\sk
La loi de Kirchhoff traduit la conservation du flux incident~$\Phi_{\lambda} = \Phi_{\lambda}^r + \Phi_{\lambda}^t + \Phi_{\lambda}^a$ par une relation entre les coefficients spectraux de réflexion, transmission et absorption $$ \boxed{ \rho_{\lambda} + \tau_{\lambda} + \alpha_{\lambda} = 1 } $$ On peut également considérer le flux énergétique incident~$\Phi$ intégré selon toutes les longueurs d'onde. Les coefficients de réflexion~$\rho$, de transmission~$\tau$, d'absorption~$\alpha$ peuvent alors être définis par $$\Phi^r = \rho \, \Phi \qquad\qquad \Phi^t = \tau \, \Phi \qquad\qquad \Phi^a = \alpha \, \Phi $$ et la loi de Kirchhoff s'écrit alors $$ \boxed{ \rho + \tau + \alpha = 1 } $$ On note par ailleurs que les définitions des coefficients impliquent que $\rho \ne \int_{\lambda} \rho_\lambda \, \dd\lambda$
