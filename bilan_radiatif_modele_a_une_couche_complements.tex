\sk
La température de l'atmosphère dans le modèle à une couche est, d'après les équations qui précèdent,
\[ \boxed{ T\e{a} = \bigg[ \frac{ \mathcal{F}\e{s}'\,(1-A\e{b}) }{ \sigma } \bigg]^{\frac{1}{4}} = T\e{eq} } \]
Un résultat équivalent peut être obtenu en faisant un bilan des flux reçus/cédés pour l'interface \ofg{espace}
\[ \underbrace{\mathcal{F}\e{s}'\,A\e{b} + \sigma \, {T\e{a}}^4}_{\text{bilan des flux reçus}} = \underbrace{\mathcal{F}\e{s}'}_{\text{bilan des flux cédés}} \] 
Deux aspects de ce modèle simple de l'effet de serre sont importants:
\begin{enumerate}
\item Que l'on considère un équilibre radiatif simple [figures~\ref{fig:eqrad2}], ou un modèle à une couche [figures~\ref{fig:modun}], la température à laquelle est émise le rayonnement infrarouge sortant vers l'espace doit être (en moyenne) égale à $T\e{eq}$. Sans atmosphère, cette température est celle de la surface, avec une atmosphère opaque dans l'infrarouge, il s'agit de celle de l'atmosphère.
\item Il n'y a un effet de serre que si la température d'émission vers l'espace est inférieure à la température de la surface. On peut l'imaginer dans le cas où l'atmosphère est également opaque dans les longueurs d'onde visible~: la surface échange alors uniquement du rayonnement avec l'atmosphère, et est à la même température à l'équilibre~: $T\e{s}=T\e{a}=T\e{eq}$ [voir section suivante].
\end{enumerate}
Pour obtenir des températures atmosphériques plus en accord avec les variations verticales observées (qui servent à définir les différentes couches atmosphériques comme abordé au chapitre d'introduction), on peut adopter un modèle à~$2$, $3$, \ldots couches\footnote{Ce point est abordé en travaux dirigés.}.
