\sk
Les molécules de l'atmosphère absorbent donc le rayonnement à diverses longueurs d'onde. En conséquence, on comprend que les coefficients d'absorption des gaz qui composent l'atmosphère sont extrêmement variables en fonction de~$\lambda$ et présentent une structure très complexe. Un domaine limité de longueurs d'onde contigues où une certaine espèce atmosphérique est très absorbante est appelé \voc{bande d'absorption}. Certaines espèces possèdent des bandes d'absorption dans les longueurs d'onde visible, comme l'ozone~O$_3$, d'autres dans les longueurs d'onde infrarouge, comme les gaz à effet de serre CO$_2$ et H$_2$O [Figure~\ref{fig:atmspectrum} et table~\ref{tab:abs}]. Un domaine limité de longueurs d'onde contigues où les espèces principales qui composent une atmosphère ne sont pas (trop) absorbantes est appelée \voc{fenêtre atmosphérique}, car alors le coefficient de transmission atmosphérique est proche de~$1$.

\small
\begin{table}\label{tab:abs}
\begin{center}
\begin{tabular}{|c|c|}
\hline
Molécules & Principales bandes d'absorption (en $\mu$m) \\
\hline
O$_3$ & 0,242-0,31 (Hartley) / 0,31-0,4 (Huggins) / 0,4-0,85 (Chappuis) / 3,3 / 4,74 \\
O$_2$ & 0,175-0,2 (Schumann-Runge) / 0,2-0,26 (Herzberg) / 0,628 / 0,688 / 0,762 / 1,06 / 1,27 / 1,58 \\
CO$_2$ & 1,4 / 1,6 / 2,0 / 2,7 / 4,3 / >15 \\
H$_2$O & 0,72 / 0,82 / 0,94 / 1,1 / 1,38 / 1,87 / 2,7-3,2 / 6,25 / >14 \\
CH$_4$ & 1,66 / 2,2 / 2,3 / 2,37 / 3,26 / 3,31 / 3,53 / 3,83 / 3,55 / 7,65 \\
CO & 2,34 / 4,67 \\
N2O & 2,87 / 2,97 / 3,9 / 4,06 / 4,5 \\
\hline
\end{tabular}
\caption{\emph{Principales bandes d'absorption pour les gaz composant l'atmosphère terrestre. Voir la figure~\ref{fig:atmspectrum}}}
\end{center}
\end{table}
\normalsize

\figside{0.7}{0.35}{decouverte/cours_dyn/absorption.png}{Spectres d'absorption de l'atmosphère en fonction de la longueur d'onde. [Haut] Courbes d'émittance normalisée de corps noirs à 5780~K (rayonnement solaire) et 255~K (rayonnement terrestre). [Bas] Coefficients d'absorption (en~$\%$) entre le sommet de l'atmosphère et la surface. Les principaux gaz responsables de l'absorption à différentes longueurs d'onde sont indiqués en bas. Source: McBride and Gilmour, An Introduction to the Solar System, 2004 ; d'après Goody and Yung, Atmospheric radiation, 1989}{fig:atmspectrum}
