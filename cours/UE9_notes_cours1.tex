\documentclass[a4paper,DIV16,10pt]{scrartcl}
%%%%%%%%%%%%%%%%%%%%%%%%%%%%%%%%%%%%%%%%%%%%%%%%%%%%%%%%%%%%%%%%%%%%%%%%%%%%%%%%%%%
\usepackage{texcours}
%%%%%%%%%%%%%%%%%%%%%%%%%%%%%%%%%%%%%%%%%%%%%%%%%%%%%%%%%%%%%%%%%%%%%%%%%%%%%%%%%%%
\newcommand{\zauthor}{Aymeric SPIGA}
\newcommand{\zaffil}{Laboratoire de Météorologie Dynamique}
\newcommand{\zemail}{aymeric.spiga@upmc.fr}
\newcommand{\zcourse}{Physique des atmosphères planétaires}
\newcommand{\zcode}{UE9}
\newcommand{\zuniversity}{UPMC}
\newcommand{\zlevel}{M2 Planétologie}
\newcommand{\zsubtitle}{Fiches complémentaires du cours 1}
\newcommand{\zlogo}{\includegraphics[height=1.5cm]{decouverte/cours_meteo/UPMC_cart-blanc-Q_7504-703-3.png}}
\newcommand{\zrights}{Copie et usage interdits sans autorisation explicite de l'auteur}
\newcommand{\zdate}{\today}
%%%%%%%%%%%%%%%%%%%%%%%%%%%%%%%%%%%%%%%%%%%%%%%%%%%%%%%%%%%%%%%%%%%%%%%%%%%%%%%%%%%
\begin{document} \inidoc
%%%%%%%%%%%%%%%%%%%%%%%%%%%%%%%%%%%%%%%%%%%%%%%%%%%%%%%%%%%%%%%%%%%%%%%%%%%%%%%%%%%

\newpage
\section{Parcelle}
\sk
L'atmosphère est composée d'un ensemble de molécules. Pour la description de la plupart des phénomènes étudiés, le suivi des comportements individuels de chacunes des molécules composant l'atmosphère est impossible. On s'intéresse donc aux effets de comportement d'ensemble, ou moyen. Les principales variables thermodynamiques utilisées pour décrire l'atmosphère sont donc des grandeurs \voc{intensives} dont la valeur ne dépend pas du volume d'air considéré.
\begin{finger}
\item La \voc{température} $T$ est exprimée en K (kelvin) dans le système international. Elle est un paramètre macroscopique qui représente l'agitation thermique des molécules microscopiques. Les mesures de température usuelles font parfois référence à des quantités en \deg C, auxquelles il faut ajouter la valeur $273.15$ pour convertir en kelvins.
\item La \voc{pression} $P$ est exprimée en Pa dans le système international. La pression fait référence à une force par unité de surface ($1$~Pa correspond à l'unité~N~m$^{-2}$). Paramètre macroscopique, elle est reliée à la quantité de mouvement des molécules microscopiques qui subissent des chocs sur une surface donnée. Les mesures et raisonnements météorologiques font souvent référence à des quantités en hPa ou en mbar. Ces deux unités sont équivalentes : 1~hPa correspond à~$10^{2}$~Pa, 1~mbar correspond à $10^{-3}$~bar, ce qui correspond bien à 1~hPa, puisque 1 bar est $10^{5}$~Pa. La pression atmosphérique vaut $1013.25$~hPa (ou mbar) en moyenne au niveau de la mer sur Terre. On utilise parfois l'unité d'$1$~atm (atmosphère) qui correspond à cette valeur de~$101325$~Pa.
\item La \voc{masse volumique} ou densité~$\rho$ est exprimée en~kg~m$^{-3}$ dans le système international. Elle représente une quantité de matière par unité de volume. Elle vaut environ $1.217$~kg~m$^{-3}$ proche de la surface sur Terre.
\end{finger}

\sk
L'atmosphère est composée d'un ensemble de molécules microscopiques et l'on s'intéresse aux effets de comportement d'ensemble, qualifiés de \voc{macroscopiques}. Les variables thermodynamiques utilisées pour décrire l'atmosphère (pression~$P$, température~$T$, densité~$\rho$) sont des grandeurs macroscopiques \voc{intensives} dont la valeur ne dépend pas du volume d'air considéré. 
%Une façon d'y parvenir est d'utiliser des grandeurs volumiques ou massiques.

\sk
Le système que l'on étudie est appelé \voc{parcelle d'air}. Il s'agit d'un volume d'air dont les dimensions sont %à la fois
\begin{citemize}
\item assez grandes pour contenir un grand nombre de molécules et pouvoir moyenner leur comportement, afin d'exprimer un équilibre thermodynamique~;
\item assez petites par rapport au phénomène considéré, afin de décrire le fluide atmosphérique de façon continue~; la parcelle d'air peut donc être considérée comme un volume élémentaire.
\end{citemize}
On peut donc supposer que les variables macroscopiques d'intérêt sont quasiment constantes à l'échelle de la parcelle. Autrement dit, une parcelle est caractérisée par sa pression~$P$, sa température~$T$, sa densité~$\rho$. Les limites d'une parcelle sont arbitraires, mais ne correspondent pas en général à des barrières physiques. 



\sk
Une parcelle est en \voc{équilibre mécanique} avec son environnement, c'est-à-dire que la pression de la parcelle~$P\e{p}$ et la pression de l'environnement~$P\e{e}$ dans lequel elle se trouve sont égales
\[ \boxed{P\e{p} = P\e{e}} \]
Néanmoins, une parcelle n'est pas en général en \voc{équilibre thermique} avec son environnement, c'est-à-dire que la température de la parcelle et la température de l'environnement dans lequel elle se trouve ne sont pas nécessairement égales
\[ \boxed{T\e{p} \neq T\e{e}} \]
Cette dernière propriété provient du fait que l'air est un très bon isolant thermique\footnote{Une telle propriété est utilisée dans le principe du double vitrage}.




\newpage
\section{Gaz parfait}
\sk
On appelle \voc{gaz parfait} un gaz suffisament dilué pour que les interactions entre les molécules du gaz, autres que les chocs, soient négligeables. L'air composant l'atmosphère peut être considéré en bonne approximation comme un mélange de gaz parfaits\footnote{On peut en général considérer que le gaz est parfait si~$P < 1$~kbar. C'est le cas dans la plupart des atmosphères planétaires rencontrées. Il n'y a guère qu'au coeur des planètes géantes, où la pression dépasse cette limite, que l'approximation du gaz parfait doit être abandonnée.} notés~$i$, dont le nombre de moles est~$n_i$ pour un volume donné~$V$ d'air à la température~$T$. Chaque espèce gazeuse composant l'air est caractérisée par une \voc{pression partielle}~$P_i$ qui est définie comme la pression qu'aurait l'espèce gazeuse si elle occupait à elle seule le volume~$V$ à la température~$T$. Chacune de ces espèces gazeuses~$i$ se caractérise par une même température~$T$ et vérifie l'équation d'état du gaz parfait $$ P_i \, V = n_i \, R^* \, T $$ où~$R^*$=8.31 J~K\mo~mol\mo~est la constante des gaz parfaits (produit du nombre d'Avogadro et de la constante de Boltzmann). La pression totale de l'air~$P$ est, d'après la loi de Dalton, la somme des pressions partielles~$P_i$ des espèces gazeuses composant le mélange $P=\Sigma P_i$. En faisant la somme des lois du gaz parfait appliquées pour chacune des espèces gazeuses, on obtient $$ P \, V = \big( \Sigma n_i \big) \, R^* \, T $$ ce qui montre qu'un mélange de gaz parfaits est aussi un gaz parfait. Cette équation permet de relier la pression totale~$P$ à la température~$T$, mais présente l'inconvénient de contenir les grandeurs \voc{extensives}~$V$ et $n_i$ qui dépendent du volume d'air considéré. Il reste donc à donner une traduction intensive à la loi du gaz parfait pour un mélange de gaz. La masse totale contenue dans le volume~$V$ peut s'écrire $m=\Sigma n_iM_i$ où $M_i$ est la masse molaire du gaz $i$. En divisant l'équation précédente par $m$, et en utilisant la définition de la masse volumique~$\rho = m / V$, on obtient $$ \frac{P}{\rho} = \frac{\Sigma n_i}{\Sigma n_iM_i} \, R^* \, T $$ Or, d'une part, la \voc{masse molaire de l'air} composé d'un mélange de gaz~$i$ est $$ \boxed{ M=\frac{\Sigma n_iM_i}{\Sigma n_i} }$$ et d'autre part, on peut définir une \voc{constante de l'air sec} de la façon suivante $$ R=\frac{R^*}{M} $$ On a alors l'équation des gaz parfaits pour l'air atmosphérique qui permet de relier les trois paramètres intensifs importants : pression~$P$, température~$T$ et densité~$\rho$ $$ \boxed{ P = \rho \, R \,T } $$ L'état thermodynamique d'un élément d'air est donc déterminé uniquement par deux paramètres sur les trois~$P$, $T$, $\rho$. En météorologie par exemple, on travaille principalement avec la pression et la température qui sont plus aisées à mesurer que la densité. Les valeurs numériques de~$M$, et donc~$R$, dépendent de la planète considérée et de sa composition atmosphérique. 



\newpage
\section{Equilibre hydrostatique}
\sk
On considère une parcelle d'air cubique de dimensions élémentaires~$(\dd x,\dd y, \dd z)$, au repos et située à une altitude~$z$. La pression atmosphérique vaut~$P(z)$ sur la face du dessous et~$P(z+\dd z)$ sur la face du dessus. Pour le moment, on ne considère pas de variations de pression~$P$ selon l'horizontale\footnote{En pratique, les variations de pression selon l'horizontale sont effectivement négligeables par rapport aux variations de pression selon la verticale. On revient sur ce point dans le chapitre consacré à la dynamique}. Il y a équilibre des forces s'exerçant sur cette parcelle. On appelle \voc{équilibre hydrostatique} l'équilibre des forces selon la verticale, à savoir~:
\begin{citemize}
\item Son poids de module\footnote{On néglige les variations de~$g$ avec~$z$.}~$m \, g$ (où~$m = \rho \, \dd x \, \dd y \, \dd z$) dirigé vers le bas
\item La force de pression sur la face du dessous de module~$P(z) \, \dd x \, \dd y$ dirigée vers le haut
\item La force de pression sur la face du dessus de module~$P(z+\dd z) \, \dd x \, \dd y$ dirigée vers le bas
\item La force de viscosité qui est négligée
\end{citemize}
On note que, contrairement au poids qui s'applique de façon homogène sur tout le volume de la parcelle d'air, les forces de pression s'appliquent sur les surfaces frontières de la parcelle d'air. 
Pour une parcelle au repos, la résultante selon la verticale des forces de pression exercées par le fluide environnant (ici, l'air) n'est autre que la poussée d'Archimède.
%%Par ailleurs, l'équilibre hydrostatique suppose implicitement que la parcelle est à l'équilibre thermique avec son environnement~$T\e{p} = T\e{e}$ soit~$\rho\e{p} = \rho\e{e}$. On aborde le cas général où~$T\e{p} \neq T\e{e}$ dans le chapitre suivant pour définir les notions de stabilité.

\sk
L'équilibre hydrostatique de la parcelle s'écrit donc
\[ - \rho \, g \, \dd x \, \dd y \, \dd z + P(z) \, \dd x \, \dd y - P(z+\dd z) \, \dd x \, \dd y = 0 \qquad \Rightarrow \qquad - \rho \, g \, \dd z + P(z) - P(z+\dd z) = 0 \]
soit en introduisant la dérivée partielle suivant~$z$ de~$P$
\[ \frac{P(z+\dd z) - P(z)}{\dd z} = \boxed{\EH} \]
Cette relation est appelée \voc{équation hydrostatique} (ou relation de l'équilibre hydrostatique). Elle indique que, pour la parcelle considérée, la résultante des forces de pression selon la verticale équilibre la force de gravité. En principe, cette équation est valable pour une parcelle au repos. Par extension, elle est valable lorsque l'accélération verticale d'une parcelle est négligeable devant les autres forces. L'équation hydrostatique est en excellente approximation valable pour les mouvements atmosphériques de grande échelle. 

\sk
Si l'on intègre la relation hydrostatique entre deux niveaux~$z_1$ et~$z_2$ où la pression est~$P_1$ et~$P_2$, on obtient
\[ \Delta P = P_2 - P_1 = - g \, \int_{z_1}^{z_2} \rho \, \dd z \]
L'équilibre hydrostatique peut donc s'interpréter de la façon éclairante suivante~: la différence de pression entre deux niveaux verticaux est proportionnelle à la masse d'air (par unité de surface) entre ces niveaux. Une autre façon équivalente de formuler cela est de dire que la pression atmosphérique à une altitude~$z$ correspond au poids de la colonne d'air située au-dessus de l'altitude~$z$, exercé sur une surface unité de~$1$~m$^2$. Il s'ensuit que la pression atmosphérique~$P$ est décroissante selon l'altitude~$z$. Ainsi, la pression peut être utilisée pour repérer une position verticale à la place de l'altitude. En sciences de l'atmosphère, la pression atmosphérique est une coordonnée verticale plus naturelle que l'altitude~: non seulement elle est directement reliée à la masse atmosphérique par l'équilibre hydrostatique, mais elle est également plus aisée à mesurer.


\newpage
\section{\'Echelle de hauteur}
\sk
En exprimant la densité~$\rho$ en fonction de l'équation des gaz parfaits, l'équilibre hydrostatique s'écrit
\[ \Dp{P}{z} = - g \, \frac{P}{RT} \]
On peut intégrer cette équation si on suppose que l'on connaît les variations de~$T$ en fonction de $P$ ou $z$. On suppose ici que l'on peut négliger les variations de pression selon l'horizontale devant les variations suivant la verticale, donc transformer les dérivées partielles~$\partial$ en dérivées simples~$\dd$. On effectue ensuite une séparation des variables
\[R \, T \, \frac{\dd P}{P} = - g \, \dd z\]

\sk
Cette équation peut s'écrire sous une forme dimensionnelle simple à retenir
\[ \boxed{ \frac{\dd P}{P} = - \frac{\dd z}{H(z)} \qquad \text{avec} \qquad H(z) = \frac{R \, T(z)}{g} } \]
La grandeur~$H$ se dénomme l'\voc{échelle de hauteur} et dépend des variations de la température~$T$ avec l'altitude~$z$. L'équation ci-dessus indique bien que la pression décroît avec l'altitude selon une loi exponentielle comme observé en pratique. Cette loi peut être plus ou moins complexe selon la fonction~$T(z)$. On peut néanmoins fournir une illustration simple du résultat de l'intégration dans le cas d'une atmosphère isotherme~$T(z)=T_0$
\[ P(z) = P(z=0) \, e^{-z/H} \qquad \text{avec} \qquad H = R \, T_0 / g \]



\mk\mk\mk\mk\mk
\begin{detail}
\sk
Depuis son invention en 1643 par Torricelli, disciple de Galilée, le \voc{baromètre} est l'instrument de référence pour mesurer la pression atmosphérique à la surface de l'atmosphère terrestre\footnote{Le but initial de Torricelli était de parvenir le premier à maintenir artificiellement en laboratoire une chambre sous vide. Néanmoins, il est également reporté que l'invention du baromètre découle des réflexions de Torricelli autour de l'impossibilité, constatée en pratique, de pomper l'eau d'un fleuve au-dessus d'un certain niveau.}. Trois ans après son invention, le baromètre était déjà utilisé pour sa première application en sciences de l'atmosphère~: donner une preuve expérimentale de l'équilibre hydrostatique qui gouverne la stratification en pression de l'atmosphère. Autrement dit, le baromètre est un moyen indirect de mesurer la masse de l'atmosphère à travers la pression de surface. Blaise Pascal montra ainsi, par des mesures respectivement sur la Tour Saint-Jacques à Paris ($ \Delta z = 52 \U{m} $ au-dessus du sol) et sur le Puy-de-Dôme en Auvergne ($ \Delta z = 1000 \U{m} $ au-dessus du sol), que la pression atmosphérique varie avec l'altitude\footnote{Le texte original du traité de Pascal est disponible sur Gallica~: \url{http://gallica.bnf.fr/ark:/12148/bpt6k105083f}}.

\sk
L'équation hypsométrique permet de retrouver l'écart relatif en pression mesuré par Blaise Pascal entre le sol et le haut de la Tour Saint-Jacques. Comme les variations mesurées sont petites, on peut les assimiler aux différentielles et on peut négliger les variations de température avec l'altitude. Les variations relatives de pression mesurées par Pascal peuvent alors s'écrire
\[ \f{\Delta P}{P} \simeq \f{g}{R\,T} \, \Delta z \]
L'application numérique avec~$T = 288$~K donne une variation~$\Delta P / P = 0.62 \%$. La variation de pression détectée par Blaise Pascal\footnote{On note que, par une heureuse coincidence, la variation de pression entre le pied et le sommet de la Tour Saint-Jacques est de l'ordre de grandeur de la pression atmosphérique à la surface de Mars, ce qui permet de se représenter la finesse de l'atmosphère sur cette planète, comparé à notre Terre.} est donc d'environ~$6$~hPa (avec la valeur standard de la pression de surface~$P_0 = 1013$~hPa). Bien que cette baisse de pression soit détectable à l'aide du tube de Torricelli, Blaise Pascal a reproduit l'expérience au Puy-de-Dôme avec un écart~$\Delta z$ plus grand pour une meilleure précision quantitative. 


\end{detail}

\newpage
\section{\'Equation hypsométrique}
\sk
Dans l'équation de l'échelle de hauteur, faire l'hypothèse isotherme est très simpliste et rarement rencontré en pratique dans l'atmosphère. On se place dans le cas plus général, bien que toujours simplifié, de deux niveaux atmosphériques~$a$ et~$b$ entre lesquels la température ne varie pas trop brusquement avec l'altitude~$z$. On réalise alors l'intégration entre les deux niveaux~$a$ et~$b$
\[R \, T \, \frac{\dd P}{P} = - g \, \dd z \qquad \Rightarrow \qquad R \, \int_a^b T\, \frac{\dd P}{P} = - g \, \int_a^b dz\]
puis on définit la température moyenne de la couche atmosphérique entre~$a$ et~$b$ avec une moyenne pondérée
\[ \langle T \rangle = \frac{\int_a^b T \, \frac{\dd P}{P}}{\int_a^b \frac{\dd P}{P}} \]
pour obtenir finalement
\[R \, \langle T \rangle \, \int_a^b \frac{\dd P}{P} = - g \, \int_a^b dz
\qquad \Rightarrow \qquad \boxed{ g \, (z_a - z_b) = R \, \langle T \rangle \ln \left( \frac{P_b}{P_a} \right) } \]
Cette relation est appelée \voc{équation hypsométrique}. Elle correspond à une formulation utile en météorologie du principe que \ofg{l'air chaud se dilate}. Les conséquences de l'équation hypsométrique peuvent s'exprimer de diverses façons équivalentes.
\begin{citemize}
\item Pour une masse d'air donnée, une couche d'air chaud est plus épaisse.
\item La distance entre deux isobares est plus grande si l'air est chaud.
\item La pression diminue plus vite selon l'altitude dans une couche d'air froid.
\end{citemize}
En passant le résultat précédent au logarithme, on note que l'on retrouve toujours le fait que la pression diminue avec l'altitude selon une loi exponentielle. En notant l'échelle de hauteur moyenne~$\langle H \rangle$, on a
\[ P_b = P_a \, e^{ - \frac{z_b - z_a}{\langle H \rangle}} \qquad \text{avec} \qquad \langle H \rangle = \frac{R \, \langle T \rangle}{g} \]





\newpage
\section{Circulations thermiques directes}
\sk
Toute différence de température entre deux régions (provoquée par exemple par un chauffage différentiel, ou par une différence des propriétés thermophysiques de la surface) est associée à des différences de pression, car d'après l'équation hypsométrique (équilibre hydrostatique + équation d'état du gaz parfait) la pression diminue plus vite avec l'altitude dans les couches d'air froid que dans les couches d'air chaud. Ceci donne naissance en altitude à un gradient de pression donc, en supposant que la force de pression est seule responsable de l'accélération du vent (vision à raffiner par la suite), des vents vont naître en altitude de la région chaude vers la régions froide. Ces vents induisent un flux de masse atmosphérique de la région chaude vers la région froide, donc causent, d'après l'équivalence entre pression et masse déduite de l'équilibre hydrostatique, une augmentation de la pression de surface dans la région froide par rapport à la région chaude. Ceci donne naissance proche de la surface à des vents de la région chaude vers la région froide. Par continuité, en considérant les convergences et divergences d'air proche du sol et en altitude, l'air s'élève dans les régions chaudes et redescend dans les régions froides.

\sk
Des exemples de circulations thermiques directes sont
\begin{finger}
\item les \voc{cellules de Hadley}, cellules fermées dans le plan méridien, sud-nord et verticale; sous les tropiques, l'air s'élève proche de l'équateur (suivant la saison, du côté de l'hémisphère d'été) et redescend au niveau des subtropiques.
\item les \voc{\og brises \fg~de mer et de terre} au bord de la mer sur Terre, naissant du contraste thermique entre continent et océan
\item les circulations atmosphériques sur Mars entre les régions polaires couvertes de glace et les régions de sol nu
\end{finger}

\sk
Les cellules fermées associées aux circulations thermiques directes \underline{ne sont pas des cellules de convection}. Elles résultent simplement de la déformation du champ de pression par des contrastes de température. Des cellules fermées non convectives peuvent également se développer dans le sens inverse de celui thermique direct (par exemple, les cellules de Ferrel sur Terre) : les mécanismes sont distincts des processus de circulation thermique directe et sont en général relatifs au forçage de l'écoulement moyen par les ondes atmosphériques résultant d'instabilités dans l'atmosphère.





\newpage
\section{Premier principe de la thermodynamique}
\sk
Un système thermodynamique possède, en plus de son énergie d'ensemble (cinétique, potentielle), une \voc{énergie interne}~$U$. Comme la température~$T$, l'énergie interne~$U$ est une grandeur macroscopique qui représente les phénomènes microscopiques au sein d'un gaz. Le premier principe indique que les variations d'énergie interne sont égales à la somme du travail et de
la chaleur algébriquement reçus~:
\[ \dd U = \delta W + \delta Q\]

\sk
Dans le cas d'un gaz parfait, l'énergie potentielle d'interaction des molécules du gaz est négligeable, et l'énergie interne est égale à l'énergie cinétique des molécules, qui dépend seulement de la température. On peut montrer que $U = n \, \frac{\zeta \, R^* \, T}{2}$ où $\zeta$ est le nombre de degrés de liberté des molécules. Pour un gaz (principalement) diatomique comme l'air, $\zeta = 5$. 

\sk
Dans le cas de variations quasi-statiques d'un gaz, ce qui est supposé être le cas dans l'atmosphère, le travail s'exprime en fonction de la pression~$P$ du gaz et de la variation de volume~$\dd V$
\[ \delta W = - P \, \dd V \]

\sk
L'expérience montre que la quantité de chaleur échangée au cours d'une transformation à volume ou pression constant est proportionnelle à la variation de température du système~: $\delta Q = n \, C_V^* \, \dd T$ à volume constant, $\delta Q = n \, C_P^* \, \dd T$ à pression constante. $C_P^*, C_V^*$ sont les \voc{chaleurs molaires}, également appelées \voc{capacités calorifiques}. Il s'agit de l'énergie qu'il faut fournir à un gaz pour faire augmenter sa température de~$1$~K dans les conditions indiquées (à volume constant ou à pression constante). Pour une transformation à volume constant (isochore), $\dd U = \delta Q$ donc $C_V^*=\frac{\zeta \, R^*}{2}$.

\sk
Pour l'étude de l'atmosphère, toujours dans la logique de travailler sur des grandeurs intensives, il est  bien plus utile de s'intéresser aux variations de pression plutôt qu'à celles de volume. On utilise donc l'\voc{enthalpie}~$H = U + P \, V$. On a alors par dérivation $ \dd H = \dd U + \dd (P\,V) $ puis, en utilisant le premier principe
\[ \dd H = V \, \dd P + \delta Q \]
Pour une transformation à pression constante (isobare) on a $\dd H = \delta Q$. On en déduit pour une transformation quelconque\footnote{
D'autre part, en utilisant conjointement la dérivation de l'équation d'état du gaz parfait~$\dd (P\,V) = n \, R^* \, \dd T$ et l'expression de l'énergie interne~$U = n \, C_V^* \, \dd T$, on obtient $\dd H = n \, C_V^* \, \dd T + n \, R^* \, \dd T$ pour une transformation quelconque. On en déduit la relation de Mayer \[ C_P^* = C_V^* + R^* = \frac{(\zeta+2) \, R^*}{2}\]
} 
que $\dd H = n \, C_P^* \, dT$, ce qui permet d'écrire
\[ n \, C_P^* \, dT = V \, \dd P + \delta Q \]





\newpage
\section{Premier principe de la thermodynamique (application)}
\sk
Afin de travailler sur des grandeurs intensives, on divise la relation précédente par la masse~$m$ de la parcelle pour obtenir
\[ C_P \, \dd T = \frac{\dd P}{\rho} + \delta q \]
où $\delta q$ est la chaleur massique reçue et $C_P = C_P^* / M$ est la \voc{chaleur massique de l'air} ($C_P$=1004 J~K$^{-1}$~kg$^{-1}$). Nous disposons alors d'une autre version du premier principe, très utile en météorologie et valable pour une transformation quelconque d'une parcelle d'air
\[ \boxed{ \underbrace{\textcolor{white}{\frac{R^2}{C_P}} \dd T \textcolor{white}{\frac{R}{C_P}}}_{\text{variation de température de la parcelle}} = \underbrace{\frac{R}{C_P} \, \frac{T}{P} \, \dd P}_{\text{travail expansion/compression}} + \underbrace{\frac{1}{C_P} \, \delta q}_{\text{chauffage diabatique}} } \]

\sk
Autrement dit, la température de la parcelle augmente si elle subit une compression ($\dd P > 0$) et/ou si on lui apporte de la chaleur ($\delta q > 0$). La température de la parcelle à l'inverse diminue si elle subit une détente ($\dd P < 0$) et/ou si elle cède de la chaleur à l'extérieur ($\delta q < 0$). Il est donc important de retenir que la température de la parcelle peut très bien varier quand bien même la parcelle n'échange aucune chaleur avec l'extérieur~: dans ce cas, $\delta q = 0$ et l'on parle de \voc{transformation adiabatique}. 

\sk
L'équation fondamentale ci-dessus est directement dérivée du premier principe, mais prend une forme plus pratique en sciences de l'atmosphère du fait que les transformations que subit une parcelle atmosphérique se réduisent en général aux transformations \voc{isobares} (à pression constante $\dd P = 0$) et aux transformations \voc{adiabatiques} (sans échanges de chaleur avec l'extérieur $\delta q = 0$). Les transformations isothermes, au cours de laquelle la température de la parcelle ne varie pas, sont plus rarement rencontrées en sciences de l'atmosphère.



\section{Transformations non adiabatiques}
\sk
Dans le cas où la transformation n'est pas adiabatique, les échanges de chaleur~$\delta q$ d'une parcelle d'air avec son environnement sont non nuls et peuvent s'effectuer par~:
\begin{itemize}
\item Transfert radiatif~: l'atmosphère se refroidit en émettant dans l'infrarouge, ou se réchauffe en absorbant du rayonnement électromagnétique dans l'infrarouge [cas des gaz à effet de serre] ou dans le visible [cas de l'ozone dans la stratosphère].
%Ces échanges sont faibles et peuvent être négligés sauf à l'échelle de la circulation générale\footnote{Le refroidissement/réchauffement peut être localement élevé au sommet/à la base de nuages.}
\item Condensation ou évaporation d'eau~: le changement d'état consomme ou relâche de la chaleur (ceci n'a lieu que lorsque l'air est à saturation).
\item Diffusion moléculaire (conduction thermique)~: ces transferts sont très négligeables sauf à quelques centimètres du sol.
\end{itemize}
Un cas notamment souvent cité en météorologie est celui d'une parcelle d'air située proche du sol, à la tombée de la nuit, qui subit peu de variations de pression ($\dd P \sim 0$) mais dont la température diminue sous l'effet du refroidissement radiatif ($\delta q < 0$). Ceci explique la présence de rosée sur le sol et de brouillard proche de la surface au petit matin.




\newpage
\section{Transformation adiabatique et température potentielle}
\sk
Dans de nombreuses situations en sciences de l'atmosphère, on peut considérer que l'évolution de la parcelle est \voc{adiabatique} et se fait sans échange de chaleur avec l'extérieur ($\delta q=0$). En vertu de l'équilibre hydrostatique qui relie pression~$P$ et altitude~$z$~:
\begin{citemize}
\item une parcelle dont l'altitude~$z$ augmente sans apport extérieur de chaleur, subit une \voc{ascendance} adiabatique, donc une détente telle que~$\dd P < 0$ et sa température diminue ;
\item inversement, une parcelle dont l'altitude~$z$ diminue sans apport extérieur de chaleur, subit une \voc{subsidence} adiabatique, donc une compression telle que~$\dd P > 0$ et sa température augmente. 
\end{citemize}

\sk
Dans le cas où la transformation est adiabatique, pression et température sont intimement liées en vetu du premier principe. La version du premier principe encadrée ci-dessus avec~$\delta q = 0$ indique
\[ \dd T = \frac{R}{C_P} \, \frac{T}{P} \, \dd P\]
\[ \Rightarrow \qquad \frac{\dd T}{T} - \frac{R}{C_P} \, \frac{\dd P}{P} = 0 \]
soit par intégration
\[ T \, P^{- \kappa} = \text{constante} \qquad \text{avec} \qquad \kappa = R / C_P \]
Autrement dit, dans le cas où une parcelle subit une transformation adiabatique, sa température varie proportionnellement à~$P^{\kappa}$. Il s'agit d'une version, avec les grandeurs intensives utiles en sciences de l'atmosphère, de l'équation~$P\,V^{\gamma}$, avec $\gamma = C_P / C_V$, vue dans les cours de thermodynamique générale pour les transformations adiabatiques.



\section{Gradient adiabatique sec}
\sk
D'après les seules équations thermodynamiques, on peut trouver une loi simple des variations de température avec l'altitude pour une parcelle qui ne subit que des transformations adiabatiques. Considérons le cas d'une parcelle subissant un déplacement vertical quasi-statique et adiabatique tel que~$\delta q = 0$. Elle vérifie en première approximation l'équilibre hydrostatique~$\dd P\e{p} / \rho = - g \, \dd z$. L'équation du premier principe modifiée pour le cas atmosphérique indique alors que
\[  \dd T\e{p}  = - \frac{g}{C_P} \, \dd z \]
d'où on tire le profil vertical adopté dans l'atmosphère sèche par une parcelle ne subissant pas d'échange de chaleur avec l'extérieur
\[  \boxed{ \ddf{T\e{p}}{z}  = \Gamma\e{sec} \qquad \text{avec} \qquad \Gamma\e{sec} = \frac{-g}{C_P} } \]
On note qu'il ne s'agit pas nécessairement du profil vertical suivi par l'environnement (voir section~\ref{parcenv}).

\sk
Le résultat trouvé ci-dessus revêt une importance particulière en sciences de l'atmosphère. La température d'une parcelle en ascension adiabatique décroît avec l'altitude selon un taux de variation constant, indépendamment des effets de pression. La constante~$\Gamma\e{sec}$ est appelée le \voc{gradient adiabatique sec} de température. Il n'est valable que pour une parcelle d'air non saturée en vapeur d'eau. Le calcul pour la Terre donne un refroidissement de l'ordre de~$10^{\circ}$C/km (ou K/km). 

\sk
Pourquoi cette valeur est-elle en désaccord avec la décroissance de~$6.5^{\circ}$C/km effectivement constatée dans l'atmosphère terrestre~? Cet écart est relatif aux processus humides qui ont une grande importance dans l'atmosphère terrestre.
%Le chapitre suivant apporte des éléments de réponse à ce paradoxe apparent.





\newpage
\section{Force de flottaison}
\sk
Soit une parcelle dont la température $T\e{p}$ n'est pas égale à celle de l'environnement~$T\e{e}$, que ce soit sous l'effet d'un chauffage diabatique (par exemple~: chaleur latente, effets radiatifs) ou d'une compression / détente adiabatique. On reprend le calcul réalisé précédemment pour l'équilibre hydrostatique, avec la différence notable que l'on n'est plus dans le cas statique~: on étudie le mouvement vertical d'une parcelle. 

\sk
La somme des forces massiques s'exerçant sur la parcelle suivant la verticale est
\[ - g  - \frac{1}{\rho\e{p}}  \, \Dp{P\e{e}}{z} \]
où~$\rho\e{p}$ est la masse volumique de la parcelle. L'environnement est à l'équilibre hydrostatique donc
\[ \Dp{P\e{e}}{z} = - \rho\e{e} \, g \]
Ainsi la résultante~$F_z$ des forces massiques qui s'exercent sur la parcelle selon la verticale vaut
\[ F_z = g \, \left( \frac{\rho\e{e}}{\rho\e{p}} - 1 \right) = g \, \frac{\rho\e{e}-\rho\e{p}}{\rho\e{p}} \]
En utilisant l'équation du gaz parfait pour la parcelle~$\rho\e{p}=P/RT\e{p}$ et l'environnement~$\rho\e{e}=P/RT\e{e}$, on a
\[ \boxed{ F_z = g \, \frac{T\e{p}-T\e{e}}{T\e{e}} } \]
La résultante des forces est donc dirigée vers le haut, donc la parcelle s'élève, si la parcelle est plus chaude (donc moins dense) que son environnement. 
Elle est dirigée vers le bas si la parcelle est plus froide (donc plus dense) que son environnement.
En d'autres termes, on écrit ici la version météorologique de la force ascendante ou descendante 
provoquée par la poussée d'Archimède, également appelée \voc{force de flottaison}.

%%% CAPE?

\newpage
\section{(In)stabilité}
\sk
Ces considérations permettent de définir le concept de stabilité et instabilité verticale de l'atmosphère.
On considère l'atmosphère à un endroit donné de la planète, à une saison donnée, à une heure donnée de la journée.
On suppose que la température de l'environnement varie linéairement avec l'altitude
\[ \ddf{T\e{e}}{z} = \Gamma\e{env} \]
A une altitude~$z_0$ proche de la surface, la température de l'environnement est~$T\e{e}(z_0)=T_0$.

\sk
On considère une parcelle initialement à l'altitude~$z_0$ dont la température initiale~$T\e{p}(z_0)$ est également~$T_0$. On suppose que la parcelle subit une ascension verticale d'amplitude~$\delta z > 0$. Le profil de température suivi par la parcelle lors de son ascension est
\[ \ddf{T\e{p}}{z} = \Gamma\e{parcelle} \]
\begin{citemize}
\item Si la parcelle est non saturée, elle suit un profil adiabatique sec tel que $\Gamma\e{parcelle} = \Gamma\e{sec} \simeq - 10 \, \text{K/km}$.
\item Si elle est saturée, elle suit un profil adiabatique saturé tel que $\Gamma\e{parcelle} = \Gamma\e{saturé} \simeq - 6.5 \, \text{K/km}$. 
\end{citemize}
On rappelle qu'en général, à l'échelle où l'on étudie les mouvements de la parcelle
\[ \Gamma\e{parcelle} \neq \Gamma\e{env} \]

\sk
Quel est l'effet de la perturbation~$\delta z > 0$ sur le mouvement de la parcelle~? A l'altitude~$z_0 + \delta z$, les températures de la parcelle et de l'environnement sont respectivement
\[ T\e{p}(z_0 + \delta z) = T_0 + \Gamma\e{parcelle} \, \delta z 
\qquad \text{et} \qquad
T\e{e}(z_0 + \delta z) = T_0 + \Gamma\e{env} \, \delta z \]
\begin{finger}
\item Si $\Gamma\e{parcelle} > \Gamma\e{env}$, la température~$T\e{e}$ de l'environnement décroît plus vite que la température~$T\e{p}$ de la parcelle. Il en résulte que~$T\e{p}(z_0 + \delta z) > T\e{e}(z_0 + \delta z)$ et le mouvement de la parcelle est ascendant. La perturbation initiale est donc amplifiée par les forces de flottabilité. On parle de \voc{situation instable}. La situation est d'autant plus instable que la température de l'environnement décroît rapidement avec l'altitude. Lorsque la situation est instable, les mouvements verticaux sont amplifiés~: on parle parfois de \voc{situation convective}.
\item Si $\Gamma\e{parcelle} < \Gamma\e{env}$, la température~$T\e{e}$ de l'environnement décroît moins vite que la température~$T\e{p}$ de la parcelle. Il en résulte que~$T\e{p}(z_0 + \delta z) < T\e{e}(z_0 + \delta z)$ et le mouvement de la parcelle est descendant. La perturbation initiale n'est donc pas amplifiée et la parcelle revient à son état initial. On parle de \voc{situation stable}. La stabilité est d'autant plus grande que la température de l'environnement décroît lentement (ou augmente, dans le cas d'une inversion de température). Lorsque la situation est stable, les mouvements verticaux sont inhibés.
\end{finger}
La résultante des forces verticales s'exerçant sur la parcelle peut s'écrire en fonction des taux de variation~$\Gamma$ de la température
\[ F_z = g \, \frac{\Gamma\e{parcelle}-\Gamma\e{env}}{T\e{env}} \, \delta z \]
\noindent Un raisonnement similaire permet d'obtenir la fréquence de Brunt-V{\"a}is{\"a}l{\"a}.


\newpage
\section{Transformations pseudo-adiabatiques}
\sk
On considère tout d'abord une parcelle d'air (contenant de la vapeur d'eau) en évolution isobare. Le premier principe appliqué à la parcelle indique donc
\[ \dd T = \frac{1}{C_P} \, \delta q \]
Lors de l'évaporation, les molécules d'eau liquide voient les liaisons hydrogène avec leurs proches voisins être brisées. Le passage de l'eau de la phase liquide à la phase vapeur consomme donc de l'énergie\footnote{On peut s'en convaincre en notant la sensation de froid immédiate que provoque la sortie d'un bain à cause de l'évaporation de l'eau liquide sur le corps mouillé~; ou en se souvenant que lorsque l'on souffle sur la soupe pour la refroidir, c'est précisément pour favoriser l'évaporation et la refroidir efficacement.}~: pour l'air qui compose la parcelle, $\delta q < 0$ et il y a refroidissement. 
A l'inverse, lors de la condensation, les molécules d'eau sous forme gazeuse créent des liaisons hydrogène avec les molécules d'eau de la phase liquide pour atteindre un état énergétique plus faible. Le passage de l'eau de la phase vapeur à la phase liquide libère donc de l'énergie~: pour l'air qui compose la parcelle, $\delta q > 0$ et il y a chauffage.

\sk
L'énergie~$\delta q$ consommée ou libérée par les changements d'état s'appelle~\voc{chaleur latente}, on la note~$\delta q\e{latent}$. Si une masse de vapeur~$\dd m\e{vapeur d'eau}$ est condensée ou évaporée, on a
\[ \delta q\e{latent} = \frac{- L \, \dd m\e{vapeur d'eau}}{m\e{air sec}} \qquad \Rightarrow \qquad \boxed{ \delta q\e{latent} = - L \, \dd r } \]
où~$L$ est la chaleur latente massique en~J~kg$^{-1}$. La formule ci-dessus comporte un signe négatif. La quantité~$\delta q\e{latent}$ est positive lorsqu'il y a condensation (le rapport de mélange en vapeur d'eau diminue $\dd r < 0$) et négative lorsqu'il y a évaporation (le rapport de mélange en vapeur d'eau augmente $\dd r > 0$).

\sk
On considère désormais une parcelle d'air en évolution adiabatique, à l'exception des échanges de chaleur latente~: $\delta q = \delta q\e{latent}$. On appelle une telle transformation \voc{pseudo-adiabatique} ou encore \voc{adiabatique saturée}. On fait l'approximation que la chaleur latente consommée ou dégagée est seulement échangée avec l'air sec~:
\begin{citemize}
\item La chaleur latente consommée/dégagée n'est pas utilisée pour refroidir/chauffer les gouttes d'eau présentes.
\item On néglige les pertes de masse par précipitation~: la masse d'air sec considérée est constante.
\end{citemize}
Pour une telle transformation, la variation de température s'écrit ainsi
\[ \dd T = \frac{R}{C_P} \, \frac{T}{P} \, \dd P - \frac{L}{C_P} \, \dd r \]


\newpage
\section{Gradient adiabatique humide}
\sk
Considérons une parcelle en ascension adiabatique saturée (et non plus sèche comme dans la section~\ref{adiabsec}). Pour une parcelle saturée, c'est-à-dire à l'équilibre liquide/vapeur, l'équation qui précède peut s'écrire, en utilisant l'équilibre hydrostatique
\[ c_p \, \dd T + g \, \dd z + L \, \dd r = 0 \]
Or, puisque la parcelle est saturée, on a~$r = r\e{sat}(T)$ et on peut écrire $\dd r\e{sat} = \ddf{r\e{sat}}{T} \, \dd T$. On a alors
\[ \left( c_p + L \, \ddf{r\e{sat}}{T} \right) \dd T + g \, \dd z = 0\]
Cette expression est similaire au cas sec, à l'exception notable du terme supplémentaire~$L \, \ddf{r\e{sat}}{T}$ lié aux échanges latents. On peut alors obtenir le profil vertical adopté dans l'atmosphère saturée par une parcelle ne subissant pas d'échange de chaleur avec l'extérieur autre que les échanges de chaleur latente
\[  \ddf{T}{z}  = \Gamma\e{saturé} \qquad \text{avec} \qquad \Gamma\e{saturé} = \frac{-g}{c_p+L \, \ddf{r\e{sat}}{T} } \]
On a vu que $\ddf{r\e{sat}}{T}$ est toujours positif, on en déduit donc
\[ \boxed{ \Gamma\e{saturé} > \Gamma\e{sec} \qquad \text{ou} \qquad |\Gamma\e{saturé}| < |\Gamma\e{sec}| } \]
A cause du dégagement de chaleur latente, la température diminue moins vite pour une parcelle saturée en ascension que pour une parcelle non saturée. Le calcul pour l'atmosphère terrestre montre que
\[ \Gamma\e{saturé} = -6.5 \, \text{K~km}^{-1} \] 
ce qui correspond à la valeur observée dans la troposphère sur Terre. %[Figure~\ref{fig:tempvert}].

\sk
La constatation que~$\Gamma\e{saturé}$ correspond au profil d'environnement effectivement mesuré dans la troposphère appelle un commentaire important. Les profils verticaux secs ou saturés sont ceux suivis par une parcelle en ascension~: autrement dit, ils donnent les variations de~$T\e{p}$ avec l'altitude~$z$. D'un point de vue instantané, ils ne correspondent pas aux profils d'environnement~$T\e{e}$ tels qu'ils peuvent être par exemple mesurés par des ballons-sonde lâchés dans l'atmosphère. La parcelle n'est pas nécessairement à l'équilibre thermique avec l'environnement. On peut néanmoins constater sur la figure~\ref{fig:tempvert} que la température de l'environnement diminue avec une pente très proche de~$\Gamma\e{saturé}$. Ceci s'explique par le fait que cette figure montre une moyenne sur tout le globe à toutes les saisons. La situation moyenne ainsi décrite correspond aux mouvements d'une multitude de parcelles en ascension qui finissent par définir l'environnement atmosphérique\footnote{Ce phénomène porte le nom d'ajustement convectif.}. Pour comprendre la formation des nuages, et plus généralement les mouvements atmosphériques, il faut néanmoins se placer dans le cas local où l'équilibre thermique n'est pas vérifié. C'est l'objet de la section suivante.
%Comme pour le cas adiabatique, on peut aussi intégrer l'équation pour obtenir:
%\begin{equation} e_h=c_pT+gz+Lr=cste \label{estath} \end{equation}  
%La quantité $e_h$ est appelée {\em énergie statique humide} et est conservée
%pour des mouvements adiabatiques ($r$ et $e_s$ sont séparément conservés) ou
%saturés (pseudo-adiabatiques).


\newpage
\section{Formation d'un cumulonimbus sur Terre}
\sk
On s'intéresse ici au développement des nuages cumuliformes, en particulier les cumulonimbus. Le cas d'étude donné dans le radiosonsage exemple permet de suivre graphiquement les concepts de cette partie. Le point de départ est une parcelle d'air non saturée, c'est-à-dire dont l'humidité est inférieure à~$1$, située proche de la surface. On suppose que son rapport de mélange en vapeur d'eau~$r$ est conservé au cours de l'ascension. 

\sk
En premier lieu, de nombreux phénomènes atmosphériques vont provoquer une élévation de la parcelle que l'on considère initialement proche de la surface.
\begin{citemize}
\item{\textbf{Soulèvement frontal}} Un front est une variation marquée et localisée de température. Lorsqu'un front se déplace horizontalement, l'air chaud passe au-dessus de l'air froid de densité moindre. 
\item{\textbf{Soulèvement orographique}} La présence d'un relief face au vent force les parcelles d'air à s'élever.
\item{\textbf{Convection sèche}} Un sol très chaud l'après-midi peut induire un profil de température de l'environnement très instable proche de la surface. Dans ce cas, les mouvements verticaux sont amplifiés proche de la surface par la poussée d'Archimède (voir chapitre précédent).
%%% circulation thermique. brise de terre et brise de mer.
\end{citemize}
Tous ces mécanismes expliquent que des nuages cumuliformes sont souvent trouvés au-dessus de régions soumises au passage de fronts, montagneuses ou dont la surface est particulièrement chaude. Ces nuages évoluent parfois vers un état de type cumulonimbus.

\sk
En second lieu, lorsqu'une parcelle d'air est soulevée vers les plus hauts niveaux de l'atmosphère par les phénomènes atmosphériques précités, elle subit un refroidissement par détente adiabatique. Le taux de refroidissement de la parcelle est~$|\Gamma\e{sec}|$. Sur un émagramme tel celui de la figure~\ref{fig:sounding}, la parcelle suit une \voc{courbe adiabatique sèche}. Cette décroissance de la température de la parcelle au cours de l'ascension a pour principale conséquence d'abaisser la valeur de $r\e{sat}$, de par les variations exponentielles de cette quantité avec la température. Il en résulte que l'humidité relative~$H = r / r\e{sat}$ augmente. Si la quantité de vapeur d'eau initiale~$r$ et/ou le soulèvement de la parcelle sont suffisants, $H$ peut atteindre~$1$ au cours de l'ascension~: la parcelle devient alors saturée. Des gouttelettes nuageuses apparaissent par condensation, autrement dit un nuage se forme. Le niveau d'altitude ou de pression auquel la condensation se produit suite à un refroidissement par soulèvement adiabatique s'appelle le \voc{niveau de condensation} ou la \voc{base du nuage}. A ce stade, le nuage n'est pas encore nécessairement cumuliforme.
%%Pour la convection. Les bases des nuages sont horizontales, leurs sommets évoluent en fonction de la température.

\sk
En troisième lieu, si la parcelle continue son ascension au-delà du niveau de condensation, sa température ne décroît plus d'un taux~$|\Gamma\e{sec}|$, mais d'un taux $|\Gamma\e{saturé}|$ plus faible, puisque la parcelle est désormais saturée (son humidité vaut~$1$ et son rapport de mélange~$r$ vaut~$r\e{sat}$). Sur l'émagramme, la parcelle suit une \voc{courbe adiabatique saturée}. Au cours de l'ascension, la parcelle reste saturée mais, puisque sa température continue de diminuer, $r\e{sat}$ diminue de concours, ce qui induit une diminution du rapport de mélange en vapeur d'eau~$r$ et une augmentation du rapport de mélange en eau liquide (qui prend la forme de gouttelettes nuageuses ou, si les conditions de croissance sont réunies, de précipitations pluvieuses).

\sk
En quatrième lieu, la forme du profil de température d'environnement détermine si, une fois le niveau de condensation atteint, le nuage va suivre ou non un développement vertical marqué. On rappelle que le profil d'environnement n'est pas celui suivi par la parcelle considérée, mais représente l'état atmosphérique tel qu'il peut être mesuré par un ballon-sonde météorologique par exemple.

\begin{finger}

\item Si les soulèvements initiaux de la parcelle ne l'amènent que dans des niveaux atmosphériques où sa température reste plus faible que celle de l'environnement, alors il n'y a pas de mouvements verticaux spontanés au sein du nuage. Le nuage formé est plutôt de type stratiforme (ou faiblement cumuliforme).

\item Si les soulèvements initiaux de la parcelle parviennent à la hisser à des niveaux atmosphériques où sa température devient plus élevée que celle de l'environnement, alors des mouvements verticaux spontanés apparaissent au sein du nuage. On parle de convection humide (ou convection profonde). Le nuage ainsi formé est cumuliforme. Le niveau atmosphérique à partir duquel la température de la parcelle en ascension adiabatique devient plus élevée que la température de l'environnement s'appelle le \voc{niveau de convection libre}. Le niveau atmosphérique à partir duquel la température de la parcelle redevient plus faible que l'environnement s'appelle le \voc{sommet théorique du nuage}. Si le sommet théorique du nuage est très élevé, la formation de cumulonimbus, donc d'orage, est très probable. 

\end{finger}




%\newpage
%\section{Formation d'un cumulonimbus sur Terre (compléments)}
%\figside{0.4}{0.2}{decouverte/cours_meteo/anvil.png}{Vue lointaine d'un cumulonimbus à un stade avancé de développement, où l'on peut observer la structure aplatie en forme d'enclume au sommet du nuage. Source~: Wallace and Hobbs, Atmospheric Science, 2006~; d'après une photographie du Bureau Australien de Météorologie.}{fig:enclume}

\sk
L'étude de la formation des cumulonimbus appelle deux remarques importantes qui illustrent les concepts de stabilité et instabilité atmosphérique.

\begin{finger}

\item Le sommet des cumulonimbus atteint très fréquemment la tropopause. Lorsque c'est le cas, ils prennent alors une apparence aplatie et la forme d'enclume comme présenté dans la figure~\ref{fig:enclume}. Cela provient du fait que la stratosphère voit la température de l'environnement augmenter avec l'altitude, contrairement à ce qui peut se passer dans la troposphère. Un tel profil de température est extrêmement stable, donc a tendance à inhiber les mouvements verticaux. Ainsi, le fort développement vertical des cumulonimbus est stoppé net lorsque les couches stables de la stratosphère sont atteintes. En conséquence, le nuage s'étale selon l'horizontale au voisinage de la tropopause. C'est la raison pour laquelle le sommet réel des nuages cumuliformes est le minimum du sommet théorique des nuages et de la hauteur de la tropopause. Plus généralement, des conditions stables peuvent conduire à la formation de nuages stratiformes, ce qui nuance un peu la distinction faite dans la section~\ref{classphys}.

\item On entrevoit par les développement précédents qu'il est possible qu'une couche atmosphérique donnée, dont la température suit le taux de décroissance~$|\Gamma\e{env}|$ selon l'altitude, apparaisse comme stable si l'on considère l'ascension d'une parcelle non saturée, mais instable si l'on considère l'ascension d'une parcelle saturée. Cette situation se présente lorsque 
\[ |\Gamma\e{saturée}| <  |\Gamma\e{env}|  < |\Gamma\e{sec}| \]
On parle alors d'\voc{instabilité conditionnelle}. Il s'agit de conditions où seule l'apparition d'un nuage peut donner lieu à une instabilité et au développement de mouvements verticaux potentiellement étendus. Dans ce cas de figure, les nuages qui se forment sont essentiellement cumuliformes.

\end{finger}




\newpage
\section{Rayonnement thermique et corps noir}
\sk
Le Soleil qui se situe à une distance considérable dans le vide spatial nous procure une sensation de chaleur. De même, placer sa main sur le côté d'un radiateur en fonctionnement sans le toucher procure une sensation de chaleur instantanée qui ne peut être attribuée à un transfert convectif entre le radiateur et la main. Cet échange de chaleur est attribué au contraire à l'émission d'ondes électromagnétiques par la matière du fait de sa température; on parle d'émission de \voc{rayonnement thermique}. Tous les corps émettent du rayonnement thermique. La transmission de cette énergie entre une source et une cible ne nécessite pas la présence d'un milieu intermédiaire matériel. 
%Le but de cette section est d'en étudier les principales propriétés.

\sk
On appelle \voc{corps noir} un objet dont la surface est idéale et satisfait les trois conditions suivantes~:
\begin{description}
\item[émetteur parfait] un corps noir rayonne plus d’énergie radiative à chaque température et pour chaque longueur d’onde que n'importe quelle autre surface,
\item[absorbant parfait] un corps noir absorbe complètement le rayonnement incident selon toutes les directions de l'espace et toutes les longueurs d'onde,
\item[source lambertienne] un corps noir émet du rayonnement de façon isotrope
\end{description}

\sk
Un corps noir est à l'équilibre thermodynamique avec son environnement. On peut montrer qu'un tel corps émet du rayonnement qui dépend seulement de sa température et non de sa nature. La définition du corps noir, et les développements théoriques qui l'accompagnent, sont partis du constat, fait notamment par les céramistes, qu'un objet placé dans un four à haute température devient rouge en même temps que les parois du four quelle que soit sa taille, sa forme ou le matériau qui le compose. Un exemple de source utilisée pour étudier expérimentalement le modèle du corps noir consiste à construire une enceinte chauffée, totalement hermétique, et y percer un trou pour y mesurer le flux énergétique émis [figure~\ref{fig:four}]

\figside{0.35}{0.15}{\figwallace/Radiation/radiation_Page_10_Image_0001.png}{L'énergie entrant par une petite fente dans une enceinte subit des réflexions sur la paroi jusqu'à ce qu'elle soit absorbée. L'ouverture dans la paroi d'une enceinte chauffée apparaît comme une source de type corps noir. Un absorbant presque parfait est aussi un émetteur presque parfait. Ce type de four a été employé au début du XXe siècle pour évaluer expérimentalement les prédictions théoriques de Planck. Source~: Wallace and Hobbs, Atmospheric Science, 2006.}{fig:four}



\newpage
\section{Loi de Planck}
\sk
L'émission de rayonnement par le corps noir est décrite par une luminance énergétique spectrale~$L_{\lambda}$, notée $B_\lambda$ dans ce qui suit\footnote{Correspond au nom anglais \emph{blackbody}}. La loi de variation de~$B_\lambda$ selon la température~$T$ est donnée par la \voc{loi de Planck}\footnote{La luminance spectrale $B_\nu$ est déterminée d'une façon similaire. La démonstration de la loi de Planck fait appel à des notions de quantification d'énergie et de thermodynamique statistique qui sont hors programme dans le cadre de ce cours.} $$ B_\lambda(T) = \frac{C_1 \, \lambda^{-5}}{\pi \, \left( e^{ C_2 / \lambda T}-1\right) } $$ où $C_1$ et $C_2$ sont des constantes. Comme le rayonnement du corps noir est isotrope, l'émittance spectrale du corps noir, obtenue par intégration sur toutes les directions de l'espace, vaut $ M_\lambda(T) = \pi \, B_\lambda(T) $. 

%\figun{0.5}{0.25}{\figfrancis/WH_BBrad}{Courbes de luminance spectrale d'un corps noir pour différentes températures. La courbe en pointillés indique la position du maximum en fonction de $T$.}{fig:BBrad} 
\figside{0.5}{0.25}{\figwallace/Radiation/radiation_Page_11_Image_0001.png}{Courbes de luminance spectrale d'un corps noir pour différentes températures. La courbe en pointillés indique la position du maximum en fonction de $T$. Source~: Wallace and Hobbs, Atmospheric Science, 2006.}{fig:BBrad} 

\sk
Les variations de la fonction~$B_\lambda$ sont illustrées sur la figure~\ref{fig:BBrad}. L'émission de rayonnement par le corps noir ne dépend que de la longueur d'onde~$\lambda$ et de la température~$T$ du corps. A une température donnée, le rayonnement émis est parfaitement déterminé pour chaque longueur d'onde; dans un domaine spectral particulier, le rayonnement émis ne dépend que de la température du corps noir.




\newpage
\section{Loi de Wien du corps noir}
\sk
On observe sur la figure \ref{fig:BBrad} que, lorsque~$T$ augmente, la maximum de la luminance spectrale~$B_\lambda$, appelé \voc{maximum d'émission}, se décale vers les longueurs d'onde courtes, c'est-à-dire correspond à des photons de plus en plus énergétiques. La loi exacte, appelée \voc{loi de déplacement de Wien}, s'obtient en dérivant $B_\lambda$ par rapport à $\lambda$, ce qui permet d'obtenir $$ \boxed{ \lambda\e{max} \, T = 2.898 \times 10^{-3} \, \textrm{(mètres~K)} } $$ où $\lambda_{max}$ est la longueur d'onde du maximum de luminance spectrale~$B_\lambda$. La longueur d'onde du maximum d'émission~$\lambda\e{max}$ est ainsi inversement proportionnelle à la température du corps émetteur. Une formulation alternative est que $\nu\e{max}$ est proportionelle à $T$.

\figun{0.6}{0.45}{/home/aymeric/Big_Data/BOOKS/pierrehumbert_pics/9780521865562c03_fig001.jpg}{Source~: R. Pierrehumbert, Principles of Planetary Climates, CUP, 2010.}{wvl} 



\newpage
\section{Domaine de l'énergie solaire}
\sk
Le Soleil peut être considéré en bonne approximation comme un corps noir car il absorbe tout le rayonnement incident. Sa \ofg{couleur} est dûe à du rayonnement émis et, plus précisément, correspond aux longueurs d'onde où le maximum de rayonnement est émis. D'après la loi de Wien, le Soleil, dont l'enveloppe externe a une température autour de~$6000$~K, a donc un maximum d'émission situé dans le visible à $\lambda\e{max} = 0.5 \mu$m, proche du maximum de sensibilité de l'oeil humain [figure~\ref{fig:BBmax} haut]. Au contraire, la surface terrestre, dont la température typique est d'environ~$288$~K, voit son maximum d'émission situé dans l'infrarouge vers 10~$\mu$m, alors que le rayonnement émis dans les longueurs d'ondes visible est négligeable [figure~\ref{fig:BBmax} bas]. Un raccourci usuel est donc de dire que \ofg{la Terre émet du rayonnement (thermique) dans l'infrarouge alors que le Soleil émet dans le visible}. En toute rigueur, cette affirmation ne parle que du voisinage du maximum d'émission, où la contribution au flux intégré selon toutes les longueurs d'onde est la plus significative. Il est ainsi plus exact de dire que, dans l'atmosphère, la région du spectre où~$\lambda$ est inférieure à environ 4~$\mu$m est dominée par le rayonnement d'origine solaire, alors qu'au-delà, le rayonnement est surtout d'origine terrestre. Il n’y a pratiquement pas de recouvrement entre la partie utile du spectre du rayonnement solaire et celui d’un corps de température ambiante; ce fait est d'une grande importance pour les phénomènes de type effet de serre, qui sont abordés plus loin dans ce cours. On désigne ainsi souvent le rayonnement d'origine solaire par le terme \voc{ondes courtes} et le rayonnement d'origine terrestre par le terme \voc{ondes longues}.

\figsup{0.65}{0.2}{decouverte/cours_meteo/6000K.jpg}{decouverte/cours_meteo/earth.jpg}{Courbes de luminance spectrale d'un corps noir pour différentes températures correspondant notamment au Soleil (haut) et à la Terre (bas). La quantité représentée ici est l'émittance spectrale~$M_\lambda = \pi \, B_\lambda$. Noter la différence d'indexation de l'abscisse et l'ordonnée sur les deux schémas. Le rayonnement thermique émis par la Terre est plusieurs ordres de grandeur moins énergétique que celui émis par le Soleil et le maximum d'émission se trouve à des longueurs d'onde plus grandes (infrarouge pour la Terre au lieu de visible pour le Soleil). Source : \url{http://hyperphysics.phy-astr.gsu.edu/hbase/bbrc.html}.}{fig:BBmax}


\newpage
\section{Loi de Stefan-Boltzmann du corps noir}
\sk
La \voc{loi de Stefan-Boltzmann}\footnote{Joseph Stefan met expérimentalement en évidence en 1879 la dépendance de l'émittance en puissance quatrième de la température. Ludwig Boltzmann, à qui l'on doit également des résultats fondamentaux sur l'entropie et l'atomisme, prouve en 1884 le résultat par des arguments théoriques.} donne la valeur de l'intégrale sur toutes les longueurs d'ondes et dans tout l'espace\footnote{On entend par là toutes les directions du demi-espace extérieur au corps considéré.} de la courbe du corps noir, décrite par la loi de Planck et illustrée par les figures \ref{fig:BBrad} et \ref{fig:BBmax}. Cette loi donne donc l'expression d'une densité de flux énergétique~$F$ ou plus spécifiquement, puisque le corps noir est une source de rayonnement, d'une émittance totale~$M$. Cette dernière s'obtient tout d'abord avec une intégration par rapport à~$\lambda$ de la luminance énergétique spectrale~$B_\lambda$ donnée par la loi de Planck, afin d'obtenir la luminance énergétique~$B$. On déduit ensuite l'émittance totale~$M$ en intégrant selon toutes les directions de l'espace; comme le rayonnement du corps noir est isotrope, $M$ s'obtient à partir de~$B$ simplement en multipliant par~$\pi$. La loi de Stefan-Boltzmann établit que le flux net surfacique~$M$ émis par un corps noir ne dépend que de sa température par une dépendance type loi de puissance $$ \boxed{ M\e{corps noir} = \sigma \, T^4 } $$ avec~$\sigma=5.67 \times 10^{-8} \textrm{~W~m}^{-2}\textrm{~K}^{-4}$ appelée constante de Stefan-Boltzmann. La loi de Stefan-Boltzmann, comme la loi de Planck dont elle dérive, stipule que l'émittance~$M$ d'un corps pouvant être considéré en bonne approximation comme un corps noir ne dépend que de sa température et non de sa nature. Cette loi indique par ailleurs que l'émittance~$M$ augmente très rapidement avec la température -- de par la puissance quatrième impliquée.

\section{Corps gris et émissivité}
\sk
Le corps noir est un modèle idéal d'absorbant qu'en pratique on ne rencontre pas dans la nature. Par exemple, le charbon noir est un absorbant parfait, mais seulement dans les longueurs d'onde visible. La plupart des objets ressemblent néanmoins au corps noir, au moins à certaines températures et pour certaines longueurs d'onde considérées en pratique. Dans le cas d'un corps qui n'est pas un absorbant parfait, on parle d'un \voc{corps gris}. A température égale, un corps gris n'émet pas autant qu'un corps noir dans les mêmes conditions. Pour évaluer l'énergie émise par un corps gris par comparaison à celle qu'émettrait le corps noir dans les mêmes conditions, on définit un coefficient appelé \voc{émissivité} $\epsilon_\lambda$ compris entre~$0$ et~$1$ et égal au rapport entre la luminance spectrale du corps~$L_\lambda$ et celle du corps noir~$B_\lambda$: $ \epsilon_\lambda=L_\lambda / B_\lambda(T)$ En toute généralité, l'émissivité~$\epsilon_{\lambda}$ d'une surface à une longueur d'onde~$\lambda$ dépend de ses propriétés physico-chimiques, de sa température et de la direction d'émission\footnote{Par exemple, les métaux, matériaux conducteurs de l'électricité, ont une émissivité faible (sauf dans les directions rasantes) qui croît lentement avec la température et décroît avec la longueur d'onde ; au contraire, les diélectriques, matériaux isolant de l'électricité, ont une émissivité élevée qui augmente avec la longueur d'onde et se révèlent lambertiens sauf pour les directions rasantes où l'émissivité décroît significativement.}.

\sk
On peut définir une émissivité totale intégrée~$\epsilon$ qui permet d'exprimer l'émittance~$M$ d'un corps gris $$ \boxed{\SB} $$ Des valeurs de l'émissivité totale~$\epsilon$ pour certains matériaux sont données dans le tableau~\ref{tab:emiss}~: l'eau, la neige, les roches basaltiques ont des émissivités proches de~$1$ et sont donc des corps noirs en bonne approximation. 


\begin{table}[h!]
\label{tab:emiss}
\begin{center}
\footnotesize
\begin{tabular}{||c|c||c|c||c|c||}
\hline
Matériau & Emissivité~$\epsilon$ & Matériau & Emissivité~$\epsilon$ & Matériau & Emissivité~$\epsilon$ \\
\hline
Cuivre poli & 0.03 		& Cuivre oxydé & 0.5 		& Béton & 0.7 à 0.9 	\\
Carbone & 0.8 			& Lave (volcan actif) & 0.8 	& Suie & 0.95		\\
Ville & 0.85 			& Peinture blanche & 0.87 	& Peinture noire & 0.94 \\
Désert & 0.85 à 0.9 		& Herbe & 0.9 à 0.95		& Forêt & 0.95 		\\
Nuages cirrus & 0.10 à 0.90 	& Nuages cumulus & 0.25 à 0.99	& Eau & 0.92 à 0.97  	\\
Neige âgée & 0.8 		& Neige fraîche & 0.99		& &			\\
\hline
\end{tabular}
\normalsize
\caption{\emph{Quelques valeurs usuelles d'émissivité à la température ambiante (pour un rayonnement infrarouge). Source~: Hecht, Physique, 1999 -- avec quelques ajouts d'après site CNES}}
\end{center}
\end{table}



\newpage
\section{Constante solaire}
\sk
La distance Soleil-Terre est beaucoup plus grande que les rayons de la Terre et du Soleil. Ainsi, d'une part, le rayonnement solaire arrive au niveau de l'orbite terrestre en faisceaux pratiquement parallèles. D'autre part, la luminance en différents points de la Terre ne varie pas. On peut par conséquent définir une valeur moyenne de la densité de flux énergétique du rayonnement solaire au niveau de l'orbite terrestre, reçue par le système surface~+~atmosphère. Elle est désignée par le terme de \voc{constante solaire} notée~$\mathcal{F}\e{s}$. Les mesures indiquent que
\[ \mathcal{F}\e{s} = 1368 \text{~W~m}^{-2} \qquad \text{pour la Terre} \]

\sk
La constante solaire est une valeur instantanée côté jour~: le rayonnement solaire reçu au sommet de l'atmosphère en un point donné de l'orbite varie en fonction de l'heure de la journée et de la saison considérée (c'est-à-dire la position de la Terre au cours de sa révolution annuelle autour du Soleil)\footnote{En réalité, la constante solaire~$\mathcal{F}\e{s}$ varie elle-même d'environ~$3$~W~m$^{-2}$ en fonction des saisons à cause de l'excentricité de l'orbite terrestre, qui n'est pas exactement circulaire. De plus, elle peut varier évidemment en fonction des cycles solaires, néanmoins sans influence majeure sur la température des basses couches atmosphériques (troposphère et stratosphère).}. On peut donc définir un \voc{éclairement solaire moyen} noté~$\mathcal{F}\e{s}'$ reçu par la Terre qui intègre les effets diurnes et saisonniers. Autrement dit, $\mathcal{F}\e{s}$~est l'éclairement instantané reçu par un satellite en orbite autour de la Terre~; $\mathcal{F}\e{s}'$ est la valeur que l'on obtiendrait si l'on faisait la moyenne d'un grand nombre de mesures instantanées du satellite à diverses heures et saisons. 

\figside{0.5}{0.2}{decouverte/cours_dyn/incoming.png}{Energie reçue du Soleil par le système Terre. Source~: McBride and Gilmour, \emph{An Introduction to the Solar System}, CUP 2004.}{fig:eqrad}

\sk
On admet ici que~$\mathcal{F}\e{s}'$ peut être calculé en considérant que le flux total reçu du Soleil l'est à travers un disque de rayon le rayon~$R$ de la Terre (il s'agit de l'ombre projetée de la planète, voir Figure~\ref{fig:eqrad}). A cause de l'incidence parallèle, le flux énergétique intercepté par la Terre vaut donc~$\Phi = \pi \, R^2 \, \mathcal{F}\e{s}$. L'éclairement moyen à la surface de la Terre est alors $$\mathcal{F}\e{s}' = \frac{\Phi}{4 \, \pi \, R^2}$$ le dénominateur étant l'aire de la surface complète de la Terre. On obtient ainsi
\[ \boxed{ \mathcal{F}\e{s}' = \frac{\mathcal{F}\e{s}}{4} } \]

\sk
La valeur de la constante solaire peut s'obtenir par le calcul. Le soleil est considéré en bonne approximation comme un corps noir de température~$T_{\sun} = 5780$~K. D'après la loi de Stefan-Boltzmann, son émittance est $M = \sigma \, T_{\sun}^4$ donc le flux énergétique~$\Phi_{\sun}$ émis par le Soleil de rayon~$R_{\sun} = 7 \times 10^5$~km est~$\Phi_{\sun} = 4 \, \pi \, R_{\sun}^2 \, \sigma \, T_{\sun}^4$. Ce flux énergétique est rayonné dans tout l'espace~: à une distance~$d$ du soleil il est réparti sur une sphère de centre le soleil et de rayon~$d$, donc de surface~$4 \, \pi \, d^2$. A cette distance, l'éclairement~$\mathcal{F}$, c'est-à-dire la densité de flux énergétique reçue en W~m$^{-2}$, s'écrit donc
\[ \mathcal{F} = \frac{\Phi_{\sun}}{4 \, \pi \, d^2} = \frac{4 \, \pi \, R_{\sun}^2 \, \sigma \, T_{\sun}^4}{4 \, \pi \, d^2} = \sigma \, T_{\sun}^4 \, \left( \frac{R_{\sun}}{d} \right)^2 \]
Si l'on prend~$d$ égal à la distance Terre-Soleil, $\mathcal{F}$ définit ainsi la constante solaire~$\mathcal{F}\e{s}$.
%\[ \mathcal{F}\e{s} = \frac{{\mathcal{F}\e{s}}^{\text{Terre}}}{d\e{soleil}^2} \]

%Variation de la constante solaire : Bien que l’intensité du soleil ait subit des variations depuis la formation de la Terre, on peut s’attendre à ce qu’elle soit stable sur une période de 1000 ans. On mesure mal la constante solaire, mais les mesures récentes, même avec leurs incertitudes, semblent indiquer que le soleil ne peut pas expliquer le réchauffement récent. Notons toutefois que les simulations actuelles ne tiennent pas compte des fluctuations possibles du rayonnement solaire (négligeable a priori).
%%%% pas sûr du dernier point.


\newpage 
\section{Albédo} 
\sk
En sciences de l'atmosphère, les coefficients de réflexion~$\rho$ et~$\rho_{\lambda}$ sont souvent désignés sous le nom respectivement d'\voc{albédo} noté~$A$ et d'albédo spectral noté~$A_{\lambda}$. Plus la surface réfléchit une grande partie du rayonnement électromagnétique incident, plus l'albédo est proche de~$1$. L'albédo spectral~$A_{\lambda}$ peut varier significativement en fonction de la longueur d'onde : voir l'exemple de la neige fraîche donné ci-dessus. 

\sk
De par la diversité des surfaces terrestres, et de la variabilité de la couverture nuageuse, les valeurs de l'albédo~$A$ varient fortement d'un point à l'autre du globe terrestre~: il est élevé pour de la neige fraîche et faible pour de la végétation et des roches sombres [table~\ref{tab:albedo}]. L'albédo de l'océan est faible, particulièrement pour des angles d'incidence rasants -- il dépend ainsi beaucoup de la distribution des vagues. 

\begin{table}\label{tab:albedo}
\begin{center}
\begin{tabular}{|c|c|c|c|}
\hline
Type & albédo~$A$ & Type & albédo~$A$ \\
\hline
Surface de lac & 0.02 à 0.04 & Surface de la mer & 0.05 à 0.15 \\
Asphalte & 0.07 & Mer calme (soleil au zenith) & 0.10 \\
Forêt équatoriale & 0.10 & Roches sombres, humus & 0.10 à 0.15 \\
Ville & 0.10 à 0.30 & Forêt de conifères & 0.12 \\
Cultures & 0.15 à 0.25 & Végétation basse, verte & 0.17 \\
Béton & 0.20 & Sable mouillé & 0.25 \\
Végétation sèche & 0.25 & Sable léger et sec & 0.25 à 0.45 \\
Forêt avec neige au sol & 0.25 & Glace & 0.30 à 0.40 \\
Neige tassée & 0.40 à 0.70 & Sommet de certains nuages & 0.70 \\
Neige fraîche & 0.75 à 0.95 & & \\
\hline
\end{tabular}
\caption{\emph{Quelques valeurs usuelles d'albédo (rayonnement visible). D'après mesures missions NASA et ESA.}}
\end{center}
\end{table}

\sk
L'\voc{albédo planétaire} est noté~$A\e{b}$ et défini comme la fraction moyenne de l'éclairement~$E$ au sommet de l'atmosphère (noté également~$\mathcal{F}\e{s}'$) qui est réfléchie vers l'espace~: il comprend donc la contribution des surfaces continentales, de l'océan et de l'atmosphère. Il vaut~$0.31$ pour la planète Terre~: une partie significative du rayonnement reçu du Soleil par la Terre est réfléchie vers l'espace\footnote{L'albédo planétaire est par exemple encore plus élevé sur Vénus ($0.75$) à cause de la couverture nuageuse permanente et très réfléchissante de cette planète.}. Ainsi le système Terre reçoit une densité de flux énergétique moyenne~$F\e{reçu}$ en W~m$^{-2}$ telle que
\[ F\e{reçu} = (1-A\e{b}) \, \mathcal{F}\e{s}' \] 
donc un flux énergétique~$\Phi\e{reçu}$ (en W) qui s'exprime
\[ \Phi\e{reçu} = \pi \, R^2 \, (1-A\e{b}) \, \mathcal{F}\e{s} \]
%L'albédo de Bond~ désigne l'albédo intégré sur toutes les longueurs d'onde et tous les angles d'incidence.

\sk
La valeur de~$30\%$ de l'albédo planétaire sur Terre est en fait majoritairement dû à l'atmosphère~:  seuls 4\% de l'énergie solaire incidente sont réfléchis par la surface terrestre comme indiqué sur la figure~\ref{fig:diffsep}. L'énergie réfléchie par l'atmosphère vers l'espace, responsable de plus de~$85\%$ de l'albedo planétaire, est diffusée par les molécules ou par des particules en suspension, gouttelettes nuageuses, gouttes de pluie ou aérosols.

\figside{0.4}{0.15}{\figpayan/LP211_Chap2_Page_27_Image_0001.png}{L'énergie solaire incidente est réfléchie vers l'espace par la surface et l'atmosphère d'une planète. La figure montre les différentes contributions à l'albédo planétaire total.}{fig:diffsep}


\newpage
\section{\'Equilibre TOA}
\sk
Nous pouvons exprimer le rayonnement reçu du Soleil par la Terre par une densité de flux énergétique moyenne~$F\e{reçu}$ en W~m$^{-2}$ ou un flux énergétique~$\Phi\e{reçu}$ (en W)
\[ 
F\e{reçu} = (1-A\e{b}) \, \mathcal{F}\e{s}' 
\qquad \qquad
\Phi\e{reçu} = \pi \, R^2 \, (1-A\e{b}) \, \mathcal{F}\e{s}
\] 
La partie du rayonnement reçue du soleil qui est réfléchie vers l'espace est prise en compte via l'albédo planétaire noté~$A\e{b}$. On rappelle par ailleurs que~$\mathcal{F}\e{s}' = \mathcal{F}\e{s} / 4$ où $\mathcal{F}\e{s}$ est la constante solaire.


\sk
Par ailleurs, le système Terre émet également du rayonnement principalement dans les longueurs d'onde infrarouge [figure \ref{fig:eqrad2}]. 
Cette quantité de rayonnement émise au sommet de l'atmosphère radiative est notée $OLR$ pour \emph{Outgoing Longwave Radiation} en anglais.
A l'équilibre, la planète Terre doit émettre vers l'espace autant d'énergie qu'elle en reçoit du Soleil, donc
on obtient la relation générale appelée \emph{TOA} pour \emph{Top-Of-Atmosphere} en anglais, correspondant
au bilan radiatif au sommet de l'atmosphère
\[ \boxed{\TOA} \] 
La principale difficulté qui sous-tend les divers modèles pouvant être proposés réside dans l'expression du terme~$OLR$.




\newpage
\section{Bilan simple : température équivalente}
\sk
Dans l'équilibre~\emph{TOA}, la manière la plus simple de définir~$OLR$ pour entamer un calcul préliminaire est comme suit. On fait l'hypothèse, assez réaliste en pratique, que la surface de la Terre est comme un corps noir, c'est-à-dire que son émissivité est très proche de~$1$ dans l'infrarouge où se trouve le maximum d'émission. D'après la loi de Stefan-Boltzmann, la densité de flux énergétique~$F\e{émis}$ émise par la Terre en W~m$^{-2}$ s'exprime
\[ F\e{émis} = \sigma \, {T\e{eq}}^4 \]
où~T\e{eq} est la \voc{température équivalente} du système Terre que l'on suppose uniforme sur toute la planète. Autrement dit, $T\e{eq}$ est la température équivalente d'un corps noir qui émettrait la quantité d'énergie~$F\e{émis}$. Le flux énergétique~$\Phi\e{émis}$ émis par la surface de la planète Terre s'exprime
\[ \Phi\e{émis} = 4 \, \pi \, R^2 \, F\e{émis} = 4 \, \pi \, R^2 \, \sigma \, {T\e{eq}}^4 \]
Contrairement au cas de l'énergie visible, il n'y a pas lieu de prendre en compte le contraste jour/nuit, car le rayonnement thermique émis par la Terre l'est à tout instant par l'intégralité de sa surface. La seule limite éventuellement discutable est l'uniformité de la température de la surface de la Terre, ce qui est irréaliste en pratique. On peut souligner cependant que, même dans le cas d'une planète n'ayant pas une température uniforme ou ne se comportant pas comme un corps noir, le rayonnement émis vers l'espace doit être égal en moyenne à $\sigma \, {T\e{eq}}^4$.
%% CHANGER LES SLIDES, ne pas utiliser P

\figsup{0.31}{0.17}{decouverte/cours_dyn/incoming.png}{decouverte/cours_dyn/emission.png}{Equilibre radiatif simple : à gauche, l'énergie reçue du Soleil par le système Terre ; à droite, l'énergie émise par le système Terre. Source~: McBride and Gilmour, \emph{An Introduction to the Solar System}, CUP 2004.}{fig:eqrad2}

\sk
A l'équilibre, la planète Terre doit émettre vers l'espace autant d'énergie qu'elle en reçoit du Soleil (équilibre \emph{TOA}). Ceci peut s'exprimer par unité de surface
\[ \boxed{ F\e{reçu} = F\e{émis} } \]
ou, pour un résultat similaire, en considérant l'intégralité de la surface planétaire
\[ \Phi\e{reçu} = \Phi\e{émis} \]
ce qui permet de déterminer la température équivalente en fonction des paramètres planétaires
\[ \boxed{
T\e{eq} = \bigg[ \frac{\mathcal{F}\e{s}'\,(1-A\e{b})}{\sigma} \bigg]^{\frac{1}{4}}
} \]


Le calcul présenté ici porte le nom d'\voc{équilibre radiatif simple}. On y néglige les effets de l'atmosphère (sauf l'albédo) puisqu'on suppose que le rayonnement atteint la surface, ou est rayonné vers l'espace, sans être absorbé par l'atmosphère. La température équivalente est ainsi la température qu'aurait la Terre si l'on négligeait tout autre influence atmosphérique que la réflexion du rayonnement solaire incident. Les valeurs de $T\e{eq}$ pour quelques planètes telluriques sont données dans la table \ref{tab:planets}. On note que la température équivalente de Vénus est plus faible que celle de la Terre, bien qu'elle soit plus proche du Soleil, à cause de son fort albédo~; la formule indique bien que, plus le pouvoir réfléchissant d'une planète est grand, plus la température de sa surface est froide. Par ailleurs, comme indiqué par les calculs du tableau~\ref{tab:planets}, on remarque que la température équivalente, si elle peut renseigner sur le bilan énergétique simple de la planète, ne représente pas correctement la valeur de la température de surface. Par exemple, la température équivalente pour la Terre est~$T\e{eq} = 255 K = -18^{\circ}$C, bien trop faible par rapport à la température de surface effectivement mesurée. Il faut donc avoir recours à un modèle plus élaboré.

\begin{table}\label{tab:planets} \begin{center} \begin{tabular}{lccccc} & \emph{Mercure} & \emph{V\'enus} & \emph{Terre} & \emph{Mars} & \emph{ Titan} \\ \hline $d\e{soleil}$ (UA) & 0.39 & 0.72 & 1 & 1.5 & 9.5 \\ $\mathcal{F}\e{s}\,$(W~m$^{-2}$) & $8994$ & $2614$ & $1367$ & $589$ & $15$ \\ $A\e{b}$ & $0.06$ & $0.75$ & $0.31$ & $0.25$ & $0.2$ \\ \textcolor{blue}{$T\e{surface}$ (K)} & \textcolor{blue}{$100/700$~K} & \textcolor{blue}{$730$} & \textcolor{blue}{$288$} & \textcolor{blue}{$220$} & \textcolor{blue}{$95$} \\ \hline $T\e{eq}$~(K) & $439$ & $232$ & $254$ & $210$ & $86$\\ \end{tabular} \caption{\emph{Comparaison des facteurs influençant la température équivalente du corps noir pour différentes planètes du système solaire.}} \end{center} \end{table}
%    Mercure & 0.39 & 8994 & 0.06 & 439 \\
%    Vénus & 0.72 & 2639 & 0.78 & 225 \\
%    Terre & 1 & 1368 & 0.30 & 255 \\
%    Mars & 1.52 & 592 & 0.17 & 216 \\


\newpage
\section{\'Epaisseur optique}
\sk
Considérons une espèce~$X$ bien mélangée dans l'atmosphère, qui absorbe dans un intervalle de longueur d'onde donné. A la longueur d'onde~$\lambda$, son \voc{épaisseur optique}~$t_\lambda$ s'écrit
\[ \boxed{ t_\lambda = \int_{0}^{z\e{sommet}} \, k_\lambda \, \rho_X \, \dd z } \]
où $k_\lambda$ est un coefficient d'absorption massique en m$^2$~kg$^{-1}$ et $\rho_X$ est la densité d'absorbant~X. Le nom d'épaisseur optique est assez intuitif. Si un flux de rayonnement~$\Phi_\lambda$ à la longueur d'onde~$\lambda$ est émis à la base de l'atmosphère, le flux observé au sommet de l'atmosphère est d'autant plus réduit qu'à cette longueur d'onde l'épaisseur optique de l'atmosphère traversée est grande\footnote{Si l'extinction est uniquement due à de l'absorption, sans diffusion, on a une relation directe entre l'épaisseur optique et le coefficient d'absorption de la couche~: 
\[\alpha_\lambda = 1 - e^{- \frac{t_\lambda}{\cos\theta}} \] où~$\theta$ est l'angle d'incidence du rayonnement danns la couche.}. La formule ci-dessus ne fait qu'exprimer le fait que la réduction du flux (l'extinction) est plus d'autant plus marquée que 
\begin{citemize}
\item l'espèce considérée est très absorbante dans la longueur d'onde considérée ($k_\lambda$ grand)~;
\item l'espèce considérée est présente en grande quantité ($\rho_X$ grand).
\end{citemize}
Ainsi, le dioxyde de carbone~CO$_2$, bien qu'étant un composant minoritaire ($\rho$ faible), peut atteindre des épaisseurs optiques très grandes dans les intervalles de longueur d'onde où il est très fortement absorbant ($k_\lambda$ élevé), par exemple dans l'infrarouge autour de~$15$~$\mu$m. Autrement dit, un composant minoritaire en quantité peut avoir un rôle majoritaire radiativement.


\newpage
\section{Modèle à deux faisceaux : écriture}
\sk
Le modèle à deux faisceaux est un bon compromis entre simplicité
et illustration de concepts importants. Il est une version simplifiée
de l'équation de Schwarzschild du transfert radiatif.
Ce modèle entend élucider
les transferts de rayonnement dans l'infrarouge entre
les couches qui composent la colonne atmosphérique. Les
hypothèses simplificatrices suivantes sont réalisées
\begin{citemize}
\item couches atmosphériques plan-parallèle (sphéricité négligée)
\item phénomènes d'absorption négligés dans le visible (transparence au visible)
\item phénomènes de diffusion (\emph{scattering}) négligés dans l'infrarouge
\item \emph{gray gas} dans l'infra-rouge : on considère que le coefficient d'absorption
du gaz est indépendant de la longueur d'onde~$\lambda$ ($k_{\lambda} = k$ pour tout~$\lambda$),
ce qui implique une hypothèse similaire pour l'épaisseur optique ($\tau_{\lambda} = \tau$ pour tout~$\lambda$).
\end{citemize}
En d'autres termes, on se cantonne dans ce modèle à deux types de phénomènes
\begin{enumerate}
\item Un faisceau de rayonnement infra-rouge de flux~$F$ traversant une couche 
atmosphérique donnée
subit une extinction à cause de l'absorption selon une loi de type Beer-Lambert
\[
\dd F = - F \dd \tau
\]
avec~$\tau$ l'épaisseur optique \emph{gray gas} 
dans l'infra-rouge.
\item Une couche atmosphérique émet un flux de rayonnement thermique~$M$ 
calculé par la loi intégrée de Stefan-Boltzmann ($M=\epsilon\,\sigma\,T^4$)
puisque la majorité de l'émittance est émise dans l'infrarouge pour les températures considérées.
\end{enumerate}

\sk
L'épaisseur optique~$\tau$ peut servir de coordonnée verticale à la place de~$z$
en utilisant la relation entre les deux quantités. La couche atmosphérique
élémentaire considérée est ainsi d'épaisseur~$\dd\tau$ et située à une coordonnée
verticale~$\tau$ qui croît avec l'altitude. 

\sk
Si l'on considère un faisceau ascendant~$F^+(\tau)$ au bas de la couche considérée,
une fois la couche traversée son amplitude est
\[
F^+(\tau) - F^+(\tau) \dd\tau
\]
A ce flux au sommet de la couche, il convient d'ajouter
la contribution de l'émission thermique de la couche vers 
le haut, à savoir~$M\,\dd\tau$.
Le flux total ascendant au sommet de la couche est donc
\[
F^+(\tau+\dd\tau) = F^+(\tau) - F^+(\tau) \dd\tau + M\dd\tau
\]

\sk
Même raisonnement avec le flux descendant~$F^-(\tau+\dd\tau)$ au sommet de la couche considérée,
une fois la couche traversée son amplitude est~$F^-(\tau+\dd\tau) - F^-(\tau) \dd\tau$, où 
l'approximation du terme du second ordre~$F^-(\tau+\dd\tau) \dd\tau \simeq F^-(\tau) \dd\tau$
a été effectuée.
Le flux total descendant au bas de la couche est donc
\[
F^-(\tau+\dd\tau) = F^-(\tau) - F^-(\tau) \dd\tau + M\dd\tau
\]

\sk
Les deux résultats qui précèdent peuvent être transformés 
afin de faire apparaître une dérivée
en utilisant le théorème des accroissements finis
\[
\ddf{F^+}{\tau} = \frac{F^+(\tau+\dd\tau) - F^+(\tau)}{\dd\tau}
\]
\noindent ce qui permet d'obtenir au final
\[
\ddf{F^+}{\tau} = - F^+(\tau) + \epsilon\,\sigma\,T(\tau)^4 \quad [S^+]
\qquad\qquad 
\ddf{F^-}{\tau} = F^-(\tau) - \epsilon\,\sigma\,T(\tau)^4 \quad [S^-]
\]
\noindent $[S^+]$ et~$[S^-]$ sont parfois appelées les relations de Schwarzschild à deux faisceaux.
Il s'agit d'une version très simplifiée des équations de Schwarzschild du transfert radiatif.

\sk
Si l'on souhaite adopter la convention~$\tau=0$ au sommet de l'atmosphère,
et $\tau=\tau_{\infty}$ à la surface en $z=0$, donc adopter un axe
vertical d'épaisseur optique avec~$\tau$ croissant de haut en bas, il
suffit de remplacer~$\tau$ par~$-\tau$ dans les équations précédentes pour obtenir
\[
\boxed{\ddf{F^+}{\tau} = F^+(\tau) - \epsilon\,\sigma\,T(\tau)^4 \quad [S^+]} 
\qquad\qquad 
\boxed{\ddf{F^-}{\tau} = -F^-(\tau) + \epsilon\,\sigma\,T(\tau)^4 \quad [S^-]}
\]









\newpage
\section{Modèle à deux faisceaux : résolution 1}
\sk
Le système d'équations~$[S^+]$ et~$[S^-]$ du modèle à deux faisceaux
est plus simple à résoudre si l'on considère les deux quantités~$\Sigma(\tau)=F^{+}(\tau)+F^{-}(\tau)$ et~$\Delta(\tau)=F^{+}(\tau)-F^{-}(\tau)$ car on obtient
\[
\ddf{\Sigma}{\tau} = \Delta(\tau) \quad [E_\Sigma] 
\qquad\qquad 
\ddf{\Delta}{\tau} = \Sigma(\tau) - 2\,\epsilon\,\sigma\,T(\tau)^4 \quad [E_\Delta]
\]
\noindent Ensuite la résolution impose d'expliciter les conditions aux limites
\begin{enumerate}[label=$\mathcal{C}_\arabic*$]
\item on se place à l'équilibre radiatif donc le flux net~$\Delta$ est constant à tout niveau : $\ddf{\Delta}{\tau}=0$
\item au sommet de l'atmosphère $F^+(\tau=0) = OLR$ (définition de $OLR$) et $F^-(\tau=0) = 0$ (contribution
incidente négligeable du Soleil dans l'infra-rouge), ce qui s'écrit encore~$\Delta(\tau=0)=\Sigma(\tau=0)=OLR$
\item à la surface de température~$T_s$ le bilan radiatif est le suivant : la surface reçoit l'intégralité du rayonnement
solaire incident~$(1-A\e{b}) \, \mathcal{F}\e{s}'$ (visible) plus du rayonnement de l'atmosphère située
juste au-dessus d'elle~$F^-(\tau=\tau_{\infty})$ (infra-rouge) ; de plus elle émet un rayonnement
$\epsilon\,\sigma\,T\e{s}^4$ dans l'infra-rouge vers l'atmosphère\footnote{On a supposé ici pour simplifier les calculs que l'émissivité
de la surface était similaire à l'émissivité de l'atmosphère}
\item on rappelle que selon la relation \emph{TOA}, nous avons $OLR = (1-A\e{b}) \, \mathcal{F}\e{s}'$
\end{enumerate}


\newpage
\section{Modèle à deux faisceaux : résolution 2}
\sk
\paragraph{Conséquence 1} Il est alors possible d'obtenir deux expressions différentes pour~$\Sigma(\tau)$.
Premièrement, en utilisant $[E_\Sigma]$ avec $\mathcal{C}_1$ et $\mathcal{C}_2$, on obtient~$\Sigma(\tau)=OLR \, (1+\tau)$.
Deuxièmement, en utilisant $[E_\Delta]$ avec $\mathcal{C}_1$, on obtient~$\Sigma(\tau)=2\,\epsilon\,\sigma\,T(\tau)^4$.
On obtient le \voc{profil radiatif}, 
c'est-à-dire le profil vertical de température\footnote{Suivant la géométrie
équivalente choisie pour le modèle plan-parallèle, le terme $1+\tau$
peut s'écrire un peu différemment, mais quoiqu'il en soit toujours sous une
forme~$a+b\,\tau$ avec $a,b$ constants. Les conclusions énoncées ici ne sont pas
modifiées.} imposé par les transferts radiatifs dans l'infrarouge
\[
T(\tau) = \sqrt[4]{\frac{OLR\,(1+\tau)}{2\,\sigma\,\epsilon}}
\]

\sk
\paragraph{Conséquence 2} Reste à calculer la température de surface avec ce modèle. D'après $\mathcal{C}_3$, le bilan
au sol s'écrit~$(1-A\e{b}) \, \mathcal{F}\e{s}' + F^-(\tau=\tau_{\infty}) = \epsilon\,\sigma\,T\e{s}^4$.
Il faut donc exprimer les flux ascendant et descendant dans l'infrarouge.
Du fait que $\mathcal{C}_1$ et $\mathcal{C}_2$ nous indiquent que~$\Delta=OLR$, on obtient aisément
\[
F^+(\tau) = \frac{\Sigma+\Delta}{2} = OLR \, (1+\frac{\tau}{2})
\qquad \qquad
F^-(\tau) = \frac{\Sigma-\Delta}{2} = OLR \, \frac{\tau}{2}
\]
\noindent On obtient alors l'expression liant $OLR$
et température de surface~$T\e{s}$
\[
\boxed{\epsilon\,\sigma\,T\e{s}^4 = OLR \, \left( 1 + \frac{\tau_{\infty}}{2} \right)}
\]
\noindent On obtient ainsi une définition quantitative de \voc{l'effet de serre}
\begin{citemize}
\item Dans l'infrarouge, le rayonnement sortant au sommet de l'atmosphère ($OLR$)
est inférieur au rayonnement émis par la surface ($\epsilon\,\sigma\,T\e{s}^4$).
Une partie du rayonnement émis par la surface reste donc piégée par la planète.
\item Avec un albédo et un rayonnement incident constant, donc à~$OLR$ constant (d'après $\mathcal{C}_4$),
augmenter la quantité de gaz à effet de serre (donc augmenter~$\tau_{\infty}$)
conduit à une augmentation de la température de surface~$T\e{s}$.
\end{citemize}

\sk
\paragraph{Conséquence 3} Il est alors instructif de s'intéresser à la température atmosphérique 
proche de la surface~$T(\tau_\infty)$
donnée par le profil radiatif. Cette température ne dépend que de~$OLR$
et s'obtient totalement indépendamment de la température de surface.
On peut alors montrer que
\[
T\e{s} = T(\tau_\infty) \, \sqrt[4]{\frac{2+\tau_\infty}{1+\tau_\infty}} > T(\tau_\infty)
\]
\noindent Tant que l'atmosphère n'est pas
optiquement épaisse dans l'infrarouge (donc tant que~$\tau_\infty$ reste fini),
il existe une \voc{discontinuité entre la surface et l'atmosphère}, la surface
étant toujours plus chaude que l'atmosphère. Cela implique que l'atmosphère
est instable proche de la surface, donc que du mélange turbulent / convectif
apparaît, donc que l'équilibre proche de la surface ne peut être simplement
radiatif mais \voc{radiatif-convectif}. Notons que dans le cas où l'atmosphère est optiquement épaisse,
$T\e{s} = T(\tau_\infty)$, ce qui est tout à fait représentatif des conditions sur Vénus.


%%% figure Salby


\end{document}
