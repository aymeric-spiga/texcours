\documentclass[a4paper,DIV16,10pt]{scrartcl}
%%%%%%%%%%%%%%%%%%%%%%%%%%%%%%%%%%%%%%%%%%%%%%%%%%%%%%%%%%%%%%%%%%%%%%%%%%%%%%%%%%%
\usepackage{texcours}
%%%%%%%%%%%%%%%%%%%%%%%%%%%%%%%%%%%%%%%%%%%%%%%%%%%%%%%%%%%%%%%%%%%%%%%%%%%%%%%%%%%
\newcommand{\zauthor}{Aymeric SPIGA}
\newcommand{\zaffil}{Laboratoire de Météorologie Dynamique}
\newcommand{\zemail}{aymeric.spiga@upmc.fr}
\newcommand{\zcourse}{Physique des atmosphères planétaires}
\newcommand{\zcode}{UE9}
\newcommand{\zuniversity}{Sorbonne Université (Faculté des Sciences)}
\newcommand{\zlevel}{M2 Planétologie}
\newcommand{\zsubtitle}{Fiches complémentaires du cours 3}
\newcommand{\zlogo}{\includegraphics[height=1.5cm]{/home/aspiga/images/logo/LOGO_SU_HORIZ_SIGNATURE_CMJN_JPEG.jpg}}
\newcommand{\zrights}{Copie et usage interdits sans autorisation explicite de l'auteur}
\newcommand{\zdate}{\today}
%%%%%%%%%%%%%%%%%%%%%%%%%%%%%%%%%%%%%%%%%%%%%%%%%%%%%%%%%%%%%%%%%%%%%%%%%%%%%%%%%%%
\begin{document} \inidoc
%%%%%%%%%%%%%%%%%%%%%%%%%%%%%%%%%%%%%%%%%%%%%%%%%%%%%%%%%%%%%%%%%%%%%%%%%%%%%%%%%%%

\newpage
\section{Expression des forces en coordonnées sphériques}
\paragraph{L'équation fondamentale de la dynamique} dans le référentiel tournant
\[
\vec{\gamma}_r = -2 \, \vec{\Omega} \wedge \vec{U}_r + \vec{g} - \dfrac{\vec{\nabla}p}{\rho} + \vec{Fr}
\]

\paragraph{Composantes de l'acc\'el\'eration}
\[
\vec{\gamma}_r=\left[\begin{array}{ccc} u_t & - \dfrac{u\,v\tan\phi}{a} & + \dfrac{u\,w}{a}\\ & & \\ v_t & + \dfrac{u^2\tan\phi}{a} & + \dfrac{v\,w}{a}\\ & & \\ w_t & - \dfrac{u^2+v^2}{a} & \\ \end{array}\right] \qquad \text{notation} \quad u_t = \derd{u}{t}
\]

\paragraph{Composantes de la \ofg{force} de Coriolis}
\[
-2\,\vec{\Omega}\wedge\vec{U}_r=-2 \left[ \begin{array}{c} 0\\ \Omega\,\cos\phi\\ \Omega\,\sin\phi \end{array} \right] \wedge \left[ \begin{array}{c} u\\ v\\ w \end{array} \right] = \left[ \begin{array}{c} 2 \, \Omega \, ( v\sin\phi-w\cos\phi )\\ -2 \, \Omega \, u\sin\phi\\ 2 \, \Omega \, u\cos\phi \end{array} \right]
\]


\newpage
\section{Passage aux coordonnées sphériques}
\sk
Vitesse dans le référentiel tournant $\vec{U_r} = u\,\vec{i} + v\,\vec{j} + w\,\vec{k} $
\noindent Accélération (dérivée lagrangienne)
\[
\gamma_r = \derd{\vec{U_r}}{t}=\derd{u}{t}\,\vec{i}+\derd{v}{t}\,\vec{j}+\derd{w}{t}\,\vec{k} +u\,\derd{\vec{i}}{t}+v\,\derd{\vec{j}}{t}+w\,\derd{\vec{k}}{t}
\]

\figsup{0.35}{0.2}{decouverte/cours_dyn/didt1.png}{decouverte/cours_dyn/didt2.png}{Axe méridional, axe zonal}{fig:spher}

\sk
Décomposition sur les trois axes (zonal, méridien et vertical)
\[
\derd{\vec{i}}{t} = \left[ \derd{\vec{i}}{t} \right]_u + \left[ \derd{\vec{i}}{t} \right]_v + \left[ \derd{\vec{i}}{t} \right]_w
\]
\noindent Axe vertical
\[
\left[ \derd{\vec{i}}{t} \right]_w = \left[ \derd{\vec{j}}{t} \right]_w = \left[ \derd{\vec{k}}{t} \right]_w = \vec{0}
\]
\noindent Axe méridien \textcolor{magenta}{(aisé à expliquer)}
\[
\textcolor{magenta}{\left[ \derd{\vec{j}}{t} \right]_v = - \derd{\phi}{t} \, \vec{k} = - \dfrac{v}{a} \, \vec{k}}
\]
\[
\left[ \derd{\vec{i}}{t} \right]_v = \vec{0}
\qquad \qquad
\left[ \derd{\vec{k}}{t} \right]_v = + \dfrac{v}{a} \, \vec{j}
\]
\noindent Axe zonal
\[
\vec{j} = -\sin\phi \, \vec{m} + \cos\phi \, \vec{n} \qquad \vec{k} =  \cos\phi \, \vec{m} + \sin\phi \, \vec{n} \qquad \textrm{avec} \qquad \vec{m} = -\sin\phi \, \vec{j} + \cos\phi \, \vec{k}
\]
\[
\left[ \derd{\vec{n}}{t} \right]_u = \vec{0} \qquad \left[ \derd{\vec{m}}{t} \right]_u = \derd{\lambda}{t} \, \vec{i}
\]
\[
\left[ \derd{\vec{j}}{t} \right]_u = -\sin\phi \, \left[ \derd{\lambda}{t} \, \vec{i} \right] = \dfrac{-u}{a} \, \tan\phi \, \vec{i}
\]
\[
\left[ \derd{\vec{k}}{t} \right]_u =  \cos\phi \, \left[ \derd{\lambda}{t} \, \vec{i} \right] = \dfrac{u}{a} \, \vec{i}
\]
\[
\left[ \derd{\vec{i}}{t} \right]_u = - \derd{\lambda}{t} \, \vec{m} = \dfrac{u}{a} \tan\phi \, \vec{j} - \dfrac{u}{a} \,\vec{k}
\]



\newpage
\section{Eulérien vs. Lagrangien}
\sk
Comment caractériser un écoulement ?

\sk
\paragraph{Point de vue lagrangien} Le plus intuitif~:~Suivre les particules le long de leur trajectoire.
\paragraph{Point de vue eulérien} Le plus pratique~:~Suivre le courant depuis un point géométrique. Les points sont fixés ce qui est plus aisé en première approche pour modéliser l'écoulement sur une grille.
\centers{Variations lagrangiennes \quad = \quad Variations eulériennes \quad + \quad Terme d'advection}
Le terme d'advection transport concentre le caractère non-linéaire de la dynamique atmosphérique

\sk
On passe de l'un à l'autre des formalismes avec la formule de la dérivée d'une fonction composée~$\mathcal{F}[x(t)]$ où~$x$ est la position.
\[
\underbrace{\derd{\mathcal{F}}{t}}_{\text{En suivant la particule}}
= 
\underbrace{\Dp{\mathcal{F}}{t}}_{\text{En un point géométrique}} 
+ 
\underbrace{\left(\v U \cdot \v \nabla \right)\,\mathcal{F} }_{\text{Lié au déplacement de la particule}}
\]



\section{Equations complètes du mouvement}
\sk
L'équation fondamentale de la dynamique des fluides géophysiques 
en projection sur les coordonnées sphériques avec l'approximation de couche mince
s'écrit finalement

\begin{center}
\begin{tabular}{ccccccccc}
%%%%%%
\textcolor{blue}{$\ddf{u}{t}$} & 
\textcolor{brown}{$-\dfrac{uv\tan\phi}{a}$} & 
$+\dfrac{uw}{a}$ & 
= & 
\textcolor{red}{$2\Omega\sin\phi v$} & 
$-2\Omega \cos\phi w$ & 
\textcolor{green!75!black}{$-\dfrac{1}{\rho}\Dp{p}{x}$} & 
& 
$+Fr_x$\\
~\\
%%%%%%
\textcolor{blue}{$\ddf{v}{t}$} & 
\textcolor{brown}{$+\dfrac{u^2\tan\phi}{a}$} & 
$+\dfrac{vw}{a}$ & 
= & 
\textcolor{red}{$-2\Omega\sin\phi u$} &
&
\textcolor{green!75!black}{$-\dfrac{1}{\rho}\Dp{p}{y}$} &
&
$+Fr_y$\\
~\\
%%%%%%
$\ddf{w}{t}$ & 
$-\dfrac{u^2+v^2}{a}$ & 
&
=&
$2\Omega\cos\phi u$ & 
& 
\textcolor{green!75!black}{$-\dfrac{1}{\rho}\Dp{p}{z}$} & 
\textcolor{green!75!black}{$ -g$} & 
$+Fr_z$\\ 
\end{tabular}
\end{center}

\sk
Les termes s'interprètent comme suit
\begin{citemize}
\item{Pression} \textcolor{green!75!black}{$\bullet$} 
\item{Coriolis} \textcolor{red}{$\bullet$} où l'on définit~$f = 2 \, \Omega \, \sin\phi$
\item{Inertiels (sphéricité)} \textcolor{brown}{$\bullet$}
\item{Inertials (accélération)} \textcolor{blue}{$\bullet$}
\end{citemize}
Les termes en noir sont liés au déplacements verticaux et sont en général négligeables quand on considère la circulation générale de l'atmosphère et de l'océan.


%\newpage 
%\section{\'Echelles}
%\sk
Tous les termes de l'équation du mouvement n'ont pas la même importance lorsqu'on considère des mouvements atmosphériques de grande échelle. On définit donc des échelles caractéristiques du mouvement étudié. Pour simplifier, on choisit des échelles qui sont des puissances de 10.
\begin{description}
\item[longueur] Les échelles de longueur sont $L$ sur l'horizontale, et $H$ sur la verticale. Pour des mouvements qui s'étendent sur la hauteur de la troposphère, $H\sim 10$~km. $L$ peut varier beaucoup, mais l'échelle dite synoptique $L=$1000~km, qui est celle des perturbations des latitudes moyennes, est d'un intérêt particulier. La dernière échelle de longueur est celle du rayon de la Terre~$a$, qui est de l'ordre de 10000~km. 
\item[vitesse] Les échelles de vitesse horizontale et verticale sont notées $U$ et $W$. On a typiquement $U$=10~m~s$^{-1}$ dans l'atmosphère. Le rapport d'aspect du mouvement impose d'autre part que $W\le UH/L$.
\item[temps] L'échelle de durée du mouvement est construite à partir de celles de vitesse et de longueur: $T=L/U$. L'autre échelle de temps est celle liée à la rotation de la Terre, qui apparait dans le terme de Coriolis.
\item[variables thermodynamiques] Les variations des variables thermodynamiques $P,T,\rho$ sur la verticale sont celles des profils moyens donnés en introduction. En un point donné, les variations à l'échelle synoptique $\delta P,\delta T,\delta\rho$ sont de l'ordre de 1\% de la valeur moyenne.
\end{description}



%
%\newpage 
%\section{\'Echelles (évaluation)}
%
%	\sk \subsection{\'Echelles (mouvement vertical)}
%	\sk
L'ordre de grandeur des termes de l'équation du mouvement 
%\ref{eq:qtemvt} 
projetée sur la verticale (dirigée suivant \v k) est indiqué dans la table \ref{tab:vqmouv}. On voit que l'équilibre hydrostatique est vérifié avec une très bonne approximation\footnote{On peut noter qu'on vérifie également l'équilibre hydrostatique entre des anomalies de densité et des anomalies de variations de pression sur la verticale. Les termes $\rho g$ et $\partial P/\partial z$ sont alors cent fois plus faibles que pour l'état moyen, mais toujours supérieurs aux autres termes de l'équation.}. Notamment la composante verticale de la force de Coriolis~$\v F_C$ est négligeable devant~\v g et les forces de pression. Le seul autre terme qui peut devenir important est l'accélération relative~$dw/dt$, lors de mouvements verticaux intenses à petite échelle, comme dans un nuage d'orage ou près de topographie raide.  
%\begin{equation}
%  \frac{\partial P}{\partial z}=-\rho  g
%  \label{eq:hydro}
%\end{equation}

\begin{table}
  \centering
  \begin{tabular}{ccccccc}
    \hline
    Équation & $dw/dt$ & $-2\Omega u\cos\phi$ & $-\left(u^2+v^2\right)/a$ & = &
    $-\rho^{-1}\partial P/\partial z$ & $-g$ \\
    Échelle & $UW/L$ & $fU$ & $U^2/a$ && $P_0/(\rho_0H)$ & $g$ \\
    m.s\md & 10$^{-7}$ & 10$^{-3}$ & 10$^{-5}$  && 10 & 10 \\ 
    \hline
  \end{tabular}
  \caption{\emph{Analyse d'échelle de l'équation du mouvement vertical (avec
  $L$=1000~km et $W$=1~cm.s\mo).}}
  \label{tab:vqmouv}
\end{table}



%
%	\sk \subsection{\'Echelles (mouvement horizontal)}
%	\sk
Le détail de l'équation horizontale projetée en coordonnées sphériques est donné dans la table \ref{tab:hqmouv} pour $L$=1000~km. Sur les composantes horizontales (\v i, \v j), l'expression de la force de Coriolis se réduit aux contributions des mouvements horizontaux dans la mesure où~$W<<U$ pour des mouvements d'échelle supérieure à 10~km. 
\[\v F_C = \binom{f \, v}{-f \, u} \qquad \textrm{ou} \qquad \v F_C = -f \, \v k \wedge \v V_h \]
où $\v V_h = u \v i + v \v j$ est la vitesse horizontale et 
\[ \boxed{ f = 2 \, \Omega \, \sin \phi } \]
est appelé \voc{facteur de Coriolis}. Aux moyennes latitudes ($\phi=45$\deg), la valeur de~$f$ est environ~$10^{-4}$~s$^{-1}$. 
%Les composantes de la force de Coriolis sont \[\v F_C=-2\Omega\left(\begin{array}{c}0\\\cos\phi\\\sin\phi\end{array}\right) \wedge\left(\begin{array}{c}u\\v\\w\end{array}\right) =-2\Omega\left(\begin{array}{c}w\cos\phi-v\sin \phi\\u\sin \phi\\-u\cos \phi\end{array}\right)\]
%\footnote{Pour des mouvements de
%type ``chute libre'', la vitesse verticale $w$ domine. On peut alors mettre en
%évidence une déviation vers l'est, mais qui reste très faible (de l'ordre de
%1cm pour 80m de chute).} 

\begin{table}
  \centering
  \begin{tabular}{cccccccc}
    \hline
    Équation-$x$ & $\frac{du}{dt}$ & $-2\Omega v\sin\phi$ & $+2\Omega
    w\cos\phi$ & $+\frac{uw}{a}$ & $-\frac{uv\tan\phi}{a}$ &=&
    $-\frac{1}{\rho}\frac{\partial P}{\partial x}$ \\
    Équation-$y$ & $\frac{dv}{dt}$ & $+2\Omega u\sin\phi$ &&         
                $+\frac{vw}{a}$ & $+\frac{u^2\tan\phi}{a}$ &=&
    $-\frac{1}{\rho}\frac{\partial P}{\partial y}$ \\
    Échelles & $U^2/L$ & $fU$ & $fW$ & $UW/a$ & $U^2/a$ && $\delta P/(\rho L)$
    \\
    m.s\md & 10$^{-4}$ & 10$^{-3}$ & 10$^{-6}$ & 10$^{-8}$ & 10$^{-5}$ &&
    10$^{-3}$ \\
    \hline
  \end{tabular}
  \caption{\emph{Analyse en ordre de grandeur de l'équation du mouvement
  horizontale.}}
  \label{tab:hqmouv}
\end{table}

\sk
Sur un plan horizontal, les termes restants de l'équation du mouvement sont ainsi:
%\begin{equation}
\[  \frac{d\v V_h}{dt}+f\v k\wedge\v V_h=\v F_P  \]
%  \label{eq:hqmouv}
%\end{equation}
avec $\v V_h$ la vitesse horizontale, et $\v F_P$ les forces de pression horizontales massiques. Pour évaluer lequel des deux termes à gauche domine, on définit le \voc{nombre de Rossby} $\mathcal{R}$, rapport entre accélération relative et de Coriolis
\[ \mathcal{R} = \frac{U^2/L}{f\,U} = \frac{U}{f\,L} \]
Avec $f$=10$^{-4}$~s$^{-1}$ aux moyennes latitudes et $U$=10~m~s$^{-1}$, on a $\mathcal{R}=0.1$ aux grandes échelles de la circulation terrestre ($L$=1000~km), donc Coriolis domine. Au contraire, à une échelle plus petite de $L$=10~km, $\mathcal{R}=10$ et Coriolis devient négligeable.




\newpage
\section{Grands équilibres}
\sk
Suivant les termes dominants, on peut définir un certain nombre d'équilibres (stationnaires) ou de modèles / équations (pouvant servir à la prédiction de l'écoulement au cours du temps):
\begin{description}
\item{Equilibre hydrostatique} \DDD{\bullet} 
\item{Equilibre g\'eostrophique} \DDD{\bullet}\AAA{\bullet}
\item{Equilibre cyclostrophique} \DDD{\bullet}\CCC{\bullet}
\item{Equilibre du vent gradient} \DDD{\bullet}\AAA{\bullet}\CCC{\bullet}
\item{Modèle quasi-g\'eostrophique} \DDD{\bullet}\AAA{\bullet}\BBB{\bullet}
\item{Equations primitives} \DDD{\bullet}\AAA{\bullet}\BBB{\bullet}\CCC{\bullet}
\end{description}

\mk
\paragraph{Nombre de Rossby} Le nombre de Rossby permet d'évaluer l'importance relative de l'accélération de Coriolis, impulsée par la rotation de la planète, par rapport aux autres mouvements de rotation. Il permet de savoir si l'on se trouve dans le domaine de validité de l'équilibre géostrophique ou de l'équilibre cyclostrophique
\[
R_o=\f{\text{accélération horizontale (inertielle + sphéricité)}}{\text{accélération de Coriolis}}\qquad\boxed{R_o=\frac{U}{L\,\Omega}}
\]
\begin{table}[h!]
\begin{tabular}{cccc}
$R_o \ll 1$ & \DDD{\bullet}\AAA{\bullet} & Equilibre g\'eostrophique & [Terre, Mars]\\
$R_o \gg 1$ & \DDD{\bullet}\CCC{\bullet} & Equilibre cyclostrophique & [Vénus, Titan]\\
$R_o$~tous & \DDD{\bullet}\AAA{\bullet}\CCC{\bullet} & Equilibre du vent gradient & [Toutes]\\
~ & & & \\
$R_o \ll 1$ & \DDD{\bullet}\AAA{\bullet}\BBB{\bullet} & Modèle quasi-g\'eostrophique & [Terre, Mars]\\
$R_o$~tous & \DDD{\bullet}\AAA{\bullet}\BBB{\bullet}\CCC{\bullet} & Equations primitives  & [Toutes]
\end{tabular}
\end{table}

\mk
Sur les planètes à rotation rapide, l'équilibre géostrophique est le développement des équations du mouvement à l'ordre 1 en le nombre de Rossby, qui décrit un écoulement bidimensionnel, stationnaire et non divergent. A un ordre supérieur en $\textrm{Ro}$, l'évolution lente de la fonction de courant géostrophique peut être diagnostiquée par un nouvel équilibre dit quasi-géostrophique (QG). Couplé à l'équation de conservation de la vorticité potentielle de Rossby, le modèle approché QG a permis à Charney dans les années 50 de faire fonctionner sur un ordinateur le premier modèle de prévision numérique du temps.


\newpage
\section{\'Equilibre géostrophique}
\sk
Dans le cas d'un nombre de Rossby petit (donc $L$>1000~km aux moyennes latitudes), on est proche d'un équilibre appelé \voc{équilibre géostrophique} entre les forces de Coriolis et de pression
\[ \boxed{ \v F_C+\v F_P=\v 0 } \]
qui s'écrit selon les deux composantes horizontales
\[ \boxed{ \binom{f \, v}{-f \, u} = \binom{\frac{1}{\rho} \,\frac{\partial P}{\partial x}}{\frac{1}{\rho} \,\frac{\partial P}{\partial y}} } \]
Le vent qui vérifie exactement cet équilibre est appelé \voc{vent géostrophique}~$\v V_g$. Sous forme vectorielle on a $f\v k\wedge\v V_g=\v F_P$ et sous forme projetée
\[ \v V_g = \binom{u}{v} = \binom{- \frac{1}{\rho \, f} \, \frac{\partial P}{\partial y}}{\frac{1}{\rho \, f} \, \frac{\partial P}{\partial x}} \]
%\begin{equation}
%  \v V_g=\frac{1}{\rho f}\v k\wedge\vl{grad}_z(P)=\frac{g}{f}\v k\wedge\vl{grad}_P(Z)
%  \label{eq:geost}
%\end{equation}

\figun{1.1}{0.3}{\figfrancis/geost}{Forces et vent dans l'équilibre géostrophique (hémisphère nord).}{fig:geost}

\sk
L'équilibre géostrophique peut s'illustrer graphiquement (voir figure~\ref{fig:geost}), formant ce que l'on appelle la loi de Buys-Ballot\footnote{Comme l'indique Buys-Ballot dans son article de 1857~: \emph{Note sur le rapport de l'intensité et de la direction du vent avec les écarts simultanés du baromètre ; [...] Ce n'est pas la girouette, mais c'est le baromètre d'après lequel on doit juger le vent [...] La grande force du vent est annoncée par une grande différence des écarts simultanés du baromètre dans les Pays-Bas [...] Pour un autre pays, on devra étudier les modifications.}}.
\begin{description}
\item[Direction] Comme la force de Coriolis est orthogonale au vecteur vitesse, et opposée à la force de pression, le vent géostrophique est lui-même orthogonal aux variations horizontales de pression donc parallèle aux isobares.
\item[Sens] Dans l'hémisphère nord, les basses pressions sont à gauche du vent, à droite dans l'hémisphère sud.
\item[Module] La vitesse du vent géostrophique est proportionnelle aux variations horizontales de pression~; autrement dit, plus les isobares sont resserrées, plus le vent est fort.
\end{description}



\newpage
\section{Vent gradient}
\sk
Dans toutes les atmosphères connues, à grande échelle, un quasi-équilibre s'établit entre le champ de masse atmosphérique (lié au gradient de pression) et la composante horizontale de la force centrifuge (entraînement + Coriolis), liée au vent zonal~$u$ et 
\[
\v F\e{e} = - m \, \left( 2\,\Omega\,\sin\varphi\,u + \f{u^2\,\tan\varphi}{a} \right) \v j
\]
%% $F\e{e}$ s'oppose au gradient de pression pour un vent prograde~$u > 0$
\noindent Il s'agit de l'\voc{équilibre du vent gradient}, vrai en moyenne zonale (Figure~\ref{fig:vg}). 
\[
\boxed{\dfrac{u^2\tan\phi}{a} + 2\Omega\sin\phi u = -\dfrac{1}{\rho}\Dp{p}{y}}
\]
%\sk Le vent gradient est un équilibre diagnostic alors que les équilibres géostrophiques et cyclostrophiques sont des équations prognostiques

\figsup{0.35}{0.15}{decouverte/cours_dyn/td2_pression.png}{decouverte/cours_dyn/td2_centrifuge.png}{Gradient de pression (gauche). Force centrifuge (droite).}{fig:vg}

\sk
L'advection de l'air liée à la force de pression devient de moins en moins efficace quand on s'éloigne de l'équateur : l'équilibre du vent gradient provient d'un effet de frein exercé par la composante horizontale de la force centrifuge. Plus la planète tourne vite, plus cet effet de frein est prépondérant. Ainsi, la rotation de la planète contrôle l'extension en latitude des cellules de Hadley -- tout comme le déplacement en latitude du maximum saisonnier de température la contrôle~: 
\begin{finger}
\item L'extension limitée des cellules de Hadley sur Terre est majoritairement causée par la position du maximum saisonnier de température (qui reste confinée aux tropiques en raison de l'inertie thermique élevée des océans), alors que leur extension limitée sur les planètes géantes est le fait de leur rotation rapide.
\item L'extension des cellules de Hadley vers les pôles sur Mars est majoritairement causée par les effets de position du maximum saisonnier de température\footnote{Aux solstices, sous l'effet de la faible inertie thermique de la surface martienne et des constantes de temps radiatifs réduits dans l'atmosphère, la structure thermique de l'atmosphère martienne est composée d'un gradient de température d'un pôle à l'autre et conduit à une circulation de Hadley interhémisphérique. La circulation méridienne est particulièrement intense en raison du forçage diabatique des poussières en suspension dans l'atmosphère, surtout au solstice d'hiver nord où l'opacité moyenne des poussières atteint $1$ et l'insolation est maximale.}, alors que la grande extension vers les pôles des cellules de Hadley sur Vénus est avant tout reliée à la rotation lente de ce corps qui limite l'effet de frein de~$F\e{e}$.
\end{finger}

\sk 
Dans certains cas particuliers, notamment si~$u<0$ et~$u<2 \, \Omega \, a \, \cos \varphi$ (autrement dit, pour un courant-jet prograde dans le cas où la force de Coriolis domine la force d'entraînement), la force~$\v F\e{e}$ induit une accélération vers le pôle et non un frein. Cet effet peut favoriser l'extension vers les pôles des cellules de Hadley dans le cas de planètes à rotation rapide comme Mars. Exemple, au solstice d'hiver nord de Mars, une particule de vitesse zonale nulle partant d'un point de l'hémisphère sud de latitude $-\varphi_0$ (typiquement $60^{\circ}S$) et parcourant la branche haute de la cellule de Hadley adopte un mouvement rétrograde $u<0$ jusque la latitude opposée $\varphi_0$ par conservation du moment cinétique $\mathcal{M} = a \cos \varphi \left( \Omega \, a \, \cos \varphi + u \right)$. Contrairement au cas terrestre, la résultante des forces d'entraînement~$\v F\e{e}$ s'ajoute entre les latitudes $0$ et $\varphi_0$ au gradient de pression et la circulation méridienne s'intensifie jusqu'à la latitude $\varphi_0$, rejetant la limite des cellules de Hadley beaucoup plus loin que sur Terre. Entre les latitudes $-\varphi_0$ et $\varphi_0$, les isolignes du transport méridien de masse se confondent donc avec les isolignes du moment cinétique. Aux plus hautes latitudes $\varphi > \varphi_0$, dès que la vitesse zonale devient négative, le jet d'ouest se forme et la résultante~$\v F\e{e}$ s'oppose aux gradients de pression comme sur Terre.
%Seules les cellules de Hadley autour des équinoxes martiens, symétriques entre les deux hémisphères, ressemblent aux équivalents terrestres.









\newpage
\section{Géopotentiel et coordonnées isobares}
\sk
\paragraph{Coordonnées isobares} On note~$x$ la coordonnée sur l'axe est-ouest (axe zonal), $y$ la coordonnée sur l'axe sud-nord (axe méridional), $P$ la pression atmosphérique. On commence tout d'abord par considérer que la pression atmosphérique~$P$ remplace l'altitude comme coordonnée verticale~: on raisonne donc sur des surfaces isobares. La pression~$P$ peut être utilisée comme coordonnée verticale car monotone en vertu de l'équilibre hydrostatique. Le vent géostrophique zonal~$u$ s'exprime comme une fonction de~$x$, $y$, $P$, tout comme la température atmosphérique~$T$ et la masse volumiquede l'air~$\rho$. Les dérivées partielles (notées~$\partial$) de ces fonctions de trois variables se comprennent comme les dérivées selon la coordonnée indiquée avec les deux autres fixées. Par exemple $\frac{\partial T}{\partial y}$ est la dérivée de~$T$ uniquement selon la coordonnée~$y$, en considérant que~$x$ et~$P$ ne varient pas ; $\frac{\partial u}{\partial P}$ est la dérivée de~$u$ uniquement selon la coordonnée~$P$, avec les deux autres coordonnées fixées.

\sk
\paragraph{Géopotentiel} On définit le géopotentiel~$\Phi$ comme une fonction des coordonnées~$x$, $y$ et~$P$ qui s'écrit simplement
\[ \Phi(x,y,P)=g \, z(x,y,P) \] 
\noindent avec~$z$ l'altitude (également fonction des coordonnées~$x$, $y$ et~$P$) et~$g$ l'accélération de la gravité (supposée ici ne pas varier avec~$z$). 
%On rappelle que les dérivées partielles commutent, c'est-à-dire par exemple 
%\[ \frac{\partial}{\partial P} \frac{\partial \Phi}{\partial y} = \frac{\partial}{\partial y} \frac{\partial \Phi}{\partial P} \]

\sk
\paragraph{Dérivée verticale du géopotentiel} On utilise tout d'abord l'équilibre hydrostatique pour exprimer très simplement la dérivée du géopotentiel~$\Phi$ en fonction de la coordonnée verticale~$P$
\[ \frac{\partial \Phi}{\partial P} = g \, \frac{\partial z}{\partial P} = -\f{1}{\rho} \] 
\noindent ce qui permet de relier simplement les variations verticales de géopotentiel (sur les lignes isobares) au champ de masse. 

\sk
\paragraph{Dérivée horizontale du géopotentiel} On utilise une propriété de changement de coordonnée dans les dérivées partielles
\[ 
\left[ \frac{\partial P}{\partial y} \right]_z
\simeq
\left[ \frac{(P_0 + \delta p) - P_0}{\delta y} \right]_z 
=
\left[ \frac{(P_0 + \delta p) - P_0}{\delta z} \right]_y
\left[ \frac{\delta z}{\delta y} \right]_P
\simeq
-\left[ \frac{\partial P}{\partial z} \right]_y \left[ \frac{\partial z}{\partial y} \right]_P
\]
(le signe moins apparaît car~$P$ décroît avec l'altitude~$z$) pour exprimer très simplement la force de pression comme la dérivée spatiale du géopotentiel (en utilisant à nouveau au passage l'équilibre hydrostatique)
\[ -\frac{1}{\rho} \left[ \frac{\partial P}{\partial y} \right]_z
= \frac{1}{\rho} \left[ \frac{\partial P}{\partial z} \right]_y \left[ \frac{\partial z}{\partial y} \right]_P
= -g \left[ \frac{\partial z}{\partial y} \right]_P
= -\left[ \frac{\partial \Phi}{\partial y} \right]_P
\]

\sk
\paragraph{Bilan} En coordonnées isobares, 
le terme de gradient de pression horizontal devient irrotationnel
et dérive du géopotentiel~$\Phi$.


\newpage
\section{Equilibre du vent thermique}
\sk
On cherche à relier simplement un équilibre entre température et vent. Si on suppose l'équilibre du vent gradient vérifié, il donne déjà une relation entre champ de vent et champ de pression, donc seules quelques transformations de cette relation sont nécessaires. On note~$x$ la coordonnée sur l'axe est-ouest (axe zonal), $y$ la coordonnée sur l'axe sud-nord (axe méridional), $P$ la pression atmosphérique. Cette dernière est utilisée comme coordonnée verticale en vertu de l'équilibre hydrostatique.

\sk
On définit le géopotentiel~$\Phi$ comme une fonction des coordonnées~$x$, $y$ et~$P$ qui s'écrit simplement
\[ \Phi(x,y,P)=g \, z(x,y,P) \] 
\noindent avec~$z$ l'altitude (également fonction des coordonnées~$x$, $y$ et~$P$) et~$g$ l'accélération de la gravité. Le vent géostrophique zonal~$u$ s'exprime comme une fonction de~$x$, $y$, $P$, tout comme la température atmosphérique~$T$ et la masse volumiquede l'air~$\rho$. Les dérivées partielles (notées~$\partial$) de ces fonctions de trois variables se comprennent comme les dérivées selon la coordonnée indiquée avec les deux autres fixées. Par exemple $\frac{\partial \Phi}{\partial y}$ est la dérivée du géopotentiel~$\Phi$ uniquement selon la coordonnée~$y$, en considérant que~$x$ et~$P$ ne varient pas. 
%On rappelle que les dérivées partielles commutent, c'est-à-dire par exemple 
%\[ \frac{\partial}{\partial P} \frac{\partial \Phi}{\partial y} = \frac{\partial}{\partial y} \frac{\partial \Phi}{\partial P} \]

\sk
On utilise tout d'abord l'équilibre hydrostatique pour exprimer très simplement la dérivée du géopotentiel~$\Phi$ en fonction de la coordonnée verticale~$P$
\[ \frac{\partial \Phi}{\partial P} = g \, \frac{\partial z}{\partial P} = -\f{1}{\rho} \] 
\noindent ce qui permet de relier simplement les variations verticales de géopotentiel (sur les lignes isobares) au champ de masse.
On utilise directement ce résultat, combiné à une propriété de changement de coordonnée dans les dérivées partielles, pour exprimer très simplement la force de pression comme la dérivée spatiale du géopotentiel
\[ \frac{\partial \Phi}{\partial y} = \frac{\partial \Phi}{\partial P} \, \frac{\partial P}{\partial y} = -\frac{1}{\rho} \, \frac{\partial P}{\partial y} \]

\sk
Munis de cette expression simple de la force de pression, on peut alors modifier l'équilibre du vent gradient
\[ \dfrac{u^2\tan\phi}{a} + \fcoriolis u = -\dfrac{1}{\rho}\der{p}{y} = \frac{\partial \Phi}{\partial y} \]
\noindent que l'on peut ensuite dériver par rapport à la coordonnée verticale pression~$P$ 
\[ \left[ 2 \, u \, \dfrac{\tan\phi}{a} + \fcoriolis \right] \der{u}{P} = \der{~}{P} \left[ \frac{\partial \Phi}{\partial y} \right] \]
\noindent afin de pouvoir commuter les dérivées partielles puis utiliser la version de l'équilibre hydrostatique formulée ci-dessus avec le géopotentiel~$\Phi$
\[ \left[ 2 \, u \, \dfrac{\tan\phi}{a} + \fcoriolis \right] \der{u}{P} = \der{~}{y} \left[ \frac{\partial \Phi}{\partial P} \right] = \der{~}{y} \left[ \frac{1}{\rho} \right] \]
\noindent Reste à employer l'équation d'état des gaz parfaits~$P=\rho\,R\,T$ pour faire apparaître la température
\[ \left[ 2 \, u \, \dfrac{\tan\phi}{a} + \fcoriolis \right] \der{u}{P} = R \, \frac{\partial}{\partial y} \left[ \frac{T}{P} \right] 
\qquad \Rightarrow \qquad
\boxed{ \left[ \textcolor{brown}{2\,u\,\frac{\tan\phi}{a}} + \textcolor{red}{\fcoriolis} \right] \der{u}{P} = \frac{R}{P} \, \frac{\partial T}{\partial y} }
\]
\noindent L'expression encadrée de l'équilibre du vent thermique vient de la constatation finale que l'on peut sortir le terme en pression à l'intérieur de la dérivée à droite puisque~$P$ est une coordonnée supposée fixe par définition de la dérivée partielle suivant~$y$. Les termes sont colorés en fonction de l'équilibre dans lequel on se trouve : \textcolor{brown}{équilibre cyclostrophique} (exemple sur Vénus) ou \textcolor{red}{équilibre géostrophique} (exemple sur Mars ou la Terre). Dans le cas où la situation est ambigüe, il faut conserver les deux termes.

\sk
L'\voc{équilibre du vent thermique} exprime un lien diagnostique entre les variations verticales du vent zonal et les variations méridiennes de la température. Sur des planètes où la mesure de température est aisée à mesurer (e.g. par télédétection infrarouge) par rapport au vent, cet équilibre est employé pour calculer un champ de vent (appelé vent thermique) associé à un champ de température. Invariablement, l'équilibre du vent thermique peut permettre de déduire les variations de température associées à un vent donné. Il s'agit d'un équilibre entre deux champs~température$\leftrightarrow$vent, sans relation de causalité température$\rightarrow$vent ou vent$\rightarrow$température.



\newpage
\section{Instabilités barotropes et baroclines}
\sk
L'équation du vent thermique indique que les jets d'altitude dans la branche descendante de la cellule de Hadley (aux moyennes latitudes) conduisent à un renforcement des gradients latitudinaux de température, qui ne peuvent être résorbés par la circulation de Hadley. Aux moyennes latitudes, les énergies cinétique et thermique sont plus facilement redistribuées par les instabilités non-axisymétriques que par la circulation zonale axisymétrique. L'écoulement zonal axisymétrique des moyennes latitudes terrestres et martiennes peut ainsi donner naissance à une circulation non axisymétrique par des instabilités barotropes et baroclines. A partir d'une certaine latitude, ces instabilités sont bien plus efficaces pour redistribuer l'énergie.

\sk
Les perturbations \voc{barotropes} de l'écoulement moyen se développent en extrayant de l'énergie cinétique au cisaillement horizontal de vent de cet écoulement moyen (ex: courant-jet de forte amplitude). Les tourbillons barotropes ont une structure verticale constante avec l'altitude et transportent de la quantité de mouvement selon la latitude, afin de réduire le cisaillement qui leur a donné naissance.

\sk
L'instabilité \voc{barocline} résulte au contraire des gradients latitudinaux de température aux moyennes latitudes, associés à un cisaillement vertical de vent par l'équilibre du vent thermique. Les ondes baroclines générées transportent de la chaleur (et un peu de quantité de mouvement) en latitude et en altitude pour réduire l'inclinaison des isentropes qui leur a donné naissance. Les perturbations baroclines se développent par conversion de l'énergie potentielle disponible de l'écoulement zonal moyen en énergie cinétique.

%%\note{Stabilité des états d'équilibre~\donc~Réponse à une perturbation~:~s'amplifie-t-elle ? A chaque équilibre son instabilité associée~:~ exemple équilibre hydrostatique et instabilités convectives} \note{Courants-Jet d'altitude branche Hadley descendante~$+$~Renforcement gradients latitudinaux de température~\donc~Instabilités aux moyennes latitudes, Notion d'Energie Potentielle Disponible de l'écoulement zonal moyen.}




\newpage
\section{\'Equation de la vorticité}
\sk
La vorticité est définie généralement par le rotationnel du champ de vent
$\v \zeta = \v \nabla \wedge \v V$. La composante verticale de la vorticité~$\zeta$
quantifie la rotation du fluide dans le plan horizontal
\[ 
\zeta = \v k \cdot \left( \v \nabla \wedge \v u \right) 
\qquad\qquad \Rightarrow \qquad\qquad
\zeta = \Dp{v}{x} - \Dp{u}{y}
\]

\sk
Considérons les équations quasi-géostrophiques pour le mouvement horizontal,
en négligeant les termes de sphéricité pour plus de simplicité.
\[ \ddf{u}{t} - f\,v = -\frac{1}{\rho} \Dp{p}{x} \]
\[ \ddf{v}{t} + f\,u = -\frac{1}{\rho} \Dp{p}{y} \]
\noindent L'évolution 
de la vorticité relative 
selon la verticale
peut être obtenue simplement en
dérivant selon~$y$ l'équation du mouvement selon~$x$
(et vice versa) puis en soustrayant les deux termes.
L'on obtient alors l'\voc{équation de la vorticité}
\[
\ddf{\zeta}{t} + (\zeta+f) \, \v \nabla \cdot \vec v + \textcolor{red}{v \ddf{f}{y}}
= \frac{1}{\rho^2} \left( \Dp{\rho}{x} \Dp{p}{y} - \Dp{\rho}{y} \Dp{p}{x} \right)
\]
\noindent en notant que l'on a négligé les termes
$\Dp{w}{x}$ et $\Dp{w}{y}$ en se limitant au mouvement quasi-horizontal.
Le terme en \textcolor{red}{rouge} est en réalité égal à~$\ddf{f}{t}$
et correspond à l'\voc{effet~$\beta$} c'est-à-dire
l'effet sur l'écoulement de la variation avec la
latitude du paramètre de Coriolis.
Ainsi on obtient la notation plus compacte
\[
\ddf{(\zeta + f)}{t} + (\zeta+f) \, \v \nabla \cdot \vec v
= \frac{1}{\rho^2} \left( \Dp{\rho}{x} \Dp{p}{y} - \Dp{\rho}{y} \Dp{p}{x} \right)
\]
\noindent Le terme de droite est la projection verticale
du terme $\v \nabla p \wedge \v \nabla \rho$
appelé solénoïdal ou, plus classiquement, terme de production barocline.
Un fluide est appelé \voc{barocline} lorsque
ses surfaces d'égale masse volumique
ne sont pas alignées avec ses surfaces d'égale pression.
En pratique, un fluide est barocline
lorsqu'il y a des variations verticales de température
et de vent; sinon il est appelé \voc{barotrope}.


\sk
Un écoulement barotrope peut être considéré comme bi-dimensionnel (quasi-horizontal).
Alors l'équation de la vorticité s'écrit plus simplement
\[
\ddf{\xi}{t}
=
\Dp{\xi}{t} + u \Dp{\xi}{x} + v \Dp{\xi}{y}
=
- \xi \, \left( \Dp{u}{x} + \Dp{v}{y} \right)
\]
\noindent avec la vorticité absolue~$\xi = \zeta + f$
(rappelons qu'il s'agit de composantes verticales). 
Cette quantité se comprend comme une \og composée de vorticité \fg~entre
vorticité relative~$\zeta$ (calculée via le mouvement dans le référentiel tournant)
et vorticité planétaire~$f = 2\,\Omega\,\sin\phi$ (correspondant à la rotation de la planète).
Dans le cas d'un écoulement non-divergent, on obtient alors
\[
\ddf{\xi}{t} = 0
\qquad
\Rightarrow
\qquad
\Dp{\xi}{t} + u \Dp{\xi}{x} + v \Dp{\xi}{y}
=
0
\]
\noindent Il y a donc conservation de la vorticité totale~$\xi$
dans le cas d'un écoulement barotrope non-divergent.
Ce modèle très simplifié de l'évolution du fluide atmosphérique
est à la base des premiers modèles de prévision numérique du temps.
Par ailleurs, ce modèle très simple permet d'illustrer aisément
l'apparition d'ondes de Rossby dans l'écoulement 
sous l'effet de la conservation de la vorticité.

\sk
L'approche peut se généraliser au cas barocline, 
moyennant des calculs plus élaborés.
La quantité qui se conserve dans le cas général
est la \voc{vorticité potentielle d'Ertel} notée~$PV$
\[
PV = \left( \zeta_{\theta} + f \right) \left( -g \Dp{\theta}{p} \right)
\]
\noindent avec~$\zeta_{\theta}$ vorticité relative
calculée sur une surface isentrope (à~$\theta = \cte$).



\newpage
\section{Ondes de Rossby}
\sk
Les ondes de Rossby sont des ondes atmosphériques planétaires qui
naissent du fait de la variation du paramètre de Coriolis avec la latitude,
sous la contrainte de conservation de la vorticité totale~$\xi = \zeta + f$
avec~$f=2\Omega\sin\phi$ la vorticité planétaire.

\sk
\paragraph{Approche qualitative} Considérons une parcelle d'air à l'équateur, de vorticité relative~$\zeta = 0$.
Admettons qu'une perturbation déplace légèrement cette parcelle vers le nord.
La vorticité planétaire~$f$ de cette parcelle augmente.
La conservation de~$\xi = \zeta + f$ impose donc
que la vorticité relative~$\zeta$ de la parcelle devienne négative,
ce qui impose un mouvement tournant de sens des aiguilles d'une montre
au voisinage de la parcelle.
A gauche de la parcelle, un déplacement vers le nord se fait
donc sentir, donnant naissance à une nouvelle perturbation~$\zeta < 0$ par conservation de~$\xi$.
A droite de la parcelle, un déplacement vers le sud se fait
donc sentir, donnant naissance à une nouvelle perturbation~$\zeta > 0$ par conservation de~$\xi$.
Il en résulte ainsi une propagation conjointe vers l'ouest 
des perturbations~$\zeta > 0$ et~$\zeta < 0$,
caractérisant les \voc{ondes de Rossby}.

\sk
\paragraph{Approche quantitative}
Ce résultat qualitatif peut se retrouver quantitativement
dans un contexte barotrope quasi-horizontal non-divergent,
en utilisant la composante verticale de la vorticité absolue~$\xi = \zeta + f$
qui est conservée en suivant le mouvement horizontal
\[ 
\ddf{\xi}{t} = 0 
\qquad
\textrm{avec}
\qquad
\ddf{}{t} = \Dp{}{t} + u \Dp{}{x} + v \Dp{}{y}
\]
Pour illustrer aisément l'effet de~$\beta = \Dp{f}{y}$,
on se place dans un plan~$\beta$ tangent à la sphère localement
et pour lequel~$\beta = \cte$ (ainsi $f = f_0 + \beta\,y$).
On considère, pour réaliser une \voc{analyse linéaire},
un état de base~$\moyenne{\cdot}$ (écoulement zonal)
et une perturbation horizontale~$\cdot'$
\[
u = \moyenne{u} + u' 
\qquad 
v = v' 
\qquad 
\zeta = \Dp{v'}{x} - \Dp{u'}{y} = \zeta'
\]
\noindent L'équation barotrope de la vorticité s'écrit ainsi en version linéarisée
\[ 
\left( \Dp{}{t} + u \Dp{}{x} + v \Dp{}{y} \right) \zeta + \beta v = 0
\qquad
\Rightarrow
\qquad
\left( \Dp{}{t} + \moyenne{u} \Dp{}{x} \right) \zeta' + \beta v' = 0
\]

\sk
Dans le cas d'un écoulement bidimensionnel non divergent,
il est possible d'exprimer les coordonnées de la vitesse
à l'aide d'une \voc{fonction de courant}~$\Psi'$ telle que
%\[
%u' = - \Dp{\Psi'}{y} \qquad v' = \Dp{\Psi'}{x}
%\]
$u' = - \partial \Psi' / \partial y$ et $v' = \partial \Psi' / \partial x$
\noindent possédant la propriété intéressante suivante
\[
\zeta' = \nabla^2 \Psi'
\qquad
\textrm{avec}
\qquad
\nabla^2 \equiv \DDp{}{x} + \DDp{}{y}
\]
\noindent ce qui permet d'exprimer l'équation linéarisée de la vorticité barotrope
en fonction de la seule variable~$\Psi'$
\[
\left( \Dp{}{t} + \moyenne{u} \Dp{}{x} \right) \nabla^2 \Psi' + \beta \Dp{\Psi'}{x} = 0
\]
\noindent On cherche des \voc{solutions harmoniques}
de type onde monochromatique 
de vecteur d'onde horizontal~$(k,l)$ 
et de fréquence absolue $\omega$
s'écrivant
\[
\Psi' = Re \left[ \hat{\Psi} \, \exi{(kx+ly-\omega t)} \right]
\]
\noindent D'après l'équation linéarisée, les ondes
qui se propagent vérifient ainsi
\[
- \left( \omega + k \moyenne{u} \right) \left( - k^2 - l^2 \right) + k\,\beta = 0
\]
et donc leur fréquence intrinsèque~$\tilde{\omega} = \omega - k\moyenne{u}$ 
dans le référentiel attaché à l'écoulement de base est
\[
\boxed{
\tilde{\omega} = - \beta \, \f{k}{k^2+l^2}
}
\]
\noindent La relation de dispersion ainsi obtenue pour
les ondes de Rossby traduit à la fois leur
propagation systématique vers l'ouest
et l'importance centrale de l'effet~$\beta$ 
de variation du paramètre de Coriolis avec la latitude.


\sk
Les ondes de Rossby se développent sur Terre
dans les moyennes latitudes avec un nombre d'onde~$4-6$.
Elles peuvent également être responsables de structures 
sur les planètes géantes, telles l'hexagone sur Saturne par exemple.
La forme en Y du nuage global de Vénus pourrait être liée
à une onde de Rossby, mais d'autres types d'ondes 
(par exemple, ondes de Kelvin) pourrait être impliquées.


%\newpage
%\section{Note importante}
%\Huge
%Ce qui suit a été abordé rapidement
%(Boussinesq, ondes de gravité)
%ou n'a pas été abordé du tout.
%\normalsize

\newpage
\section{Instabilité barotrope}
\sk
Pour aborder l'instabilité barotrope de manière simplifiée,
on reprend l'équation de la vorticité barotrope 
linéarisée pour un écoulement quasi-horizontal,
employée pour décrire les ondes de Rossby barotropes
\[
\left( \Dp{}{t} + \moyenne{u} \Dp{}{x} \right) \zeta' + v' \Dp{f}{y} = 0
\]
\noindent L'unique différence est néanmoins que l'on considère ici
que le vent zonal de base~$\moyenne{u}$ a une dépendance latitudinale,
comme dans un courant-jet réel, ce qui conduit à une vorticité relative
de base non nulle~$\moyenne{\zeta}$, donc l'équation de la vorticité suivante
\[
\left( \Dp{}{t} + \moyenne{u} \Dp{}{x} \right) \zeta' + v' \Dp{(\moyenne{\zeta}+f)}{y} = 0
\]
En utilisant~$\Dp{\moyenne{\zeta}}{y} = -\DDp{u}{y}$,
et en définissant le potentiel~$\Psi'$ 
tel que~$u'=-\Dp{\Psi'}{y}$
et~$v'=\Dp{\Psi'}{x}$, on parvient à
\[
\left( \Dp{}{t} + \moyenne{u} \Dp{}{x} \right) \nabla^2 \Psi' + \left( \beta - \DDp{\moyenne{u}}{y} \right) \Dp{\Psi'}{x} = 0
\]

\sk
Considérons une solution monochromatique se propageant uniquement
selon la longitude (axe~$x$) et le temps
\[
\Psi'(x,y) = Re \left[ \hat{\Psi}(y) \, \exi{(k(x-c\,t)} \right]
\]
\noindent avec~$c$ une vitesse de phase complexe. Si la partie
imaginaire~$c_i$ de la vitesse de phase~$c$ est positive, l'onde
croît exponentiellement avec le temps, dénotant l'instabilité de l'écoulement.
On obtient alors une équation différentielle du second degré
\[
\left[ \moyenne{u} - c \right] \left( \ddf{^2\hat{\Psi}}{y^2} - k^2 \, \hat{\Psi} \right) + \left( \beta - \DDp{\moyenne{u}}{y} \right) \hat{\Psi} = 0
\]

\sk
Supposons que l'onde est confinée dans un canal entre $y=0$ et $y=L$ avec~$\hat{\Psi}=0$
(situation plutôt réaliste si l'on considère l'instabilité d'un jet).
En multipliant par le complexe conjugué~$\hat{\Psi}^*$ et en divisant par~$\moyenne{u}-c$,
la partie imaginaire de l'intégrale suivant~$y$ des termes ci-dessus
s'annule pour le premier terme, ce qui conduit à
\[
c_i \int_0^L \left( \beta - \DDp{\moyenne{u}}{y} \right) \f{|\hat{\Psi}|^2}{|\moyenne{u}-c|^2} \dd y = 0
\]
Une instabilité ne peut se développer ($c_i > 0$) que
si l'intégrale est nulle, ce qui n'est possible que si
\[
\boxed{
\beta - \DDp{\moyenne{u}}{y} \quad \textrm{change de signe dans le canal } [0,L]
}
\]
\noindent Le critère ci-dessus est le critère d'instabilité barotrope de Rayleigh-Kuo.
Un courant-jet barotrope peut ainsi devenir instable 
si sa courbure selon la latitude est particulièrement marquée.
L'instabilité barotrope peut survenir
sous les tropiques terrestres,
dans les moyennes latitudes sur Vénus,
ou le courant circumpolaire martien.
Dans les courants-jets des planètes géantes,
l'instabilité barotrope est également probable,
notamment dans les jets dirigés vers l'ouest.
Elle pourrait jouer un rôle également
dans le développement de la forme hexagonale de l'hexagone de Saturne.






\end{document}

\newpage
\section{Approximation de Boussinesq et force de flottaison}
\sk
Un fluide de Boussinesq est un fluide pour lequel les variations de densité peuvent être négligées 
sauf lorsqu'elles apparaissent dans des termes dont $g$, l'accélération de la gravité, est facteur.
La plupart des phénomènes atmosphériques, 
qu'ils soient de grande échelle (mouvement quasi-géostrophique),
de méso-échelle (fronts météorologiques, vents catabatiques, ondes de gravité)
ou 
de micro-échelle (convection de couche limite)
peuvent être considérés comme des mouvements
d'un fluide de Boussinesq, pour lequel $\rho=\cte$
sauf dans le terme de flottabilité verticale.
En d'autres termes, on ne retient que l'impact sous l'action de la gravité
de la stratification en densité du fluide sur les mouvements verticaux.
L'équation de continuité s'écrit alors
\[ \v \nabla \cdot \v v = 0 \]
\noindent Cette approximation est valable dans le cas de mouvements  
verticaux relativement confinés (d'extension verticale de l'ordre de $H$).

\sk
On se place dans le cadre d'une analyse linéaire de perturbations
autour d'un état de base moyen qui vérifie l'équilibre hydrostatique.
\[ \rho = \rho_0 + \rho' \qquad p = \moyenne{p}(z) + p' \]
L'approximation de Boussinesq permet 
de remplacer~$\rho$ par~$\rho_0$ partout, sauf 
lorsqu'on considère les deux termes dominants 
de l'équation du mouvement vertical -- car ils font apparaître la force de flottaison.
\[ 
\f{1}{\rho} \Dp{p}{z} + g 
= 
\f{1}{\rho_0 + \rho'} \left( \Dp{\moyenne{p}}{z} + \Dp{p'}{z} \right) + g  
\]
\noindent Au premier ordre
\[ 
\f{1}{\rho_0 + \rho'} 
= 
\f{1}{\rho_0 \left( 1 + \f{\rho'}{\rho_0} \right) } 
\simeq 
\f{1}{\rho_0} \left( 1 - \f{\rho'}{\rho_0} \right)
\]
\noindent De plus, l'état de base vérifie l'équilibre hydrostatique
\[ 
\f{1}{\rho_0} \Dp{\moyenne{p}}{z} + g = 0
\]
\noindent Ainsi on parvient à
\[ 
\f{1}{\rho} \Dp{p}{z} + g 
= 
\f{1}{\rho_0} \Dp{p'}{z} - g \f{\rho'}{\rho_0}
\]
\noindent Le second terme est l'expression de la force de flottaison (la poussée d'Archimède).
En utilisant l'équation d'état, et en introduisant le température
potentielle~$\theta = \moyenne{\theta} + \theta'$, on a
\[
\rho' \simeq - \rho_0 \f{\theta'}{\moyenne{\theta}} + \f{p'}{c_s^2}
\]
\noindent avec~$c_s$ la célérité du son, bien plus rapide que
l'écrasante majorité des mouvements atmosphériques, ce qui donne
une autre expression de la force de flottaison
\[
-\f{\rho'}{\moyenne{\rho}} = \f{\theta'}{\moyenne{\theta}}
\]
\noindent Ce qui montre qu'on peut raisonner indifféremment
sur les perturbations de densité ou de température potentielle.
Les premières sont plutôt utilisées en océanographie,
les secondes en météorologie.

%force de flottaison est une perturbation de l'équilibre hydrostatique.


\newpage
\section{Système complet pour la modélisation}
\sk
Bjerknes, 1904~:~6 équations pour 6 inconnues 
\begin{finger}
\item variables \textcolor{red}{dynamiques} ou \textcolor{brown}{thermodynamiques}
\item ce qui dépend de la planète considérée~:~\textcolor{blue}{forçages} et \textcolor{green!75!black}{constantes planétaires} 
\item rappel: formalisme eulérien vs. lagrangien $\derd{\mathcal{F}}{t}=\der{\mathcal{F}}{t}+\textcolor{red}{u}\der{\mathcal{F}}{x}+\textcolor{red}{v}\der{\mathcal{F}}{y}+\textcolor{red}{w}\der{\mathcal{F}}{z}$
\end{finger}

\sk
\noindent Les 6 équations qui permettent d'évaluer l'évolution déterministe du fluide atmosphérique soumis aux forçages
\begin{enumerate}
\item Mouvement horizontal
\[ \derd{\textcolor{red}{u}}{t} - \dfrac{\textcolor{red}{u}\textcolor{red}{v}\tan\phi}{\textcolor{green!75!black}{a}} = 2\textcolor{green!75!black}{\Omega}\sin\phi \, \textcolor{red}{v} - \dfrac{1}{\textcolor{brown}{\rho}} \, \der{\textcolor{brown}{p}}{x} + \textcolor{blue}{F_u} \]
\[ \derd{\textcolor{red}{v}}{t} + \dfrac{\textcolor{red}{u}^2\tan\phi}{\textcolor{green!75!black}{a}} = -2\textcolor{green!75!black}{\Omega}\sin\phi \, \textcolor{red}{u} - \dfrac{1}{\textcolor{brown}{\rho}} \, \der{\textcolor{brown}{p}}{y} + \textcolor{blue}{F_v} \]
\item Equilibre hydrostatique vertical
\[
 - \dfrac{1}{\textcolor{brown}{\rho}} \, \der{\textcolor{brown}{p}}{z} - \textcolor{green!75!black}{g} = 0
\]
\item Conservation de la masse
\[
\der{\textcolor{brown}{\rho} }{t} + \div\dep{\textcolor{brown}{\rho} \textcolor{red}{\vec{V}}}=0
\]
\item Premier principe 
\[
\f{\textcolor{green!75!black}{c_p}}{\theta} \, \derd{\theta}{t} = \f{\textcolor{blue}{\mathcal{Q}}}{\textcolor{brown}{T}}
\qquad \text{avec} \qquad
\theta=\textcolor{brown}{T} \, \left[ \f{\textcolor{green!75!black}{p_0}}{\textcolor{brown}{p}} \right]^{\textcolor{green!75!black}{\kappa}} 
\]
\item Gaz parfait
\[
\textcolor{brown}{p} = \textcolor{brown}{\rho} \, \textcolor{green!75!black}{R} \, \textcolor{brown}{T}
\]
\end{enumerate}



\newpage
\section{Ondes de gravité}

\sk
Les ondes de gravité sont des oscillations atmosphériques
causés par le rappel de la force de flottaison.
%On se place pour simplifier dans un cas bidimensionnel~$(x,z)$.
L'approximation de Boussinesq est respectée.
L'écoulement est supposé adiabatique.
On réalise une analyse linéaire
\[
\rho =\rho_0 + \rho' 
\qquad
p = \moyenne{p}(z) + p' 
\qquad
\theta = \moyenne{\theta}(z) + \theta'
\qquad
u = \moyenne{u}(z) + u'
\qquad
v = \moyenne{v}(z) + v'
\qquad
w = w'
\]
\noindent où l'état de base vérifie les équations du mouvement.
L'écoulement moyen est supposé purement horizontal, avec~$\moyenne{w}=0$.

\sk
Les équations primitives se réduisent à
\[ 
\Dp{u'}{t} 
+ \moyenne{u} \Dp{u'}{x} 
+ \moyenne{v} \Dp{u'}{x} 
\textcolor{lightgray}{+ w' \Dp{\moyenne{u}}{z}}
- f v' 
+ \f{1}{\rho_0} \Dp{p'}{x}
= 0
\]
\[ 
\Dp{v'}{t} 
+ \moyenne{u} \Dp{v'}{x} 
+ \moyenne{v} \Dp{v'}{x} 
\textcolor{lightgray}{+ w' \Dp{\moyenne{v}}{z}}
+ f u' 
+ \f{1}{\rho_0} \Dp{p'}{y}
= 0
\]
\[ 
\Dp{w'}{t} 
+ \moyenne{u} \Dp{w'}{x} 
+ \moyenne{v} \Dp{w'}{x} 
+ \f{1}{\rho_0} \Dp{p'}{z}
- g \f{\theta'}{\moyenne{\theta}}
= 0
\]
\[ 
\Dp{u'}{x} 
+ \Dp{v'}{y}
+ \Dp{w'}{z}
= 0
\]
%% dernier: terme en D/Dt (rho'/rho_0) ??
\[ 
\Dp{\theta'}{t} 
+ \moyenne{u} \Dp{\theta'}{x} 
+ \moyenne{v} \Dp{\theta'}{x} 
+ w' \Dp{\moyenne{\theta}}{z}
= 0
\]
\noindent Les termes en grisé sont ici négligés pour simplifier. 
Ils représentent la possibilité d'avoir un vent d'environnement
variant avec l'altitude~$z$.


\sk
Pour déterminer les propriétés des ondes de gravité 
qui apparaissent dans l'atmosphère,
on poursuit l'analyse linéaire en substituant
aux perturbations une solution
de type onde monochromatique 
de vecteur d'onde 
$(k=\f{2\pi}{\lambda_x},l=\f{2\pi}{\lambda_y},m=\f{2\pi}{\lambda_z})$ 
et de fréquence absolue $\omega$, 
(autrement dit, un élément harmonique d'un développement de Fourier)
\[
\mathcal{F}'
= 
Re \left[ 
\hat{\mathcal{F}} \, \exi{(kx+ly+mz-\omega t)} 
\right]
\quad \textrm{avec} \quad 
\mathcal{F}' \equiv u', v', w', \f{p'}{\rho_0}, \f{\theta'}{\moyenne{\theta}}
\]
\noindent Ainsi les équations ci-dessus deviennent 
des équations dites de polarisation 
(5 équations pour 5 inconnues $\hat{u} \, \hat{v} \, \hat{w} \, \hat{p} \, \hat{\theta}$), 
où~$\tilde{\omega} = \omega - k \moyenne{u} - l \moyenne{v}$
est la \voc{fréquence intrinsèque} de l'onde considérée
dans un référentiel attaché à l'écoulement moyen
de composantes $\moyenne{u}$ et~$\moyenne{v}$

\begin{minipage}{.32\linewidth}
\begin{equation}\label{polau}
- \ir \, \tilde{\omega} \, \hat{u}
- f \hat{v}
+ \ir \, k \, \hat{p} 
= 0
\end{equation}
\begin{equation}\label{polav}
- \ir \, \tilde{\omega} \, \hat{v}
+ f \hat{u}
+ \ir \, l \, \hat{p} 
= 0
\end{equation}
\end{minipage}
\begin{minipage}{.32\linewidth}
\begin{equation}\label{polaw} 
- \ir \, \tilde{\omega} \, \hat{w}
+ \ir \, m \, \hat{p} 
- g \hat{\theta}
= 0
\end{equation}
\begin{equation}\label{poladiv}
\ir \, k \, \hat{u}
+ \ir \, l \, \hat{v}
+ \ir \, m \, \hat{w}
= 0
\end{equation}
\end{minipage}
\begin{minipage}{.32\linewidth}
\begin{equation}\label{polat}
- \ir \, \tilde{\omega} \, g \, \hat{\theta}
+ N^2 \, \hat{w}
= 0
\end{equation}
$$ \qquad \textrm{avec} \quad N^2 = \f{g}{\moyenne{\theta}} \Dp{\moyenne{\theta}}{z}$$ % = g \Dp{\ln\theta}{z} $$
\end{minipage}

\sk
$N^2$ est appelée la fréquence de Brunt-V{\"a}is{\"a}l{\"a}.



\newpage
\section{Ondes de gravité : relation de dispersion}
\sk
Nous allons résoudre ce système 
en tentant d'obtenir une relation entre~$\tilde{\omega},k,l,m$.
Pour cela, obtenons une équation contenant uniquement~$\hat{w}$
Une manipulation type~$k \ref{polau} + l \ref{polav}$ donne
\[
- \ir \, k \, \tilde{\omega} \, \hat{u}
- f \, k \, \hat{v}
+ \ir \, k^2 \, \hat{p} 
- \ir \, l \, \tilde{\omega} \, \hat{v}
+ f \, l \, \hat{u}
+ \ir \, l^2 \, \hat{p} 
= 0
\]
\noindent Les deux termes inertiels $\ir \, k \, \tilde{\omega} \, \hat{u}$ et~$\ir \, k \, \tilde{\omega} \, \hat{v}$ 
se rassemblent en utilisant l'équation~\ref{poladiv} liée à la divergence
\[
\ir \, m \, \tilde{\omega} \, \hat{w}
- f \, k \, \hat{v}
+ f \, l \, \hat{u}
+ \ir \, (k^2 + l^2) \, \hat{p}
= 0
\]
\noindent Le terme en pression peut être lié à~$\hat{w}$ et~$\hat{\theta}$ grâce à l'équation~\ref{polaw}
\[
\ir \, m \, \tilde{\omega} \, \hat{w}
- f \, k \, \hat{v}
+ f \, l \, \hat{u}
+ \frac{k^2 + l^2}{m} \left( g \hat{\theta} + \ir \, \tilde{\omega} \, \hat{w} \right)
= 0
\]
\noindent Par ailleurs, $\hat{w}$ et~$\hat{\theta}$ sont liés par l'équation~\ref{polat} qui s'écrit également $g \, \hat{\theta} = - \ir \frac{N^2}{\tilde{\omega}} \, \hat{w}$. En multiplant par~$- \ir \tilde{\omega} / m$, on obtient
\[
\tilde{\omega}^2 \, \hat{w}
+ f \, \f{k}{m} \, \ir \, \tilde{\omega} \, \hat{v}
- f \, \f{l}{m} \, \ir \, \tilde{\omega} \, \hat{u}
- \frac{k^2 + l^2}{m^2} \, \hat{w} \, \left( N^2 - \tilde{\omega}^2 \right)
= 0
\]
\noindent Il ne reste plus qu'à transformer les termes inertiels restant, contenant $\ir \, \tilde{\omega} \, \hat{u}$ et~$\ir \, \tilde{\omega} \, \hat{v}$, en utilisant les équations~\ref{polau} et~\ref{polav}
\[
\tilde{\omega}^2 \, \hat{w}
+ f \, \f{k}{m} \, \left( + f \hat{u} + \ir \, l \, \hat{p} \right)
- f \, \f{l}{m} \, \left( - f \hat{v} + \ir \, k \, \hat{p} \right) 
- \frac{k^2 + l^2}{m^2} \, \hat{w} \, \left( N^2 - \tilde{\omega}^2 \right)
= 0
\]
\noindent En remarquant que les termes en~$\hat{p}$ se compensent, on parvient à
\[
\tilde{\omega}^2 \, \hat{w}
+ \frac{f^2}{m} \, \left( k \, \hat{u} + l \, \hat{v} \right)
- \frac{k^2 + l^2}{m^2} \, \hat{w} \, \left( N^2 - \tilde{\omega}^2 \right)
= 0
\]
\noindent puis l'équation~\ref{polaw} peut à nouveau être utilisée pour éliminer~$\hat{u}$ et~$\hat{v}$
pour finalement obtenir 
\[
\left[
\tilde{\omega}^2 - f^2
- \frac{k^2 + l^2}{m^2} \, \left( N^2 - \tilde{\omega}^2 \right)
\right] \hat{w}
= 0
\]
\noindent donc la \voc{relation de dispersion} des ondes de gravité
\[ \boxed{
\tilde{\omega}^2 - f^2 = \frac{k^2 + l^2}{m^2} \, \left( \tilde{\omega}^2 - N^2 \right)
} \]

\sk
La relation de dispersion indique que les ondes de gravité
\begin{finger}
\item se propagent horizontalement et verticalement, avec longueurs d'onde horizontale et verticale intimement liées;
\item ont une fréquence intrinsèque telle que~$N^2 < \tilde{\omega} < f^2$.
\end{finger}


\end{document}
