\documentclass[a4paper,DIV16,10pt]{scrartcl}
%%%%%%%%%%%%%%%%%%%%%%%%%%%%%%%%%%%%%%%%%%%%%%%%%%%%%%%%%%%%%%%%%%%%%%%%%%%%%%%%%%%
\usepackage{texcours}
%%%%%%%%%%%%%%%%%%%%%%%%%%%%%%%%%%%%%%%%%%%%%%%%%%%%%%%%%%%%%%%%%%%%%%%%%%%%%%%%%%%
\newcommand{\zauthor}{Aymeric SPIGA}
\newcommand{\zaffil}{Laboratoire de Météorologie Dynamique}
\newcommand{\zemail}{aymeric.spiga@sorbonne-universite.fr}
\newcommand{\zcourse}{Planétologie}
\newcommand{\zcode}{ENS}
\newcommand{\zuniversity}{ENS / Sorbonne Université (Faculté des Sciences)}
\newcommand{\zlevel}{M1 Géosciences}
\newcommand{\zsubtitle}{Fiches complémentaires de cours}
\newcommand{\zlogo}{\includegraphics[height=1.5cm]{/home/aspiga/images/logo/LOGO_SU_HORIZ_SIGNATURE_CMJN_JPEG.jpg}}
\newcommand{\zrights}{Copie et usage interdits sans autorisation explicite de l'auteur}
\newcommand{\zdate}{\today}
%%%%%%%%%%%%%%%%%%%%%%%%%%%%%%%%%%%%%%%%%%%%%%%%%%%%%%%%%%%%%%%%%%%%%%%%%%%%%%%%%%%
\begin{document} \inidoc
%%%%%%%%%%%%%%%%%%%%%%%%%%%%%%%%%%%%%%%%%%%%%%%%%%%%%%%%%%%%%%%%%%%%%%%%%%%%%%%%%%%

\newpage
\section{Premier principe appliqué à l'atmosphère}
\sk
Afin de travailler sur des grandeurs intensives, on divise la relation précédente par la masse~$m$ de la parcelle pour obtenir
\[ C_P \, \dd T = \frac{\dd P}{\rho} + \delta q \]
où $\delta q$ est la chaleur massique reçue et $C_P = C_P^* / M$ est la \voc{chaleur massique de l'air} ($C_P$=1004 J~K$^{-1}$~kg$^{-1}$). Nous disposons alors d'une autre version du premier principe, très utile en météorologie et valable pour une transformation quelconque d'une parcelle d'air
\[ \boxed{ \underbrace{\textcolor{white}{\frac{R^2}{C_P}} \dd T \textcolor{white}{\frac{R}{C_P}}}_{\text{variation de température de la parcelle}} = \underbrace{\frac{R}{C_P} \, \frac{T}{P} \, \dd P}_{\text{travail expansion/compression}} + \underbrace{\frac{1}{C_P} \, \delta q}_{\text{chauffage diabatique}} } \]

\sk
Autrement dit, la température de la parcelle augmente si elle subit une compression ($\dd P > 0$) et/ou si on lui apporte de la chaleur ($\delta q > 0$). La température de la parcelle à l'inverse diminue si elle subit une détente ($\dd P < 0$) et/ou si elle cède de la chaleur à l'extérieur ($\delta q < 0$). Il est donc important de retenir que la température de la parcelle peut très bien varier quand bien même la parcelle n'échange aucune chaleur avec l'extérieur~: dans ce cas, $\delta q = 0$ et l'on parle de \voc{transformation adiabatique}. 

\sk
L'équation fondamentale ci-dessus est directement dérivée du premier principe, mais prend une forme plus pratique en sciences de l'atmosphère du fait que les transformations que subit une parcelle atmosphérique se réduisent en général aux transformations \voc{isobares} (à pression constante $\dd P = 0$) et aux transformations \voc{adiabatiques} (sans échanges de chaleur avec l'extérieur $\delta q = 0$). Les transformations isothermes, au cours de laquelle la température de la parcelle ne varie pas, sont plus rarement rencontrées en sciences de l'atmosphère.




%\newpage
\section{Types de transformations : non adiabatiques}
\sk
Dans le cas où la transformation n'est pas adiabatique, les échanges de chaleur~$\delta q$ d'une parcelle d'air avec son environnement sont non nuls et peuvent s'effectuer par~:
\begin{itemize}
\item Transfert radiatif~: l'atmosphère se refroidit en émettant dans l'infrarouge, ou se réchauffe en absorbant du rayonnement électromagnétique dans l'infrarouge [cas des gaz à effet de serre] ou dans le visible [cas de l'ozone dans la stratosphère].
%Ces échanges sont faibles et peuvent être négligés sauf à l'échelle de la circulation générale\footnote{Le refroidissement/réchauffement peut être localement élevé au sommet/à la base de nuages.}
\item Condensation ou évaporation d'eau~: le changement d'état consomme ou relâche de la chaleur (ceci n'a lieu que lorsque l'air est à saturation).
\item Diffusion moléculaire (conduction thermique)~: ces transferts sont très négligeables sauf à quelques centimètres du sol.
\end{itemize}
Un cas notamment souvent cité en météorologie est celui d'une parcelle d'air située proche du sol, à la tombée de la nuit, qui subit peu de variations de pression ($\dd P \sim 0$) mais dont la température diminue sous l'effet du refroidissement radiatif ($\delta q < 0$). Ceci explique la présence de rosée sur le sol et de brouillard proche de la surface au petit matin.




\newpage
\section{Types de transformations : adiabatiques}
\sk
Dans de nombreuses situations en sciences de l'atmosphère, on peut considérer que l'évolution de la parcelle est \voc{adiabatique} et se fait sans échange de chaleur avec l'extérieur ($\delta q=0$). En vertu de l'équilibre hydrostatique qui relie pression~$P$ et altitude~$z$~:
\begin{citemize}
\item une parcelle dont l'altitude~$z$ augmente sans apport extérieur de chaleur, subit une \voc{ascendance} adiabatique, donc une détente telle que~$\dd P < 0$ et sa température diminue ;
\item inversement, une parcelle dont l'altitude~$z$ diminue sans apport extérieur de chaleur, subit une \voc{subsidence} adiabatique, donc une compression telle que~$\dd P > 0$ et sa température augmente. 
\end{citemize}

\sk
Dans le cas où la transformation est adiabatique, pression et température sont intimement liées en vetu du premier principe. La version du premier principe encadrée ci-dessus avec~$\delta q = 0$ indique
\[ \dd T = \frac{R}{C_P} \, \frac{T}{P} \, \dd P\]
\[ \Rightarrow \qquad \frac{\dd T}{T} - \frac{R}{C_P} \, \frac{\dd P}{P} = 0 \]
soit par intégration
\[ T \, P^{- \kappa} = \text{constante} \qquad \text{avec} \qquad \kappa = R / C_P \]
Autrement dit, dans le cas où une parcelle subit une transformation adiabatique, sa température varie proportionnellement à~$P^{\kappa}$. Il s'agit d'une version, avec les grandeurs intensives utiles en sciences de l'atmosphère, de l'équation~$P\,V^{\gamma}$, avec $\gamma = C_P / C_V$, vue dans les cours de thermodynamique générale pour les transformations adiabatiques.




\newpage
\section{Gradient adiabatique sec}
\sk
D'après les seules équations thermodynamiques, on peut trouver une loi simple des variations de température avec l'altitude pour une parcelle qui ne subit que des transformations adiabatiques. Considérons le cas d'une parcelle subissant un déplacement vertical quasi-statique et adiabatique tel que~$\delta q = 0$. Elle vérifie en première approximation l'équilibre hydrostatique~$\dd P\e{p} / \rho = - g \, \dd z$. L'équation du premier principe modifiée pour le cas atmosphérique indique alors que
\[  \dd T\e{p}  = - \frac{g}{C_P} \, \dd z \]
d'où on tire le profil vertical adopté dans l'atmosphère sèche par une parcelle ne subissant pas d'échange de chaleur avec l'extérieur
\[  \boxed{ \ddf{T\e{p}}{z}  = \Gamma\e{sec} \qquad \text{avec} \qquad \Gamma\e{sec} = \frac{-g}{C_P} } \]
On note qu'il ne s'agit pas nécessairement du profil vertical suivi par l'environnement (voir section~\ref{parcenv}).

\sk
Le résultat trouvé ci-dessus revêt une importance particulière en sciences de l'atmosphère. La température d'une parcelle en ascension adiabatique décroît avec l'altitude selon un taux de variation constant, indépendamment des effets de pression. La constante~$\Gamma\e{sec}$ est appelée le \voc{gradient adiabatique sec} de température. Il n'est valable que pour une parcelle d'air non saturée en vapeur d'eau. Le calcul pour la Terre donne un refroidissement de l'ordre de~$10^{\circ}$C/km (ou K/km). 

\sk
Pourquoi cette valeur est-elle en désaccord avec la décroissance de~$6.5^{\circ}$C/km effectivement constatée dans l'atmosphère terrestre~? Cet écart est relatif aux processus humides qui ont une grande importance dans l'atmosphère terrestre.
%Le chapitre suivant apporte des éléments de réponse à ce paradoxe apparent.




%%%%
\section{Remarque: chauffage adiabatique}
\sk
Dans un point de vue lagrangien, on peut aisément déterminer
qu'un mouvement vertical ascendant ($w>0$) induit un refroidissement adiabatique ($\dd T<0$)
et qu'un mouvement vertical descendant ($w<0$) induit un chauffage adiabatique ($\dd T>0$).
Il suffit de combiner l'équilibre hydrostatique,
ou plutôt sa variante, l'équation hypsométrique
\[ 
\f{\dd p}{p} = -\f{g \dd z}{R\,T} 
\qquad  
\Rightarrow
\qquad
\ddf{p}{t} = - \f{p}{R\,T} \, g \, w
\]
\noindent avec le premier principe dans le cas
adiabatique %\ddf{\theta}{t} = 0
\[
c_p \, \dd T = \f{\dd p}{\rho} \qquad\qquad \Rightarrow \qquad\qquad \boxed{\ddf{T}{t} = - \f{g}{c_p} \, w}
\]
%\noindent pour obtenir
%\[
%\ddf{T}{t} = - \f{g}{c_p} \, w
%\]




\newpage
\section{Transformations pseudo-adiabatiques}
\sk
On considère tout d'abord une parcelle d'air (contenant de la vapeur d'eau) en évolution isobare. Le premier principe appliqué à la parcelle indique donc
\[ \dd T = \frac{1}{C_P} \, \delta q \]
Lors de l'évaporation, les molécules d'eau liquide voient les liaisons hydrogène avec leurs proches voisins être brisées. Le passage de l'eau de la phase liquide à la phase vapeur consomme donc de l'énergie\footnote{On peut s'en convaincre en notant la sensation de froid immédiate que provoque la sortie d'un bain à cause de l'évaporation de l'eau liquide sur le corps mouillé~; ou en se souvenant que lorsque l'on souffle sur la soupe pour la refroidir, c'est précisément pour favoriser l'évaporation et la refroidir efficacement.}~: pour l'air qui compose la parcelle, $\delta q < 0$ et il y a refroidissement. 
A l'inverse, lors de la condensation, les molécules d'eau sous forme gazeuse créent des liaisons hydrogène avec les molécules d'eau de la phase liquide pour atteindre un état énergétique plus faible. Le passage de l'eau de la phase vapeur à la phase liquide libère donc de l'énergie~: pour l'air qui compose la parcelle, $\delta q > 0$ et il y a chauffage.

\sk
L'énergie~$\delta q$ consommée ou libérée par les changements d'état s'appelle~\voc{chaleur latente}, on la note~$\delta q\e{latent}$. Si une masse de vapeur~$\dd m\e{vapeur d'eau}$ est condensée ou évaporée, on a
\[ \delta q\e{latent} = \frac{- L \, \dd m\e{vapeur d'eau}}{m\e{air sec}} \qquad \Rightarrow \qquad \boxed{ \delta q\e{latent} = - L \, \dd r } \]
où~$L$ est la chaleur latente massique en~J~kg$^{-1}$. La formule ci-dessus comporte un signe négatif. La quantité~$\delta q\e{latent}$ est positive lorsqu'il y a condensation (le rapport de mélange en vapeur d'eau diminue $\dd r < 0$) et négative lorsqu'il y a évaporation (le rapport de mélange en vapeur d'eau augmente $\dd r > 0$).

\sk
On considère désormais une parcelle d'air en évolution adiabatique, à l'exception des échanges de chaleur latente~: $\delta q = \delta q\e{latent}$. On appelle une telle transformation \voc{pseudo-adiabatique} ou encore \voc{adiabatique saturée}. On fait l'approximation que la chaleur latente consommée ou dégagée est seulement échangée avec l'air sec~:
\begin{citemize}
\item La chaleur latente consommée/dégagée n'est pas utilisée pour refroidir/chauffer les gouttes d'eau présentes.
\item On néglige les pertes de masse par précipitation~: la masse d'air sec considérée est constante.
\end{citemize}
Pour une telle transformation, la variation de température s'écrit ainsi
\[ \dd T = \frac{R}{C_P} \, \frac{T}{P} \, \dd P - \frac{L}{C_P} \, \dd r \]


\newpage
\section{Profil vertical saturé}
\sk
Considérons une parcelle en ascension adiabatique saturée (et non plus sèche comme dans la section~\ref{adiabsec}). Pour une parcelle saturée, c'est-à-dire à l'équilibre liquide/vapeur, l'équation qui précède peut s'écrire, en utilisant l'équilibre hydrostatique
\[ c_p \, \dd T + g \, \dd z + L \, \dd r = 0 \]
Or, puisque la parcelle est saturée, on a~$r = r\e{sat}(T)$ et on peut écrire $\dd r\e{sat} = \ddf{r\e{sat}}{T} \, \dd T$. On a alors
\[ \left( c_p + L \, \ddf{r\e{sat}}{T} \right) \dd T + g \, \dd z = 0\]
Cette expression est similaire au cas sec, à l'exception notable du terme supplémentaire~$L \, \ddf{r\e{sat}}{T}$ lié aux échanges latents. On peut alors obtenir le profil vertical adopté dans l'atmosphère saturée par une parcelle ne subissant pas d'échange de chaleur avec l'extérieur autre que les échanges de chaleur latente
\[  \ddf{T}{z}  = \Gamma\e{saturé} \qquad \text{avec} \qquad \Gamma\e{saturé} = \frac{-g}{c_p+L \, \ddf{r\e{sat}}{T} } \]
On a vu que $\ddf{r\e{sat}}{T}$ est toujours positif, on en déduit donc
\[ \boxed{ \Gamma\e{saturé} > \Gamma\e{sec} \qquad \text{ou} \qquad |\Gamma\e{saturé}| < |\Gamma\e{sec}| } \]
A cause du dégagement de chaleur latente, la température diminue moins vite pour une parcelle saturée en ascension que pour une parcelle non saturée. Le calcul pour l'atmosphère terrestre montre que
\[ \Gamma\e{saturé} = -6.5 \, \text{K~km}^{-1} \] 
ce qui correspond à la valeur observée dans la troposphère sur Terre. %[Figure~\ref{fig:tempvert}].

\sk
La constatation que~$\Gamma\e{saturé}$ correspond au profil d'environnement effectivement mesuré dans la troposphère appelle un commentaire important. Les profils verticaux secs ou saturés sont ceux suivis par une parcelle en ascension~: autrement dit, ils donnent les variations de~$T\e{p}$ avec l'altitude~$z$. D'un point de vue instantané, ils ne correspondent pas aux profils d'environnement~$T\e{e}$ tels qu'ils peuvent être par exemple mesurés par des ballons-sonde lâchés dans l'atmosphère. La parcelle n'est pas nécessairement à l'équilibre thermique avec l'environnement. On peut néanmoins constater sur la figure~\ref{fig:tempvert} que la température de l'environnement diminue avec une pente très proche de~$\Gamma\e{saturé}$. Ceci s'explique par le fait que cette figure montre une moyenne sur tout le globe à toutes les saisons. La situation moyenne ainsi décrite correspond aux mouvements d'une multitude de parcelles en ascension qui finissent par définir l'environnement atmosphérique\footnote{Ce phénomène porte le nom d'ajustement convectif.}. Pour comprendre la formation des nuages, et plus généralement les mouvements atmosphériques, il faut néanmoins se placer dans le cas local où l'équilibre thermique n'est pas vérifié. C'est l'objet de la section suivante.
%Comme pour le cas adiabatique, on peut aussi intégrer l'équation pour obtenir:
%\begin{equation} e_h=c_pT+gz+Lr=cste \label{estath} \end{equation}  
%La quantité $e_h$ est appelée {\em énergie statique humide} et est conservée
%pour des mouvements adiabatiques ($r$ et $e_s$ sont séparément conservés) ou
%saturés (pseudo-adiabatiques).


\newpage
\section{(In)stabilité}
\sk
Ces considérations permettent de définir le concept de stabilité et instabilité verticale de l'atmosphère.
On considère l'atmosphère à un endroit donné de la planète, à une saison donnée, à une heure donnée de la journée.
On suppose que la température de l'environnement varie linéairement avec l'altitude
\[ \ddf{T\e{e}}{z} = \Gamma\e{env} \]
A une altitude~$z_0$ proche de la surface, la température de l'environnement est~$T\e{e}(z_0)=T_0$.

\sk
On considère une parcelle initialement à l'altitude~$z_0$ dont la température initiale~$T\e{p}(z_0)$ est également~$T_0$. On suppose que la parcelle subit une ascension verticale d'amplitude~$\delta z > 0$. Le profil de température suivi par la parcelle lors de son ascension est
\[ \ddf{T\e{p}}{z} = \Gamma\e{parcelle} \]
\begin{citemize}
\item Si la parcelle est non saturée, elle suit un profil adiabatique sec tel que $\Gamma\e{parcelle} = \Gamma\e{sec} \simeq - 10 \, \text{K/km}$.
\item Si elle est saturée, elle suit un profil adiabatique saturé tel que $\Gamma\e{parcelle} = \Gamma\e{saturé} \simeq - 6.5 \, \text{K/km}$. 
\end{citemize}
On rappelle qu'en général, à l'échelle où l'on étudie les mouvements de la parcelle
\[ \Gamma\e{parcelle} \neq \Gamma\e{env} \]

\sk
Quel est l'effet de la perturbation~$\delta z > 0$ sur le mouvement de la parcelle~? A l'altitude~$z_0 + \delta z$, les températures de la parcelle et de l'environnement sont respectivement
\[ T\e{p}(z_0 + \delta z) = T_0 + \Gamma\e{parcelle} \, \delta z 
\qquad \text{et} \qquad
T\e{e}(z_0 + \delta z) = T_0 + \Gamma\e{env} \, \delta z \]
\begin{finger}
\item Si $\Gamma\e{parcelle} > \Gamma\e{env}$, la température~$T\e{e}$ de l'environnement décroît plus vite que la température~$T\e{p}$ de la parcelle. Il en résulte que~$T\e{p}(z_0 + \delta z) > T\e{e}(z_0 + \delta z)$ et le mouvement de la parcelle est ascendant. La perturbation initiale est donc amplifiée par les forces de flottabilité. On parle de \voc{situation instable}. La situation est d'autant plus instable que la température de l'environnement décroît rapidement avec l'altitude. Lorsque la situation est instable, les mouvements verticaux sont amplifiés~: on parle parfois de \voc{situation convective}.
\item Si $\Gamma\e{parcelle} < \Gamma\e{env}$, la température~$T\e{e}$ de l'environnement décroît moins vite que la température~$T\e{p}$ de la parcelle. Il en résulte que~$T\e{p}(z_0 + \delta z) < T\e{e}(z_0 + \delta z)$ et le mouvement de la parcelle est descendant. La perturbation initiale n'est donc pas amplifiée et la parcelle revient à son état initial. On parle de \voc{situation stable}. La stabilité est d'autant plus grande que la température de l'environnement décroît lentement (ou augmente, dans le cas d'une inversion de température). Lorsque la situation est stable, les mouvements verticaux sont inhibés.
\end{finger}
La résultante des forces verticales s'exerçant sur la parcelle peut s'écrire en fonction des taux de variation~$\Gamma$ de la température
\[ F_z = g \, \frac{\Gamma\e{parcelle}-\Gamma\e{env}}{T\e{env}} \, \delta z \]
\noindent Un raisonnement similaire permet d'obtenir la fréquence de Brunt-V{\"a}is{\"a}l{\"a}.


\newpage
\section{Profil radiatif convectif (telluriques)}
\sk
Les conditions atmosphériques sont très instables proche d'une surface (en présence d'une telle surface). A cause de la discontinuité entre surface et atmosphère, sous l'action de la diffusion thermique, ou turbulente, entre la surface (chaude) et l'air immédiatement adjacent (plus froid) crée une couche d'air fine approximativement à la température de la surface ; les conditions de température étant plus froides au-dessus, les conditions atmosphériques sont très instables proche de la surface et des mouvements de convection vont se mettre en place pour mélanger l'air sur une certaine épaisseur atmosphérique. Un équilibre dit \voc{radiatif-convectif} prévaut, avec une structure thermique suivant le profil adiabatique, donnant naissance à une troposphère. Au-dessus de la limite radiative-convective (correspondant peu ou prou à la tropopause), les phénomènes radiatifs dominent et donnent naissance à une mésosphère -- ou une stratosphère si un absorbant visible y est présent en quantité suffisante, donnant naissance à une inversion stable à la tropopause.
%% on passait en troposphère dès que le gradient du profil radiatif dépassait celui du profil adiabatique (-g/cp)

\figun{0.4}{0.25}{decouverte/pierrehumbert_pics/9780521865562c03_fig014.jpg}{Figure tirée de R. Pierrehumbert, Principles of Planetary Climates, CUP, 2010}{fig:effetserre2}











%\newpage
%\section{Profil radiatif convectif (géantes)}
%\sk
La présence d'une surface et une hypothèse d'équilibre radiatif imposent donc que le chauffage de la surface conduit inévitablement à de la convection. Que se passe-t-il sur les planètes géantes dépourvues de surface ? L'équilibre radiatif y prévaut également, car à partir d'une certaine profondeur, le gradient radiatif est instable -- et ce, même en l'absence d'une surface qui absorbe le rayonnement solaire. Pour formuler la stabilité de l'équilibre radiatif, on calcule le profil~$\dd T / \dd p$ et on le compare au gradient adiabatique sec ou humide dans l'atmosphère considérée. En dérivant le profil radiatif obtenu dans le cas du modèle à deux faisceaux
\[ T(\tau) = \sqrt[4]{\frac{OLR\,(1+\tau)}{2\,\sigma\,\epsilon}} \]
\noindent par rapport à l'épaisseur optique~$\tau$, nous obtenons
\[ 8 \, \sigma \, T^3 \, \ddf{T}{\tau} = OLR \]
\noindent soit, en utilisant~$\ddf{~}{p} = \ddf{\tau}{p} \ddf{~}{\tau}$
\[ \ddf{T}{\ln p} = \frac{1}{4\,(1+\tau)} \, p \, \ddf{\tau}{p} \]
Ainsi la stabilité de la couche s'écrit
\[ \frac{R}{c_p} \ge \frac{1}{4\,(1+\tau)} \, p \, \ddf{\tau}{p} \]
\noindent et dans le cas d'un coefficient d'absorption~$\kappa$ constant, nous pouvons même écrire la condition de stabilité
\[ \frac{R}{c_p} \ge \frac{\tau}{4\,(1+\tau)} \]
%% si l'absorption est constante p \, \ddf{\tau}{p} = - \kappa \, p / g costheta

\sk
Le terme en~$p$ dans ce qui précède guarantit (à moins d'une variation énorme de $\ddf{\tau}{p}$ en~$1/p$ ou plus rapide quand~$p \rightarrow 0$) que les hautes atmosphères planétaires sont toujours stables. De plus, les atmosphères optiquement fines sont toujours stables sur l'intégralité de leur épaisseur, puisque~$-p \, \ddf{\tau}{p} < \tau_\infty \ll 1$. Le critère de stabilité est en pratique un peu plus complexe qu'indiqué dans les atmosphères réelles. Bien sûr, $\tau$ et~$\kappa$ varient avec la longueur d'onde~$\lambda$ (limitation inhérente au modèle à deux faisceaux), mais surtout le coefficient d'absorption~$\kappa$ augmente avec la pression (donc la profondeur) en raison de l'élargissement collisionnel (\emph{collisional broadening}), efficace à partir de quelques bars. La loi de variation d'élargissement collisionnel peut s'écrire~$\kappa(p) = \kappa(p\e{s}) \, \frac{p}{p\e{s}}$. Les processus de changements d'état sont également susceptibles de rendre la situation plus complexe que le calcul proposé ici.


%\newpage
%\section{Effet de serre : déplacement d'équilibre}
%


%\figside{0.45}{0.3}{/home/aymeric/Big_Data/BOOKS/pierrehumbert_pics/9780521865562c03_fig005.jpg}{R. Pierrehumbert, Principles of Planetary Climates, CUP, 2010}{fig:effetserre1}

\sk
Nous présentons ici l'explication la plus simple (sans être simpliste) du déplacement
d'équilibre radiatif qu'induit l'augmentation de gaz à effet de serre.
%Le modèle à deux faisceaux ne nous aide pas énormément. diverge quand tau tend vers infini.

\sk
Il est possible de montrer par des calculs de transfert radiatif que le niveau
d'émission équivalent au sommet de l'atmosphère est tel que~$\tau = 1$.
Qualitativement, on comprend que les niveaux inférieurs sont optiquement
épais donc ne sont que marginalement ``vus'' depuis l'espace dans les longueurs d'onde infrarouges.
Ainsi dans l'équilibre TOA
\[ \TOA \] 
\noindent l'émission de rayonnement au niveau~$\tau=1$ 
à la température~$T(\tau=1)$ domine OLR.

\sk
Appelons~$P\e{rad}$ la pression du niveau~$\tau=1$. 
Pour relier
les deux quantités, on emploie la définition de l'épaisseur optique
$\EO$ que l'on combine
à l'équilibre hydrostatique pour obtenir
par intégration~$\tau = \kappa \frac{P}{g} \, q_X$,
avec $q_X$ le rapport de mélange massique 
de l'espèce~$X$ absorbante dans l'infrarouge.
Ainsi
\[ P\e{rad} = \frac{g}{\kappa \, q_X} \]

\figun{0.7}{0.25}{/home/aymeric/Big_Data/BOOKS/pierrehumbert_pics/9780521865562c03_fig006.jpg}{R. Pierrehumbert, Principles of Planetary Climates, CUP, 2010}{fig:effetserre2}

\sk
L'expression ci-dessus implique qu'une augmentation de
gaz à effet de serre ($q_X$ augmente) implique une 
élévation du niveau équivalent d'émission
($p\e{rad}$ diminue).
L'effet sur la température de surface se détermine alors
en écrivant la conservation de la température potentielle
dans la troposphère soumise à l'équilibre radiatif-convectif,
entre la surface et le niveau équivalent d'émission
\[ T_s = T\e{rad} \, \left( \frac{P\e{rad}}{P\e{s}} \right)^{-\frac{R}{c_p}} \]
\noindent où~$P_s$ est la pression de surface.
Une élévation du niveau équivalent d'émission
se traduit donc par une augmentation
de température (fournissant
un modèle à la fois simple et fidèle du 
changement climatique récent sur Terre, Figure~\ref{fig:effetserre2}). 
Approximativement, $OLR \sim \sigma \, T(P\e{rad})^4$
et~$P\e{rad}$ est alors défini par la 
condition TOA qui s'écrit~$OLR = (1-A\e{b}) \, \mathcal{F}\e{s}'$.
Le lien entre quantité de gaz à effet de serre~$q_X$
et température de surface~$T\e{s}$ peut ainsi s'écrire
\[ T\e{s} = \sqrt[4]{\frac{(1-A\e{b}) \, \mathcal{F}\e{s}'}{\sigma}} \, \left( \frac{\kappa \, q_X \,P\e{s}}{g} \right)^{\frac{R}{c_p}} \]


%\newpage
%\section{Effet de serre divergent}
%
%% http://www.skepticalscience.com/print.php?r=262

\sk
\paragraph{Approche rapide} Une augmentation de la température de surface~$T\e{s}$ 
est donc associée à une augmentation de la quantité de gaz à effet de serre~$q_X$.
Sur une planète pourvue d'océan,
une augmentation de la température de surface
provoque une augmentation de l'évaporation
donc de la quantité de vapeur d'eau dans l'atmosphère, 
par conséquent de l'effet de serre.
Il s'agit d'une \voc{rétroaction positive}~:
le système amplifie la perturbation initiale de température de surface.
La quantité~$q_X$ de vapeur d'eau dans l'atmosphère
peut donc virtuellement augmenter indéfiniment.
Néanmoins, la pression du niveau équivalent~$P\e{rad} = \frac{g}{\kappa \, q_X}$
ne peut diminuer indéfiniment, du moins continûment~:
lorsque le sommet de la couche radiative de l'atmosphère
est atteint $P\e{rad} \ll P\e{s}$ , la radiation sortante~OLR
atteint une valeur maximale~$OLR\e{max}$.

\sk
\paragraph{Approche plus subtile} On peut inverser le point de vue et se demander quel est la valeur
d'OLR qui correspond à une température de surface~$T\e{s}$.
D'après le modèle simplifié combinant la hauteur équivalente
d'émission et le profil radiatif-convectif dans la troposphère, nous avons
\[ OLR  = \textcolor{magenta}{\sigma \, T\e{s}^4} \textcolor{blue}{\left( \frac{g}{\kappa \, q_X \, P\e{s}} \right)^{\frac{4 \, R}{c_p}}} \]
Plaçons-nous toujours dans le cas d'une planète pourvue
d'océan en évaporation.
Pour les températures de surface relativement modérées,
les variations de quantité de vapeur d'eau~$q_X$
(et de pression de surface~$P\e{s}$)
sont modérées et les variations d'OLR suivent 
une loi en~$\sigma T\e{s}^4$ (terme en \textcolor{magenta}{magenta}).
Néanmoins, plus la température de surface~$T\e{s}$
augmente, plus la vapeur d'eau devient dominante
dans l'atmosphère en influençant~$q_X$, mais
surtout~$P\e{s}$ via la loi d'équilibre liquide-vapeur
de Clausius-Clapeyron
$  P\e{s}(T\e{s}) = P_0 \, \exp{ \left[ -\frac{\ell}{R\,T\e{s}} \right] } $
\noindent où~$\ell > 0$ est la chaleur latente de vaporisation.
Si $T\e{s}$ augmente, $P\e{s}(T\e{s})$ augmente, et de manière exponentielle.
Le terme en \textcolor{blue}{bleu}, qui décroît exponentiellement avec
la température~$T\e{s}$, influence de façon dominante
l'expression de l'OLR pour les température élevées.
L'effet combiné des deux termes 
(\textcolor{magenta}{magenta} et \textcolor{blue}{bleu})
impose donc que l'OLR atteint une valeur maximale~$OLR\e{max}$,
que l'on appelle limite de Komabayashi-Ingersoll
(du nom de deux auteurs d'articles indépendants parus à la fin des années 60).

%% EM: En revanche, l'asymptote n'est qu'approximativement horizontale, elle est légèrement décroissante en présence d'un gaz à effet de serre non condensable (typiquement CO2). Du coup, il vaut mieux distinguer le maximum (KI limit) et l'asymptote plus basse (limite de Nakajima, voir Fig. 3 de https://journals.ametsoc.org/doi/pdf/10.1175/1520-0469%281992%29049%3C2256%3AASOTGE%3E2.0.CO%3B2) 

\figside{0.4}{0.15}{/home/aymeric/Big_Data/BOOKS/pierrehumbert_pics/9780521865562c04_fig004.jpg}{R. Pierrehumbert, Principles of Planetary Climates, CUP, 2010}{fig:ki}

\sk
\paragraph{Effet de serre divergent} 
Quelle que soit l'approche adoptée pour définir~$OLR\e{max}$,
il existe cette limite lorsque toute l'atmosphère devient optiquement épaisse.
Si l'on se place dans un contexte de variation
(à l'échelle des temps géologiques) du flux incident
solaire~$(1-A\e{b}) \, \mathcal{F}\e{s}'$,
avec notamment une augmentation au cours du temps
étant donné l'activité radioactive du Soleil\footnote{En fait, le flux incident varie lorsque la pression atmosphérique devient conséquente à cause d'un effet de diffusion Rayleigh accru},
on constate que les valeurs du flux 
incident~$(1-A\e{b}) \, \mathcal{F}\e{s}'$ peuvent
dépasser~$OLR\e{max}$, ce qui signifie que
l'équilibre TOA ne peut être satisfait et que
l'atmosphère reçoit plus d'énergie qu'elle
n'en émet. La température de surface peut augmenter
de manière incontrôlée, au risque d'atteindre des valeurs
très élevées (plusieurs centaines de K, voire quelques milliers).
L'évaporation des océans peut alors
survenir de manière abrupte et rapide (Figure~\ref{fig:ki}),
dans ce que l'on appelle l'\voc{effet de serre divergent}
(\emph{runaway greenhouse}). De fait, 
l'équilibre radiatif type TOA peut n'être
récupéré que pour des températures de surface 
très élevées (valeurs de plus de~$1000-2000$~K,
pour lesquelles l'intégralité des océans a disparu
selon toute vraisemblance).
Au-delà de telles valeurs de température de surface~$T\e{s}$,
le flux sortant OLR se remet à augmenter avec~$T\e{s}$
en raison de la contribution grandissante de l'émission
thermique dans le visible (et de la moindre absorption
de la vapeur d'eau dans ces longueurs d'onde).


%%% KI : dépend de g


%% Fs + when Ts + because increased absorption of solar rad by water vapor
%% then - when Ts + because Rayleigh scattering



\newpage
\section{\'Echelle de hauteur}
\sk
En exprimant la densité~$\rho$ en fonction de l'équation des gaz parfaits, l'équilibre hydrostatique s'écrit
\[ \Dp{P}{z} = - g \, \frac{P}{RT} \]
On peut intégrer cette équation si on suppose que l'on connaît les variations de~$T$ en fonction de $P$ ou $z$. On suppose ici que l'on peut négliger les variations de pression selon l'horizontale devant les variations suivant la verticale, donc transformer les dérivées partielles~$\partial$ en dérivées simples~$\dd$. On effectue ensuite une séparation des variables
\[R \, T \, \frac{\dd P}{P} = - g \, \dd z\]

\sk
Cette équation peut s'écrire sous une forme dimensionnelle simple à retenir
\[ \boxed{ \frac{\dd P}{P} = - \frac{\dd z}{H(z)} \qquad \text{avec} \qquad H(z) = \frac{R \, T(z)}{g} } \]
La grandeur~$H$ se dénomme l'\voc{échelle de hauteur} et dépend des variations de la température~$T$ avec l'altitude~$z$. L'équation ci-dessus indique bien que la pression décroît avec l'altitude selon une loi exponentielle comme observé en pratique. Cette loi peut être plus ou moins complexe selon la fonction~$T(z)$. On peut néanmoins fournir une illustration simple du résultat de l'intégration dans le cas d'une atmosphère isotherme~$T(z)=T_0$
\[ P(z) = P(z=0) \, e^{-z/H} \qquad \text{avec} \qquad H = R \, T_0 / g \]



%\newpage
\section{\'Equation hypsométrique}
\sk
Dans l'équation de l'échelle de hauteur, faire l'hypothèse isotherme est très simpliste et rarement rencontré en pratique dans l'atmosphère. On se place dans le cas plus général, bien que toujours simplifié, de deux niveaux atmosphériques~$a$ et~$b$ entre lesquels la température ne varie pas trop brusquement avec l'altitude~$z$. On réalise alors l'intégration entre les deux niveaux~$a$ et~$b$
\[R \, T \, \frac{\dd P}{P} = - g \, \dd z \qquad \Rightarrow \qquad R \, \int_a^b T\, \frac{\dd P}{P} = - g \, \int_a^b dz\]
puis on définit la température moyenne de la couche atmosphérique entre~$a$ et~$b$ avec une moyenne pondérée
\[ \langle T \rangle = \frac{\int_a^b T \, \frac{\dd P}{P}}{\int_a^b \frac{\dd P}{P}} \]
pour obtenir finalement
\[R \, \langle T \rangle \, \int_a^b \frac{\dd P}{P} = - g \, \int_a^b dz
\qquad \Rightarrow \qquad \boxed{ g \, (z_a - z_b) = R \, \langle T \rangle \ln \left( \frac{P_b}{P_a} \right) } \]
Cette relation est appelée \voc{équation hypsométrique}. Elle correspond à une formulation utile en météorologie du principe que \ofg{l'air chaud se dilate}. Les conséquences de l'équation hypsométrique peuvent s'exprimer de diverses façons équivalentes.
\begin{citemize}
\item Pour une masse d'air donnée, une couche d'air chaud est plus épaisse.
\item La distance entre deux isobares est plus grande si l'air est chaud.
\item La pression diminue plus vite selon l'altitude dans une couche d'air froid.
\end{citemize}
En passant le résultat précédent au logarithme, on note que l'on retrouve toujours le fait que la pression diminue avec l'altitude selon une loi exponentielle. En notant l'échelle de hauteur moyenne~$\langle H \rangle$, on a
\[ P_b = P_a \, e^{ - \frac{z_b - z_a}{\langle H \rangle}} \qquad \text{avec} \qquad \langle H \rangle = \frac{R \, \langle T \rangle}{g} \]





\newpage
\section{Circulations thermiques directes}
\sk
Toute différence de température entre deux régions (provoquée par exemple par un chauffage différentiel, ou par une différence des propriétés thermophysiques de la surface) est associée à des différences de pression, car d'après l'équation hypsométrique (équilibre hydrostatique + équation d'état du gaz parfait) la pression diminue plus vite avec l'altitude dans les couches d'air froid que dans les couches d'air chaud. Ceci donne naissance en altitude à un gradient de pression donc, en supposant que la force de pression est seule responsable de l'accélération du vent (vision à raffiner par la suite), des vents vont naître en altitude de la région chaude vers la régions froide. Ces vents induisent un flux de masse atmosphérique de la région chaude vers la région froide, donc causent, d'après l'équivalence entre pression et masse déduite de l'équilibre hydrostatique, une augmentation de la pression de surface dans la région froide par rapport à la région chaude. Ceci donne naissance proche de la surface à des vents de la région chaude vers la région froide. Par continuité, en considérant les convergences et divergences d'air proche du sol et en altitude, l'air s'élève dans les régions chaudes et redescend dans les régions froides.

\sk
Des exemples de circulations thermiques directes sont
\begin{finger}
\item les \voc{cellules de Hadley}, cellules fermées dans le plan méridien, sud-nord et verticale; sous les tropiques, l'air s'élève proche de l'équateur (suivant la saison, du côté de l'hémisphère d'été) et redescend au niveau des subtropiques.
\item les \voc{\og brises \fg~de mer et de terre} au bord de la mer sur Terre, naissant du contraste thermique entre continent et océan
\item les circulations atmosphériques sur Mars entre les régions polaires couvertes de glace et les régions de sol nu
\end{finger}

\sk
Les cellules fermées associées aux circulations thermiques directes \underline{ne sont pas des cellules de convection}. Elles résultent simplement de la déformation du champ de pression par des contrastes de température. Des cellules fermées non convectives peuvent également se développer dans le sens inverse de celui thermique direct (par exemple, les cellules de Ferrel sur Terre) : les mécanismes sont distincts des processus de circulation thermique directe et sont en général relatifs au forçage de l'écoulement moyen par les ondes atmosphériques résultant d'instabilités dans l'atmosphère.






\newpage
\section{Accélération de Coriolis et déviation du mouvement}
\sk
L'accélération de Coriolis peut être interprétée comme une force apparente massique $\v F_C = - 2 \, \v \Omega\wedge\v V_r$. Cette force apparente étant orthogonale à la vitesse à cause de la présence du produit vectoriel, sa puissance est nulle~: la \voc{force de Coriolis} va dévier le mouvement relatif mais ne peut pas modifier la vitesse du vent ou de courants. Pour des mouvements relatifs horizontaux à la vitesse \v V, le module de la force apparente de Coriolis est~$2 \, \Omega \, \sin \phi \, V$ qui change de signe lorsqu'on change d'hémisphère en fonction de~$\sin \phi$. Dans l'hémisphère nord, où $\sin \phi>0$, la force de Coriolis est dirigée à $90^{\circ}$ à droite du vent. 

\sk
Afin de bien comprendre l'effet de la force de Coriolis, il est profitable sur une planète comme la Terre d'utiliser la conservation du moment cinétique\footnote{
Puisque le moment cinétique~$\sigma$ se conserve on a \[ \ddf{\sigma}{t} = 0 = \ddf{r}{t} \, (\Omega \, r + u) + r \, \left( \Omega \ddf{r}{t}+\ddf{u}{t} \right) \qquad \Rightarrow \qquad \ddf{u}{t} = - \ddf{r}{t}  \, \left( 2\,\Omega + \frac{u}{r} \right) \] 
Le terme en $u/r$ est dû à la courbure de la surface, mais seule la vitesse relative intervient, pas la rotation de la Terre. En pratique, ce terme est négligeable sur Terre devant~$2 \, \Omega$. L'équation ci-dessus montre donc que raisonner avec la conservation du moment cinétique permet de comprendre l'effet sur les vents de la force de Coriolis.
}
(l'équivalent pour les systèmes en rotation de la quantité de mouvement pour les systèmes en translation). En effet, la somme des forces étant dirigée vers H, M conserve son \voc{moment cinétique}~$\sigma$ par rapport à l'axe des pôles, qui s'exprime
\[ \boxed{ \sigma = u_a \, r = (\Omega \, r + u) \, r } \]
où~$r$ est la distance entre le point considéré et l'axe de rotation qui passe par les deux pôles.
%\footnote{La conservation de $\sigma$ implique des variations de l'énergie cinétique $(\Omega r+u)^2$. C'est le travail de \v G (pour un mouvement sud-nord) qui en est l'origine.}. 

\sk
Pour illustrer les effets de cette force apparente de Coriolis, on considère une parcelle initialement au repos dans le référentiel tournant (c'est à dire~$u=0$ et~$v=0$ à~$t=0$) qui se déplacerait vers le Nord suivant l'axe~$\v j$. Elle se rapproche donc de l'axe des pôles et va voir sa vitesse absolue augmenter par conservation du moment cinétique: $\sigma$ est constant et~$r$ diminue, donc $u_a$ augmente. Dans le même temps, la vitesse d'entrainement locale~$u_e=\Omega \, r$ diminue sous l'effet de la diminution de la distance~$r$ à l'axe des pôles. La parcelle va donc acquérir une vitesse relative $u>0$ vers l'est\footnote{
En fait, l'expression ci-dessus permet même de calculer la variation de vitesse associée. Pour un mouvement sud-nord, la vitesse est $v=a \, \ddf{\phi}{t}$. D'autre part $r=a \, \cos \phi$ donc~$\ddf{r}{t}=-a \, \ddf{\phi}{t} \, \sin \phi = - v \, \sin \phi$. L'équation de conservation du moment cinétique devient 
\[ \ddf{u}{t} = v \, \sin \phi \, \left( 2 \, \Omega + \frac{u}{r} \right) \simeq 2 \, \Omega \, v \, \sin \phi \] 
La parcelle est bien déviée vers l'est pour un déplacement vers le nord tel que~$v>0$.
}
comme indiqué sur le schéma \ref{fig:coriolisns}. 

\figside{0.3}{0.2}{\figfrancis/coriolis_ns}{Déviation d'une parcelle se déplaçant vers le nord. Instant initial: vitesses d'entrainement $u_e$ et absolue $u_a$ égales. Instant final: vitesse d'entrainement $u_e'$ et absolue $u_a'$ augmentée par conservation du moment cinétique $\sigma$.}{fig:coriolisns}

%\subsubsection{Force de Coriolis: mouvement vers l'est} On considère un point M en mouvement par rapport à la surface de la Terre. On rappelle que pour un mouvement circulaire, on doit avoir une accélération normale égale à $V^2/R$ dirigée vers le centre du cercle. On suppose que les forces réelles s'exerçant sur M sont les mêmes que pour un point fixe: $\Sigma \vec F=\v a_e$. La composante de la vitesse relative vers l'est (suivant \v i) est $u$, et $\dot{r}$ dans la direction \vl{HM}. La vitesse absolue de M vers l'est est $u_a=\Omega r+u$. La relation $\v a=\Sigma\v F$ s'écrit dans la direction $\v e_r$: \[-\frac{(\Omega r+u)^2}{r}+\ddot{r}=a_e=-\Omega^2r\] soit en développant: \[\ddot{r}=u\cdot(2\Omega+\frac{u}{r})\] Pour un mouvement relatif vers l'est ($u>0$), la vitesse absolue est supérieure à la vitesse d'entrainement, et la somme des forces est insuffisante pour compenser $V_a^2/r$. La parcelle va donc s'éloigner de l'axe de rotation (figure \ref{fig:coriolisew}). Elle va au contraire se rapprocher pour $u<0$ (mouvement vers l'ouest). Pour trouver l'accélération relative dans la direction sud-nord, on projette $\v e_r$ sur \v j: $\dot{v}=-\ddot{r}\sin \phi$. \[\dot{v}=-u\sin \phi\cdot(2\Omega+\frac{u}{r})\] M est donc dévié vers le sud pour un déplacement relatif vers l'est.
%\begin{figure}[tbp] \begin{center} \includegraphics[width=12cm]{\figfrancis/coriolis_ew} \end{center} \caption{Déviation d'une parcelle ayant une vitesse relative initiale non nulle vers l'est (gauche) et l'ouest (droite). Un plan parallèle à l'équateur est représenté, vu depuis le pôle nord, l'axe de rotation est au centre. Les vitesse et accélération d'entrainement (égale à la somme des forces) sont en noir, la vitesse absolue en rouge. La trajectoire future de la parcelle est en pointillés.} \label{fig:coriolisew} \end{figure}


\newpage
\section{Cellules de Hadley et courants-jets}
\sk
La structure du vent zonal est dominée aux moyennes latitudes par la présence de deux \voc{jets}, c'est-à-dire de puissants courants atmosphériques, dits \voc{jets d'ouest} car ils soufflent de l'ouest vers l'est. Leur vitesse augmente sur la verticale entre la surface et un maximum au niveau de la tropopause, autour de 50~m~s$^{-1}$. Ce comportement peut être justifié en combinant l'équilibre géostrophique à l'équilibre hydrostatique (équation du vent thermique). Dans les tropiques, les vents moyens sont d'est, surtout dominants dans la basse troposphère, mais restent néanmoins moins forts que les vents d'ouest dans les moyennes latitudes. On les appelle les \voc{alizés}.

\sk
La circulation dans le plan méridien (sud-nord et verticale) est caractérisée par une série de cellules fermées. Le chauffage différentiel explique qu'une différence de pression naisse entre les tropiques et les moyennes latitudes, car la pression diminue plus vite avec l'altitude dans les couches d'air froid des moyennes latitudes que dans les couches d'air chaud des tropiques. Ceci donne naissance en altitude à des vents de l'équateur vers les pôles. Ces vents induisent un flux de masse atmosphérique vers les moyennes latitudes, donc, d'après l'équivalence entre pression et masse, une augmentation de la pression de surface aux moyennes latitudes par rapport aux tropiques. Ceci donne naissance proche de la surface à des vents des pôles vers l'équateur. Par continuité, dans les tropiques, l'air s'élève proche de l'équateur (suivant la saison, du côté de l'hémisphère d'été) et redescend au niveau des subtropiques. Ce système est appelé \voc{cellules de Hadley}. On observe également dans les moyennes latitudes des cellules moins intenses, contrôlées par les instabilités dans l'atmosphère, appelées cellules de Ferrel. 
%%(figure \ref{fig:MMC})

\sk
La structure en latitude des vents %décrite par la figure~\ref{fig:UTlatP}, 
avec des vents d'ouest aux moyennes latitudes et d'est sous les tropiques, est très liée à la circulation de Hadley.
% décrite par la figure~\ref{fig:MMC}.
Sous l'action de la force de Coriolis, les mouvements vers les pôles sont déviés vers l'est et les mouvements vers l'équateur sont déviés vers l'ouest. Les jets d'ouest des moyennes latitudes proviennent ainsi de la déviation vers l'est de la circulation vers les pôles dans la branche supérieure de la cellule de Hadley. Les vents d'est (alizés) sous les tropiques proviennent quant à eux de la déviation vers l'ouest de la circulation vers l'équateur dans la branche inférieure de la cellule de Hadley. Les vents de grande échelle comportent donc une composante vers l'équateur et l'ouest sous les tropiques, alors qu'aux moyennes latitudes, ils comportent une composante vers les pôles et l'est [la composante vers l'est domine cependant]. Une exception à cette image est observée dans les régions de ``mousson'' (sous-continent Indien, et dans une moindre mesure Afrique de l'ouest et Amérique centrale) où la direction du vent s'inverse entre l'été (vers le continent) et l'hiver (vers l'océan).

\figside{0.6}{0.3}{\figfrancis/WH_circ_scheme}{Schéma de la circulation atmosphérique: zone de convergence et alizés dans les tropiques; gradient de pression tropiques (H) -pôle (L), vents d'ouest et ondes aux moyennes latitudes. La position des jets d'ouest et l'extension des cellules de Hadley sont représentées à droite. Figure adaptée de Wallace and Hobbs, Atmospheric Science, 2006.}{fig:circscheme}


%\begin{detail} 
%%%%%%%%%%%%%%%%%%%%%%%%%%%%%%%%%%%%%%%%%%%%%%%%%%%%%%%%%%%%%%%
%\newpage
%\section{Système et référentiel}
%	\sk
La position d'un point $M$ de l'atmosphère sera représentée dans un systèmes de coordonnées sphériques (figure~\ref{fig:repere}) par sa latitude $\varphi$, sa longitude $\lambda$, et son altitude~$z$ par rapport au niveau de la mer. Pour les déplacements horizontaux, on utilise le repère direct
$\left(M,\mathbf{i},\mathbf{j},\mathbf{k}\right)$ où $\mathbf{i}$ et $\mathbf{j}$ sont les vecteurs unitaires vers l'est et le nord, et $\mathbf{k}$ est dirigé suivant la verticale vers le haut. La direction définie par~$\mathbf{i}$ est souvent qualifiée de \voc{zonale}, celle définie par~$\mathbf{j}$ de \voc{méridienne}. 
%Pour des déplacements qui ne sont pas d'échelle planétaire, on utilisera également des distances horizontales vers l'est et le nord~$dx=a\, d\lambda\, \cos \varphi$ et~$dy=a\,d\varphi$ où~$a$ est le rayon de la Terre.
%
%\sk
On distingue deux référentiels pour l'étude des mouvements de l'air:
\begin{finger}
\item Un \voc{référentiel tournant} lié à la Terre, en rotation autour de l'axe des pôles avec la vitesse angulaire $\Omega$. La \voc{vitesse relative} est mesurée dans le référentiel tournant, par rapport à la surface de la Terre et a pour composantes~$u,v,w$ suivant \v i,\v j,\v k. Il s'agit de ce que l'on appelle communément le \voc{vent} avec le point de vue d'humain attaché à la surface de la Terre, c'est-à-dire au référentiel tournant. La composante horizontale du vecteur vitesse relative est donc~$\mathbf{V} = u \, \mathbf{i} + v \, \mathbf{j}$ et la composante verticale~$w \, \mathbf{k}$.
\item Un \voc{référentiel fixe} orienté suivant les directions de trois étoiles. La \voc{vitesse absolue} d'un point M est considérée dans le référentiel fixe et inclut donc le mouvement circulaire autour de l'axe des pôles. Ce référentiel peut être considéré comme galiléen. Il correspond à ce qu'on observerait depuis l'espace, lorsqu'on voit la Terre tourner au lieu d'être \ofg{attaché} à sa rotation.
\end{finger}

%\figun{0.4}{0.25}{\figfrancis/repere}{Schéma du système de coordonnées et du repère utilisés.}{fig:repere}
\figside{0.45}{0.22}{\figfrancis/repere}{Système de coordonnées et repère utilisés.}{fig:repere}



%\section{Changement de référentiel}
%	\sk
L'équation de base pour le mouvement de masses d'air est la relation fondamentale de la dynamique $\Sigma \v F=m \, \v a$ (seconde loi de Newton).  Cette relation est cependant valable dans un référentiel galiléen, tel le référentiel fixe. On s'intéresse plutôt au vent, c'est-à-dire que l'on souhaite considérer des mouvements atmosphériques par rapport à la surface de la Terre qui est en rotation autour de l'axe des pôles. On va donc dans un premier temps projeter l'accélération dans le référentiel tournant, puis étudier les principales forces horizontales. Autrement dit, on se donne pour objectif d'exprimer l'accélération dans le référentiel tournant, qu'on souhaite connaître, en fonction de l'accélération dans le référentiel fixe, qui est égale à la somme des forces.

\sk
La relation entre vitesse absolue~$\v V_a$ dans le référentiel fixe et vitesse relative~$\v V_r$ dans le référentiel tournant s'écrit, avec le vecteur de rotation~$\v \Omega$ de module~$\Omega$ dirigé selon l'axe des pôles~:
\[\v V_a = \v V_r + \vl{\Omega}\wedge\vl{CM}\]
Il s'agit de la relation de composition des vitesses pour un référentiel tournant. Le terme $\vl{\Omega}\wedge\vl{CM}$ est la vitesse d'un point fixe par rapport au sol ($\v V_r=0$), il est appelé \voc{vitesse d'entrainement}.
%La relation entre la dérivée temporelle d'un vecteur \v X dans le référentiel fixe (\emph{absolue}, $a$) et celle dans le référentiel tournant (\emph{relative}, $r$) s'écrit \[\frac{d\v X}{dt}_{|a}=\frac{d\v X}{dt}_{|r}+ \vl \Omega\wedge \v X\] En applicant au vecteur \vl{CM}, avec $\frac{d\vl{CM}}{dt}=\v V$, on a: \[\v V_a=\v V_r+\vl{\Omega}\wedge\vl{CM}\]
La relation entre accélération absolue~$\v a_a$, égale à la somme des forces, et accélération relative~$\v V_r$ dans le référentiel tournant s'écrit
\[ \v a_a=\Sigma\v F=\v a_r+2\vl{\Omega}\wedge\v V_r-\Omega^2\,\vl{HM} \]
Le premier terme est l'accélération relative~$\v a_r$, le deuxième l'\voc{accélération de Coriolis}~$\v a_c$, le troisième est l'\voc{accélération d'entrainement}~$\v a_e$. Les termes de Coriolis et d'entraînement induisent des \voc{forces apparentes}~$\v F_c = -m \, \v a_c$ et~$\v F_e = -m \, \v a_e$ dans le référentiel tournant. On parle de forces apparentes car du point de vue du référentiel fixe, ces termes n'apparaissent pas comme des forces~: ils ne sont que des termes d'accélération causés par le caractère non galiléen du référentiel tournant.
%En dérivant à nouveau $\v V_a$, on obtient: \[\v a_a=\left(\frac{d \v V_r}{dt}_{|r}+\vl{\Omega}\wedge\v V_r\right)+\vl{\Omega}\wedge \left(\v V_r+\vl{\Omega}\wedge\vl{CM}\right)\] soit en regroupant et avec $\vl{\Omega}\wedge(\vl{\Omega}\wedge\vl{CM})=-\Omega^2\cdot\vl{HM}$:



%\section{Eulérien vs. Lagrangien}
%	\sk
Comment caractériser un écoulement ?

\sk
\paragraph{Point de vue lagrangien} Le plus intuitif~:~Suivre les particules le long de leur trajectoire.
\paragraph{Point de vue eulérien} Le plus pratique~:~Suivre le courant depuis un point géométrique. Les points sont fixés ce qui est plus aisé en première approche pour modéliser l'écoulement sur une grille.
\centers{Variations lagrangiennes \quad = \quad Variations eulériennes \quad + \quad Terme d'advection}
Le terme d'advection transport concentre le caractère non-linéaire de la dynamique atmosphérique

\sk
On passe de l'un à l'autre des formalismes avec la formule de la dérivée d'une fonction composée~$\mathcal{F}[x(t)]$ où~$x$ est la position.
\[
\underbrace{\derd{\mathcal{F}}{t}}_{\text{En suivant la particule}}
= 
\underbrace{\Dp{\mathcal{F}}{t}}_{\text{En un point géométrique}} 
+ 
\underbrace{\left(\v U \cdot \v \nabla \right)\,\mathcal{F} }_{\text{Lié au déplacement de la particule}}
\]



%\section{Accélération d'entraînement et pesanteur}
%	\sk
On considère un point M immobile par rapport à la surface de la Terre. Les forces (massiques) subies par M sont la force de gravitation \v G, dirigée vers le centre de la Terre, et \v R la réaction du sol dirigée perpendiculairement à la surface (figure \ref{fig:centrif}). Dans le référentiel fixe, l'accélération de M est celle du mouvement circulaire uniforme: $\v a_e = - \Omega^2 \, \vl{HM}$ (accélération d'entrainement). On doit donc avoir \[\v a_e=\v G+\v R\] 
C'est impossible si la Terre est sphérique (sauf au pôle et à l'équateur): on aurait alors \v R et \v G colinéaires mais pas dans la direction de $\v a_e$. La Terre a en fait pris une forme aplatie, où la surface n'est pas perpendiculaire à~$\v G$. En posant $\v g=\v G-\v a_e$, l'équilibre devient: \[\v g+\v R=\v 0\] On a donc une gravité apparente \v g dirigée localement vers le bas (perpendiculairement à la surface) mais pas exactement vers le centre de la Terre. La gravité réelle \v G a elle une faible composante horizontale. Dans ce qui suit, on considère que l'accélération d'entraînement est inclus dans le terme~$\v g$.

%\figside{0.55}{0.25}{\figfrancis/centrif}{Equilibre d'un point posé au sol. La forme réelle de la Terre est en trait continu, la sphère en pointillés.}{fig:centrif}
\figside{0.45}{0.2}{\figfrancis/centrif}{Equilibre d'un point posé au sol. La forme réelle de la Terre est en trait continu, la sphère en pointillés.}{fig:centrif}



%\section{Forces de pression}
%	\sk
Les forces de pression horizontales se calculent comme la force de pression verticale dans la démonstration de l'équilibre hydrostatique. La force de pression s'exerçant sur une surface $S$ est normale à cette surface et vaut $P \, S$. Pour une parcelle d'air de volume $\delta x \, \delta y \, \delta z$ (figure \ref{fig:pres}), la force de pression totale dans la direction ($Ox$) vaut
\[ F_P^* = P(x) \, \delta y \, \delta z - P(x+\delta x) \, \delta y \, \delta z = - \frac{\partial P}{\partial x} \, \delta x \, \delta y \, \delta z \]
La force de pression {\em massique} est donc
\[F_P = \frac{F_P^*}{\rho \delta x \delta y \delta z}=-\frac{1}{\rho}\frac{\partial P}{\partial x}\]
On peut faire le même calcul sur ($Oy$). Finalement les deux composantes horizontales de la force de pression s'écrivent
\[\v F_P^H = -\frac{1}{\rho} \, \binom{\frac{\partial P}{\partial x}}{\frac{\partial P}{\partial y}}\] %  =-\frac{1}{\rho}\vl{grad}P\]

\sk
La force de pression est donc opposée aux variations horizontales de pression données par les dérivées partielles, ce qui lui confère des propriétés importantes.
\begin{citemize}
\item La force de pression est dirigée des hautes vers les basses pressions, perpendiculairement aux isobares.
\item La force de pression est inversement proportionelle à l'écartement des isobares.
\end{citemize}
Une région où la pression est particulièrement basse est appelée \voc{dépression}. Une région où la pression est particulièrement élevée est appelée \voc{anticyclone}.

\figside{0.6}{0.2}{\figfrancis/pressure}{Forces de pression (suivant ($Ox$)) s'exerçant sur une parcelle.}{fig:pres}

%\subsubsection{Équivalence avec le géopotentiel}
%L'équilibre hydrostatique fait que la pression décroit toujours avec
%l'altitude. Une pression localement élevée doit donc correspondre à une
%altitude élevée des surfaces isobares.
%\begin{figure}[htp]
%  \begin{center}
%    \includegraphics[width=\figwn]{\figfrancis/pres_geop}
%  \end{center}
%  \caption{Équivalence entre écarts de pression et d'altitude: les points A et
%  B sont à la même altitude, A et C à la même pression. La pression en B est
%  donc supérieure à celle en B.}
%  \label{fig:pres_geop}
%\end{figure}
%Sur la figure \ref{fig:pres_geop}, la force de pression horizonale dans la
%direction ($Ox$) est
%$F_P=-\frac{1}{\rho}\frac{P_B-P_A}{\delta x}$. Or $A$ et $C$ sont à la même
%pression, on a donc
%\[F_P=-\frac{1}{\rho}\frac{P_B-P_C}{\delta x}=-\frac{1}{\rho}\frac{P_B-P_C}{\delta z}\cdot\frac{\delta z}{\delta x}\]
%En utilisant
%\[\frac{P_B-P_C}{\delta z}=-\frac{\partial P}{\partial z}=\rho g\]
%on trouve 
%\[F_P=-g\left(\frac{\delta z}{\delta x}\right)_P\]
%On aurait une relation équivalente pour la direction ($Oy$), la
%force de pression horizontale vaut donc finalement
%\[\v F_P=-\frac{1}{\rho}\vl{grad}_Z(P)=-g\cdot\vl{grad}_P(Z)\]
%On utilise plutôt le gradient de pression horizontal avec la pression au
%niveau de la mer, et le gradient isobare de l'altitude $Z$ ou du
%{\em géopotentiel} $gZ$ dans l'atmosphère libre.
%Sur une carte d'une surface isobare, les lignes à $Z$ constant sont des
%{\em isohypses}. La force de pression est donc dirigée des hautes vers les
%basses valeurs de $Z$, perpendiculairement aux isohypses.

\sk
Les variations verticales de la pression sont données par l'équilibre hydrostatique comme indiqué dans les chapitres précédents. Cette propriété a deux conséquences importantes pour les variations de pression horizontales donc la force de pression horizontale. 
\begin{finger}
\item Une conséquence de cet équilibre est que la pression à une altitude $z$ est proportionelle à la masse de la colonne d'air située au dessus de $z$. Une diminution ou augmentation de cette masse dûe aux mouvements d'air horizontaux change donc la pression en dessous, en particulier à la surface.
\item D'autre part, même pour une masse d'air totale de la colonne constante, des écarts de température horizontaux peuvent créer des gradients de pression en changeant la répartition verticale de cette masse. L'équation hypsométrique donne l'épaisseur d'une colonne d'air de masse constante entre deux niveaux de pression donnés (voir chapitres précédents)~: la pression décroît plus vite dans une couche d'air froid que dans une couche d'air chaud. Une variation horizontale de température induit donc une force de pression horizontale selon ce principe.
\end{finger}

%\begin{equation}
%  g\cdot(Z_2-Z_1)=R<T>\ln{\frac{P_1}{P_2}}
%  \label{eq:hypso}
%\end{equation}
%La différence entre les forces de pressions aux niveaux 1 et 2 sera donc: \[\v F_{P_2}-\v F_{P_1}=-R\cdot\vl{grad}<T>\cdot\ln{\frac{P_1}{P_2}}\]



%%%%%%%%%%%%%%%%%%%%%%%%%%%%%%%%%%%%%%%%%%%%%%%%%%%%%%%%%%%%%%%
%\end{detail}




	
%\begin{detail}
%%%%%%%%%%%%%%%%%%%%%%%%%%%%%%%%%%%%%%%%%%%%%%%%%%%%%%%%%%%%%%%
%\newpage
%\section{Equation fondamentale}
%	\sk
L'équation complète de la quantité de mouvement pour les mouvements atmosphériques, qui résulte de l'application de la seconde loi de Newton, s'écrit~:
%\begin{equation}
%\ddf{\v V_r}{t} + 2 \, \v \Omega\wedge\v V_r = \v g + \v F_P + \vl{Fr} 
\[   
\ddf{\v V_r}{t} = \v g + \v F_P + \v F_C + \vl{Fr}
\] %\frac{1}{\rho}\vl{grad}P  \]
%  \label{eq:qtemvt}
%\end{equation}
Le terme~$\vl{Fr}$ représente les forces de friction qui sont négligées sauf lorsqu'on se trouve à proximité de la surface.

%\section{Passage aux coordonnées sphériques}
%	\sk
Vitesse dans le référentiel tournant 
\[
\vec{U_r} = u\,\vec{i} + v\,\vec{j} + w\,\vec{k}
\]
\noindent Accélération (dérivée lagrangienne)
\[
\gamma_r = \derd{\vec{U_r}}{t}=\derd{u}{t}\,\vec{i}+\derd{v}{t}\,\vec{j}+\derd{w}{t}\,\vec{k} +u\,\derd{\vec{i}}{t}+v\,\derd{\vec{j}}{t}+w\,\derd{\vec{k}}{t}
\]

\figsup{0.4}{0.2}{decouverte/cours_dyn/didt1.png}{decouverte/cours_dyn/didt2.png}{Axe méridional, axe zonal}{fig:spher}

\sk
Décomposition sur les trois axes (zonal, méridien et vertical)
\[
\derd{\vec{i}}{t} = \left[ \derd{\vec{i}}{t} \right]_u + \left[ \derd{\vec{i}}{t} \right]_v + \left[ \derd{\vec{i}}{t} \right]_w
\]
\noindent Axe vertical
\[
\left[ \derd{\vec{i}}{t} \right]_w = \left[ \derd{\vec{j}}{t} \right]_w = \left[ \derd{\vec{k}}{t} \right]_w = \vec{0}
\]
\noindent Axe méridien
\[
\left[ \derd{\vec{i}}{t} \right]_v = \vec{0}
\]
\[
\left[ \derd{\vec{j}}{t} \right]_v = - \derd{\phi}{t} \, \vec{k} = - \dfrac{v}{a} \, \vec{k}
\]
\[
\left[ \derd{\vec{k}}{t} \right]_v = + \dfrac{v}{a} \, \vec{j}
\]
\noindent Axe zonal
\[
\vec{j} = -\sin\phi \, \vec{m} + \cos\phi \, \vec{n} \qquad \vec{k} =  \cos\phi \, \vec{m} + \sin\phi \, \vec{n} \qquad \textrm{avec} \qquad \vec{m} = -\sin\phi \, \vec{j} + \cos\phi \, \vec{k}
\]
\[
\left[ \derd{\vec{n}}{t} \right]_u = \vec{0} \qquad \left[ \derd{\vec{m}}{t} \right]_u = \derd{\lambda}{t} \, \vec{i}
\]
\[
\left[ \derd{\vec{j}}{t} \right]_u = -\sin\phi \, \left[ \derd{\lambda}{t} \, \vec{i} \right] = \dfrac{-u}{a} \, \tan\phi \, \vec{i}
\]
\[
\left[ \derd{\vec{k}}{t} \right]_u =  \cos\phi \, \left[ \derd{\lambda}{t} \, \vec{i} \right] = \dfrac{u}{a} \, \vec{i}
\]
\[
\left[ \derd{\vec{i}}{t} \right]_u = - \derd{\lambda}{t} \, \vec{m} = \dfrac{u}{a} \tan\phi \, \vec{j} - \dfrac{u}{a} \,\vec{k}
\]


%\section{Expression des forces en coordonnées sphériques}
%	\paragraph{L'équation fondamentale de la dynamique} dans le référentiel tournant
\[
\vec{\gamma}_r = -2 \, \vec{\Omega} \wedge \vec{U}_r + \vec{g} - \dfrac{\vec{\nabla}p}{\rho} + \vec{Fr}
\]

\paragraph{Composantes de l'acc\'el\'eration}
\[
\vec{\gamma}_r=\left[\begin{array}{ccc} u_t & - \dfrac{u\,v\tan\phi}{a} & + \dfrac{u\,w}{a}\\ & & \\ v_t & + \dfrac{u^2\tan\phi}{a} & + \dfrac{v\,w}{a}\\ & & \\ w_t & - \dfrac{u^2+v^2}{a} & \\ \end{array}\right] \qquad \text{notation} \quad u_t = \derd{u}{t}
\]

\paragraph{Composantes de la \ofg{force} de Coriolis}
\[
-2\,\vec{\Omega}\wedge\vec{U}_r=-2 \left[ \begin{array}{c} 0\\ \Omega\,\cos\phi\\ \Omega\,\sin\phi \end{array} \right] \wedge \left[ \begin{array}{c} u\\ v\\ w \end{array} \right] = \left[ \begin{array}{c} 2 \, \Omega \, ( v\sin\phi-w\cos\phi )\\ -2 \, \Omega \, u\sin\phi\\ 2 \, \Omega \, u\cos\phi \end{array} \right]
\]

%%%%%%%%%%%%%%%%%%%%%%%%%%%%%%%%%%%%%%%%%%%%%%%%%%%%%%%%%%%%%%%
%\end{detail}

\newpage
\section{Equations complètes du mouvement}
\sk
L'équation fondamentale de la dynamique des fluides géophysiques 
en projection sur les coordonnées sphériques avec l'approximation de couche mince
s'écrit finalement

\begin{center}
\begin{tabular}{ccccccccc}
%%%%%%
\textcolor{blue}{$\ddf{u}{t}$} & 
\textcolor{brown}{$-\dfrac{uv\tan\phi}{a}$} & 
$+\dfrac{uw}{a}$ & 
= & 
\textcolor{red}{$2\Omega\sin\phi v$} & 
$-2\Omega \cos\phi w$ & 
\textcolor{green!75!black}{$-\dfrac{1}{\rho}\Dp{p}{x}$} & 
& 
$+Fr_x$\\
~\\
%%%%%%
\textcolor{blue}{$\ddf{v}{t}$} & 
\textcolor{brown}{$+\dfrac{u^2\tan\phi}{a}$} & 
$+\dfrac{vw}{a}$ & 
= & 
\textcolor{red}{$-2\Omega\sin\phi u$} &
&
\textcolor{green!75!black}{$-\dfrac{1}{\rho}\Dp{p}{y}$} &
&
$+Fr_y$\\
~\\
%%%%%%
$\ddf{w}{t}$ & 
$-\dfrac{u^2+v^2}{a}$ & 
&
=&
$2\Omega\cos\phi u$ & 
& 
\textcolor{green!75!black}{$-\dfrac{1}{\rho}\Dp{p}{z}$} & 
\textcolor{green!75!black}{$ -g$} & 
$+Fr_z$\\ 
\end{tabular}
\end{center}

\sk
Les termes s'interprètent comme suit
\begin{citemize}
\item{Pression} \textcolor{green!75!black}{$\bullet$} 
\item{Coriolis} \textcolor{red}{$\bullet$}
\item{Inertiels (sphéricité)} \textcolor{brown}{$\bullet$}
\item{Inertials (accélération)} \textcolor{blue}{$\bullet$}
\end{citemize}
Les termes en noir sont liés au déplacements verticaux et sont en général négligeables quand on considère la circulation générale de l'atmosphère et de l'océan.


%\section{Système complet pour la modélisation}
%\sk
Bjerknes, 1904~:~6 équations pour 6 inconnues 
\begin{finger}
\item variables \textcolor{red}{dynamiques} ou \textcolor{brown}{thermodynamiques}
\item ce qui dépend de la planète considérée~:~\textcolor{blue}{forçages} et \textcolor{green!75!black}{constantes planétaires} 
\item rappel: formalisme eulérien vs. lagrangien $\derd{\mathcal{F}}{t}=\der{\mathcal{F}}{t}+\textcolor{red}{u}\der{\mathcal{F}}{x}+\textcolor{red}{v}\der{\mathcal{F}}{y}+\textcolor{red}{w}\der{\mathcal{F}}{z}$
\end{finger}

\sk
\noindent Les 6 équations qui permettent d'évaluer l'évolution déterministe du fluide atmosphérique soumis aux forçages
\begin{enumerate}
\item Mouvement horizontal ouest-est
\[ \derd{\textcolor{red}{u}}{t} - \dfrac{\textcolor{red}{u}\textcolor{red}{v}\tan\phi}{\textcolor{green!75!black}{a}} = 2\textcolor{green!75!black}{\Omega}\sin\phi \, \textcolor{red}{v} - \dfrac{1}{\textcolor{brown}{\rho}} \, \der{\textcolor{brown}{p}}{x} + \textcolor{blue}{F_u} \]
\item Mouvement horizontal sud-nord
\[ \derd{\textcolor{red}{v}}{t} + \dfrac{\textcolor{red}{u}^2\tan\phi}{\textcolor{green!75!black}{a}} = -2\textcolor{green!75!black}{\Omega}\sin\phi \, \textcolor{red}{u} - \dfrac{1}{\textcolor{brown}{\rho}} \, \der{\textcolor{brown}{p}}{y} + \textcolor{blue}{F_v} \]
\item Equilibre hydrostatique vertical
\[
 - \dfrac{1}{\textcolor{brown}{\rho}} \, \der{\textcolor{brown}{p}}{z} - \textcolor{green!75!black}{g} = 0
\]
\item Conservation de la masse
\[
\der{\textcolor{brown}{\rho} }{t} + \div\dep{\textcolor{brown}{\rho} \textcolor{red}{\vec{V}}}=0
\]
\item Premier principe 
\[
\f{\textcolor{green!75!black}{c_p}}{\theta} \, \derd{\theta}{t} = \f{\textcolor{blue}{\mathcal{Q}}}{\textcolor{brown}{T}}
\qquad \text{avec} \qquad
\theta=\textcolor{brown}{T} \, \left[ \f{\textcolor{green!75!black}{p_0}}{\textcolor{brown}{p}} \right]^{\textcolor{green!75!black}{\kappa}} 
\]
\item Gaz parfait
\[
\textcolor{brown}{p} = \textcolor{brown}{\rho} \, \textcolor{green!75!black}{R} \, \textcolor{brown}{T}
\]
\end{enumerate}



%\newpage 
%\section{\'Echelles}
%\sk
Tous les termes de l'équation du mouvement n'ont pas la même importance lorsqu'on considère des mouvements atmosphériques de grande échelle. On définit donc des échelles caractéristiques du mouvement étudié. Pour simplifier, on choisit des échelles qui sont des puissances de 10.
\begin{description}
\item[longueur] Les échelles de longueur sont $L$ sur l'horizontale, et $H$ sur la verticale. Pour des mouvements qui s'étendent sur la hauteur de la troposphère, $H\sim 10$~km. $L$ peut varier beaucoup, mais l'échelle dite synoptique $L=$1000~km, qui est celle des perturbations des latitudes moyennes, est d'un intérêt particulier. La dernière échelle de longueur est celle du rayon de la Terre~$a$, qui est de l'ordre de 10000~km. 
\item[vitesse] Les échelles de vitesse horizontale et verticale sont notées $U$ et $W$. On a typiquement $U$=10~m~s$^{-1}$ dans l'atmosphère. Le rapport d'aspect du mouvement impose d'autre part que $W\le UH/L$.
\item[temps] L'échelle de durée du mouvement est construite à partir de celles de vitesse et de longueur: $T=L/U$. L'autre échelle de temps est celle liée à la rotation de la Terre, qui apparait dans le terme de Coriolis.
\item[variables thermodynamiques] Les variations des variables thermodynamiques $P,T,\rho$ sur la verticale sont celles des profils moyens donnés en introduction. En un point donné, les variations à l'échelle synoptique $\delta P,\delta T,\delta\rho$ sont de l'ordre de 1\% de la valeur moyenne.
\end{description}




%\newpage 
%\section{\'Echelles (évaluation)}

%	\sk \subsection{\'Echelles (mouvement vertical)}
%	\sk
L'ordre de grandeur des termes de l'équation du mouvement 
%\ref{eq:qtemvt} 
projetée sur la verticale (dirigée suivant \v k) est indiqué dans la table \ref{tab:vqmouv}. On voit que l'équilibre hydrostatique est vérifié avec une très bonne approximation\footnote{On peut noter qu'on vérifie également l'équilibre hydrostatique entre des anomalies de densité et des anomalies de variations de pression sur la verticale. Les termes $\rho g$ et $\partial P/\partial z$ sont alors cent fois plus faibles que pour l'état moyen, mais toujours supérieurs aux autres termes de l'équation.}. Notamment la composante verticale de la force de Coriolis~$\v F_C$ est négligeable devant~\v g et les forces de pression. Le seul autre terme qui peut devenir important est l'accélération relative~$dw/dt$, lors de mouvements verticaux intenses à petite échelle, comme dans un nuage d'orage ou près de topographie raide.  
%\begin{equation}
%  \frac{\partial P}{\partial z}=-\rho  g
%  \label{eq:hydro}
%\end{equation}

\begin{table}
  \centering
  \begin{tabular}{ccccccc}
    \hline
    Équation & $dw/dt$ & $-2\Omega u\cos\phi$ & $-\left(u^2+v^2\right)/a$ & = &
    $-\rho^{-1}\partial P/\partial z$ & $-g$ \\
    Échelle & $UW/L$ & $fU$ & $U^2/a$ && $P_0/(\rho_0H)$ & $g$ \\
    m.s\md & 10$^{-7}$ & 10$^{-3}$ & 10$^{-5}$  && 10 & 10 \\ 
    \hline
  \end{tabular}
  \caption{\emph{Analyse d'échelle de l'équation du mouvement vertical (avec
  $L$=1000~km et $W$=1~cm.s\mo).}}
  \label{tab:vqmouv}
\end{table}




%	\sk \subsection{\'Echelles (mouvement horizontal)}
%	\sk
Le détail de l'équation horizontale projetée en coordonnées sphériques est donné dans la table \ref{tab:hqmouv} pour $L$=1000~km. Sur les composantes horizontales (\v i, \v j), l'expression de la force de Coriolis se réduit aux contributions des mouvements horizontaux dans la mesure où~$W<<U$ pour des mouvements d'échelle supérieure à 10~km. 
\[\v F_C = \binom{f \, v}{-f \, u} \qquad \textrm{ou} \qquad \v F_C = -f \, \v k \wedge \v V_h \]
où $\v V_h = u \v i + v \v j$ est la vitesse horizontale et 
\[ \boxed{ f = 2 \, \Omega \, \sin \phi } \]
est appelé \voc{facteur de Coriolis}. Aux moyennes latitudes ($\phi=45$\deg), la valeur de~$f$ est environ~$10^{-4}$~s$^{-1}$. 
%Les composantes de la force de Coriolis sont \[\v F_C=-2\Omega\left(\begin{array}{c}0\\\cos\phi\\\sin\phi\end{array}\right) \wedge\left(\begin{array}{c}u\\v\\w\end{array}\right) =-2\Omega\left(\begin{array}{c}w\cos\phi-v\sin \phi\\u\sin \phi\\-u\cos \phi\end{array}\right)\]
%\footnote{Pour des mouvements de
%type ``chute libre'', la vitesse verticale $w$ domine. On peut alors mettre en
%évidence une déviation vers l'est, mais qui reste très faible (de l'ordre de
%1cm pour 80m de chute).} 

\begin{table}
  \centering
  \begin{tabular}{cccccccc}
    \hline
    Équation-$x$ & $\frac{du}{dt}$ & $-2\Omega v\sin\phi$ & $+2\Omega
    w\cos\phi$ & $+\frac{uw}{a}$ & $-\frac{uv\tan\phi}{a}$ &=&
    $-\frac{1}{\rho}\frac{\partial P}{\partial x}$ \\
    Équation-$y$ & $\frac{dv}{dt}$ & $+2\Omega u\sin\phi$ &&         
                $+\frac{vw}{a}$ & $+\frac{u^2\tan\phi}{a}$ &=&
    $-\frac{1}{\rho}\frac{\partial P}{\partial y}$ \\
    Échelles & $U^2/L$ & $fU$ & $fW$ & $UW/a$ & $U^2/a$ && $\delta P/(\rho L)$
    \\
    m.s\md & 10$^{-4}$ & 10$^{-3}$ & 10$^{-6}$ & 10$^{-8}$ & 10$^{-5}$ &&
    10$^{-3}$ \\
    \hline
  \end{tabular}
  \caption{\emph{Analyse en ordre de grandeur de l'équation du mouvement
  horizontale.}}
  \label{tab:hqmouv}
\end{table}

\sk
Sur un plan horizontal, les termes restants de l'équation du mouvement sont ainsi:
%\begin{equation}
\[  \frac{d\v V_h}{dt}+f\v k\wedge\v V_h=\v F_P  \]
%  \label{eq:hqmouv}
%\end{equation}
avec $\v V_h$ la vitesse horizontale, et $\v F_P$ les forces de pression horizontales massiques. Pour évaluer lequel des deux termes à gauche domine, on définit le \voc{nombre de Rossby} $\mathcal{R}$, rapport entre accélération relative et de Coriolis
\[ \mathcal{R} = \frac{U^2/L}{f\,U} = \frac{U}{f\,L} \]
Avec $f$=10$^{-4}$~s$^{-1}$ aux moyennes latitudes et $U$=10~m~s$^{-1}$, on a $\mathcal{R}=0.1$ aux grandes échelles de la circulation terrestre ($L$=1000~km), donc Coriolis domine. Au contraire, à une échelle plus petite de $L$=10~km, $\mathcal{R}=10$ et Coriolis devient négligeable.





%\newpage
\section{Grands équilibres}
\sk
Suivant les termes dominants, on peut définir un certain nombre d'équilibres (stationnaires) ou de modèles / équations (pouvant servir à la prédiction de l'écoulement au cours du temps):
\begin{description}
\item{Equilibre hydrostatique} \textcolor{green!75!black}{$\bullet$} 
\item{Equilibre g\'eostrophique} \textcolor{green!75!black}{$\bullet$}\textcolor{red}{$\bullet$}
\item{Equilibre cyclostrophique} \textcolor{green!75!black}{$\bullet$}\textcolor{brown}{$\bullet$}
\item{Equilibre du vent gradient} \textcolor{green!75!black}{$\bullet$}\textcolor{red}{$\bullet$}\textcolor{brown}{$\bullet$}
\item{Modèle quasi-g\'eostrophique} \textcolor{green!75!black}{$\bullet$}\textcolor{red}{$\bullet$}\textcolor{blue}{$\bullet$}
\item{Equations primitives} \textcolor{green!75!black}{$\bullet$}\textcolor{red}{$\bullet$}\textcolor{blue}{$\bullet$}\textcolor{brown}{$\bullet$}
\end{description}

\mk
\paragraph{Nombre de Rossby} Le nombre de Rossby permet d'évaluer l'importance relative de l'accélération de Coriolis, impulsée par la rotation de la planète, par rapport aux autres mouvements de rotation. Il permet de savoir si l'on se trouve dans le domaine de validité de l'équilibre géostrophique ou de l'équilibre cyclostrophique
\[
R_o=\f{\text{accélération horizontale (inertielle + sphéricité)}}{\text{accélération de Coriolis}}\qquad\boxed{R_o=\frac{U}{L\,\Omega}}
\]
\begin{table}[h!]
\begin{tabular}{cccc}
$R_o \ll 1$ & \textcolor{green!75!black}{$\bullet$}\textcolor{red}{$\bullet$} & Equilibre g\'eostrophique & [Terre, Mars]\\
$R_o \gg 1$ & \textcolor{green!75!black}{$\bullet$}\textcolor{brown}{$\bullet$} & Equilibre cyclostrophique & [Vénus, Titan]\\
$R_o$~tous & \textcolor{green!75!black}{$\bullet$}\textcolor{red}{$\bullet$}\textcolor{brown}{$\bullet$} & Equilibre du vent gradient & [Toutes]\\
~ & & & \\
$R_o \ll 1$ & \textcolor{green!75!black}{$\bullet$}\textcolor{red}{$\bullet$}\textcolor{blue}{$\bullet$} & Modèle quasi-g\'eostrophique & [Terre, Mars]\\
$R_o$~tous & \textcolor{green!75!black}{$\bullet$}\textcolor{red}{$\bullet$}\textcolor{blue}{$\bullet$}\textcolor{brown}{$\bullet$} & Equations primitives  & [Toutes]
\end{tabular}
\end{table}

\mk
Sur les planètes à rotation rapide, l'équilibre géostrophique est le développement des équations du mouvement à l'ordre 1 en le nombre de Rossby, qui décrit un écoulement bidimensionnel, stationnaire et non divergent. A un ordre supérieur en $\textrm{Ro}$, l'évolution lente de la fonction de courant géostrophique peut être diagnostiquée par un nouvel équilibre dit quasi-géostrophique (QG). Couplé à l'équation de conservation de la vorticité potentielle de Rossby, le modèle approché QG a permis à Charney dans les années 50 de faire fonctionner sur un ordinateur le premier modèle de prévision numérique du temps.


\newpage
\section{Vent gradient}
\sk
Dans toutes les atmosphères connues, à grande échelle, un quasi-équilibre s'établit entre le champ de masse atmosphérique (lié au gradient de pression) et la composante horizontale de la force centrifuge (entraînement + Coriolis), liée au vent zonal~$u$ et 
\[
\v F\e{e} = - m \, \left( 2\,\Omega\,\sin\varphi\,u + \f{u^2\,\tan\varphi}{a} \right) \v j
\]
%% $F\e{e}$ s'oppose au gradient de pression pour un vent prograde~$u > 0$
\noindent Il s'agit de l'\voc{équilibre du vent gradient}, vrai en moyenne zonale (Figure~\ref{fig:vg}). 
\[
\boxed{\dfrac{u^2\tan\phi}{a} + 2\Omega\sin\phi u = -\dfrac{1}{\rho}\Dp{p}{y}}
\]
%\sk Le vent gradient est un équilibre diagnostic alors que les équilibres géostrophiques et cyclostrophiques sont des équations prognostiques

\figsup{0.35}{0.15}{decouverte/cours_dyn/td2_pression.png}{decouverte/cours_dyn/td2_centrifuge.png}{Gradient de pression (haut). Force centrifuge (bas).}{fig:vg}

\sk
L'advection de l'air liée à la force de pression devient de moins en moins efficace quand on s'éloigne de l'équateur : l'équilibre du vent gradient provient d'un effet de frein exercé par la composante horizontale de la force centrifuge. Plus la planète tourne vite, plus cet effet de frein est prépondérant. Ainsi, la rotation de la planète contrôle l'extension en latitude des cellules de Hadley -- tout comme le déplacement en latitude du maximum saisonnier de température la contrôle~: 
\begin{finger}
\item L'extension limitée des cellules de Hadley sur Terre est majoritairement causée par la position du maximum saisonnier de température (qui reste confinée aux tropiques en raison de l'inertie thermique élevée des océans), alors que leur extension limitée sur les planètes géantes est le fait de leur rotation rapide.
\item L'extension des cellules de Hadley vers les pôles sur Mars est majoritairement causée par les effets de position du maximum saisonnier de température\footnote{Aux solstices, sous l'effet de la faible inertie thermique de la surface martienne et des constantes de temps radiatifs réduits dans l'atmosphère, la structure thermique de l'atmosphère martienne est composée d'un gradient de température d'un pôle à l'autre et conduit à une circulation de Hadley interhémisphérique. La circulation méridienne est particulièrement intense en raison du forçage diabatique des poussières en suspension dans l'atmosphère, surtout au solstice d'hiver nord où l'opacité moyenne des poussières atteint $1$ et l'insolation est maximale.}, alors que la grande extension vers les pôles des cellules de Hadley sur Vénus est avant tout reliée à la rotation lente de ce corps qui limite l'effet de frein de~$F\e{e}$.
\end{finger}

\sk 
Dans certains cas particuliers, notamment si~$u<0$ et~$u<2 \, \Omega \, a \, \cos \varphi$ (autrement dit, pour un courant-jet prograde dans le cas où la force de Coriolis domine la force d'entraînement), la force~$\v F\e{e}$ induit une accélération vers le pôle et non un frein. Cet effet peut favoriser l'extension vers les pôles des cellules de Hadley dans le cas de planètes à rotation rapide comme Mars. Exemple, au solstice d'hiver nord de Mars, une particule de vitesse zonale nulle partant d'un point de l'hémisphère sud de latitude $-\varphi_0$ (typiquement $60^{\circ}S$) et parcourant la branche haute de la cellule de Hadley adopte un mouvement rétrograde $u<0$ jusque la latitude opposée $\varphi_0$ par conservation du moment cinétique $\mathcal{M} = a \cos \varphi \left( \Omega \, a \, \cos \varphi + u \right)$. Contrairement au cas terrestre, la résultante des forces d'entraînement~$\v F\e{e}$ s'ajoute entre les latitudes $0$ et $\varphi_0$ au gradient de pression et la circulation méridienne s'intensifie jusqu'à la latitude $\varphi_0$, rejetant la limite des cellules de Hadley beaucoup plus loin que sur Terre. Entre les latitudes $-\varphi_0$ et $\varphi_0$, les isolignes du transport méridien de masse se confondent donc avec les isolignes du moment cinétique. Aux plus hautes latitudes $\varphi > \varphi_0$, dès que la vitesse zonale devient négative, le jet d'ouest se forme et la résultante~$\v F\e{e}$ s'oppose aux gradients de pression comme sur Terre.
%Seules les cellules de Hadley autour des équinoxes martiens, symétriques entre les deux hémisphères, ressemblent aux équivalents terrestres.









\newpage
\section{Géopotentiel et coordonnées isobares}
\sk
\paragraph{Coordonnées isobares} On note~$x$ la coordonnée sur l'axe est-ouest (axe zonal), $y$ la coordonnée sur l'axe sud-nord (axe méridional), $P$ la pression atmosphérique. On commence tout d'abord par considérer que la pression atmosphérique~$P$ remplace l'altitude comme coordonnée verticale~: on raisonne donc sur des surfaces isobares. La pression~$P$ peut être utilisée comme coordonnée verticale car monotone en vertu de l'équilibre hydrostatique. Le vent géostrophique zonal~$u$ s'exprime comme une fonction de~$x$, $y$, $P$, tout comme la température atmosphérique~$T$ et la masse volumiquede l'air~$\rho$. Les dérivées partielles (notées~$\partial$) de ces fonctions de trois variables se comprennent comme les dérivées selon la coordonnée indiquée avec les deux autres fixées. Par exemple $\frac{\partial T}{\partial y}$ est la dérivée de~$T$ uniquement selon la coordonnée~$y$, en considérant que~$x$ et~$P$ ne varient pas ; $\frac{\partial u}{\partial P}$ est la dérivée de~$u$ uniquement selon la coordonnée~$P$, avec les deux autres coordonnées fixées.

\sk
\paragraph{Géopotentiel} On définit le géopotentiel~$\Phi$ comme une fonction des coordonnées~$x$, $y$ et~$P$ qui s'écrit simplement
\[ \Phi(x,y,P)=g \, z(x,y,P) \] 
\noindent avec~$z$ l'altitude (également fonction des coordonnées~$x$, $y$ et~$P$) et~$g$ l'accélération de la gravité (supposée ici ne pas varier avec~$z$). 
%On rappelle que les dérivées partielles commutent, c'est-à-dire par exemple 
%\[ \frac{\partial}{\partial P} \frac{\partial \Phi}{\partial y} = \frac{\partial}{\partial y} \frac{\partial \Phi}{\partial P} \]

\sk
\paragraph{Dérivée verticale du géopotentiel} On utilise tout d'abord l'équilibre hydrostatique pour exprimer très simplement la dérivée du géopotentiel~$\Phi$ en fonction de la coordonnée verticale~$P$
\[ \frac{\partial \Phi}{\partial P} = g \, \frac{\partial z}{\partial P} = -\f{1}{\rho} \] 
\noindent ce qui permet de relier simplement les variations verticales de géopotentiel (sur les lignes isobares) au champ de masse. 

\sk
\paragraph{Dérivée horizontale du géopotentiel} On utilise une propriété de changement de coordonnée dans les dérivées partielles
\[ 
\left[ \frac{\partial P}{\partial y} \right]_z
\simeq
\left[ \frac{(P_0 + \delta p) - P_0}{\delta y} \right]_z 
=
\left[ \frac{(P_0 + \delta p) - P_0}{\delta z} \right]_y
\left[ \frac{\delta z}{\delta y} \right]_P
\simeq
-\left[ \frac{\partial P}{\partial z} \right]_y \left[ \frac{\partial z}{\partial y} \right]_P
\]
(le signe moins apparaît car~$P$ décroît avec l'altitude~$z$) pour exprimer très simplement la force de pression comme la dérivée spatiale du géopotentiel (en utilisant à nouveau au passage l'équilibre hydrostatique)
\[ -\frac{1}{\rho} \left[ \frac{\partial P}{\partial y} \right]_z
= \frac{1}{\rho} \left[ \frac{\partial P}{\partial z} \right]_y \left[ \frac{\partial z}{\partial y} \right]_P
= -g \left[ \frac{\partial z}{\partial y} \right]_P
= -\left[ \frac{\partial \Phi}{\partial y} \right]_P
\]

\sk
\paragraph{Bilan} En coordonnées isobares, 
le terme de gradient de pression horizontal devient irrotationnel
et dérive du géopotentiel~$\Phi$.


\newpage
\section{Equilibre du vent thermique}
\sk
On cherche à relier simplement un équilibre entre température et vent. Si on suppose l'équilibre du vent gradient vérifié, il donne déjà une relation entre champ de vent et champ de pression, donc seules quelques transformations de cette relation sont nécessaires. On note~$x$ la coordonnée sur l'axe est-ouest (axe zonal), $y$ la coordonnée sur l'axe sud-nord (axe méridional), $P$ la pression atmosphérique. Cette dernière est utilisée comme coordonnée verticale en vertu de l'équilibre hydrostatique.

\sk
On définit le géopotentiel~$\Phi$ comme une fonction des coordonnées~$x$, $y$ et~$P$ qui s'écrit simplement
\[ \Phi(x,y,P)=g \, z(x,y,P) \] 
\noindent avec~$z$ l'altitude (également fonction des coordonnées~$x$, $y$ et~$P$) et~$g$ l'accélération de la gravité. Le vent géostrophique zonal~$u$ s'exprime comme une fonction de~$x$, $y$, $P$, tout comme la température atmosphérique~$T$ et la masse volumiquede l'air~$\rho$. Les dérivées partielles (notées~$\partial$) de ces fonctions de trois variables se comprennent comme les dérivées selon la coordonnée indiquée avec les deux autres fixées. Par exemple $\frac{\partial \Phi}{\partial y}$ est la dérivée du géopotentiel~$\Phi$ uniquement selon la coordonnée~$y$, en considérant que~$x$ et~$P$ ne varient pas. 
%On rappelle que les dérivées partielles commutent, c'est-à-dire par exemple 
%\[ \frac{\partial}{\partial P} \frac{\partial \Phi}{\partial y} = \frac{\partial}{\partial y} \frac{\partial \Phi}{\partial P} \]

\sk
On utilise tout d'abord l'équilibre hydrostatique pour exprimer très simplement la dérivée du géopotentiel~$\Phi$ en fonction de la coordonnée verticale~$P$
\[ \frac{\partial \Phi}{\partial P} = g \, \frac{\partial z}{\partial P} = -\f{1}{\rho} \] 
\noindent ce qui permet de relier simplement les variations verticales de géopotentiel (sur les lignes isobares) au champ de masse.
On utilise directement ce résultat, combiné à une propriété de changement de coordonnée dans les dérivées partielles, pour exprimer très simplement la force de pression comme la dérivée spatiale du géopotentiel
\[ \frac{\partial \Phi}{\partial y} = \frac{\partial \Phi}{\partial P} \, \frac{\partial P}{\partial y} = -\frac{1}{\rho} \, \frac{\partial P}{\partial y} \]

\sk
Munis de cette expression simple de la force de pression, on peut alors modifier l'équilibre du vent gradient
\[ \dfrac{u^2\tan\phi}{a} + \fcoriolis u = -\dfrac{1}{\rho}\der{p}{y} = \frac{\partial \Phi}{\partial y} \]
\noindent que l'on peut ensuite dériver par rapport à la coordonnée verticale pression~$P$ 
\[ \left[ 2 \, u \, \dfrac{\tan\phi}{a} + \fcoriolis \right] \der{u}{P} = \der{~}{P} \left[ \frac{\partial \Phi}{\partial y} \right] \]
\noindent afin de pouvoir commuter les dérivées partielles puis utiliser la version de l'équilibre hydrostatique formulée ci-dessus avec le géopotentiel~$\Phi$
\[ \left[ 2 \, u \, \dfrac{\tan\phi}{a} + \fcoriolis \right] \der{u}{P} = \der{~}{y} \left[ \frac{\partial \Phi}{\partial P} \right] = \der{~}{y} \left[ \frac{1}{\rho} \right] \]
\noindent Reste à employer l'équation d'état des gaz parfaits~$P=\rho\,R\,T$ pour faire apparaître la température
\[ \left[ 2 \, u \, \dfrac{\tan\phi}{a} + \fcoriolis \right] \der{u}{P} = R \, \frac{\partial}{\partial y} \left[ \frac{T}{P} \right] 
\qquad \Rightarrow \qquad
\boxed{ \left[ \textcolor{brown}{2\,u\,\frac{\tan\phi}{a}} + \textcolor{red}{\fcoriolis} \right] \der{u}{P} = \frac{R}{P} \, \frac{\partial T}{\partial y} }
\]
\noindent L'expression encadrée de l'équilibre du vent thermique vient de la constatation finale que l'on peut sortir le terme en pression à l'intérieur de la dérivée à droite puisque~$P$ est une coordonnée supposée fixe par définition de la dérivée partielle suivant~$y$. Les termes sont colorés en fonction de l'équilibre dans lequel on se trouve : \textcolor{brown}{équilibre cyclostrophique} (exemple sur Vénus) ou \textcolor{red}{équilibre géostrophique} (exemple sur Mars ou la Terre). Dans le cas où la situation est ambigüe, il faut conserver les deux termes.

\sk
L'\voc{équilibre du vent thermique} exprime un lien diagnostique entre les variations verticales du vent zonal et les variations méridiennes de la température. Sur des planètes où la mesure de température est aisée à mesurer (e.g. par télédétection infrarouge) par rapport au vent, cet équilibre est employé pour calculer un champ de vent (appelé vent thermique) associé à un champ de température. Invariablement, l'équilibre du vent thermique peut permettre de déduire les variations de température associées à un vent donné. Il s'agit d'un équilibre entre deux champs~température$\leftrightarrow$vent, sans relation de causalité température$\rightarrow$vent ou vent$\rightarrow$température.



%\newpage
%\section{Retour sur cellule de Hadley}
%\sk
Sur Mars comme sur Terre, les forçages radiatifs
sont le moteur de la circulation de grande échelle.
%
L'énergie radiative absorbée dans le visible
par le système \ofg{surface - atmosphère}
subit de plus fortes variations latitudinales
que l'énergie radiative émise dans l'\IR.
%
En moyenne annuelle, il en résulte un chauffage net 
des régions équatoriales de la planète et
un refroidissement net de ses régions polaires.
%
Ce contraste thermique induit des contrastes latitudinaux d'échelle de hauteur $H$,
donc, d'après l'équation hypsométrique,
un gradient de pression latitudinal qui augmente avec l'altitude.
%
Une telle force méridienne entraîne une circulation en altitude
des régions équatoriales excédentaires en énergie
aux hautes latitudes en déficit d'énergie.
%
De ce transport de masse résulte une augmentation de la pression de surface
aux hautes latitudes et donc une circulation inversée proche du sol.
%
En moyenne zonale, ces mouvements sont conceptualisés par les
\ofg{cellules de Hadley}.


\sk
La circulation précitée prend place
dans un référentiel particulièrement non-galiléen :
la planète en rotation sur elle-même.
%
Le moment cinétique absolu $\mathcal{M}$ 
d'une particule d'air de masse $m$
par rapport à l'axe de rotation de la planète
est en coordonnées sphériques
%
\[
\mathcal{M} = m \, (a \, \cos \varphi) (\Omega \, a \cos \varphi + u)
\]
%
\noindent pour une particule située à la latitude $\varphi$. 
%
\noindent La conservation de ce moment cinétique
impose à la particule d'air advectée vers
les pôles, donc se rapprochant de son axe de rotation,
d'accélérer en un vent prograde d'altitude (jets d'ouest)
et à la particule d'air ramenée vers l'équateur de
décélérer en un vent rétrograde de surface (alizés).
%%les vents sont créés par transport de moment cinétique
%
La résultante $\vec{F\e{e}}$ des forces 
d'entraînement centrifuge et de Coriolis
%
\[
\vec{F\e{e}} = - m \, (2\,\Omega\,\sin\varphi\,u + \f{u^2\,\tan\varphi}{a}) \, \vec{y}
\]
%
\noindent est alors dirigée vers l'équateur
pour une particule animée d'un vent prograde
$u > 0$,
%tournant plus vite
%que la planète (i.e. animée d'un vent d'ouest ), 
et s'oppose au gradient de pression ayant donné
naissance à la circulation de Hadley, limitant son extension.
%
Au-delà d'une certaine latitude, la conservation du moment cinétique
autour de l'axe planétaire cesse d'être valable
et les circulations non-axisymétriques
prennent une part dominante dans le transport de moment cinétique.
%
%notamment les ondes stationnaires
%et les ondes baroclines.
%
Dans ce cas, le vent ne peut plus être déterminé
quantitativement par conservation du moment cinétique mais
par son lien diagnostique à la structure thermique
(équilibre du vent thermique)
%
\[
- \Dp{\vec{v_H}}{p} = \f{R}{p \, f} \: \vec{z} \wedge \vec{\nabla_p} T
\quad
\textrm{(coord. isobares)}
\]
%
\noindent qui combine l'équilibre vertical hydrostatique  
avec l'équilibre horizontal géostrophique aux moyennes latitudes.
%
Le modèle conceptuel axisymétrique simple de \textit{Held et Hou} [1980]\nocite{Held:80}
se base sur cette distinction entre deux régimes de vent pour en déduire
l'extension latitudinale de la cellule de Hadley $\mathcal{L}$
%et la vitesse
%du vent zonal maximal $\mathcal{U}$
%
%\quad \textrm{et} \quad \mathcal{U} = \f{ \Omega \, \mathcal{L}^2 }{ a }
%
\[
\mathcal{L} = \sqrt{ \f{ 5 \, \Delta \theta \, g \, H }{ 3 \, \Omega^2 \, \theta_0 } }
\]
%
\noindent En utilisant les constantes planétaires de la Terre et Mars,
et les contrastes thermiques typiques
$\Delta \theta\e{T} = 40\U{K}$
et $\Delta \theta\e{M} = 65\U{K}$,
nous obtenons 
$\mathcal{L}\e{M} \sim \mathcal{L}\e{T}$,
soit une cellule de Hadley
significativement plus étendue sur Mars
de rayon deux fois plus petit que celui
de la Terre.
%
Ce modèle reste néanmoins très simplifié et ne s'applique qu'aux 
moyennes annuelles principalement, ce qui en limite fortement
la portée sur des planètes aux saisons très marquées comme Mars.


%La vitesse maximale peut être déterminée
%à partir d'un arbitrage entre le vent
%limite au sens de la conservation du moment cinétique
%et le vent thermique; les valeurs 
%données par \numeroref{eq:heldhou} donnent
%ainsi $\mathcal{U}\e{T} \sim 55\U{m~s^{-1}}$
%et $\mathcal{U}\e{M} \sim 85\U{m~s^{-1}}$, en
%accord correct avec les résultats de modèles
%plus élaborés.




%%L'équilibre du vent thermique donne un 
%%sens des vents conforme à ce qu'on pourrait
%%déduire de la conservation du moment
%%cinétique, sur la base qu'une particule
%%se rapprochant de son axe de rotation
%%est accélérée dans le sens prograde.



%\newpage
%\section{Instabilités barotropes et baroclines}
%\sk
L'équation du vent thermique indique que les jets d'altitude dans la branche descendante de la cellule de Hadley (aux moyennes latitudes) conduisent à un renforcement des gradients latitudinaux de température, qui ne peuvent être résorbés par la circulation de Hadley. Aux moyennes latitudes, les énergies cinétique et thermique sont plus facilement redistribuées par les instabilités non-axisymétriques que par la circulation zonale axisymétrique. L'écoulement zonal axisymétrique des moyennes latitudes terrestres et martiennes peut ainsi donner naissance à une circulation non axisymétrique par des instabilités barotropes et baroclines. A partir d'une certaine latitude, ces instabilités sont bien plus efficaces pour redistribuer l'énergie.

\sk
Les perturbations \voc{barotropes} de l'écoulement moyen se développent en extrayant de l'énergie cinétique au cisaillement horizontal de vent de cet écoulement moyen (ex: courant-jet de forte amplitude). Les tourbillons barotropes ont une structure verticale constante avec l'altitude et transportent de la quantité de mouvement selon la latitude, afin de réduire le cisaillement qui leur a donné naissance.

\sk
L'instabilité \voc{barocline} résulte au contraire des gradients latitudinaux de température aux moyennes latitudes, associés à un cisaillement vertical de vent par l'équilibre du vent thermique. Les ondes baroclines générées transportent de la chaleur (et un peu de quantité de mouvement) en latitude et en altitude pour réduire l'inclinaison des isentropes qui leur a donné naissance. Les perturbations baroclines se développent par conversion de l'énergie potentielle disponible de l'écoulement zonal moyen en énergie cinétique.

%%\note{Stabilité des états d'équilibre~\donc~Réponse à une perturbation~:~s'amplifie-t-elle ? A chaque équilibre son instabilité associée~:~ exemple équilibre hydrostatique et instabilités convectives} \note{Courants-Jet d'altitude branche Hadley descendante~$+$~Renforcement gradients latitudinaux de température~\donc~Instabilités aux moyennes latitudes, Notion d'Energie Potentielle Disponible de l'écoulement zonal moyen.}





\end{document}
