\documentclass[a4paper,DIV16,10pt]{scrartcl}
%%%%%%%%%%%%%%%%%%%%%%%%%%%%%%%%%%%%%%%%%%%%%%%%%%%%%%%%%%%%%%%%%%%%%%%%%%%%%%%%%%%
\usepackage{texcours}
%%%%%%%%%%%%%%%%%%%%%%%%%%%%%%%%%%%%%%%%%%%%%%%%%%%%%%%%%%%%%%%%%%%%%%%%%%%%%%%%%%%
\newcommand{\zauthor}{Aymeric SPIGA}
\newcommand{\zaffil}{Laboratoire de Météorologie Dynamique}
\newcommand{\zemail}{aymeric.spiga@upmc.fr}
\newcommand{\zcourse}{Physique, chimie, écologie de l'environnement}
\newcommand{\zcode}{UE1}
\newcommand{\zuniversity}{UPMC}
\newcommand{\zlevel}{M1 Sciences et Politiques de l'Environnement}
\newcommand{\zsubtitle}{CM9: Fiches complémentaires de cours}
\newcommand{\zlogo}{\includegraphics[height=1.5cm]{decouverte/cours_meteo/UPMC_cart-blanc-Q_7504-703-3.png}}
\newcommand{\zrights}{Copie et usage interdits sans autorisation explicite de l'auteur}
\newcommand{\zdate}{\today}
%%%%%%%%%%%%%%%%%%%%%%%%%%%%%%%%%%%%%%%%%%%%%%%%%%%%%%%%%%%%%%%%%%%%%%%%%%%%%%%%%%%
\begin{document} \inidoc
%%%%%%%%%%%%%%%%%%%%%%%%%%%%%%%%%%%%%%%%%%%%%%%%%%%%%%%%%%%%%%%%%%%%%%%%%%%%%%%%%%%


%%%%
%%%%
\mk \section{Energie reçue du Soleil}
	\sk \subsection{Caractéristiques et domaine de longueurs d'onde}
		\sk
Le Soleil peut être considéré en bonne approximation comme un corps noir car il absorbe tout le rayonnement incident. Sa \ofg{couleur} est dûe à du rayonnement émis et, plus précisément, correspond aux longueurs d'onde où le maximum de rayonnement est émis. D'après la loi de Wien, le Soleil, dont l'enveloppe externe a une température autour de~$6000$~K, a donc un maximum d'émission situé dans le visible à $\lambda\e{max} = 0.5 \mu$m, proche du maximum de sensibilité de l'oeil humain [figure~\ref{fig:BBmax} haut]. Au contraire, la surface terrestre, dont la température typique est d'environ~$288$~K, voit son maximum d'émission situé dans l'infrarouge vers 10~$\mu$m, alors que le rayonnement émis dans les longueurs d'ondes visible est négligeable [figure~\ref{fig:BBmax} bas]. Un raccourci usuel est donc de dire que \ofg{la Terre émet du rayonnement (thermique) dans l'infrarouge alors que le Soleil émet dans le visible}. En toute rigueur, cette affirmation ne parle que du voisinage du maximum d'émission, où la contribution au flux intégré selon toutes les longueurs d'onde est la plus significative. Il est ainsi plus exact de dire que, dans l'atmosphère, la région du spectre où~$\lambda$ est inférieure à environ 4~$\mu$m est dominée par le rayonnement d'origine solaire, alors qu'au-delà, le rayonnement est surtout d'origine terrestre. Il n’y a pratiquement pas de recouvrement entre la partie utile du spectre du rayonnement solaire et celui d’un corps de température ambiante; ce fait est d'une grande importance pour les phénomènes de type effet de serre, qui sont abordés plus loin dans ce cours. On désigne ainsi souvent le rayonnement d'origine solaire par le terme \voc{ondes courtes} et le rayonnement d'origine terrestre par le terme \voc{ondes longues}.

\figsup{0.65}{0.2}{decouverte/cours_meteo/6000K.jpg}{decouverte/cours_meteo/earth.jpg}{Courbes de luminance spectrale d'un corps noir pour différentes températures correspondant notamment au Soleil (haut) et à la Terre (bas). La quantité représentée ici est l'émittance spectrale~$M_\lambda = \pi \, B_\lambda$. Noter la différence d'indexation de l'abscisse et l'ordonnée sur les deux schémas. Le rayonnement thermique émis par la Terre est plusieurs ordres de grandeur moins énergétique que celui émis par le Soleil et le maximum d'émission se trouve à des longueurs d'onde plus grandes (infrarouge pour la Terre au lieu de visible pour le Soleil). Source : \url{http://hyperphysics.phy-astr.gsu.edu/hbase/bbrc.html}.}{fig:BBmax}

	\sk \subsection{Constante solaire}
		\sk
La distance Soleil-Terre est beaucoup plus grande que les rayons de la Terre et du Soleil. Ainsi, d'une part, le rayonnement solaire arrive au niveau de l'orbite terrestre en faisceaux pratiquement parallèles. D'autre part, la luminance en différents points de la Terre ne varie pas. On peut par conséquent définir une valeur moyenne de la densité de flux énergétique du rayonnement solaire au niveau de l'orbite terrestre, reçue par le système surface~+~atmosphère. Elle est désignée par le terme de \voc{constante solaire} notée~$\mathcal{F}\e{s}$. Les mesures indiquent que
\[ \mathcal{F}\e{s} = 1368 \text{~W~m}^{-2} \qquad \text{pour la Terre} \]

\sk
La constante solaire est une valeur instantanée côté jour~: le rayonnement solaire reçu au sommet de l'atmosphère en un point donné de l'orbite varie en fonction de l'heure de la journée et de la saison considérée (c'est-à-dire la position de la Terre au cours de sa révolution annuelle autour du Soleil)\footnote{En réalité, la constante solaire~$\mathcal{F}\e{s}$ varie elle-même d'environ~$3$~W~m$^{-2}$ en fonction des saisons à cause de l'excentricité de l'orbite terrestre, qui n'est pas exactement circulaire. De plus, elle peut varier évidemment en fonction des cycles solaires, néanmoins sans influence majeure sur la température des basses couches atmosphériques (troposphère et stratosphère).}. On peut donc définir un \voc{éclairement solaire moyen} noté~$\mathcal{F}\e{s}'$ reçu par la Terre qui intègre les effets diurnes et saisonniers. Autrement dit, $\mathcal{F}\e{s}$~est l'éclairement instantané reçu par un satellite en orbite autour de la Terre~; $\mathcal{F}\e{s}'$ est la valeur que l'on obtiendrait si l'on faisait la moyenne d'un grand nombre de mesures instantanées du satellite à diverses heures et saisons. 

\figside{0.5}{0.2}{decouverte/cours_dyn/incoming.png}{Energie reçue du Soleil par le système Terre. Source~: McBride and Gilmour, \emph{An Introduction to the Solar System}, CUP 2004.}{fig:eqrad}

\sk
On admet ici que~$\mathcal{F}\e{s}'$ peut être calculé en considérant que le flux total reçu du Soleil l'est à travers un disque de rayon le rayon~$R$ de la Terre (il s'agit de l'ombre projetée de la planète, voir Figure~\ref{fig:eqrad}). A cause de l'incidence parallèle, le flux énergétique intercepté par la Terre vaut donc~$\Phi = \pi \, R^2 \, \mathcal{F}\e{s}$. L'éclairement moyen à la surface de la Terre est alors $$\mathcal{F}\e{s}' = \frac{\Phi}{4 \, \pi \, R^2}$$ le dénominateur étant l'aire de la surface complète de la Terre. On obtient ainsi
\[ \boxed{ \mathcal{F}\e{s}' = \frac{\mathcal{F}\e{s}}{4} } \]

		\sk
La valeur de la constante solaire peut s'obtenir par le calcul. Le soleil est considéré en bonne approximation comme un corps noir de température~$T_{\sun} = 5780$~K. D'après la loi de Stefan-Boltzmann, son émittance est $M = \sigma \, T_{\sun}^4$ donc le flux énergétique~$\Phi_{\sun}$ émis par le Soleil de rayon~$R_{\sun} = 7 \times 10^5$~km est~$\Phi_{\sun} = 4 \, \pi \, R_{\sun}^2 \, \sigma \, T_{\sun}^4$. Ce flux énergétique est rayonné dans tout l'espace~: à une distance~$d$ du soleil il est réparti sur une sphère de centre le soleil et de rayon~$d$, donc de surface~$4 \, \pi \, d^2$. A cette distance, l'éclairement~$\mathcal{F}$, c'est-à-dire la densité de flux énergétique reçue en W~m$^{-2}$, s'écrit donc
\[ \mathcal{F} = \frac{\Phi_{\sun}}{4 \, \pi \, d^2} = \frac{4 \, \pi \, R_{\sun}^2 \, \sigma \, T_{\sun}^4}{4 \, \pi \, d^2} = \sigma \, T_{\sun}^4 \, \left( \frac{R_{\sun}}{d} \right)^2 \]
Si l'on prend~$d$ égal à la distance Terre-Soleil, $\mathcal{F}$ définit ainsi la constante solaire~$\mathcal{F}\e{s}$.
%\[ \mathcal{F}\e{s} = \frac{{\mathcal{F}\e{s}}^{\text{Terre}}}{d\e{soleil}^2} \]

%Variation de la constante solaire : Bien que l’intensité du soleil ait subit des variations depuis la formation de la Terre, on peut s’attendre à ce qu’elle soit stable sur une période de 1000 ans. On mesure mal la constante solaire, mais les mesures récentes, même avec leurs incertitudes, semblent indiquer que le soleil ne peut pas expliquer le réchauffement récent. Notons toutefois que les simulations actuelles ne tiennent pas compte des fluctuations possibles du rayonnement solaire (négligeable a priori).
%%%% pas sûr du dernier point.


%%%%
%%%%
\mk \section{Equilibre radiatif simple}
	\sk \subsection{Flux reçu et flux émis}
		\sk
Nous pouvons exprimer le rayonnement reçu du Soleil par la Terre par une densité de flux énergétique moyenne~$F\e{reçu}$ en W~m$^{-2}$ ou un flux énergétique~$\Phi\e{reçu}$ (en W)
\[ 
F\e{reçu} = (1-A\e{b}) \, \mathcal{F}\e{s}' 
\qquad \qquad
\Phi\e{reçu} = \pi \, R^2 \, (1-A\e{b}) \, \mathcal{F}\e{s}
\] 
La partie du rayonnement reçue du soleil qui est réfléchie vers l'espace est prise en compte via l'albédo planétaire noté~$A\e{b}$. On rappelle par ailleurs que~$\mathcal{F}\e{s}' = \mathcal{F}\e{s} / 4$ où $\mathcal{F}\e{s}$ est la constante solaire.


\sk
Par ailleurs, le système Terre émet également du rayonnement principalement dans les longueurs d'onde infrarouge [figure \ref{fig:eqrad2}]. 
Cette quantité de rayonnement émise au sommet de l'atmosphère radiative est notée $OLR$ pour \emph{Outgoing Longwave Radiation} en anglais.
A l'équilibre, la planète Terre doit émettre vers l'espace autant d'énergie qu'elle en reçoit du Soleil, donc
on obtient la relation générale appelée \emph{TOA} pour \emph{Top-Of-Atmosphere} en anglais, correspondant
au bilan radiatif au sommet de l'atmosphère
\[ \boxed{\TOA} \] 
La principale difficulté qui sous-tend les divers modèles pouvant être proposés réside dans l'expression du terme~$OLR$.



	\sk \subsection{Equilibre et température équivalente}
		\sk
Dans l'équilibre~\emph{TOA}, la manière la plus simple de définir~$OLR$ pour entamer un calcul préliminaire est comme suit. On fait l'hypothèse, assez réaliste en pratique, que la surface de la Terre est comme un corps noir, c'est-à-dire que son émissivité est très proche de~$1$ dans l'infrarouge où se trouve le maximum d'émission. D'après la loi de Stefan-Boltzmann, la densité de flux énergétique~$F\e{émis}$ émise par la Terre en W~m$^{-2}$ s'exprime
\[ F\e{émis} = \sigma \, {T\e{eq}}^4 \]
où~T\e{eq} est la \voc{température équivalente} du système Terre que l'on suppose uniforme sur toute la planète. Autrement dit, $T\e{eq}$ est la température équivalente d'un corps noir qui émettrait la quantité d'énergie~$F\e{émis}$. Le flux énergétique~$\Phi\e{émis}$ émis par la surface de la planète Terre s'exprime
\[ \Phi\e{émis} = 4 \, \pi \, R^2 \, F\e{émis} = 4 \, \pi \, R^2 \, \sigma \, {T\e{eq}}^4 \]
Contrairement au cas de l'énergie visible, il n'y a pas lieu de prendre en compte le contraste jour/nuit, car le rayonnement thermique émis par la Terre l'est à tout instant par l'intégralité de sa surface. La seule limite éventuellement discutable est l'uniformité de la température de la surface de la Terre, ce qui est irréaliste en pratique. On peut souligner cependant que, même dans le cas d'une planète n'ayant pas une température uniforme ou ne se comportant pas comme un corps noir, le rayonnement émis vers l'espace doit être égal en moyenne à $\sigma \, {T\e{eq}}^4$.
%% CHANGER LES SLIDES, ne pas utiliser P

\figsup{0.31}{0.17}{decouverte/cours_dyn/incoming.png}{decouverte/cours_dyn/emission.png}{Equilibre radiatif simple : à gauche, l'énergie reçue du Soleil par le système Terre ; à droite, l'énergie émise par le système Terre. Source~: McBride and Gilmour, \emph{An Introduction to the Solar System}, CUP 2004.}{fig:eqrad2}

\sk
A l'équilibre, la planète Terre doit émettre vers l'espace autant d'énergie qu'elle en reçoit du Soleil (équilibre \emph{TOA}). Ceci peut s'exprimer par unité de surface
\[ \boxed{ F\e{reçu} = F\e{émis} } \]
ou, pour un résultat similaire, en considérant l'intégralité de la surface planétaire
\[ \Phi\e{reçu} = \Phi\e{émis} \]
ce qui permet de déterminer la température équivalente en fonction des paramètres planétaires
\[ \boxed{
T\e{eq} = \bigg[ \frac{\mathcal{F}\e{s}'\,(1-A\e{b})}{\sigma} \bigg]^{\frac{1}{4}}
} \]


		Le calcul présenté ici porte le nom d'\voc{équilibre radiatif simple}. On y néglige les effets de l'atmosphère (sauf l'albédo) puisqu'on suppose que le rayonnement atteint la surface, ou est rayonné vers l'espace, sans être absorbé par l'atmosphère. La température équivalente est ainsi la température qu'aurait la Terre si l'on négligeait tout autre influence atmosphérique que la réflexion du rayonnement solaire incident. Les valeurs de $T\e{eq}$ pour quelques planètes telluriques sont données dans la table \ref{tab:planets}. On note que la température équivalente de Vénus est plus faible que celle de la Terre, bien qu'elle soit plus proche du Soleil, à cause de son fort albédo~; la formule indique bien que, plus le pouvoir réfléchissant d'une planète est grand, plus la température de sa surface est froide. Par ailleurs, comme indiqué par les calculs du tableau~\ref{tab:planets}, on remarque que la température équivalente, si elle peut renseigner sur le bilan énergétique simple de la planète, ne représente pas correctement la valeur de la température de surface. Par exemple, la température équivalente pour la Terre est~$T\e{eq} = 255 K = -18^{\circ}$C, bien trop faible par rapport à la température de surface effectivement mesurée. Il faut donc avoir recours à un modèle plus élaboré.

\begin{table}\label{tab:planets} \begin{center} \begin{tabular}{lccccc} & \emph{Mercure} & \emph{V\'enus} & \emph{Terre} & \emph{Mars} & \emph{ Titan} \\ \hline $d\e{soleil}$ (UA) & 0.39 & 0.72 & 1 & 1.5 & 9.5 \\ $\mathcal{F}\e{s}\,$(W~m$^{-2}$) & $8994$ & $2614$ & $1367$ & $589$ & $15$ \\ $A\e{b}$ & $0.06$ & $0.75$ & $0.31$ & $0.25$ & $0.2$ \\ \textcolor{blue}{$T\e{surface}$ (K)} & \textcolor{blue}{$100/700$~K} & \textcolor{blue}{$730$} & \textcolor{blue}{$288$} & \textcolor{blue}{$220$} & \textcolor{blue}{$95$} \\ \hline $T\e{eq}$~(K) & $439$ & $232$ & $254$ & $210$ & $86$\\ \end{tabular} \caption{\emph{Comparaison des facteurs influençant la température équivalente du corps noir pour différentes planètes du système solaire.}} \end{center} \end{table}
%    Mercure & 0.39 & 8994 & 0.06 & 439 \\
%    Vénus & 0.72 & 2639 & 0.78 & 225 \\
%    Terre & 1 & 1368 & 0.30 & 255 \\
%    Mars & 1.52 & 592 & 0.17 & 216 \\

	\sk \subsection{Equilibre TOA}
		\sk
Nous pouvons exprimer le rayonnement reçu du Soleil par la Terre par une densité de flux énergétique moyenne~$F\e{reçu}$ en W~m$^{-2}$ ou un flux énergétique~$\Phi\e{reçu}$ (en W)
\[ 
F\e{reçu} = (1-A\e{b}) \, \mathcal{F}\e{s}' 
\qquad \qquad
\Phi\e{reçu} = \pi \, R^2 \, (1-A\e{b}) \, \mathcal{F}\e{s}
\] 
La partie du rayonnement reçue du soleil qui est réfléchie vers l'espace est prise en compte via l'albédo planétaire noté~$A\e{b}$. On rappelle par ailleurs que~$\mathcal{F}\e{s}' = \mathcal{F}\e{s} / 4$ où $\mathcal{F}\e{s}$ est la constante solaire.


\sk
Par ailleurs, le système Terre émet également du rayonnement principalement dans les longueurs d'onde infrarouge [figure \ref{fig:eqrad2}]. 
Cette quantité de rayonnement émise au sommet de l'atmosphère radiative est notée $OLR$ pour \emph{Outgoing Longwave Radiation} en anglais.
A l'équilibre, la planète Terre doit émettre vers l'espace autant d'énergie qu'elle en reçoit du Soleil, donc
on obtient la relation générale appelée \emph{TOA} pour \emph{Top-Of-Atmosphere} en anglais, correspondant
au bilan radiatif au sommet de l'atmosphère
\[ \boxed{\TOA} \] 
La principale difficulté qui sous-tend les divers modèles pouvant être proposés réside dans l'expression du terme~$OLR$.




%%%%
%%%%
\mk \section{Effet de serre : modèle à deux faisceaux}
	\sk \subsection{\'Epaisseur optique}
		\sk
Considérons une espèce~$X$ bien mélangée dans l'atmosphère, qui absorbe dans un intervalle de longueur d'onde donné. A la longueur d'onde~$\lambda$, son \voc{épaisseur optique}~$t_\lambda$ s'écrit
\[ \boxed{ t_\lambda = \int_{0}^{z\e{sommet}} \, k_\lambda \, \rho_X \, \dd z } \]
où $k_\lambda$ est un coefficient d'absorption massique en m$^2$~kg$^{-1}$ et $\rho_X$ est la densité d'absorbant~X. Le nom d'épaisseur optique est assez intuitif. Si un flux de rayonnement~$\Phi_\lambda$ à la longueur d'onde~$\lambda$ est émis à la base de l'atmosphère, le flux observé au sommet de l'atmosphère est d'autant plus réduit qu'à cette longueur d'onde l'épaisseur optique de l'atmosphère traversée est grande\footnote{Si l'extinction est uniquement due à de l'absorption, sans diffusion, on a une relation directe entre l'épaisseur optique et le coefficient d'absorption de la couche~: 
\[\alpha_\lambda = 1 - e^{- \frac{t_\lambda}{\cos\theta}} \] où~$\theta$ est l'angle d'incidence du rayonnement danns la couche.}. La formule ci-dessus ne fait qu'exprimer le fait que la réduction du flux (l'extinction) est plus d'autant plus marquée que 
\begin{citemize}
\item l'espèce considérée est très absorbante dans la longueur d'onde considérée ($k_\lambda$ grand)~;
\item l'espèce considérée est présente en grande quantité ($\rho_X$ grand).
\end{citemize}
Ainsi, le dioxyde de carbone~CO$_2$, bien qu'étant un composant minoritaire ($\rho$ faible), peut atteindre des épaisseurs optiques très grandes dans les intervalles de longueur d'onde où il est très fortement absorbant ($k_\lambda$ élevé), par exemple dans l'infrarouge autour de~$15$~$\mu$m. Autrement dit, un composant minoritaire en quantité peut avoir un rôle majoritaire radiativement.

	\sk \subsection{Modèle à deux faisceaux : écriture}
		\sk
Le modèle à deux faisceaux est un bon compromis entre simplicité
et illustration de concepts importants. Il est une version simplifiée
de l'équation de Schwarzschild du transfert radiatif.
Ce modèle entend élucider
les transferts de rayonnement dans l'infrarouge entre
les couches qui composent la colonne atmosphérique. Les
hypothèses simplificatrices suivantes sont réalisées
\begin{citemize}
\item couches atmosphériques plan-parallèle (sphéricité négligée)
\item phénomènes d'absorption négligés dans le visible (transparence au visible)
\item phénomènes de diffusion (\emph{scattering}) négligés dans l'infrarouge
\item \emph{gray gas} dans l'infra-rouge : on considère que le coefficient d'absorption
du gaz est indépendant de la longueur d'onde~$\lambda$ ($k_{\lambda} = k$ pour tout~$\lambda$),
ce qui implique une hypothèse similaire pour l'épaisseur optique ($\tau_{\lambda} = \tau$ pour tout~$\lambda$).
\end{citemize}
En d'autres termes, on se cantonne dans ce modèle à deux types de phénomènes
\begin{enumerate}
\item Un faisceau de rayonnement infra-rouge de flux~$F$ traversant une couche 
atmosphérique donnée
subit une extinction à cause de l'absorption selon une loi de type Beer-Lambert
\[
\dd F = - F \dd \tau
\]
avec~$\tau$ l'épaisseur optique \emph{gray gas} 
dans l'infra-rouge.
\item Une couche atmosphérique émet un flux de rayonnement thermique~$M$ 
calculé par la loi intégrée de Stefan-Boltzmann ($M=\epsilon\,\sigma\,T^4$)
puisque la majorité de l'émittance est émise dans l'infrarouge pour les températures considérées.
\end{enumerate}

\sk
L'épaisseur optique~$\tau$ peut servir de coordonnée verticale à la place de~$z$
en utilisant la relation entre les deux quantités. La couche atmosphérique
élémentaire considérée est ainsi d'épaisseur~$\dd\tau$ et située à une coordonnée
verticale~$\tau$ qui croît avec l'altitude. 

\sk
Si l'on considère un faisceau ascendant~$F^+(\tau)$ au bas de la couche considérée,
une fois la couche traversée son amplitude est
\[
F^+(\tau) - F^+(\tau) \dd\tau
\]
A ce flux au sommet de la couche, il convient d'ajouter
la contribution de l'émission thermique de la couche vers 
le haut, à savoir~$M\,\dd\tau$.
Le flux total ascendant au sommet de la couche est donc
\[
F^+(\tau+\dd\tau) = F^+(\tau) - F^+(\tau) \dd\tau + M\dd\tau
\]

\sk
Même raisonnement avec le flux descendant~$F^-(\tau+\dd\tau)$ au sommet de la couche considérée,
une fois la couche traversée son amplitude est~$F^-(\tau+\dd\tau) - F^-(\tau) \dd\tau$, où 
l'approximation du terme du second ordre~$F^-(\tau+\dd\tau) \dd\tau \simeq F^-(\tau) \dd\tau$
a été effectuée.
Le flux total descendant au bas de la couche est donc
\[
F^-(\tau+\dd\tau) = F^-(\tau) - F^-(\tau) \dd\tau + M\dd\tau
\]

\sk
Les deux résultats qui précèdent peuvent être transformés 
afin de faire apparaître une dérivée
en utilisant le théorème des accroissements finis
\[
\ddf{F^+}{\tau} = \frac{F^+(\tau+\dd\tau) - F^+(\tau)}{\dd\tau}
\]
\noindent ce qui permet d'obtenir au final
\[
\ddf{F^+}{\tau} = - F^+(\tau) + \epsilon\,\sigma\,T(\tau)^4 \quad [S^+]
\qquad\qquad 
\ddf{F^-}{\tau} = F^-(\tau) - \epsilon\,\sigma\,T(\tau)^4 \quad [S^-]
\]
\noindent $[S^+]$ et~$[S^-]$ sont parfois appelées les relations de Schwarzschild à deux faisceaux.
Il s'agit d'une version très simplifiée des équations de Schwarzschild du transfert radiatif.

\sk
Si l'on souhaite adopter la convention~$\tau=0$ au sommet de l'atmosphère,
et $\tau=\tau_{\infty}$ à la surface en $z=0$, donc adopter un axe
vertical d'épaisseur optique avec~$\tau$ croissant de haut en bas, il
suffit de remplacer~$\tau$ par~$-\tau$ dans les équations précédentes pour obtenir
\[
\boxed{\ddf{F^+}{\tau} = F^+(\tau) - \epsilon\,\sigma\,T(\tau)^4 \quad [S^+]} 
\qquad\qquad 
\boxed{\ddf{F^-}{\tau} = -F^-(\tau) + \epsilon\,\sigma\,T(\tau)^4 \quad [S^-]}
\]








	\sk \subsection{Modèle à deux faisceaux : résolution 1}
		\sk
Le système d'équations~$[S^+]$ et~$[S^-]$ du modèle à deux faisceaux
est plus simple à résoudre si l'on considère les deux quantités~$\Sigma(\tau)=F^{+}(\tau)+F^{-}(\tau)$ et~$\Delta(\tau)=F^{+}(\tau)-F^{-}(\tau)$ car on obtient
\[
\ddf{\Sigma}{\tau} = \Delta(\tau) \quad [E_\Sigma] 
\qquad\qquad 
\ddf{\Delta}{\tau} = \Sigma(\tau) - 2\,\epsilon\,\sigma\,T(\tau)^4 \quad [E_\Delta]
\]
\noindent Ensuite la résolution impose d'expliciter les conditions aux limites
\begin{enumerate}[label=$\mathcal{C}_\arabic*$]
\item on se place à l'équilibre radiatif donc le flux net~$\Delta$ est constant à tout niveau : $\ddf{\Delta}{\tau}=0$
\item au sommet de l'atmosphère $F^+(\tau=0) = OLR$ (définition de $OLR$) et $F^-(\tau=0) = 0$ (contribution
incidente négligeable du Soleil dans l'infra-rouge), ce qui s'écrit encore~$\Delta(\tau=0)=\Sigma(\tau=0)=OLR$
\item à la surface de température~$T_s$ le bilan radiatif est le suivant : la surface reçoit l'intégralité du rayonnement
solaire incident~$(1-A\e{b}) \, \mathcal{F}\e{s}'$ (visible) plus du rayonnement de l'atmosphère située
juste au-dessus d'elle~$F^-(\tau=\tau_{\infty})$ (infra-rouge) ; de plus elle émet un rayonnement
$\epsilon\,\sigma\,T\e{s}^4$ dans l'infra-rouge vers l'atmosphère\footnote{On a supposé ici pour simplifier les calculs que l'émissivité
de la surface était similaire à l'émissivité de l'atmosphère}
\item on rappelle que selon la relation \emph{TOA}, nous avons $OLR = (1-A\e{b}) \, \mathcal{F}\e{s}'$
\end{enumerate}

	\sk \subsection{Modèle à deux faisceaux : résolution 2}
		\sk
\paragraph{Conséquence 1} Il est alors possible d'obtenir deux expressions différentes pour~$\Sigma(\tau)$.
Premièrement, en utilisant $[E_\Sigma]$ avec $\mathcal{C}_1$ et $\mathcal{C}_2$, on obtient~$\Sigma(\tau)=OLR \, (1+\tau)$.
Deuxièmement, en utilisant $[E_\Delta]$ avec $\mathcal{C}_1$, on obtient~$\Sigma(\tau)=2\,\epsilon\,\sigma\,T(\tau)^4$.
On obtient le \voc{profil radiatif}, 
c'est-à-dire le profil vertical de température\footnote{Suivant la géométrie
équivalente choisie pour le modèle plan-parallèle, le terme $1+\tau$
peut s'écrire un peu différemment, mais quoiqu'il en soit toujours sous une
forme~$a+b\,\tau$ avec $a,b$ constants. Les conclusions énoncées ici ne sont pas
modifiées.} imposé par les transferts radiatifs dans l'infrarouge
\[
T(\tau) = \sqrt[4]{\frac{OLR\,(1+\tau)}{2\,\sigma\,\epsilon}}
\]

\sk
\paragraph{Conséquence 2} Reste à calculer la température de surface avec ce modèle. D'après $\mathcal{C}_3$, le bilan
au sol s'écrit~$(1-A\e{b}) \, \mathcal{F}\e{s}' + F^-(\tau=\tau_{\infty}) = \epsilon\,\sigma\,T\e{s}^4$.
Il faut donc exprimer les flux ascendant et descendant dans l'infrarouge.
Du fait que $\mathcal{C}_1$ et $\mathcal{C}_2$ nous indiquent que~$\Delta=OLR$, on obtient aisément
\[
F^+(\tau) = \frac{\Sigma+\Delta}{2} = OLR \, (1+\frac{\tau}{2})
\qquad \qquad
F^-(\tau) = \frac{\Sigma-\Delta}{2} = OLR \, \frac{\tau}{2}
\]
\noindent On obtient alors l'expression liant $OLR$
et température de surface~$T\e{s}$
\[
\boxed{\epsilon\,\sigma\,T\e{s}^4 = OLR \, \left( 1 + \frac{\tau_{\infty}}{2} \right)}
\]
\noindent On obtient ainsi une définition quantitative de \voc{l'effet de serre}
\begin{citemize}
\item Dans l'infrarouge, le rayonnement sortant au sommet de l'atmosphère ($OLR$)
est inférieur au rayonnement émis par la surface ($\epsilon\,\sigma\,T\e{s}^4$).
Une partie du rayonnement émis par la surface reste donc piégée par la planète.
\item Avec un albédo et un rayonnement incident constant, donc à~$OLR$ constant (d'après $\mathcal{C}_4$),
augmenter la quantité de gaz à effet de serre (donc augmenter~$\tau_{\infty}$)
conduit à une augmentation de la température de surface~$T\e{s}$.
\end{citemize}

\sk
\paragraph{Conséquence 3} Il est alors instructif de s'intéresser à la température atmosphérique 
proche de la surface~$T(\tau_\infty)$
donnée par le profil radiatif. Cette température ne dépend que de~$OLR$
et s'obtient totalement indépendamment de la température de surface.
On peut alors montrer que
\[
T\e{s} = T(\tau_\infty) \, \sqrt[4]{\frac{2+\tau_\infty}{1+\tau_\infty}} > T(\tau_\infty)
\]
\noindent Tant que l'atmosphère n'est pas
optiquement épaisse dans l'infrarouge (donc tant que~$\tau_\infty$ reste fini),
il existe une \voc{discontinuité entre la surface et l'atmosphère}, la surface
étant toujours plus chaude que l'atmosphère. Cela implique que l'atmosphère
est instable proche de la surface, donc que du mélange turbulent / convectif
apparaît, donc que l'équilibre proche de la surface ne peut être simplement
radiatif mais \voc{radiatif-convectif}. Notons que dans le cas où l'atmosphère est optiquement épaisse,
$T\e{s} = T(\tau_\infty)$, ce qui est tout à fait représentatif des conditions sur Vénus.


%%% figure Salby

	\sk \subsection{Profil radiatif convectif (telluriques)}
		\sk
Les conditions atmosphériques sont très instables proche d'une surface (en présence d'une telle surface). A cause de la discontinuité entre surface et atmosphère, sous l'action de la diffusion thermique, ou turbulente, entre la surface (chaude) et l'air immédiatement adjacent (plus froid) crée une couche d'air fine approximativement à la température de la surface ; les conditions de température étant plus froides au-dessus, les conditions atmosphériques sont très instables proche de la surface et des mouvements de convection vont se mettre en place pour mélanger l'air sur une certaine épaisseur atmosphérique. Un équilibre dit \voc{radiatif-convectif} prévaut, avec une structure thermique suivant le profil adiabatique, donnant naissance à une troposphère. Au-dessus de la limite radiative-convective (correspondant peu ou prou à la tropopause), les phénomènes radiatifs dominent et donnent naissance à une mésosphère -- ou une stratosphère si un absorbant visible y est présent en quantité suffisante, donnant naissance à une inversion stable à la tropopause.
%% on passait en troposphère dès que le gradient du profil radiatif dépassait celui du profil adiabatique (-g/cp)

\figun{0.4}{0.25}{decouverte/pierrehumbert_pics/9780521865562c03_fig014.jpg}{Figure tirée de R. Pierrehumbert, Principles of Planetary Climates, CUP, 2010}{fig:effetserre2}










%\newpage
%\section{Effet de serre : déplacement d'équilibre}
%


%\figside{0.45}{0.3}{/home/aymeric/Big_Data/BOOKS/pierrehumbert_pics/9780521865562c03_fig005.jpg}{R. Pierrehumbert, Principles of Planetary Climates, CUP, 2010}{fig:effetserre1}

\sk
Nous présentons ici l'explication la plus simple (sans être simpliste) du déplacement
d'équilibre radiatif qu'induit l'augmentation de gaz à effet de serre.
%Le modèle à deux faisceaux ne nous aide pas énormément. diverge quand tau tend vers infini.

\sk
Il est possible de montrer par des calculs de transfert radiatif que le niveau
d'émission équivalent au sommet de l'atmosphère est tel que~$\tau = 1$.
Qualitativement, on comprend que les niveaux inférieurs sont optiquement
épais donc ne sont que marginalement ``vus'' depuis l'espace dans les longueurs d'onde infrarouges.
Ainsi dans l'équilibre TOA
\[ \TOA \] 
\noindent l'émission de rayonnement au niveau~$\tau=1$ 
à la température~$T(\tau=1)$ domine OLR.

\sk
Appelons~$P\e{rad}$ la pression du niveau~$\tau=1$. 
Pour relier
les deux quantités, on emploie la définition de l'épaisseur optique
$\EO$ que l'on combine
à l'équilibre hydrostatique pour obtenir
par intégration~$\tau = \kappa \frac{P}{g} \, q_X$,
avec $q_X$ le rapport de mélange massique 
de l'espèce~$X$ absorbante dans l'infrarouge.
Ainsi
\[ P\e{rad} = \frac{g}{\kappa \, q_X} \]

\figun{0.7}{0.25}{/home/aymeric/Big_Data/BOOKS/pierrehumbert_pics/9780521865562c03_fig006.jpg}{R. Pierrehumbert, Principles of Planetary Climates, CUP, 2010}{fig:effetserre2}

\sk
L'expression ci-dessus implique qu'une augmentation de
gaz à effet de serre ($q_X$ augmente) implique une 
élévation du niveau équivalent d'émission
($p\e{rad}$ diminue).
L'effet sur la température de surface se détermine alors
en écrivant la conservation de la température potentielle
dans la troposphère soumise à l'équilibre radiatif-convectif,
entre la surface et le niveau équivalent d'émission
\[ T_s = T\e{rad} \, \left( \frac{P\e{rad}}{P\e{s}} \right)^{-\frac{R}{c_p}} \]
\noindent où~$P_s$ est la pression de surface.
Une élévation du niveau équivalent d'émission
se traduit donc par une augmentation
de température (fournissant
un modèle à la fois simple et fidèle du 
changement climatique récent sur Terre, Figure~\ref{fig:effetserre2}). 
Approximativement, $OLR \sim \sigma \, T(P\e{rad})^4$
et~$P\e{rad}$ est alors défini par la 
condition TOA qui s'écrit~$OLR = (1-A\e{b}) \, \mathcal{F}\e{s}'$.
Le lien entre quantité de gaz à effet de serre~$q_X$
et température de surface~$T\e{s}$ peut ainsi s'écrire
\[ T\e{s} = \sqrt[4]{\frac{(1-A\e{b}) \, \mathcal{F}\e{s}'}{\sigma}} \, \left( \frac{\kappa \, q_X \,P\e{s}}{g} \right)^{\frac{R}{c_p}} \]

%\newpage
%\section{Effet de serre divergent}
%
%% http://www.skepticalscience.com/print.php?r=262

\sk
\paragraph{Approche rapide} Une augmentation de la température de surface~$T\e{s}$ 
est donc associée à une augmentation de la quantité de gaz à effet de serre~$q_X$.
Sur une planète pourvue d'océan,
une augmentation de la température de surface
provoque une augmentation de l'évaporation
donc de la quantité de vapeur d'eau dans l'atmosphère, 
par conséquent de l'effet de serre.
Il s'agit d'une \voc{rétroaction positive}~:
le système amplifie la perturbation initiale de température de surface.
La quantité~$q_X$ de vapeur d'eau dans l'atmosphère
peut donc virtuellement augmenter indéfiniment.
Néanmoins, la pression du niveau équivalent~$P\e{rad} = \frac{g}{\kappa \, q_X}$
ne peut diminuer indéfiniment, du moins continûment~:
lorsque le sommet de la couche radiative de l'atmosphère
est atteint $P\e{rad} \ll P\e{s}$ , la radiation sortante~OLR
atteint une valeur maximale~$OLR\e{max}$.

\sk
\paragraph{Approche plus subtile} On peut inverser le point de vue et se demander quel est la valeur
d'OLR qui correspond à une température de surface~$T\e{s}$.
D'après le modèle simplifié combinant la hauteur équivalente
d'émission et le profil radiatif-convectif dans la troposphère, nous avons
\[ OLR  = \textcolor{magenta}{\sigma \, T\e{s}^4} \textcolor{blue}{\left( \frac{g}{\kappa \, q_X \, P\e{s}} \right)^{\frac{4 \, R}{c_p}}} \]
Plaçons-nous toujours dans le cas d'une planète pourvue
d'océan en évaporation.
Pour les températures de surface relativement modérées,
les variations de quantité de vapeur d'eau~$q_X$
(et de pression de surface~$P\e{s}$)
sont modérées et les variations d'OLR suivent 
une loi en~$\sigma T\e{s}^4$ (terme en \textcolor{magenta}{magenta}).
Néanmoins, plus la température de surface~$T\e{s}$
augmente, plus la vapeur d'eau devient dominante
dans l'atmosphère en influençant~$q_X$, mais
surtout~$P\e{s}$ via la loi d'équilibre liquide-vapeur
de Clausius-Clapeyron
$  P\e{s}(T\e{s}) = P_0 \, \exp{ \left[ -\frac{\ell}{R\,T\e{s}} \right] } $
\noindent où~$\ell > 0$ est la chaleur latente de vaporisation.
Si $T\e{s}$ augmente, $P\e{s}(T\e{s})$ augmente, et de manière exponentielle.
Le terme en \textcolor{blue}{bleu}, qui décroît exponentiellement avec
la température~$T\e{s}$, influence de façon dominante
l'expression de l'OLR pour les température élevées.
L'effet combiné des deux termes 
(\textcolor{magenta}{magenta} et \textcolor{blue}{bleu})
impose donc que l'OLR atteint une valeur maximale~$OLR\e{max}$,
que l'on appelle limite de Komabayashi-Ingersoll
(du nom de deux auteurs d'articles indépendants parus à la fin des années 60).

%% EM: En revanche, l'asymptote n'est qu'approximativement horizontale, elle est légèrement décroissante en présence d'un gaz à effet de serre non condensable (typiquement CO2). Du coup, il vaut mieux distinguer le maximum (KI limit) et l'asymptote plus basse (limite de Nakajima, voir Fig. 3 de https://journals.ametsoc.org/doi/pdf/10.1175/1520-0469%281992%29049%3C2256%3AASOTGE%3E2.0.CO%3B2) 

\figside{0.4}{0.15}{/home/aymeric/Big_Data/BOOKS/pierrehumbert_pics/9780521865562c04_fig004.jpg}{R. Pierrehumbert, Principles of Planetary Climates, CUP, 2010}{fig:ki}

\sk
\paragraph{Effet de serre divergent} 
Quelle que soit l'approche adoptée pour définir~$OLR\e{max}$,
il existe cette limite lorsque toute l'atmosphère devient optiquement épaisse.
Si l'on se place dans un contexte de variation
(à l'échelle des temps géologiques) du flux incident
solaire~$(1-A\e{b}) \, \mathcal{F}\e{s}'$,
avec notamment une augmentation au cours du temps
étant donné l'activité radioactive du Soleil\footnote{En fait, le flux incident varie lorsque la pression atmosphérique devient conséquente à cause d'un effet de diffusion Rayleigh accru},
on constate que les valeurs du flux 
incident~$(1-A\e{b}) \, \mathcal{F}\e{s}'$ peuvent
dépasser~$OLR\e{max}$, ce qui signifie que
l'équilibre TOA ne peut être satisfait et que
l'atmosphère reçoit plus d'énergie qu'elle
n'en émet. La température de surface peut augmenter
de manière incontrôlée, au risque d'atteindre des valeurs
très élevées (plusieurs centaines de K, voire quelques milliers).
L'évaporation des océans peut alors
survenir de manière abrupte et rapide (Figure~\ref{fig:ki}),
dans ce que l'on appelle l'\voc{effet de serre divergent}
(\emph{runaway greenhouse}). De fait, 
l'équilibre radiatif type TOA peut n'être
récupéré que pour des températures de surface 
très élevées (valeurs de plus de~$1000-2000$~K,
pour lesquelles l'intégralité des océans a disparu
selon toute vraisemblance).
Au-delà de telles valeurs de température de surface~$T\e{s}$,
le flux sortant OLR se remet à augmenter avec~$T\e{s}$
en raison de la contribution grandissante de l'émission
thermique dans le visible (et de la moindre absorption
de la vapeur d'eau dans ces longueurs d'onde).


%%% KI : dépend de g


%% Fs + when Ts + because increased absorption of solar rad by water vapor
%% then - when Ts + because Rayleigh scattering



%%%%
%%%%
\mk \section{Description complète du bilan radiatif du système Terre}
	\sk \subsection{Mesures en moyenne dans le temps et dans l'espace}
		\sk
Une représentation détaillée des différents flux échangés en moyenne temporelle et spatiale sur la Terre est présentée sur la figure~\ref{fig:bilflux}, qui est dérivée d'observations satellite les plus récentes. La figure est construite conformément à la séparation visible / infrarouge dictée par les résultats de la figure~\ref{fig:atmspectrum}. 

\sk
\subsubsection{Domaine visible}

\sk
Seulement~$50\%$ du rayonnement solaire incident dans les longueurs d'onde visible parviennent à la surface à cause, d'une part, de la réflexion/diffusion sur les molécules de l'air (diffusion Rayleigh dans toutes les directions, responsable de la couleur bleue du ciel), sur les gouttelettes nuageuses (diffusion de Mie) et sur la surface, et d'autre part, de l'absorption du rayonnement solaire incident par les molécules\footnote{Dans la mésosphère, c'est l'oxygène qui absorbe les radiations les plus énergétiques~; dans la stratosphère, l'absorption des radiations dans l’ultraviolet est assurée par différentes bandes d'absorption de l'ozone~; cette absorption peut avoir lieu dans certaines bandes jusque dans la troposphère.} et les aérosols composant l'atmosphère [relativement modérée dans les longueurs d'onde visible]. On note que la partie du rayonnement visible diffusée vers l'espace par les molécules de l'air, les nuages et la surface définit l'albédo planétaire mentionné précédemment~: un albédo élevé contribue à refroidir la surface et l'atmosphère. L'absorption de la lumière ultraviolet/visible, quant à elle, réchauffe directement l'atmosphère (notamment dans la stratosphère, car la troposphère n'est que très faiblement chauffée par les radiations solaires) et contribue à refroidir la surface par extinction du flux solaire incident. Dans le domaine visible, l'extinction est causée principalement par la diffusion et moins par l'absorption. La partie du rayonnement qui parvient à la surface est absorbée par la surface et convertie en énergie interne, c'est-à-dire contribue à élever sa température. On remarque que la surface terrestre ne peut être considérée tout à fait comme un corps noir puisqu'elle n'absorbe pas toute l'énergie incidente~: une petite partie du rayonnement incident est réfléchie par la surface. Cette composante réfléchie par la surface dépend fortement de la nature des sols (océans, forêts, déserts, glace, \ldots) et de leur répartition géographique.

\sk
\subsubsection{Domaine infrarouge}

\sk
Chauffée par l'absorption du rayonnement solaire incident, la surface terrestre se refroidit en émettant du rayonnement surtout dans l'infrarouge d'après la loi de Wien. La troposphère est ainsi principalement chauffée par l'absorption, par les gaz à effet de serre et les nuages, du rayonnement infrarouge émis par la surface. Dans l'infrarouge, à part quelques fenêtres atmosphériques à des longueurs d'onde bien précises, seule une petite partie du flux total émis par la surface s'échappe directement vers l'espace. A leur tour, les gaz à effet de serre émettent du rayonnement dans l'infrarouge, à la fois vers l'espace et vers la surface, ce qui refroidit localement l'atmosphère mais réchauffe la surface par effet de serre comme décrit précédemment avec le modèle à une couche [figure~\ref{fig:modun}]. L'atmosphère piège ainsi~$150$~W~m$^{-2}$ par effet de serre, puisque le rayonnement infrarouge sortant est~$240$~W~m$^{-2}$. On ajoute que la stratosphère est également refroidie par émission infrarouge du gaz carbonique, principalement dans la bande d'absorption à~$15$~$\mu$m. Du point de vue de l'atmosphère, émission infrarouge et refroidissement sont donc intimement liés.
%Émission nette par la vapeur d'eau, l'ozone, le CO2 et les autres gaz à effet de serre : Il s'agit du flux énergétique net émis sous forme de rayonnement énergétique (infrarouge) par l'ensemble des molécules de l'atmosphère. L'émission infrarouge est associée à un refroidissement local. Comme le Corps Noir, les molécules émettent un rayonnement pour se refroidir et équilibrer l'énergie absorbée. L'émission n'a lieu que dans les bandes d'absorption (ou d'émission). Il faut donc que la température locale soit celle du Corps Noir émettant à la longueur d'onde de la bande d'émission. Ainsi, plus on descend dans l'atmosphère plus l'émission se fera par les bandes centrées sur de faibles longueurs d'onde. Émission IR et refroidissement atmosphériques sont doncintimement liés. La stratosphère est principalement refroidie par l'émission IR du gaz carbonique. Ce refroidissement est associé à l'émission par la bande située à 15 μm. Dans la haute stratosphère, la bande d'émission de l'ozone à 9.6 μm permet l’émission IR et le refroidissement atmosphérique. Cependant l'ozone absorbe principalement les radiations solaires et ne peut être considérée comme un gaz à effet de serre (dans la stratosphère). La vapeur d'eau émet également dans la stratosphère dans la bande à 8 μm. La troposphère est principalement refroidie par l'émission de la vapeur d'eau dans la bande située à 6.3 micromètres.

\sk
\subsubsection{Autres échanges d'énergie}

\sk
Le bilan net en surface dans l'infrarouge de $65$~W~m$^{-2}$ est une petite différence entre le flux émis par la surface $\sigma \, {T\e{s}}^4$ et celui reçu depuis l'atmosphère. Si le bilan radiatif est bien équilibré au sommet de l'atmosphère, la surface gagne en moyenne de l'énergie et l'atmosphère en perd. En l'absence d'autres mécanismes de transfert d'énergie, cela conduirait à un refroidissement de l'atmosphère, et à une discontinuité de température à la surface entre le sol et l'air. En pratique, ce déséquilibre radiatif est compensé par des flux qui dépendent des mouvements et des changements de phase dans l'atmosphère
\begin{citemize}
\item de chaleur sensible (transport vertical de chaleur par la conduction et les mouvements de convection) 
\item de chaleur latente (évaporation depuis la surface et condensation dans l'atmosphère) 
\end{citemize}
depuis la surface vers l'atmosphère. Du fait que le transfert d'énergie du sol vers l'atmosphère se fait également sous forme d'un flux de chaleur sensible et latente, le sol n'émet donc que~$396$~W~m$^{-2}$ (au lieu de~$495$~W~m$^{-2}$) ce qui équivaut à une température de~$15^{\circ}$C, soit la température moyenne de la surface terrestre effectivement constatée. En l'absence de convection et de changements d'état dans l'atmosphère, la température de la surface et des basses couches atmosphériques serait beaucoup plus élevée. 
%%% les 240 W/m2 qui sortent sont les mêmes que dans la version sans atmosphère.
%%% noter la fenêtre atmosphérique dans l'infrarouge

%\figun{1.0}{0.4}{\figfrancis/bilan_rad_glob}{Schéma des flux moyens échangés entre la surface de la Terre, l'atmosphère, et l'espace: flux radiatifs ondes courtes (jaune) et infrarouge (rouge), et flux sensibles et latents (violet).}{fig:bilanrad}
\figun{1.0}{0.45}{decouverte/meteo_terre/bilanflux00004.png}{Bilan énergétique moyen de la Terre (en W~m$^{-2}$)~: flux échangés entre la surface de la Terre, l'atmosphère et l'espace. On distingue les flux radiatifs ondes courtes (rayonnement visible, en jaune) des flux radiatifs ondes longues (rayonnement infrarouge, en rouge). Noter les flux sensibles et latents qui ne sont pas relatifs au transfert radiatif. Source~: Planton CNRS editions 2011 ; adapté de Trenberth et al. BAMS 2009}{fig:bilflux}

%%% MANQUE UN TOPO SUR LE FORçAGE RADIATIF ????
%%% POUR REBRANCHER SUR LE CHANGEMENT CLIMATIQUE. VOIR PAYAN 10-12.


	\sk \subsection{Variations géographiques}
		\sk
\subsubsection{Influence de la latitude}

\sk
Localement, l'éclairement varie suivant la latitude et la saison, en plus de l'alternance jour/nuit: il est proportionnel à $\cos\theta$ où $\theta$ est l'angle d'incidence avec la surface.
%[figure~\ref{fig:senslat1}]. 
En moyenne annuelle, le maximum d'ensoleillement est donc aux latitudes tropicales, mais il varie au cours de l'année et est même maximal aux pôles pendant l'été local [figure~\ref{fig:senslat2}]~: la durée du jour de 24h fait plus que compenser l'angle d'incidence réduit dû à la latitude élevée (ce qui peut paraître de prime abord contre-intuitif).
%\figside{0.3}{0.1}{\figfrancis/swcoslat.jpg}{Schéma de la relation entre densité de flux du rayonnement incident parallèle et éclairement de la surface suivant l'angle d'incidence.}{fig:senslat1}
\figside{0.65}{0.25}{\figfrancis/swtoaseas}{Cycle saisonnier de l'éclairement dû au rayonnement solaire incident au sommet de l'atmosphère.}{fig:senslat2}
%%%% http://www.energieplus-lesite.be/energieplus/page_16761.htm

\sk
\subsubsection{Rôle des nuages}

\sk
La présence de différents types de nuages est très variable, à la fois géographiquement et dans le temps. Ils ont pourtant une influence très grande sur le bilan radiatif, par deux mécanismes distincts [figure \ref{fig:schemacrf}].
\begin{finger}
\item Effet d'albédo~: les nuages réfléchissent une partie importante du rayonnement solaire incident (par rétro\-diffusion par les gouttes d'eau). Cet effet est d'autant plus fort que le nuage contient d'eau et que les gouttes sont fines. Un nuage très réfléchissant apparaitra sombre vu d'en dessous. Au total, les nuages sont responsables des 2/3 de l'albédo planétaire.
\item Effet de serre~: Les gouttes d'eau (ou la glace) des nuages sont d'excellents absorbants dans l'infrarouge. Un nuage même peu épais absorbe donc très rapidement tout le rayonnement infrarouge provenant des couches plus basses. Il émet lui même vers le haut du rayonnement suivant sa propre température: $\sigma T_N^4$ où $T_N$ est la température au sommet du nuage. Un nuage au sommet élevé (donc froid) aura donc un effet de serre très important.
\end{finger}
Au final, l'effet d'albédo l'emporte pour les nuages bas (type stratus), qui sont typiquement épais (albédo élevé) et dont le sommet est chaud. Au contraire, les fins nuages d'altitude (cirrus) ont un albédo faible mais un sommet très froid donc ont un effet net réchauffant. Pour les nuages de type orageux, qui sont épais avec un sommet froid, les deux effets tendent à se compenser.
\figside{0.6}{0.2}{\figfrancis/schema_crf}{Schema de l'influence des nuages sur le bilan radiatif: effet d'albédo dans le visible (jaune), et absorption et émission dans l'infrarouge (rouge). L'effet de serre vient du rayonnement émis vers l'espace plus faible que celui venant de la surface, qui est absorbé.}{fig:schemacrf}
%% nuages comme les lunettes dans la caméra infrarouge. faire également référence à la vidéo tirée du satellite.


	\sk \subsection{Moyennes annuelles~: cartes}
		\sk
On présente dans cette section des cartes des différents termes du bilan radiatif de la terre, tels qu'observés par satellite depuis l'espace. On observe un bilan moyen sur une année qui est variable en fonction de la position géographique : si l'on a, en moyenne, égalité entre absorption du rayonnement solaire incident et émission de la Terre vers l’espace, ce n’est plus vrai si on considère une région donnée. 
\begin{finger}
\item
Le flux solaire absorbé (figure \ref{fig:swtoa}) montre essentiellement une dépendance en latitude. L'effet de l'ensoleil\-lement au sommet de l'atmosphère, plus fort dans les tropiques, est amplifié par un albédo élevé aux latitudes polaires à cause de la présence de neige et de glace au sol. En plus de ces variations en latitudes, on observe des différences locales dûes à l'albédo des régions nuageuses (zone de convergence intertropicale, bords est des océans) ou du sol (Sahara).
\item
Le flux infrarouge sortant au sommet de l'atmosphère (figure \ref{fig:olr}) a lui aussi une structure en latitude, mais moins marquée que pour les ondes courtes: les hautes latitudes, plus froides, émettent moins de rayonnement. On voit d'autre part nettement le flux sortant plus faible dans les régions humides des tropiques (continents et zone de convergence) où des nuages convectifs d'altitude élevée se forment.
\item
La signature des régions humides est nettement plus faible sur la carte du bilan net au sommet de l'atmosphère (figure \ref{fig:nettoa}); les effets de serre et d'albédo des nuages se compensant en grande partie. On retrouve par contre un bilan moins positif dans les régions où un albédo élevé provient du sol (Sahara) ou de nuages bas (Chili, Californie). D'autre part, on observe un gain net d'énergie dans les tropiques, et une perte dans les hautes latitudes; la distribution du bilan dans le visible qui est plus inégale que celle dans l'infrarouge détermine donc la structure globale. 
\end{finger}

\figside{0.65}{0.25}{\figfrancis/erbe_stoa_ann}{Rayonnement visible absorbé par la Terre, en moyenne annuelle (données ERBE).}{fig:swtoa}
\figside{0.65}{0.25}{\figfrancis/erbe_olr_ann}{Rayonnement infrarouge sortant au sommet de l'atmosphère, en moyenne annuelle.}{fig:olr}
\figside{0.65}{0.25}{\figfrancis/erbe_ntoa_ann}{Flux net absorbé par la Terre (visible  - infrarouge sortant) en moyenne annuelle.}{fig:nettoa}

%• Disparités régionales :
%o Sahara en été boréal : fort albédo + forte perte IR + capacité
%calorifique faible + atmosphère sèche
%o Océan : albédo faible compense la perte radiative IR + forte
%capacité calorifique

\sk
Ces excès et déficit d'énergie locaux doivent, en moyenne, être compensés par des transports d'énergie par les circulations atmosphérique et océanique. Ils fournissent la source d'énergie pour la \voc{dynamique atmosphérique} qui va contribuer à répartir l'énergie des régions excédentaires en énergie vers les régions déficitaires en énergie~[figure~\ref{fig:hadley}].

\figun{0.75}{0.35}{decouverte/cours_meteo/energiedyn.png}{Schéma représentant les latitudes où l'atmosphère est excédentaire ou, au contraire, déficitaire en énergie. La courbe bleue représente l'énergie radiative reçue du Soleil, principalement dans les courtes longueurs d'onde (noté \emph{shortwave} sur la figure, correspond au rayonnement visible et ultraviolet). La courbe rouge représente l'émission par la surface terrestre, principalement dans les longues longueurs d'onde (noté \emph{longwave} sur la figure, correspond au rayonnement infrarouge). Une circulation atmosphérique de grande échelle se met en place entre les régions excédentaires (équatoriales et tropicales) et déficitaires (hautes latitudes).}{fig:hadley}






\end{document}
%%%%%%%%%%%%%%%%%%%%%%%%%
%%%%%%%%%%%%%%%%%%%%%%%%%
