\documentclass[
	a4paper,
	DIV16,
	10pt]{scrartcl}

%%% MOI
\usepackage{scrpage2}
\usepackage{my_latex}
\usepackage{patch_french}
\setcapindent{1em} %%% parce que sinon l'indentation "Figure ... -- texte" est trop grande

%%% BANDEAU
\deftripstyle{Vide}[0pt][.4pt]
{}{}{}
{}{}{}
\deftripstyle{Bandeau}[0pt][.4pt]
{3T054}{}{Universit\'e Pierre et Marie Curie}
{Aymeric Spiga (UPMC/LMD)}{Tous droits réservés}{\pagemark} \pagestyle{scrheadings} \pagestyle{Bandeau}

%%% DEBUT
\begin{document}

\subject{\includegraphics[height=1.5cm]{/home/aymeric/Images/Logo/UPMC_cart-blanc-Q_7504-703-3.png}}
\title{Transfert radiatif}
\subtitle{UE 3T054. Fiches explicatives.}
\author{Aymeric SPIGA}
\date{\email{aymeric.spiga@upmc.fr}}
\publishers{\small Copie et usage interdits sans autorisation explicite de l'auteur.}

\maketitle


\newpage
\section{Le rôle central de l'énergie solaire}
\bk
Comment déterminer les processus dynamiques, physiques, chimiques à l'oeuvre dans l'atmosphère ? Il faut commencer par faire le point sur les sources d'énergie pour l'atmosphère, les océans et la surface. La principale source d'énergie pour l'atmosphère et le système climatique de la Terre est le Soleil\footnote{Ce n'est pas le cas pour les géantes gazeuses Jupiter et Saturne où il existe un flux de chaleur interne significatif en regard du flux d'énergie reçu du Soleil. Ce flux est un reste de la contraction gravitationnelle au cours de la formation de ces géantes gazeuses.}. La figure~\ref{fig:flux} montre que d’autres sources existent mais en quantité réduite : l'énergie reçue par la géothermie, ou par les activités humaines, est~$4$ ordres de grandeur plus faible que la source solaire; celle reçue des étoiles~$8$ ordres de grandeur plus faible. L’énergie solaire est transmise principalement à la Terre au moyen du rayonnement électromagnétique~: on qualifie cette énergie de \voc{radiative}. %L'objet de ce chapitre est de s'intéresser au rayonnement électromagnétique et plus particulièrement au phénomène d'émission thermique.

\figside{0.75}{0.18}{decouverte/cours_meteo/fluxenergsurf.png}{Ordres de grandeur des flux énergétiques reçus à la surface de la Terre. Source~:~P.~von Balmoos \emph{in} Le Climat à Découvert, CNRS éditions, 2011}{fig:flux}



\newpage
\section{Spectre électromagnétique}
\sk
Les échanges d'\voc{énergie radiative} se font à distance par le biais du \voc{rayonnement électromagnétique}. Le rayonnement électromagnétique est composé d'une superposition d'ondes monochromatiques de longueurs d'onde~$\lambda$ se propageant à la vitesse de la lumière~$c$ (dans le vide~$c=3 \times 10^8$~m~s$^{-1}$). Le rayonnement électromagnétique parcourt la distance Terre-Soleil en $8$~minutes; à l'échelle des processus atmosphériques terrestres, la propagation des ondes électromagnétiques est si rapide qu'elle peut être considérée en première approximation comme immédiate. 

\sk
Les ondes composant le rayonnement électromagnétique peuvent être caractérisées indifféremment par leur \voc{longueur d'onde}~$\lambda$, leur \voc{fréquence}~$\nu = c / \lambda$ ou leur \voc{nombre d'onde}\footnote{Le nombre d'onde est souvent exprimé en cm$^{-1}$. Pour obtenir~$\overline{\nu}$ dans cette unité à partir de~$\lambda$ en microns, on utilise~$\overline{\nu} = 10^{4} / \lambda$.}~$\overline{\nu} = 1 / \lambda$. L'ensemble de ces ondes constitue le \voc{spectre} électromagnétique. Selon le principe de De Broglie, à chaque onde électromagnétique de fréquence~$\nu$ est associée une particule sans masse nommée \voc{photon} dont l'énergie est~$h \, \nu$ où $h = 6.63 \times 10^{-34}$~J~s est appelée la constante de Planck. Cette énergie est souvent exprimée en électron-volts eV ($1$~eV~$= 1.6 \times 10^{-19}$~J~s).

\sk
Le rayonnement visible occupe une bande très étroite du spectre aux longueurs d'ondes comprises entre 0.4 et 0.76~$\mu$m [figure~\ref{fig:spectrum}]. Lorsque l'on considère des longueurs d'ondes plus courtes (c'est-à-dire des fréquences plus élevées) que le rayonnement visible, on passe dans le domaine du rayonnement ultraviolet, puis celui des rayons X et gamma~$\gamma$. Lorsque l'on considère des longueurs d'ondes plus grandes (c'est-à-dire des fréquences plus faibles) que le rayonnement visible, on passe dans le domaine du rayonnement infrarouge, puis celui des micro-ondes et des ondes radio. Les photons les plus énergétiques correspondent aux rayons X; les moins énergétiques aux ondes radio.

\figsup{1}{0.1}{\figpayan/LP211_Chap2_Page_09_Image_0001.png}{\figpayan/LP211_Chap2_Page_09_Image_0002.png}{Classification du rayonnement électromagnétique en fonction de la longueur d'onde. On rappelle que 1~$\mu$m (micron) correspond à $10^{-6}$~m et 1~nm (nanomètre) correspond à $10^{-9}$~m.}{fig:spectrum}


\newpage
\section{Rayonnement thermique et corps noir}
\sk
Le Soleil qui se situe à une distance considérable dans le vide spatial nous procure une sensation de chaleur. De même, placer sa main sur le côté d'un radiateur en fonctionnement sans le toucher procure une sensation de chaleur instantanée qui ne peut être attribuée à un transfert convectif entre le radiateur et la main. Cet échange de chaleur est attribué au contraire à l'émission d'ondes électromagnétiques par la matière du fait de sa température; on parle d'émission de \voc{rayonnement thermique}. Tous les corps émettent du rayonnement thermique. La transmission de cette énergie entre une source et une cible ne nécessite pas la présence d'un milieu intermédiaire matériel. 
%Le but de cette section est d'en étudier les principales propriétés.

\sk
On appelle \voc{corps noir} un objet dont la surface est idéale et satisfait les trois conditions suivantes~:
\begin{description}
\item[émetteur parfait] un corps noir rayonne plus d’énergie radiative à chaque température et pour chaque longueur d’onde que n'importe quelle autre surface,
\item[absorbant parfait] un corps noir absorbe complètement le rayonnement incident selon toutes les directions de l'espace et toutes les longueurs d'onde,
\item[source lambertienne] un corps noir émet du rayonnement de façon isotrope
\end{description}

\sk
Un corps noir est à l'équilibre thermodynamique avec son environnement. On peut montrer qu'un tel corps émet du rayonnement qui dépend seulement de sa température et non de sa nature. La définition du corps noir, et les développements théoriques qui l'accompagnent, sont partis du constat, fait notamment par les céramistes, qu'un objet placé dans un four à haute température devient rouge en même temps que les parois du four quelle que soit sa taille, sa forme ou le matériau qui le compose. Un exemple de source utilisée pour étudier expérimentalement le modèle du corps noir consiste à construire une enceinte chauffée, totalement hermétique, et y percer un trou pour y mesurer le flux énergétique émis [figure~\ref{fig:four}]

\figside{0.35}{0.15}{\figwallace/Radiation/radiation_Page_10_Image_0001.png}{L'énergie entrant par une petite fente dans une enceinte subit des réflexions sur la paroi jusqu'à ce qu'elle soit absorbée. L'ouverture dans la paroi d'une enceinte chauffée apparaît comme une source de type corps noir. Un absorbant presque parfait est aussi un émetteur presque parfait. Ce type de four a été employé au début du XXe siècle pour évaluer expérimentalement les prédictions théoriques de Planck. Source~: Wallace and Hobbs, Atmospheric Science, 2006.}{fig:four}



\newpage
\section{Loi de Planck}
\sk
L'émission de rayonnement par le corps noir est décrite par une luminance énergétique spectrale~$L_{\lambda}$, notée $B_\lambda$ dans ce qui suit\footnote{Correspond au nom anglais \emph{blackbody}}. La loi de variation de~$B_\lambda$ selon la température~$T$ est donnée par la \voc{loi de Planck}\footnote{La luminance spectrale $B_\nu$ est déterminée d'une façon similaire. La démonstration de la loi de Planck fait appel à des notions de quantification d'énergie et de thermodynamique statistique qui sont hors programme dans le cadre de ce cours.} $$ B_\lambda(T) = \frac{C_1 \, \lambda^{-5}}{\pi \, \left( e^{ C_2 / \lambda T}-1\right) } $$ où $C_1$ et $C_2$ sont des constantes. Comme le rayonnement du corps noir est isotrope, l'émittance spectrale du corps noir, obtenue par intégration sur toutes les directions de l'espace, vaut $ M_\lambda(T) = \pi \, B_\lambda(T) $. 

%\figun{0.5}{0.25}{\figfrancis/WH_BBrad}{Courbes de luminance spectrale d'un corps noir pour différentes températures. La courbe en pointillés indique la position du maximum en fonction de $T$.}{fig:BBrad} 
\figside{0.5}{0.25}{\figwallace/Radiation/radiation_Page_11_Image_0001.png}{Courbes de luminance spectrale d'un corps noir pour différentes températures. La courbe en pointillés indique la position du maximum en fonction de $T$. Source~: Wallace and Hobbs, Atmospheric Science, 2006.}{fig:BBrad} 

\sk
Les variations de la fonction~$B_\lambda$ sont illustrées sur la figure~\ref{fig:BBrad}. L'émission de rayonnement par le corps noir ne dépend que de la longueur d'onde~$\lambda$ et de la température~$T$ du corps. A une température donnée, le rayonnement émis est parfaitement déterminé pour chaque longueur d'onde; dans un domaine spectral particulier, le rayonnement émis ne dépend que de la température du corps noir.



\newpage
\section{Loi de Wien du corps noir}
\sk
On observe sur la figure \ref{fig:BBrad} que, lorsque~$T$ augmente, la maximum de la luminance spectrale~$B_\lambda$, appelé \voc{maximum d'émission}, se décale vers les longueurs d'onde courtes, c'est-à-dire correspond à des photons de plus en plus énergétiques. La loi exacte, appelée \voc{loi de déplacement de Wien}, s'obtient en dérivant $B_\lambda$ par rapport à $\lambda$, ce qui permet d'obtenir $$ \boxed{ \lambda\e{max} \, T = 2898 \quad (\mu\textrm{m~K}) } $$ où $\lambda_{max}$ est la longueur d'onde du maximum de luminance spectrale~$B_\lambda$. La longueur d'onde du maximum d'émission~$\lambda\e{max}$ est ainsi inversement proportionnelle à la température du corps émetteur. Une formulation alternative est que $\nu\e{max}$ est proportionelle à $T$.

\figun{0.6}{0.45}{/home/aymeric/Big_Data/BOOKS/pierrehumbert_pics/9780521865562c03_fig001.jpg}{Source~: R. Pierrehumbert, Principles of Planetary Climates, CUP, 2010.}{wvl} 



\newpage
\section{Loi de Stefan-Boltzmann du corps noir}
\sk
La \voc{loi de Stefan-Boltzmann}\footnote{Joseph Stefan met expérimentalement en évidence en 1879 la dépendance de l'émittance en puissance quatrième de la température. Ludwig Boltzmann, à qui l'on doit également des résultats fondamentaux sur l'entropie et l'atomisme, prouve en 1884 le résultat par des arguments théoriques.} donne la valeur de l'intégrale sur toutes les longueurs d'ondes et dans tout l'espace\footnote{On entend par là toutes les directions du demi-espace extérieur au corps considéré.} de la courbe du corps noir, décrite par la loi de Planck et illustrée par les figures \ref{fig:BBrad} et \ref{fig:BBmax}. Cette loi donne donc l'expression d'une densité de flux énergétique~$F$ ou plus spécifiquement, puisque le corps noir est une source de rayonnement, d'une émittance totale~$M$. Cette dernière s'obtient tout d'abord avec une intégration par rapport à~$\lambda$ de la luminance énergétique spectrale~$B_\lambda$ donnée par la loi de Planck, afin d'obtenir la luminance énergétique~$B$. On déduit ensuite l'émittance totale~$M$ en intégrant selon toutes les directions de l'espace; comme le rayonnement du corps noir est isotrope, $M$ s'obtient à partir de~$B$ simplement en multipliant par~$\pi$. La loi de Stefan-Boltzmann établit que le flux net surfacique~$M$ émis par un corps noir ne dépend que de sa température par une dépendance type loi de puissance $$ \boxed{ M\e{corps noir} = \sigma \, T^4 } $$ avec~$\sigma=5.67 \times 10^{-8} \textrm{~W~m}^{-2}\textrm{~K}^{-4}$ appelée constante de Stefan-Boltzmann. La loi de Stefan-Boltzmann, comme la loi de Planck dont elle dérive, stipule que l'émittance~$M$ d'un corps pouvant être considéré en bonne approximation comme un corps noir ne dépend que de sa température et non de sa nature. Cette loi indique par ailleurs que l'émittance~$M$ augmente très rapidement avec la température -- de par la puissance quatrième impliquée.


\newpage
\section{Corps gris et émissivité}
\sk
Le corps noir est un modèle idéal d'absorbant qu'en pratique on ne rencontre pas dans la nature. Par exemple, le charbon noir est un absorbant parfait, mais seulement dans les longueurs d'onde visible. La plupart des objets ressemblent néanmoins au corps noir, au moins à certaines températures et pour certaines longueurs d'onde considérées en pratique. Dans le cas d'un corps qui n'est pas un absorbant parfait, on parle d'un \voc{corps gris}. A température égale, un corps gris n'émet pas autant qu'un corps noir dans les mêmes conditions. Pour évaluer l'énergie émise par un corps gris par comparaison à celle qu'émettrait le corps noir dans les mêmes conditions, on définit un coefficient appelé \voc{émissivité} $\epsilon_\lambda$ compris entre~$0$ et~$1$ et égal au rapport entre la luminance spectrale du corps~$L_\lambda$ et celle du corps noir~$B_\lambda$ $$ \epsilon_\lambda=\frac{L_\lambda}{B_\lambda(T)} $$ En toute généralité, l'émissivité~$\epsilon_{\lambda}$ d'une surface à une longueur d'onde~$\lambda$ dépend de ses propriétés physico-chimiques, de sa température et de la direction d'émission\footnote{Par exemple, les métaux, matériaux conducteurs de l'électricité, ont une émissivité faible (sauf dans les directions rasantes) qui croît lentement avec la température et décroît avec la longueur d'onde ; au contraire, les diélectriques, matériaux isolant de l'électricité, ont une émissivité élevée qui augmente avec la longueur d'onde et se révèlent lambertiens sauf pour les directions rasantes où l'émissivité décroît significativement.}.

\sk
On peut également définir une émissivité totale intégrée~$\epsilon$ qui permet d'exprimer l'émittance~$M$ d'un corps gris $$ \boxed{\SB} $$ Des valeurs de l'émissivité totale~$\epsilon$ pour certains matériaux sont données dans le tableau~\ref{tab:emiss}~: l'eau, la neige, les roches basaltiques ont des émissivités proches de~$1$ et peuvent être considérées comme des corps noirs en bonne approximation. 

\begin{table}\label{tab:emiss}
\begin{center}
\begin{tabular}{|c|c|}
\hline
Matériau & Emissivité~$\epsilon$ \\
\hline
Aluminium & 0.02 \\
Cuivre poli & 0.03 \\
Nuages type cirrus & 0.10 à 0.90 \\
Nuages type cumulus & 0.25 à 0.99 \\
Cuivre oxydé & 0.5 \\
Béton & 0.7 à 0.9 \\
Carbone & 0.8 \\
Lave (volcan actif) & 0.8 \\
Neige âgée & 0.8 \\
Ville & 0.85 \\
Désert & 0.85 à 0.9 \\
Peinture blanche & 0.87 \\
Brique rouge & 0.9 \\
Herbe & 0.9 à 0.95 \\
Eau & 0.92 à 0.97 \\
Peinture noire & 0.94 \\
Forêt & 0.95 \\
Suie & 0.95 \\
Neige fraîche & 0.99 \\
\hline
\end{tabular}
\caption{\emph{Quelques valeurs usuelles d'émissivité à la température ambiante (pour un rayonnement infrarouge). Source~: Hecht, Physique, 1999 -- avec quelques ajouts d'après site CNES}}
\end{center}
\end{table}


\newpage
\section{Domaine de l'énergie solaire}
\sk
Le Soleil peut être considéré en bonne approximation comme un corps noir car il absorbe tout le rayonnement incident. Sa \ofg{couleur} est dûe à du rayonnement émis et, plus précisément, correspond aux longueurs d'onde où le maximum de rayonnement est émis. D'après la loi de Wien, le Soleil, dont l'enveloppe externe a une température autour de~$6000$~K, a donc un maximum d'émission situé dans le visible à $\lambda\e{max} = 0.5 \mu$m, proche du maximum de sensibilité de l'oeil humain [figure~\ref{fig:BBmax} haut]. Au contraire, la surface terrestre, dont la température typique est d'environ~$288$~K, voit son maximum d'émission situé dans l'infrarouge vers 10~$\mu$m, alors que le rayonnement émis dans les longueurs d'ondes visible est négligeable [figure~\ref{fig:BBmax} bas]. Un raccourci usuel est donc de dire que \ofg{la Terre émet du rayonnement (thermique) dans l'infrarouge alors que le Soleil émet dans le visible}. En toute rigueur, cette affirmation ne parle que du voisinage du maximum d'émission, où la contribution au flux intégré selon toutes les longueurs d'onde est la plus significative. Il est ainsi plus exact de dire que, dans l'atmosphère, la région du spectre où~$\lambda$ est inférieure à environ 4~$\mu$m est dominée par le rayonnement d'origine solaire, alors qu'au-delà, le rayonnement est surtout d'origine terrestre. Il n’y a pratiquement pas de recouvrement entre la partie utile du spectre du rayonnement solaire et celui d’un corps de température ambiante; ce fait est d'une grande importance pour les phénomènes de type effet de serre, qui sont abordés plus loin dans ce cours. On désigne ainsi souvent le rayonnement d'origine solaire par le terme \voc{ondes courtes} et le rayonnement d'origine terrestre par le terme \voc{ondes longues}.

\figsup{0.65}{0.2}{decouverte/cours_meteo/6000K.jpg}{decouverte/cours_meteo/earth.jpg}{Courbes de luminance spectrale d'un corps noir pour différentes températures correspondant notamment au Soleil (haut) et à la Terre (bas). La quantité représentée ici est l'émittance spectrale~$M_\lambda = \pi \, B_\lambda$. Noter la différence d'indexation de l'abscisse et l'ordonnée sur les deux schémas. Le rayonnement thermique émis par la Terre est plusieurs ordres de grandeur moins énergétique que celui émis par le Soleil et le maximum d'émission se trouve à des longueurs d'onde plus grandes (infrarouge pour la Terre au lieu de visible pour le Soleil). Source : \url{http://hyperphysics.phy-astr.gsu.edu/hbase/bbrc.html}.}{fig:BBmax}


\newpage
\section{Constante solaire}
\sk
La distance Soleil-Terre est beaucoup plus grande que les rayons de la Terre et du Soleil. Ainsi, d'une part, le rayonnement solaire arrive au niveau de l'orbite terrestre en faisceaux pratiquement parallèles. D'autre part, la luminance en différents points de la Terre ne varie pas. On peut par conséquent définir une valeur moyenne de la densité de flux énergétique du rayonnement solaire au niveau de l'orbite terrestre, reçue par le système surface~+~atmosphère. Elle est désignée par le terme de \voc{constante solaire} notée~$\mathcal{F}\e{s}$. Les mesures indiquent que
\[ \mathcal{F}\e{s} = 1368 \text{~W~m}^{-2} \qquad \text{pour la Terre} \]

\sk
La constante solaire est une valeur instantanée côté jour~: le rayonnement solaire reçu au sommet de l'atmosphère en un point donné de l'orbite varie en fonction de l'heure de la journée et de la saison considérée (c'est-à-dire la position de la Terre au cours de sa révolution annuelle autour du Soleil)\footnote{En réalité, la constante solaire~$\mathcal{F}\e{s}$ varie elle-même d'environ~$3$~W~m$^{-2}$ en fonction des saisons à cause de l'excentricité de l'orbite terrestre, qui n'est pas exactement circulaire. De plus, elle peut varier évidemment en fonction des cycles solaires, néanmoins sans influence majeure sur la température des basses couches atmosphériques (troposphère et stratosphère).}. On peut donc définir un \voc{éclairement solaire moyen} noté~$\mathcal{F}\e{s}'$ reçu par la Terre qui intègre les effets diurnes et saisonniers. Autrement dit, $\mathcal{F}\e{s}$~est l'éclairement instantané reçu par un satellite en orbite autour de la Terre~; $\mathcal{F}\e{s}'$ est la valeur que l'on obtiendrait si l'on faisait la moyenne d'un grand nombre de mesures instantanées du satellite à diverses heures et saisons. 

\figside{0.5}{0.2}{decouverte/cours_dyn/incoming.png}{Energie reçue du Soleil par le système Terre. Source~: McBride and Gilmour, \emph{An Introduction to the Solar System}, CUP 2004.}{fig:eqrad}

\sk
On admet ici que~$\mathcal{F}\e{s}'$ peut être calculé en considérant que le flux total reçu du Soleil l'est à travers un disque de rayon le rayon~$R$ de la Terre (il s'agit de l'ombre projetée de la planète, voir Figure~\ref{fig:eqrad}). A cause de l'incidence parallèle, le flux énergétique intercepté par la Terre vaut donc~$\Phi = \pi \, R^2 \, \mathcal{F}\e{s}$. L'éclairement moyen à la surface de la Terre est alors $$\mathcal{F}\e{s}' = \frac{\Phi}{4 \, \pi \, R^2}$$ le dénominateur étant l'aire de la surface complète de la Terre. On obtient ainsi
\[ \boxed{ \mathcal{F}\e{s}' = \frac{\mathcal{F}\e{s}}{4} } \]

%\sk
La valeur de la constante solaire peut s'obtenir par le calcul. Le soleil est considéré en bonne approximation comme un corps noir de température~$T_{\sun} = 5780$~K. D'après la loi de Stefan-Boltzmann, son émittance est $M = \sigma \, T_{\sun}^4$ donc le flux énergétique~$\Phi_{\sun}$ émis par le Soleil de rayon~$R_{\sun} = 7 \times 10^5$~km est~$\Phi_{\sun} = 4 \, \pi \, R_{\sun}^2 \, \sigma \, T_{\sun}^4$. Ce flux énergétique est rayonné dans tout l'espace~: à une distance~$d$ du soleil il est réparti sur une sphère de centre le soleil et de rayon~$d$, donc de surface~$4 \, \pi \, d^2$. A cette distance, l'éclairement~$\mathcal{F}$, c'est-à-dire la densité de flux énergétique reçue en W~m$^{-2}$, s'écrit donc
\[ \mathcal{F} = \frac{\Phi_{\sun}}{4 \, \pi \, d^2} = \frac{4 \, \pi \, R_{\sun}^2 \, \sigma \, T_{\sun}^4}{4 \, \pi \, d^2} = \sigma \, T_{\sun}^4 \, \left( \frac{R_{\sun}}{d} \right)^2 \]
Si l'on prend~$d$ égal à la distance Terre-Soleil, $\mathcal{F}$ définit ainsi la constante solaire~$\mathcal{F}\e{s}$.
%\[ \mathcal{F}\e{s} = \frac{{\mathcal{F}\e{s}}^{\text{Terre}}}{d\e{soleil}^2} \]

%Variation de la constante solaire : Bien que l’intensité du soleil ait subit des variations depuis la formation de la Terre, on peut s’attendre à ce qu’elle soit stable sur une période de 1000 ans. On mesure mal la constante solaire, mais les mesures récentes, même avec leurs incertitudes, semblent indiquer que le soleil ne peut pas expliquer le réchauffement récent. Notons toutefois que les simulations actuelles ne tiennent pas compte des fluctuations possibles du rayonnement solaire (négligeable a priori).
%%%% pas sûr du dernier point.


\newpage 
\section{Albédo} 
\sk
En sciences de l'atmosphère, les coefficients de réflexion~$\rho$ et~$\rho_{\lambda}$ sont souvent désignés sous le nom respectivement d'\voc{albédo} noté~$A$ et d'albédo spectral noté~$A_{\lambda}$. Plus la surface réfléchit une grande partie du rayonnement électromagnétique incident, plus l'albédo est proche de~$1$. L'albédo spectral~$A_{\lambda}$ peut varier significativement en fonction de la longueur d'onde : voir l'exemple de la neige fraîche donné ci-dessus. 

\sk
De par la diversité des surfaces terrestres, et de la variabilité de la couverture nuageuse, les valeurs de l'albédo~$A$ varient fortement d'un point à l'autre du globe terrestre~: il est élevé pour de la neige fraîche et faible pour de la végétation et des roches sombres [table~\ref{tab:albedo}]. L'albédo de l'océan est faible, particulièrement pour des angles d'incidence rasants -- il dépend ainsi beaucoup de la distribution des vagues. 

\begin{table}\label{tab:albedo}
\begin{center}
\begin{tabular}{|c|c|c|c|}
\hline
Type & albédo~$A$ & Type & albédo~$A$ \\
\hline
Surface de lac & 0.02 à 0.04 & Surface de la mer & 0.05 à 0.15 \\
Asphalte & 0.07 & Mer calme (soleil au zenith) & 0.10 \\
Forêt équatoriale & 0.10 & Roches sombres, humus & 0.10 à 0.15 \\
Ville & 0.10 à 0.30 & Forêt de conifères & 0.12 \\
Cultures & 0.15 à 0.25 & Végétation basse, verte & 0.17 \\
Béton & 0.20 & Sable mouillé & 0.25 \\
Végétation sèche & 0.25 & Sable léger et sec & 0.25 à 0.45 \\
Forêt avec neige au sol & 0.25 & Glace & 0.30 à 0.40 \\
Neige tassée & 0.40 à 0.70 & Sommet de certains nuages & 0.70 \\
Neige fraîche & 0.75 à 0.95 & & \\
\hline
\end{tabular}
\caption{\emph{Quelques valeurs usuelles d'albédo (rayonnement visible). D'après mesures missions NASA et ESA.}}
\end{center}
\end{table}

\sk
L'\voc{albédo planétaire} est noté~$A\e{b}$ et défini comme la fraction moyenne de l'éclairement~$E$ au sommet de l'atmosphère (noté également~$\mathcal{F}\e{s}'$) qui est réfléchie vers l'espace~: il comprend donc la contribution des surfaces continentales, de l'océan et de l'atmosphère. Il vaut~$0.31$ pour la planète Terre~: une partie significative du rayonnement reçu du Soleil par la Terre est réfléchie vers l'espace\footnote{L'albédo planétaire est par exemple encore plus élevé sur Vénus ($0.75$) à cause de la couverture nuageuse permanente et très réfléchissante de cette planète.}. Ainsi le système Terre reçoit une densité de flux énergétique moyenne~$F\e{reçu}$ en W~m$^{-2}$ telle que
\[ F\e{reçu} = (1-A\e{b}) \, \mathcal{F}\e{s}' \] 
donc un flux énergétique~$\Phi\e{reçu}$ (en W) qui s'exprime
\[ \Phi\e{reçu} = \pi \, R^2 \, (1-A\e{b}) \, \mathcal{F}\e{s} \]
%L'albédo de Bond~ désigne l'albédo intégré sur toutes les longueurs d'onde et tous les angles d'incidence.

\sk
La valeur de~$30\%$ de l'albédo planétaire sur Terre est en fait majoritairement dû à l'atmosphère~:  seuls 4\% de l'énergie solaire incidente sont réfléchis par la surface terrestre comme indiqué sur la figure~\ref{fig:diffsep}. L'énergie réfléchie par l'atmosphère vers l'espace, responsable de plus de~$85\%$ de l'albedo planétaire, est diffusée par les molécules ou par des particules en suspension, gouttelettes nuageuses, gouttes de pluie ou aérosols.

\figside{0.4}{0.15}{\figpayan/LP211_Chap2_Page_27_Image_0001.png}{L'énergie solaire incidente est réfléchie vers l'espace par la surface et l'atmosphère d'une planète. La figure montre les différentes contributions à l'albédo planétaire total.}{fig:diffsep}


\newpage
\section{Bilan simple : température équivalente}
\sk
Nous pouvons exprimer le rayonnement reçu du Soleil par la Terre par une densité de flux énergétique moyenne~$F\e{reçu}$ en W~m$^{-2}$ ou un flux énergétique~$\Phi\e{reçu}$ (en W)
\[ 
F\e{reçu} = (1-A\e{b}) \, \mathcal{F}\e{s}' 
\qquad \qquad
\Phi\e{reçu} = \pi \, R^2 \, (1-A\e{b}) \, \mathcal{F}\e{s}
\] 
La partie du rayonnement reçue du soleil qui est réfléchie vers l'espace est prise en compte via l'albédo planétaire noté~$A\e{b}$. On rappelle par ailleurs que~$\mathcal{F}\e{s}' = \mathcal{F}\e{s} / 4$ où $\mathcal{F}\e{s}$ est la constante solaire.


\sk
Par ailleurs, le système Terre émet également du rayonnement principalement dans les longueurs d'onde infrarouge [figure \ref{fig:eqrad2}]. 
Cette quantité de rayonnement émise au sommet de l'atmosphère radiative est notée $OLR$ pour \emph{Outgoing Longwave Radiation} en anglais.
A l'équilibre, la planète Terre doit émettre vers l'espace autant d'énergie qu'elle en reçoit du Soleil, donc
on obtient la relation générale appelée \emph{TOA} pour \emph{Top-Of-Atmosphere} en anglais, correspondant
au bilan radiatif au sommet de l'atmosphère
\[ \boxed{\TOA} \] 
La principale difficulté qui sous-tend les divers modèles pouvant être proposés réside dans l'expression du terme~$OLR$.



\sk
A l'équilibre, la planète Terre doit émettre vers l'espace autant d'énergie qu'elle en reçoit du Soleil. Ceci peut s'exprimer par unité de surface
\[ \boxed{ F\e{reçu} = F\e{émis} } \]
ou, pour un résultat similaire, en considérant l'intégralité de la surface planétaire
\[ \Phi\e{reçu} = \Phi\e{émis} \]
ce qui permet de déterminer la température équivalente en fonction des paramètres planétaires
\[ \boxed{
T\e{eq} = \bigg[ \frac{\mathcal{F}\e{s}'\,(1-A\e{b})}{\sigma} \bigg]^{\frac{1}{4}}
} \]



\newpage
\section{Applications de la température équivalente}
Le calcul présenté ici porte le nom d'\voc{équilibre radiatif simple}. On y néglige les effets de l'atmosphère (sauf l'albédo) puisqu'on suppose que le rayonnement atteint la surface, ou est rayonné vers l'espace, sans être absorbé par l'atmosphère. La température équivalente est ainsi la température qu'aurait la Terre si l'on négligeait tout autre influence atmosphérique que la réflexion du rayonnement solaire incident. Les valeurs de $T\e{eq}$ pour quelques planètes telluriques sont données dans la table \ref{tab:planets}. On note que la température équivalente de Vénus est plus faible que celle de la Terre, bien qu'elle soit plus proche du Soleil, à cause de son fort albédo~; la formule indique bien que, plus le pouvoir réfléchissant d'une planète est grand, plus la température de sa surface est froide. Par ailleurs, comme indiqué par les calculs du tableau~\ref{tab:planets}, on remarque que la température équivalente, si elle peut renseigner sur le bilan énergétique simple de la planète, ne représente pas correctement la valeur de la température de surface. Par exemple, la température équivalente pour la Terre est~$T\e{eq} = 255 K = -18^{\circ}$C, bien trop faible par rapport à la température de surface effectivement mesurée. Il faut donc avoir recours à un modèle plus élaboré.

\begin{table}\label{tab:planets} \begin{center} \begin{tabular}{lccccc} &{\bf Mercure} &{\bf V\'enus}&{\bf Terre}&{\bf Mars} &{\bf Titan} \\ \hline $d\e{soleil}$ (UA) & 0.39 & 0.72 & 1 & 1.5 & 9.5 \\ $\mathcal{F}\e{s}\,$(W~m$^{-2}$) & $8994$ & $2614$ & $1367$ & $589$ & $15$ \\ $A\e{b}$ & $0.06$ & $0.75$ & $0.31$ & $0.25$ & $0.2$ \\ \textcolor{blue}{$T\e{surface}$ (K)} & \textcolor{blue}{$100/700$~K} & \textcolor{blue}{$730$} & \textcolor{blue}{$288$} & \textcolor{blue}{$220$} & \textcolor{blue}{$95$} \\ \hline $T\e{eq}$~(K) & $439$ & $232$ & $254$ & $210$ & $86$\\ \end{tabular} \caption{\emph{Comparaison des facteurs influençant la température équivalente du corps noir pour différentes planètes du système solaire.}} \end{center} \end{table}
%    Mercure & 0.39 & 8994 & 0.06 & 439 \\
%    Vénus & 0.72 & 2639 & 0.78 & 225 \\
%    Terre & 1 & 1368 & 0.30 & 255 \\
%    Mars & 1.52 & 592 & 0.17 & 216 \\


\newpage
\section{Absorption, réflexion, transmission}
\sk
Tout rayonnement se propageant dans un milieu matériel subit trois phénomènes~: réflexion, absorption, transmission. Autrement dit, tout corps cible irradié par une source voit le flux énergétique incident spectral~$\Phi_{\lambda}$ se répartir selon trois termes
\begin{citemize}
\item une partie~$\Phi_{\lambda}^r$ du flux incident est réfléchie ou diffusée;
\item une partie~$\Phi_{\lambda}^t$ du flux incident traverse le corps sans interactions;
\item une partie~$\Phi_{\lambda}^a$ du flux incident est absorbée, c'est-à-dire transformée en énergie interne.
\end{citemize}
Afin de définir les contributions respectives de ces trois phénomènes, on définit des coefficients spectraux de \voc{réflexion}~$\rho_{\lambda}$, de \voc{transmission}~$\tau_{\lambda}$, d'\voc{absorption}~$\alpha_{\lambda}$ compris entre~$0$ et~$1$
$$\Phi_{\lambda}^r = \rho_{\lambda} \, \Phi_{\lambda} \qquad\qquad \Phi_{\lambda}^t = \tau_{\lambda} \, \Phi_{\lambda} \qquad\qquad \Phi_{\lambda}^a = \alpha_{\lambda} \, \Phi_{\lambda}$$
Ces coefficients sont également appelés \voc{réflectivité}~$\rho_{\lambda}$, \voc{transmittivité}~$\tau_{\lambda}$, \voc{absorptivité}~$\alpha_{\lambda}$. Ils dépendent de la longueur d'onde~$\lambda$ du rayonnement incident, de l'angle d'incidence et des propriétés physiques et chimiques du corps récepteur (par exemple, température, composition). Lorsque le coefficient de réflexion~$\rho_{\lambda}$ ne dépend pas de l'angle d'incidence\footnote{L'énergie incidente à une surface pénètre dans celle-ci et est réfléchie aléatoirement à l'intérieur de l'objet par de microscopiques in-homogénéités du matériau. Au cours de ces multiples réflexions une partie de l'énergie incidente ressort de l'objet suivant une direction aléatoire. Bien souvent les réflexions multiples dans le matériau ne subissent aucune contrainte particulière, l'énergie est donc réfléchie de façon uniforme et isotrope par la surface. Le flux réfléchi est alors uniquement fonction de la quantité d'énergie incidente tombant sur la surface, qui s'exprime souvent simplement comme un cosinus de l'angle entre la normale à la surface et la direction de la source.}, on parle de l'objet cible comme d'un \voc{réflecteur lambertien}. 

%\sk
%\subsection{Rappel sur la distinction entre grandeurs intégrées et spectrales}
%\sk
%Dans le chapitre précédent, les grandeurs caractéristiques~$\Phi$,~$F$ ou~$L$, ainsi que les coefficients d'absorption~$\rho$, de transmission~$\tau$, de réflexion~$\rho$ (également appelé albédo~$A$), ont été décrits 
%\begin{citemize}
%\item soit d'une façon intégrée selon toutes les longueurs d'onde (par exemple l'émittance~$M$ dans la loi de Stefan-Boltzmann) ;
%\item soit en prenant en compte la dépendance spectrale, c'est-à-dire en considérant un petit intervalle~$\dd \lambda$ autour d'une longueur d'onde~$\lambda$ donnée (par exemple la luminance énergétique spectrale~$B_{\lambda}$ dans la loi de Planck). 
%\end{citemize}
%Dans le premier cas, on emploie simplement les symboles décrivant les grandeurs (exemple, luminance~$L$). Dans le second cas, on ajoute $\lambda$ en indice de ces symboles (exemple, luminance spectrale~$L_\lambda$). Ce qui est clair pour les variables l'est beaucoup moins dans le vocabulaire couramment utilisé, y compris dans certains ouvrages.

\sk
La loi de Kirchhoff traduit la conservation du flux incident~$\Phi_{\lambda} = \Phi_{\lambda}^r + \Phi_{\lambda}^t + \Phi_{\lambda}^a$ par une relation entre les coefficients spectraux de réflexion, transmission et absorption $$ \boxed{ \rho_{\lambda} + \tau_{\lambda} + \alpha_{\lambda} = 1 } $$ On peut également considérer le flux énergétique incident~$\Phi$ intégré selon toutes les longueurs d'onde. Les coefficients de réflexion~$\rho$, de transmission~$\tau$, d'absorption~$\alpha$ peuvent alors être définis par $$\Phi^r = \rho \, \Phi \qquad\qquad \Phi^t = \tau \, \Phi \qquad\qquad \Phi^a = \alpha \, \Phi $$ et la loi de Kirchhoff s'écrit alors $$ \boxed{ \rho + \tau + \alpha = 1 } $$ On note par ailleurs que les définitions des coefficients impliquent que $\rho \ne \int_{\lambda} \rho_\lambda \, \dd\lambda$


\newpage
\section{Absorption dans l'atmosphère}
\sk
Les molécules de l'atmosphère absorbent donc le rayonnement à diverses longueurs d'onde. En conséquence, on comprend que les coefficients d'absorption des gaz qui composent l'atmosphère sont extrêmement variables en fonction de~$\lambda$ et présentent une structure très complexe. Un domaine limité de longueurs d'onde contigues où une certaine espèce atmosphérique est très absorbante est appelé \voc{bande d'absorption}. Certaines espèces possèdent des bandes d'absorption dans les longueurs d'onde visible, comme l'ozone~O$_3$, d'autres dans les longueurs d'onde infrarouge, comme les gaz à effet de serre CO$_2$ et H$_2$O [Figure~\ref{fig:atmspectrum} et table~\ref{tab:abs}]. Un domaine limité de longueurs d'onde contigues où les espèces principales qui composent une atmosphère ne sont pas (trop) absorbantes est appelée \voc{fenêtre atmosphérique}, car alors le coefficient de transmission atmosphérique est proche de~$1$.

\small
\begin{table}\label{tab:abs}
\begin{center}
\begin{tabular}{|c|c|}
\hline
Molécules & Principales bandes d'absorption (en $\mu$m) \\
\hline
O$_3$ & 0,242-0,31 (Hartley) / 0,31-0,4 (Huggins) / 0,4-0,85 (Chappuis) / 3,3 / 4,74 \\
O$_2$ & 0,175-0,2 (Schumann-Runge) / 0,2-0,26 (Herzberg) / 0,628 / 0,688 / 0,762 / 1,06 / 1,27 / 1,58 \\
CO$_2$ & 1,4 / 1,6 / 2,0 / 2,7 / 4,3 / >15 \\
H$_2$O & 0,72 / 0,82 / 0,94 / 1,1 / 1,38 / 1,87 / 2,7-3,2 / 6,25 / >14 \\
CH$_4$ & 1,66 / 2,2 / 2,3 / 2,37 / 3,26 / 3,31 / 3,53 / 3,83 / 3,55 / 7,65 \\
CO & 2,34 / 4,67 \\
N2O & 2,87 / 2,97 / 3,9 / 4,06 / 4,5 \\
\hline
\end{tabular}
\caption{\emph{Principales bandes d'absorption pour les gaz composant l'atmosphère terrestre. Voir la figure~\ref{fig:atmspectrum}}}
\end{center}
\end{table}
\normalsize

\figside{0.7}{0.35}{decouverte/cours_dyn/absorption.png}{Spectres d'absorption de l'atmosphère en fonction de la longueur d'onde. [Haut] Courbes d'émittance normalisée de corps noirs à 5780~K (rayonnement solaire) et 255~K (rayonnement terrestre). [Bas] Coefficients d'absorption (en~$\%$) entre le sommet de l'atmosphère et la surface. Les principaux gaz responsables de l'absorption à différentes longueurs d'onde sont indiqués en bas. Source: McBride and Gilmour, An Introduction to the Solar System, 2004 ; d'après Goody and Yung, Atmospheric radiation, 1989}{fig:atmspectrum}

\sk
Il existe une seconde loi de Kirchhoff, différente de celle précitée, qui stipule que l'émissivité spectrale doit être égale au coefficient d'absorption du corps $$ \epsilon_\lambda = \alpha_\lambda $$ pour des quantités intégrées selon toutes les directions de l'espace. Un corps ne peut émettre que les radiations qu'il est capable d'absorber. En d'autres termes, pour une température et une longueur d'onde donnée, un bon émetteur est souvent un bon absorbant (et vice versa). On retrouve par ce principe que le corps noir est le corps idéal qui rayonne un maximum d'énergie radiative à chaque température et pour chaque longueur d'onde. 
%Un corollaire est qu'un corps transparent ou réfléchissant à une certaine longueur d'onde émet peu de rayonnement thermique à cette même longueur d'onde. NON CAR IL SUFFIT QUE LA TEMPERATURE SOIT ELEVEE !



%\newpage
%\section{Effet de serre}
%\sk
Quelle température de surface est prédite par le modèle à une couche décrit par la figure~\ref{fig:modun} ? On considère toujours une planète d'albédo planétaire $A\e{b}$ recevant l'éclairement moyen $\mathcal{F}\e{s}'$ du Soleil. Ce bilan correspond à la partie visible de la figure~\ref{fig:modun}. L'atmosphère est considérée comme transparente dans ce domaine de longueur d'onde. Dans la partie infrarouge, au contraire on ne néglige plus l'absorption, par les gaz à effet de serre présents dans l'atmosphère, du rayonnement infrarouge émis par la surface de la planète à la température~$T\e{s}$~: on représente ainsi l'atmosphère par une couche isotherme de température~$T\e{a}$, parfaitement absorbante dans l'infrarouge. Le rayonnement infrarouge émis par la surface est complètement absorbé dans l'atmosphère, qui émet à son tour~$\sigma {T\e{a}}^4$ à la fois vers l'espace et vers la surface comme indiqué dans le domaine infrarouge de la figure~\ref{fig:modun}. Une partie du rayonnement infrarouge émis par la Terre n'est donc pas évacuée vers l'espace et reste \ofg{piégée} dans le système atmosphère~+~surface, contribuant ainsi à élever la température de la surface~$T\e{s}$.

\figside{0.6}{0.2}{decouverte/cours_meteo/une_couche.png}{Modèle à une couche~: schéma des flux échangés dans le visible et dans l'infrarouge pour une planète dont l'atmosphère de température~$T\e{a}$ est opaque dans l'infrarouge.}{fig:modun}

\sk
Il s'agit ensuite d'effectuer le bilan des flux reçus et cédés en chacune des interfaces en rassemblant les termes des deux domaines visible et infrarouge.
%%\footnote{Les modèles du type de celui présenté ici sont parfois également appelés modèles aux puissances échangées}
\begin{finger}
\item pour l'atmosphère
\[ \underbrace{\sigma \, {T\e{s}}^4}_{\text{bilan des flux reçus}} = \underbrace{\sigma \, {T\e{a}}^4 + \sigma \, {T\e{a}}^4}_{\text{bilan des flux cédés}} \] 
On note que le rayonnement visible reçu du Soleil n'intervient pas dans le bilan pour l'atmosphère, ce qui est normal puisque l'absorption est négligée. Ainsi, comme indiqué sur le schéma~\ref{fig:modun}, l'atmosphère reçoit un rayonnement~$\mathcal{F}\e{s}'$ dont la partie~$\mathcal{F}\e{s}'\,(1-A\e{b})$ qui n'est pas réfléchie/diffusée est entièrement transmise à la surface. Tout se passe comme si l'atmosphère recevait~$\mathcal{F}\e{s}'$ et cédait~$\mathcal{F}\e{s}'\,(1-A\e{b})$ à la surface et~$\mathcal{F}\e{s}'\,A\e{b}$ à l'espace~; son bilan d'énergie dans le visible est donc nul puisque tous ces termes se compensent.
\item pour la surface
\[ \underbrace{\mathcal{F}\e{s}'\,(1-A\e{b}) + \sigma \, {T\e{a}}^4}_{\text{bilan des flux reçus}} = \underbrace{\sigma \, {T\e{s}}^4}_{\text{bilan des flux cédés}} \]
\end{finger}
On dispose alors de deux équations qui permettent de déterminer les deux inconnues~$T\e{a}$ et~$T\e{s}$. Ainsi la température à la surface de la planète dans le modèle à une couche est 
\[ \boxed{ T\e{s} = \bigg[ \frac{ 2 \, \mathcal{F}\e{s}'\,(1-A\e{b}) }{ \sigma } \bigg]^{\frac{1}{4}} = \sqrt[4]{2} \, T\e{eq} } \]

\sk
Le calcul numérique donne une température de~$303$~K (environ~$30^{\circ}$C) pour la Terre, une valeur à la fois bien supérieure à~$T\e{eq}$, qui vaut~$255$~K, et plus proche de la température effectivement constatée à la surface, quoiqu'un peu surévaluée. Ainsi les gaz à effet de serre présents dans l'atmosphère contribuent à réchauffer significativement la surface d'une planète. Le modèle à une couche est le modèle le plus simple de l'effet de serre qui permet d'en rendre compte qualitativement et, dans une certaine mesure, quantitativement.

%
%\newpage
%\section{Effet de serre : modèle plus fin et effet parasol}
%\sk Le modèle à une couche peut être généralisé quelque peu en considérant deux raffinements.
\begin{finger}
\item L'atmosphère est absorbante dans le visible avec un coefficient d'absorption~$\alpha$. Plus~$\alpha$ est grand, plus le rayonnement incident dans les longueurs d'onde visible reçu par la surface terrestre est atténué. Ce phénomène porte le nom d'\voc{effet parasol} (ou parfois \ofg{anti-effet de serre}). En réalité cet effet est très modéré sur Terre. Certes, l'ozone stratosphérique absorbe complètement le rayonnement ultraviolet, mais ceci représente une contribution faible du flux total. Les aérosols, tels que les poussières désertiques ou les particules d'origine volcanique, absorbent dans le visible et peuvent contribuer lors d'événements particuliers, telles les éruptions volcaniques ou les tempêtes de poussière, à augmenter~$\alpha$.
\item L'atmosphère n'est pas tout à fait un corps noir~: son émissivité dans l'infrarouge est~$\epsilon$ comprise entre~$0$ et~$1$. D'après la seconde loi de Kirchhoff, ceci indique que la couche atmosphérique n'est pas parfaitement absorbante dans l'infrarouge~: elle absorbe une partie~$\epsilon \, F$ du flux incident~$F$ et en transmet une partie~$(1-\epsilon) \, F$. Reste qu'en pratique, comme mentionné dans la partie précédente, l'atmosphère se comporte comme un corps presque noir dans l'infrarouge et l'émissivité~$\epsilon$ est relativement proche de~$1$. 
\end{finger}
Un tel modèle porte le nom de \voc{modèle gris à une couche}. Comme dans la figure~\ref{fig:modun}, la méthode pour calculer les températures consiste à reporter tous les flux échangés entre chacune des couches, comme indiqué dans la figure~\ref{fig:modgris}, puis de faire le bilan des énergies reçues et cédées aux interfaces.
\[ \begin{aligned} & & \boxed{\text{bilan des flux reçus}} & = \boxed{\text{bilan des flux cédés}} \\ 
& \text{[espace]} & \mathcal{F}\e{s}'\,A\e{b} + (1-\epsilon) \, \sigma \, {T\e{s}}^4 + \epsilon \, \sigma \, {T\e{a}}^4 & = \mathcal{F}\e{s}' \\
& \text{[atmos.]} & \mathcal{F}\e{s}' + \sigma \, {T\e{s}}^4 & = \mathcal{F}\e{s}'\,A\e{b} + \mathcal{F}\e{s}'\,(1-A\e{b}-\alpha) + (1-\epsilon) \, \sigma \, {T\e{s}}^4 + \epsilon \, \sigma \, {T\e{a}}^4 + \epsilon \, \sigma \, {T\e{a}}^4 \\ 
& \text{[surface]} & \mathcal{F}\e{s}'\,(1-A\e{b}-\alpha) + \epsilon \, \sigma \, {T\e{a}}^4 & = \sigma \, {T\e{s}}^4 \\ \end{aligned} \]
%& \text{[surface]} & \mathcal{F}\e{s}'\,(1-A\e{b})\,(1-\alpha) + \epsilon \, \sigma \, {T\e{a}}^4 & = \sigma \, {T\e{s}}^4 \\ \end{aligned} \]

\sk
On obtient alors l'expression de la température de la surface de la planète dans le cadre du modèle gris en combinant les équations de bilan de deux des trois interfaces (par exemple espace et surface)
\[ \boxed{T\e{s} = \sqrt[4]{\frac{1 - \frac{\alpha^{\prime}}{2}}{1 - \frac{\epsilon}{2}}} \, T\e{eq}} \qquad  \text{avec} \quad \alpha^{\prime} = \frac{\alpha}{1-A\e{b}} \] 
\begin{citemize}
\item Dans le cas où l'atmosphère est un corps noir dans l'infrarouge~($\epsilon=1$) et qu'elle est transparente dans le visible~($\alpha=0$), on se retrouve dans la situation du modèle à une couche avec~$T\e{s} = \sqrt[4]{2} \, T\e{eq}$.
\item Dans le cas où l'atmosphère est opaque dans le visible~($\alpha^{\prime}=1$) et dans l'infrarouge~($\epsilon=1$), la surface échange alors uniquement du rayonnement avec l'atmosphère et~$T\e{s}=T\e{a}=T\e{eq}$.
\item Si~$\epsilon$ augmente, l'effet de serre augmente, donc la température de la surface de la planète augmente.
\item Si~$\alpha$ augmente, l'effet parasol augmente, donc la température de la surface de la planète diminue. C'est l'un des effets observés dans les mois qui suivent une éruption volcanique majeure.
\end{citemize}
%De façon plus générale, on a vu que le rayonnement sortant provenait majoritairement de la région de l'atmosphère autour d'une épaisseur optique de 1 à partir du sommet. Cette région dépend de la longueur d'onde: proche de la surface dans la fenêtre transparente, dans la haute troposphère dans les bandes d'absorption du CO$_2$, autour de 2~km dans celles de la vapeur d'eau. Comme la température décroit à partir de la surface, le rayonnement sortant est donc émis à des températures inférieures à $T_s$, et on peut écrire qu'il vaut \[IR_{sommet}=\sigma T_s^4 (1-\epsilon)=\sigma T_{eq}^4\] Où $\epsilon>0$ représente l'effet de serre. La valeur de $\epsilon$ augmente quand la température d'émission vers l'espace diminue par rapport à celle de surface, typiquement parce que l'altitude d'émission augmente.

\figside{0.7}{0.17}{decouverte/cours_meteo/une_couche_gris.png}{Modèle gris à une couche~: schéma des flux échangés dans le visible et dans l'infrarouge pour une planète comme la figure~\ref{fig:modun} sauf que l'atmosphère de température~$T\e{a}$ est opaque dans l'infrarouge, mais sans être un corps totalement noir, et est absorbante dans le visible avec un coefficient d'absorption~$\alpha$.}{fig:modgris}
%\figun{0.6}{0.2}{\figfrancis/GH_1lay_atm}{Comme la figure \ref{fig:GH1laynoatm} mais avec une atmosphère opaque dans l'infrarouge et de coefficient d'absorption $a$ dans le visible, de température $T_a$.}{fig:GH1layatm}


\end{document}
