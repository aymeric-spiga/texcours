\documentclass[a4paper,DIV16,10pt]{scrartcl}
%%%%%%%%%%%%%%%%%%%%%%%%%%%%%%%%%%%%%%%%%%%%%%%%%%%%%%%%%%%%%%%%%%%%%%%%%%%%%%%%%%%
\usepackage{texcours}
%%%%%%%%%%%%%%%%%%%%%%%%%%%%%%%%%%%%%%%%%%%%%%%%%%%%%%%%%%%%%%%%%%%%%%%%%%%%%%%%%%%
\newcommand{\zauthor}{Aymeric SPIGA}
\newcommand{\zaffil}{Laboratoire de Météorologie Dynamique}
\newcommand{\zemail}{aymeric.spiga@sorbonne-universite.fr}
\newcommand{\zcourse}{Planétologie}
\newcommand{\zcode}{ENS}
\newcommand{\zuniversity}{ENS / Sorbonne Université (Faculté des Sciences)}
\newcommand{\zlevel}{M1 Géosciences}
\newcommand{\zsubtitle}{Fiches complémentaires de cours}
\newcommand{\zlogo}{\includegraphics[height=1.5cm]{/home/aspiga/images/logo/LOGO_SU_HORIZ_SIGNATURE_CMJN_JPEG.jpg}}
\newcommand{\zrights}{Copie et usage interdits sans autorisation explicite de l'auteur}
\newcommand{\zdate}{\today}
%%%%%%%%%%%%%%%%%%%%%%%%%%%%%%%%%%%%%%%%%%%%%%%%%%%%%%%%%%%%%%%%%%%%%%%%%%%%%%%%%%%
\begin{document} \inidoc
%%%%%%%%%%%%%%%%%%%%%%%%%%%%%%%%%%%%%%%%%%%%%%%%%%%%%%%%%%%%%%%%%%%%%%%%%%%%%%%%%%%

%\newpage
%\section{Fil rouge}
%La force gravitationnelle. 
%On connaît son influence sur les orbites des planètes et satellites.
%Découvrons des effets importants en planétologie
%\begin{itemize}
%\item induit des forces de marée qui ont des effets des orbites aux intérieurs
%\item détermine la stabilité d'une atmosphère
%\item (anticipant un peu sur la suite : énergies) est une source d'énergie interne
%\end{itemize}


\newpage
\section{Marées}
\sk
La dépendance de la force gravitationnelle en $r^{-2}$ explique qu'un corps attracteur puisse exercer une force gravitationnelle différente en divers points d'un objet. Ce phénomène est appelé marée. L'effet de la force de marée sur un objet peut aller de la déformation simple, à la flexure produisant une chaleur interne, jusqu'à la destruction pure et simple de l'objet. Cette force est également responsable de certaines caractéristiques orbitales. L'action des marées explique que la Lune présente toujours la même face à la Terre, que la comète Shoemaker-Levy s'est détruite à l'approche de Jupiter, que Io soit le théâtre une activité volcanique soutenue, que les étoiles extrasolaires proches de leur étoile voient leur orbite perturbée. La formulation mathématique des marées en terme d'équilibre entre gravité et force centrifuge date de 1687 par Isaac Newton. 

\sk
L'accélération~$\vec{a}$ d'un objet de masse~$m$ sous l'action des forces~$\vec{F}$ vérifie la seconde loi de Newton~$m \vec{a} = \Sigma \vec{F}$. L'objet subit de la part d'un corps attracteur de masse~$M$ la force de gravitation
\[ \vec{F_g} = -\mathcal{G} \frac{M\,m}{r^3} \vec{r} \]
\noindent où~$\mathcal{G}$ est la constante de gravitation, $\vec{r}$ est le vecteur radial reliant les deux corps. 

\sk
L'objet ainsi dans le champ de gravitation du corps attracteur, à supposer que sa trajectoire soit fermée, décrit une ellipse d'excentricité~$e$ avec une certaine fréquence de rotation~$\Omega$. Dans le référentiel tournant, en interprétant les termes cinématiques liés au changement de référentiel comme des forces apparentes, on montre que l'objet de masse~$m$ subit une force d'entraînement (appelé communément, de manière un peu abusive, force centrifuge)
\[ \vec{F_e} = m \, \Omega^2 \vec{r} \]

\sk
L'accélération est nulle en tout point de l'orbite, il y a équilibre entre force de gravitation et force centrifuge, ce qui permet de retrouver la troisième loi de Kepler
\[ \Omega^2 \, r^3 = \mathcal{G} \, M  \]

\sk
L'action des marées se comprend en considérant que l'objet n'est pas un point matériel mais, par exemple, une planète de rayon~$R$. Au centre de masse~O de l'objet, fixe dans le référentiel tournant, l'équilibre entre force de gravitation et force centrifuge s'écrit
\[ \mathcal{E}(r) = -\mathcal{G} \frac{M\,m}{r^2} + m \, \Omega^2 r = 0 \]
Sur le lobe de l'objet le plus proche du corps attracteur (subplanétaire S), la force de gravitation domine la force centrifuge, attirant ledit lobe vers le corps attracteur : $\mathcal{E}(r-R) < 0$. Sur le lobe de l'objet le plus éloigné du corps attracteur (antiplanétaire A), l'inverse se produit, éloignant ledit lobe du corps attracteur : $\mathcal{E}(r+R) > 0$. Les forces de marées ont donc tendance à allonger le corps en lui conférant une forme avec deux lobes qui donne la périodicité semi-diurne des marées. Plus le corps attracteur est gros, plus l'effet est marqué, pouvant conduire jusqu'à la destruction du corps.

\sk
L'accélération résultante au point S est
\[ -\mathcal{G} \frac{M}{(r-R)^2} + \Omega^2 (r-R) = -\mathcal{G} \frac{M}{(r-R)^2} + \mathcal{G} \, \frac{M}{r^3} \, (r-R) \]
\noindent en notant~$\epsilon = R/r \ll 1$ et réalisant un développement limité au premier ordre
\[ -\frac{\mathcal{G}\,M}{r^2} \left[ \frac{-1}{(1-\epsilon)^2} + 1-\epsilon \right] \sim - 3 \, \frac{\mathcal{G}\,M}{r^2} \,\epsilon \]
La force de marée~$f_m$ exercée par unité de masse est donc
\[ f_m = - 3 \, \frac{\mathcal{G}\,M\,R}{r^3} = \frac{-3\,\mathcal{G}}{R^2} \, \textcolor{blue}{ M \, \left[ \frac{R}{r} \right]^3 } \]
\noindent Le terme en couleur, qui inclut à la fois la masse~$M$ de l'objet attracteur et le rapport~$R/r$ entre rayon planétaire et distance au corps attracteur, détermine l'intensité des forces de marées. Noter la puissance trois en le rapport~$R/r$ qui rend ce terme souvent prépondérant sur le terme~$M$. La Terre subit ainsi des forces de marée de la Lune bien plus intenses que de la part du Soleil bien que ce dernier soit plus massif.













\newpage
\section{Calcul de la force de marée}
%% AlexBarbo: Faire un récapitulatif de toutes les notations ? Notamment à cause des risques de confusion M/m, R/r/a, surtout dans la partie limite de Roche

\sk
La formulation mathématique des marées en terme d'équilibre entre gravité et force centrifuge date de 1687 par Isaac Newton. L'accélération~$\vec{A}$ d'un objet de masse~$m$ sous l'action des forces~$\vec{F}$ vérifie la seconde loi de Newton~$m \vec{A} = \Sigma \vec{F}$. L'objet subit de la part d'un corps attracteur de masse~$M$ la force de gravitation
\[ \vec{F_g} = -\mathcal{G} \frac{M\,m}{a^2} \vec{u_r} \]
\noindent où~$\mathcal{G}$ est la constante de gravitation, $\vec{u_r}$ est le vecteur unitaire reliant les deux corps. 
\sk
L'objet ainsi dans le champ de gravitation du corps attracteur, à supposer que sa trajectoire soit fermée, décrit une ellipse d'excentricité~$e$ avec une certaine fréquence de rotation~$\Omega$. Dans le référentiel tournant (autour du corps attracteur), en interprétant les termes cinématiques liés au changement de référentiel comme des forces apparentes, on montre que l'objet de masse~$m$ subit une force d'entraînement (appelé communément, de manière un peu abusive, force centrifuge)
\[ \vec{F_e} = m \, \Omega^2 \, a \, \vec{u_r} \]

\sk
L'accélération est nulle en tout point de l'orbite, il y a équilibre entre force de gravitation et force centrifuge, ce qui permet de retrouver la troisième loi de Kepler
\[ \Omega^2 \, a^3 = \mathcal{G} \, M  \]

\sk
L'action des marées se comprend en considérant que l'objet n'est pas un point matériel mais, par exemple, une planète de rayon~$r$. Au centre de masse~O de l'objet, fixe dans le référentiel tournant, l'équilibre entre force de gravitation et force centrifuge s'écrit
\[ \mathcal{F}(a) = -\mathcal{G} \frac{M\,m}{a^2} + m \, \Omega^2 a = 0 \]
Sur le lobe de l'objet le plus proche du corps attracteur (subplanétaire S), la force de gravitation domine la force centrifuge, attirant ledit lobe vers le corps attracteur : $\mathcal{F}(a-r) < 0$. Sur le lobe de l'objet le plus éloigné du corps attracteur (antiplanétaire A), l'inverse se produit, éloignant ledit lobe du corps attracteur : $\mathcal{F}(a+r) > 0$. Les forces de marées ont donc tendance à allonger le corps en lui conférant une forme avec deux lobes qui donne la périodicité semi-diurne des marées. Plus le corps attracteur est gros, plus l'effet est marqué, pouvant conduire jusqu'à la destruction du corps.

%% AlexBarbo: On peut aussi dire  directement qu'il s'agit du développement de Taylor de la force de gravité de l'astre attracteur en r+/-R, non? Je trouve juste l'interprétation plus physique, d'autant qu'elle donne directement la dépendance en r^-3
%\[ 
%-\mathcal{G} \frac{M\,m}{(r \pm R)^2} = 
%-\mathcal{G} \frac{M\,m}{R^2} \frac{1}{1 \pm (\frac{R}{r}})^2}
%\]

\sk
L'accélération résultante au point S est
\[ -\mathcal{G} \frac{M}{(a-r)^2} + \Omega^2 (a-r) = -\mathcal{G} \frac{M}{(a-r)^2} + \mathcal{G} \, \frac{M}{a^3} \, (a-r) \]
\noindent en notant~$\epsilon = r/a \ll 1$ et réalisant un développement limité au premier ordre
\[ -\frac{\mathcal{G}\,M}{a^2} \left[ \frac{-1}{(1-\epsilon)^2} + 1-\epsilon \right] \sim - 3 \, \frac{\mathcal{G}\,M}{a^2} \,\epsilon \]
La force de marée~$f_m$ exercée par unité de masse est donc
\[ f_m = - 3 \, \frac{\mathcal{G}\,M\,r}{a^3} = \frac{-3\,\mathcal{G}}{a^2} \, \textcolor{blue}{ M \, \left[ \frac{r}{a} \right]^3 } \]
\noindent Le terme en couleur, qui inclut à la fois la masse~$M$ de l'objet attracteur et le rapport~$r/a$ entre rayon planétaire et distance au corps attracteur, détermine l'intensité des forces de marées. Noter la puissance trois en le rapport~$r/a$ qui rend ce terme souvent prépondérant sur le terme~$M$. La Terre subit ainsi des forces de marée de la Lune bien plus intenses que de la part du Soleil bien que ce dernier soit plus massif.


\newpage
\section{Couple de marée}

\figside{0.4}{0.05}{/home/aspiga/images/decouverte/cours_geomorpho/maree_1.png}{Mécanisme de l'effet de marée gravitationnelle. $\omega$ représente le sens de rotation de la planète. Crédits J. Laskar}{fig:couplemaree}

\sk
\paragraph{Couple des marées} Si les planètes étaient parfaitement fluides, l'objet serait déformé par les forces de marée dans la direction donnée par le centre de masse de l'objet et celui du corps attracteur, comme indiqué par le calcul des forces de marée. En pratique, puisque les objets ne sont pas parfaitement élastiques, la déformation des forces de marée n'est pas instantanée, et se réalise pendant un certain temps caractéristique. Ainsi la direction de déformation (des points subplanétaire S et antiplanétaire A) forme un angle (décalage angulaire~$\delta$ sur la figure~\ref{fig:couplemaree}) avec la direction donnée par le centre de masse de l'objet et celui du corps attracteur. 

\sk
Il s'exerce alors un couple de force de rappel en A et en S~: ce couple agit toujours dans le sens s'opposant à la rotation propre~$\omega$ de l'objet. On définit 
\begin{citemize}
\item la déformation élastique de l'objet en réponse aux perturbations de marée, notée $k_T$ (appelée \emph{tidal Love number} qui contient l'information sur la rhéologie du corps)
\item le facteur de qualité~$Q$ donnant en un cycle le rapport entre l'énergie maximale stockée dans le bourrelet de marée et l'énergie dissipée  
\end{citemize}
\noindent Le couple~$\Gamma_m$ exercé par les forces de marée dépend donc 
\begin{citemize}
\item de la taille du bourrelet induit (loi en~$k_T \, M \, a^{-3}$), 
\item du décalage de phase entre l'orientation de ce bourrelet et l'axe des centres de masse (loi en~$Q^{-1}\,\textrm{sgn}(\omega-\Omega)$), 
\item et s'exprime comme le couple d'une force de marée~$f_m$ (loi proportionnelle à~$\mathcal{G}\,M\,a^{-3}$). 
\end{citemize}
Finalement
\[ \Gamma_m = \frac{3}{2} \, \frac{k_T}{Q} \, \frac{\mathcal{G}\,M^2\,r^5}{a^6} \, \textrm{sgn}(\omega-\Omega) \]
\noindent Le couple des marées~$\Gamma_m$ permet de transférer du moment angulaire entre rotation propre et révolution orbitale.

\sk
\paragraph{Synchronisation et modification d'orbites: rotation propre~$\omega$} Le couple de marée cause au long terme une synchronisation des orbites, dans la plupart des cas\footnote{La modification de la rotation propre d'un corps par les forces de marée ne conduit pas toujours à une rotation synchrone, en témoigne le cas de Mercure en résonance $3:2$ (3 rotations propres pour 2 révolutions orbitales) sous l'effet des forces de marée du Soleil. Correia et Laskar dans un article Nature de 2004 ont montré que cela résulte de l'évolution chaotique (déterministe) de l'orbite de Mercure, et non de la friction noyau-manteau comme cela avait été proposé précédemment.}. La synchronisation peut se décrire ainsi~: le point de départ est un objet en rotation asynchrone, dont le bourrelet induit par les forces de marée est \og en retard \fg~par rapport à la rotation de la planète (figure~\ref{fig:couplemaree}). Le couple des marées décrit ci-dessus va toujours dans le sens d'une synchronisation, car il tend à s'opposer au sens de rotation de la planète de manière à faire tendre celle-ci vers la vitesse de révolution. Par ce mécanisme, la Lune est en rotation synchrone autour de la Terre : elle lui présente toujours la même face car sa période propre~$\omega$ est égale à sa période de révolution~$\Omega$.

\sk
\paragraph{Temps de synchronisation} Le temps caractéristique~$\tau_s$ de synchronisation de l'orbite d'un objet autour d'un corps attracteur, dans le cas le plus simple, est
\[ \tau_s = \frac{\omega \, a^6 \, I \, Q}{3 \, \mathcal{G} \, M^2 \, k_T \, r^5} \]
\noindent avec I le moment d'inertie de l'objet cible ($I=\frac{2}{5} m r^2$ pour une sphère). Le temps de synchronisation est donc proportionnel à~$a^6 / M^2$. Avec ce type de loi, on comprend qu'une planète dans la Zone Habitable d'une naine brune de $M = 0.2$~masses solaires, de luminosité bien plus faible que le Soleil, est située bien plus près de son étoile que la Terre du Soleil. Bien que son étoile soit moins massive que le Soleil, la variation en~$a^6$ indique un temps de synchronisation de l'orbite de cette planète 5 ordres de grandeur plus rapide que la Terre située dans la Zone Habitable du Soleil\footnote{Exemple inspiré de la fiche en ligne de Martin Turbet \url{https://media4.obspm.fr/public/ressources_lu/pages_planetologie-habitabilite/torque.html}}. 


\newpage
\section{Marée et orbites}

\sk
\paragraph{Synchronisation et modification d'orbites: demi-grand axe~$a$} En plus d'elle-même se synchroniser, la Lune elle-même ralentit la rotation propre de la Terre (2 millisecondes en plus par siècle) pour tendre virtuellement (à très long terme \ldots) vers une synchronisation. Ce lent ralentissement de la rotation propre de la Terre implique donc que la Terre perd du moment angulaire, qui est transféré à la Lune. Comme cette dernière voit sa rotation synchronisée plus rapidement, l'ajustement se fait via une augmentation de la distance orbitale Terre-Lune.
%\footnote{Autre exemple d'ajustement (inspiré par Tristan Guillot), lorsque le Soleil aura consommé son hydrogène dans~$\sim 5$~milliards d'années et sera devenu une géante rouge (pouvant s'étendre potentiellement jusque l'orbite de Mars). Le Soleil perdra de la masse, donc du moment angulaire, et comme la Lune par rapport à la Terre, la Terre s'éloignera du corps attracteur responsable des marées, ici le Soleil. A moins que les marées exercées par la Terre sur l'enveloppe du Soleil ne la conduisent au contraire à se rapprocher du Soleil.}.

\sk
\paragraph{Modification d'orbites (quantitatif)} Pour un objet de masse~$m$ suivant une orbite elliptique d'excentricité~$e$ et de demi grand axe~$a$ sous l'effet de l'attraction d'un corps de masse~$M$, l'énergie totale~$E = - \frac{\mathcal{G}\,m\,M}{2\,a}$ et le moment angulaire~$L_0 = m \, \sqrt{\mathcal{G}\,M\,a\,(1-e^2)}$ sont conservés\footnote{L'énergie totale~$E$ correspond à l'énergie cinétique
plus l'énergie potentielle de gravitation. %% Etot = Egrav + Ecin L0= m*v*a Et on a Egrav avec Fgrav=-grad(E)
Le moment angulaire~$L_0$ se conserve car la force est centrale; 
il s'obtient simplement à l'apoastre (point 2) et périastre (point 1)
$L_0 = m v_1 d_1 = m v_2 d_2$, puis en utilisant
l'équation de l'énergie~$\left( v_1^2 - v_2^2 \right) = 2 \mathcal{G} M \left( \frac{1}{d_1} - \frac{1}{d_2} \right)$
on en déduit
\[ \frac{L_0^2}{2 m^2} = \frac{\mathcal{G}M}{(\frac{1}{d_1}-\frac{1}{d_2})} \]
puis le résultat~$L_0 = m \, \sqrt{\mathcal{G}\,M\,a\,(1-e^2)}$
étant donné que~$r_1 = a (1-e)$ et~$d_2 = a (1+e)$.}
En éliminant~$m$ dans les deux équations qui précèdent, on obtient une relation de proportionalité entre~$E$ et~$L_0$
\[ E = - \frac{\mathcal{G}\,M}{2\,a\,\sqrt{\mathcal{G} \, M \, a \, (1-e^2)}} \,L_0 \qquad \Rightarrow \qquad \ddf{E}{L_0} = \frac{E}{L_0} = - \frac{1}{2} \sqrt{ \frac{\mathcal{G}\,M}{a^3 \, (1-e^2)} } \]
En introduisant~$\Gamma_m = \ddf{L_0}{t}$, le couple exercé par les forces de marée, on obtient la variation temporelle d'énergie orbitale par les forces de marée
\[ \ddf{E}{t} = \ddf{E}{L_0} \ddf{L_0}{t} = - \frac{\Gamma_m}{2} \, \sqrt{ \frac{\mathcal{G}\,M}{a^3 \, (1-e^2) } } \]
\noindent En exprimant par ailleurs la dérivée~$\ddf{E}{t} = \ddf{a}{t} \dfrac{\mathcal{G}\,m\,M}{2\,a^2}$, on parvient à
%% = \Gamma_m \, \sqrt{ \frac{\mathcal{G}\,M}{a^3 \, (1-e^2) } }
\[ \ddf{a}{t} = \Gamma_m \, \frac{2\,\mathcal{G}^{-1/2}\,m^{-1/2}}{M \, a^{-1/2} \, (1-e^2) } \]
\noindent En remplaçant par l'expression du couple des marées 
%\[ \Gamma_m = \frac{3}{2} \, \frac{k_T}{Q} \, \frac{\mathcal{G}\,M^2\,r^5}{a^6} \, \textrm{sgn}(\omega-\Omega) \]
et en supposant une orbite de faible excentricité~$e \ll 1$, le changement de rayon orbital (contraction ou expansion) s'écrit
\[
\ddf{a}{t} = 3 \, \frac{k_T}{Q} \, \frac{\mathcal{G}^{1/2}\,M\,r^5}{m^{1/2} \, a^{11/2} } \, \textrm{sgn}(\omega-\Omega)
\]
Ce type de formule permet de déterminer que la Lune s'éloigne de la Terre à une vitesse d'environ 4~cm~an$^{-1}$.
Elle peut permettre de comprendre pourquoi une exoplanète géante ($r$ grand) si près de son étoile ($a$ petit),
comme l'était 51 Peg b première exoplanète découverte en 1995, était considérée comme 
improbable (car instable) avant sa découverte. D'où la nécessité d'adopter des modèles plus sophistiqués.

\sk
\paragraph{Circularisation} Les forces de marée ont également pour effet de rendre les orbites circulaires en faisant tendre l'excentricité~$e$ vers~$0$. Si la révolution du corps autour du corps attracteur est excentrique, en supposant la rotation synchrone, la différence de vitesse orbitale entre le passage au périastre plus rapide que le passage à l'apoastre fait qu'un couple de force va s'exercer pour repousser le corps au périastre et l'attirer à l'apoastre, rendant ainsi l'orbite plus circulaire. La littérature (e.g. Peale 2003) fournit une formule quantitative
\[ \ddf{e}{t} = - \frac{21}{2} \, \frac{k_T}{Q} \, \frac{M}{m} \, \left( \frac{r}{a} \right)^5 \, e  \]
\noindent ce qui indique que l'échelle de temps caractéristique~$\tau = \frac{-e}{\ddf{e}{t}} $ de la circularisation varie en~$\left( \frac{a}{r} \right)^5$, donc très abruptement avec le demi-grand axe~$a$. Ceci explique que les orbites des satellites des planètes géantes ont une faible excentricité, et que la plupart des exoplanètes telles que~$a < 0.07$~UA ont des orbites circulaires.


\newpage
\section{Limite de Roche}
\sk
Les forces de marée peuvent être si intenses qu'elles peuvent affecter les caractéristiques physiques des corps au point de les détruire, particulièrement si le corps est fluide ou faiblement agrégé. La comète Shoemaker-Levy 9 a ainsi été détruite en 1994 lors d'une rencontre proche avec Jupiter -- avant que ses débris n'impactent la planète. Ceci a montré que l'agrégat de glaces et roches qui formait Shoemaker-Levy 9 était peu compact et peu dense. On appelle \voc{limite de Roche} la distance orbitale en deçà de laquelle des lunes d'une taille non négligeable ne peuvent se former par accrétion, en raison d'une prédominance des forces de marée sur les forces de cohésion du corps. Entre un corps attracteur et sa limite de Roche, des anneaux se forment plutôt que des satellites (exemple le plus célèbre, Saturne).

\sk
Une première approche est de considérer que les forces de cohésion sont purement gravitationnelles. La \voc{sphère d'influence de Hill} est telle qu'au delà d'un rayon~$R_H$ de la planète, la force de marée exercée par le corps attracteur sur un point de la planète devient plus importante que l'attraction gravitationnelle de la planète elle-même. Cela s'écrit pour des forces massiques:
\[ |f_m| \geq \mathcal{G} \frac{m}{R^2} \qquad \Rightarrow \qquad R \geq R_H = r \, \sqrt[3]{ \frac{m}{3\,M} } \]

\sk
La sphère de Hill indique la région dans laquelle des corps peuvent orbiter autour d'une planète. Pour obtenir une première approximation de la limite de Roche, en fixant au contraire le rayon~$R$ de l'objet, et en introduisant sa masse volumique~$\rho$ ainsi que rayon~$R_{\star}$ et masse volumique~$\rho_{\star}$, nous obtenons
\[ R = r \, \sqrt[3]{ \frac{\rho \, R^3}{3\, \rho_{\star} \, {R_{\star}}^3} } \]
d'où la distance orbitale limite~$r_R$ en deçà de laquelle les forces de marées dominent
\[ \frac{r_R}{R_{\star}} = 1.44 \, \sqrt[3]{ \frac{\rho_{\star}}{\rho} } \]

\sk
Une seconde approche, plus réaliste que l'approche de Hill ci-dessus, est de considérer que les forces de cohésion sont en lien avec la résistance du matériau et les forces de friction dans l'objet considéré. Un calcul avec un corps liquide (complètement déformable) permet d'obtenir une estimation raisonnable de la limite de Roche
\[ \frac{r_R}{R_{\star}} = 2.456 \, \sqrt[3]{ \frac{\rho_{\star}}{\rho} } \]
\noindent Dans le cas où le corps est résistant à la déformation, il peut être stable même à l'intérieur de la limite de Roche. Les particules formant les anneaux des planètes sont si petites que les forces de marée ne peuvent être assez puissantes pour dominer les forces de cohésion. Si les masses volumiques des deux corps sont proches, la limite de Roche est simplement située à~$r_R \simeq 2.456 \, R_{\star}$.

\sk
La limite de Roche permet qualitativement (et, dans une certaine mesure, quantitativement) d'expliquer l'observation typique autour des planètes géantes d'une configuration~: planète -- anneaux -- petites lunes -- larges lunes. Un modèle de niveau de sophistication supérieure doit bien souvent être invoqué, puisque des petites lunes cohabitent à proximité d'anneaux (\emph{shepherding}).

%% Figure 11.5 DePater et Lissauer


\newpage
\section{Chauffage par les marées}

\sk
Les forces de marées peuvent conduire à un chauffage interne des corps,
car elles produisent des déformations (flexures) à l'intérieur de ces corps
et les mouvements relatifs de ces masses produisent des frottements internes
qui induisent le chauffage de friction.
Autrement dit, via les forces de marée, l'énergie orbitale et de rotation propre sont dissipées par chauffage frictionnel dans l'intérieur du corps.
Lorsque la contraction gravitationnelle ou la radioactivité ne conduisent
pas à des flux dominants (comme c'est le cas respectivement sur les planètes géantes et sur Terre),
le chauffage par les forces de marées
peut être la source de chaleur principale à l'intérieur des corps.
Ainsi, le chauffage par les forces de marée de Jupiter et Saturne sur 
les lunes Io, Europe, Encelade est dominant.
Le taux~$\dot{E}$ de dissipation d'énergie par les marées peut s'écrire
selon la littérature (Showman et al. 1997, citant Peale et Cassen 1978)
\[ \mathcal{P} = \dot{E} = \frac{21}{2} \, \frac{k_T}{Q} \, \frac{R^5\,\mathcal{G}M^2\,\Omega}{a^6} \, e^2 \]
\noindent Il convient de remarquer que cette formule est valable
pour les faibles excentricités.
%%%% Peale S.J. & Cassen P. "Melting of Io by tidal dissipation", Science 203, 892 (1979)
%Peale and Cassen \[ \ddf{E}{t} = \frac{36}{19} \, \frac{\pi\,\rho^2\,n^5\,R^7\,e^2}{\mu\,Q} \]
%%% chapitre 4 thèse Jérémy

%% AlexBarbo: Et du coup aucun chauffage interne pour une excentricité nulle, pourquoi ? Quel est le lien précis entre excentricité et friction, même s'il est sous-jacent ?

\sk
Le chauffage par les forces de marée est un mécanisme central pour
expliquer un chauffage interne au sein de certains satellites
qui possède une atmosphère fine. L'exemple le plus frappant
est Io satellite de Jupiter dont l'activité volcanique
(source principale de sa fine atmosphère)
est causé par le chauffage interne par les forces de 
marées\footnote{Les fortes marées devraient également provoquer
une circularisation de l'orbite de Io, mais ce n'est pas le cas
en raison des perturbations causées par les autres satellites galiléens}.
Le chauffage par les marées est également la source d'énergie
principale d'Europe et Ganymède, satellite de Jupiter -- dans les
deux cas, il est responsable de la présence d'océans souterrains.
Enfin, le chauffage des marées est le meilleur candidate pour
expliquer les panaches observés sur Encelade (lune de Saturne)
et Triton (lune de Neptune).











%\newpage
%\section{Echappement}
%\sk
La région de l'atmosphère d'une planète soumise
à l'échappement est appelée \voc{exosphère} et sa base
est l'exobase. Plus précisément, l'\voc{exobase} est
définie comme l'altitude à laquelle le libre
parcours moyen d'une particule (distance qu'elle peut parcourir sans
entrer en collision avec une autre particule)
est égal à une échelle de hauteur atmosphérique.
Sur Mercure ou la Lune, l'exobase est confondue avec la surface;
sur les planètes telluriques, elle est à quelques centaines de kilomètres de la surface.
La définition de la pression et de la température y sont sujettes
à caution.

\sk
Les vitesses~$v$ des molécules de masse~$m$ 
composant un gaz avec une densité particulaire~$N$,
en équilibre thermique à la température~$T$, 
suivent une distribution maxwellienne
\[ f(v) \dd v = N 
\, \left( \frac{2}{\pi} \right)^{1/2}
\, \left( \frac{m}{k_B \, T} \right)^{3/2}
v^2 \, e^{ \frac{-m\,v^2}{2\,k_B\,T}} \, \dd v \]
\noindent En dessous de l'exobase, les collisions entre molécules
assurent le caractère maxwellien de la distribution.
Au dessus de l'exobase, les collisions sont beaucoup
plus rares (trajectoires ballistiques) et les molécules 
dans la queue de la distribution
maxwellienne ayant une vitesse~$v > v_e$ peuvent s'échapper.
La distribution maxwellienne des vitesses s'étend
jusque l'infini, mais la distribution des molécules
est plutôt gaussienne, ce qui implique en pratique
que quasiment aucune molécule avec des vitesses
plus grandes qu'environ 4 fois la vitesse thermique moyenne
\textcolor{red}{$v_0=\sqrt{\frac{2 k_B T}{m}}$}.

\sk
Pour quantifier au premier ordre à quel point une planète
peut retenir son atmosphère, on définit le paramètre d'échappement~$\lambda_e$,
rapport entre
vitesse d'échappement~\textcolor{magenta}{$v\e{e} = \sqrt{\frac{2\,\mathcal{G}\,M\e{p}}{R+h}}$}
(vitesse d'une molécule permettant de se libérer 
de l'attraction gravitationnelle de la planète)
et vitesse thermique moyenne~$v_0$
\[ \lambda\e{e} = \left( \frac{v_{e}}{v_0} \right)^2 = \frac{\mathcal{G}\,M\e{p}\,m}{k_B\,T\,(R+z)} \] %= \frac{R+z}{H(z)}
\noindent Le paramètre d'échappement peut également être vu comme
le rapport entre énergie potentielle et cinétique.
Si le paramètre d'échappement~$\lambda\e{e}$ est significativement plus petit que 1 ($v_{e} < v_0$), 
les molécules considérées sont susceptibles de s'échapper en nombre.
Ainsi, seules les planètes froides et massives 
peuvent retenir les éléments légers comme l'hydrogène.

\sk
En intégrant le flux de molécules avec une distribution
maxwellienne des vitesses au-dessus de l'exobase,
avec l'approximation d'une atmosphère en équilibre hydrostatique,
on obtient le taux d'échappement thermique de Jeans (atomes par cm$^2$ par s)
\[ \Phi_J = \frac{N_{ex} \, v_0}{2\,\sqrt{\pi}} \, (1+\lambda_e) \, e^{-\lambda_e} \]
\noindent avec les indices $\emph{ex}$ pour l'exobase. Les paramètres typiques pour la Terre : $N\e{ex} = 10^5$~cm$^{-3}$, $T\e{ex} = 900$~K, $\lambda\e{e} \simeq 8$ pour l'hydrogène atomique, donnent~$\Phi\e{J} \simeq 6 \times 10^7$~cm$^{-2}$~s$^{-1}$.
%Le flux hors la planète ne peut dépasser le flux limite imposé par la diffusion.
%
Les éléments ou isotopes les plus légers sont perdus par échappement
à un taux bien plus élevé ; l'échappement de Jeans peut donc
générer un fractionnement isotopique significatif.
La présence ou non d'une atmosphère peut être prédite (au premier ordre)
en évaluant le taux d'échappement de Jeans.
Le temps caractéristique de stabilité d'une atmosphère peut
être obtenu par~$\tau\e{e} = \frac{H \, N}{\Phi\e{e}}$.


%Vitesse thermique $V_{th}$: Vitesse d'une molécule atmosphérique d'énergie cinétique~$\frac{3}{2}kT$
%$$V_{th}=\sqrt{\frac{3RT}{M}}$$
%\begin{itemize} \item \'Echappement thermique \begin{itemize} \item Négligeable uniquement si $\lambda = \left( \frac{V_{e}}{V_{th}}   \right) \gg 1$ \item Flux de Jeans : $\Phi_e = \frac{n_c}{2\sqrt{\pi}} V_{th} (1+\lambda) \exp (-\lambda)$ \item Inefficace sur Vénus \item Echappement de H, H$_2$ et D pour Mars, Titan et la Terre. \end{itemize} \item \'Echappement non thermique \begin{itemize} \item Processus majoritaires \item \'Echappement d'ions positifs excités. \item Excitations et ionisation par collisions (avec autres atomes ou électrons libres) et/ou photochimie. \end{itemize} \end{itemize}
%\begin{tabular}{|c|c|c|c|c|}
%\hline
%                & Vénus & Terre & Mars & Titan \\
%\hline
%$V_{e}$ [km/s]  & 10,3  & 11,2  & 5,0  & 2,6   \\
%\hline
%$V_{th}$ [km/s] & 2,2   & 4,1   & 2,4  & 1,8   \\
%\hline
%\end{tabular}


\figside{0.6}{0.15}{decouverte/cours_gqe/escape.png}{McBride and Gilmour, An Introduction to the Solar System, 2004}{echapplanete}


















%% voir slides Emmanuel




%\newpage
%\section{Echappement (autre)}
%\sk
\paragraph{Echappement non thermique} Le flux d'échappement de Jeans donne une limite inférieure au flux d'échappement de l'atmosphère d'un corps céleste. Le flux d'échappement peut en réalité être plus élevé du fait de phénomènes qualifiés de non-thermiques, ayant trait à la photochimie de la haute atmosphère et des phénomènes ioniques. Ces mécanismes peuvent se révéler importants pour les atomes plus lourds que le formalisme de Jeans n'aurait pas indiqué en échappement.
\begin{description}
\item{\emph{dissociation et recombinaison dissociative}} : une molécule est dissociée par les radiations UV ou un électron impactant, ou un ion se dissocie par recombinaison, formant des produits qui ont une énergie suffisante pour échapper à l'attraction gravitationnelle d'un corps
\item{\emph{réaction ion-neutre}} : une réaction entre un ion atomique et une molécule peut donner un ion moléculaire et un atome rapide
\item{\emph{échange de charge}} : un ion rapide peut échanger sa charge avec un neutre sans perdre son énergie cinétique, résultant en un neutre rapide pouvant s'échapper de l'attraction gravitationnelle du corps\footnote{Phénomène clé sur Io où des atomes de sodium rapides sont créés par échange de charge avec le plasma magnétosphérique.}
%\item{\emph{champs électriques}} les champs électriques dans la ionosphère accélèrent les ions, qui à leur tour peuvent accélérer les neutres par collision.
\item{\emph{bombardement (\emph{sputtering})}} : quand un atome, ou plus généralement un ion, rapide entre en collision avec un atome, ce dernier peut s'échapper ; ce mécanisme est important dans une variété d'atmosphères (épaisses ou ténues) et même sur les surfaces planétaires des corps dépourvus d'atmosphère.
\item{\emph{balayage \emph{sweeping} par le vent solaire}} sans champ magnétique planétaire, les particules chargées peuvent réagir directement avec le vent solaire et être emportées. %% voir SL p89
\end{description}

\sk
\paragraph{Echappement fluide} Ces mécanismes d'échappement -- de caractère plus extrême voire catastrophique -- ont pu être dominants par le passé pour les corps que nous connaissons.
\begin{description}
\item{\emph{échappement hydrodynamique}} : un courant (super)sonique formé des atomes les plus légers entraîne les atomes et molécules plus lourdes -- qui n'auraient pas subi d'échappement thermique au sens de Jeans ; une source d'énergie intense dans la haute atmosphère est requise pour que l'échappement hydrodynamique soit significatif : pour les atmosphères de Vénus, la Terre et Mars dans un lointain passé, cela a pu être possible en raison d'un flux extrême UV soutenu.
\item{\emph{érosion par impact}} : si l'impacteur est plus grand qu'une échelle de hauteur, une fraction significative du gaz chaud suite au choc peut s'échapper si la vitesse d'impact la vitesse d'échappement.
\end{description}

%% exercice: calculer la taille typique d'un impacteur


%\newpage
%\section{Energie gravitationnelle}
%\sk
L'énergie gravitationnelle est accumulée au cours de la croissance de la planète
par accrétion de matériel la composant.
L'énergie potentielle gravitationnelle acquise par une sphère planétaire
de masse~$m(r)$ et de rayon~$r$ lorsqu'une mass infinitésimale $\dd m$
est ajoutée depuis une distance infinie
est
\[ \dd E\e{p} = - \mathcal{G} \, \frac{m(r) \, \dd m}{r}  \]

\sk
Supposons que le corps planétaire est formé en maintenant une densité constante~$\rho_0$
jusqu'à ce que son rayon soit~$R$ ; alors l'incrément infinitésimal de masse~$\dd m$
est~$\dd m = 4 \, \pi \, r^2 \, \rho_0 \, \dd r$, et~$E\e{p}$ peut être déterminée par
intégration
\[ E\e{p} = - \mathcal{G} \, \int_{0}^{R} \frac{m(r) \, \dd m}{r}  
= - 3 \, \mathcal{G} \, \left( \frac{4\,\pi}{3} \right)^2 \, \rho_0^2 \, \frac{R^5}{5}
= - \frac{3}{5} \, \frac{\mathcal{G}\,M^2}{R} \] 
\noindent Plus généralement, pour une distribution sphérique quelconque~$\rho(r)$,
une analyse dimensionnelle indique que~$E\e{p}$ est proportionnelle à~$-\mathcal{G}\,M^2/R$.
La formule simplifiée ci-dessus donne le bon résultat pour la Terre avec seulement~$10\%$ d'erreur.

\sk
Ensuite supposons simplement que cette énergie est convertie en énergie interne
(ne faisant qu'appliquer en cela le premier principe de la thermodynamique, puisque
l'énergie gravitationnelle~$E_p$ n'est autre que l'opposé du travail de la force gravitationnelle)
\[ M \, c_p \, \Delta T = E\e{p} \]
\noindent (où~$c_p$ est la capacité calorifique massique du matériau composant la planète)
ce qui fournit une estimation de l'augmentation de température interne de la planète
\[ \Delta T = \frac{3}{5} \, \frac{\mathcal{G}\,M}{R\,C_p} \]
\noindent Comme attendu, une contraction gravitationnelle 
(diminution du rayon planétaire~$R$ à masse totale~M constante)
entraîne une augmentation de la température interne du corps.

\sk
Une partie de cette énergie est perdue par radiation par la surface.
En fait, une large part est perdue pendant la formation de la planète,
phase pendant laquelle à la fois 
la conduction de chaleur dans la planète
et la radiation d'énergie vers l'espace
sont très efficaces (autrement dit, leur temps caractéristique
est initialement petit devant la durée de vie d'une planète).
Néanmoins, la poursuite de la contraction,
et la différentiation qui provoque la descente
des éléments lourds vers le centre de la planète,
contribue à réchauffer l'intérieur de la planète
après la formation.

\sk
La mesure du flux surfacique (en W~m$^{-2}$) 
rayonné par une planète peut mettre à jour, en retranchant 
le flux surfacique reçu du Soleil (équilibre TOA), une contribution 
du flux de chaleur interne provenant de la contraction
gravitationnelle initiale
\[ \frac{1}{4\,\pi\,R^2}\,\ddf{E\e{p}}{t} \]
\noindent Le flux de chaleur interne de Jupiter peut être
expliqué complètement par l'énergie interne accumulée lors de la phase
de contraction initiale, mais ce n'est pas le cas de Saturne
qui est plus ``brillante'' que ne semble indiquer son âge.
Un processus de différentiation pourrait expliquer ce phénomène
via un phénomène de pluie d'hélium (causé par l'immiscibilité
de ce dernier dans l'hydrogène) ; plus récemment, il a été
proposé que l'intérieur de Saturne refroidit plus lentement
à cause d'une convection en couches causée par les
gradients compositionnels.
Uranus a, comme les planètes telluriques,
quasiment perdu sa source de chaleur interne -- alors
que Neptune émet toujours une quantité importante
de chaleur qui peut être liée à une température d'accrétion initialement très élevée.









\end{document}
