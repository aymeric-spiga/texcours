\documentclass[a4paper,DIV16,10pt]{scrartcl}
%%%%%%%%%%%%%%%%%%%%%%%%%%%%%%%%%%%%%%%%%%%%%%%%%%%%%%%%%%%%%%%%%%%%%%%%%%%%%%%%%%%
\usepackage{texcours}
%%%%%%%%%%%%%%%%%%%%%%%%%%%%%%%%%%%%%%%%%%%%%%%%%%%%%%%%%%%%%%%%%%%%%%%%%%%%%%%%%%%
\newcommand{\zauthor}{Aymeric SPIGA}
\newcommand{\zaffil}{Laboratoire de Météorologie Dynamique}
\newcommand{\zemail}{aymeric.spiga@sorbonne-universite.fr}
\newcommand{\zcourse}{Atmosphères planétaires}
\newcommand{\zcode}{T02}
\newcommand{\zuniversity}{Sorbonne Université (Faculté des Sciences)}
\newcommand{\zlevel}{M2 Astronomie-Astrophysique}
\newcommand{\zsubtitle}{Fiches complémentaires du cours 1}
\newcommand{\zlogo}{\includegraphics[height=1.5cm]{/home/aspiga/images/logo/LOGO_SU_HORIZ_SIGNATURE_CMJN_JPEG.jpg}}
\newcommand{\zrights}{Copie et usage interdits sans autorisation explicite de l'auteur}
\newcommand{\zdate}{\today}
%%%%%%%%%%%%%%%%%%%%%%%%%%%%%%%%%%%%%%%%%%%%%%%%%%%%%%%%%%%%%%%%%%%%%%%%%%%%%%%%%%%
\begin{document} \inidoc
%%%%%%%%%%%%%%%%%%%%%%%%%%%%%%%%%%%%%%%%%%%%%%%%%%%%%%%%%%%%%%%%%%%%%%%%%%%%%%%%%%%

%% point de vue du cours: météorologie
%% approche physique et non phénoménologique !

\newpage \section{Parcelle} \sk
L'atmosphère est composée d'un ensemble de molécules. Pour la description de la plupart des phénomènes étudiés, le suivi des comportements individuels de chacunes des molécules composant l'atmosphère est impossible. On s'intéresse donc aux effets de comportement d'ensemble, ou moyen. Les principales variables thermodynamiques utilisées pour décrire l'atmosphère sont donc des grandeurs \voc{intensives} dont la valeur ne dépend pas du volume d'air considéré.
\begin{finger}
\item La \voc{température} $T$ est exprimée en K (kelvin) dans le système international. Elle est un paramètre macroscopique qui représente l'agitation thermique des molécules microscopiques. Les mesures de température usuelles font parfois référence à des quantités en \deg C, auxquelles il faut ajouter la valeur $273.15$ pour convertir en kelvins.
\item La \voc{pression} $P$ est exprimée en Pa dans le système international. La pression fait référence à une force par unité de surface ($1$~Pa correspond à l'unité~N~m$^{-2}$). Paramètre macroscopique, elle est reliée à la quantité de mouvement des molécules microscopiques qui subissent des chocs sur une surface donnée. Les mesures et raisonnements météorologiques font souvent référence à des quantités en hPa ou en mbar. Ces deux unités sont équivalentes : 1~hPa correspond à~$10^{2}$~Pa, 1~mbar correspond à $10^{-3}$~bar, ce qui correspond bien à 1~hPa, puisque 1 bar est $10^{5}$~Pa. La pression atmosphérique vaut $1013.25$~hPa (ou mbar) en moyenne au niveau de la mer sur Terre. On utilise parfois l'unité d'$1$~atm (atmosphère) qui correspond à cette valeur de~$101325$~Pa.
\item La \voc{masse volumique} ou densité~$\rho$ est exprimée en~kg~m$^{-3}$ dans le système international. Elle représente une quantité de matière par unité de volume. Elle vaut environ $1.217$~kg~m$^{-3}$ proche de la surface sur Terre.
\end{finger}
 \sk
L'atmosphère est composée d'un ensemble de molécules microscopiques et l'on s'intéresse aux effets de comportement d'ensemble, qualifiés de \voc{macroscopiques}. Les variables thermodynamiques utilisées pour décrire l'atmosphère (pression~$P$, température~$T$, densité~$\rho$) sont des grandeurs macroscopiques \voc{intensives} dont la valeur ne dépend pas du volume d'air considéré. 
%Une façon d'y parvenir est d'utiliser des grandeurs volumiques ou massiques.

\sk
Le système que l'on étudie est appelé \voc{parcelle d'air}. Il s'agit d'un volume d'air dont les dimensions sont %à la fois
\begin{citemize}
\item assez grandes pour contenir un grand nombre de molécules et pouvoir moyenner leur comportement, afin d'exprimer un équilibre thermodynamique~;
\item assez petites par rapport au phénomène considéré, afin de décrire le fluide atmosphérique de façon continue~; la parcelle d'air peut donc être considérée comme un volume élémentaire.
\end{citemize}
On peut donc supposer que les variables macroscopiques d'intérêt sont quasiment constantes à l'échelle de la parcelle. Autrement dit, une parcelle est caractérisée par sa pression~$P$, sa température~$T$, sa densité~$\rho$. Les limites d'une parcelle sont arbitraires, mais ne correspondent pas en général à des barrières physiques. 


 \sk
Une parcelle est en \voc{équilibre mécanique} avec son environnement, c'est-à-dire que la pression de la parcelle~$P\e{p}$ et la pression de l'environnement~$P\e{e}$ dans lequel elle se trouve sont égales
\[ \boxed{P\e{p} = P\e{e}} \]
Néanmoins, une parcelle n'est pas en général en \voc{équilibre thermique} avec son environnement, c'est-à-dire que la température de la parcelle et la température de l'environnement dans lequel elle se trouve ne sont pas nécessairement égales
\[ \boxed{T\e{p} \neq T\e{e}} \]
Cette dernière propriété provient du fait que l'air est un très bon isolant thermique\footnote{Une telle propriété est utilisée dans le principe du double vitrage}.


 %
\sk
Tout le but de ce chapitre est de décrire les relations thermodynamiques qui lient les grandeurs intensives qui caractérisent l'état de la parcelle. Une première de ces relations a été obtenue en introduction~: il s'agit de l'équation des gaz parfaits pour l'air atmosphérique, qui relie les trois paramètres intensifs $P$, $T$ et $\rho$ 
\[ \boxed{ P = \rho \, R \,T } \] 
avec la \voc{constante de l'air sec} $R=\frac{R^*}{M}$ où~$R^*$ est la constante des gaz parfaits et~$M$ est la masse molaire de l'air. On rappelle que sur Terre~$R = 287$~J~K$^{-1}$~kg$^{-1}$. L'état thermodynamique d'une parcelle d'air est donc déterminé uniquement par deux paramètres sur les trois~$P$, $T$, $\rho$. Pour les applications météorologiques, on caractérise en général l'état de la parcelle par sa pression~$P$ et sa température~$T$, plus aisées à mesurer, par exemple via des ballons-sondes, que la densité~$\rho$.



\newpage \section{Gaz parfait} \sk
On appelle \voc{gaz parfait} un gaz suffisament dilué pour que les interactions entre les molécules du gaz, autres que les chocs, soient négligeables. L'air composant l'atmosphère peut être considéré en bonne approximation comme un mélange de gaz parfaits\footnote{On peut en général considérer que le gaz est parfait si~$P < 1$~kbar. C'est le cas dans la plupart des atmosphères planétaires rencontrées. Il n'y a guère qu'au coeur des planètes géantes, où la pression dépasse cette limite, que l'approximation du gaz parfait doit être complètement abandonnée. Le domaine de validité de l'approximation du gaz parfait dépend tout de même fortement de la chimie du gaz considéré ; par exemple, dans l'atmosphère profonde de Vénus avec du CO$_2$ à 90 bars, des corrections de Van der Waals sont nécessaires.} notés~$i$, dont le nombre de moles est~$n_i$ pour un volume donné~$V$ d'air à la température~$T$. Chaque espèce gazeuse composant l'air est caractérisée par une \voc{pression partielle}~$P_i$ qui est définie comme la pression qu'aurait l'espèce gazeuse si elle occupait à elle seule le volume~$V$ à la température~$T$. Chacune de ces espèces gazeuses~$i$ se caractérise par une même température~$T$ et vérifie l'équation d'état du gaz parfait $$ P_i \, V = n_i \, R^* \, T $$ où~$R^*$=8.31 J~K\mo~mol\mo~est la constante des gaz parfaits (produit du nombre d'Avogadro et de la constante de Boltzmann). La pression totale de l'air~$P$ est, d'après la loi de Dalton, la somme des pressions partielles~$P_i$ des espèces gazeuses composant le mélange $P=\Sigma P_i$. En faisant la somme des lois du gaz parfait appliquées pour chacune des espèces gazeuses, on obtient $$ P \, V = \big( \Sigma n_i \big) \, R^* \, T $$ ce qui montre qu'un mélange de gaz parfaits est aussi un gaz parfait. Cette équation permet de relier la pression totale~$P$ à la température~$T$, mais présente l'inconvénient de contenir les grandeurs \voc{extensives}~$V$ et $n_i$ qui dépendent du volume d'air considéré. Il reste donc à donner une traduction intensive à la loi du gaz parfait pour un mélange de gaz. La masse totale contenue dans le volume~$V$ peut s'écrire $m=\Sigma n_iM_i$ où $M_i$ est la masse molaire du gaz $i$. En divisant l'équation précédente par $m$, et en utilisant la définition de la masse volumique~$\rho = m / V$, on obtient $$ \frac{P}{\rho} = \frac{\Sigma n_i}{\Sigma n_iM_i} \, R^* \, T $$ Or, d'une part, la \voc{masse molaire de l'air} composé d'un mélange de gaz~$i$ est $$ \boxed{ M=\frac{\Sigma n_iM_i}{\Sigma n_i} }$$ et d'autre part, on peut définir une \voc{constante de l'air sec} de la façon suivante $$ R=\frac{R^*}{M} $$ On a alors l'équation des gaz parfaits pour l'air atmosphérique qui permet de relier les trois paramètres intensifs importants : pression~$P$, température~$T$ et densité~$\rho$ $$ \boxed{\GP} $$ L'état thermodynamique d'un élément d'air est donc déterminé uniquement par deux paramètres sur les trois~$P$, $T$, $\rho$. En météorologie par exemple, on travaille principalement avec la pression et la température qui sont plus aisées à mesurer que la densité. Les valeurs numériques de~$M$, et donc~$R$, dépendent de la planète considérée et de sa composition atmosphérique. 



\newpage \section{Homosphère vs. hétérosphère} \sk
\begin{finger}
\item Dans toute discussion de la composition atmosphérique, il est important de faire la distinction entre composés minoritaires et majoritaires [figure~\ref{fig:minor}]. Alors que les composés majoritaires suivent une distribution verticale en accord avec l'état énergétique et dynamique de l'atmosphère globale, les composés minoritaires peuvent avoir des comportements très différents qui dépendent à la fois des mécanismes photochimiques de production et de perte ainsi que des phénomènes de transport.
\item La composition de l'air donnée ici est valide sur les premiers~$80$ à~$100$ kilomètres d'altitude, à part quelques constituants mineurs. On appelle cette région l'\voc{homosphère}, elle correspond approximativement à la troposphère, la stratosphère et la mésosphère (Figure \ref{fig:tempvert}). Dans l'homosphère, l'atmosphère est un mélange homogène de différents gaz. Au dessus de cette altitude, le libre parcours moyen des molécules devient très grand et on a une \ofg{décantation} où les éléments plus légers dominent progressivement aux altitudes élevées. On parle alors d'\voc{hétérosphère}; elle regroupe approximativement la thermosphère et l'exosphère. 
\end{finger}
%Au niveau du sol, l'atmosphère standard sèche est caractérisée par une pression d’environ 1013 hPa et une concentration totale de 2,69 x 1019 molécule~cm$^{-3}$ lorsque la température est de 273~K.

\figun{0.9}{0.4}{\figpayan/LP211_Chap1_Page_06_Image_0001.png}{Composition de l’atmosphère~: des espèces en très faibles quantités jouent un rôle très important. Sur la figure sont données quelques mesures de constituants minoritaires dans l’homosphère. Les courbes en trait fin correspondent aux concentrations résultant de rapports de mélange volumiques constants de~$10^{-1}$ à~$10^{-13}$. (Source: Kockarts, Aéronomie, 2000).}{fig:minor}



%%%%%%%%%%

\newpage \section{Equilibre hydrostatique} \sk
On considère une parcelle d'air cubique de dimensions élémentaires~$(\dd x,\dd y, \dd z)$, au repos et située à une altitude~$z$. La pression atmosphérique vaut~$P(z)$ sur la face du dessous et~$P(z+\dd z)$ sur la face du dessus. Pour le moment, on ne considère pas de variations de pression~$P$ selon l'horizontale\footnote{En pratique, les variations de pression selon l'horizontale sont effectivement négligeables par rapport aux variations de pression selon la verticale. On revient sur ce point dans le chapitre consacré à la dynamique}. Il y a équilibre des forces s'exerçant sur cette parcelle. On appelle \voc{équilibre hydrostatique} l'équilibre des forces selon la verticale, à savoir~:
\begin{citemize}
\item Son poids de module\footnote{On néglige les variations de~$g$ avec~$z$.}~$m \, g$ (où~$m = \rho \, \dd x \, \dd y \, \dd z$) dirigé vers le bas
\item La force de pression sur la face du dessous de module~$P(z) \, \dd x \, \dd y$ dirigée vers le haut
\item La force de pression sur la face du dessus de module~$P(z+\dd z) \, \dd x \, \dd y$ dirigée vers le bas
\item La force de viscosité qui est négligée
\end{citemize}
On note que, contrairement au poids qui s'applique de façon homogène sur tout le volume de la parcelle d'air, les forces de pression s'appliquent sur les surfaces frontières de la parcelle d'air. 
Pour une parcelle au repos, la résultante selon la verticale des forces de pression exercées par le fluide environnant (ici, l'air) n'est autre que la poussée d'Archimède.
%%Par ailleurs, l'équilibre hydrostatique suppose implicitement que la parcelle est à l'équilibre thermique avec son environnement~$T\e{p} = T\e{e}$ soit~$\rho\e{p} = \rho\e{e}$. On aborde le cas général où~$T\e{p} \neq T\e{e}$ dans le chapitre suivant pour définir les notions de stabilité.

\sk
L'équilibre hydrostatique de la parcelle s'écrit donc
\[ - \rho \, g \, \dd x \, \dd y \, \dd z + P(z) \, \dd x \, \dd y - P(z+\dd z) \, \dd x \, \dd y = 0 \qquad \Rightarrow \qquad - \rho \, g \, \dd z + P(z) - P(z+\dd z) = 0 \]
soit en introduisant la dérivée partielle suivant~$z$ de~$P$
\[ \frac{P(z+\dd z) - P(z)}{\dd z} = \boxed{ \Dp{P}{z} = - \rho \, g } \]
Cette relation est appelée \voc{équation hydrostatique} (ou relation de l'équilibre hydrostatique). Elle indique que, pour la parcelle considérée, la résultante des forces de pression selon la verticale équilibre la force de gravité. En principe, cette équation est valable pour une parcelle au repos. Par extension, elle est valable lorsque l'accélération verticale d'une parcelle est négligeable devant les autres forces. L'équation hydrostatique est en excellente approximation valable pour les mouvements atmosphériques de grande échelle. 

\sk
Si l'on intègre la relation hydrostatique entre deux niveaux~$z_1$ et~$z_2$ où la pression est~$P_1$ et~$P_2$, on obtient
\[ \Delta P = P_2 - P_1 = - g \, \int_{z_1}^{z_2} \rho \, \dd z \]
L'équilibre hydrostatique peut donc s'interpréter de la façon éclairante suivante~: la différence de pression entre deux niveaux verticaux est proportionnelle à la masse d'air (par unité de surface) entre ces niveaux. Une autre façon équivalente de formuler cela est de dire que la pression atmosphérique à une altitude~$z$ correspond au poids de la colonne d'air située au-dessus de l'altitude~$z$, exercé sur une surface unité de~$1$~m$^2$. Il s'ensuit que la pression atmosphérique~$P$ est décroissante selon l'altitude~$z$. Ainsi, la pression peut être utilisée pour repérer une position verticale à la place de l'altitude. En sciences de l'atmosphère, la pression atmosphérique est une coordonnée verticale plus naturelle que l'altitude~: non seulement elle est directement reliée à la masse atmosphérique par l'équilibre hydrostatique, mais elle est également plus aisée à mesurer.


\newpage \section{\'Echelle de hauteur} \sk
En exprimant la densité~$\rho$ en fonction de l'équation des gaz parfaits, l'équilibre hydrostatique s'écrit
\[ \Dp{P}{z} = - g \, \frac{P}{RT} \]
On peut intégrer cette équation si on suppose que l'on connaît les variations de~$T$ en fonction de $P$ ou $z$. On suppose ici que l'on peut négliger les variations de pression selon l'horizontale devant les variations suivant la verticale, donc transformer les dérivées partielles~$\partial$ en dérivées simples~$\dd$. On effectue ensuite une séparation des variables
\[R \, T \, \frac{\dd P}{P} = - g \, \dd z\]

\sk
Cette équation peut s'écrire sous une forme dimensionnelle simple à retenir
\[ \boxed{ \frac{\dd P}{P} = - \frac{\dd z}{H(z)} \qquad \text{avec} \qquad H(z) = \frac{R \, T(z)}{g} } \]
La grandeur~$H$ se dénomme l'\voc{échelle de hauteur} et dépend des variations de la température~$T$ avec l'altitude~$z$. L'équation ci-dessus indique bien que la pression décroît avec l'altitude selon une loi exponentielle comme observé en pratique. Cette loi peut être plus ou moins complexe selon la fonction~$T(z)$. On peut néanmoins fournir une illustration simple du résultat de l'intégration dans le cas d'une atmosphère isotherme~$T(z)=T_0$
\[ P(z) = P(z=0) \, e^{-z/H} \qquad \text{avec} \qquad H = R \, T_0 / g \]




\section{\'Equation hypsométrique} \sk
Dans l'équation de l'échelle de hauteur, faire l'hypothèse isotherme est très simpliste et rarement rencontré en pratique dans l'atmosphère. On se place dans le cas plus général, bien que toujours simplifié, de deux niveaux atmosphériques~$a$ et~$b$ entre lesquels la température ne varie pas trop brusquement avec l'altitude~$z$. On réalise alors l'intégration entre les deux niveaux~$a$ et~$b$
\[R \, T \, \frac{\dd P}{P} = - g \, \dd z \qquad \Rightarrow \qquad R \, \int_a^b T\, \frac{\dd P}{P} = - g \, \int_a^b dz\]
puis on définit la température moyenne de la couche atmosphérique entre~$a$ et~$b$ avec une moyenne pondérée
\[ \langle T \rangle = \frac{\int_a^b T \, \frac{\dd P}{P}}{\int_a^b \frac{\dd P}{P}} \]
pour obtenir finalement
\[R \, \langle T \rangle \, \int_a^b \frac{\dd P}{P} = - g \, \int_a^b dz
\qquad \Rightarrow \qquad \boxed{ g \, (z_a - z_b) = R \, \langle T \rangle \ln \left( \frac{P_b}{P_a} \right) } \]
Cette relation est appelée \voc{équation hypsométrique}. Elle correspond à une formulation utile en météorologie du principe que \ofg{l'air chaud se dilate}. Les conséquences de l'équation hypsométrique peuvent s'exprimer de diverses façons équivalentes.
\begin{citemize}
\item Pour une masse d'air donnée, une couche d'air chaud est plus épaisse.
\item La distance entre deux isobares est plus grande si l'air est chaud.
\item La pression diminue plus vite selon l'altitude dans une couche d'air froid.
\end{citemize}
En passant le résultat précédent au logarithme, on note que l'on retrouve toujours le fait que la pression diminue avec l'altitude selon une loi exponentielle. En notant l'échelle de hauteur moyenne~$\langle H \rangle$, on a
\[ P_b = P_a \, e^{ - \frac{z_b - z_a}{\langle H \rangle}} \qquad \text{avec} \qquad \langle H \rangle = \frac{R \, \langle T \rangle}{g} \]





%%%%%%%%%%

\newpage \section{Circulations thermiques directes} \sk
Toute différence de température entre deux régions (provoquée par exemple par un chauffage différentiel, ou par une différence des propriétés thermophysiques de la surface) est associée à des différences de pression, car d'après l'équation hypsométrique (équilibre hydrostatique + équation d'état du gaz parfait) la pression diminue plus vite avec l'altitude dans les couches d'air froid que dans les couches d'air chaud. Ceci donne naissance en altitude à un gradient de pression donc, en supposant que la force de pression est seule responsable de l'accélération du vent (vision à raffiner par la suite), des vents vont naître en altitude de la région chaude vers la régions froide. Ces vents induisent un flux de masse atmosphérique de la région chaude vers la région froide, donc causent, d'après l'équivalence entre pression et masse déduite de l'équilibre hydrostatique, une augmentation de la pression de surface dans la région froide par rapport à la région chaude. Ceci donne naissance proche de la surface à des vents de la région chaude vers la région froide. Par continuité, en considérant les convergences et divergences d'air proche du sol et en altitude, l'air s'élève dans les régions chaudes et redescend dans les régions froides.

\sk
Des exemples de circulations thermiques directes sont
\begin{finger}
\item les \voc{cellules de Hadley}, cellules fermées dans le plan méridien, sud-nord et verticale; sous les tropiques, l'air s'élève proche de l'équateur (suivant la saison, du côté de l'hémisphère d'été) et redescend au niveau des subtropiques.
\item les \voc{\og brises \fg~de mer et de terre} au bord de la mer sur Terre, naissant du contraste thermique entre continent et océan
\item les circulations atmosphériques sur Mars entre les régions polaires couvertes de glace et les régions de sol nu
\end{finger}

\sk
Les cellules fermées associées aux circulations thermiques directes \underline{ne sont pas des cellules de convection}. Elles résultent simplement de la déformation du champ de pression par des contrastes de température. Des cellules fermées non convectives peuvent également se développer dans le sens inverse de celui thermique direct (par exemple, les cellules de Ferrel sur Terre) : les mécanismes sont distincts des processus de circulation thermique directe et sont en général relatifs au forçage de l'écoulement moyen par les ondes atmosphériques résultant d'instabilités dans l'atmosphère.





%\newpage \section{Conduction} \sk
Les transferts thermiques par conduction se font par diffusion thermique. L'énergie est transférée via les collisions entre molécules. Ce type de transfert est dominant dans les intérieurs planétaires et dans les hautes atmosphères (thermosphère). Dans le dernier cas, le libre parcours moyen est si long que les atomes / molécules peuvent se mouvoir très rapidement d'une localisation à une autre, résultant en une conduction très efficace et un profil en général proche de l'isotherme.

\sk
Par analogie avec la diffusion moléculaire, pour définir la diffusion thermique, il s'agit de définir une loi phénoménologique (loi de Fourier, analogue de la loi de Fick) et une équation de conservation dans un volume de contrôle (conservation de l'énergie, analogue de la conservation de la matière). On définit ainsi pour le cas de la diffusion thermique uni-dimensionnelle selon~$x$
\begin{citemize}
\item \textit{Cause} inhomogénéité spatiale : Différence de température~$T(x,t)$
\item \textit{Conséquence} Densité de courant de chaleur~$\vec{J_Q}$ (W~m$^{-2}$)
\item \textit{\'Echange} Chaleur~$\delta Q = J_Q \, S \, \dd t$
\item \textit{Loi phénoménologique} Loi de Fourier~$J_Q = - \lambda\e{T} \, \Dp{T}{x}$
\item \textit{Conductivité thermique} en W~m$^{-1}$~K$^{-1}$~: $\lambda\e{T,roche} = 1-2$, $\lambda\e{T,eau} = 0.5$, $\lambda\e{T,air} = 0.02$.
\item \textit{Equation bilan locale} Conservation de l'énergie interne~$ \rho \, c_p \, \Dp{T}{t} + \Dp{J_Q}{x} = 0 $
\end{citemize}

\sk
Les équations tridimensionnelles sont
\[  
\textrm{Loi de Fourier} \quad \vec{J_Q} = - \lambda\e{T} \, \nabla T 
\qquad \qquad
\textrm{Conservation de l'énergie} \quad \rho \, c_p \, \Dp{T}{t} + \nabla \cdot \vec{J_Q} = 0
\]
\noindent L'équation de conservation de l'énergie n'est rien d'autre que le premier principe appliqué à un volume de contrôle~: la variation temporelle d'énergie interne est égale à la divergence du flux de chaleur (flux sortant moins flux entrant).

\sk
Combiner loi phénoménologique et équation de conservation permet d'obtenir ce qui est communément appelé l'équation de la chaleur, ou plus précisément l'équation de diffusion thermique
\[ \Dp{T}{t} = - D\e{T} \, \nabla^2 T \quad \textrm{[3D]} \qquad \qquad \Dp{T}{t} = - D\e{T} \, \DDp{T}{x} \quad \textrm{[1D]} \]
\noindent où~$D\e{T}$ est la diffusivité thermique notée
\[ D\e{T} = \frac{\lambda\e{T}}{\rho \, c_p} \]

\sk
Dans le cas unidimensionnel de la diffusion thermique dans un sol uniforme à la profondeur~$z$, en supposant un forçage périodique~$T(0,t) = T_0 + T_0' \, \cos \omega t$ ($\omega$ étant adapté au cas considéré selon si forçage diurne, saisonnier, \ldots), l'équation de diffusion thermique permet d'obtenir l'expression des variations spatiales et temporelles de~$T$
\[ T(x,t) = T_0 + T_0' \, e^{-\frac{z}{\delta}} \, \cos (\omega t - \frac{z}{\delta}) \]
\noindent où l'atténuation avec la profondeur est~$e^{-\frac{z}{\delta}}$ et le déphasage du maximum du forçage est~$\Delta t = \frac{z}{\omega \delta}$, avec~$\delta$ l'épaisseur de peau qui s'exprime
\[ \delta = \frac{2\,D\e{T}}{\omega} \]
\noindent En pratique, l'atténuation et le déphasage sont très marquées pour des profondeurs dans le sol même très modérées.








\section{Inertie thermique} L'inertie thermique $I$ mesure la résistance
thermique d'un milieu à un apport ou un
déficit de chaleur.
%
L'expression de $I$ (J~m$^{-2}$~s$^{-1/2}$~K$^{-1}$) 
s'obtient en déduisant d'une équation
simple de conduction thermique de Fourier,
par analyse dimensionnelle, l'épaisseur
de peau thermique $\delta$ 
%
\[
\rho \, c_p \, \Dp{T}{t} = \Dp{}{x} \left( \lambda \, \Dp{T}{x} \right)
\quad
\to
\quad
\delta = \sqrt{\f{\lambda \, \tau}{\rho \, c_p}}
\]
%
\noindent (où $\tau$ est une constante caractéristique de temps)
ce qui permet de mettre en évidence l'inertie
thermique dans le terme de flux de chaleur $\phi\e{c}$
à la surface
%
\[
\phi\e{c} = - \lambda \, \Dp{T}{x} = - \f{\lambda}{\delta} \, \Dp{T}{x'} 
= - \sqrt{\f{\lambda \, \rho \, c_p}{\tau}} \, \Dp{T}{x'}
\quad 
\textrm{avec}
\quad
x'=x/\delta 
\]
\noindent en ne retenant que les termes qui dépendent du milieu
dans la caractérisation du flux de chaleur :
%
%\[
$\boxed{
I = \sqrt{\lambda \, \rho \, c_p}
}$
%\]


Un milieu est donc de faible inertie thermique
lorsqu'il ne peut stocker que de petites quantités de chaleur
(faible capacité calorifique $c_p$)
et/ou qu'il ne peut transmettre cette chaleur que dans ses couches superficielles
(faible conductivité thermique $\lambda$).
%
Les océans terrestres constituent un exemple 
bien connu de milieu à très forte inertie thermique, 
de par leur grande capacité calorifique.
%
Autre exemple bien connu, 
l'inertie thermique des terrains rocheux
martiens est plus élevée que 
l'inertie thermique des terrains
poussiéreux, principalement
pour des raison de conductivité thermique.
%
L'inertie thermique
peut d'ailleurs permettre 
sous certaines conditions d'estimer
la taille des grains dans les sols
non consolidés.

Dépourvue d'océans, Mars forme
un gigantesque désert de faible inertie
thermique : $I$~dépasse 
rarement $400\U{J~m^{-2}~s^{-1/2}~K^{-1}}$
pour la plupart des sols martiens.
%
Pour qualifier les grands
ensembles sur le champ d'inertie
thermique planétaire,
le terme de \ofg{continents thermiques}
est parfois employé.
%
L'inertie thermique n'est pas une
quantité observable directement
et sa détermination requiert la combinaison
de mesures de température de surface et
d'un modèle simulant les variations thermiques du sol.


%%%%%%%%%%

\newpage \section{Premier principe de la thermodynamique} \sk
Un système thermodynamique possède, en plus de son énergie d'ensemble (cinétique, potentielle), une \voc{énergie interne}~$U$. Comme la température~$T$, l'énergie interne~$U$ est une grandeur macroscopique qui représente les phénomènes microscopiques au sein d'un gaz. Le premier principe indique que les variations d'énergie interne sont égales à la somme du travail et de
la chaleur algébriquement reçus~:
\[ \dd U = \delta W + \delta Q\]

\sk
Dans le cas d'un gaz parfait, l'énergie potentielle d'interaction des molécules du gaz est négligeable, et l'énergie interne est égale à l'énergie cinétique des molécules, qui dépend seulement de la température. On peut montrer que $U = n \, \frac{\zeta \, R^* \, T}{2}$ où $\zeta$ est le nombre de degrés de liberté des molécules. Pour un gaz (principalement) diatomique comme l'air, $\zeta = 5$. 

\sk
Dans le cas de variations quasi-statiques d'un gaz, ce qui est supposé être le cas dans l'atmosphère, le travail s'exprime en fonction de la pression~$P$ du gaz et de la variation de volume~$\dd V$
\[ \delta W = - P \, \dd V \]

\sk
L'expérience montre que la quantité de chaleur échangée au cours d'une transformation à volume ou pression constant est proportionnelle à la variation de température du système~: $\delta Q = n \, C_V^* \, \dd T$ à volume constant, $\delta Q = n \, C_P^* \, \dd T$ à pression constante. $C_P^*, C_V^*$ sont les \voc{chaleurs molaires}, également appelées \voc{capacités calorifiques}. Il s'agit de l'énergie qu'il faut fournir à un gaz pour faire augmenter sa température de~$1$~K dans les conditions indiquées (à volume constant ou à pression constante). Pour une transformation à volume constant (isochore), $\dd U = \delta Q$ donc $C_V^*=\frac{\zeta \, R^*}{2}$.

\sk
Pour l'étude de l'atmosphère, toujours dans la logique de travailler sur des grandeurs intensives, il est  bien plus utile de s'intéresser aux variations de pression plutôt qu'à celles de volume. On utilise donc l'\voc{enthalpie}~$H = U + P \, V$. On a alors par dérivation $ \dd H = \dd U + \dd (P\,V) $ puis, en utilisant le premier principe
\[ \dd H = V \, \dd P + \delta Q \]
Pour une transformation à pression constante (isobare) on a $\dd H = \delta Q$. On en déduit pour une transformation quelconque\footnote{
D'autre part, en utilisant conjointement la dérivation de l'équation d'état du gaz parfait~$\dd (P\,V) = n \, R^* \, \dd T$ et l'expression de l'énergie interne~$U = n \, C_V^* \, \dd T$, on obtient $\dd H = n \, C_V^* \, \dd T + n \, R^* \, \dd T$ pour une transformation quelconque. On en déduit la relation de Mayer \[ C_P^* = C_V^* + R^* = \frac{(\zeta+2) \, R^*}{2}\]
} 
que $\dd H = n \, C_P^* \, dT$, ce qui permet d'écrire
\[ n \, C_P^* \, dT = V \, \dd P + \delta Q \]





\newpage \section{Premier principe de la thermodynamique (application)} \sk
Afin de travailler sur des grandeurs intensives, on divise la relation précédente par la masse~$m$ de la parcelle pour obtenir
\[ \EE \]
où $\delta q$ est la chaleur massique reçue et $C_P = C_P^* / M$ est la \voc{chaleur massique de l'air} ($C_P$=1004 J~K$^{-1}$~kg$^{-1}$). Nous disposons alors d'une autre version du premier principe, très utile en météorologie et valable pour une transformation quelconque d'une parcelle d'air
\[ \boxed{ \underbrace{\textcolor{white}{\frac{R^2}{C_P}} \dd T \textcolor{white}{\frac{R}{C_P}}}_{\text{variation de température de la parcelle}} = \underbrace{\frac{R}{C_P} \, \frac{T}{P} \, \dd P}_{\text{travail expansion/compression}} + \underbrace{\frac{1}{C_P} \, \delta q}_{\text{chauffage diabatique}} } \]

\sk
Autrement dit, la température de la parcelle augmente si elle subit une compression ($\dd P > 0$) et/ou si on lui apporte de la chaleur ($\delta q > 0$). La température de la parcelle à l'inverse diminue si elle subit une détente ($\dd P < 0$) et/ou si elle cède de la chaleur à l'extérieur ($\delta q < 0$). Il est donc important de retenir que la température de la parcelle peut très bien varier quand bien même la parcelle n'échange aucune chaleur avec l'extérieur~: dans ce cas, $\delta q = 0$ et l'on parle de \voc{transformation adiabatique}. 

\sk
L'équation fondamentale ci-dessus est directement dérivée du premier principe, mais prend une forme plus pratique en sciences de l'atmosphère du fait que les transformations que subit une parcelle atmosphérique se réduisent en général aux transformations \voc{isobares} (à pression constante $\dd P = 0$) et aux transformations \voc{adiabatiques} (sans échanges de chaleur avec l'extérieur $\delta q = 0$). Les transformations isothermes, au cours de laquelle la température de la parcelle ne varie pas, sont plus rarement rencontrées en sciences de l'atmosphère.



\section{Transformations non adiabatiques} \sk
Dans le cas où la transformation n'est pas adiabatique, les échanges de chaleur~$\delta q$ d'une parcelle d'air avec son environnement sont non nuls et peuvent s'effectuer par~:
\begin{itemize}
\item Transfert radiatif~: l'atmosphère se refroidit en émettant dans l'infrarouge, ou se réchauffe en absorbant du rayonnement électromagnétique dans l'infrarouge [cas des gaz à effet de serre] ou dans le visible [cas de l'ozone dans la stratosphère].
%Ces échanges sont faibles et peuvent être négligés sauf à l'échelle de la circulation générale\footnote{Le refroidissement/réchauffement peut être localement élevé au sommet/à la base de nuages.}
\item Condensation ou évaporation d'eau~: le changement d'état consomme ou relâche de la chaleur (ceci n'a lieu que lorsque l'air est à saturation).
\item Diffusion moléculaire (conduction thermique)~: ces transferts sont très négligeables sauf à quelques centimètres du sol.
\end{itemize}
Un cas notamment souvent cité en météorologie est celui d'une parcelle d'air située proche du sol, à la tombée de la nuit, qui subit peu de variations de pression ($\dd P \sim 0$) mais dont la température diminue sous l'effet du refroidissement radiatif ($\delta q < 0$). Ceci explique la présence de rosée sur le sol et de brouillard proche de la surface au petit matin.




\newpage \section{Transformation adiabatique et température potentielle} \sk
Dans de nombreuses situations en sciences de l'atmosphère, on peut considérer que l'évolution de la parcelle est \voc{adiabatique} et se fait sans échange de chaleur avec l'extérieur ($\delta q=0$). En vertu de l'équilibre hydrostatique qui relie pression~$P$ et altitude~$z$~:
\begin{citemize}
\item une parcelle dont l'altitude~$z$ augmente sans apport extérieur de chaleur, subit une \voc{ascendance} adiabatique, donc une détente telle que~$\dd P < 0$ et sa température diminue ;
\item inversement, une parcelle dont l'altitude~$z$ diminue sans apport extérieur de chaleur, subit une \voc{subsidence} adiabatique, donc une compression telle que~$\dd P > 0$ et sa température augmente. 
\end{citemize}

\sk
Dans le cas où la transformation est adiabatique, pression et température sont intimement liées en vetu du premier principe. La version du premier principe encadrée ci-dessus avec~$\delta q = 0$ indique
\[ \dd T = \frac{R}{c_p} \, \frac{T}{P} \, \dd P \qquad \Rightarrow \qquad \frac{\dd T}{T} - \frac{R}{c_p} \, \frac{\dd P}{P} = 0 \]
soit par intégration
\[ T \, P^{- \kappa} = \text{constante} \qquad \text{avec} \qquad \kappa = R / c_p \]
Autrement dit, dans le cas où une parcelle subit une transformation adiabatique, sa température varie proportionnellement à~$P^{\kappa}$. Il s'agit d'une version, avec les grandeurs intensives utiles en sciences de l'atmosphère, de l'équation~$P\,V^{\gamma}$, avec $\gamma = c_p / c_v$, vue dans les cours de thermodynamique générale pour les transformations adiabatiques.

\sk
En se basant sur les considérations précédentes, il est possible de définir une quantité nommée \voc{température potentielle}~$\theta$ (en Kelvin) qui se conserve au cours de transformations adiabatiques
\[ \theta  = T \, \Pi^{-1} \qquad \textrm{avec} \qquad \Pi = \left( \f{P}{P_0} \right)^{R/c_p} 
\qquad \qquad
\f{\dd \theta}{\theta} = \f{\dd T}{T} - \f{R}{c_p} \f{\dd P}{P} = 0
\]
\noindent avec $\Pi$ la fonction adimensionnelle d'Exner et $p_0$ est une valeur de référence pour la pression (par exemple, $1000$~hPa pour la Terre). La température potentielle est donc égale à la température d'une parcelle ramenée de façon adiabatique à une pression~$P_0$. Cette quantité donne des informations fiables sur les échanges de chaleur d'une parcelle avec l'extérieur, contrairement à la température.

\section{Gradient adiabatique sec} \sk
D'après les seules équations thermodynamiques, on peut trouver une loi simple des variations de température avec l'altitude pour une parcelle qui ne subit que des transformations adiabatiques. Considérons le cas d'une parcelle subissant un déplacement vertical quasi-statique et adiabatique tel que~$\delta q = 0$. Elle vérifie en première approximation l'équilibre hydrostatique~$\dd P\e{p} / \rho = - g \, \dd z$. L'équation du premier principe modifiée pour le cas atmosphérique indique alors que
\[  \dd T\e{p}  = - \frac{g}{C_P} \, \dd z \]
d'où on tire le profil vertical adopté dans l'atmosphère sèche par une parcelle ne subissant pas d'échange de chaleur avec l'extérieur
\[  \boxed{ \ddf{T\e{p}}{z}  = \Gamma\e{sec} \qquad \text{avec} \qquad \Gamma\e{sec} = \frac{-g}{C_P} } \]
On note qu'il ne s'agit pas nécessairement du profil vertical suivi par l'environnement (voir section~\ref{parcenv}).

\sk
Le résultat trouvé ci-dessus revêt une importance particulière en sciences de l'atmosphère. La température d'une parcelle en ascension adiabatique décroît avec l'altitude selon un taux de variation constant, indépendamment des effets de pression. La constante~$\Gamma\e{sec}$ est appelée le \voc{gradient adiabatique sec} de température. Il n'est valable que pour une parcelle d'air non saturée en vapeur d'eau. Le calcul pour la Terre donne un refroidissement de l'ordre de~$10^{\circ}$C/km (ou K/km). 

\sk
Pourquoi cette valeur est-elle en désaccord avec la décroissance de~$6.5^{\circ}$C/km effectivement constatée dans l'atmosphère terrestre~? Cet écart est relatif aux processus humides qui ont une grande importance dans l'atmosphère terrestre.
%Le chapitre suivant apporte des éléments de réponse à ce paradoxe apparent.





\newpage \section{Remarque: chauffage adiabatique} \sk
Dans un point de vue lagrangien, on peut aisément déterminer
qu'un mouvement vertical ascendant ($w>0$) induit un chauffage adiabatique
et qu'un mouvement vertical descendant ($w<0$) induit un refroidissement adiabatique.
Il suffit de combiner l'équilibre hydrostatique,
ou plutôt sa variante, l'équation hypsométrique
\[ 
\f{\dd p}{p} = -\f{g \dd z}{R\,T} 
\qquad  
\Rightarrow
\qquad
\ddf{p}{t} = - \f{p}{R\,T} \, g \, w
\]
\noindent avec le premier principe dans le cas
adiabatique
\[
\ddf{\theta}{t} = 0
\]
\noindent pour obtenir
\[
\ddf{T}{t} = - \f{g}{c_p} \, w
\]


\section{Force de flottaison} \sk
Soit une parcelle dont la température $T\e{p}$ n'est pas égale à celle de l'environnement~$T\e{e}$, que ce soit sous l'effet d'un chauffage diabatique (par exemple~: chaleur latente, effets radiatifs) ou d'une compression / détente adiabatique. On reprend le calcul réalisé précédemment pour l'équilibre hydrostatique, avec la différence notable que l'on n'est plus dans le cas statique~: on étudie le mouvement vertical d'une parcelle. 

\sk
La somme des forces massiques s'exerçant sur la parcelle suivant la verticale est
\[ - g  - \frac{1}{\rho\e{p}}  \, \Dp{P\e{e}}{z} \]
où~$\rho\e{p}$ est la masse volumique de la parcelle. L'environnement est à l'équilibre hydrostatique donc
\[ \Dp{P\e{e}}{z} = - \rho\e{e} \, g \]
Ainsi la résultante~$F_z$ des forces massiques qui s'exercent sur la parcelle selon la verticale vaut
\[ F_z = g \, \left( \frac{\rho\e{e}}{\rho\e{p}} - 1 \right) = g \, \frac{\rho\e{e}-\rho\e{p}}{\rho\e{p}} \]
En utilisant l'équation du gaz parfait pour la parcelle~$\rho\e{p}=P/RT\e{p}$ et l'environnement~$\rho\e{e}=P/RT\e{e}$, on a
\[ \boxed{ F_z = g \, \frac{T\e{p}-T\e{e}}{T\e{e}} } \]
La résultante des forces est donc dirigée vers le haut, donc la parcelle s'élève, si la parcelle est plus chaude (donc moins dense) que son environnement. 
Elle est dirigée vers le bas si la parcelle est plus froide (donc plus dense) que son environnement.
En d'autres termes, on écrit ici la version météorologique de la force ascendante ou descendante 
provoquée par la poussée d'Archimède, également appelée \voc{force de flottaison}.


\section{Potentiel convectif et CAPE} \sk
Une quantité utile pour quantifier à l'ordre 0 l'amplitude des mouvements atmosphériques verticaux causés par l'instabilité convective est de considérer le travail de cette force de flottaison, que l'on appelle CAPE~$\mathcal{C}$ ou \voc{Convectively Available Potential Energy} et de calculer l'énergie cinétique verticale associée. Cela donne accès à une borne supérieure de la vitesse verticale~$w\e{max}$ atteinte dans l'ascendance car, en réalité, toute l'énergie potentielle n'est pas (loin de là) convertie en énergie cinétique.
\[ \mathcal{C} = \int_{\textrm{ascendance}}^{} g \, \frac{T\e{p}-T\e{e}}{T\e{e}} \, \dd z \]
\[ w\e{max} = \sqrt{2\,\mathcal{C}}  \]


\newpage \section{(In)stabilité} \sk
Ces considérations permettent de définir le concept de stabilité et instabilité verticale de l'atmosphère.
On considère l'atmosphère à un endroit donné de la planète, à une saison donnée, à une heure donnée de la journée.
On suppose que la température de l'environnement varie linéairement avec l'altitude
\[ \ddf{T\e{e}}{z} = \Gamma\e{env} \]
A une altitude~$z_0$ proche de la surface, la température de l'environnement est~$T\e{e}(z_0)=T_0$.

\sk
On considère une parcelle initialement à l'altitude~$z_0$ dont la température initiale~$T\e{p}(z_0)$ est également~$T_0$. On suppose que la parcelle subit une ascension verticale d'amplitude~$\delta z > 0$. Le profil de température suivi par la parcelle lors de son ascension est
\[ \ddf{T\e{p}}{z} = \Gamma\e{parcelle} \]
\begin{citemize}
\item Si la parcelle est non saturée, elle suit un profil adiabatique sec tel que $\Gamma\e{parcelle} = \Gamma\e{sec} \simeq - 10 \, \text{K/km}$.
\item Si elle est saturée, elle suit un profil adiabatique saturé tel que $\Gamma\e{parcelle} = \Gamma\e{saturé} \simeq - 6.5 \, \text{K/km}$. 
\end{citemize}
On rappelle qu'en général, à l'échelle où l'on étudie les mouvements de la parcelle
\[ \Gamma\e{parcelle} \neq \Gamma\e{env} \]

\sk
Quel est l'effet de la perturbation~$\delta z > 0$ sur le mouvement de la parcelle~? A l'altitude~$z_0 + \delta z$, les températures de la parcelle et de l'environnement sont respectivement
\[ T\e{p}(z_0 + \delta z) = T_0 + \Gamma\e{parcelle} \, \delta z 
\qquad \text{et} \qquad
T\e{e}(z_0 + \delta z) = T_0 + \Gamma\e{env} \, \delta z \]
\begin{finger}
\item Si $\Gamma\e{parcelle} > \Gamma\e{env}$, la température~$T\e{e}$ de l'environnement décroît plus vite que la température~$T\e{p}$ de la parcelle. Il en résulte que~$T\e{p}(z_0 + \delta z) > T\e{e}(z_0 + \delta z)$ et le mouvement de la parcelle est ascendant. La perturbation initiale est donc amplifiée par les forces de flottabilité. On parle de \voc{situation instable}. La situation est d'autant plus instable que la température de l'environnement décroît rapidement avec l'altitude. Lorsque la situation est instable, les mouvements verticaux sont amplifiés~: on parle parfois de \voc{situation convective}.
\item Si $\Gamma\e{parcelle} < \Gamma\e{env}$, la température~$T\e{e}$ de l'environnement décroît moins vite que la température~$T\e{p}$ de la parcelle. Il en résulte que~$T\e{p}(z_0 + \delta z) < T\e{e}(z_0 + \delta z)$ et le mouvement de la parcelle est descendant. La perturbation initiale n'est donc pas amplifiée et la parcelle revient à son état initial. On parle de \voc{situation stable}. La stabilité est d'autant plus grande que la température de l'environnement décroît lentement (ou augmente, dans le cas d'une inversion de température). Lorsque la situation est stable, les mouvements verticaux sont inhibés.
\end{finger}
La résultante des forces verticales s'exerçant sur la parcelle peut s'écrire en fonction des taux de variation~$\Gamma$ de la température
\[ F_z = g \, \frac{\Gamma\e{parcelle}-\Gamma\e{env}}{T\e{env}} \, \delta z \]
\noindent Un raisonnement similaire permet d'obtenir la fréquence de Brunt-V{\"a}is{\"a}l{\"a}.


\newpage \section{Oscillations de flottaison et fréquence de Brunt-V{\"a}is{\"a}l{\"a}} \sk
La force de flottaison peut s'écrire de diverses manières,
(p désignant la parcelle et e l'environnement)
\[ F_z = g \, \frac{\rho\e{e}-\rho\e{p}}{\rho\e{p}} = g \, \frac{T\e{p}-T\e{e}}{T\e{e}}  \]
\noindent ou encore
\[ F_z = g \, \frac{\Gamma\e{p}-\Gamma\e{e}}{T\e{e}} \, \delta z \]

\sk
Si l'on se place dans un contexte d'une perturbation d'altitude~$z'$
pour la parcelle
\[ T' = T_0 - \Gamma_d \, z' \]
alors que pour l'environnement
\[ T = T_0 - \Gamma \, z' \qquad \textrm{avec} \qquad \Gamma = \ddf{T}{z} \]
\noindent D'après les expressions du paragraphe précédent,
on peut ainsi écrire l'effet de la force ascensionnelle
sur l'accélération comme
\[ \ddf{^2 z'}{t^2} = g \left( \f{T'}{T} - 1 \right) \]
\noindent ou encore
\[ 
\ddf{^2 z'}{t^2} + \left[ \f{g}{T} \, \left( \ddf{T}{z} + \f{g}{c_p} \right)  \right] \, z' = 0
\]
\noindent Nous avons donc un système qui
peut causer des oscillations de la parcelle sous l'effet de la force de flottaison.
Par analogie avec l'équation du second ordre d'un
oscillateur harmonique, on définit le terme en
facteur de $z'$ comme une fréquence au carré, 
nommée fréquence de Brunt-V{\"a}is{\"a}l{\"a}~$N$
exprimée comme
\[ N^2 = \f{g}{T} \, \left( \ddf{T}{z} + \f{g}{c_p} \right) \]
\noindent ou encore en utilisant la température potentielle
\[ N^2 = \f{g}{\theta} \, \ddf{\theta}{z} = g \, \ddf{\ln\theta}{z} \]
\noindent La fréquence de Brunt-V{\"a}is{\"a}l{\"a}~$N$
traduit l'instabilité convective si~$N<0$
et la stabilité convective si~$N>0$ -- avec dans ce cas,
les oscillations de la parcelle comme mécanisme de rappel.

\sk
Nous voyons ici un cas très particulier d'un effet plus général appelé onde de gravité.


%%%%%%%%%%%%%%%%%%%%%%%%%%%%%%%%%%%%%%%%%ù

\newpage \section{Nuages, classification physique} \sk
La cause générale de la formation d'un nuage est le refroidissement d'une masse d'air. Dans la grande majorité des cas, les transformations qui provoquent ce refroidissement sont soit isobares, soit adiabatiques. Ceci est illustré par la formulation du premier principe adoptée dans les chapitre précédents.
%\[ \underbrace{\textcolor{white}{\frac{R^2}{C_P}} \dd T \textcolor{white}{\frac{R}{C_P}}}_{\text{variation de température de la parcelle}} = \underbrace{\frac{R}{C_P} \, \frac{T}{P} \, \dd P}_{\text{travail expansion/compression}} + \underbrace{\frac{1}{C_P} \, \delta q}_{\text{chauffage diabatique}} \]

\sk
Suivant la transformation qui va donner naissance au nuage, la morphologie de ce dernier est très différente. Par opposition à la classification phénoménologique donnée en début de chapitre, on peut alors établir une classification physique des nuages en fonction de la transformation thermodynamique qui leur donne naissance.
\begin{finger}
\item Si la transformation est isobare, cela signifie que le nuage se forme sans que la pression ne varie significativement dans la parcelle d'air considérée. D'après l'équilibre hydrostatique, et le fait que les variations de pression selon l'horizontale sont négligeables par rapport aux variations de pression selon la verticale, un tel phénomène ne peut exister que si l'altitude de la parcelle varie peu. La parcelle subit un refroidissement diabatique, qui peut être provoqué par exemple par les pertes radiatives (comme le brouillard nocturne cité au chapitre précédent) ou par un déplacement horizontal par les vents vers une région plus froide de l'atmosphère. Les nuages obtenus ne présentent pas de développement selon la verticale et sont même souvent particulièrement étendus selon l'horizontale. Il s'agit des nuages de type cirrus et stratus. Ces nuages étant organisés en strates, puisque non étendus selon la verticale, on dit qu'ils sont \voc{stratiformes}.
\item Si la transformation est adiabatique, le nuage se forme par le biais de parcelles d'air qui n'échangent pas de chaleur avec l'air environnant (ou, du moins, pour lesquelles les échanges diabatiques sont négligeables par rapport au terme de travail d'expansion/compression). Cette condition n'est réalisée que pour des parcelles subissant une variation de pression significative donc, selon l'équilibre hydrostatique, une variation d'altitude importante. Les nuages obtenus ne présentent donc pas d'étendue horizontale et sont au contraire très développés selon la verticale. Il s'agit des nuages de type cumulus et cumulonimbus. On les qualifie de nuages \voc{cumuliformes}.
\end{finger}
%%Ces nuages sont gouvernés par la convection~: la parcelle d'air plus chaude que son environnement a plus de facilité à refroidir en s'élevant qu'en échangeant de la chaleur avec son environnement.

\sk
La plupart des nuages se forment ainsi par refroidissement (isobare ou adiabatique) d'une masse d'air. Il est cependant à noter que le brassage d'une masse d'air chaude d'humidité voisine de~$100\%$ avec une masse d'air froide relativement sèche peut également donner naissance à des nuages de type stratiforme. %% avions. %% brise de terre.

%Quelques remarques sur les processus de formation: la plupart des nuages stratiforme se forment aussi au départ par refroidissement adiabatique -- même si le chauffage / refroidissement radiatif joue ensuite un rôle important dans le cycle de vie, en particulier dans les transitions type stratus -> strato-cumulus.
%Pour les nuages de couche limite, c'est le refroidissement lors du transport vertical turbulent ; pour les nuages type cirrus / altostratus / nimbi-stratus c'est le soulèvement en bloc de couches, dans les zones frontales par exemple. En fait, il n'y a guère que le brouillard radiatif ou d'advection qui soit vraiment d'origine adiabatique...
%Sinon, il y a un type un peu particulier qu'on appelle nuage de mélange, qu'on voit par exemple dans l'haleine qui condense en hiver. C'est ici le mélange de 2 masses d'air humides (mais non saturées) qui donne de l'air à une température intermédiaire mais saturé cette fois à cause de la forme de la courbe rsat(T).


\newpage \section{Grandeurs saturantes} %\sk
%\subsubsection{Définition}

\sk
La pression partielle~$e$ pour laquelle l'équilibre liquide-vapeur est atteint est appelée \voc{pression de vapeur saturante} que l'on note~$e\e{sat}$. Tant que~$e<e\e{sat}$, les échanges par évaporation dominent les échanges par condensation et $e$ augmente jusqu'à atteindre~$e\e{sat}$. Lorsque~$e=e\e{sat}$, la quantité de vapeur d'eau dans l'enceinte n'augmente plus\footnote{Il est important de noter que cet état stationnaire n'est pas dénué d'échanges entre les phases liquide et gaz par condensation et évaporation. Par analogie, on peut penser au remplissage d'une baignoire équipée d'un siphon~: le niveau de l'eau est constant à l'état stationnaire bien qu'il y ait en permanence un apport d'eau par le robinet et une perte d'eau par le siphon -- l'état stationnaire signifie juste que ces échanges se compensent.}. Ainsi, si l'on considère une enceinte avec de l'eau sous forme vapeur et liquide
\begin{citemize}
\item si la pression partielle de vapeur d'eau~$e$ dans l'enceinte est supérieure à la pression de vapeur saturante~$e\e{sat}$, il y a condensation jusqu'à ce que~$e=e\e{sat}$.
\item si la pression partielle de vapeur d'eau~$e$ dans l'enceinte est inférieure à la pression de vapeur saturante~$e\e{sat}$, il y a évaporation jusqu'à ce que~$e=e\e{sat}$.
\end{citemize}
Si l'on considère une enceinte contenant de l'eau sous forme vapeur uniquement, le premier point est toujours valable alors que le second point n'est pas vrai~: si la pression partielle de vapeur d'eau~$e$ dans l'enceinte est inférieure à la pression de vapeur saturante~$e\e{sat}$, rien ne se passe, car aucune phase liquide ne peut être évaporée. La pression partielle de vapeur d'eau~$e$ est donc toujours inférieure ou égale à la pression de vapeur saturante~$e\e{sat}$. 

\sk
Le rapport de mélange~$r\e{sat}$ correspondant à l'équilibre liquide/vapeur où $e=e\e{sat}$ est appelé \voc{rapport de mélange saturant}. D'après l'équation encadrée à la section précédente, on a 
\[ r\e{sat} \simeq 0.622 \, \frac{e\e{sat}}{P} \]
Les mêmes raisonnements qu'avec les pressions partielles~$e$ et~$e\e{sat}$ peuvent être faits avec les rapports de mélange~$r$ et~$r\e{sat}$. Ces quantités servent à définir l'\voc{humidité relative}~$H$
\[ \boxed{ H = \frac{e}{e\e{sat}} = \frac{r}{r\e{sat}} } \]
De ce qui précède, on déduit que l'humidité~$H$ est toujours inférieure à~$1$ ($100\%$) et que, lorsqu'il y a équilibre liquide/vapeur (\ofg{conditions saturées}), $H$ vaut~$1$ ($100\%$).

%\sk
%\subsubsection{Variation avec la température}

\sk
La pression de vapeur saturante~$e\e{sat}$ augmente exponentiellement avec la température~$T$ du gaz à l'équilibre liquide-vapeur d'après la \voc{relation de Clausius-Clapeyron}\footnote{La pression de vapeur saturante est proportionnelle à la probabilité de rupture d'une liaison, elle-même variant exponentiellement suivant la température.} 
\[ \ddf{P\e{sat}}{T} = \frac{L\,P\e{sat}}{R\,T^2} \qquad \Rightarrow \qquad P\e{sat}(T) = P_0 \, \exp{ \left[ -\frac{L}{R\,T} \right] } \]
%\noindent où~$\ell > 0$ est la chaleur latente de vaporisation.
\noindent où la chaleur latente de vaporisation est supposée constante avec la température pour simplifier. Ainsi elle double pour une élévation de température de~$10$~K. Plus le gaz dans l'enceinte est chaud, plus la quantité de vapeur d'eau au terme de l'expérience est élevée. En pratique, le terme~$e\e{sat}$, qui varie exponentiellement avec la température~$T$, domine très fréquemment les variations de pression~$P$. Ainsi, en bonne approximation, le rapport de mélange saturant~$r\e{sat}$ varie également exponentiellement avec la température~$T$. 

\sk
La dépendance de~$e\e{sat}$ avec~$T$ permet par ailleurs de définir la \voc{température de rosée}~$T\e{rosée}$ associée à une valeur donnée de la pression partielle~$e$ de l'eau. Il s'agit de la température~$T\e{rosée}$ à laquelle la pression partielle~$e$ devient saturante, c'est-à-dire qui vérifie
\[ \boxed{ e\e{sat}(T\e{rosée}) = e } \]

%\subsection{Ébullition} L'ébullition est un cas particulier: des bulles de gaz se forment à l'{\em intérieur} du liquide bouillant. Dans le cas de l'eau, ce gaz est donc de la vapeur d'eau. La pression dans ces bulles est égale à celle du liquide, soit à peu près la pression atmosphérique si le liquide est en contact avec l'air. Les bulles sont d'autre part stables si leur pression est supérieure à la pression saturante. L'ébullition se produit donc à une température $T_b$ telle que \[e_{sat}(T_b)=P_{atm}\]


\newpage \section{Déplacement d'équilibre et application aux gouttelettes nuageuses} \sk
On applique ici les raisonnements de la section précédente pour une interface plane liquide/vapeur à une goutte sphérique comme rencontrée dans les brouillards ou les nuages. La réalité est un peu plus complexe et fait intervenir les concepts de noyaux de condensation et de sursaturation, qui ne sont pas abordés dans ce cours. Les raisonnements présentés ci-dessous restent cependant valables au premier ordre.
%%Dans l'atmosphère, loin de la surface, il n'y a pas d'interface liquide/gaz permanente. Si $e<e_{sat}$, il n'y a ni condensation ni évaporation. Si $e$ devient supérieure à $e_{sat}$, il y a condensation sous forme de gouttes d'eau liquide (qui se forment plus vite qu'elles ne s'évaporent). Ces gouttes s'évaporent dès que $e<e_{sat}$

\sk
Soit une parcelle d'air à la température~$T_0$ et à la pression~$P$. Elle contient de la vapeur d'eau en équilibre avec des gouttelettes d'eau en suspension, en pratique cela correspond à une parcelle dans laquelle des gouttelettes nuageuses se sont formées. A l'équilibre liquide-vapeur, la pression partielle de vapeur d'eau dans la parcelle vaut~$e=e\e{sat}(T_0)$, le rapport de mélange de vapeur d'eau vaut~$r=r\e{sat}(T_0)$ et l'humidité~$H$ vaut~$100\%$. Si la température de la parcelle change, il y a déplacement de l'équilibre liquide/vapeur\footnote{Par abus de langage, on dit parfois que \ofg{l'air chaud peut contenir plus de vapeur d'eau que l'air froid}. Il est autorisé de garder cette phrase en tête en tant que moyen mnémotechnique, cependant elle est incorrecte physiquement car elle ne rend pas compte de l'équilibre liquide/vapeur.}.
\begin{finger}
\item
Supposons que l'on \underline{chauffe la parcelle} à une température~$T\e{c}>T_0$. Sa pression partielle en vapeur d'eau~$e$ est toujours proche de~$e\e{sat}(T_0)$, mais la pression de vapeur saturante~$e\e{sat}$ a augmenté de façon exponentielle de~$e\e{sat}(T_0)$ à~$e\e{sat}(T\e{c})$. On est alors dans la situation où~$e < e\e{sat}(T\e{c})$, donc~$H < 1$. Il y a alors évaporation d'eau liquide jusqu'à ce qu'un nouvel équilibre liquide/vapeur soit atteint, où~$e = e\e{sat}(T\e{c})$. Une façon équivalente de décrire ce déplacement d'équilibre est de dire que, lorsque la parcelle chauffe, la quantité de vapeur d'eau~$r=r\e{sat}(T_0)$ devient très inférieure à la quantité de vapeur d'eau à saturation~$r\e{sat}(T\e{c})$. De l'eau liquide doit passer sous forme gazeuse par évaporation pour compenser ce déséquilibre, de manière à ce que la quantité de vapeur d'eau~$r$ dans la parcelle augmente à~$r\e{sat}(T\e{c})$. Ainsi lorsque l'on chauffe la parcelle, des gouttelettes nuageuses disparaissent, le nuage se dissipe.
\item
Supposons à l'inverse que l'on \underline{refroidisse la parcelle} à une température~$T\e{f}<T_0$. La pression de vapeur saturante~$e\e{sat}$ a diminué de façon exponentielle de~$e\e{sat}(T_0)$ à~$e\e{sat}(T\e{f})$. On est alors dans la situation où~$e > e\e{sat}(T\e{f})$, donc~$H > 1$, qui est impossible. Il y a alors condensation d'eau liquide jusqu'à ce qu'un nouvel équilibre liquide/vapeur soit atteint, où~$e = e\e{sat}(T\e{f})$. Autrement dit, lorsque la parcelle refroidit, la quantité de vapeur d'eau~$r=r\e{sat}(T_0)$ devient très supérieure à la quantité de vapeur d'eau à saturation~$r\e{sat}(T\e{f})$. De l'eau sous forme gazeuse doit passer sous forme liquide par condensation pour compenser ce déséquilibre, de manière à ce que la quantité de vapeur d'eau~$r$ dans la parcelle diminue à~$r\e{sat}(T\e{f})$. Ainsi lorsque l'on refroidit la parcelle, de nouvelles gouttelettes nuageuses apparaissent, le nuage s'épaissit.
\end{finger}

\sk
Le second point s'applique également au cas d'une parcelle d'air ne contenant pas initialement de gouttelettes nuageuses. 

\sk
Le chapitre précédent a proposé une expression du premier principe qui distingue deux manières de faire varier la température d'une parcelle atmosphérique d'air sec~: transformations isobares et transformations adiabatiques. On peut désormais illustrer la formation de nuages associée à chacune des transformations appliquée à une parcelle de rapport de mélange en vapeur d'eau~$r \neq 0$ qui reste constant au cours de la transformation.
\begin{finger}
\item Lorsqu'une parcelle d'air proche de la surface subit un refroidissement isobare à la tombée de la nuit, sous l'influence du flux radiatif infrarouge, des gouttelettes nuageuses se forment car le rapport de mélange saturant~$r\e{sat}$ diminue jusqu'à devenir plus faible que~$r$. Il s'agit du brouillard nocturne~; la formation de rosée obéit à un principe similaire. La température de rosée~$T\e{rosée}$ peut ainsi être définie comme la température à laquelle la condensation se produit suite à un refroidissement isobare. 
\item Lorsqu'une parcelle d'air subit une élévation adiabatique, à cause par exemple de la présence d'une montagne, elle se refroidit et le rapport de mélange saturant~$r\e{sat}$ diminue. Le rapport de mélange~$r$ peut alors éventuellement devenir supérieur à~$r\e{sat}$ et des gouttelettes se forment pour que~$r=r\e{sat}$. Ceci explique par exemple que les montagnes soient souvent couvertes de nuages.
\end{finger}
Le chapitre suivant se propose de reprendre avec plus de précisions la formation des nuages.


\newpage \section{Transformations pseudo-adiabatiques} \sk
On considère tout d'abord une parcelle d'air (contenant de la vapeur d'eau) en évolution isobare. Le premier principe appliqué à la parcelle indique donc
\[ \dd T = \frac{1}{c_p} \, \delta q \]
Lors de l'évaporation, les molécules d'eau liquide voient les liaisons hydrogène avec leurs proches voisins être brisées. Le passage de l'eau de la phase liquide à la phase vapeur consomme donc de l'énergie\footnote{On peut s'en convaincre en notant la sensation de froid immédiate que provoque la sortie d'un bain à cause de l'évaporation de l'eau liquide sur le corps mouillé~; ou en se souvenant que lorsque l'on souffle sur la soupe pour la refroidir, c'est précisément pour favoriser l'évaporation et la refroidir efficacement.}~: pour l'air qui compose la parcelle, $\delta q < 0$ et il y a refroidissement. 
A l'inverse, lors de la condensation, les molécules d'eau sous forme gazeuse créent des liaisons hydrogène avec les molécules d'eau de la phase liquide pour atteindre un état énergétique plus faible. Le passage de l'eau de la phase vapeur à la phase liquide libère donc de l'énergie~: pour l'air qui compose la parcelle, $\delta q > 0$ et il y a chauffage.

\sk
L'énergie~$\delta q$ consommée ou libérée par les changements d'état s'appelle~\voc{chaleur latente}, on la note~$\delta q\e{latent}$. Si une masse de vapeur~$\dd m\e{vapeur d'eau}$ est condensée ou évaporée, on a
\[ \delta q\e{latent} = \frac{- L \, \dd m\e{vapeur d'eau}}{m\e{air sec}} \qquad \Rightarrow \qquad \boxed{ \delta q\e{latent} = - L \, \dd r } \]
où~$L$ est la chaleur latente massique en~J~kg$^{-1}$. La formule ci-dessus comporte un signe négatif. La quantité~$\delta q\e{latent}$ est positive lorsqu'il y a condensation (le rapport de mélange en vapeur d'eau diminue $\dd r < 0$) et négative lorsqu'il y a évaporation (le rapport de mélange en vapeur d'eau augmente $\dd r > 0$).

\sk
On considère désormais une parcelle d'air en évolution adiabatique, à l'exception des échanges de chaleur latente~: $\delta q = \delta q\e{latent}$. On appelle une telle transformation \voc{pseudo-adiabatique} ou encore \voc{adiabatique saturée}. On fait l'approximation que la chaleur latente consommée ou dégagée est seulement échangée avec l'air sec~:
\begin{citemize}
\item La chaleur latente consommée/dégagée n'est pas utilisée pour refroidir/chauffer les gouttes d'eau présentes.
\item On néglige les pertes de masse par précipitation~: la masse d'air sec considérée est constante.
\end{citemize}
Pour une telle transformation, la variation de température s'écrit ainsi
\[ \dd T = \frac{R}{c_p} \, \frac{T}{P} \, \dd P - \frac{L}{c_p} \, \dd r \]
\noindent ou encore
\[ c_p \, \dd T + g \, \dd z + L \, \dd r = 0 \]


\newpage \section{Gradient adiabatique humide} \sk
Considérons une parcelle en ascension adiabatique saturée (et non plus sèche comme dans la section~\ref{adiabsec}). Pour une parcelle saturée, c'est-à-dire à l'équilibre liquide/vapeur, l'équation qui précède peut s'écrire, en utilisant l'équilibre hydrostatique
\[ C_P \, \dd T + g \, \dd z + L \, \dd r = 0 \]
Or, puisque la parcelle est saturée, on a~$r = r\e{sat}(T)$ et on peut écrire $\dd r\e{sat} = \ddf{r\e{sat}}{T} \, \dd T$. On a alors
\[ \left( C_P + L \, \ddf{r\e{sat}}{T} \right) \dd T + g \, \dd z = 0\]
Cette expression est similaire au cas sec, à l'exception notable du terme supplémentaire~$L \, \ddf{r\e{sat}}{T}$ lié aux échanges latents. On peut alors obtenir le profil vertical adopté dans l'atmosphère saturée par une parcelle ne subissant pas d'échange de chaleur avec l'extérieur autre que les échanges de chaleur latente
\[  \ddf{T}{z}  = \Gamma\e{saturé} \qquad \text{avec} \qquad \Gamma\e{saturé} = \frac{-g}{C_P+L \, \ddf{r\e{sat}}{T} } \]
On a vu que $\ddf{r\e{sat}}{T}$ est toujours positif, on en déduit donc
\[ \boxed{ \Gamma\e{saturé} > \Gamma\e{sec} \qquad \text{ou} \qquad |\Gamma\e{saturé}| < |\Gamma\e{sec}| } \]
A cause du dégagement de chaleur latente, la température diminue moins vite pour une parcelle saturée en ascension que pour une parcelle non saturée. Le calcul pour l'atmosphère terrestre montre que
\[ \Gamma\e{saturé} = -6.5 \, \text{K~km}^{-1} \] 
ce qui correspond à la valeur observée dans la troposphère sur Terre. %[Figure~\ref{fig:tempvert}].

\sk
La constatation que~$\Gamma\e{saturé}$ correspond au profil d'environnement effectivement mesuré dans la troposphère appelle un commentaire important. Les profils verticaux secs ou saturés sont ceux suivis par une parcelle en ascension~: autrement dit, ils donnent les variations de~$T\e{p}$ avec l'altitude~$z$. D'un point de vue instantané, ils ne correspondent pas aux profils d'environnement~$T\e{e}$ tels qu'ils peuvent être par exemple mesurés par des ballons-sonde lâchés dans l'atmosphère. La parcelle n'est pas nécessairement à l'équilibre thermique avec l'environnement. On peut néanmoins constater sur la figure~\ref{fig:tempvert} que la température de l'environnement diminue avec une pente très proche de~$\Gamma\e{saturé}$. Ceci s'explique par le fait que cette figure montre une moyenne sur tout le globe à toutes les saisons. La situation moyenne ainsi décrite correspond aux mouvements d'une multitude de parcelles en ascension qui finissent par définir l'environnement atmosphérique\footnote{Ce phénomène porte le nom d'ajustement convectif.}. Pour comprendre la formation des nuages, et plus généralement les mouvements atmosphériques, il faut néanmoins se placer dans le cas local où l'équilibre thermique n'est pas vérifié. C'est l'objet de la section suivante.
%Comme pour le cas adiabatique, on peut aussi intégrer l'équation pour obtenir:
%\begin{equation} e_h=C_PT+gz+Lr=cste \label{estath} \end{equation}  
%La quantité $e_h$ est appelée {\em énergie statique humide} et est conservée
%pour des mouvements adiabatiques ($r$ et $e_s$ sont séparément conservés) ou
%saturés (pseudo-adiabatiques).


\newpage \section{Formation d'un cumulonimbus sur Terre} \sk
On s'intéresse ici au développement des nuages cumuliformes, en particulier les cumulonimbus. Le cas d'étude donné dans le radiosonsage exemple permet de suivre graphiquement les concepts de cette partie. Le point de départ est une parcelle d'air non saturée, c'est-à-dire dont l'humidité est inférieure à~$1$, située proche de la surface. On suppose que son rapport de mélange en vapeur d'eau~$r$ est conservé au cours de l'ascension. 

\sk
En premier lieu, de nombreux phénomènes atmosphériques vont provoquer une élévation de la parcelle que l'on considère initialement proche de la surface.
\begin{citemize}
\item{\textbf{Soulèvement frontal}} Un front est une variation marquée et localisée de température. Lorsqu'un front se déplace horizontalement, l'air chaud passe au-dessus de l'air froid de densité moindre. 
\item{\textbf{Soulèvement orographique}} La présence d'un relief face au vent force les parcelles d'air à s'élever.
\item{\textbf{Convection sèche}} Un sol très chaud l'après-midi peut induire un profil de température de l'environnement très instable proche de la surface. Dans ce cas, les mouvements verticaux sont amplifiés proche de la surface par la poussée d'Archimède (voir chapitre précédent).
%%% circulation thermique. brise de terre et brise de mer.
\end{citemize}
Tous ces mécanismes expliquent que des nuages cumuliformes sont souvent trouvés au-dessus de régions soumises au passage de fronts, montagneuses ou dont la surface est particulièrement chaude. Ces nuages évoluent parfois vers un état de type cumulonimbus.

\sk
En second lieu, lorsqu'une parcelle d'air est soulevée vers les plus hauts niveaux de l'atmosphère par les phénomènes atmosphériques précités, elle subit un refroidissement par détente adiabatique. Le taux de refroidissement de la parcelle est~$|\Gamma\e{sec}|$. Sur un émagramme tel celui de la figure~\ref{fig:sounding}, la parcelle suit une \voc{courbe adiabatique sèche}. Cette décroissance de la température de la parcelle au cours de l'ascension a pour principale conséquence d'abaisser la valeur de $r\e{sat}$, de par les variations exponentielles de cette quantité avec la température. Il en résulte que l'humidité relative~$H = r / r\e{sat}$ augmente. Si la quantité de vapeur d'eau initiale~$r$ et/ou le soulèvement de la parcelle sont suffisants, $H$ peut atteindre~$1$ au cours de l'ascension~: la parcelle devient alors saturée. Des gouttelettes nuageuses apparaissent par condensation, autrement dit un nuage se forme. Le niveau d'altitude ou de pression auquel la condensation se produit suite à un refroidissement par soulèvement adiabatique s'appelle le \voc{niveau de condensation} ou la \voc{base du nuage}. A ce stade, le nuage n'est pas encore nécessairement cumuliforme.
%%Pour la convection. Les bases des nuages sont horizontales, leurs sommets évoluent en fonction de la température.

\sk
En troisième lieu, si la parcelle continue son ascension au-delà du niveau de condensation, sa température ne décroît plus d'un taux~$|\Gamma\e{sec}|$, mais d'un taux $|\Gamma\e{saturé}|$ plus faible, puisque la parcelle est désormais saturée (son humidité vaut~$1$ et son rapport de mélange~$r$ vaut~$r\e{sat}$). Sur l'émagramme, la parcelle suit une \voc{courbe adiabatique saturée}. Au cours de l'ascension, la parcelle reste saturée mais, puisque sa température continue de diminuer, $r\e{sat}$ diminue de concours, ce qui induit une diminution du rapport de mélange en vapeur d'eau~$r$ et une augmentation du rapport de mélange en eau liquide (qui prend la forme de gouttelettes nuageuses ou, si les conditions de croissance sont réunies, de précipitations pluvieuses).

\sk
En quatrième lieu, la forme du profil de température d'environnement détermine si, une fois le niveau de condensation atteint, le nuage va suivre ou non un développement vertical marqué. On rappelle que le profil d'environnement n'est pas celui suivi par la parcelle considérée, mais représente l'état atmosphérique tel qu'il peut être mesuré par un ballon-sonde météorologique par exemple.

\begin{finger}

\item Si les soulèvements initiaux de la parcelle ne l'amènent que dans des niveaux atmosphériques où sa température reste plus faible que celle de l'environnement, alors il n'y a pas de mouvements verticaux spontanés au sein du nuage. Le nuage formé est plutôt de type stratiforme (ou faiblement cumuliforme).

\item Si les soulèvements initiaux de la parcelle parviennent à la hisser à des niveaux atmosphériques où sa température devient plus élevée que celle de l'environnement, alors des mouvements verticaux spontanés apparaissent au sein du nuage. On parle de convection humide (ou convection profonde). Le nuage ainsi formé est cumuliforme. Le niveau atmosphérique à partir duquel la température de la parcelle en ascension adiabatique devient plus élevée que la température de l'environnement s'appelle le \voc{niveau de convection libre}. Le niveau atmosphérique à partir duquel la température de la parcelle redevient plus faible que l'environnement s'appelle le \voc{sommet théorique du nuage}. Si le sommet théorique du nuage est très élevé, la formation de cumulonimbus, donc d'orage, est très probable. 

\end{finger}




\newpage \section{Formation d'un cumulonimbus sur Terre (compléments)} \figside{0.4}{0.2}{decouverte/cours_meteo/anvil.png}{Vue lointaine d'un cumulonimbus à un stade avancé de développement, où l'on peut observer la structure aplatie en forme d'enclume au sommet du nuage. Source~: Wallace and Hobbs, Atmospheric Science, 2006~; d'après une photographie du Bureau Australien de Météorologie.}{fig:enclume}

\sk
L'étude de la formation des cumulonimbus appelle deux remarques importantes qui illustrent les concepts de stabilité et instabilité atmosphérique.

\begin{finger}

\item Le sommet des cumulonimbus atteint très fréquemment la tropopause. Lorsque c'est le cas, ils prennent alors une apparence aplatie et la forme d'enclume comme présenté dans la figure~\ref{fig:enclume}. Cela provient du fait que la stratosphère voit la température de l'environnement augmenter avec l'altitude, contrairement à ce qui peut se passer dans la troposphère. Un tel profil de température est extrêmement stable, donc a tendance à inhiber les mouvements verticaux. Ainsi, le fort développement vertical des cumulonimbus est stoppé net lorsque les couches stables de la stratosphère sont atteintes. En conséquence, le nuage s'étale selon l'horizontale au voisinage de la tropopause. C'est la raison pour laquelle le sommet réel des nuages cumuliformes est le minimum du sommet théorique des nuages et de la hauteur de la tropopause. Plus généralement, des conditions stables peuvent conduire à la formation de nuages stratiformes, ce qui nuance un peu la distinction faite dans la section~\ref{classphys}.

\item On entrevoit par les développement précédents qu'il est possible qu'une couche atmosphérique donnée, dont la température suit le taux de décroissance~$|\Gamma\e{env}|$ selon l'altitude, apparaisse comme stable si l'on considère l'ascension d'une parcelle non saturée, mais instable si l'on considère l'ascension d'une parcelle saturée. Cette situation se présente lorsque 
\[ |\Gamma\e{saturée}| <  |\Gamma\e{env}|  < |\Gamma\e{sec}| \]
On parle alors d'\voc{instabilité conditionnelle}. Il s'agit de conditions où seule l'apparition d'un nuage peut donner lieu à une instabilité et au développement de mouvements verticaux potentiellement étendus. Dans ce cas de figure, les nuages qui se forment sont essentiellement cumuliformes.

\end{finger}




\newpage \section{Nuages (microphysique)} Un \voc{nuage} se définit comme un regroupement localisé de gouttelettes d'eau et/ou de cristaux de glace ou de neige en suspension dans l'atmosphère. 

\figside{0.5}{0.2}{decouverte/cours_meteo/gouttes.png}{Vue schématique des composants d'un nuage pluvieux chaud. Les tailles indicatives sont précisées afin de donner une idée des quelques ordres de grandeur en taille qui séparent une gouttelette nuageuse d'une goutte de pluie. La traduction des termes est la suivante: \emph{cloud-condensation nuclei} \donc~noyaux de condensation, \emph{moisture droplets} \donc~gouttelettes nuageuses, \emph{typical raindrop} \donc~goutte de pluie. Les diamètres sont indiqués en microns (10$^{-6}$~m). Source: à partir de MacDonald Adv. Geophys. 1958.}{fig:cloud}

\begin{finger}
\item Dans les nuages chauds, la vapeur d'eau condense en gouttelettes nuageuses dont la taille est de quelques microns. Les conditions locales de saturation (autrement dit, l'équilibre liquide-vapeur) déterminent le taux de condensation et de croissance des gouttes. Même si cela n'est pas l'objet du présent cours, il convient de noter que l'interface courbée des gouttes a un certain coût énergétique~: il est difficile de former des gouttes à moins d'atteindre une sursaturation très élevée (c'est-à-dire une humidité très supérieure à~$1$). La formation des gouttelettes nuageuses par condensation est par contre facilitée par la présence de \voc{noyaux de condensation} (par exemple, les poussières atmosphériques). Ensuite, les gouttelettes nuageuses peuvent, par collision ou coalescence, croître de plusieurs ordres de grandeurs en taille pour donner naissance à des gouttes de pluie de plusieurs millimètres de large qui donnent lieu à des précipitations. La figure~\ref{fig:cloud} donne un aperçu très schématique de l'intérieur d'un nuage pluvieux chaud.
\item Dans les nuages froids, i.e. ceux qui se trouvent dans une zone où la température est plus faible que 0 degrés celsius, règne un équilibre à trois phases (solide, liquide, gaz). Les gouttelettes nuageuses et cristaux de glace peuvent se former par condensation directe (des phénomènes de surfusion expliquent que les gouttelettes d'eau restent à l'état liquide). La neige se forme par agrégation de cristaux de glace. Par accrétion, plus précisément coalescence liquide sur glace, la grêle peut apparaître dans un nuage froid et exceptionnellement former des espèces précipitantes d'un diamètre important.
\end{finger}

\sk
La formation des gouttes et cristaux dans un nuage obéit à un ensemble de lois dites microphysiques dont la complexité dépasse le présent cours. Pour simplifier la description, on s'intéresse souvent aux nuages chauds uniquement. De plus, on se place dans le cas d'un équilibre thermodynamique liquide-vapeur pour une interface plane, pour laquelle la valeur maximum de l'humidité est~$1$ et suffit à déclencher la formation d'un nuage dans l'atmosphère.




\newpage \section{Nuages (microphysique 2)} \sk
La contribution énergétique d'une interface obéit à l'équation de Kelvin.

\sk
Energie libre de Gibbs (à minimiser). Nucléation particule rayon~$R$, premier terme est le travail nécessaire pour former l'interface et second terme l'échange d'énergie associé aux molécules de vapeur allant vers la phase condensée
\[ 
\Delta G = 
\underbrace{\textcolor{white}{-\frac{4}{3}} 4 \, \pi \, R^2 \, \Sigma \textcolor{white}{\ln \frac{e}{e\e{sat}}}}_{\text{création de l'interface}} 
\quad
\underbrace{- \frac{4}{3} \, \pi \, R^3 \, n \, k \, T \, \ln \frac{e}{e\e{sat}}}_{\text{changement d'état}}
\]

L'équation montre que l'énergie libre de Gibbs dépend de l'humidité de l'atmosphère entourant la surface de la particule. En dessous de la saturation, le logarithme est négatif ou zéro et $\Delta G$ est une fonction croissante avec le rayon, mais pour des valeurs en sursaturation $e>e\e{sat}$ le logarithme est négatif et la fonction de Gibbs a un maximum.

Formule de Kelvin pour un rayon critique~$R\e{c}$
\[
R\e{c} = \frac{2 \, \Sigma}{n\,k\,T\,\ln \frac{e}{e\e{sat}}}
\]
noter la forte variabilité en l'humidité relative

La stabilité de petites gouttes requiert des conditions fortement sur-saturées ($e \gg e\e{sat}$).


%%%%%%%%%%%%%%%%%%%%%%%%%%%%%%%%%%%%%%%%%ù

\end{document}


%\newpage
%\section{Processus humides (définitions)}
%\sk
L'eau est présente dans l'atmosphère sous trois phases différentes, de la moins à la plus ordonnée~: gazeuse (vapeur d'eau), liquide (fines gouttelettes en suspension formant les nuages, précipitations pluvieuses), solide (cristaux de glace dans les fins nuages de haute altitude, intempéries de type neige et grêle). On s'intéresse principalement aux phases liquide et gazeuse afin de préfigurer l'étude des nuages sur Terre. Des raisonnements similaires sont possibles, avec quelques subtilités, pour la phase solide afin de décrire les nuages formés de cristaux de glace lorsque la température de l'atmosphère est suffisamment basse.

\sk
\subsection{Quantification de la vapeur d'eau dans l'atmosphère}\label{rappmel}

\sk
Soit une parcelle contenant un mélange de gaz parfaits notés~$i$, dont un est la vapeur d'eau. On a défini la \voc{pression partielle}~$P_i$ et le \voc{rapport de mélange massique}~$r_i = \frac{m\e{gaz i}}{m\e{air}}$ dans le chapitre introductif. Ces deux quantités peuvent servir à définir la quantité de vapeur d'eau présente dans la parcelle d'air. Pour simplifier, on note
\[ P\e{vapeur d'eau} = e \qquad \text{et} \qquad r\e{vapeur d'eau} = \frac{m\e{vapeur d'eau}}{m\e{air}} = r \] 
%La quantité~$q$ est également appelée \voc{humidité spécifique}. 
Le rapport de mélange en vapeur d'eau~$r$ est conservé dans la parcelle si il n'y a pas de changement de phase.

\sk
La pression partielle de l'air sec est~$P - e$. Comme mentionné dans le chapitre d'introduction, la vapeur d'eau vérifie l'équation d'état des gaz parfaits tout comme l'air sec, mélange de gaz parfaits, d'où
\[  e \, V = \frac{m\e{vapeur d'eau}}{M\e{vapeur d'eau}} \, R^* \, T  \qquad \qquad \qquad (P-e) \, V = \frac{m\e{air sec}}{M\e{air sec}} \, R^* \, T  \]
On forme le rapport des deux expressions pour obtenir une expression en fonction de paramètres intensifs et ne dépendant pas de la température
\[ \frac{e}{P-e} = \frac{m\e{vapeur d'eau}}{m\e{air sec}} \, \frac{M\e{air sec}}{M\e{vapeur d'eau}} \]

\sk
L'expression ci-dessus peut être grandement simplifiée. L'eau est un composant minoritaire dans l'atmosphère terrestre~: l'ordre de grandeur de~$r$ est de l'ordre de~$0$ à~$20$~g~kg$^{-1}$. On a donc toujours~$r \ll 1$ et~$e \ll P$, soit~$P-e \simeq P$. Ainsi la masse d'air sec~$m\e{air sec}$ dans la parcelle est en très bonne approximation égale à la masse d'air~$m\e{air}$ dans la parcelle, ce qui vaut également pour la masse molaire. Le rapport de mélange en vapeur d'eau s'écrit alors~$r = \frac{m\e{vapeur d'eau}}{m\e{air}} \simeq \frac{m\e{vapeur d'eau}}{m\e{air sec}}$. L'expression ci-dessus se simplifie donc en
\[ r = \frac{M\e{vapeur d'eau}}{M\e{air}} \, \frac{e}{P} \qquad \Rightarrow \qquad \boxed{ r \simeq 0.622 \, \frac{e}{P} } \]
Cette équation signifie que, pour une pression~$P$ donnée, le rapport de mélange de vapeur d'eau~$r$ est en bonne approximation proportionnel à la pression partielle de vapeur d'eau~$e$.

\sk
%\subsection{Évaporation, Saturation}
\subsection{Equilibre liquide / vapeur}

\sk
L'\voc{évaporation} est l'échappement de molécules d'eau depuis une phase liquide vers une phase gazeuse. A l'interface liquide-gaz, sous l'effet de l'agitation thermique, certaines molécules d'eau dans le liquide vont voir les liaisons hydrogène rompues avec leurs plus proches voisins. L'échappement est ainsi plus facile pour des molécules ayant une énergie cinétique importante~: le taux d'évaporation~$\mathcal{E}$ à partir d'une surface dépend donc de la température de l'eau. 

\sk
La \voc{condensation} est le passage de molécules d'eau de la phase gazeuse à la phase liquide. A l'interface liquide-gaz, certaines molécules d'eau dans le gaz vont se lier à des molécules d'eau dans le liquide par le biais de liaisons hydrogène. Le taux de condensation~$\mathcal{C}$ dépend de la pression de la phase gazeuse, à savoir~$e$ dans le cas de la vapeur d'eau. 

\sk
Soit une enceinte remplie d'air totalement sec, c'est-à-dire qui ne contient aucune molécule d'eau sous forme vapeur. On introduit dans cette enceinte une quantité donnée d'eau liquide. Comme décrit ci-dessus, il va y avoir spontanément évaporation avec un taux d'évaporation~$\mathcal{E}$ (supposé constant) à la surface du liquide, d'autant plus que la température de l'eau est élevée. Des molécules d'eau s'échappent donc dans l'espace au-dessus du liquide et forment une phase gazeuse dont la pression partielle~$e$ augmente peu à peu. Des molécules de cette phase gazeuse subissent à leur tour un phénomène de condensation et repassent en phase liquide. Le taux de condensation~$\mathcal{C}$ est, au début de l'expérience, très petit devant~$\mathcal{E}$ car la pression partielle~$e$ est extrêmement faible. Puisque l'évaporation domine la condensation, le bilan est donc en faveur d'une augmentation des molécules sous forme gazeuse. Néanmoins, plus le nombre de molécules d'eau sous forme gazeuse augmente, plus la pression partielle~$e$ augmente, donc plus le taux de condensation~$\mathcal{C}$ augmente. Ce phénomène va continuer jusqu'à atteindre un équilibre stationnaire où les taux de condensation~$\mathcal{C}$ et~$\mathcal{E}$ se compensent. Cet équilibre est appelé \voc{équilibre liquide-vapeur}, on parle également souvent, par abus de langage, de \voc{\ofg{saturation}}.% ou de \voc{\ofg{conditions saturées}}. 
















%\newpage
%\section{{\'E}nergie interne}
%\sk
L'énergie gravitationnelle est accumulée au cours de la croissance de la planète
par accrétion de matériel la composant.
L'énergie potentielle gravitationnelle acquise par une sphère planétaire
de masse~$m(r)$ et de rayon~$r$ lorsqu'une mass infinitésimale $\dd m$
est ajoutée depuis une distance infinie
est
\[ \dd E\e{p} = - \mathcal{G} \, \frac{m(r) \, \dd m}{r}  \]

\sk
Supposons que le corps planétaire est formé en maintenant une densité constante~$\rho_0$
jusqu'à ce que son rayon soit~$R$ ; alors l'incrément infinitésimal de masse~$\dd m$
est~$\dd m = 4 \, \pi \, r^2 \, \rho_0 \, \dd r$, et~$E\e{p}$ peut être déterminée par
intégration
\[ E\e{p} = - \mathcal{G} \, \int_{0}^{R} \frac{m(r) \, \dd m}{r}  
= - 3 \, \mathcal{G} \, \left( \frac{4\,\pi}{3} \right)^2 \, \rho_0^2 \, \frac{R^5}{5}
= - \frac{3}{5} \, \frac{\mathcal{G}\,M^2}{R} \] 
\noindent Plus généralement, pour une distribution sphérique quelconque~$\rho(r)$,
une analyse dimensionnelle indique que~$E\e{p}$ est proportionnelle à~$-\mathcal{G}\,M^2/R$.
La formule simplifiée ci-dessus donne le bon résultat pour la Terre avec seulement~$10\%$ d'erreur.

\sk
Ensuite supposons simplement que cette énergie est convertie en énergie interne
(ne faisant qu'appliquer en cela le premier principe de la thermodynamique, puisque
l'énergie gravitationnelle~$E_p$ n'est autre que l'opposé du travail de la force gravitationnelle)
\[ M \, c_p \, \Delta T = E\e{p} \]
\noindent (où~$c_p$ est la capacité calorifique massique du matériau composant la planète)
ce qui fournit une estimation de l'augmentation de température interne de la planète
\[ \Delta T = \frac{3}{5} \, \frac{\mathcal{G}\,M}{R\,C_p} \]
\noindent Comme attendu, une contraction gravitationnelle 
(diminution du rayon planétaire~$R$ à masse totale~M constante)
entraîne une augmentation de la température interne du corps.

\sk
Une partie de cette énergie est perdue par radiation par la surface.
En fait, une large part est perdue pendant la formation de la planète,
phase pendant laquelle à la fois 
la conduction de chaleur dans la planète
et la radiation d'énergie vers l'espace
sont très efficaces (autrement dit, leur temps caractéristique
est initialement petit devant la durée de vie d'une planète).
Néanmoins, la poursuite de la contraction,
et la différentiation qui provoque la descente
des éléments lourds vers le centre de la planète,
contribue à réchauffer l'intérieur de la planète
après la formation.

\sk
La mesure du flux surfacique (en W~m$^{-2}$) 
rayonné par une planète peut mettre à jour, en retranchant 
le flux surfacique reçu du Soleil (équilibre TOA), une contribution 
du flux de chaleur interne provenant de la contraction
gravitationnelle initiale
\[ \frac{1}{4\,\pi\,R^2}\,\ddf{E\e{p}}{t} \]
\noindent Le flux de chaleur interne de Jupiter peut être
expliqué complètement par l'énergie interne accumulée lors de la phase
de contraction initiale, mais ce n'est pas le cas de Saturne
qui est plus ``brillante'' que ne semble indiquer son âge.
Un processus de différentiation pourrait expliquer ce phénomène
via un phénomène de pluie d'hélium (causé par l'immiscibilité
de ce dernier dans l'hydrogène) ; plus récemment, il a été
proposé que l'intérieur de Saturne refroidit plus lentement
à cause d'une convection en couches causée par les
gradients compositionnels.
Uranus a, comme les planètes telluriques,
quasiment perdu sa source de chaleur interne -- alors
que Neptune émet toujours une quantité importante
de chaleur qui peut être liée à une température d'accrétion initialement très élevée.










%%%% PIEGE FROID
%%%% slide 4 cours 3 Bezard








%%% dans les applications donner les GES responsables
%%% voir Bezard III-4


%\newpage
%\section{Couches atmosphériques sur Terre}
%\sk
Les variations verticales de température sont très différentes des variations de pression et de densité: la température décroît et augmente alternativement avec l'altitude%, de façon quasi-linéaire [figure \ref{fig:tempvert}], en restant comprise entre environ~$200$ et~$300$~K. 
Cette structure verticale de la température permet de diviser l'atmosphère en un certain nombre de couches aux propriétés différentes, dont les noms comportent le suffixe \emph{-sphère}. La limite supérieure d'une couche atmosphérique donnée porte un nom similaire, où le suffixe \emph{-sphère} est remplacé par le suffixe \emph{-pause}. Par exemple, la limite entre la troposphère et la stratosphère s'appelle la \voc{tropopause}. Les couches atmosphériques en partant de la surface vers l'espace sont détaillées ci-dessous.

\sk
\begin{description} 
\item[La \voc{troposphère}] \normalsize s'étend jusqu'à environ 11 km d'altitude et contient 80\% de la masse de l'atmosphère. La température y décroit en moyenne de 6.5\deg C par kilomètre (nous verrons pourquoi dans un chapitre ultérieur). La troposphère est une couche relativement bien mélangée sur la verticale (échelle de temps de quelques jours), sauf en certaines couches minces, appelées \voc{inversions}, où la température décroit peu ou même augmente avec l'altitude. La troposphère est la couche où ont lieu la plupart des phénomènes météorologiques acessibles à l'expérience humaine (par exemple, les nuages montrés en figure~\ref{fig:blue}). La partie inférieure de la troposphère contient la \voc{couche limite atmosphérique} située juste au dessus de la surface, d'épaisseur variant de quelques centaines de~m à 3 km et définie comme la partie de l'atmosphère influencée par la surface sur de courtes échelles de temps (typiquement un cycle diurne). La couche limite atmosphérique est le siège de mouvements turbulents organisés au cours de l'après-midi qui opèrent un mélange des espèces chimiques qui y sont émises. \normalsize
\item[La \voc{stratosphère}] \normalsize est située au dessus de la troposphère. L'altitude au-dessus du sol de la tropopause peut varier entre~$5$ et~$15$~km. Contrairement à la troposphère, la stratosphère contient très peu de vapeur d'eau (à cause des températures très basses rencontrées à la tropopause) mais la majorité de l'ozone~O$_3$. L'absorption par l'ozone du rayonnement solaire \voc{ultraviolet}, de longueur d'onde moindre que le rayonnement visible et plus énergétique, explique que la température dans la stratosphère est d'abord isotherme, puis augmente avec l'altitude jusqu'à un maximum à la stratopause. Cette structure verticale très stable inhibe fortement les mouvements verticaux, ce qui explique que la stratosphère soit organisée en couches horizontales (comme l'indique l'étymologie de son nom). Le temps de résidence de particules dans la stratosphère est très long à cause de l'absence de nuages et précipitations. \normalsize
\item[La \voc{mésosphère}] \normalsize voit sa température décroître selon la verticale. Contrairement à la troposphère, elle ne contient pas de vapeur d'eau et contrairement à la stratosphère, elle ne contient que peu d'ozone. Elle se situe sur Terre à des altitudes entre~$50$ et~$85$~km. La mésopause est souvent le point le plus froid de l'atmosphère terrestre, la température peut y atteindre des valeurs aussi basses que~$130$~K. \normalsize
\end{description}

\figside{0.45}{0.3}{\figfrancis/WH_stdatm}{Structure verticale idéalisée de la température correspondant au profil moyenné global annuel.}{fig:tempvert}

%\newpage
%\section{Couches atmosphériques sur Terre -- haute atmosphère}
%\sk
\begin{description} 
\item[La \voc{thermosphère}] s'étend jusque des altitudes très élevées (800 km) et voit sa température contrôlée par l'absorption du rayonnement solaire ultraviolet. La température dans la thermosphère varie souvent d'un facteur deux suivant l'activité solaire et l'alternance jour-nuit. Les aurores surviennent dans cette couche atmosphérique. Les missions spatiales \ofg{basse orbite} telles que la Station Spatiale Internationale sont localisées au milieu de la thermosphère. \normalsize
\item[L'\voc{exosphère}] est située au-dessus de la thermosphère à partir d'une altitude d'environ~$800$~km sur Terre. Il s'agit de la zone où l'atmosphère subit un \voc{échappement} : les molécules peuvent s'échapper vers l'espace sans que des chocs avec d'autres molécules ne les renvoient dans l'atmosphère. L'exosphère constitue la dernière zone de transition entre l'atmosphère et l'espace. \normalsize
\end{description}

\figside{0.45}{0.3}{\figpayan/LP211_Chap1_Page_05_Image_0001.png}{Structure verticale idéalisée de la température étendue aux hautes atmosphères.}{fig:tempvert}
%% Voir figure~\ref{fig:presvert} pour la distinction entre figure de gauche et figure de droite

\sk
D'autres couches atmosphériques sont définies non pas à partir de la température mais à partir des propriétés électriques de l'atmosphère terrestre. On fait référence ici au vent solaire, qui est un flux de particules chargées (ions et électrons) formant un plasma qui s’échappe en permanence du Soleil vers l’espace interplanétaire
\begin{description}
\item[L'ionosphère] Comme son étymologie l'indique, l’ionosphère est une région de notre haute atmosphère contenant des ions et des électrons formés par photo-ionisation des molécules neutres qui s’y trouvent. C’est le Soleil, et plus particulièrement ses rayonnements énergétiques ultraviolets et X, mais aussi les particules du vent solaire et le rayonnement cosmique, qui sont à l’origine de cette ionisation de la haute atmosphère. L’ionosphère se situe entre~$50$ et~$1000$ kilomètres d’altitude (elle s'étend donc de la mésosphère à la thermosphère). L’ionosphère est habituellement divisée horizontalement en différentes couches, baptisées D, E et F dans lesquelles l’ionisation croît avec l’altitude. Ces couches proviennent des différences de pénétration dans l’atmosphère des rayonnements solaires selon leur énergie.
\item[La magnétosphère] Notre planète génère son propre champ magnétique, un peu à la manière d’une dynamo. C’est la différence de vitesse entre la rotation de la planète et de son coeur liquide qui, par induction, génère ce champ magnétique. Ce champ magnétique protège la Terre des agressions extérieures comme les rayons cosmiques et les particules énergétiques du vent solaire. Cette zone protégée s'appelle magnétosphère. Elle démarre au dessus de l’ionosphère, à plusieurs milliers de kilomètres de la surface du sol, et s’étend jusqu’à 70 000 kilomètres environ du côté du Soleil. Du côté opposé, la queue de la magnétosphère s’étire sur plusieurs millions de kilomètres. Les contours de la magnétosphère évoluent continuellement sous l’action du vent solaire et de sa variabilité.
\end{description}
\normalsize



%% MANQUE QQCH SUR STRATOSPHERE
%% critère pour stratosphère



%% COURS 2
%\newpage

%%% paramètres planétaires

%\section{Structures thermiques}
%

Profil radiatif-convectif + profil radiatif + effet de la convection humide

%\figun{0.2}{0.1}{decouverte/pierrehumbert_pics/9780521865562c02_fig001.jpg}{R. Pierrehumbert, Principles of Planetary Climates, CUP, 2010}{fig:profearth}

\figun{0.65}{0.5}{decouverte/pierrehumbert_pics/9780521865562c02_fig002.jpg}{R. Pierrehumbert, Principles of Planetary Climates, CUP, 2010}{fig:profplanet}








%fiches/bilan_radiatif_Terre_geographique.tex

%%% schéma Jérémy sur origine des atmosphères --> après cours 4
%%% sources de matière etc... Bruno

%% COURS 3
%\newpage \section{Oscillations de flottaison et fréquence de Brunt-V{\"a}is{\"a}l{\"a}} \sk
La force de flottaison peut s'écrire de diverses manières,
(p désignant la parcelle et e l'environnement)
\[ F_z = g \, \frac{\rho\e{e}-\rho\e{p}}{\rho\e{p}} = g \, \frac{T\e{p}-T\e{e}}{T\e{e}}  \]
\noindent ou encore
\[ F_z = g \, \frac{\Gamma\e{p}-\Gamma\e{e}}{T\e{e}} \, \delta z \]

\sk
Si l'on se place dans un contexte d'une perturbation d'altitude~$z'$
pour la parcelle
\[ T' = T_0 - \Gamma_d \, z' \]
alors que pour l'environnement
\[ T = T_0 - \Gamma \, z' \qquad \textrm{avec} \qquad \Gamma = \ddf{T}{z} \]
\noindent D'après les expressions du paragraphe précédent,
on peut ainsi écrire l'effet de la force ascensionnelle
sur l'accélération comme
\[ \ddf{^2 z'}{t^2} = g \left( \f{T'}{T} - 1 \right) \]
\noindent ou encore
\[ 
\ddf{^2 z'}{t^2} + \left[ \f{g}{T} \, \left( \ddf{T}{z} + \f{g}{c_p} \right)  \right] \, z' = 0
\]
\noindent Nous avons donc un système qui
peut causer des oscillations de la parcelle sous l'effet de la force de flottaison.
Par analogie avec l'équation du second ordre d'un
oscillateur harmonique, on définit le terme en
facteur de $z'$ comme une fréquence au carré, 
nommée fréquence de Brunt-V{\"a}is{\"a}l{\"a}~$N$
exprimée comme
\[ N^2 = \f{g}{T} \, \left( \ddf{T}{z} + \f{g}{c_p} \right) \]
\noindent ou encore en utilisant la température potentielle
\[ N^2 = \f{g}{\theta} \, \ddf{\theta}{z} = g \, \ddf{\ln\theta}{z} \]
\noindent La fréquence de Brunt-V{\"a}is{\"a}l{\"a}~$N$
traduit l'instabilité convective si~$N<0$
et la stabilité convective si~$N>0$ -- avec dans ce cas,
les oscillations de la parcelle comme mécanisme de rappel.

\sk
Nous voyons ici un cas très particulier d'un effet plus général appelé onde de gravité.



%%% exo géantes à CO --> car équilibre thermochimique CH4 se transpose à CO
%%% du coup pas de stratosphère
%%% ---> faire le lien avec Catling et Robinson (même chose que la condition Bezard cours 3 slide 8
%%% --> lien pour exoplanètes



\end{document}


