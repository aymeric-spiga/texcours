\documentclass[a4paper,DIV16,10pt]{scrartcl}
%%%%%%%%%%%%%%%%%%%%%%%%%%%%%%%%%%%%%%%%%%%%%%%%%%%%%%%%%%%%%%%%%%%%%%%%%%%%%%%%%%%
\usepackage{texcours}
%%%%%%%%%%%%%%%%%%%%%%%%%%%%%%%%%%%%%%%%%%%%%%%%%%%%%%%%%%%%%%%%%%%%%%%%%%%%%%%%%%%
\newcommand{\zauthor}{Aymeric SPIGA}
\newcommand{\zaffil}{Laboratoire de Météorologie Dynamique}
\newcommand{\zemail}{aymeric.spiga@sorbonne-universite.fr}
\newcommand{\zcourse}{Atmosphères planétaires}
\newcommand{\zcode}{T02}
\newcommand{\zuniversity}{Sorbonne Université (Faculté des Sciences)}
\newcommand{\zlevel}{M2 Astronomie-Astrophysique}
\newcommand{\zsubtitle}{Fiches complémentaires du cours 2}
\newcommand{\zlogo}{\includegraphics[height=1.5cm]{/home/aspiga/images/logo/LOGO_SU_HORIZ_SIGNATURE_CMJN_JPEG.jpg}}
\newcommand{\zrights}{Copie et usage interdits sans autorisation explicite de l'auteur}
\newcommand{\zdate}{\today}
%%%%%%%%%%%%%%%%%%%%%%%%%%%%%%%%%%%%%%%%%%%%%%%%%%%%%%%%%%%%%%%%%%%%%%%%%%%%%%%%%%%
\begin{document} \inidoc
%%%%%%%%%%%%%%%%%%%%%%%%%%%%%%%%%%%%%%%%%%%%%%%%%%%%%%%%%%%%%%%%%%%%%%%%%%%%%%%%%%%

%%%%%%%%%%%%%%%%%%%%%%%%%%%%%%%%%%%%
\begin{detail}
\newpage
\section{Rayonnement thermique et corps noir}
\sk
Le Soleil qui se situe à une distance considérable dans le vide spatial nous procure une sensation de chaleur. De même, placer sa main sur le côté d'un radiateur en fonctionnement sans le toucher procure une sensation de chaleur instantanée qui ne peut être attribuée à un transfert convectif entre le radiateur et la main. Cet échange de chaleur est attribué au contraire à l'émission d'ondes électromagnétiques par la matière du fait de sa température; on parle d'émission de \voc{rayonnement thermique}. Tous les corps émettent du rayonnement thermique. La transmission de cette énergie entre une source et une cible ne nécessite pas la présence d'un milieu intermédiaire matériel. 
%Le but de cette section est d'en étudier les principales propriétés.

\sk
On appelle \voc{corps noir} un objet dont la surface est idéale et satisfait les trois conditions suivantes~:
\begin{description}
\item[émetteur parfait] un corps noir rayonne plus d’énergie radiative à chaque température et pour chaque longueur d’onde que n'importe quelle autre surface,
\item[absorbant parfait] un corps noir absorbe complètement le rayonnement incident selon toutes les directions de l'espace et toutes les longueurs d'onde,
\item[source lambertienne] un corps noir émet du rayonnement de façon isotrope
\end{description}

\sk
Un corps noir est à l'équilibre thermodynamique avec son environnement. On peut montrer qu'un tel corps émet du rayonnement qui dépend seulement de sa température et non de sa nature. La définition du corps noir, et les développements théoriques qui l'accompagnent, sont partis du constat, fait notamment par les céramistes, qu'un objet placé dans un four à haute température devient rouge en même temps que les parois du four quelle que soit sa taille, sa forme ou le matériau qui le compose. Un exemple de source utilisée pour étudier expérimentalement le modèle du corps noir consiste à construire une enceinte chauffée, totalement hermétique, et y percer un trou pour y mesurer le flux énergétique émis [figure~\ref{fig:four}]

\figside{0.35}{0.15}{\figwallace/Radiation/radiation_Page_10_Image_0001.png}{L'énergie entrant par une petite fente dans une enceinte subit des réflexions sur la paroi jusqu'à ce qu'elle soit absorbée. L'ouverture dans la paroi d'une enceinte chauffée apparaît comme une source de type corps noir. Un absorbant presque parfait est aussi un émetteur presque parfait. Ce type de four a été employé au début du XXe siècle pour évaluer expérimentalement les prédictions théoriques de Planck. Source~: Wallace and Hobbs, Atmospheric Science, 2006.}{fig:four}


\section{Loi de Planck}
\sk
L'émission de rayonnement par le corps noir est décrite par une luminance énergétique spectrale~$L_{\lambda}$, notée $B_\lambda$ dans ce qui suit\footnote{Correspond au nom anglais \emph{blackbody}}. La loi de variation de~$B_\lambda$ selon la température~$T$ est donnée par la \voc{loi de Planck}\footnote{La luminance spectrale $B_\nu$ est déterminée d'une façon similaire. La démonstration de la loi de Planck fait appel à des notions de quantification d'énergie et de thermodynamique statistique qui sont hors programme dans le cadre de ce cours.} $$ B_\lambda(T) = \frac{C_1 \, \lambda^{-5}}{\pi \, \left( e^{ C_2 / \lambda T}-1\right) } $$ où $C_1$ et $C_2$ sont des constantes. Comme le rayonnement du corps noir est isotrope, l'émittance spectrale du corps noir, obtenue par intégration sur toutes les directions de l'espace, vaut $ M_\lambda(T) = \pi \, B_\lambda(T) $. 

%\figun{0.5}{0.25}{\figfrancis/WH_BBrad}{Courbes de luminance spectrale d'un corps noir pour différentes températures. La courbe en pointillés indique la position du maximum en fonction de $T$.}{fig:BBrad} 
\figside{0.5}{0.25}{\figwallace/Radiation/radiation_Page_11_Image_0001.png}{Courbes de luminance spectrale d'un corps noir pour différentes températures. La courbe en pointillés indique la position du maximum en fonction de $T$. Source~: Wallace and Hobbs, Atmospheric Science, 2006.}{fig:BBrad} 

\sk
Les variations de la fonction~$B_\lambda$ sont illustrées sur la figure~\ref{fig:BBrad}. L'émission de rayonnement par le corps noir ne dépend que de la longueur d'onde~$\lambda$ et de la température~$T$ du corps. A une température donnée, le rayonnement émis est parfaitement déterminé pour chaque longueur d'onde; dans un domaine spectral particulier, le rayonnement émis ne dépend que de la température du corps noir.


\newpage
\section{Loi de Wien du corps noir}
\sk
On observe sur la figure \ref{fig:BBrad} que, lorsque~$T$ augmente, la maximum de la luminance spectrale~$B_\lambda$, appelé \voc{maximum d'émission}, se décale vers les longueurs d'onde courtes, c'est-à-dire correspond à des photons de plus en plus énergétiques. La loi exacte, appelée \voc{loi de déplacement de Wien}, s'obtient en dérivant $B_\lambda$ par rapport à $\lambda$, ce qui permet d'obtenir $$ \boxed{ \lambda\e{max} \, T = 2898 \quad (\mu\textrm{m~K}) } $$ où $\lambda_{max}$ est la longueur d'onde du maximum de luminance spectrale~$B_\lambda$. La longueur d'onde du maximum d'émission~$\lambda\e{max}$ est ainsi inversement proportionnelle à la température du corps émetteur. Une formulation alternative est que $\nu\e{max}$ est proportionelle à $T$.

\figun{0.6}{0.45}{/home/aymeric/Big_Data/BOOKS/pierrehumbert_pics/9780521865562c03_fig001.jpg}{Source~: R. Pierrehumbert, Principles of Planetary Climates, CUP, 2010.}{wvl} 


\newpage
\section{Domaine de l'énergie solaire}
\sk
Le Soleil peut être considéré en bonne approximation comme un corps noir car il absorbe tout le rayonnement incident. Sa \ofg{couleur} est dûe à du rayonnement émis et, plus précisément, correspond aux longueurs d'onde où le maximum de rayonnement est émis. D'après la loi de Wien, le Soleil, dont l'enveloppe externe a une température autour de~$6000$~K, a donc un maximum d'émission situé dans le visible à $\lambda\e{max} = 0.5 \mu$m, proche du maximum de sensibilité de l'oeil humain [figure~\ref{fig:BBmax} haut]. Au contraire, la surface terrestre, dont la température typique est d'environ~$288$~K, voit son maximum d'émission situé dans l'infrarouge vers 10~$\mu$m, alors que le rayonnement émis dans les longueurs d'ondes visible est négligeable [figure~\ref{fig:BBmax} bas]. Un raccourci usuel est donc de dire que \ofg{la Terre émet du rayonnement (thermique) dans l'infrarouge alors que le Soleil émet dans le visible}. En toute rigueur, cette affirmation ne parle que du voisinage du maximum d'émission, où la contribution au flux intégré selon toutes les longueurs d'onde est la plus significative. Il est ainsi plus exact de dire que, dans l'atmosphère, la région du spectre où~$\lambda$ est inférieure à environ 4~$\mu$m est dominée par le rayonnement d'origine solaire, alors qu'au-delà, le rayonnement est surtout d'origine terrestre. Il n’y a pratiquement pas de recouvrement entre la partie utile du spectre du rayonnement solaire et celui d’un corps de température ambiante; ce fait est d'une grande importance pour les phénomènes de type effet de serre, qui sont abordés plus loin dans ce cours. On désigne ainsi souvent le rayonnement d'origine solaire par le terme \voc{ondes courtes} et le rayonnement d'origine terrestre par le terme \voc{ondes longues}.

\figsup{0.65}{0.2}{decouverte/cours_meteo/6000K.jpg}{decouverte/cours_meteo/earth.jpg}{Courbes de luminance spectrale d'un corps noir pour différentes températures correspondant notamment au Soleil (haut) et à la Terre (bas). La quantité représentée ici est l'émittance spectrale~$M_\lambda = \pi \, B_\lambda$. Noter la différence d'indexation de l'abscisse et l'ordonnée sur les deux schémas. Le rayonnement thermique émis par la Terre est plusieurs ordres de grandeur moins énergétique que celui émis par le Soleil et le maximum d'émission se trouve à des longueurs d'onde plus grandes (infrarouge pour la Terre au lieu de visible pour le Soleil). Source : \url{http://hyperphysics.phy-astr.gsu.edu/hbase/bbrc.html}.}{fig:BBmax}

\newpage
\section{Loi de Stefan-Boltzmann du corps noir}
\sk
La \voc{loi de Stefan-Boltzmann}\footnote{Joseph Stefan met expérimentalement en évidence en 1879 la dépendance de l'émittance en puissance quatrième de la température. Ludwig Boltzmann, à qui l'on doit également des résultats fondamentaux sur l'entropie et l'atomisme, prouve en 1884 le résultat par des arguments théoriques.} donne la valeur de l'intégrale sur toutes les longueurs d'ondes et dans tout l'espace\footnote{On entend par là toutes les directions du demi-espace extérieur au corps considéré.} de la courbe du corps noir, décrite par la loi de Planck et illustrée par les figures \ref{fig:BBrad} et \ref{fig:BBmax}. Cette loi donne donc l'expression d'une densité de flux énergétique~$F$ ou plus spécifiquement, puisque le corps noir est une source de rayonnement, d'une émittance totale~$M$. Cette dernière s'obtient tout d'abord avec une intégration par rapport à~$\lambda$ de la luminance énergétique spectrale~$B_\lambda$ donnée par la loi de Planck, afin d'obtenir la luminance énergétique~$B$. On déduit ensuite l'émittance totale~$M$ en intégrant selon toutes les directions de l'espace; comme le rayonnement du corps noir est isotrope, $M$ s'obtient à partir de~$B$ simplement en multipliant par~$\pi$. La loi de Stefan-Boltzmann établit que le flux net surfacique~$M$ émis par un corps noir ne dépend que de sa température par une dépendance type loi de puissance $$ \boxed{ M\e{corps noir} = \sigma \, T^4 } $$ avec~$\sigma=5.67 \times 10^{-8} \textrm{~W~m}^{-2}\textrm{~K}^{-4}$ appelée constante de Stefan-Boltzmann. La loi de Stefan-Boltzmann, comme la loi de Planck dont elle dérive, stipule que l'émittance~$M$ d'un corps pouvant être considéré en bonne approximation comme un corps noir ne dépend que de sa température et non de sa nature. Cette loi indique par ailleurs que l'émittance~$M$ augmente très rapidement avec la température -- de par la puissance quatrième impliquée.

\section{Corps gris et émissivité}
\sk
Le corps noir est un modèle idéal d'absorbant qu'en pratique on ne rencontre pas dans la nature. Par exemple, le charbon noir est un absorbant parfait, mais seulement dans les longueurs d'onde visible. La plupart des objets ressemblent néanmoins au corps noir, au moins à certaines températures et pour certaines longueurs d'onde considérées en pratique. Dans le cas d'un corps qui n'est pas un absorbant parfait, on parle d'un \voc{corps gris}. A température égale, un corps gris n'émet pas autant qu'un corps noir dans les mêmes conditions. Pour évaluer l'énergie émise par un corps gris par comparaison à celle qu'émettrait le corps noir dans les mêmes conditions, on définit un coefficient appelé \voc{émissivité} $\epsilon_\lambda$ compris entre~$0$ et~$1$ et égal au rapport entre la luminance spectrale du corps~$L_\lambda$ et celle du corps noir~$B_\lambda$ $$ \epsilon_\lambda=\frac{L_\lambda}{B_\lambda(T)} $$ En toute généralité, l'émissivité~$\epsilon_{\lambda}$ d'une surface à une longueur d'onde~$\lambda$ dépend de ses propriétés physico-chimiques, de sa température et de la direction d'émission\footnote{Par exemple, les métaux, matériaux conducteurs de l'électricité, ont une émissivité faible (sauf dans les directions rasantes) qui croît lentement avec la température et décroît avec la longueur d'onde ; au contraire, les diélectriques, matériaux isolant de l'électricité, ont une émissivité élevée qui augmente avec la longueur d'onde et se révèlent lambertiens sauf pour les directions rasantes où l'émissivité décroît significativement.}.

\sk
On peut également définir une émissivité totale intégrée~$\epsilon$ qui permet d'exprimer l'émittance~$M$ d'un corps gris $$ \boxed{\SB} $$ Des valeurs de l'émissivité totale~$\epsilon$ pour certains matériaux sont données dans le tableau~\ref{tab:emiss}~: l'eau, la neige, les roches basaltiques ont des émissivités proches de~$1$ et peuvent être considérées comme des corps noirs en bonne approximation. 

\begin{table}\label{tab:emiss}
\begin{center}
\begin{tabular}{|c|c|}
\hline
Matériau & Emissivité~$\epsilon$ \\
\hline
Aluminium & 0.02 \\
Cuivre poli & 0.03 \\
Nuages type cirrus & 0.10 à 0.90 \\
Nuages type cumulus & 0.25 à 0.99 \\
Cuivre oxydé & 0.5 \\
Béton & 0.7 à 0.9 \\
Carbone & 0.8 \\
Lave (volcan actif) & 0.8 \\
Neige âgée & 0.8 \\
Ville & 0.85 \\
Désert & 0.85 à 0.9 \\
Peinture blanche & 0.87 \\
Brique rouge & 0.9 \\
Herbe & 0.9 à 0.95 \\
Eau & 0.92 à 0.97 \\
Peinture noire & 0.94 \\
Forêt & 0.95 \\
Suie & 0.95 \\
Neige fraîche & 0.99 \\
\hline
\end{tabular}
\caption{\emph{Quelques valeurs usuelles d'émissivité à la température ambiante (pour un rayonnement infrarouge). Source~: Hecht, Physique, 1999 -- avec quelques ajouts d'après site CNES}}
\end{center}
\end{table}

\end{detail}
%%%%%%%%%%%%%%%%%%%%%%%%%%%%%%%%%%%%

\newpage
\section{Constante solaire}
\sk
La distance Soleil-Terre est beaucoup plus grande que les rayons de la Terre et du Soleil. Ainsi, d'une part, le rayonnement solaire arrive au niveau de l'orbite terrestre en faisceaux pratiquement parallèles. D'autre part, la luminance en différents points de la Terre ne varie pas. On peut par conséquent définir une valeur moyenne de la densité de flux énergétique du rayonnement solaire au niveau de l'orbite terrestre, reçue par le système surface~+~atmosphère. Elle est désignée par le terme de \voc{constante solaire} notée~$\mathcal{F}\e{s}$. Les mesures indiquent que
\[ \mathcal{F}\e{s} = 1368 \text{~W~m}^{-2} \qquad \text{pour la Terre} \]

\sk
La constante solaire est une valeur instantanée côté jour~: le rayonnement solaire reçu au sommet de l'atmosphère en un point donné de l'orbite varie en fonction de l'heure de la journée et de la saison considérée (c'est-à-dire la position de la Terre au cours de sa révolution annuelle autour du Soleil)\footnote{En réalité, la constante solaire~$\mathcal{F}\e{s}$ varie elle-même d'environ~$3$~W~m$^{-2}$ en fonction des saisons à cause de l'excentricité de l'orbite terrestre, qui n'est pas exactement circulaire. De plus, elle peut varier évidemment en fonction des cycles solaires, néanmoins sans influence majeure sur la température des basses couches atmosphériques (troposphère et stratosphère).}. On peut donc définir un \voc{éclairement solaire moyen} noté~$\mathcal{F}\e{s}'$ reçu par la Terre qui intègre les effets diurnes et saisonniers. Autrement dit, $\mathcal{F}\e{s}$~est l'éclairement instantané reçu par un satellite en orbite autour de la Terre~; $\mathcal{F}\e{s}'$ est la valeur que l'on obtiendrait si l'on faisait la moyenne d'un grand nombre de mesures instantanées du satellite à diverses heures et saisons. 

\figside{0.5}{0.2}{decouverte/cours_dyn/incoming.png}{Energie reçue du Soleil par le système Terre. Source~: McBride and Gilmour, \emph{An Introduction to the Solar System}, CUP 2004.}{fig:eqrad}

\sk
On admet ici que~$\mathcal{F}\e{s}'$ peut être calculé en considérant que le flux total reçu du Soleil l'est à travers un disque de rayon le rayon~$R$ de la Terre (il s'agit de l'ombre projetée de la planète, voir Figure~\ref{fig:eqrad}). A cause de l'incidence parallèle, le flux énergétique intercepté par la Terre vaut donc~$\Phi = \pi \, R^2 \, \mathcal{F}\e{s}$. L'éclairement moyen à la surface de la Terre est alors $$\mathcal{F}\e{s}' = \frac{\Phi}{4 \, \pi \, R^2}$$ le dénominateur étant l'aire de la surface complète de la Terre. On obtient ainsi
\[ \boxed{ \mathcal{F}\e{s}' = \frac{\mathcal{F}\e{s}}{4} } \]

\sk
La valeur de la constante solaire peut s'obtenir par le calcul. Le soleil est considéré en bonne approximation comme un corps noir de température~$T_{\sun} = 5780$~K. D'après la loi de Stefan-Boltzmann, son émittance est $M = \sigma \, T_{\sun}^4$ donc le flux énergétique~$\Phi_{\sun}$ émis par le Soleil de rayon~$R_{\sun} = 7 \times 10^5$~km est~$\Phi_{\sun} = 4 \, \pi \, R_{\sun}^2 \, \sigma \, T_{\sun}^4$. Ce flux énergétique est rayonné dans tout l'espace~: à une distance~$d$ du soleil il est réparti sur une sphère de centre le soleil et de rayon~$d$, donc de surface~$4 \, \pi \, d^2$. A cette distance, l'éclairement~$\mathcal{F}$, c'est-à-dire la densité de flux énergétique reçue en W~m$^{-2}$, s'écrit donc
\[ \mathcal{F} = \frac{\Phi_{\sun}}{4 \, \pi \, d^2} = \frac{4 \, \pi \, R_{\sun}^2 \, \sigma \, T_{\sun}^4}{4 \, \pi \, d^2} = \sigma \, T_{\sun}^4 \, \left( \frac{R_{\sun}}{d} \right)^2 \]
Si l'on prend~$d$ égal à la distance Terre-Soleil, $\mathcal{F}$ définit ainsi la constante solaire~$\mathcal{F}\e{s}$.
%\[ \mathcal{F}\e{s} = \frac{{\mathcal{F}\e{s}}^{\text{Terre}}}{d\e{soleil}^2} \]

%Variation de la constante solaire : Bien que l’intensité du soleil ait subit des variations depuis la formation de la Terre, on peut s’attendre à ce qu’elle soit stable sur une période de 1000 ans. On mesure mal la constante solaire, mais les mesures récentes, même avec leurs incertitudes, semblent indiquer que le soleil ne peut pas expliquer le réchauffement récent. Notons toutefois que les simulations actuelles ne tiennent pas compte des fluctuations possibles du rayonnement solaire (négligeable a priori).
%%%% pas sûr du dernier point.


\newpage 
\section{Albédo} 
\sk
En sciences de l'atmosphère, les coefficients de réflexion~$\rho$ et~$\rho_{\lambda}$ sont souvent désignés sous le nom respectivement d'\voc{albédo} noté~$A$ et d'albédo spectral noté~$A_{\lambda}$. Plus la surface réfléchit une grande partie du rayonnement électromagnétique incident, plus l'albédo est proche de~$1$. L'albédo spectral~$A_{\lambda}$ peut varier significativement en fonction de la longueur d'onde : voir l'exemple de la neige fraîche donné ci-dessus. 

\sk
De par la diversité des surfaces terrestres, et de la variabilité de la couverture nuageuse, les valeurs de l'albédo~$A$ varient fortement d'un point à l'autre du globe terrestre~: il est élevé pour de la neige fraîche et faible pour de la végétation et des roches sombres [table~\ref{tab:albedo}]. L'albédo de l'océan est faible, particulièrement pour des angles d'incidence rasants -- il dépend ainsi beaucoup de la distribution des vagues. 

\begin{table}\label{tab:albedo}
\begin{center}
\begin{tabular}{|c|c|c|c|}
\hline
Type & albédo~$A$ & Type & albédo~$A$ \\
\hline
Surface de lac & 0.02 à 0.04 & Surface de la mer & 0.05 à 0.15 \\
Asphalte & 0.07 & Mer calme (soleil au zenith) & 0.10 \\
Forêt équatoriale & 0.10 & Roches sombres, humus & 0.10 à 0.15 \\
Ville & 0.10 à 0.30 & Forêt de conifères & 0.12 \\
Cultures & 0.15 à 0.25 & Végétation basse, verte & 0.17 \\
Béton & 0.20 & Sable mouillé & 0.25 \\
Végétation sèche & 0.25 & Sable léger et sec & 0.25 à 0.45 \\
Forêt avec neige au sol & 0.25 & Glace & 0.30 à 0.40 \\
Neige tassée & 0.40 à 0.70 & Sommet de certains nuages & 0.70 \\
Neige fraîche & 0.75 à 0.95 & & \\
\hline
\end{tabular}
\caption{\emph{Quelques valeurs usuelles d'albédo (rayonnement visible). D'après mesures missions NASA et ESA.}}
\end{center}
\end{table}

\sk
L'\voc{albédo planétaire} est noté~$A\e{b}$ et défini comme la fraction moyenne de l'éclairement~$E$ au sommet de l'atmosphère (noté également~$\mathcal{F}\e{s}'$) qui est réfléchie vers l'espace~: il comprend donc la contribution des surfaces continentales, de l'océan et de l'atmosphère. Il vaut~$0.31$ pour la planète Terre~: une partie significative du rayonnement reçu du Soleil par la Terre est réfléchie vers l'espace\footnote{L'albédo planétaire est par exemple encore plus élevé sur Vénus ($0.75$) à cause de la couverture nuageuse permanente et très réfléchissante de cette planète.}. Ainsi le système Terre reçoit une densité de flux énergétique moyenne~$F\e{reçu}$ en W~m$^{-2}$ telle que
\[ F\e{reçu} = (1-A\e{b}) \, \mathcal{F}\e{s}' \] 
donc un flux énergétique~$\Phi\e{reçu}$ (en W) qui s'exprime
\[ \Phi\e{reçu} = \pi \, R^2 \, (1-A\e{b}) \, \mathcal{F}\e{s} \]
%L'albédo de Bond~ désigne l'albédo intégré sur toutes les longueurs d'onde et tous les angles d'incidence.

\sk
La valeur de~$30\%$ de l'albédo planétaire sur Terre est en fait majoritairement dû à l'atmosphère~:  seuls 4\% de l'énergie solaire incidente sont réfléchis par la surface terrestre comme indiqué sur la figure~\ref{fig:diffsep}. L'énergie réfléchie par l'atmosphère vers l'espace, responsable de plus de~$85\%$ de l'albedo planétaire, est diffusée par les molécules ou par des particules en suspension, gouttelettes nuageuses, gouttes de pluie ou aérosols.

\figside{0.4}{0.15}{\figpayan/LP211_Chap2_Page_27_Image_0001.png}{L'énergie solaire incidente est réfléchie vers l'espace par la surface et l'atmosphère d'une planète. La figure montre les différentes contributions à l'albédo planétaire total.}{fig:diffsep}


\newpage
\section{\'Equilibre TOA}
\sk
Nous pouvons exprimer le rayonnement reçu du Soleil par la Terre par une densité de flux énergétique moyenne~$F\e{reçu}$ en W~m$^{-2}$ ou un flux énergétique~$\Phi\e{reçu}$ (en W)
\[ 
F\e{reçu} = (1-A\e{b}) \, \mathcal{F}\e{s}' 
\qquad \qquad
\Phi\e{reçu} = \pi \, R^2 \, (1-A\e{b}) \, \mathcal{F}\e{s}
\] 
La partie du rayonnement reçue du soleil qui est réfléchie vers l'espace est prise en compte via l'albédo planétaire noté~$A\e{b}$. On rappelle par ailleurs que~$\mathcal{F}\e{s}' = \mathcal{F}\e{s} / 4$ où $\mathcal{F}\e{s}$ est la constante solaire.


\sk
Par ailleurs, le système Terre émet également du rayonnement principalement dans les longueurs d'onde infrarouge [figure \ref{fig:eqrad2}]. 
Cette quantité de rayonnement émise au sommet de l'atmosphère radiative est notée $OLR$ pour \emph{Outgoing Longwave Radiation} en anglais.
A l'équilibre, la planète Terre doit émettre vers l'espace autant d'énergie qu'elle en reçoit du Soleil, donc
on obtient la relation générale appelée \emph{TOA} pour \emph{Top-Of-Atmosphere} en anglais, correspondant
au bilan radiatif au sommet de l'atmosphère
\[ \boxed{\TOA} \] 
La principale difficulté qui sous-tend les divers modèles pouvant être proposés réside dans l'expression du terme~$OLR$.




\newpage
\section{Bilan simple : température équivalente}
\sk
A l'équilibre, la planète Terre doit émettre vers l'espace autant d'énergie qu'elle en reçoit du Soleil. Ceci peut s'exprimer par unité de surface
\[ \boxed{ F\e{reçu} = F\e{émis} } \]
ou, pour un résultat similaire, en considérant l'intégralité de la surface planétaire
\[ \Phi\e{reçu} = \Phi\e{émis} \]
ce qui permet de déterminer la température équivalente en fonction des paramètres planétaires
\[ \boxed{
T\e{eq} = \bigg[ \frac{\mathcal{F}\e{s}'\,(1-A\e{b})}{\sigma} \bigg]^{\frac{1}{4}}
} \]


Le calcul présenté ici porte le nom d'\voc{équilibre radiatif simple}. On y néglige les effets de l'atmosphère (sauf l'albédo) puisqu'on suppose que le rayonnement atteint la surface, ou est rayonné vers l'espace, sans être absorbé par l'atmosphère. La température équivalente est ainsi la température qu'aurait la Terre si l'on négligeait tout autre influence atmosphérique que la réflexion du rayonnement solaire incident. Les valeurs de $T\e{eq}$ pour quelques planètes telluriques sont données dans la table \ref{tab:planets}. On note que la température équivalente de Vénus est plus faible que celle de la Terre, bien qu'elle soit plus proche du Soleil, à cause de son fort albédo~; la formule indique bien que, plus le pouvoir réfléchissant d'une planète est grand, plus la température de sa surface est froide. Par ailleurs, comme indiqué par les calculs du tableau~\ref{tab:planets}, on remarque que la température équivalente, si elle peut renseigner sur le bilan énergétique simple de la planète, ne représente pas correctement la valeur de la température de surface. Par exemple, la température équivalente pour la Terre est~$T\e{eq} = 255 K = -18^{\circ}$C, bien trop faible par rapport à la température de surface effectivement mesurée. Il faut donc avoir recours à un modèle plus élaboré.

\begin{table}\label{tab:planets} \begin{center} \begin{tabular}{lccccc} &{\bf Mercure} &{\bf V\'enus}&{\bf Terre}&{\bf Mars} &{\bf Titan} \\ \hline $d\e{soleil}$ (UA) & 0.39 & 0.72 & 1 & 1.5 & 9.5 \\ $\mathcal{F}\e{s}\,$(W~m$^{-2}$) & $8994$ & $2614$ & $1367$ & $589$ & $15$ \\ $A\e{b}$ & $0.06$ & $0.75$ & $0.31$ & $0.25$ & $0.2$ \\ \textcolor{blue}{$T\e{surface}$ (K)} & \textcolor{blue}{$100/700$~K} & \textcolor{blue}{$730$} & \textcolor{blue}{$288$} & \textcolor{blue}{$220$} & \textcolor{blue}{$95$} \\ \hline $T\e{eq}$~(K) & $439$ & $232$ & $254$ & $210$ & $86$\\ \end{tabular} \caption{\emph{Comparaison des facteurs influençant la température équivalente du corps noir pour différentes planètes du système solaire.}} \end{center} \end{table}
%    Mercure & 0.39 & 8994 & 0.06 & 439 \\
%    Vénus & 0.72 & 2639 & 0.78 & 225 \\
%    Terre & 1 & 1368 & 0.30 & 255 \\
%    Mars & 1.52 & 592 & 0.17 & 216 \\


\newpage
\section{\'Epaisseur optique}
\sk
Considérons une espèce~$X$ bien mélangée dans l'atmosphère, qui absorbe dans un intervalle de longueur d'onde donné. A la longueur d'onde~$\lambda$, son \voc{épaisseur optique}~$t_\lambda$ s'écrit
\[ \boxed{ t_\lambda = \int_{0}^{z\e{sommet}} \, k_\lambda \, \rho_X \, \dd z } \]
où $k_\lambda$ est un coefficient d'absorption massique en m$^2$~kg$^{-1}$ et $\rho_X$ est la densité d'absorbant~X. Le nom d'épaisseur optique est assez intuitif. Si un flux de rayonnement~$\Phi_\lambda$ à la longueur d'onde~$\lambda$ est émis à la base de l'atmosphère, le flux observé au sommet de l'atmosphère est d'autant plus réduit qu'à cette longueur d'onde l'épaisseur optique de l'atmosphère traversée est grande\footnote{Si l'extinction est uniquement due à de l'absorption, sans diffusion, on a une relation directe entre l'épaisseur optique et le coefficient d'absorption de la couche~: 
\[\alpha_\lambda = 1 - e^{- \frac{t_\lambda}{\cos\theta}} \] où~$\theta$ est l'angle d'incidence du rayonnement danns la couche.}. La formule ci-dessus ne fait qu'exprimer le fait que la réduction du flux (l'extinction) est plus d'autant plus marquée que 
\begin{citemize}
\item l'espèce considérée est très absorbante dans la longueur d'onde considérée ($k_\lambda$ grand)~;
\item l'espèce considérée est présente en grande quantité ($\rho_X$ grand).
\end{citemize}
Ainsi, le dioxyde de carbone~CO$_2$, bien qu'étant un composant minoritaire ($\rho$ faible), peut atteindre des épaisseurs optiques très grandes dans les intervalles de longueur d'onde où il est très fortement absorbant ($k_\lambda$ élevé), par exemple dans l'infrarouge autour de~$15$~$\mu$m. Autrement dit, un composant minoritaire en quantité peut avoir un rôle majoritaire radiativement.


\newpage
\section{Constante de temps radiative}
A condition qu'elle abrite
des particules radiativement actives,
une atmosphère de pression
plus faible se caractérise par une constante
de relaxation radiative $\tau\e{R}$ plus
courte.
%
En effet, lorsque la pression
diminue, la densité diminue également
mais l'énergie radiative absorbée n'est
pas proportionnelle à la densité.
%
Nous pouvons illustrer ce point avec
un calcul simpl(ist)e sur une couche
atmosphérique d'épaisseur $e$, de densité $\rho$ 
et se comportant comme un corps noir
de température $T\e{e}$
($\sigma$ est la constante de Stefan-Boltzmann).
%
Le temps caractéristique $\tau\e{R}$ pour 
dissiper radiativement une
perturbation thermique $\Theta=\Delta T$
de l'équilibre radiatif de la couche
avec les couches environnantes est
donné par la conservation de l'énergie
%%
%Le temps de relaxation radiatif $\tau\e{R}$, 
%inverse de la constante
%d'amortissement radiatif, est le temps 
%nécessaire pour dissiper une perturbation
%thermique par des processus radiatifs.
%%
\[
\ddt{\Theta} + \f{\Theta}{\tau\e{R}} = 0
\quad \textrm{avec} \quad
\tau\e{R} = \f{c_p \, \rho \, e}{8 \, \sigma \, T\e{e}^3}
\]
%
\noindent L'expression de $\tau\e{R}$ traduit
bien le point mentionné précédemment.
%
Le rapport entre les constantes
de temps radiatives martiennes et terrestres
est donc principalement contrôlé
par la différence de densité.
%
Les valeurs planétaires donnent par exemple
%
\[
\f{ \tau\e{Mars} }{ \tau\e{Terre} }
=
\f{ \left( c_p \, \rho / T\e{e}^3 \right)\e{Mars} }{ \left( c_p \, \rho / T\e{e}^3 \right)\e{Terre} }
\sim 
\f{1}{40}
\]
%
\noindent soit un très fort amortissement
radiatif dans l'atmosphère martienne, deux ordres
de grandeur plus élevé que sur Terre.
%
Dans les conditions typiques pour la basse
atmosphère terrestre et martienne, 
$\tau\e{Mars}$ est de l'ordre de la journée
alors que $\tau\e{Terre}$ est de l'ordre du mois.
%
Sur Terre, les différences de constante radiative
entre la basse et la haute atmosphère sont expliquées
de la même façon.
%
L'estimation ci-dessus reste illustrative plus que quantitativement
valable.
%
Cependant, des calculs plus élaborés 
distinguant les molécules radiativement actives 
dans l'\IR~thermique sur Mars (\carb) et sur Terre (\eau),
donnent $\tau\e{Mars}/\tau\e{Terre}$
entre $1/5$ et $1/100$, ce qui
ne contredit pas l'ordre de grandeur
trouvé par le calcul simpliste
précédent. 
%
%
%%La grande concentration en \carb~et
%%la densité faible font que le temps
%%de relaxation radiatif de \lam~est
%%très faible.


\newpage
\section{Modèle à deux faisceaux : écriture}
\sk
Le modèle à deux faisceaux est un bon compromis entre simplicité
et illustration de concepts importants. Il est une version simplifiée
de l'équation de Schwarzschild du transfert radiatif.
Ce modèle entend élucider
les transferts de rayonnement dans l'infrarouge entre
les couches qui composent la colonne atmosphérique. Les
hypothèses simplificatrices suivantes sont réalisées
\begin{citemize}
\item couches atmosphériques plan-parallèle (sphéricité négligée)
\item phénomènes d'absorption négligés dans le visible (transparence au visible)
\item phénomènes de diffusion (\emph{scattering}) négligés dans l'infrarouge
\item \emph{gray gas} dans l'infra-rouge : on considère que le coefficient d'absorption
du gaz est indépendant de la longueur d'onde~$\lambda$ ($k_{\lambda} = k$ pour tout~$\lambda$),
ce qui implique une hypothèse similaire pour l'épaisseur optique ($\tau_{\lambda} = \tau$ pour tout~$\lambda$).
\end{citemize}
En d'autres termes, on se cantonne dans ce modèle à deux types de phénomènes
\begin{enumerate}
\item Un faisceau de rayonnement infra-rouge de flux~$F$ traversant une couche 
atmosphérique donnée
subit une extinction à cause de l'absorption selon une loi de type Beer-Lambert
\[
\dd F = - F \dd \tau
\]
avec~$\tau$ l'épaisseur optique \emph{gray gas} 
dans l'infra-rouge.
\item Une couche atmosphérique émet un flux de rayonnement thermique~$M$ 
calculé par la loi intégrée de Stefan-Boltzmann ($\SB$)
puisque la majorité de l'émittance est émise dans l'infrarouge pour les températures considérées.
\end{enumerate}

\sk
L'épaisseur optique~$\tau$ peut servir de coordonnée verticale à la place de~$z$
en utilisant la relation entre les deux quantités. La couche atmosphérique
élémentaire considérée est ainsi d'épaisseur~$\dd\tau$ et située à une coordonnée
verticale~$\tau$ qui croît avec l'altitude. 

\sk
Si l'on considère un faisceau ascendant~$F^+(\tau)$ au bas de la couche considérée,
une fois la couche traversée son amplitude est
\[
F^+(\tau) - F^+(\tau) \dd\tau
\]
A ce flux au sommet de la couche, il convient d'ajouter
la contribution de l'émission thermique de la couche vers 
le haut, à savoir~$M\,\dd\tau$.
Le flux total ascendant au sommet de la couche est donc
\[
F^+(\tau+\dd\tau) = F^+(\tau) - F^+(\tau) \dd\tau + M\dd\tau
\]

\sk
Même raisonnement avec le flux descendant~$F^-(\tau+\dd\tau)$ au sommet de la couche considérée,
une fois la couche traversée son amplitude est~$F^-(\tau+\dd\tau) - F^-(\tau) \dd\tau$, où 
l'approximation du terme du second ordre~$F^-(\tau+\dd\tau) \dd\tau \simeq F^-(\tau) \dd\tau$
a été effectuée.
Le flux total descendant au bas de la couche est donc
\[
F^-(\tau+\dd\tau) = F^-(\tau) - F^-(\tau) \dd\tau + M\dd\tau
\]

\sk
Les deux résultats qui précèdent peuvent être transformés 
afin de faire apparaître une dérivée
en utilisant le théorème des accroissements finis
\[
\ddf{F^+}{\tau} = \frac{F^+(\tau+\dd\tau) - F^+(\tau)}{\dd\tau}
\]
\noindent ce qui permet d'obtenir au final
\[
\ddf{F^+}{\tau} = - F^+(\tau) + \epsilon\,\sigma\,T(\tau)^4 \quad [S^+]
\qquad\qquad 
\ddf{F^-}{\tau} = F^-(\tau) - \epsilon\,\sigma\,T(\tau)^4 \quad [S^-]
\]
\noindent $[S^+]$ et~$[S^-]$ sont parfois appelées les relations de Schwarzschild à deux faisceaux.
Il s'agit d'une version très simplifiée des équations de Schwarzschild du transfert radiatif.

\sk
Si l'on souhaite adopter la convention~$\tau=0$ au sommet de l'atmosphère,
et $\tau=\tau_{\infty}$ à la surface en $z=0$, donc adopter un axe
vertical d'épaisseur optique avec~$\tau$ croissant de haut en bas, il
suffit de remplacer~$\tau$ par~$\tau_{\infty}-\tau$ dans les équations précédentes pour obtenir
\[
\boxed{\ddf{F^+}{\tau} = F^+(\tau) - \epsilon\,\sigma\,T(\tau)^4 \quad [S^+]} 
\qquad\qquad 
\boxed{\ddf{F^-}{\tau} = -F^-(\tau) + \epsilon\,\sigma\,T(\tau)^4 \quad [S^-]}
\]









\newpage
\section{Modèle à deux faisceaux : résolution 1}
\sk
Le système d'équations~$[S^+]$ et~$[S^-]$ du modèle à deux faisceaux
est plus simple à résoudre si l'on considère les deux quantités~$\Sigma(\tau)=F^{+}(\tau)+F^{-}(\tau)$ et~$\Delta(\tau)=F^{+}(\tau)-F^{-}(\tau)$ car on obtient
\[
\ddf{\Sigma}{\tau} = \Delta(\tau) \quad [E_\Sigma] 
\qquad\qquad 
\ddf{\Delta}{\tau} = \Sigma(\tau) - 2\,\epsilon\,\sigma\,T(\tau)^4 \quad [E_\Delta]
\]
\noindent Ensuite la résolution impose d'expliciter les conditions aux limites
\begin{enumerate}[$\mathcal{C}_1$]
\item on se place à l'équilibre radiatif donc le flux net~$\Delta$ est constant à tout niveau : $\ddf{\Delta}{\tau}=0$
\item au sommet de l'atmosphère $F^+(\tau=0) = OLR$ (définition de $OLR$) et $F^-(\tau=0) = 0$ (contribution
incidente négligeable du Soleil dans l'infra-rouge), ce qui s'écrit encore~$\Delta(\tau=0)=\Sigma(\tau=0)=OLR$
\item à la surface de température~$T_s$ le bilan radiatif est le suivant : la surface reçoit l'intégralité du rayonnement
solaire incident~$(1-A\e{b}) \, \mathcal{F}\e{s}'$ (visible) plus du rayonnement de l'atmosphère située
juste au-dessus d'elle~$F^-(\tau=\tau_{\infty})$ (infra-rouge) ; de plus elle émet un rayonnement
$\epsilon\,\sigma\,T\e{s}^4$ dans l'infra-rouge vers l'atmosphère\footnote{On a supposé ici pour simplifier les calculs que l'émissivité
de la surface était similaire à l'émissivité de l'atmosphère}
\item on rappelle que selon la relation \emph{TOA}, nous avons $OLR = (1-A\e{b}) \, \mathcal{F}\e{s}'$
\end{enumerate}


\newpage
\section{Modèle à deux faisceaux : résolution 2}
\sk
\paragraph{Conséquence 1} Il est alors possible d'obtenir deux expressions différentes pour~$\Sigma(\tau)$.
Premièrement, en utilisant $[E_\Sigma]$ avec $\mathcal{C}_1$ et $\mathcal{C}_2$, on obtient~$\Sigma(\tau)=OLR \, (1+\tau)$.
Deuxièmement, en utilisant $[E_\Delta]$ avec $\mathcal{C}_1$, on obtient~$\Sigma(\tau)=2\,\epsilon\,\sigma\,T(\tau)^4$.
On obtient le \voc{profil radiatif}, 
c'est-à-dire le profil vertical de température\footnote{Suivant la géométrie
équivalente choisie pour le modèle plan-parallèle, le terme $1+\tau$
peut s'écrire un peu différemment, mais quoiqu'il en soit toujours sous une
forme~$a+b\,\tau$ avec $a,b$ constants. Les conclusions énoncées ici ne sont pas
modifiées.} imposé par les transferts radiatifs dans l'infrarouge
\[
T(\tau) = \sqrt[4]{\frac{OLR\,(1+\tau)}{2\,\sigma\,\epsilon}}
\]

\sk
\paragraph{Conséquence 2} Reste à calculer la température de surface avec ce modèle. D'après $\mathcal{C}_3$, le bilan
au sol s'écrit~$(1-A\e{b}) \, \mathcal{F}\e{s}' + F^-(\tau=\tau_{\infty}) = \epsilon\,\sigma\,T\e{s}^4$.
Il faut donc exprimer les flux ascendant et descendant dans l'infrarouge.
Du fait que $\mathcal{C}_1$ et $\mathcal{C}_2$ nous indiquent que~$\Delta=OLR$, on obtient aisément
%% = F+(tau_inf) directement?
\[
F^+(\tau) = \frac{\Sigma+\Delta}{2} = OLR \, (1+\frac{\tau}{2})
\qquad \qquad
F^-(\tau) = \frac{\Sigma-\Delta}{2} = OLR \, \frac{\tau}{2}
\]
\noindent On obtient alors l'expression liant $OLR$
et température de surface~$T\e{s}$
\[
\boxed{\epsilon\,\sigma\,T\e{s}^4 = OLR \, \left( 1 + \frac{\tau_{\infty}}{2} \right)}
\]
\noindent On obtient ainsi une définition quantitative de \voc{l'effet de serre}
\begin{citemize}
\item Dans l'infrarouge, le rayonnement sortant au sommet de l'atmosphère ($OLR$)
est inférieur au rayonnement émis par la surface ($\epsilon\,\sigma\,T\e{s}^4$).
Une partie du rayonnement émis par la surface reste donc piégée par la planète.
\item Avec un albédo et un rayonnement incident constant, donc à~$OLR$ constant (d'après $\mathcal{C}_4$),
augmenter la quantité de gaz à effet de serre (donc augmenter~$\tau_{\infty}$)
conduit à une augmentation de la température de surface~$T\e{s}$.
\end{citemize}

\sk
\paragraph{Conséquence 3} Il est alors instructif de s'intéresser à la température atmosphérique 
proche de la surface~$T(\tau_\infty)$
donnée par le profil radiatif. Cette température ne dépend que de~$OLR$
et s'obtient totalement indépendamment de la température de surface.
On peut alors montrer que
\[
T\e{s} = T(\tau_\infty) \, \sqrt[4]{\frac{2+\tau_\infty}{1+\tau_\infty}} > T(\tau_\infty)
\]
\noindent Tant que l'atmosphère n'est pas
optiquement épaisse dans l'infrarouge (donc tant que~$\tau_\infty$ reste fini),
il existe une \voc{discontinuité entre la surface et l'atmosphère}, la surface
étant toujours plus chaude que l'atmosphère. Cela implique que l'atmosphère
est instable proche de la surface, donc que du mélange turbulent / convectif
apparaît, donc que l'équilibre proche de la surface ne peut être simplement
radiatif mais \voc{radiatif-convectif}. Notons que dans le cas où l'atmosphère est optiquement épaisse,
$T\e{s} = T(\tau_\infty)$, ce qui est tout à fait représentatif des conditions sur Vénus.


%%% figure Salby


\newpage
\section{Profil radiatif convectif (telluriques)}
\sk
Les conditions atmosphériques sont très instables proche d'une surface (en présence d'une telle surface). A cause de la discontinuité entre surface et atmosphère, sous l'action de la diffusion thermique, ou turbulente, entre la surface (chaude) et l'air immédiatement adjacent (plus froid) crée une couche d'air fine approximativement à la température de la surface ; les conditions de température étant plus froides au-dessus, les conditions atmosphériques sont très instables proche de la surface et des mouvements de convection vont se mettre en place pour mélanger l'air sur une certaine épaisseur atmosphérique. Un équilibre dit \voc{radiatif-convectif} prévaut, avec une structure thermique suivant le profil adiabatique, donnant naissance à une troposphère. Au-dessus de la limite radiative-convective (correspondant peu ou prou à la tropopause), les phénomènes radiatifs dominent et donnent naissance à une mésosphère -- ou une stratosphère si un absorbant visible y est présent en quantité suffisante, donnant naissance à une inversion stable à la tropopause.
%% on passait en troposphère dès que le gradient du profil radiatif dépassait celui du profil adiabatique (-g/cp)

\figun{0.4}{0.25}{/home/aymeric/Big_Data/BOOKS/pierrehumbert_pics/9780521865562c03_fig014.jpg}{Figure tirée de R. Pierrehumbert, Principles of Planetary Climates, CUP, 2010}{fig:effetserre2}











\newpage
\section{Profil radiatif convectif (géantes)}
\sk
La présence d'une surface et une hypothèse d'équilibre radiatif imposent donc que le chauffage de la surface conduit inévitablement à de la convection. Que se passe-t-il sur les planètes géantes dépourvues de surface ? L'équilibre radiatif y prévaut également, car à partir d'une certaine profondeur, le gradient radiatif est instable -- et ce, même en l'absence d'une surface qui absorbe le rayonnement solaire. Pour formuler la stabilité de l'équilibre radiatif, on calcule le profil~$\dd T / \dd p$ et on le compare au gradient adiabatique sec ou humide dans l'atmosphère considérée. En dérivant le profil radiatif obtenu dans le cas du modèle à deux faisceaux
\[ T(\tau) = \sqrt[4]{\frac{OLR\,(1+\tau)}{2\,\sigma\,\epsilon}} \]
\noindent par rapport à l'épaisseur optique~$\tau$, nous obtenons
\[ 8 \, \sigma \, T^3 \, \ddf{T}{\tau} = OLR \]
\noindent soit, en utilisant~$\ddf{~}{p} = \ddf{\tau}{p} \ddf{~}{\tau}$
\[ \ddf{T}{\ln p} = \frac{1}{4\,(1+\tau)} \, p \, \ddf{\tau}{p} \]
Ainsi la stabilité de la couche s'écrit
\[ \frac{R}{c_p} \ge \frac{1}{4\,(1+\tau)} \, p \, \ddf{\tau}{p} \]
\noindent et dans le cas d'un coefficient d'absorption~$\kappa$ constant, nous pouvons même écrire la condition de stabilité
\[ \frac{R}{c_p} \ge \frac{\tau}{4\,(1+\tau)} \]
%% si l'absorption est constante p \, \ddf{\tau}{p} = - \kappa \, p / g costheta

\sk
Le terme en~$p$ dans ce qui précède guarantit (à moins d'une variation énorme de $\ddf{\tau}{p}$ en~$1/p$ ou plus rapide quand~$p \rightarrow 0$) que les hautes atmosphères planétaires sont toujours stables. De plus, les atmosphères optiquement fines sont toujours stables sur l'intégralité de leur épaisseur, puisque~$-p \, \ddf{\tau}{p} < \tau_\infty \ll 1$. Le critère de stabilité est en pratique un peu plus complexe qu'indiqué dans les atmosphères réelles. Bien sûr, $\tau$ et~$\kappa$ varient avec la longueur d'onde~$\lambda$ (limitation inhérente au modèle à deux faisceaux), mais surtout le coefficient d'absorption~$\kappa$ augmente avec la pression (donc la profondeur) en raison de l'élargissement collisionnel (\emph{collisional broadening}), efficace à partir de quelques bars. La loi de variation d'élargissement collisionnel peut s'écrire~$\kappa(p) = \kappa(p\e{s}) \, \frac{p}{p\e{s}}$. Les processus de changements d'état sont également susceptibles de rendre la situation plus complexe que le calcul proposé ici.


\newpage
\section{Transfert radiatif}
\sk
Le rayonnement électromagnétique, lors de sa traversée de l'atmosphère terrestre, est perturbé par deux processus que l'on peut analyser séparément~: un processus d'\voc{absorption} par certains gaz atmosphériques (O$_2$, H$_2$O, O$_3$, etc.) et un processus de \voc{diffusion} par les molécules et les aérosols (poussières, cristaux de glace, gouttes nuageuses et gouttes de pluie). 
\begin{finger}
\item Dans le processus d'absorption, un certain nombre de photons disparaissent, une partie du rayonnement incident est convertie en énergie interne, et il y a une extinction du signal dans la direction de propagation. 
\item Au contraire, dans le processus de diffusion, les photons sont simplement redistribués dans toutes les directions avec une certaine probabilité définie par ce qu'on appelle la fonction de phase de diffusion~; on peut alors observer une extinction dans certaines directions et une augmentation dans d'autres. Lors de la diffusion, il n'y a pas de changement de longueur d'onde de l'onde incidente et de l'onde diffusée.
\end{finger}
%Ainsi la réflexion peut être \voc{diffuse} (dans toutes les directions), \voc{spéculaire} (dans la direction symétrique du rayonnement incident) ou quelconque. 
%Les deux effets peuvent être analysés séparément.
Autrement dit, on s'intéresse à l'\voc{extinction} progressive du rayonnement incident par absorption et diffusion. Deux phénomènes très importants sont mis de côté~: l'émission de rayonnement thermique, déjà abordée au chapitre précédent, et la diffusion multiple, que l'on néglige (rayonnement diffusé qui viendrait depuis d'autres directions).


\newpage
\section{Section efficace}
\sk
Que l'on s'intéresse à la diffusion ou l'absorption, l’interaction entre le rayonnement et la matière dépend de la rencontre entre les photons et les éléments (atomes, molécules, particules) du milieu considéré. La diffusion et l'absorption dépendent donc de la probabilité que le rayonnement, c'est-à-dire les photons qui le constituent, rencontre les éléments constitutifs de la matière. 

\sk
De façon évidente, cette probabilité est liée 
\begin{citemize}
\item au flux de photons~: la probabilité est plus élevée s'il y a un plus grand nombre de photons incidents, ou de façon équivalente une plus grande énergie radiative incidente ;
\item au nombre d'éléments (atomes, molécules, particules) dans le milieu matériel~: la probabilité augmente avec le nombre d'éléments, autrement dit dans un milieu plus dense, le rayonnement aura une plus forte probabilité de rencontrer des éléments matériels avec lesquels interagir. 
\end{citemize}
Cependant, même si ces deux quantités sont élevées, la probabilité peut rester faible, car elle dépend également d'un paramètre qui traduit l'efficacité de la rencontre entre un photon de longueur d'onde donnée~$\lambda$ et l'espèce absorbante. Cette grandeur est appelée \voc{section efficace} et est décrite ci-dessous.

\figun{0.5}{0.2}{\figfrancis/Beer_upside}{Variation du rayonnement incident avec un angle $\theta$ sur une couche d'épaisseur $dz$}{fig:beer}

\sk
On considère la situation décrite dans la figure~\ref{fig:beer}. Soit une tranche d'atmosphère horizontale\footnote{On dit qu'on fait l'approximation plan-parallèle car on néglige la courbure de la Terre ainsi que les variations horizontales des paramètres géophysiques (température et profils de gaz).} d'épaisseur élémentaire~$\dd z$ qui reçoit un rayonnement monochromatique de longueur d'onde~$\lambda$ caractérisé par sa luminance énergétique spectrale~$L_\lambda$. Le rayonnement incident traverse la tranche d'atmosphère en faisant un angle~$\theta$ par rapport à la verticale. La distance parcourue par le rayonnement à travers la fine couche d'épaisseur~$\dd z$ vaut\footnote{On parle d'abscisse curviligne pour qualifier~$s$.} 
\[ \dd s = \frac{1}{\cos\theta} \, \dd z \]
%%%% OK avec Beer_upside
autrement dit l'inclinaison du rayonnement impose un chemin optique plus grand. A la sortie de la tranche d'atmosphère, le rayonnement a subi une extinction à cause des phénomènes de diffusion et absorption dans la tranche d'atmosphère. On caractérise alors l'extinction causée par la diffusion et l'absorption par une quantité appelée section efficace~$\Sigma_\lambda$ qui a la dimension d'une surface, exprimée en m$^2$. A la sortie de la tranche d'atmosphère, la luminance spectrale est~$L_\lambda + \dd L_\lambda$ avec 
\[ \dd L_\lambda = - L_\lambda(s) \, \Sigma_\lambda(s) \, N \, \dd s \qquad \textrm{ou de manière équivalente} \qquad \boxed{ \dd L_\lambda = - L_\lambda(z) \, \Sigma_\lambda(z) \, N \, \frac{1}{\cos\theta} \, \dd z } \]
où $N$ est le nombre de particules par unité de volume. La section efficace~$\Sigma_{\lambda}$ prend en compte l'extinction causée par les phénomènes d'absorption et de diffusion~: plus la section efficace est grande, plus l'extinction du rayonnement incident est élevée. Il est possible de séparer les deux contributions en définissant une section efficace d'absorption~$\Sigma_\lambda^a$ et une section efficace de diffusion/réflexion~$\Sigma_\lambda^r$ telles que~$\Sigma_{\lambda} = \Sigma_\lambda^a + \Sigma_\lambda^r$. La section efficace dépend de la longueur d'onde et de la nature physico-chimique du milieu absorbant~: par exemple, la composition de l'air dans le cas de l'atmosphère, ou la température dans le cas de la majorité des matériaux. Ainsi, en toute généralité, elle n'a pas de raison particulière de rester constante pour les différentes parties du milieu matériel traversé.

%Ce n'est pas le but de ce cours que de proposer une vision complète du transfert radiatif, mais le lecteur intéressé peut noter que l'équation complète du transfert radiatif dans l'atmosphère prend la forme indiquée précédemment avec cependant l'ajout des deux termes \ofg{sources} précités \[ \dd L_\lambda = - L_\lambda(s) \, \Sigma_\lambda(s) \, N \, \dd s + \textrm{émission thermique} + \textrm{diffusion multiple} \]



\newpage 
\section{Diffusion}
\sk
La diffusion est un phénomène macroscopique résultant de la réflexion, de la réfraction et de la diffraction du rayonnement incident, qui se produisent au niveau microscopique en raison des inhomogénéités du milieu matériel traversé. Par abus de langage, auquel les présentes notes n'échappent pas, on identifie souvent réflexion et diffusion. 

\sk
On distingue différents mécanismes de diffusion selon la taille relative des cibles (molécules ou particules) par rapport à la longueur d'onde du rayonnement électromagnétique incident [figure \ref{fig:diffsize}]. Les connaître permet de comprendre certains phénomènes atmosphériques perçus au quotidien [figure~\ref{fig:ciel}]. Les radiations solaires situées dans l'ultraviolet sont absorbées dans la haute atmosphère (notamment par l'ozone dans la stratosphère) si bien que l'on considère principalement les radiations visibles.
\begin{finger}
\item \underline{Taille des cibles petite devant la longueur d'onde du rayonnement incident} La \voc{diffusion Rayleigh} est la diffusion par les molécules\footnote{ou par des particules significativement petites devant la longueur d'onde, mais ce cas de figure est relativement rare en pratique} qui constituent l'atmosphère. La diffusion Rayleigh dépend fortement de la longueur d’onde incidente. Lord Rayleigh à démontré en 1873 que cette dépendance s’exprimait selon l'inverse de la puissance quatrième de la longueur d'onde
\[ \boxed{ \Sigma_\lambda^r \propto \lambda^{-4} } \]
\begin{citemize}
\item Cette dépendance en longueur d'onde a un effet très notable sur le rayonnement thermique reçu du Soleil, dont nous avons vu au précédent chapitre qu'il est maximum dans les longueurs d'onde visibles. Les molécules d'air de l'atmosphère diffusent plus les photons de courte longueur d'onde à cause de la loi en puissance quatrième de~$\lambda$~: ainsi, le violet et le bleu sont~$16$ fois plus diffusés que le rouge par le mécanisme de Rayleigh. C'est pour cette raison que l'on voit le ciel bleu depuis la surface~: il s'agit de la couleur émanant du rayonnement solaire incident diffusé en majorité par le mécanisme de Rayleigh\footnote{Le fait qu'on ne voit pas le ciel violet est dû à une moindre sensibilité de l'oeil à ces longueurs d'onde, ainsi qu'un moindre flux incident que dans le bleu d'après le spectre solaire.}. La figure~\ref{fig:diffsep} nous indique que~$6\%$ du rayonnement incident sont ainsi diffusés, soit une contribution d'environ~$20\%$ à l'albédo planétaire. 
\item La diffusion Rayleigh ne montre pas de direction préférentielle significative, à part une tendance légèrement supérieure à la diffusion vers l'arrière (rétrodiffusion) et vers l'avant [figure~\ref{fig:diffdir}]. Ceci explique que le ciel apparaisse bleu qu'on le regarde depuis la surface ou depuis un avion. Ceci explique également que la diffusion Rayleigh fasse apparaître le Soleil de la couleur la moins diffusée, à savoir jaune à rouge suivant l'importance de la diffusion, alors qu'il apparaîtrait blanc sans diffusion.
\item Au lever et au coucher du Soleil, lorsque la lumière solaire traverse une couche importante d'atmosphère, la diffusion Rayleigh est plus grande qu'en journée lorsque le Soleil est proche du zénith. La raison est purement géométrique, comme l'on peut s'en convaincre d'après la section~\ref{sec:efficace}~: si l'angle d'incidence~$\theta$ est plus grand, le facteur~$1/\cos\theta$ est plus grand, donc, pour une même section efficace de diffusion~$\Sigma_\lambda^r$, la variation du flux incident~$dL_\lambda$ est plus grande. Ainsi, le soleil est vu rouge le soir car plus de rayonnement incident dans les longueurs d'onde bleues est diffusé qu'en journée.
\item Au contraire du rayonnement thermique solaire, dont le maximum d'émission est dans le domaine visible, l'effet de la diffusion Rayleigh sur le rayonnement thermique émis par la Terre est négligeable, car ce dernier est situé dans l'infrarouge à des longueurs d'onde~$\lambda$ plus élevées pour lesquelles~$\Sigma_\lambda^r \sim 0$. 
\end{citemize}

\figside{0.6}{0.3}{\figfrancis/WH_diff_size}{Type de mécanisme de diffusion dominant en fonction de la longueur d'onde (en abscisse) et de la taille des particules (en ordonnée, l'unité est en $\mu$m). Figure adaptée de Wallace and Hobbs, Atmospheric Science, 2006.}{fig:diffsize}

\figsup{0.48}{0.2}{decouverte/cours_meteo/nuage_ciel.jpg}{decouverte/cours_meteo/ciel_rouge.jpg}{Ciel bleu et nuage blanc. Coucher de soleil rouge. Crédits photos: \url{http://www.meteofrance.com} et \url{http://www.exworld.fr}}{fig:ciel}

\item \underline{Taille des cibles grande devant la longueur d'onde du rayonnement incident} La diffusion par les particules les plus grosses, par exemple les gouttes de brume de quelques centaines de microns, les gouttes de pluie de l'ordre du mm, les cristaux de glace de quelques dizaines de microns, ou les poussières les plus grosses, peut être expliquée par les lois de l'\voc{optique géométrique}, les lois qui gouvernent le fonctionnement des lentilles convergentes/divergentes. Contrairement à la diffusion de Rayleigh, la diffusion est non sélective, c'est-à-dire qu'elle ne dépend pas de la longueur d'onde. Les gouttes d'eau de l'atmosphère diffusent toutes les longueurs d'onde de façon quasiment équivalente, ce qui produit un rayonnement blanc. Ceci explique pourquoi le brouillard et les nuages nous paraissent blancs. La réalité d'un nuage est parfois plus complexe~: ses propriétés radiatives dépendent de la taille des particules et leur nombre par unité de volume.
\item \underline{Taille des cibles grande devant la longueur d'onde du rayonnement incident} La diffusion par les particules de taille intermédiaire, par exemple les gouttes nuageuses ou les aérosols de plus petite taille (quelques microns), est plus délicate à étudier que les deux mécanismes précédemment cités. On parle de \voc{diffusion de Mie}. Les particules soulevées pendant une tempête de poussière sur Terre ou sur Mars causent par diffusion de Mie une couleur orangée au ciel. Suivant la taille et la nature de la particule interagissant avec le rayonnement, la diffusion de Mie peut avoir des caractéristiques très directionnelles [figure \ref{fig:diffdir}]. La section efficace~$\Sigma_\lambda^r$ de la diffusion de Mie suit une loi en l'inverse de~$\lambda^2$ avec la longueur d'onde~$\lambda$ du rayonnement incident. Ces variations sont donc moins sensibles à la longueur d'onde que dans le cas de la diffusion de Rayleigh.
\end{finger}

\figside{0.6}{0.25}{\figfrancis/WH_diff_dir}{Répartition de la probabilité de diffusion dans différentes directions, pour différents types de diffusion: (a) Rayleigh, (b) et (c) Mie avec une particule plus grande en (c).}{fig:diffdir}



\end{document}
