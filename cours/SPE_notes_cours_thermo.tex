\documentclass[a4paper,DIV16,10pt]{scrartcl}
%%%%%%%%%%%%%%%%%%%%%%%%%%%%%%%%%%%%%%%%%%%%%%%%%%%%%%%%%%%%%%%%%%%%%%%%%%%%%%%%%%%
\usepackage{texcours}
%%%%%%%%%%%%%%%%%%%%%%%%%%%%%%%%%%%%%%%%%%%%%%%%%%%%%%%%%%%%%%%%%%%%%%%%%%%%%%%%%%%
\newcommand{\zauthor}{Aymeric SPIGA}
\newcommand{\zaffil}{Laboratoire de Météorologie Dynamique}
\newcommand{\zemail}{aymeric.spiga@upmc.fr}
\newcommand{\zcourse}{Physique, chimie, écologie de l'environnement}
\newcommand{\zcode}{UE1}
\newcommand{\zuniversity}{UPMC}
\newcommand{\zlevel}{M1 Sciences et Politiques de l'Environnement}
\newcommand{\zsubtitle}{Fiches complémentaires de cours}
\newcommand{\zlogo}{\includegraphics[height=1.5cm]{decouverte/cours_meteo/UPMC_cart-blanc-Q_7504-703-3.png}}
\newcommand{\zrights}{Copie et usage interdits sans autorisation explicite de l'auteur}
\newcommand{\zdate}{\today}
%%%%%%%%%%%%%%%%%%%%%%%%%%%%%%%%%%%%%%%%%%%%%%%%%%%%%%%%%%%%%%%%%%%%%%%%%%%%%%%%%%%
\begin{document} \inidoc
%%%%%%%%%%%%%%%%%%%%%%%%%%%%%%%%%%%%%%%%%%%%%%%%%%%%%%%%%%%%%%%%%%%%%%%%%%%%%%%%%%%

\newpage \sk
La figure~\ref{fig:blue} illustre la présence d'une atmosphère très active sur Terre par les nuages qui y prennent naissance. Il ne s'agit que d'un exemple parmi tant d'autres pour appréhender l'atmosphère. C'est le but de ce chapitre d'introduction d'évoquer la diversité des points de vue pouvant être adoptés pour étudier l'atmosphère, un système complexe où se mêlent processus physiques, dynamiques, chimiques, biologiques, et même sociétaux. Sont également abordées dans ce chapitre quelques notions de base nécessaires pour la suite du cours.

\figsup{0.45}{0.25}{decouverte/cours_meteo/blue_50.png}{decouverte/cours_meteo/blueclouds_50.png}{La planète Terre avec et sans les nuages de son atmosphère. Les nuages couvrent très souvent au moins la moitié du globe. Construit d'après une image \ofg{Blue Marble} NASA du projet \ofg{Visible Earth}. Des versions haute-résolution des images planes et des explications complètes peuvent être trouvées aux adresses suivantes \url{http://visibleearth.nasa.gov/view_rec.php?id=2430} et \url{http://visibleearth.nasa.gov/view_rec.php?id=2431}.}{fig:blue} %%%http://visibleearth.nasa.gov/view_rec.php?id=2429



\mk \section{Quelques définitions et généralités}

	\sk \subsection{Vocabulaire}
	\sk
L'objectif des sciences de l'atmosphère est d'étudier la structure et l'évolution de l'atmosphère en caractérisant et en expliquant les phénomènes qui s'y déroulent. Les sciences de l'atmosphère font principalement appel à des notions de physique, chimie, et mécanique des fluides.
\begin{description}
\item[\voc{Atmosphère}] Ensemble de couches, principalement gazeuses, qui entourent la masse condensée, solide ou liquide, d'une planète (voir également citation de Laplace en en-tête).
\item[\voc{Air}] Mélange gazeux constituant l'atmosphère terrestre.
\item[\voc{Aéronomie}] Science dont l'objet est la connaissance de l'état physique de l'atmosphère terrestre et des lois qui la gouvernent.
\item[\voc{Météorologie}] Discipline ayant pour objet l'étude des phénomènes atmosphériques et de leurs variations, et qui a pour objectif de prévoir à court terme les variations du temps.
\item[\voc{Climat}] Ensemble des conditions atmosphériques et météorologiques d'un pays, d'une région. Le climat peut également être défini comme un système thermo-hydrodynamique non isolé dont les composantes sont les principales « enveloppes » externes de la Terre : on parle également de~\voc{système climatique} [figure~\ref{fig:pluri}]. 
\begin{citemize} \item L'atmosphère : l’air, les nuages, les aérosols, \ldots 
\item L’hydrosphère : les océans, les rivières, les précipitations, \ldots 
\item La lithosphère : les terres immergées, les sols, \ldots 
\item La cryosphère : glace, neige, banquise, glaciers, \ldots 
\item La biosphère : les organismes vivants, \ldots
\item L’anthroposphère : l’activité humaine, \ldots 
\end{citemize} 
\end{description}

\figside{0.75}{0.35}{decouverte/cours_meteo/joussaume_pluri.png}{Schéma du système climatique présentant les différentes composantes du système : atmosphère, océans, cryosphère, biosphère et lithosphère, leurs constantes de temps et leurs interactions en termes d’échanges d’énergie, d’eau et de carbone. Source~:~S.~Joussaume \emph{in} Le Climat à Découvert, CNRS éditions, 2011}{fig:pluri}


	\sk \subsection{Grandeurs utiles}
	\sk
L'atmosphère est composée d'un ensemble de molécules. Pour la description de la plupart des phénomènes étudiés, le suivi des comportements individuels de chacunes des molécules composant l'atmosphère est impossible. On s'intéresse donc aux effets de comportement d'ensemble, ou moyen. Les principales variables thermodynamiques utilisées pour décrire l'atmosphère sont donc des grandeurs \voc{intensives} dont la valeur ne dépend pas du volume d'air considéré.
\begin{finger}
\item La \voc{température} $T$ est exprimée en K (kelvin) dans le système international. Elle est un paramètre macroscopique qui représente l'agitation thermique des molécules microscopiques. Les mesures de température usuelles font parfois référence à des quantités en \deg C, auxquelles il faut ajouter la valeur $273.15$ pour convertir en kelvins.
\item La \voc{pression} $P$ est exprimée en Pa dans le système international. La pression fait référence à une force par unité de surface ($1$~Pa correspond à l'unité~N~m$^{-2}$). Paramètre macroscopique, elle est reliée à la quantité de mouvement des molécules microscopiques qui subissent des chocs sur une surface donnée. Les mesures et raisonnements météorologiques font souvent référence à des quantités en hPa ou en mbar. Ces deux unités sont équivalentes : 1~hPa correspond à~$10^{2}$~Pa, 1~mbar correspond à $10^{-3}$~bar, ce qui correspond bien à 1~hPa, puisque 1 bar est $10^{5}$~Pa. La pression atmosphérique vaut $1013.25$~hPa (ou mbar) en moyenne au niveau de la mer sur Terre. On utilise parfois l'unité d'$1$~atm (atmosphère) qui correspond à cette valeur de~$101325$~Pa.
\item La \voc{masse volumique} ou densité~$\rho$ est exprimée en~kg~m$^{-3}$ dans le système international. Elle représente une quantité de matière par unité de volume. Elle vaut environ $1.217$~kg~m$^{-3}$ proche de la surface sur Terre.
\end{finger}

	\sk
Les trois paramètres pression, température, densité varient en théorie selon les trois directions de l'espace. On constate cependant que, pour une unité de longueur donnée, leurs variations selon la verticale sont beaucoup plus significatives que leurs variations selon l'horizontale. On peut donc définir une structure moyenne en fonction de l'altitude dont sera toujours relativement proche la structure verticale en chaque jour et chaque région de la planète. 

\figsup{0.48}{0.35}{\figfrancis/WH_vert_struct}{\figpayan/LP211_Chap1_Page_03_Image_0001.png}{[Gauche] Structure verticale de la pression, la densité et du libre parcours moyen des molécules (distance moyenne parcourue par une molécule avant de subir un choc sur une autre molécule). Noter l'échelle logarithmique en abscisse : une droite sur ce schéma dénote donc une variation exponentielle des quantités avec l'altitude. [Droite] Plus haut dans l'atmosphère, la variation verticale de la pression est dépendante pour plusieurs ordres de grandeur avec l'activité solaire. Les courbes indiquées correspondent respectivement à une très faible activité solaire (température de la thermopause de 600 K) et une très forte activité solaire (température de la thermopause de 2000K).}{fig:presvert}

\sk
Pression et densité décroissent de façon approximativement exponentielle selon l'altitude~$z$ [figure \ref{fig:presvert}] $$ P\sim P_0 \, e^{-z/H} $$ où $H$ est appelée \voc{hauteur d'échelle} et~$P_0$ une valeur de pression de référence. Cette loi de variation découle du fait que la pression atmosphérique à une altitude~$z$ est due au poids de la colonne d'air située au-dessus de l'altitude~$z$. En pratique sur Terre, la pression est divisée par deux environ tous les 5 km. On évalue la masse de l'atmosphère terrestre à~$5 \times 10^{18}$~kg, soit environ un millionième de la masse de la Terre. La moitié de la masse de l'atmosphère se situe au dessous de~$5500$~m, les~$2/3$ au dessous de~$8400$~m, les~$3/4$ au dessous de~$10300$~m, les~9/10~au dessous de~$16100$~m. Si l'on considère que les neuf dixièmes de l’atmosphère sont situés dans les $16$~premiers kilomètres, l’atmosphère ne forme donc qu'une mince pellicule gazeuse en comparaison des $6367$~km du rayon terrestre. On dit que l'on peut faire l'\voc{approximation de la couche mince}.


	\sk \subsection{Structure verticale : couches atmosphériques}
	\sk
Les variations verticales de température sont très différentes des variations de pression et de densité: la température décroît et augmente alternativement avec l'altitude%, de façon quasi-linéaire [figure \ref{fig:tempvert}], en restant comprise entre environ~$200$ et~$300$~K. 
Cette structure verticale de la température permet de diviser l'atmosphère en un certain nombre de couches aux propriétés différentes, dont les noms comportent le suffixe \emph{-sphère}. La limite supérieure d'une couche atmosphérique donnée porte un nom similaire, où le suffixe \emph{-sphère} est remplacé par le suffixe \emph{-pause}. Par exemple, la limite entre la troposphère et la stratosphère s'appelle la \voc{tropopause}. Les couches atmosphériques en partant de la surface vers l'espace sont détaillées ci-dessous.

\sk
\begin{description} 
\item[La \voc{troposphère}] \normalsize s'étend jusqu'à environ 11 km d'altitude et contient 80\% de la masse de l'atmosphère. La température y décroit en moyenne de 6.5\deg C par kilomètre (nous verrons pourquoi dans un chapitre ultérieur). La troposphère est une couche relativement bien mélangée sur la verticale (échelle de temps de quelques jours), sauf en certaines couches minces, appelées \voc{inversions}, où la température décroit peu ou même augmente avec l'altitude. La troposphère est la couche où ont lieu la plupart des phénomènes météorologiques acessibles à l'expérience humaine (par exemple, les nuages montrés en figure~\ref{fig:blue}). La partie inférieure de la troposphère contient la \voc{couche limite atmosphérique} située juste au dessus de la surface, d'épaisseur variant de quelques centaines de~m à 3 km et définie comme la partie de l'atmosphère influencée par la surface sur de courtes échelles de temps (typiquement un cycle diurne). La couche limite atmosphérique est le siège de mouvements turbulents organisés au cours de l'après-midi qui opèrent un mélange des espèces chimiques qui y sont émises. \normalsize
\item[La \voc{stratosphère}] \normalsize est située au dessus de la troposphère. L'altitude au-dessus du sol de la tropopause peut varier entre~$5$ et~$15$~km. Contrairement à la troposphère, la stratosphère contient très peu de vapeur d'eau (à cause des températures très basses rencontrées à la tropopause) mais la majorité de l'ozone~O$_3$. L'absorption par l'ozone du rayonnement solaire \voc{ultraviolet}, de longueur d'onde moindre que le rayonnement visible et plus énergétique, explique que la température dans la stratosphère est d'abord isotherme, puis augmente avec l'altitude jusqu'à un maximum à la stratopause. Cette structure verticale très stable inhibe fortement les mouvements verticaux, ce qui explique que la stratosphère soit organisée en couches horizontales (comme l'indique l'étymologie de son nom). Le temps de résidence de particules dans la stratosphère est très long à cause de l'absence de nuages et précipitations. \normalsize
\item[La \voc{mésosphère}] \normalsize voit sa température décroître selon la verticale. Contrairement à la troposphère, elle ne contient pas de vapeur d'eau et contrairement à la stratosphère, elle ne contient que peu d'ozone. Elle se situe sur Terre à des altitudes entre~$50$ et~$85$~km. La mésopause est souvent le point le plus froid de l'atmosphère terrestre, la température peut y atteindre des valeurs aussi basses que~$130$~K. \normalsize
\end{description}

\figside{0.45}{0.3}{\figfrancis/WH_stdatm}{Structure verticale idéalisée de la température correspondant au profil moyenné global annuel.}{fig:tempvert}

	\sk
\begin{description} 
\item[La \voc{thermosphère}] s'étend jusque des altitudes très élevées (800 km) et voit sa température contrôlée par l'absorption du rayonnement solaire ultraviolet. La température dans la thermosphère varie souvent d'un facteur deux suivant l'activité solaire et l'alternance jour-nuit. Les aurores surviennent dans cette couche atmosphérique. Les missions spatiales \ofg{basse orbite} telles que la Station Spatiale Internationale sont localisées au milieu de la thermosphère. \normalsize
\item[L'\voc{exosphère}] est située au-dessus de la thermosphère à partir d'une altitude d'environ~$800$~km sur Terre. Il s'agit de la zone où l'atmosphère subit un \voc{échappement} : les molécules peuvent s'échapper vers l'espace sans que des chocs avec d'autres molécules ne les renvoient dans l'atmosphère. L'exosphère constitue la dernière zone de transition entre l'atmosphère et l'espace. \normalsize
\end{description}

\figside{0.45}{0.3}{\figpayan/LP211_Chap1_Page_05_Image_0001.png}{Structure verticale idéalisée de la température étendue aux hautes atmosphères.}{fig:tempvert}
%% Voir figure~\ref{fig:presvert} pour la distinction entre figure de gauche et figure de droite

\sk
D'autres couches atmosphériques sont définies non pas à partir de la température mais à partir des propriétés électriques de l'atmosphère terrestre. On fait référence ici au vent solaire, qui est un flux de particules chargées (ions et électrons) formant un plasma qui s’échappe en permanence du Soleil vers l’espace interplanétaire
\begin{description}
\item[L'ionosphère] Comme son étymologie l'indique, l’ionosphère est une région de notre haute atmosphère contenant des ions et des électrons formés par photo-ionisation des molécules neutres qui s’y trouvent. C’est le Soleil, et plus particulièrement ses rayonnements énergétiques ultraviolets et X, mais aussi les particules du vent solaire et le rayonnement cosmique, qui sont à l’origine de cette ionisation de la haute atmosphère. L’ionosphère se situe entre~$50$ et~$1000$ kilomètres d’altitude (elle s'étend donc de la mésosphère à la thermosphère). L’ionosphère est habituellement divisée horizontalement en différentes couches, baptisées D, E et F dans lesquelles l’ionisation croît avec l’altitude. Ces couches proviennent des différences de pénétration dans l’atmosphère des rayonnements solaires selon leur énergie.
\item[La magnétosphère] Notre planète génère son propre champ magnétique, un peu à la manière d’une dynamo. C’est la différence de vitesse entre la rotation de la planète et de son coeur liquide qui, par induction, génère ce champ magnétique. Ce champ magnétique protège la Terre des agressions extérieures comme les rayons cosmiques et les particules énergétiques du vent solaire. Cette zone protégée s'appelle magnétosphère. Elle démarre au dessus de l’ionosphère, à plusieurs milliers de kilomètres de la surface du sol, et s’étend jusqu’à 70 000 kilomètres environ du côté du Soleil. Du côté opposé, la queue de la magnétosphère s’étire sur plusieurs millions de kilomètres. Les contours de la magnétosphère évoluent continuellement sous l’action du vent solaire et de sa variabilité.
\end{description}
\normalsize



\mk \section{Composition atmosphérique}
	
	\sk \subsection{Un mélange de gaz parfaits}
	\sk
On appelle \voc{gaz parfait} un gaz suffisament dilué pour que les interactions entre les molécules du gaz, autres que les chocs, soient négligeables. L'air composant l'atmosphère peut être considéré en bonne approximation comme un mélange de gaz parfaits\footnote{On peut en général considérer que le gaz est parfait si~$P < 1$~kbar. C'est le cas dans la plupart des atmosphères planétaires rencontrées. Il n'y a guère qu'au coeur des planètes géantes, où la pression dépasse cette limite, que l'approximation du gaz parfait doit être abandonnée.} notés~$i$, dont le nombre de moles est~$n_i$ pour un volume donné~$V$ d'air à la température~$T$. Chaque espèce gazeuse composant l'air est caractérisée par une \voc{pression partielle}~$P_i$ qui est définie comme la pression qu'aurait l'espèce gazeuse si elle occupait à elle seule le volume~$V$ à la température~$T$. Chacune de ces espèces gazeuses~$i$ se caractérise par une même température~$T$ et vérifie l'équation d'état du gaz parfait $$ P_i \, V = n_i \, R^* \, T $$ où~$R^*$=8.31 J~K\mo~mol\mo~est la constante des gaz parfaits (produit du nombre d'Avogadro et de la constante de Boltzmann). La pression totale de l'air~$P$ est, d'après la loi de Dalton, la somme des pressions partielles~$P_i$ des espèces gazeuses composant le mélange $P=\Sigma P_i$. En faisant la somme des lois du gaz parfait appliquées pour chacune des espèces gazeuses, on obtient $$ P \, V = \big( \Sigma n_i \big) \, R^* \, T $$ ce qui montre qu'un mélange de gaz parfaits est aussi un gaz parfait. Cette équation permet de relier la pression totale~$P$ à la température~$T$, mais présente l'inconvénient de contenir les grandeurs \voc{extensives}~$V$ et $n_i$ qui dépendent du volume d'air considéré. Il reste donc à donner une traduction intensive à la loi du gaz parfait pour un mélange de gaz. La masse totale contenue dans le volume~$V$ peut s'écrire $m=\Sigma n_iM_i$ où $M_i$ est la masse molaire du gaz $i$. En divisant l'équation précédente par $m$, et en utilisant la définition de la masse volumique~$\rho = m / V$, on obtient $$ \frac{P}{\rho} = \frac{\Sigma n_i}{\Sigma n_iM_i} \, R^* \, T $$ Or, d'une part, la \voc{masse molaire de l'air} composé d'un mélange de gaz~$i$ est $$ \boxed{ M=\frac{\Sigma n_iM_i}{\Sigma n_i} }$$ et d'autre part, on peut définir une \voc{constante de l'air sec} de la façon suivante $$ R=\frac{R^*}{M} $$ On a alors l'équation des gaz parfaits pour l'air atmosphérique qui permet de relier les trois paramètres intensifs importants : pression~$P$, température~$T$ et densité~$\rho$ $$ \boxed{ P = \rho \, R \,T } $$ L'état thermodynamique d'un élément d'air est donc déterminé uniquement par deux paramètres sur les trois~$P$, $T$, $\rho$. En météorologie par exemple, on travaille principalement avec la pression et la température qui sont plus aisées à mesurer que la densité. Les valeurs numériques de~$M$, et donc~$R$, dépendent de la planète considérée et de sa composition atmosphérique. 



	\sk \subsection{Composition moléculaire}
	\sk
La concentration, au sens où elle est définie dans la plupart des cours de physique / chimie, est une quantité très peu utilisée en sciences de l'atmosphère. La composition chimique de l'atmosphère s'exprime préférentiellement en utilisant le \voc{rapport de mélange volumique}, soit la proportion d'un volume d'air occupée par un gaz particulier. L'air étant un gaz parfait, ce rapport de mélange volumique est simplement égal au rapport du nombre de molécules/atomes du gaz sur le nombre total de molécules d'air $$ \boxed{ r_i=\frac{n_i}{\Sigma n_k} } \qquad \textrm{\footnotesize (parfois également noté $q_i$)} $$ D'après la loi de Dalton, il correspond également au rapport entre la pression partielle du gaz considéré avec la pression totale du mélange. Le rapport de mélange n'est exprimé en pourcentage que pour les composés les plus abondants. Pour les \voc{gaz traces}, moins abondants, on exprime le rapport de mélange en parties par million (1 ppmv = $10^{-6}$), ou par milliards (1 ppbv = $10^{-9}$), voire le pptv tel que~1 pptv = $10^{-12}$. Dire que le rapport de mélange de CO$_2$ au sol est d’environ 380 ppmv signifie que sur un million de molécules d’air, 380 sont des molécules de CO$_2$. %On utilise également parfois le rapport de mélange massique, soit la proportion d'une masse d'air occupée par un gaz particulier. 


	\begin{table}
\centering
\begin{tabular}{lcc}
%\toprule
%\makecell[l]{\textbf{Constituant}} &
%\makecell{\textbf{Masse}\\\textbf{Molaire}} &
%\makecell{\textbf{Rapport}\\\textbf{de Mélange}} \\
%\midrule
\hline
\textbf{Constituant} &
\textbf{Masse molaire} &
\textbf{Rapport de mélange} \\
\hline
Azote (N$_2$) & 28 & 78\% \\
Oxygène (O$_2$) & 32 & 21\% \\
Argon (Ar) & 40 & 0.93\% \\
\textbf{Vapeur d'eau (H$_2$O)} & 18 & 0-5\% \\
\textbf{Dioxyde de Carbone (CO$_2$)} & 44 & 380 ppmv \\
Néon (Ne) & 20 & 18 ppmv \\
Hélium (He) & 4 & 5 ppmv \\
\textbf{Méthane (CH$_4$)} & 16 & 1.75 ppmv \\
Krypton (Kr) & 84 & 1 ppmv \\
Hydrogène (H$_2$) & 2 & 0.5 ppmv \\
\textbf{Oxide nitreux (N$_2$O)} & 56 & 0.3 ppmv \\
\textbf{Ozone (O$_3$)} & 48 & 0-0.1 ppmv \\
%\bottomrule
\hline
\end{tabular}
\caption{\emph{Principaux composants de l'atmosphère. Les gaz à effet de serre sont indiqués en gras.}}
\label{tab:compos}
\end{table}


\sk
L'azote et l'oxygène dominent largement la composition de l'atmosphère terrestre (tableau \ref{tab:compos}), suivis par l'argon et d'autres gaz rares beaucoup moins abondants. Les rapports de mélange de vapeur d'eau et d'ozone sont très variables : la vapeur d'eau est présente surtout dans la troposphère, avec un maximum près de la surface et dans les tropiques, alors que l'ozone est principalement présente dans la stratosphère. Un certain nombre de gaz traces sont émis régulièrement au niveau de la surface, par des phénomènes naturels ou les activités humaines. Leur répartition dépend alors beaucoup de leur \voc{durée de vie} dans l'atmosphère. Le CO$_2$ qui est très stable est bien mélangé. Le méthane, qui a une durée de vie d'une dizaine d'années, est bien réparti dans la troposphère mais son rapport de mélange varie dans la stratosphère. Des polluants à durée de vie courte (quelques jours) comme l'ozone troposphérique, se retrouveront surtout au voisinage des sources. Les activités humaines ont également contribué à modifier le rapport de mélange de certains de ces gaz (par exemple, le~CO$_2$).

\sk
\begin{finger}
\item En employant les formules obtenues à la section précédente, et les informations ci-dessus sur la composition de l'atmosphère terrestre, il est possible de déterminer des valeurs numériques utiles
\begin{citemize}
\item La masse molaire de l'air est~$M = 28.966$~g~mol$^{-1}$ (on emploie souvent~$M \simeq 29$~g~mol$^{-1}$). 
\item La constante de l'air sec est~$R = 287$~J~K$^{-1}$~kg$^{-1}$.
\end{citemize}  
\end{finger}

	\sk
\begin{finger}
\item Dans toute discussion de la composition atmosphérique, il est important de faire la distinction entre composés minoritaires et majoritaires [figure~\ref{fig:minor}]. Alors que les composés majoritaires suivent une distribution verticale en accord avec l'état énergétique et dynamique de l'atmosphère globale, les composés minoritaires peuvent avoir des comportements très différents qui dépendent à la fois des mécanismes photochimiques de production et de perte ainsi que des phénomènes de transport.
\item La composition de l'air donnée ici est valide sur les premiers~$80$ à~$100$ kilomètres d'altitude, à part quelques constituants mineurs. On appelle cette région l'\voc{homosphère}, elle correspond approximativement à la troposphère, la stratosphère et la mésosphère (Figure \ref{fig:tempvert}). Dans l'homosphère, l'atmosphère est un mélange homogène de différents gaz, l'échelle de hauteur est la même pour tous les gaz. Au dessus de cette altitude, le libre parcours moyen des molécules devient très grand et on a une \ofg{décantation} où les éléments plus légers dominent progressivement aux altitudes élevées. Chaque composant suit sa propre échelle de hauteur. On parle alors d'\voc{hétérosphère}; elle regroupe approximativement la thermosphère et l'exosphère. 
\end{finger}
%Au niveau du sol, l'atmosphère standard sèche est caractérisée par une pression d’environ 1013 hPa et une concentration totale de 2,69 x 1019 molécule~cm$^{-3}$ lorsque la température est de 273~K.

\figun{0.7}{0.3}{\figpayan/LP211_Chap1_Page_06_Image_0001.png}{Composition de l’atmosphère~: des espèces en très faibles quantités jouent un rôle très important. Sur la figure sont données quelques mesures de constituants minoritaires dans l’homosphère. Les courbes en trait fin correspondent aux concentrations résultant de rapports de mélange volumiques constants de~$10^{-1}$ à~$10^{-13}$. (Source: Kockarts, Aéronomie, 2000).}{fig:minor}






\begin{itemize}
\item Fractionnement isotopique: Homosphère-hétérosphère
\begin{itemize}
\item A haute altitude $z>z_h$, la diffusion turbulente est moins efficace que la \voc{diffusion moléculaire} : \voc{hétérosphère}
\item Homopause : $z_h \sim 100\,km$ (telluriques), $\sim 750\,km$ (Titan)
\item Chaque espèce suit alors sa propre échelle de hauteur : les plus légères deviennent plus abondantes à haute altitude
\end{itemize}
\item 2 facteurs influencent le fractionnement
{\footnotesize \begin{description}
\item[\'Equilibre diffusif] : différence de composition atmosphérique entre l'homopause et l'exobase
\item[\'Echappement différentiel] : $V_{th}/V_e$ plus grand pour les isotopes légers.
\end{description} }
%\item Exemples
%\begin{itemize}
%\item {\bf D/H} (Vénus)
%\item {\bf $^{14}$N/$^{15}$N}, {\bf $^{16}$O/$^{18}$O} (Mars)
%\end{itemize}
\end{itemize}


Note: Diffusion turbulente, paramétrisée par un coefficient de diffusion turbulente qui le fait ressembler à un paramètre de diffusion moléculaire.
mais provient de la convection thermique (flottaison) dans la basse atmosphère et les ondes de gravité dans la haute atmosphère

	\figun{1}{0.35}{decouverte/cours_dyn/composition.png}{Objets du système solaire présentant une atmosphère substantielle et leurs caractéristiques.}{fig:composition}

\sk
Les compositions atmosphériques sont très distinctes selon les planètes du système solaire considérées [figure~\ref{fig:composition}]. L'atmosphère terrestre est unique parmi les atmosphères des autres planètes du système solaire. Elle est riche en azote et oxygène et pauvre en \carb, contrairement aux atmosphères de Venus et Mars, qui contiennent plus de~$90\%$ de \carb. On pourrait penser que la Terre a acquis son atmosphère pendant sa formation à partir des gaz présents dans la nébuleuse solaire. Une telle atmosphère serait alors primaire, et contiendrait des gaz de composition cosmique, c'est-à-dire similaire aux abondances chimiques du système solaire. Or, les gaz dominants dans le système solaire sont l'hydrogène et l'hélium. Ces gaz légers sont pratiquement absents dans notre atmosphère, car la gravitation terrestre est trop faible pour les retenir. Les planètes géantes comme Jupiter ou Saturne ont conservé ces gaz primordiaux dans leur atmosphère, au contraire des planètes internes du système solaire Venus, Terre et Mars qui ont des atmosphères de composition bien différente. Si la Terre a eu une telle atmosphère primaire pendant sa formation, elle l'a perdu rapidement. L'atmosphère actuelle doit donc être secondaire. L'évolution de sa composition résulte en partie de l'apparition de la vie [figure~\ref{fig:vie}].
%% parler des puits de carbone et des carbonates. comparaison entre Vénus et Mars.

\figun{0.75}{0.32}{\figpayan/vie.png}{Evolution parallèle de l’atmosphère et de la vie.}{fig:vie}


	\sk \subsection{Aérosols et hydrométéores}
	%%\footnote{Cette partie est inspirée d'éléments trouvés dans le cours de S. Jacquemoud de \emph{Méthodes physique en télédétection.}}

\sk
Les \voc{aérosols} sont constitués de petites particules solides ou liquides en suspension dans les basses couches de l’atmosphère. Environ trois milliards de tonnes de particules sont injectés chaque année dans l’atmosphère par les processus naturels ou les activités humaines. On distingue plusieurs types d'aérosols.
\begin{citemize}
\item[\emph{poussières d'origine désertique}] Il s'agit de la première source mondiale d’aérosols. Elles sont soulevées par des vents violents lors des tempêtes de sable. Les grosses particules retombent rapidement au sol alors que les plus petites forment un nuage sec qui peut s’élever jusqu’à 4-6 km d’altitude et s’étendre sur des milliers de kilomètres. On peut retrouver en Europe ou en Amérique des particules en provenance d'Afrique.
\item[\emph{aérosols solubles dans l’eau}] Ils peuvent être d'origine naturelle (substances organiques émises par la végétation) ou liés à l'activité industrielle (sulfates, nitrates).
\item[\emph{aérosols d'origine marine}] Ils sont formés à partir des bulles résultant du déferlement des vagues et des courants marins. Outre le dioxyde de carbone, les bulles transportent quantité de substances, notamment des particules de sel microscopiques (NaCl) qui participent à la formation de brumes. L'éclatement de ces bulles à la surface des océans donne naissance à un très grand nombre de gouttelettes (parfois une centaine pour une bulle d'un diamètre de l'ordre du millimètre) qui ne se brisent pas et sont à l'origine des aérosols marins (également appelés embruns).
\item[\emph{aérosols carbonés}] Ils sont présents dans les régions tropicales et boréales en raison de nombreux feux de forêt. Par exemple les brûlis de la végétation intertropicale en période sèche occasionnent chaque année des brumes sèches qui disparaissent une fois les pluies revenues.
\item[\emph{aérosols de sulfates}] Ils sont d'origine volcanique. Le dioxyde de soufre SO$_2$ émis lors des éruptions volcaniques produit ces fines particules d'acide sulfurique (SO$_2$ + H$_2$O~$\rightarrow$~H$_2$SO$_4$) qui s'entourent de glace et forment avec les cendres un écran empêchant le rayonnement solaire d'arriver jusqu'au sol.
\end{citemize}
La plupart des aérosols se trouvent dans la troposphère où ils résident en moyenne une semaine. En raison de leur petite taille, les aérosols peuvent être transportés sur de longues distances. Ils sont en général ramenés au sol par les précipitations (pluie, neige). Les aérosols de plus petite taille ($0.01-0.1$~$\mu$m) jouent un rôle important de \voc{noyaux de condensation} dans la formation des nuages en favorisant la condensation de vapeur d'eau en gouttelettes d'eau et/ou de cristaux de glace. Les aérosols de plus grande taille (0.1-1.0~$\mu$m), les plus nombreux, interceptent la lumière du Soleil. La stratosphère contient aussi des aérosols (principalement d'origine volcanique) jusqu'à 18-20 km d’altitude. Contrairement aux aérosols troposphériques, leur concentration est relativement uniforme et leur durée de vie beaucoup plus longue, de plusieurs mois à plusieurs années.  

\sk
Certaines molécules peuvent s’agréger pour former des particules liquides ou solides. Dans l'atmosphère terrestre, ceci concerne principalement l'eau à l'état liquide ou solide dans l'atmosphère qui participe à la formation d'\voc{hydrométéores}. Ce sont des particules d’eau liquide (gouttelettes d'eau) et/ou solide (cristaux de glace) suspendues dans l'atmosphère dont la taille varie de~$1$~$\mu$m à~$1$~cm. Les brumes et nuages sont formés de ces fines gouttelettes d'eau en suspension dans l'atmosphère, qui apparaissent dès que le seuil de saturation de l'air en vapeur d'eau est dépassé (ces mécanismes sont détaillés dans un chapitre ultérieur). Les nuages bas et intermédiaires sont constitués de gouttelettes d’eau liquide ; les nuages d’altitude de cristaux de glace de différentes formes géométriques. Certains sont accompagnés de précipitations lorsque les gouttes ou cristaux sont assez gros pour former de la pluie, neige, grêle ou verglas. %La figure~\ref{fig:droplet} donne un ordre d'idée des tailles respectives des noyaux de condensation, des gouttelettes nuageuses et des gouttes de pluie.
%\figside{0.4}{0.15}{decouverte/cours_meteo/gouttes.png}{Taille comparée des noyaux de condensation (\emph{cloud condensation nuclei}), des gouttelettes de brume ou de nuage (\emph{moisture droplets}), des gouttes de pluie (\emph{raindrop})}{fig:droplet}



\mk \section{Parcelle d'air}

	\sk \subsection{Définition et caractérisation}
	\sk
L'atmosphère est composée d'un ensemble de molécules microscopiques et l'on s'intéresse aux effets de comportement d'ensemble, qualifiés de \voc{macroscopiques}. Les variables thermodynamiques utilisées pour décrire l'atmosphère (pression~$P$, température~$T$, densité~$\rho$) sont des grandeurs macroscopiques \voc{intensives} dont la valeur ne dépend pas du volume d'air considéré. 
%Une façon d'y parvenir est d'utiliser des grandeurs volumiques ou massiques.

\sk
Le système que l'on étudie est appelé \voc{parcelle d'air}. Il s'agit d'un volume d'air dont les dimensions sont %à la fois
\begin{citemize}
\item assez grandes pour contenir un grand nombre de molécules et pouvoir moyenner leur comportement, afin d'exprimer un équilibre thermodynamique~;
\item assez petites par rapport au phénomène considéré, afin de décrire le fluide atmosphérique de façon continue~; la parcelle d'air peut donc être considérée comme un volume élémentaire.
\end{citemize}
On peut donc supposer que les variables macroscopiques d'intérêt sont quasiment constantes à l'échelle de la parcelle. Autrement dit, une parcelle est caractérisée par sa pression~$P$, sa température~$T$, sa densité~$\rho$. Les limites d'une parcelle sont arbitraires, mais ne correspondent pas en général à des barrières physiques. 



	
\sk
Tout le but de ce chapitre est de décrire les relations thermodynamiques qui lient les grandeurs intensives qui caractérisent l'état de la parcelle. Une première de ces relations a été obtenue en introduction~: il s'agit de l'équation des gaz parfaits pour l'air atmosphérique, qui relie les trois paramètres intensifs $P$, $T$ et $\rho$ 
\[ \boxed{ P = \rho \, R \,T } \] 
avec la \voc{constante de l'air sec} $R=\frac{R^*}{M}$ où~$R^*$ est la constante des gaz parfaits et~$M$ est la masse molaire de l'air. On rappelle que sur Terre~$R = 287$~J~K$^{-1}$~kg$^{-1}$. L'état thermodynamique d'une parcelle d'air est donc déterminé uniquement par deux paramètres sur les trois~$P$, $T$, $\rho$. Pour les applications météorologiques, on caractérise en général l'état de la parcelle par sa pression~$P$ et sa température~$T$, plus aisées à mesurer, par exemple via des ballons-sondes, que la densité~$\rho$.



	\sk \subsection{Parcelle et environnement} \label{parcenv}
	\sk
Une parcelle est en \voc{équilibre mécanique} avec son environnement, c'est-à-dire que la pression de la parcelle~$P\e{p}$ et la pression de l'environnement~$P\e{e}$ dans lequel elle se trouve sont égales
\[ \boxed{P\e{p} = P\e{e}} \]
Néanmoins, une parcelle n'est pas en général en \voc{équilibre thermique} avec son environnement, c'est-à-dire que la température de la parcelle et la température de l'environnement dans lequel elle se trouve ne sont pas nécessairement égales
\[ \boxed{T\e{p} \neq T\e{e}} \]
Cette dernière propriété provient du fait que l'air est un très bon isolant thermique\footnote{Une telle propriété est utilisée dans le principe du double vitrage}.




\mk \section{Équilibre hydrostatique}

	\sk \subsection{Bilan des forces}
	\sk
On considère une parcelle d'air cubique de dimensions élémentaires~$(\dd x,\dd y, \dd z)$, au repos et située à une altitude~$z$. La pression atmosphérique vaut~$P(z)$ sur la face du dessous et~$P(z+\dd z)$ sur la face du dessus. Pour le moment, on ne considère pas de variations de pression~$P$ selon l'horizontale\footnote{En pratique, les variations de pression selon l'horizontale sont effectivement négligeables par rapport aux variations de pression selon la verticale. On revient sur ce point dans le chapitre consacré à la dynamique}. Il y a équilibre des forces s'exerçant sur cette parcelle. On appelle \voc{équilibre hydrostatique} l'équilibre des forces selon la verticale, à savoir~:
\begin{citemize}
\item Son poids de module\footnote{On néglige les variations de~$g$ avec~$z$.}~$m \, g$ (où~$m = \rho \, \dd x \, \dd y \, \dd z$) dirigé vers le bas
\item La force de pression sur la face du dessous de module~$P(z) \, \dd x \, \dd y$ dirigée vers le haut
\item La force de pression sur la face du dessus de module~$P(z+\dd z) \, \dd x \, \dd y$ dirigée vers le bas
\item La force de viscosité qui est négligée
\end{citemize}
On note que, contrairement au poids qui s'applique de façon homogène sur tout le volume de la parcelle d'air, les forces de pression s'appliquent sur les surfaces frontières de la parcelle d'air. 
Pour une parcelle au repos, la résultante selon la verticale des forces de pression exercées par le fluide environnant (ici, l'air) n'est autre que la poussée d'Archimède.
%%Par ailleurs, l'équilibre hydrostatique suppose implicitement que la parcelle est à l'équilibre thermique avec son environnement~$T\e{p} = T\e{e}$ soit~$\rho\e{p} = \rho\e{e}$. On aborde le cas général où~$T\e{p} \neq T\e{e}$ dans le chapitre suivant pour définir les notions de stabilité.

\sk
L'équilibre hydrostatique de la parcelle s'écrit donc
\[ - \rho \, g \, \dd x \, \dd y \, \dd z + P(z) \, \dd x \, \dd y - P(z+\dd z) \, \dd x \, \dd y = 0 \qquad \Rightarrow \qquad - \rho \, g \, \dd z + P(z) - P(z+\dd z) = 0 \]
soit en introduisant la dérivée partielle suivant~$z$ de~$P$
\[ \frac{P(z+\dd z) - P(z)}{\dd z} = \boxed{\EH} \]
Cette relation est appelée \voc{équation hydrostatique} (ou relation de l'équilibre hydrostatique). Elle indique que, pour la parcelle considérée, la résultante des forces de pression selon la verticale équilibre la force de gravité. En principe, cette équation est valable pour une parcelle au repos. Par extension, elle est valable lorsque l'accélération verticale d'une parcelle est négligeable devant les autres forces. L'équation hydrostatique est en excellente approximation valable pour les mouvements atmosphériques de grande échelle. 

\sk
Si l'on intègre la relation hydrostatique entre deux niveaux~$z_1$ et~$z_2$ où la pression est~$P_1$ et~$P_2$, on obtient
\[ \Delta P = P_2 - P_1 = - g \, \int_{z_1}^{z_2} \rho \, \dd z \]
L'équilibre hydrostatique peut donc s'interpréter de la façon éclairante suivante~: la différence de pression entre deux niveaux verticaux est proportionnelle à la masse d'air (par unité de surface) entre ces niveaux. Une autre façon équivalente de formuler cela est de dire que la pression atmosphérique à une altitude~$z$ correspond au poids de la colonne d'air située au-dessus de l'altitude~$z$, exercé sur une surface unité de~$1$~m$^2$. Il s'ensuit que la pression atmosphérique~$P$ est décroissante selon l'altitude~$z$. Ainsi, la pression peut être utilisée pour repérer une position verticale à la place de l'altitude. En sciences de l'atmosphère, la pression atmosphérique est une coordonnée verticale plus naturelle que l'altitude~: non seulement elle est directement reliée à la masse atmosphérique par l'équilibre hydrostatique, mais elle est également plus aisée à mesurer.


	\sk \subsection{\'Equation hypsométrique}

		\sk \subsubsection{\'Echelle de hauteur}
		\sk
En exprimant la densité~$\rho$ en fonction de l'équation des gaz parfaits, l'équilibre hydrostatique s'écrit
\[ \Dp{P}{z} = - g \, \frac{P}{RT} \]
On peut intégrer cette équation si on suppose que l'on connaît les variations de~$T$ en fonction de $P$ ou $z$. On suppose ici que l'on peut négliger les variations de pression selon l'horizontale devant les variations suivant la verticale, donc transformer les dérivées partielles~$\partial$ en dérivées simples~$\dd$. On effectue ensuite une séparation des variables
\[R \, T \, \frac{\dd P}{P} = - g \, \dd z\]

\sk
Cette équation peut s'écrire sous une forme dimensionnelle simple à retenir
\[ \boxed{ \frac{\dd P}{P} = - \frac{\dd z}{H(z)} \qquad \text{avec} \qquad H(z) = \frac{R \, T(z)}{g} } \]
La grandeur~$H$ se dénomme l'\voc{échelle de hauteur} et dépend des variations de la température~$T$ avec l'altitude~$z$. L'équation ci-dessus indique bien que la pression décroît avec l'altitude selon une loi exponentielle comme observé en pratique. Cette loi peut être plus ou moins complexe selon la fonction~$T(z)$. On peut néanmoins fournir une illustration simple du résultat de l'intégration dans le cas d'une atmosphère isotherme~$T(z)=T_0$
\[ P(z) = P(z=0) \, e^{-z/H} \qquad \text{avec} \qquad H = R \, T_0 / g \]




		\sk \subsubsection{\'Epaisseur des couches atmosphériques~: équation hypsométrique}
		\sk
Dans l'équation de l'échelle de hauteur, faire l'hypothèse isotherme est très simpliste et rarement rencontré en pratique dans l'atmosphère. On se place dans le cas plus général, bien que toujours simplifié, de deux niveaux atmosphériques~$a$ et~$b$ entre lesquels la température ne varie pas trop brusquement avec l'altitude~$z$. On réalise alors l'intégration entre les deux niveaux~$a$ et~$b$
\[R \, T \, \frac{\dd P}{P} = - g \, \dd z \qquad \Rightarrow \qquad R \, \int_a^b T\, \frac{\dd P}{P} = - g \, \int_a^b dz\]
puis on définit la température moyenne de la couche atmosphérique entre~$a$ et~$b$ avec une moyenne pondérée
\[ \langle T \rangle = \frac{\int_a^b T \, \frac{\dd P}{P}}{\int_a^b \frac{\dd P}{P}} \]
pour obtenir finalement
\[R \, \langle T \rangle \, \int_a^b \frac{\dd P}{P} = - g \, \int_a^b dz
\qquad \Rightarrow \qquad \boxed{ g \, (z_a - z_b) = R \, \langle T \rangle \ln \left( \frac{P_b}{P_a} \right) } \]
Cette relation est appelée \voc{équation hypsométrique}. Elle correspond à une formulation utile en météorologie du principe que \ofg{l'air chaud se dilate}. Les conséquences de l'équation hypsométrique peuvent s'exprimer de diverses façons équivalentes.
\begin{citemize}
\item Pour une masse d'air donnée, une couche d'air chaud est plus épaisse.
\item La distance entre deux isobares est plus grande si l'air est chaud.
\item La pression diminue plus vite selon l'altitude dans une couche d'air froid.
\end{citemize}
En passant le résultat précédent au logarithme, on note que l'on retrouve toujours le fait que la pression diminue avec l'altitude selon une loi exponentielle. En notant l'échelle de hauteur moyenne~$\langle H \rangle$, on a
\[ P_b = P_a \, e^{ - \frac{z_b - z_a}{\langle H \rangle}} \qquad \text{avec} \qquad \langle H \rangle = \frac{R \, \langle T \rangle}{g} \]





	\sk \subsection{Applications pratiques}

		\sk \subsubsection{Expérience de Pascal}
		\sk
Depuis son invention en 1643 par Torricelli, disciple de Galilée, le \voc{baromètre} est l'instrument de référence pour mesurer la pression atmosphérique à la surface de l'atmosphère terrestre\footnote{Le but initial de Torricelli était de parvenir le premier à maintenir artificiellement en laboratoire une chambre sous vide. Néanmoins, il est également reporté que l'invention du baromètre découle des réflexions de Torricelli autour de l'impossibilité, constatée en pratique, de pomper l'eau d'un fleuve au-dessus d'un certain niveau.}. Trois ans après son invention, le baromètre était déjà utilisé pour sa première application en sciences de l'atmosphère~: donner une preuve expérimentale de l'équilibre hydrostatique qui gouverne la stratification en pression de l'atmosphère. Autrement dit, le baromètre est un moyen indirect de mesurer la masse de l'atmosphère à travers la pression de surface. Blaise Pascal montra ainsi, par des mesures respectivement sur la Tour Saint-Jacques à Paris ($ \Delta z = 52 \U{m} $ au-dessus du sol) et sur le Puy-de-Dôme en Auvergne ($ \Delta z = 1000 \U{m} $ au-dessus du sol), que la pression atmosphérique varie avec l'altitude\footnote{Le texte original du traité de Pascal est disponible sur Gallica~: \url{http://gallica.bnf.fr/ark:/12148/bpt6k105083f}}.

\sk
L'équation hypsométrique permet de retrouver l'écart relatif en pression mesuré par Blaise Pascal entre le sol et le haut de la Tour Saint-Jacques. Comme les variations mesurées sont petites, on peut les assimiler aux différentielles et on peut négliger les variations de température avec l'altitude. Les variations relatives de pression mesurées par Pascal peuvent alors s'écrire
\[ \f{\Delta P}{P} \simeq \f{g}{R\,T} \, \Delta z \]
L'application numérique avec~$T = 288$~K donne une variation~$\Delta P / P = 0.62 \%$. La variation de pression détectée par Blaise Pascal\footnote{On note que, par une heureuse coincidence, la variation de pression entre le pied et le sommet de la Tour Saint-Jacques est de l'ordre de grandeur de la pression atmosphérique à la surface de Mars, ce qui permet de se représenter la finesse de l'atmosphère sur cette planète, comparé à notre Terre.} est donc d'environ~$6$~hPa (avec la valeur standard de la pression de surface~$P_0 = 1013$~hPa). Bien que cette baisse de pression soit détectable à l'aide du tube de Torricelli, Blaise Pascal a reproduit l'expérience au Puy-de-Dôme avec un écart~$\Delta z$ plus grand pour une meilleure précision quantitative. 



		\sk \subsubsection{Pression au niveau de la mer, altimétrie et cartes météorologiques}
		\sk
Comme le montre la figure~\ref{fig:press} (haut), la pression~$P$ à la surface de la Terre est au premier ordre sensible à l'altimétrie (hauteur topographique), puisque la pression correspond au poids de la colonne d'air située au-dessus du point considéré. Pour produire des cartes de prévision du temps, on souhaite éliminer du champ de pression~$P$ cette composante topographique de premier ordre et mettre en évidence les variations de second ordre digne d'intérêt en météorologie.

\figsup{0.62}{0.32}{decouverte/cours_meteo/SURFPRESS/outputvar134_200.png}{decouverte/cours_meteo/SURFPRESS/outputvar151_200.png}{Champs de pression prédits au 01/09/2009 par les réanalyses ERA-INTERIM de l'ECMWF (Centre Européen de Prévision du Temps à Moyen Terme). La réprésentation graphique est basée sur une projection stéréographique polaire centrée sur le pôle Nord et les structures topographiques sont ajoutées dans le fond de la figure pour repérage. En haut, le champ de pression brut est tracé en hPa~: les valeurs de pression les plus basses correspondent aux reliefs topographiques les plus élevés. En bas, le champ de pression ramené au niveau de la mer est tracé en hPa~: la composante de premier ordre topographique sur le champ de pression a disparu pour laisser place aux composantes météorologiques de la pression~: dépressions (zones de basses pression) en bleu et anticyclones (zones de haute pression) en rouge. On peut d'ailleurs noter dans ce champ de pression normalisé l'activité ondulatoire de l'atmosphère aux moyennes latitudes. Les cartes de pression des bulletins météorologiques sont exclusivement des cartes de pression ramenées au niveau de la mer comme celle-ci.}{fig:press} 

\sk
Quand la pression de surface est mesurée à une station située à une altitude~$z \ll H$, on peut utiliser l'équation hypsométrique avec la température mesurée à la station pour déterminer une valeur approximative de la pression au niveau de la mer à~$z=0$. On suppose fréquemment que la température décroît linéairement avec l'altitude~$z$ selon un taux constant négatif~$\Gamma\e{e}$ en~$^{\circ}$C/km (ou K/km). On appelle d'ailleurs la loi~$T = T_0 + \Gamma\e{e} \, z$ le profil de l'atmosphère standard. En intégrant entre le niveau de la mer ($z=0, P=P_0$) et la station à ($z,P$), on obtient:
\[ \ln \left( \frac{P_0}{P} \right) = \frac{g}{R \, \Gamma\e{e}} \, \ln \left( \frac{T_0 + \Gamma\e{e} \, z}{T_0} \right) \]
\[ \Rightarrow \qquad P_0 = P \left( 1 + \frac{\Gamma\e{e} \, z}{T_0} \right)^{\frac{g}{R\,\Gamma\e{e}}} \]
La carte météorologique sur la Figure~\ref{fig:press} bas est obtenue en employant cette équation. L'équation qui précède est aussi utilisée avec $P_0=1013.25$~hPa par les altimètres des avions de ligne pour convertir~$P$ mesurée en~$z$.



\mk \section{Premier principe et thermodynamique de l'air sec} 

	\sk \subsection{\'Energie interne, chaleurs molaires et enthalpie}
	\sk
Un système thermodynamique possède, en plus de son énergie d'ensemble (cinétique, potentielle), une \voc{énergie interne}~$U$. Comme la température~$T$, l'énergie interne~$U$ est une grandeur macroscopique qui représente les phénomènes microscopiques au sein d'un gaz. Le premier principe indique que les variations d'énergie interne sont égales à la somme du travail et de
la chaleur algébriquement reçus~:
\[ \dd U = \delta W + \delta Q\]

\sk
Dans le cas d'un gaz parfait, l'énergie potentielle d'interaction des molécules du gaz est négligeable, et l'énergie interne est égale à l'énergie cinétique des molécules, qui dépend seulement de la température. On peut montrer que $U = n \, \frac{\zeta \, R^* \, T}{2}$ où $\zeta$ est le nombre de degrés de liberté des molécules. Pour un gaz (principalement) diatomique comme l'air, $\zeta = 5$. 

\sk
Dans le cas de variations quasi-statiques d'un gaz, ce qui est supposé être le cas dans l'atmosphère, le travail s'exprime en fonction de la pression~$P$ du gaz et de la variation de volume~$\dd V$
\[ \delta W = - P \, \dd V \]

\sk
L'expérience montre que la quantité de chaleur échangée au cours d'une transformation à volume ou pression constant est proportionnelle à la variation de température du système~: $\delta Q = n \, C_V^* \, \dd T$ à volume constant, $\delta Q = n \, C_P^* \, \dd T$ à pression constante. $C_P^*, C_V^*$ sont les \voc{chaleurs molaires}, également appelées \voc{capacités calorifiques}. Il s'agit de l'énergie qu'il faut fournir à un gaz pour faire augmenter sa température de~$1$~K dans les conditions indiquées (à volume constant ou à pression constante). Pour une transformation à volume constant (isochore), $\dd U = \delta Q$ donc $C_V^*=\frac{\zeta \, R^*}{2}$.

\sk
Pour l'étude de l'atmosphère, toujours dans la logique de travailler sur des grandeurs intensives, il est  bien plus utile de s'intéresser aux variations de pression plutôt qu'à celles de volume. On utilise donc l'\voc{enthalpie}~$H = U + P \, V$. On a alors par dérivation $ \dd H = \dd U + \dd (P\,V) $ puis, en utilisant le premier principe
\[ \dd H = V \, \dd P + \delta Q \]
Pour une transformation à pression constante (isobare) on a $\dd H = \delta Q$. On en déduit pour une transformation quelconque\footnote{
D'autre part, en utilisant conjointement la dérivation de l'équation d'état du gaz parfait~$\dd (P\,V) = n \, R^* \, \dd T$ et l'expression de l'énergie interne~$U = n \, C_V^* \, \dd T$, on obtient $\dd H = n \, C_V^* \, \dd T + n \, R^* \, \dd T$ pour une transformation quelconque. On en déduit la relation de Mayer \[ C_P^* = C_V^* + R^* = \frac{(\zeta+2) \, R^*}{2}\]
} 
que $\dd H = n \, C_P^* \, dT$, ce qui permet d'écrire
\[ n \, C_P^* \, dT = V \, \dd P + \delta Q \]





	\sk \subsection{Transformations dans l'atmosphère~: cas général}
	\sk
Afin de travailler sur des grandeurs intensives, on divise la relation précédente par la masse~$m$ de la parcelle pour obtenir
\[ C_P \, \dd T = \frac{\dd P}{\rho} + \delta q \]
où $\delta q$ est la chaleur massique reçue et $C_P = C_P^* / M$ est la \voc{chaleur massique de l'air} ($C_P$=1004 J~K$^{-1}$~kg$^{-1}$). Nous disposons alors d'une autre version du premier principe, très utile en météorologie et valable pour une transformation quelconque d'une parcelle d'air
\[ \boxed{ \underbrace{\textcolor{white}{\frac{R^2}{C_P}} \dd T \textcolor{white}{\frac{R}{C_P}}}_{\text{variation de température de la parcelle}} = \underbrace{\frac{R}{C_P} \, \frac{T}{P} \, \dd P}_{\text{travail expansion/compression}} + \underbrace{\frac{1}{C_P} \, \delta q}_{\text{chauffage diabatique}} } \]

\sk
Autrement dit, la température de la parcelle augmente si elle subit une compression ($\dd P > 0$) et/ou si on lui apporte de la chaleur ($\delta q > 0$). La température de la parcelle à l'inverse diminue si elle subit une détente ($\dd P < 0$) et/ou si elle cède de la chaleur à l'extérieur ($\delta q < 0$). Il est donc important de retenir que la température de la parcelle peut très bien varier quand bien même la parcelle n'échange aucune chaleur avec l'extérieur~: dans ce cas, $\delta q = 0$ et l'on parle de \voc{transformation adiabatique}. 

\sk
L'équation fondamentale ci-dessus est directement dérivée du premier principe, mais prend une forme plus pratique en sciences de l'atmosphère du fait que les transformations que subit une parcelle atmosphérique se réduisent en général aux transformations \voc{isobares} (à pression constante $\dd P = 0$) et aux transformations \voc{adiabatiques} (sans échanges de chaleur avec l'extérieur $\delta q = 0$). Les transformations isothermes, au cours de laquelle la température de la parcelle ne varie pas, sont plus rarement rencontrées en sciences de l'atmosphère.




	\sk \subsection{Transformations non adiabatiques}
	\sk
Dans le cas où la transformation n'est pas adiabatique, les échanges de chaleur~$\delta q$ d'une parcelle d'air avec son environnement sont non nuls et peuvent s'effectuer par~:
\begin{itemize}
\item Transfert radiatif~: l'atmosphère se refroidit en émettant dans l'infrarouge, ou se réchauffe en absorbant du rayonnement électromagnétique dans l'infrarouge [cas des gaz à effet de serre] ou dans le visible [cas de l'ozone dans la stratosphère].
%Ces échanges sont faibles et peuvent être négligés sauf à l'échelle de la circulation générale\footnote{Le refroidissement/réchauffement peut être localement élevé au sommet/à la base de nuages.}
\item Condensation ou évaporation d'eau~: le changement d'état consomme ou relâche de la chaleur (ceci n'a lieu que lorsque l'air est à saturation).
\item Diffusion moléculaire (conduction thermique)~: ces transferts sont très négligeables sauf à quelques centimètres du sol.
\end{itemize}
Un cas notamment souvent cité en météorologie est celui d'une parcelle d'air située proche du sol, à la tombée de la nuit, qui subit peu de variations de pression ($\dd P \sim 0$) mais dont la température diminue sous l'effet du refroidissement radiatif ($\delta q < 0$). Ceci explique la présence de rosée sur le sol et de brouillard proche de la surface au petit matin.




	\sk \subsection{Transformations adiabatiques}
	\sk
Dans de nombreuses situations en sciences de l'atmosphère, on peut considérer que l'évolution de la parcelle est \voc{adiabatique} et se fait sans échange de chaleur avec l'extérieur ($\delta q=0$). En vertu de l'équilibre hydrostatique qui relie pression~$P$ et altitude~$z$~:
\begin{citemize}
\item une parcelle dont l'altitude~$z$ augmente sans apport extérieur de chaleur, subit une \voc{ascendance} adiabatique, donc une détente telle que~$\dd P < 0$ et sa température diminue ;
\item inversement, une parcelle dont l'altitude~$z$ diminue sans apport extérieur de chaleur, subit une \voc{subsidence} adiabatique, donc une compression telle que~$\dd P > 0$ et sa température augmente. 
\end{citemize}

\sk
Dans le cas où la transformation est adiabatique, pression et température sont intimement liées en vetu du premier principe. La version du premier principe encadrée ci-dessus avec~$\delta q = 0$ indique
\[ \dd T = \frac{R}{C_P} \, \frac{T}{P} \, \dd P\]
\[ \Rightarrow \qquad \frac{\dd T}{T} - \frac{R}{C_P} \, \frac{\dd P}{P} = 0 \]
soit par intégration
\[ T \, P^{- \kappa} = \text{constante} \qquad \text{avec} \qquad \kappa = R / C_P \]
Autrement dit, dans le cas où une parcelle subit une transformation adiabatique, sa température varie proportionnellement à~$P^{\kappa}$. Il s'agit d'une version, avec les grandeurs intensives utiles en sciences de l'atmosphère, de l'équation~$P\,V^{\gamma}$, avec $\gamma = C_P / C_V$, vue dans les cours de thermodynamique générale pour les transformations adiabatiques.




	\sk \subsection{Gradient adiabatique sec} \label{adiabsec}
	\sk
D'après les seules équations thermodynamiques, on peut trouver une loi simple des variations de température avec l'altitude pour une parcelle qui ne subit que des transformations adiabatiques. Considérons le cas d'une parcelle subissant un déplacement vertical quasi-statique et adiabatique tel que~$\delta q = 0$. Elle vérifie en première approximation l'équilibre hydrostatique~$\dd P\e{p} / \rho = - g \, \dd z$. L'équation du premier principe modifiée pour le cas atmosphérique indique alors que
\[  \dd T\e{p}  = - \frac{g}{C_P} \, \dd z \]
d'où on tire le profil vertical adopté dans l'atmosphère sèche par une parcelle ne subissant pas d'échange de chaleur avec l'extérieur
\[  \boxed{ \ddf{T\e{p}}{z}  = \Gamma\e{sec} \qquad \text{avec} \qquad \Gamma\e{sec} = \frac{-g}{C_P} } \]
On note qu'il ne s'agit pas nécessairement du profil vertical suivi par l'environnement (voir section~\ref{parcenv}).

\sk
Le résultat trouvé ci-dessus revêt une importance particulière en sciences de l'atmosphère. La température d'une parcelle en ascension adiabatique décroît avec l'altitude selon un taux de variation constant, indépendamment des effets de pression. La constante~$\Gamma\e{sec}$ est appelée le \voc{gradient adiabatique sec} de température. Il n'est valable que pour une parcelle d'air non saturée en vapeur d'eau. Le calcul pour la Terre donne un refroidissement de l'ordre de~$10^{\circ}$C/km (ou K/km). 

\sk
Pourquoi cette valeur est-elle en désaccord avec la décroissance de~$6.5^{\circ}$C/km effectivement constatée dans l'atmosphère terrestre~? Cet écart est relatif aux processus humides qui ont une grande importance dans l'atmosphère terrestre.
%Le chapitre suivant apporte des éléments de réponse à ce paradoxe apparent.





\mk \section{Evolution hors équilibre d'une parcelle d'air}

	\sk \subsection{Transformations pseudo-adiabatiques}
	\sk
On considère tout d'abord une parcelle d'air (contenant de la vapeur d'eau) en évolution isobare. Le premier principe appliqué à la parcelle indique donc
\[ \dd T = \frac{1}{C_P} \, \delta q \]
Lors de l'évaporation, les molécules d'eau liquide voient les liaisons hydrogène avec leurs proches voisins être brisées. Le passage de l'eau de la phase liquide à la phase vapeur consomme donc de l'énergie\footnote{On peut s'en convaincre en notant la sensation de froid immédiate que provoque la sortie d'un bain à cause de l'évaporation de l'eau liquide sur le corps mouillé~; ou en se souvenant que lorsque l'on souffle sur la soupe pour la refroidir, c'est précisément pour favoriser l'évaporation et la refroidir efficacement.}~: pour l'air qui compose la parcelle, $\delta q < 0$ et il y a refroidissement. 
A l'inverse, lors de la condensation, les molécules d'eau sous forme gazeuse créent des liaisons hydrogène avec les molécules d'eau de la phase liquide pour atteindre un état énergétique plus faible. Le passage de l'eau de la phase vapeur à la phase liquide libère donc de l'énergie~: pour l'air qui compose la parcelle, $\delta q > 0$ et il y a chauffage.

\sk
L'énergie~$\delta q$ consommée ou libérée par les changements d'état s'appelle~\voc{chaleur latente}, on la note~$\delta q\e{latent}$. Si une masse de vapeur~$\dd m\e{vapeur d'eau}$ est condensée ou évaporée, on a
\[ \delta q\e{latent} = \frac{- L \, \dd m\e{vapeur d'eau}}{m\e{air sec}} \qquad \Rightarrow \qquad \boxed{ \delta q\e{latent} = - L \, \dd r } \]
où~$L$ est la chaleur latente massique en~J~kg$^{-1}$. La formule ci-dessus comporte un signe négatif. La quantité~$\delta q\e{latent}$ est positive lorsqu'il y a condensation (le rapport de mélange en vapeur d'eau diminue $\dd r < 0$) et négative lorsqu'il y a évaporation (le rapport de mélange en vapeur d'eau augmente $\dd r > 0$).

\sk
On considère désormais une parcelle d'air en évolution adiabatique, à l'exception des échanges de chaleur latente~: $\delta q = \delta q\e{latent}$. On appelle une telle transformation \voc{pseudo-adiabatique} ou encore \voc{adiabatique saturée}. On fait l'approximation que la chaleur latente consommée ou dégagée est seulement échangée avec l'air sec~:
\begin{citemize}
\item La chaleur latente consommée/dégagée n'est pas utilisée pour refroidir/chauffer les gouttes d'eau présentes.
\item On néglige les pertes de masse par précipitation~: la masse d'air sec considérée est constante.
\end{citemize}
Pour une telle transformation, la variation de température s'écrit ainsi
\[ \dd T = \frac{R}{C_P} \, \frac{T}{P} \, \dd P - \frac{L}{C_P} \, \dd r \]


	\sk \subsection{Profil vertical saturé}
	\sk
Considérons une parcelle en ascension adiabatique saturée (et non plus sèche comme dans la section~\ref{adiabsec}). Pour une parcelle saturée, c'est-à-dire à l'équilibre liquide/vapeur, l'équation qui précède peut s'écrire, en utilisant l'équilibre hydrostatique
\[ c_p \, \dd T + g \, \dd z + L \, \dd r = 0 \]
Or, puisque la parcelle est saturée, on a~$r = r\e{sat}(T)$ et on peut écrire $\dd r\e{sat} = \ddf{r\e{sat}}{T} \, \dd T$. On a alors
\[ \left( c_p + L \, \ddf{r\e{sat}}{T} \right) \dd T + g \, \dd z = 0\]
Cette expression est similaire au cas sec, à l'exception notable du terme supplémentaire~$L \, \ddf{r\e{sat}}{T}$ lié aux échanges latents. On peut alors obtenir le profil vertical adopté dans l'atmosphère saturée par une parcelle ne subissant pas d'échange de chaleur avec l'extérieur autre que les échanges de chaleur latente
\[  \ddf{T}{z}  = \Gamma\e{saturé} \qquad \text{avec} \qquad \Gamma\e{saturé} = \frac{-g}{c_p+L \, \ddf{r\e{sat}}{T} } \]
On a vu que $\ddf{r\e{sat}}{T}$ est toujours positif, on en déduit donc
\[ \boxed{ \Gamma\e{saturé} > \Gamma\e{sec} \qquad \text{ou} \qquad |\Gamma\e{saturé}| < |\Gamma\e{sec}| } \]
A cause du dégagement de chaleur latente, la température diminue moins vite pour une parcelle saturée en ascension que pour une parcelle non saturée. Le calcul pour l'atmosphère terrestre montre que
\[ \Gamma\e{saturé} = -6.5 \, \text{K~km}^{-1} \] 
ce qui correspond à la valeur observée dans la troposphère sur Terre. %[Figure~\ref{fig:tempvert}].

\sk
La constatation que~$\Gamma\e{saturé}$ correspond au profil d'environnement effectivement mesuré dans la troposphère appelle un commentaire important. Les profils verticaux secs ou saturés sont ceux suivis par une parcelle en ascension~: autrement dit, ils donnent les variations de~$T\e{p}$ avec l'altitude~$z$. D'un point de vue instantané, ils ne correspondent pas aux profils d'environnement~$T\e{e}$ tels qu'ils peuvent être par exemple mesurés par des ballons-sonde lâchés dans l'atmosphère. La parcelle n'est pas nécessairement à l'équilibre thermique avec l'environnement. On peut néanmoins constater sur la figure~\ref{fig:tempvert} que la température de l'environnement diminue avec une pente très proche de~$\Gamma\e{saturé}$. Ceci s'explique par le fait que cette figure montre une moyenne sur tout le globe à toutes les saisons. La situation moyenne ainsi décrite correspond aux mouvements d'une multitude de parcelles en ascension qui finissent par définir l'environnement atmosphérique\footnote{Ce phénomène porte le nom d'ajustement convectif.}. Pour comprendre la formation des nuages, et plus généralement les mouvements atmosphériques, il faut néanmoins se placer dans le cas local où l'équilibre thermique n'est pas vérifié. C'est l'objet de la section suivante.
%Comme pour le cas adiabatique, on peut aussi intégrer l'équation pour obtenir:
%\begin{equation} e_h=c_pT+gz+Lr=cste \label{estath} \end{equation}  
%La quantité $e_h$ est appelée {\em énergie statique humide} et est conservée
%pour des mouvements adiabatiques ($r$ et $e_s$ sont séparément conservés) ou
%saturés (pseudo-adiabatiques).


\mk \section{Stabilité et instabilité verticale}

	\sk \subsection{Force de flottaison}
	\sk
Soit une parcelle dont la température $T\e{p}$ n'est pas égale à celle de l'environnement~$T\e{e}$, que ce soit sous l'effet d'un chauffage diabatique (par exemple~: chaleur latente, effets radiatifs) ou d'une compression / détente adiabatique. On reprend le calcul réalisé précédemment pour l'équilibre hydrostatique, avec la différence notable que l'on n'est plus dans le cas statique~: on étudie le mouvement vertical d'une parcelle. 

\sk
La somme des forces massiques s'exerçant sur la parcelle suivant la verticale est
\[ - g  - \frac{1}{\rho\e{p}}  \, \Dp{P\e{e}}{z} \]
où~$\rho\e{p}$ est la masse volumique de la parcelle. L'environnement est à l'équilibre hydrostatique donc
\[ \Dp{P\e{e}}{z} = - \rho\e{e} \, g \]
Ainsi la résultante~$F_z$ des forces massiques qui s'exercent sur la parcelle selon la verticale vaut
\[ F_z = g \, \left( \frac{\rho\e{e}}{\rho\e{p}} - 1 \right) = g \, \frac{\rho\e{e}-\rho\e{p}}{\rho\e{p}} \]
En utilisant l'équation du gaz parfait pour la parcelle~$\rho\e{p}=P/RT\e{p}$ et l'environnement~$\rho\e{e}=P/RT\e{e}$, on a
\[ \boxed{ F_z = g \, \frac{T\e{p}-T\e{e}}{T\e{e}} } \]
La résultante des forces est donc dirigée vers le haut, donc la parcelle s'élève, si la parcelle est plus chaude (donc moins dense) que son environnement. 
Elle est dirigée vers le bas si la parcelle est plus froide (donc plus dense) que son environnement.
En d'autres termes, on écrit ici la version météorologique de la force ascendante ou descendante 
provoquée par la poussée d'Archimède, également appelée \voc{force de flottaison}.


	\sk \subsection{Stabilité et instabilité}
	\sk
Ces considérations permettent de définir le concept de stabilité et instabilité verticale de l'atmosphère.
On considère l'atmosphère à un endroit donné de la planète, à une saison donnée, à une heure donnée de la journée.
On suppose que la température de l'environnement varie linéairement avec l'altitude
\[ \ddf{T\e{e}}{z} = \Gamma\e{env} \]
A une altitude~$z_0$ proche de la surface, la température de l'environnement est~$T\e{e}(z_0)=T_0$.

\sk
On considère une parcelle initialement à l'altitude~$z_0$ dont la température initiale~$T\e{p}(z_0)$ est également~$T_0$. On suppose que la parcelle subit une ascension verticale d'amplitude~$\delta z > 0$. Le profil de température suivi par la parcelle lors de son ascension est
\[ \ddf{T\e{p}}{z} = \Gamma\e{parcelle} \]
\begin{citemize}
\item Si la parcelle est non saturée, elle suit un profil adiabatique sec tel que $\Gamma\e{parcelle} = \Gamma\e{sec} \simeq - 10 \, \text{K/km}$.
\item Si elle est saturée, elle suit un profil adiabatique saturé tel que $\Gamma\e{parcelle} = \Gamma\e{saturé} \simeq - 6.5 \, \text{K/km}$. 
\end{citemize}
On rappelle qu'en général, à l'échelle où l'on étudie les mouvements de la parcelle
\[ \Gamma\e{parcelle} \neq \Gamma\e{env} \]

\sk
Quel est l'effet de la perturbation~$\delta z > 0$ sur le mouvement de la parcelle~? A l'altitude~$z_0 + \delta z$, les températures de la parcelle et de l'environnement sont respectivement
\[ T\e{p}(z_0 + \delta z) = T_0 + \Gamma\e{parcelle} \, \delta z 
\qquad \text{et} \qquad
T\e{e}(z_0 + \delta z) = T_0 + \Gamma\e{env} \, \delta z \]
\begin{finger}
\item Si $\Gamma\e{parcelle} > \Gamma\e{env}$, la température~$T\e{e}$ de l'environnement décroît plus vite que la température~$T\e{p}$ de la parcelle. Il en résulte que~$T\e{p}(z_0 + \delta z) > T\e{e}(z_0 + \delta z)$ et le mouvement de la parcelle est ascendant. La perturbation initiale est donc amplifiée par les forces de flottabilité. On parle de \voc{situation instable}. La situation est d'autant plus instable que la température de l'environnement décroît rapidement avec l'altitude. Lorsque la situation est instable, les mouvements verticaux sont amplifiés~: on parle parfois de \voc{situation convective}.
\item Si $\Gamma\e{parcelle} < \Gamma\e{env}$, la température~$T\e{e}$ de l'environnement décroît moins vite que la température~$T\e{p}$ de la parcelle. Il en résulte que~$T\e{p}(z_0 + \delta z) < T\e{e}(z_0 + \delta z)$ et le mouvement de la parcelle est descendant. La perturbation initiale n'est donc pas amplifiée et la parcelle revient à son état initial. On parle de \voc{situation stable}. La stabilité est d'autant plus grande que la température de l'environnement décroît lentement (ou augmente, dans le cas d'une inversion de température). Lorsque la situation est stable, les mouvements verticaux sont inhibés.
\end{finger}
La résultante des forces verticales s'exerçant sur la parcelle peut s'écrire en fonction des taux de variation~$\Gamma$ de la température
\[ F_z = g \, \frac{\Gamma\e{parcelle}-\Gamma\e{env}}{T\e{env}} \, \delta z \]
\noindent Un raisonnement similaire permet d'obtenir la fréquence de Brunt-V{\"a}is{\"a}l{\"a}.

	\sk
On peut illustrer la stabilité/instabilité atmosphérique dans le cas des polluants émis proche de la surface par les activités humaines [Figure~\ref{fig:pollution}]. Dans l'après-midi, du fait que le sol est chaud, le profil d'environnement est tel que la situation est très instable~: les mouvements verticaux qui transportent les polluants plus haut dans l'atmosphère sont encouragés et les polluants ne restent pas proches de la surface. A l'inverse, en soirée, du fait que le sol refroidit radiativement, le profil d'environnement est tel que la situation est stable~: les mouvements verticaux qui pourraient transporter les polluants plus haut dans l'atmosphère sont inhibés et les polluants sont confinés proche de la surface. Pour être moins exposé aux polluants dans les zones urbaines, il est donc préférable d'y effectuer son jogging en fin de matinée plutôt qu'en soirée !

\figside{0.65}{0.25}{decouverte/cours_meteo/inversion-temperature.png}{Stabilité et pollution atmosphérique. On notera que cette figure est très illustrative, mais présente une situation simplifiée. Le transport vertical de polluants dans l'atmosphère est en réalité inhibé dès que la couche atmosphérique est stable, ce qui est plus général que considérer uniquement une inversion thermique comme à droite de la figure. Source~: Airparif}{fig:pollution} 


\end{document}

\newpage
\section{Modèle à deux faisceaux : écriture}
\sk
Le modèle à deux faisceaux est un bon compromis entre simplicité
et illustration de concepts importants. Il est une version simplifiée
de l'équation de Schwarzschild du transfert radiatif.
Ce modèle entend élucider
les transferts de rayonnement dans l'infrarouge entre
les couches qui composent la colonne atmosphérique. Les
hypothèses simplificatrices suivantes sont réalisées
\begin{citemize}
\item couches atmosphériques plan-parallèle (sphéricité négligée)
\item phénomènes d'absorption négligés dans le visible (transparence au visible)
\item phénomènes de diffusion (\emph{scattering}) négligés dans l'infrarouge
\item \emph{gray gas} dans l'infra-rouge : on considère que le coefficient d'absorption
du gaz est indépendant de la longueur d'onde~$\lambda$ ($k_{\lambda} = k$ pour tout~$\lambda$),
ce qui implique une hypothèse similaire pour l'épaisseur optique ($\tau_{\lambda} = \tau$ pour tout~$\lambda$).
\end{citemize}
En d'autres termes, on se cantonne dans ce modèle à deux types de phénomènes
\begin{enumerate}
\item Un faisceau de rayonnement infra-rouge de flux~$F$ traversant une couche 
atmosphérique donnée
subit une extinction à cause de l'absorption selon une loi de type Beer-Lambert
\[
\dd F = - F \dd \tau
\]
avec~$\tau$ l'épaisseur optique \emph{gray gas} 
dans l'infra-rouge.
\item Une couche atmosphérique émet un flux de rayonnement thermique~$M$ 
calculé par la loi intégrée de Stefan-Boltzmann ($M=\epsilon\,\sigma\,T^4$)
puisque la majorité de l'émittance est émise dans l'infrarouge pour les températures considérées.
\end{enumerate}

\sk
L'épaisseur optique~$\tau$ peut servir de coordonnée verticale à la place de~$z$
en utilisant la relation entre les deux quantités. La couche atmosphérique
élémentaire considérée est ainsi d'épaisseur~$\dd\tau$ et située à une coordonnée
verticale~$\tau$ qui croît avec l'altitude. 

\sk
Si l'on considère un faisceau ascendant~$F^+(\tau)$ au bas de la couche considérée,
une fois la couche traversée son amplitude est
\[
F^+(\tau) - F^+(\tau) \dd\tau
\]
A ce flux au sommet de la couche, il convient d'ajouter
la contribution de l'émission thermique de la couche vers 
le haut, à savoir~$M\,\dd\tau$.
Le flux total ascendant au sommet de la couche est donc
\[
F^+(\tau+\dd\tau) = F^+(\tau) - F^+(\tau) \dd\tau + M\dd\tau
\]

\sk
Même raisonnement avec le flux descendant~$F^-(\tau+\dd\tau)$ au sommet de la couche considérée,
une fois la couche traversée son amplitude est~$F^-(\tau+\dd\tau) - F^-(\tau) \dd\tau$, où 
l'approximation du terme du second ordre~$F^-(\tau+\dd\tau) \dd\tau \simeq F^-(\tau) \dd\tau$
a été effectuée.
Le flux total descendant au bas de la couche est donc
\[
F^-(\tau+\dd\tau) = F^-(\tau) - F^-(\tau) \dd\tau + M\dd\tau
\]

\sk
Les deux résultats qui précèdent peuvent être transformés 
afin de faire apparaître une dérivée
en utilisant le théorème des accroissements finis
\[
\ddf{F^+}{\tau} = \frac{F^+(\tau+\dd\tau) - F^+(\tau)}{\dd\tau}
\]
\noindent ce qui permet d'obtenir au final
\[
\ddf{F^+}{\tau} = - F^+(\tau) + \epsilon\,\sigma\,T(\tau)^4 \quad [S^+]
\qquad\qquad 
\ddf{F^-}{\tau} = F^-(\tau) - \epsilon\,\sigma\,T(\tau)^4 \quad [S^-]
\]
\noindent $[S^+]$ et~$[S^-]$ sont parfois appelées les relations de Schwarzschild à deux faisceaux.
Il s'agit d'une version très simplifiée des équations de Schwarzschild du transfert radiatif.

\sk
Si l'on souhaite adopter la convention~$\tau=0$ au sommet de l'atmosphère,
et $\tau=\tau_{\infty}$ à la surface en $z=0$, donc adopter un axe
vertical d'épaisseur optique avec~$\tau$ croissant de haut en bas, il
suffit de remplacer~$\tau$ par~$-\tau$ dans les équations précédentes pour obtenir
\[
\boxed{\ddf{F^+}{\tau} = F^+(\tau) - \epsilon\,\sigma\,T(\tau)^4 \quad [S^+]} 
\qquad\qquad 
\boxed{\ddf{F^-}{\tau} = -F^-(\tau) + \epsilon\,\sigma\,T(\tau)^4 \quad [S^-]}
\]









\newpage
\section{Modèle à deux faisceaux : résolution}
\sk
Le système d'équations~$[S^+]$ et~$[S^-]$ du modèle à deux faisceaux
est plus simple à résoudre si l'on considère les deux quantités~$\Sigma(\tau)=F^{+}(\tau)+F^{-}(\tau)$ et~$\Delta(\tau)=F^{+}(\tau)-F^{-}(\tau)$ car on obtient
\[
\ddf{\Sigma}{\tau} = \Delta(\tau) \quad [E_\Sigma] 
\qquad\qquad 
\ddf{\Delta}{\tau} = \Sigma(\tau) - 2\,\epsilon\,\sigma\,T(\tau)^4 \quad [E_\Delta]
\]
\noindent Ensuite la résolution impose d'expliciter les conditions aux limites
\begin{enumerate}[$\mathcal{C}_1$]
\item on se place à l'équilibre radiatif donc le flux net~$\Delta$ est constant à tout niveau : $\ddf{\Delta}{\tau}=0$
\item au sommet de l'atmosphère $F^+(\tau=0) = OLR$ (définition de $OLR$) et $F^-(\tau=0) = 0$ (contribution
incidente négligeable du Soleil dans l'infra-rouge), ce qui s'écrit encore~$\Delta(\tau=0)=\Sigma(\tau=0)=OLR$
\item à la surface de température~$T_s$ le bilan radiatif est le suivant : la surface reçoit l'intégralité du rayonnement
solaire incident~$(1-A\e{b}) \, \mathcal{F}\e{s}'$ (visible) plus du rayonnement de l'atmosphère située
juste au-dessus d'elle~$F^-(\tau=\tau_{\infty})$ (infra-rouge) ; de plus elle émet un rayonnement
$\epsilon\,\sigma\,T\e{s}^4$ dans l'infra-rouge vers l'atmosphère\footnote{On a supposé ici pour simplifier les calculs que l'émissivité
de la surface était similaire à l'émissivité de l'atmosphère}
\item on rappelle que selon la relation \emph{TOA}, nous avons $OLR = (1-A\e{b}) \, \mathcal{F}\e{s}'$
\end{enumerate}


\sk
Il est alors possible d'obtenir deux expressions différentes pour~$\Sigma(\tau)$.
Premièrement, en utilisant $[E_\Sigma]$ avec $\mathcal{C}_1$ et $\mathcal{C}_2$, on obtient~$\Sigma(\tau)=OLR \, (1+\tau)$.
Deuxièmement, en utilisant $[E_\Delta]$ avec $\mathcal{C}_1$, on obtient~$\Sigma(\tau)=2\,\epsilon\,\sigma\,T(\tau)^4$.
\textbf{Conclusion 1} : on obtient le \voc{profil radiatif}, 
c'est-à-dire le profil vertical de température\footnote{Suivant la géométrie
équivalente choisie pour le modèle plan-parallèle, le terme $1+\tau$
peut s'écrire un peu différemment, mais quoiqu'il en soit toujours sous une
forme~$a+b\,\tau$ avec $a,b$ constants. Les conclusions énoncées ici ne sont pas
modifiées.} imposé par les transferts radiatifs dans l'infrarouge
\[
T(\tau) = \sqrt[4]{\frac{OLR\,(1+\tau)}{2\,\sigma\,\epsilon}}
\]

\sk
Reste à calculer la température de surface avec ce modèle. D'après $\mathcal{C}_3$, le bilan
au sol s'écrit~$(1-A\e{b}) \, \mathcal{F}\e{s}' + F^-(\tau=\tau_{\infty}) = \epsilon\,\sigma\,T\e{s}^4$.
Il faut donc exprimer les flux ascendant et descendant dans l'infrarouge.
Du fait que $\mathcal{C}_1$ et $\mathcal{C}_2$ nous indiquent que~$\Delta=OLR$, on obtient aisément
\[
F^+(\tau) = \frac{\Sigma+\Delta}{2} = OLR \, (1+\frac{\tau}{2})
\qquad \qquad
F^-(\tau) = \frac{\Sigma-\Delta}{2} = OLR \, \frac{\tau}{2}
\]
\noindent On obtient alors l'expression liant $OLR$
et température de surface~$T\e{s}$
\[
\boxed{\epsilon\,\sigma\,T\e{s}^4 = OLR \, \left( 1 + \frac{\tau_{\infty}}{2} \right)}
\]
\textbf{Conclusion 2} : on obtient une définition quantitative de \voc{l'effet de serre}
\begin{citemize}
\item Dans l'infrarouge, le rayonnement sortant au sommet de l'atmosphère ($OLR$)
est inférieur au rayonnement émis par la surface ($\epsilon\,\sigma\,T\e{s}^4$).
Une partie du rayonnement émis par la surface reste donc piégée par la planète.
\item Avec un albédo et un rayonnement incident constant, donc à~$OLR$ constant (d'après $\mathcal{C}_4$),
augmenter la quantité de gaz à effet de serre (donc augmenter~$\tau_{\infty}$)
conduit à une augmentation de la température de surface~$T\e{s}$.
\end{citemize}

\sk
Il est alors instructif de s'intéresser à la température atmosphérique 
proche de la surface~$T(\tau_\infty)$
donnée par le profil radiatif. Cette température ne dépend que de~$OLR$
et s'obtient totalement indépendamment de la température de surface.
On peut alors montrer que
\[
T\e{s} = T(\tau_\infty) \, \sqrt[4]{\frac{2+\tau_\infty}{1+\tau_\infty}} > T(\tau_\infty)
\]
\noindent \textbf{Conclusion 3} : tant que l'atmosphère n'est pas
optiquement épaisse dans l'infrarouge (donc tant que~$\tau_\infty$ reste fini),
il existe une discontinuité entre la surface et l'atmosphère, la surface
étant toujours plus chaude que l'atmosphère. Cela implique que l'atmosphère
est instable proche de la surface, donc que du mélange turbulent / convectif
apparaît, donc que l'équilibre proche de la surface ne peut être simplement
radiatif mais \voc{radiatif-convectif}. Notons que dans le cas où l'atmosphère est optiquement épaisse,
$T\e{s} = T(\tau_\infty)$, ce qui est tout à fait représentatif des conditions sur Vénus.


%%% figure Salby


\newpage
\section{Inertie thermique}
L'inertie thermique $I$ mesure la résistance
thermique d'un milieu à un apport ou un
déficit de chaleur.
%
L'expression de $I$ (J~m$^{-2}$~s$^{-1/2}$~K$^{-1}$) 
s'obtient en déduisant d'une équation
simple de conduction thermique de Fourier,
par analyse dimensionnelle, l'épaisseur
de peau thermique $\delta$ 
%
\[
\rho \, c_p \, \Dp{T}{t} = \Dp{}{x} \left( \lambda \, \Dp{T}{x} \right)
\quad
\to
\quad
\delta = \sqrt{\f{\lambda \, \tau}{\rho \, c_p}}
\]
%
\noindent (où $\tau$ est une constante caractéristique de temps)
ce qui permet de mettre en évidence l'inertie
thermique dans le terme de flux de chaleur $\phi\e{c}$
à la surface
%
\[
\phi\e{c} = - \lambda \, \Dp{T}{x} = - \f{\lambda}{\delta} \, \Dp{T}{x'} 
= - \sqrt{\f{\lambda \, \rho \, c_p}{\tau}} \, \Dp{T}{x'}
\quad 
\textrm{avec}
\quad
x'=x/\delta 
\]
\noindent en ne retenant que les termes qui dépendent du milieu
dans la caractérisation du flux de chaleur :
%
\[
I = \sqrt{\lambda \, \rho \, c_p}
\]


Un milieu est donc de faible inertie thermique
lorsqu'il ne peut stocker que de petites quantités de chaleur
(faible capacité calorifique $c_p$)
et/ou qu'il ne peut transmettre cette chaleur que dans ses couches superficielles
(faible conductivité thermique $\lambda$).
%
Les océans terrestres constituent un exemple 
bien connu de milieu à très forte inertie thermique, 
de par leur grande capacité calorifique.
%
Autre exemple bien connu, 
l'inertie thermique des terrains rocheux
martiens est plus élevée que 
l'inertie thermique des terrains
poussiéreux, principalement
pour des raison de conductivité thermique.
%
L'inertie thermique
peut d'ailleurs permettre 
sous certaines conditions d'estimer
la taille des grains dans les sols
non consolidés.

Dépourvue d'océans, Mars forme
un gigantesque désert de faible inertie
thermique : $I$~dépasse 
rarement $400\U{J~m^{-2}~s^{-1/2}~K^{-1}}$
pour la plupart des sols martiens.
%
Pour qualifier les grands
ensembles sur le champ d'inertie
thermique planétaire,
le terme de \ofg{continents thermiques}
est parfois employé.
%
L'inertie thermique n'est pas une
quantité observable directement
et sa détermination requiert la combinaison
de mesures de température de surface et
d'un modèle simulant les variations thermiques du sol.


\newpage
\section{Eulérien vs. Lagrangien}
\sk
Comment caractériser un écoulement ?

\sk
\paragraph{Point de vue lagrangien} Le plus intuitif~:~Suivre les particules le long de leur trajectoire.
\paragraph{Point de vue eulérien} Le plus pratique~:~Suivre le courant depuis un point géométrique. Les points sont fixés ce qui est plus aisé en première approche pour modéliser l'écoulement sur une grille.
\centers{Variations lagrangiennes \quad = \quad Variations eulériennes \quad + \quad Terme d'advection}
Le terme d'advection transport concentre le caractère non-linéaire de la dynamique atmosphérique

\sk
On passe de l'un à l'autre des formalismes avec la formule de la dérivée d'une fonction composée~$\mathcal{F}[x(t)]$ où~$x$ est la position.
\[
\underbrace{\derd{\mathcal{F}}{t}}_{\text{En suivant la particule}}
= 
\underbrace{\Dp{\mathcal{F}}{t}}_{\text{En un point géométrique}} 
+ 
\underbrace{\left(\v U \cdot \v \nabla \right)\,\mathcal{F} }_{\text{Lié au déplacement de la particule}}
\]




\newpage
\section{Passage aux coordonnées sphériques}
\sk
Vitesse dans le référentiel tournant 
\[
\vec{U_r} = u\,\vec{i} + v\,\vec{j} + w\,\vec{k}
\]
\noindent Accélération (dérivée lagrangienne)
\[
\gamma_r = \derd{\vec{U_r}}{t}=\derd{u}{t}\,\vec{i}+\derd{v}{t}\,\vec{j}+\derd{w}{t}\,\vec{k} +u\,\derd{\vec{i}}{t}+v\,\derd{\vec{j}}{t}+w\,\derd{\vec{k}}{t}
\]

\figsup{0.4}{0.2}{decouverte/cours_dyn/didt1.png}{decouverte/cours_dyn/didt2.png}{Axe méridional, axe zonal}{fig:spher}

\sk
Décomposition sur les trois axes (zonal, méridien et vertical)
\[
\derd{\vec{i}}{t} = \left[ \derd{\vec{i}}{t} \right]_u + \left[ \derd{\vec{i}}{t} \right]_v + \left[ \derd{\vec{i}}{t} \right]_w
\]
\noindent Axe vertical
\[
\left[ \derd{\vec{i}}{t} \right]_w = \left[ \derd{\vec{j}}{t} \right]_w = \left[ \derd{\vec{k}}{t} \right]_w = \vec{0}
\]
\noindent Axe méridien
\[
\left[ \derd{\vec{i}}{t} \right]_v = \vec{0}
\]
\[
\left[ \derd{\vec{j}}{t} \right]_v = - \derd{\phi}{t} \, \vec{k} = - \dfrac{v}{a} \, \vec{k}
\]
\[
\left[ \derd{\vec{k}}{t} \right]_v = + \dfrac{v}{a} \, \vec{j}
\]
\noindent Axe zonal
\[
\vec{j} = -\sin\phi \, \vec{m} + \cos\phi \, \vec{n} \qquad \vec{k} =  \cos\phi \, \vec{m} + \sin\phi \, \vec{n} \qquad \textrm{avec} \qquad \vec{m} = -\sin\phi \, \vec{j} + \cos\phi \, \vec{k}
\]
\[
\left[ \derd{\vec{n}}{t} \right]_u = \vec{0} \qquad \left[ \derd{\vec{m}}{t} \right]_u = \derd{\lambda}{t} \, \vec{i}
\]
\[
\left[ \derd{\vec{j}}{t} \right]_u = -\sin\phi \, \left[ \derd{\lambda}{t} \, \vec{i} \right] = \dfrac{-u}{a} \, \tan\phi \, \vec{i}
\]
\[
\left[ \derd{\vec{k}}{t} \right]_u =  \cos\phi \, \left[ \derd{\lambda}{t} \, \vec{i} \right] = \dfrac{u}{a} \, \vec{i}
\]
\[
\left[ \derd{\vec{i}}{t} \right]_u = - \derd{\lambda}{t} \, \vec{m} = \dfrac{u}{a} \tan\phi \, \vec{j} - \dfrac{u}{a} \,\vec{k}
\]



\newpage
\section{Expression des forces en coordonnées sphériques}
\paragraph{L'équation fondamentale de la dynamique} dans le référentiel tournant
\[
\vec{\gamma}_r = -2 \, \vec{\Omega} \wedge \vec{U}_r + \vec{g} - \dfrac{\vec{\nabla}p}{\rho} + \vec{Fr}
\]

\paragraph{Composantes de l'acc\'el\'eration}
\[
\vec{\gamma}_r=\left[\begin{array}{ccc} u_t & - \dfrac{u\,v\tan\phi}{a} & + \dfrac{u\,w}{a}\\ & & \\ v_t & + \dfrac{u^2\tan\phi}{a} & + \dfrac{v\,w}{a}\\ & & \\ w_t & - \dfrac{u^2+v^2}{a} & \\ \end{array}\right] \qquad \text{notation} \quad u_t = \derd{u}{t}
\]

\paragraph{Composantes de la \ofg{force} de Coriolis}
\[
-2\,\vec{\Omega}\wedge\vec{U}_r=-2 \left[ \begin{array}{c} 0\\ \Omega\,\cos\phi\\ \Omega\,\sin\phi \end{array} \right] \wedge \left[ \begin{array}{c} u\\ v\\ w \end{array} \right] = \left[ \begin{array}{c} 2 \, \Omega \, ( v\sin\phi-w\cos\phi )\\ -2 \, \Omega \, u\sin\phi\\ 2 \, \Omega \, u\cos\phi \end{array} \right]
\]


\newpage
\section{Equations complètes du mouvement}
\sk
L'équation fondamentale de la dynamique des fluides géophysiques 
en projection sur les coordonnées sphériques avec l'approximation de couche mince
s'écrit finalement
\begin{center}
\begin{tabular}{ccccccccc}
%%%%%%
%\BBB{\derd{u}{t}} & \CCC{-\dfrac{uv\tan\phi}{a}} & \ZZZ{+\dfrac{uw}{a}} & = & \AAA{\fcoriolis v} & \ZZZ{-\fcoriolis w\cos\phi} & \DDD{-\dfrac{1}{\rho}\der{p}{x}} & & \ZZZ{+Fr_x}\\
\BBB{\derd{u}{t}} & \CCC{-\dfrac{uv\tan\phi}{a}} & \ZZZ{+\dfrac{uw}{a}} & = & \AAA{\fcoriolis v} & \ZZZ{-2\Omega \cos\phi w} & \DDD{-\dfrac{1}{\rho}\der{p}{x}} & & \ZZZ{+Fr_x}\\
~\\
%%%%%%
%\BBB{\derd{v}{t}} & \CCC{+\dfrac{u^2\tan\phi}{a}} & \ZZZ{+\dfrac{wu}{a}} & = & \AAA{-\fcoriolis u} & & \DDD{-\dfrac{1}{\rho}\der{p}{y}} & & \ZZZ{+Fr_y}\\
\BBB{\derd{v}{t}} & \CCC{+\dfrac{u^2\tan\phi}{a}} & \ZZZ{+\dfrac{vw}{a}} & = & \AAA{-\fcoriolis u} & & \DDD{-\dfrac{1}{\rho}\der{p}{y}} & & \ZZZ{+Fr_y}\\
~\\
%%%%%%
%\ZZZ{\derd{w}{t}} & \ZZZ{-\dfrac{u^2+v^2}{a}}&&=&\ZZZ{\fcoriolis u} & & \DDD{-\dfrac{1}{\rho}\der{p}{z}}& \DDD{ -g}&\ZZZ{+Fr_z}   \\ 
\ZZZ{\derd{w}{t}} & \ZZZ{-\dfrac{u^2+v^2}{a}}&&=&\ZZZ{2\Omega\cos\phi u} & & \DDD{-\dfrac{1}{\rho}\der{p}{z}}& \DDD{ -g}&\ZZZ{+Fr_z}   \\ 
\end{tabular}
\end{center}

\sk
Suivant les termes dominants, on peut définir un certain nombre d'équilibres (stationnaires) ou de modèles / équations (pouvant servir à la prédiction de l'écoulement au cours du temps):
\begin{description}
\item{Equilibre hydrostatique} \DDD{\bullet} 
\item{Equilibre g\'eostrophique} \DDD{\bullet}\AAA{\bullet}
\item{Equilibre cyclostrophique} \DDD{\bullet}\CCC{\bullet}
\item{Equilibre du vent gradient} \DDD{\bullet}\AAA{\bullet}\CCC{\bullet}
\item{Modèle quasi-g\'eostrophique} \DDD{\bullet}\AAA{\bullet}\BBB{\bullet}
\item{Equations primitives} \DDD{\bullet}\AAA{\bullet}\BBB{\bullet}\CCC{\bullet}
\end{description}

\mk
\paragraph{Nombre de Rossby} Le nombre de Rossby permet d'évaluer l'importance relative de l'accélération de Coriolis, impulsée par la rotation de la planète, par rapport aux autres mouvements de rotation. Il permet de savoir si l'on se trouve dans le domaine de validité de l'équilibre géostrophique ou de l'équilibre cyclostrophique
\[
R_o=\f{\text{accélération horizontale (inertielle + sphéricité)}}{\text{accélération de Coriolis}}\qquad\boxed{R_o=\frac{U}{L\,\Omega}}
\]
\begin{table}[h!]
\begin{tabular}{cccc}
$R_o \ll 1$ & \DDD{\bullet}\AAA{\bullet} & Equilibre g\'eostrophique & [Terre, Mars]\\
$R_o \gg 1$ & \DDD{\bullet}\CCC{\bullet} & Equilibre cyclostrophique & [Vénus, Titan]\\
$R_o$~tous & \DDD{\bullet}\AAA{\bullet}\CCC{\bullet} & Equilibre du vent gradient & [Toutes]\\
~ & & & \\
$R_o \ll 1$ & \DDD{\bullet}\AAA{\bullet}\BBB{\bullet} & Modèle quasi-g\'eostrophique & [Terre, Mars]\\
$R_o$~tous & \DDD{\bullet}\AAA{\bullet}\BBB{\bullet}\CCC{\bullet} & Equations primitives  & [Toutes]
\end{tabular}
\end{table}

\mk
Sur les planètes à rotation rapide, l'équilibre géostrophique est le développement des équations du mouvement à l'ordre 1 en le nombre de Rossby, qui décrit un écoulement bidimensionnel, stationnaire et non divergent. A un ordre supérieur en $\textrm{Ro}$, l'évolution lente de la fonction de courant géostrophique peut être diagnostiquée par un nouvel équilibre dit quasi-géostrophique (QG). Couplé à l'équation de conservation de la vorticité potentielle de Rossby, le modèle approché QG a permis à Charney dans les années 50 de faire fonctionner sur un ordinateur le premier modèle de prévision numérique du temps.


\newpage
\section{Système complet pour la modélisation}
\sk
Bjerknes, 1904~:~6 équations pour 6 inconnues 
\begin{finger}
\item variables \textcolor{red}{dynamiques} ou \textcolor{brown}{thermodynamiques}
\item ce qui dépend de la planète considérée~:~\textcolor{blue}{forçages} et \textcolor{green!75!black}{constantes planétaires} 
\item rappel: formalisme eulérien vs. lagrangien $\derd{\mathcal{F}}{t}=\der{\mathcal{F}}{t}+\textcolor{red}{u}\der{\mathcal{F}}{x}+\textcolor{red}{v}\der{\mathcal{F}}{y}+\textcolor{red}{w}\der{\mathcal{F}}{z}$
\end{finger}

\sk
\noindent Les 6 équations qui permettent d'évaluer l'évolution déterministe du fluide atmosphérique soumis aux forçages
\begin{enumerate}
\item Mouvement horizontal ouest-est
\[ \derd{\textcolor{red}{u}}{t} - \dfrac{\textcolor{red}{u}\textcolor{red}{v}\tan\phi}{\textcolor{green!75!black}{a}} = 2\textcolor{green!75!black}{\Omega}\sin\phi \, \textcolor{red}{v} - \dfrac{1}{\textcolor{brown}{\rho}} \, \der{\textcolor{brown}{p}}{x} + \textcolor{blue}{F_u} \]
\item Mouvement horizontal sud-nord
\[ \derd{\textcolor{red}{v}}{t} + \dfrac{\textcolor{red}{u}^2\tan\phi}{\textcolor{green!75!black}{a}} = -2\textcolor{green!75!black}{\Omega}\sin\phi \, \textcolor{red}{u} - \dfrac{1}{\textcolor{brown}{\rho}} \, \der{\textcolor{brown}{p}}{y} + \textcolor{blue}{F_v} \]
\item Equilibre hydrostatique vertical
\[
 - \dfrac{1}{\textcolor{brown}{\rho}} \, \der{\textcolor{brown}{p}}{z} - \textcolor{green!75!black}{g} = 0
\]
\item Conservation de la masse
\[
\der{\textcolor{brown}{\rho} }{t} + \div\dep{\textcolor{brown}{\rho} \textcolor{red}{\vec{V}}}=0
\]
\item Premier principe 
\[
\f{\textcolor{green!75!black}{c_p}}{\theta} \, \derd{\theta}{t} = \f{\textcolor{blue}{\mathcal{Q}}}{\textcolor{brown}{T}}
\qquad \text{avec} \qquad
\theta=\textcolor{brown}{T} \, \left[ \f{\textcolor{green!75!black}{p_0}}{\textcolor{brown}{p}} \right]^{\textcolor{green!75!black}{\kappa}} 
\]
\item Gaz parfait
\[
\textcolor{brown}{p} = \textcolor{brown}{\rho} \, \textcolor{green!75!black}{R} \, \textcolor{brown}{T}
\]
\end{enumerate}



\newpage
\section{Vent gradient}
\sk
Dans toutes les atmosphères connues, à grande échelle, un quasi-équilibre s'établit entre le champ de masse atmosphérique (lié au gradient de pression) et la composante horizontale de la force centrifuge (entraînement + Coriolis), liée au vent zonal~$u$ et 
\[
\v F\e{e} = - m \, \left( 2\,\Omega\,\sin\varphi\,u + \f{u^2\,\tan\varphi}{a} \right) \v j
\]
%% $F\e{e}$ s'oppose au gradient de pression pour un vent prograde~$u > 0$
\noindent Il s'agit de l'\voc{équilibre du vent gradient}, vrai en moyenne zonale (Figure~\ref{fig:vg}). 
\[
\boxed{\dfrac{u^2\tan\phi}{a} + 2\Omega\sin\phi u = -\dfrac{1}{\rho}\Dp{p}{y}}
\]
%\sk Le vent gradient est un équilibre diagnostic alors que les équilibres géostrophiques et cyclostrophiques sont des équations prognostiques

\figsup{0.35}{0.15}{decouverte/cours_dyn/td2_pression.png}{decouverte/cours_dyn/td2_centrifuge.png}{Gradient de pression (haut). Force centrifuge (bas).}{fig:vg}

\sk
L'advection de l'air liée à la force de pression devient de moins en moins efficace quand on s'éloigne de l'équateur : l'équilibre du vent gradient provient d'un effet de frein exercé par la composante horizontale de la force centrifuge. Plus la planète tourne vite, plus cet effet de frein est prépondérant. Ainsi, la rotation de la planète contrôle l'extension en latitude des cellules de Hadley -- tout comme le déplacement en latitude du maximum saisonnier de température la contrôle~: 
\begin{finger}
\item L'extension limitée des cellules de Hadley sur Terre est majoritairement causée par la position du maximum saisonnier de température (qui reste confinée aux tropiques en raison de l'inertie thermique élevée des océans), alors que leur extension limitée sur les planètes géantes est le fait de leur rotation rapide.
\item L'extension des cellules de Hadley vers les pôles sur Mars est majoritairement causée par les effets de position du maximum saisonnier de température\footnote{Aux solstices, sous l'effet de la faible inertie thermique de la surface martienne et des constantes de temps radiatifs réduits dans l'atmosphère, la structure thermique de l'atmosphère martienne est composée d'un gradient de température d'un pôle à l'autre et conduit à une circulation de Hadley interhémisphérique. La circulation méridienne est particulièrement intense en raison du forçage diabatique des poussières en suspension dans l'atmosphère, surtout au solstice d'hiver nord où l'opacité moyenne des poussières atteint $1$ et l'insolation est maximale.}, alors que la grande extension vers les pôles des cellules de Hadley sur Vénus est avant tout reliée à la rotation lente de ce corps qui limite l'effet de frein de~$F\e{e}$.
\end{finger}

\sk 
Dans certains cas particuliers, notamment si~$u<0$ et~$u<2 \, \Omega \, a \, \cos \varphi$ (autrement dit, pour un courant-jet prograde dans le cas où la force de Coriolis domine la force d'entraînement), la force~$\v F\e{e}$ induit une accélération vers le pôle et non un frein. Cet effet peut favoriser l'extension vers les pôles des cellules de Hadley dans le cas de planètes à rotation rapide comme Mars. Exemple, au solstice d'hiver nord de Mars, une particule de vitesse zonale nulle partant d'un point de l'hémisphère sud de latitude $-\varphi_0$ (typiquement $60^{\circ}S$) et parcourant la branche haute de la cellule de Hadley adopte un mouvement rétrograde $u<0$ jusque la latitude opposée $\varphi_0$ par conservation du moment cinétique $\mathcal{M} = a \cos \varphi \left( \Omega \, a \, \cos \varphi + u \right)$. Contrairement au cas terrestre, la résultante des forces d'entraînement~$\v F\e{e}$ s'ajoute entre les latitudes $0$ et $\varphi_0$ au gradient de pression et la circulation méridienne s'intensifie jusqu'à la latitude $\varphi_0$, rejetant la limite des cellules de Hadley beaucoup plus loin que sur Terre. Entre les latitudes $-\varphi_0$ et $\varphi_0$, les isolignes du transport méridien de masse se confondent donc avec les isolignes du moment cinétique. Aux plus hautes latitudes $\varphi > \varphi_0$, dès que la vitesse zonale devient négative, le jet d'ouest se forme et la résultante~$\v F\e{e}$ s'oppose aux gradients de pression comme sur Terre.
%Seules les cellules de Hadley autour des équinoxes martiens, symétriques entre les deux hémisphères, ressemblent aux équivalents terrestres.









\newpage
\section{Equilibre du vent thermique}
\sk
On cherche à relier simplement un équilibre entre température et vent. Si on suppose l'équilibre du vent gradient vérifié, il donne déjà une relation entre champ de vent et champ de pression, donc seules quelques transformations de cette relation sont nécessaires. On note~$x$ la coordonnée sur l'axe est-ouest (axe zonal), $y$ la coordonnée sur l'axe sud-nord (axe méridional), $P$ la pression atmosphérique. Cette dernière est utilisée comme coordonnée verticale en vertu de l'équilibre hydrostatique.

\sk
On définit le géopotentiel~$\Phi$ comme une fonction des coordonnées~$x$, $y$ et~$P$ qui s'écrit simplement
\[ \Phi(x,y,P)=g \, z(x,y,P) \] 
\noindent avec~$z$ l'altitude (également fonction des coordonnées~$x$, $y$ et~$P$) et~$g$ l'accélération de la gravité. Le vent géostrophique zonal~$u$ s'exprime comme une fonction de~$x$, $y$, $P$, tout comme la température atmosphérique~$T$ et la masse volumiquede l'air~$\rho$. Les dérivées partielles (notées~$\partial$) de ces fonctions de trois variables se comprennent comme les dérivées selon la coordonnée indiquée avec les deux autres fixées. Par exemple $\frac{\partial \Phi}{\partial y}$ est la dérivée du géopotentiel~$\Phi$ uniquement selon la coordonnée~$y$, en considérant que~$x$ et~$P$ ne varient pas. 
%On rappelle que les dérivées partielles commutent, c'est-à-dire par exemple 
%\[ \frac{\partial}{\partial P} \frac{\partial \Phi}{\partial y} = \frac{\partial}{\partial y} \frac{\partial \Phi}{\partial P} \]

\sk
On utilise tout d'abord l'équilibre hydrostatique pour exprimer très simplement la dérivée du géopotentiel~$\Phi$ en fonction de la coordonnée verticale~$P$
\[ \frac{\partial \Phi}{\partial P} = g \, \frac{\partial z}{\partial P} = -\f{1}{\rho} \] 
\noindent ce qui permet de relier simplement les variations verticales de géopotentiel (sur les lignes isobares) au champ de masse.
On utilise directement ce résultat, combiné à une propriété de changement de coordonnée dans les dérivées partielles, pour exprimer très simplement la force de pression comme la dérivée spatiale du géopotentiel
\[ \frac{\partial \Phi}{\partial y} = \frac{\partial \Phi}{\partial P} \, \frac{\partial P}{\partial y} = -\frac{1}{\rho} \, \frac{\partial P}{\partial y} \]

\sk
Munis de cette expression simple de la force de pression, on peut alors modifier l'équilibre du vent gradient
\[ \dfrac{u^2\tan\phi}{a} + \fcoriolis u = -\dfrac{1}{\rho}\der{p}{y} = \frac{\partial \Phi}{\partial y} \]
\noindent que l'on peut ensuite dériver par rapport à la coordonnée verticale pression~$P$ 
\[ \left[ 2 \, u \, \dfrac{\tan\phi}{a} + \fcoriolis \right] \der{u}{P} = \der{~}{P} \left[ \frac{\partial \Phi}{\partial y} \right] \]
\noindent afin de pouvoir commuter les dérivées partielles puis utiliser la version de l'équilibre hydrostatique formulée ci-dessus avec le géopotentiel~$\Phi$
\[ \left[ 2 \, u \, \dfrac{\tan\phi}{a} + \fcoriolis \right] \der{u}{P} = \der{~}{y} \left[ \frac{\partial \Phi}{\partial P} \right] = \der{~}{y} \left[ \frac{1}{\rho} \right] \]
\noindent Reste à employer l'équation d'état des gaz parfaits~$P=\rho\,R\,T$ pour faire apparaître la température
\[ \left[ 2 \, u \, \dfrac{\tan\phi}{a} + \fcoriolis \right] \der{u}{P} = R \, \frac{\partial}{\partial y} \left[ \frac{T}{P} \right] 
\qquad \Rightarrow \qquad
\boxed{ \left[ \textcolor{brown}{2\,u\,\frac{\tan\phi}{a}} + \textcolor{red}{\fcoriolis} \right] \der{u}{P} = \frac{R}{P} \, \frac{\partial T}{\partial y} }
\]
\noindent L'expression encadrée de l'équilibre du vent thermique vient de la constatation finale que l'on peut sortir le terme en pression à l'intérieur de la dérivée à droite puisque~$P$ est une coordonnée supposée fixe par définition de la dérivée partielle suivant~$y$. Les termes sont colorés en fonction de l'équilibre dans lequel on se trouve : \textcolor{brown}{équilibre cyclostrophique} (exemple sur Vénus) ou \textcolor{red}{équilibre géostrophique} (exemple sur Mars ou la Terre). Dans le cas où la situation est ambigüe, il faut conserver les deux termes.

\sk
L'\voc{équilibre du vent thermique} exprime un lien diagnostique entre les variations verticales du vent zonal et les variations méridiennes de la température. Sur des planètes où la mesure de température est aisée à mesurer (e.g. par télédétection infrarouge) par rapport au vent, cet équilibre est employé pour calculer un champ de vent (appelé vent thermique) associé à un champ de température. Invariablement, l'équilibre du vent thermique peut permettre de déduire les variations de température associées à un vent donné. Il s'agit d'un équilibre entre deux champs~température$\leftrightarrow$vent, sans relation de causalité température$\rightarrow$vent ou vent$\rightarrow$température.



\newpage
\section{Instabilités barotropes et baroclines}
\sk
L'équation du vent thermique indique que les jets d'altitude dans la branche descendante de la cellule de Hadley (aux moyennes latitudes) conduisent à un renforcement des gradients latitudinaux de température, qui ne peuvent être résorbés par la circulation de Hadley. Aux moyennes latitudes, les énergies cinétique et thermique sont plus facilement redistribuées par les instabilités non-axisymétriques que par la circulation zonale axisymétrique. L'écoulement zonal axisymétrique des moyennes latitudes terrestres et martiennes peut ainsi donner naissance à une circulation non axisymétrique par des instabilités barotropes et baroclines. A partir d'une certaine latitude, ces instabilités sont bien plus efficaces pour redistribuer l'énergie.

\sk
Les perturbations \voc{barotropes} de l'écoulement moyen se développent en extrayant de l'énergie cinétique au cisaillement horizontal de vent de cet écoulement moyen (ex: courant-jet de forte amplitude). Les tourbillons barotropes ont une structure verticale constante avec l'altitude et transportent de la quantité de mouvement selon la latitude, afin de réduire le cisaillement qui leur a donné naissance.

\sk
L'instabilité \voc{barocline} résulte au contraire des gradients latitudinaux de température aux moyennes latitudes, associés à un cisaillement vertical de vent par l'équilibre du vent thermique. Les ondes baroclines générées transportent de la chaleur (et un peu de quantité de mouvement) en latitude et en altitude pour réduire l'inclinaison des isentropes qui leur a donné naissance. Les perturbations baroclines se développent par conversion de l'énergie potentielle disponible de l'écoulement zonal moyen en énergie cinétique.

%%\note{Stabilité des états d'équilibre~\donc~Réponse à une perturbation~:~s'amplifie-t-elle ? A chaque équilibre son instabilité associée~:~ exemple équilibre hydrostatique et instabilités convectives} \note{Courants-Jet d'altitude branche Hadley descendante~$+$~Renforcement gradients latitudinaux de température~\donc~Instabilités aux moyennes latitudes, Notion d'Energie Potentielle Disponible de l'écoulement zonal moyen.}





\end{document}
