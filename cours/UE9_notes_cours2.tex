\documentclass[a4paper,DIV16,10pt]{scrartcl}
%%%%%%%%%%%%%%%%%%%%%%%%%%%%%%%%%%%%%%%%%%%%%%%%%%%%%%%%%%%%%%%%%%%%%%%%%%%%%%%%%%%
\usepackage{texcours}
%%%%%%%%%%%%%%%%%%%%%%%%%%%%%%%%%%%%%%%%%%%%%%%%%%%%%%%%%%%%%%%%%%%%%%%%%%%%%%%%%%%
\newcommand{\zauthor}{Aymeric SPIGA}
\newcommand{\zaffil}{Laboratoire de Météorologie Dynamique}
\newcommand{\zemail}{aymeric.spiga@upmc.fr}
\newcommand{\zcourse}{Physique des atmosphères planétaires}
\newcommand{\zcode}{UE9}
\newcommand{\zuniversity}{Sorbonne Université (Faculté des Sciences)}
\newcommand{\zlevel}{M2 Planétologie}
\newcommand{\zsubtitle}{Fiches complémentaires du cours 2}
\newcommand{\zlogo}{\includegraphics[height=1.5cm]{/home/aspiga/images/logo/LOGO_SU_HORIZ_SIGNATURE_CMJN_JPEG.jpg}}
\newcommand{\zrights}{Copie et usage interdits sans autorisation explicite de l'auteur}
\newcommand{\zdate}{\today}
%%%%%%%%%%%%%%%%%%%%%%%%%%%%%%%%%%%%%%%%%%%%%%%%%%%%%%%%%%%%%%%%%%%%%%%%%%%%%%%%%%%
\begin{document} \inidoc
%%%%%%%%%%%%%%%%%%%%%%%%%%%%%%%%%%%%%%%%%%%%%%%%%%%%%%%%%%%%%%%%%%%%%%%%%%%%%%%%%%%

%\newpage
%\section{Rayonnement thermique et corps noir}
%\sk
Le Soleil qui se situe à une distance considérable dans le vide spatial nous procure une sensation de chaleur. De même, placer sa main sur le côté d'un radiateur en fonctionnement sans le toucher procure une sensation de chaleur instantanée qui ne peut être attribuée à un transfert convectif entre le radiateur et la main. Cet échange de chaleur est attribué au contraire à l'émission d'ondes électromagnétiques par la matière du fait de sa température; on parle d'émission de \voc{rayonnement thermique}. Tous les corps émettent du rayonnement thermique. La transmission de cette énergie entre une source et une cible ne nécessite pas la présence d'un milieu intermédiaire matériel. 
%Le but de cette section est d'en étudier les principales propriétés.

%\sk
On appelle \voc{corps noir} un objet dont la surface est idéale et satisfait les trois conditions suivantes~:
\begin{description}
\item[émetteur parfait] un corps noir rayonne plus d’énergie radiative à chaque température et pour chaque longueur d’onde que n'importe quelle autre surface,
\item[absorbant parfait] un corps noir absorbe complètement le rayonnement incident selon toutes les directions de l'espace et toutes les longueurs d'onde,
\item[source lambertienne] un corps noir émet du rayonnement de façon isotrope
\end{description}

\sk
Un corps noir est à l'équilibre thermodynamique avec son environnement. On peut montrer qu'un tel corps émet du rayonnement qui dépend seulement de sa température et non de sa nature. La définition du corps noir, et les développements théoriques qui l'accompagnent, sont partis du constat, fait notamment par les céramistes, qu'un objet placé dans un four à haute température devient rouge en même temps que les parois du four quelle que soit sa taille, sa forme ou le matériau qui le compose. Un exemple de source utilisée pour étudier expérimentalement le modèle du corps noir consiste à construire une enceinte chauffée, totalement hermétique, et y percer un trou pour y mesurer le flux énergétique émis [figure~\ref{fig:four}]

\figside{0.35}{0.15}{\figwallace/Radiation/radiation_Page_10_Image_0001.png}{L'énergie entrant par une petite fente dans une enceinte subit des réflexions sur la paroi jusqu'à ce qu'elle soit absorbée. L'ouverture dans la paroi d'une enceinte chauffée apparaît comme une source de type corps noir. Un absorbant presque parfait est aussi un émetteur presque parfait. Ce type de four a été employé au début du XXe siècle pour évaluer expérimentalement les prédictions théoriques de Planck. Source~: Wallace and Hobbs, Atmospheric Science, 2006.}{fig:four}


%\section{Loi de Planck}
%\sk
L'émission de rayonnement par le corps noir est décrite par une luminance énergétique spectrale~$L_{\lambda}$, notée $B_\lambda$ dans ce qui suit\footnote{Correspond au nom anglais \emph{blackbody}}. La loi de variation de~$B_\lambda$ selon la température~$T$ est donnée par la \voc{loi de Planck}\footnote{La luminance spectrale $B_\nu$ est déterminée d'une façon similaire. La démonstration de la loi de Planck fait appel à des notions de quantification d'énergie et de thermodynamique statistique qui sont hors programme dans le cadre de ce cours.} $$ B_\lambda(T) = \frac{C_1 \, \lambda^{-5}}{\pi \, \left( e^{ C_2 / \lambda T}-1\right) } $$ où $C_1$ et $C_2$ sont des constantes. Comme le rayonnement du corps noir est isotrope, l'émittance spectrale du corps noir, obtenue par intégration sur toutes les directions de l'espace, vaut $ M_\lambda(T) = \pi \, B_\lambda(T) $. 

%\figun{0.5}{0.25}{\figfrancis/WH_BBrad}{Courbes de luminance spectrale d'un corps noir pour différentes températures. La courbe en pointillés indique la position du maximum en fonction de $T$.}{fig:BBrad} 
\figside{0.5}{0.25}{\figwallace/Radiation/radiation_Page_11_Image_0001.png}{Courbes de luminance spectrale d'un corps noir pour différentes températures. La courbe en pointillés indique la position du maximum en fonction de $T$. Source~: Wallace and Hobbs, Atmospheric Science, 2006.}{fig:BBrad} 

\sk
Les variations de la fonction~$B_\lambda$ sont illustrées sur la figure~\ref{fig:BBrad}. L'émission de rayonnement par le corps noir ne dépend que de la longueur d'onde~$\lambda$ et de la température~$T$ du corps. A une température donnée, le rayonnement émis est parfaitement déterminé pour chaque longueur d'onde; dans un domaine spectral particulier, le rayonnement émis ne dépend que de la température du corps noir.



%
%\newpage
%\section{Loi de Wien du corps noir}
%\sk
On observe sur la figure \ref{fig:BBrad} que, lorsque~$T$ augmente, la maximum de la luminance spectrale~$B_\lambda$, appelé \voc{maximum d'émission}, se décale vers les longueurs d'onde courtes, c'est-à-dire correspond à des photons de plus en plus énergétiques. La loi exacte, appelée \voc{loi de déplacement de Wien}, s'obtient en dérivant $B_\lambda$ par rapport à $\lambda$, ce qui permet d'obtenir $$ \boxed{ \lambda\e{max} \, T = 2.898 \times 10^{-3} \, \textrm{(mètres~K)} } $$ où $\lambda_{max}$ est la longueur d'onde du maximum de luminance spectrale~$B_\lambda$. La longueur d'onde du maximum d'émission~$\lambda\e{max}$ est ainsi inversement proportionnelle à la température du corps émetteur. Une formulation alternative est que $\nu\e{max}$ est proportionelle à $T$.

\figun{0.6}{0.45}{/home/aymeric/Big_Data/BOOKS/pierrehumbert_pics/9780521865562c03_fig001.jpg}{Source~: R. Pierrehumbert, Principles of Planetary Climates, CUP, 2010.}{wvl} 


%
%\newpage
%\section{Domaine de l'énergie solaire}
%\sk
Le Soleil peut être considéré en bonne approximation comme un corps noir car il absorbe tout le rayonnement incident. Sa \ofg{couleur} est dûe à du rayonnement émis et, plus précisément, correspond aux longueurs d'onde où le maximum de rayonnement est émis. D'après la loi de Wien, le Soleil, dont l'enveloppe externe a une température autour de~$6000$~K, a donc un maximum d'émission situé dans le visible à $\lambda\e{max} = 0.5 \mu$m, proche du maximum de sensibilité de l'oeil humain [figure~\ref{fig:BBmax} haut]. Au contraire, la surface terrestre, dont la température typique est d'environ~$288$~K, voit son maximum d'émission situé dans l'infrarouge vers 10~$\mu$m, alors que le rayonnement émis dans les longueurs d'ondes visible est négligeable [figure~\ref{fig:BBmax} bas]. Un raccourci usuel est donc de dire que \ofg{la Terre émet du rayonnement (thermique) dans l'infrarouge alors que le Soleil émet dans le visible}. En toute rigueur, cette affirmation ne parle que du voisinage du maximum d'émission, où la contribution au flux intégré selon toutes les longueurs d'onde est la plus significative. Il est ainsi plus exact de dire que, dans l'atmosphère, la région du spectre où~$\lambda$ est inférieure à environ 4~$\mu$m est dominée par le rayonnement d'origine solaire, alors qu'au-delà, le rayonnement est surtout d'origine terrestre. Il n’y a pratiquement pas de recouvrement entre la partie utile du spectre du rayonnement solaire et celui d’un corps de température ambiante; ce fait est d'une grande importance pour les phénomènes de type effet de serre, qui sont abordés plus loin dans ce cours. On désigne ainsi souvent le rayonnement d'origine solaire par le terme \voc{ondes courtes} et le rayonnement d'origine terrestre par le terme \voc{ondes longues}.

\figsup{0.65}{0.2}{decouverte/cours_meteo/6000K.jpg}{decouverte/cours_meteo/earth.jpg}{Courbes de luminance spectrale d'un corps noir pour différentes températures correspondant notamment au Soleil (haut) et à la Terre (bas). La quantité représentée ici est l'émittance spectrale~$M_\lambda = \pi \, B_\lambda$. Noter la différence d'indexation de l'abscisse et l'ordonnée sur les deux schémas. Le rayonnement thermique émis par la Terre est plusieurs ordres de grandeur moins énergétique que celui émis par le Soleil et le maximum d'émission se trouve à des longueurs d'onde plus grandes (infrarouge pour la Terre au lieu de visible pour le Soleil). Source : \url{http://hyperphysics.phy-astr.gsu.edu/hbase/bbrc.html}.}{fig:BBmax}

%
%\newpage
%\section{Loi de Stefan-Boltzmann du corps noir}
%\sk
La \voc{loi de Stefan-Boltzmann}\footnote{Joseph Stefan met expérimentalement en évidence en 1879 la dépendance de l'émittance en puissance quatrième de la température. Ludwig Boltzmann, à qui l'on doit également des résultats fondamentaux sur l'entropie et l'atomisme, prouve en 1884 le résultat par des arguments théoriques.} donne la valeur de l'intégrale sur toutes les longueurs d'ondes et dans tout l'espace\footnote{On entend par là toutes les directions du demi-espace extérieur au corps considéré.} de la courbe du corps noir, décrite par la loi de Planck et illustrée par les figures \ref{fig:BBrad} et \ref{fig:BBmax}. Cette loi donne donc l'expression d'une densité de flux énergétique~$F$ ou plus spécifiquement, puisque le corps noir est une source de rayonnement, d'une émittance totale~$M$. Cette dernière s'obtient tout d'abord avec une intégration par rapport à~$\lambda$ de la luminance énergétique spectrale~$B_\lambda$ donnée par la loi de Planck, afin d'obtenir la luminance énergétique~$B$. On déduit ensuite l'émittance totale~$M$ en intégrant selon toutes les directions de l'espace; comme le rayonnement du corps noir est isotrope, $M$ s'obtient à partir de~$B$ simplement en multipliant par~$\pi$. La loi de Stefan-Boltzmann établit que le flux net surfacique~$M$ émis par un corps noir ne dépend que de sa température par une dépendance type loi de puissance $$ \boxed{ M\e{corps noir} = \sigma \, T^4 } $$ avec~$\sigma=5.67 \times 10^{-8} \textrm{~W~m}^{-2}\textrm{~K}^{-4}$ appelée constante de Stefan-Boltzmann. La loi de Stefan-Boltzmann, comme la loi de Planck dont elle dérive, stipule que l'émittance~$M$ d'un corps pouvant être considéré en bonne approximation comme un corps noir ne dépend que de sa température et non de sa nature. Cette loi indique par ailleurs que l'émittance~$M$ augmente très rapidement avec la température -- de par la puissance quatrième impliquée.

%\section{Corps gris et émissivité}
%\sk
Le corps noir est un modèle idéal d'absorbant qu'en pratique on ne rencontre pas dans la nature. Par exemple, le charbon noir est un absorbant parfait, mais seulement dans les longueurs d'onde visible. La plupart des objets ressemblent néanmoins au corps noir, au moins à certaines températures et pour certaines longueurs d'onde considérées en pratique. Dans le cas d'un corps qui n'est pas un absorbant parfait, on parle d'un \voc{corps gris}. A température égale, un corps gris n'émet pas autant qu'un corps noir dans les mêmes conditions. Pour évaluer l'énergie émise par un corps gris par comparaison à celle qu'émettrait le corps noir dans les mêmes conditions, on définit un coefficient appelé \voc{émissivité} $\epsilon_\lambda$ compris entre~$0$ et~$1$ et égal au rapport entre la luminance spectrale du corps~$L_\lambda$ et celle du corps noir~$B_\lambda$: $ \epsilon_\lambda=L_\lambda / B_\lambda(T)$ En toute généralité, l'émissivité~$\epsilon_{\lambda}$ d'une surface à une longueur d'onde~$\lambda$ dépend de ses propriétés physico-chimiques, de sa température et de la direction d'émission\footnote{Par exemple, les métaux, matériaux conducteurs de l'électricité, ont une émissivité faible (sauf dans les directions rasantes) qui croît lentement avec la température et décroît avec la longueur d'onde ; au contraire, les diélectriques, matériaux isolant de l'électricité, ont une émissivité élevée qui augmente avec la longueur d'onde et se révèlent lambertiens sauf pour les directions rasantes où l'émissivité décroît significativement.}.

\sk
On peut définir une émissivité totale intégrée~$\epsilon$ qui permet d'exprimer l'émittance~$M$ d'un corps gris $$ \boxed{\SB} $$ Des valeurs de l'émissivité totale~$\epsilon$ pour certains matériaux sont données dans le tableau~\ref{tab:emiss}~: l'eau, la neige, les roches basaltiques ont des émissivités proches de~$1$ et sont donc des corps noirs en bonne approximation. 


\begin{table}[h!]
\label{tab:emiss}
\begin{center}
\footnotesize
\begin{tabular}{||c|c||c|c||c|c||}
\hline
Matériau & Emissivité~$\epsilon$ & Matériau & Emissivité~$\epsilon$ & Matériau & Emissivité~$\epsilon$ \\
\hline
Cuivre poli & 0.03 		& Cuivre oxydé & 0.5 		& Béton & 0.7 à 0.9 	\\
Carbone & 0.8 			& Lave (volcan actif) & 0.8 	& Suie & 0.95		\\
Ville & 0.85 			& Peinture blanche & 0.87 	& Peinture noire & 0.94 \\
Désert & 0.85 à 0.9 		& Herbe & 0.9 à 0.95		& Forêt & 0.95 		\\
Nuages cirrus & 0.10 à 0.90 	& Nuages cumulus & 0.25 à 0.99	& Eau & 0.92 à 0.97  	\\
Neige âgée & 0.8 		& Neige fraîche & 0.99		& &			\\
\hline
\end{tabular}
\normalsize
\caption{\emph{Quelques valeurs usuelles d'émissivité à la température ambiante (pour un rayonnement infrarouge). Source~: Hecht, Physique, 1999 -- avec quelques ajouts d'après site CNES}}
\end{center}
\end{table}


%
\newpage
\section{Constante solaire}
\sk
La distance Soleil-Terre est beaucoup plus grande que les rayons de la Terre et du Soleil. Ainsi, d'une part, le rayonnement solaire arrive au niveau de l'orbite terrestre en faisceaux pratiquement parallèles. D'autre part, la luminance en différents points de la Terre ne varie pas. On peut par conséquent définir une valeur moyenne de la densité de flux énergétique du rayonnement solaire au niveau de l'orbite terrestre, reçue par le système surface~+~atmosphère. Elle est désignée par le terme de \voc{constante solaire} notée~$\mathcal{F}\e{s}$. Les mesures indiquent que
\[ \mathcal{F}\e{s} = 1368 \text{~W~m}^{-2} \qquad \text{pour la Terre} \]

\sk
La constante solaire est une valeur instantanée côté jour~: le rayonnement solaire reçu au sommet de l'atmosphère en un point donné de l'orbite varie en fonction de l'heure de la journée et de la saison considérée (c'est-à-dire la position de la Terre au cours de sa révolution annuelle autour du Soleil)\footnote{En réalité, la constante solaire~$\mathcal{F}\e{s}$ varie elle-même d'environ~$3$~W~m$^{-2}$ en fonction des saisons à cause de l'excentricité de l'orbite terrestre, qui n'est pas exactement circulaire. De plus, elle peut varier évidemment en fonction des cycles solaires, néanmoins sans influence majeure sur la température des basses couches atmosphériques (troposphère et stratosphère).}. On peut donc définir un \voc{éclairement solaire moyen} noté~$\mathcal{F}\e{s}'$ reçu par la Terre qui intègre les effets diurnes et saisonniers. Autrement dit, $\mathcal{F}\e{s}$~est l'éclairement instantané reçu par un satellite en orbite autour de la Terre~; $\mathcal{F}\e{s}'$ est la valeur que l'on obtiendrait si l'on faisait la moyenne d'un grand nombre de mesures instantanées du satellite à diverses heures et saisons. 

\figside{0.5}{0.2}{decouverte/cours_dyn/incoming.png}{Energie reçue du Soleil par le système Terre. Source~: McBride and Gilmour, \emph{An Introduction to the Solar System}, CUP 2004.}{fig:eqrad}

\sk
On admet ici que~$\mathcal{F}\e{s}'$ peut être calculé en considérant que le flux total reçu du Soleil l'est à travers un disque de rayon le rayon~$R$ de la Terre (il s'agit de l'ombre projetée de la planète, voir Figure~\ref{fig:eqrad}). A cause de l'incidence parallèle, le flux énergétique intercepté par la Terre vaut donc~$\Phi = \pi \, R^2 \, \mathcal{F}\e{s}$. L'éclairement moyen à la surface de la Terre est alors $$\mathcal{F}\e{s}' = \frac{\Phi}{4 \, \pi \, R^2}$$ le dénominateur étant l'aire de la surface complète de la Terre. On obtient ainsi
\[ \boxed{ \mathcal{F}\e{s}' = \frac{\mathcal{F}\e{s}}{4} } \]

\sk
La valeur de la constante solaire peut s'obtenir par le calcul. Le soleil est considéré en bonne approximation comme un corps noir de température~$T_{\sun} = 5780$~K. D'après la loi de Stefan-Boltzmann, son émittance est $M = \sigma \, T_{\sun}^4$ donc le flux énergétique~$\Phi_{\sun}$ émis par le Soleil de rayon~$R_{\sun} = 7 \times 10^5$~km est~$\Phi_{\sun} = 4 \, \pi \, R_{\sun}^2 \, \sigma \, T_{\sun}^4$. Ce flux énergétique est rayonné dans tout l'espace~: à une distance~$d$ du soleil il est réparti sur une sphère de centre le soleil et de rayon~$d$, donc de surface~$4 \, \pi \, d^2$. A cette distance, l'éclairement~$\mathcal{F}$, c'est-à-dire la densité de flux énergétique reçue en W~m$^{-2}$, s'écrit donc
\[ \mathcal{F} = \frac{\Phi_{\sun}}{4 \, \pi \, d^2} = \frac{4 \, \pi \, R_{\sun}^2 \, \sigma \, T_{\sun}^4}{4 \, \pi \, d^2} = \sigma \, T_{\sun}^4 \, \left( \frac{R_{\sun}}{d} \right)^2 \]
Si l'on prend~$d$ égal à la distance Terre-Soleil, $\mathcal{F}$ définit ainsi la constante solaire~$\mathcal{F}\e{s}$.
%\[ \mathcal{F}\e{s} = \frac{{\mathcal{F}\e{s}}^{\text{Terre}}}{d\e{soleil}^2} \]

%Variation de la constante solaire : Bien que l’intensité du soleil ait subit des variations depuis la formation de la Terre, on peut s’attendre à ce qu’elle soit stable sur une période de 1000 ans. On mesure mal la constante solaire, mais les mesures récentes, même avec leurs incertitudes, semblent indiquer que le soleil ne peut pas expliquer le réchauffement récent. Notons toutefois que les simulations actuelles ne tiennent pas compte des fluctuations possibles du rayonnement solaire (négligeable a priori).
%%%% pas sûr du dernier point.

%
%\newpage 
%\section{Albédo} 
%\sk
En sciences de l'atmosphère, les coefficients de réflexion~$\rho$ et~$\rho_{\lambda}$ sont souvent désignés sous le nom respectivement d'\voc{albédo} noté~$A$ et d'albédo spectral noté~$A_{\lambda}$. Plus la surface réfléchit une grande partie du rayonnement électromagnétique incident, plus l'albédo est proche de~$1$. L'albédo spectral~$A_{\lambda}$ peut varier significativement en fonction de la longueur d'onde : voir l'exemple de la neige fraîche donné ci-dessus. 

\sk
De par la diversité des surfaces terrestres, et de la variabilité de la couverture nuageuse, les valeurs de l'albédo~$A$ varient fortement d'un point à l'autre du globe terrestre~: il est élevé pour de la neige fraîche et faible pour de la végétation et des roches sombres [table~\ref{tab:albedo}]. L'albédo de l'océan est faible, particulièrement pour des angles d'incidence rasants -- il dépend ainsi beaucoup de la distribution des vagues. 

\begin{table}\label{tab:albedo}
\begin{center}
\begin{tabular}{|c|c|c|c|}
\hline
Type & albédo~$A$ & Type & albédo~$A$ \\
\hline
Surface de lac & 0.02 à 0.04 & Surface de la mer & 0.05 à 0.15 \\
Asphalte & 0.07 & Mer calme (soleil au zenith) & 0.10 \\
Forêt équatoriale & 0.10 & Roches sombres, humus & 0.10 à 0.15 \\
Ville & 0.10 à 0.30 & Forêt de conifères & 0.12 \\
Cultures & 0.15 à 0.25 & Végétation basse, verte & 0.17 \\
Béton & 0.20 & Sable mouillé & 0.25 \\
Végétation sèche & 0.25 & Sable léger et sec & 0.25 à 0.45 \\
Forêt avec neige au sol & 0.25 & Glace & 0.30 à 0.40 \\
Neige tassée & 0.40 à 0.70 & Sommet de certains nuages & 0.70 \\
Neige fraîche & 0.75 à 0.95 & & \\
\hline
\end{tabular}
\caption{\emph{Quelques valeurs usuelles d'albédo (rayonnement visible). D'après mesures missions NASA et ESA.}}
\end{center}
\end{table}

\sk
L'\voc{albédo planétaire} est noté~$A\e{b}$ et défini comme la fraction moyenne de l'éclairement~$E$ au sommet de l'atmosphère (noté également~$\mathcal{F}\e{s}'$) qui est réfléchie vers l'espace~: il comprend donc la contribution des surfaces continentales, de l'océan et de l'atmosphère. Il vaut~$0.31$ pour la planète Terre~: une partie significative du rayonnement reçu du Soleil par la Terre est réfléchie vers l'espace\footnote{L'albédo planétaire est par exemple encore plus élevé sur Vénus ($0.75$) à cause de la couverture nuageuse permanente et très réfléchissante de cette planète.}. Ainsi le système Terre reçoit une densité de flux énergétique moyenne~$F\e{reçu}$ en W~m$^{-2}$ telle que
\[ F\e{reçu} = (1-A\e{b}) \, \mathcal{F}\e{s}' \] 
donc un flux énergétique~$\Phi\e{reçu}$ (en W) qui s'exprime
\[ \Phi\e{reçu} = \pi \, R^2 \, (1-A\e{b}) \, \mathcal{F}\e{s} \]
%L'albédo de Bond~ désigne l'albédo intégré sur toutes les longueurs d'onde et tous les angles d'incidence.

\sk
La valeur de~$30\%$ de l'albédo planétaire sur Terre est en fait majoritairement dû à l'atmosphère~:  seuls 4\% de l'énergie solaire incidente sont réfléchis par la surface terrestre comme indiqué sur la figure~\ref{fig:diffsep}. L'énergie réfléchie par l'atmosphère vers l'espace, responsable de plus de~$85\%$ de l'albedo planétaire, est diffusée par les molécules ou par des particules en suspension, gouttelettes nuageuses, gouttes de pluie ou aérosols.

\figside{0.4}{0.15}{\figpayan/LP211_Chap2_Page_27_Image_0001.png}{L'énergie solaire incidente est réfléchie vers l'espace par la surface et l'atmosphère d'une planète. La figure montre les différentes contributions à l'albédo planétaire total.}{fig:diffsep}


\newpage
\section{\'Equilibre TOA}
\sk
Nous pouvons exprimer le rayonnement reçu du Soleil par la Terre par une densité de flux énergétique moyenne~$F\e{reçu}$ en W~m$^{-2}$ ou un flux énergétique~$\Phi\e{reçu}$ (en W)
\[ 
F\e{reçu} = (1-A\e{b}) \, \mathcal{F}\e{s}' 
\qquad \qquad
\Phi\e{reçu} = \pi \, R^2 \, (1-A\e{b}) \, \mathcal{F}\e{s}
\] 
La partie du rayonnement reçue du soleil qui est réfléchie vers l'espace est prise en compte via l'albédo planétaire noté~$A\e{b}$. On rappelle par ailleurs que~$\mathcal{F}\e{s}' = \mathcal{F}\e{s} / 4$ où $\mathcal{F}\e{s}$ est la constante solaire.


\sk
Par ailleurs, le système Terre émet également du rayonnement principalement dans les longueurs d'onde infrarouge [figure \ref{fig:eqrad2}]. 
Cette quantité de rayonnement émise au sommet de l'atmosphère radiative est notée $OLR$ pour \emph{Outgoing Longwave Radiation} en anglais.
A l'équilibre, la planète Terre doit émettre vers l'espace autant d'énergie qu'elle en reçoit du Soleil, donc
on obtient la relation générale appelée \emph{TOA} pour \emph{Top-Of-Atmosphere} en anglais, correspondant
au bilan radiatif au sommet de l'atmosphère
\[ \boxed{\TOA} \] 
La principale difficulté qui sous-tend les divers modèles pouvant être proposés réside dans l'expression du terme~$OLR$.




\newpage
\section{Bilan simple : température équivalente}
\sk
Dans l'équilibre~\emph{TOA}, la manière la plus simple de définir~$OLR$ pour entamer un calcul préliminaire est comme suit. On fait l'hypothèse, assez réaliste en pratique, que la surface de la Terre est comme un corps noir, c'est-à-dire que son émissivité est très proche de~$1$ dans l'infrarouge où se trouve le maximum d'émission. D'après la loi de Stefan-Boltzmann, la densité de flux énergétique~$F\e{émis}$ émise par la Terre en W~m$^{-2}$ s'exprime
\[ F\e{émis} = \sigma \, {T\e{eq}}^4 \]
où~T\e{eq} est la \voc{température équivalente} du système Terre que l'on suppose uniforme sur toute la planète. Autrement dit, $T\e{eq}$ est la température équivalente d'un corps noir qui émettrait la quantité d'énergie~$F\e{émis}$. Le flux énergétique~$\Phi\e{émis}$ émis par la surface de la planète Terre s'exprime
\[ \Phi\e{émis} = 4 \, \pi \, R^2 \, F\e{émis} = 4 \, \pi \, R^2 \, \sigma \, {T\e{eq}}^4 \]
Contrairement au cas de l'énergie visible, il n'y a pas lieu de prendre en compte le contraste jour/nuit, car le rayonnement thermique émis par la Terre l'est à tout instant par l'intégralité de sa surface. La seule limite éventuellement discutable est l'uniformité de la température de la surface de la Terre, ce qui est irréaliste en pratique. On peut souligner cependant que, même dans le cas d'une planète n'ayant pas une température uniforme ou ne se comportant pas comme un corps noir, le rayonnement émis vers l'espace doit être égal en moyenne à $\sigma \, {T\e{eq}}^4$.
%% CHANGER LES SLIDES, ne pas utiliser P

\figsup{0.31}{0.17}{decouverte/cours_dyn/incoming.png}{decouverte/cours_dyn/emission.png}{Equilibre radiatif simple : à gauche, l'énergie reçue du Soleil par le système Terre ; à droite, l'énergie émise par le système Terre. Source~: McBride and Gilmour, \emph{An Introduction to the Solar System}, CUP 2004.}{fig:eqrad2}

\sk
A l'équilibre, la planète Terre doit émettre vers l'espace autant d'énergie qu'elle en reçoit du Soleil (équilibre \emph{TOA}). Ceci peut s'exprimer par unité de surface
\[ \boxed{ F\e{reçu} = F\e{émis} } \]
ou, pour un résultat similaire, en considérant l'intégralité de la surface planétaire
\[ \Phi\e{reçu} = \Phi\e{émis} \]
ce qui permet de déterminer la température équivalente en fonction des paramètres planétaires
\[ \boxed{
T\e{eq} = \bigg[ \frac{\mathcal{F}\e{s}'\,(1-A\e{b})}{\sigma} \bigg]^{\frac{1}{4}}
} \]


Le calcul présenté ici porte le nom d'\voc{équilibre radiatif simple}. On y néglige les effets de l'atmosphère (sauf l'albédo) puisqu'on suppose que le rayonnement atteint la surface, ou est rayonné vers l'espace, sans être absorbé par l'atmosphère. La température équivalente est ainsi la température qu'aurait la Terre si l'on négligeait tout autre influence atmosphérique que la réflexion du rayonnement solaire incident. Les valeurs de $T\e{eq}$ pour quelques planètes telluriques sont données dans la table \ref{tab:planets}. On note que la température équivalente de Vénus est plus faible que celle de la Terre, bien qu'elle soit plus proche du Soleil, à cause de son fort albédo~; la formule indique bien que, plus le pouvoir réfléchissant d'une planète est grand, plus la température de sa surface est froide. Par ailleurs, comme indiqué par les calculs du tableau~\ref{tab:planets}, on remarque que la température équivalente, si elle peut renseigner sur le bilan énergétique simple de la planète, ne représente pas correctement la valeur de la température de surface. Par exemple, la température équivalente pour la Terre est~$T\e{eq} = 255 K = -18^{\circ}$C, bien trop faible par rapport à la température de surface effectivement mesurée. Il faut donc avoir recours à un modèle plus élaboré.

\begin{table}\label{tab:planets} \begin{center} \begin{tabular}{lccccc} & \emph{Mercure} & \emph{V\'enus} & \emph{Terre} & \emph{Mars} & \emph{ Titan} \\ \hline $d\e{soleil}$ (UA) & 0.39 & 0.72 & 1 & 1.5 & 9.5 \\ $\mathcal{F}\e{s}\,$(W~m$^{-2}$) & $8994$ & $2614$ & $1367$ & $589$ & $15$ \\ $A\e{b}$ & $0.06$ & $0.75$ & $0.31$ & $0.25$ & $0.2$ \\ \textcolor{blue}{$T\e{surface}$ (K)} & \textcolor{blue}{$100/700$~K} & \textcolor{blue}{$730$} & \textcolor{blue}{$288$} & \textcolor{blue}{$220$} & \textcolor{blue}{$95$} \\ \hline $T\e{eq}$~(K) & $439$ & $232$ & $254$ & $210$ & $86$\\ \end{tabular} \caption{\emph{Comparaison des facteurs influençant la température équivalente du corps noir pour différentes planètes du système solaire.}} \end{center} \end{table}
%    Mercure & 0.39 & 8994 & 0.06 & 439 \\
%    Vénus & 0.72 & 2639 & 0.78 & 225 \\
%    Terre & 1 & 1368 & 0.30 & 255 \\
%    Mars & 1.52 & 592 & 0.17 & 216 \\


\newpage
\section{\'Epaisseur optique}
\sk
Considérons une espèce~$X$ bien mélangée dans l'atmosphère, qui absorbe dans un intervalle de longueur d'onde donné. A la longueur d'onde~$\lambda$, son \voc{épaisseur optique}~$t_\lambda$ s'écrit
\[ \boxed{ t_\lambda = \int_{0}^{z\e{sommet}} \, k_\lambda \, \rho_X \, \dd z } \]
où $k_\lambda$ est un coefficient d'absorption massique en m$^2$~kg$^{-1}$ et $\rho_X$ est la densité d'absorbant~X. Le nom d'épaisseur optique est assez intuitif. Si un flux de rayonnement~$\Phi_\lambda$ à la longueur d'onde~$\lambda$ est émis à la base de l'atmosphère, le flux observé au sommet de l'atmosphère est d'autant plus réduit qu'à cette longueur d'onde l'épaisseur optique de l'atmosphère traversée est grande\footnote{Si l'extinction est uniquement due à de l'absorption, sans diffusion, on a une relation directe entre l'épaisseur optique et le coefficient d'absorption de la couche~: 
\[\alpha_\lambda = 1 - e^{- \frac{t_\lambda}{\cos\theta}} \] où~$\theta$ est l'angle d'incidence du rayonnement danns la couche.}. La formule ci-dessus ne fait qu'exprimer le fait que la réduction du flux (l'extinction) est plus d'autant plus marquée que 
\begin{citemize}
\item l'espèce considérée est très absorbante dans la longueur d'onde considérée ($k_\lambda$ grand)~;
\item l'espèce considérée est présente en grande quantité ($\rho_X$ grand).
\end{citemize}
Ainsi, le dioxyde de carbone~CO$_2$, bien qu'étant un composant minoritaire ($\rho$ faible), peut atteindre des épaisseurs optiques très grandes dans les intervalles de longueur d'onde où il est très fortement absorbant ($k_\lambda$ élevé), par exemple dans l'infrarouge autour de~$15$~$\mu$m. Autrement dit, un composant minoritaire en quantité peut avoir un rôle majoritaire radiativement.


\newpage
\section{Modèle à deux faisceaux : écriture}
\sk
Le modèle à deux faisceaux est un bon compromis entre simplicité
et illustration de concepts importants. Il est une version simplifiée
de l'équation de Schwarzschild du transfert radiatif.
Ce modèle entend élucider
les transferts de rayonnement dans l'infrarouge entre
les couches qui composent la colonne atmosphérique. Les
hypothèses simplificatrices suivantes sont réalisées
\begin{citemize}
\item couches atmosphériques plan-parallèle (sphéricité négligée)
\item phénomènes d'absorption négligés dans le visible (transparence au visible)
\item phénomènes de diffusion (\emph{scattering}) négligés dans l'infrarouge
\item \emph{gray gas} dans l'infra-rouge : on considère que le coefficient d'absorption
du gaz est indépendant de la longueur d'onde~$\lambda$ ($k_{\lambda} = k$ pour tout~$\lambda$),
ce qui implique une hypothèse similaire pour l'épaisseur optique ($\tau_{\lambda} = \tau$ pour tout~$\lambda$).
\end{citemize}
En d'autres termes, on se cantonne dans ce modèle à deux types de phénomènes
\begin{enumerate}
\item Un faisceau de rayonnement infra-rouge de flux~$F$ traversant une couche 
atmosphérique donnée
subit une extinction à cause de l'absorption selon une loi de type Beer-Lambert
\[
\dd F = - F \dd \tau
\]
avec~$\tau$ l'épaisseur optique \emph{gray gas} 
dans l'infra-rouge.
\item Une couche atmosphérique émet un flux de rayonnement thermique~$M$ 
calculé par la loi intégrée de Stefan-Boltzmann ($M=\epsilon\,\sigma\,T^4$)
puisque la majorité de l'émittance est émise dans l'infrarouge pour les températures considérées.
\end{enumerate}

\sk
L'épaisseur optique~$\tau$ peut servir de coordonnée verticale à la place de~$z$
en utilisant la relation entre les deux quantités. La couche atmosphérique
élémentaire considérée est ainsi d'épaisseur~$\dd\tau$ et située à une coordonnée
verticale~$\tau$ qui croît avec l'altitude. 

\sk
Si l'on considère un faisceau ascendant~$F^+(\tau)$ au bas de la couche considérée,
une fois la couche traversée son amplitude est
\[
F^+(\tau) - F^+(\tau) \dd\tau
\]
A ce flux au sommet de la couche, il convient d'ajouter
la contribution de l'émission thermique de la couche vers 
le haut, à savoir~$M\,\dd\tau$.
Le flux total ascendant au sommet de la couche est donc
\[
F^+(\tau+\dd\tau) = F^+(\tau) - F^+(\tau) \dd\tau + M\dd\tau
\]

\sk
Même raisonnement avec le flux descendant~$F^-(\tau+\dd\tau)$ au sommet de la couche considérée,
une fois la couche traversée son amplitude est~$F^-(\tau+\dd\tau) - F^-(\tau) \dd\tau$, où 
l'approximation du terme du second ordre~$F^-(\tau+\dd\tau) \dd\tau \simeq F^-(\tau) \dd\tau$
a été effectuée.
Le flux total descendant au bas de la couche est donc
\[
F^-(\tau+\dd\tau) = F^-(\tau) - F^-(\tau) \dd\tau + M\dd\tau
\]

\sk
Les deux résultats qui précèdent peuvent être transformés 
afin de faire apparaître une dérivée
en utilisant le théorème des accroissements finis
\[
\ddf{F^+}{\tau} = \frac{F^+(\tau+\dd\tau) - F^+(\tau)}{\dd\tau}
\]
\noindent ce qui permet d'obtenir au final
\[
\ddf{F^+}{\tau} = - F^+(\tau) + \epsilon\,\sigma\,T(\tau)^4 \quad [S^+]
\qquad\qquad 
\ddf{F^-}{\tau} = F^-(\tau) - \epsilon\,\sigma\,T(\tau)^4 \quad [S^-]
\]
\noindent $[S^+]$ et~$[S^-]$ sont parfois appelées les relations de Schwarzschild à deux faisceaux.
Il s'agit d'une version très simplifiée des équations de Schwarzschild du transfert radiatif.

\sk
Si l'on souhaite adopter la convention~$\tau=0$ au sommet de l'atmosphère,
et $\tau=\tau_{\infty}$ à la surface en $z=0$, donc adopter un axe
vertical d'épaisseur optique avec~$\tau$ croissant de haut en bas, il
suffit de remplacer~$\tau$ par~$-\tau$ dans les équations précédentes pour obtenir
\[
\boxed{\ddf{F^+}{\tau} = F^+(\tau) - \epsilon\,\sigma\,T(\tau)^4 \quad [S^+]} 
\qquad\qquad 
\boxed{\ddf{F^-}{\tau} = -F^-(\tau) + \epsilon\,\sigma\,T(\tau)^4 \quad [S^-]}
\]









\newpage
\section{Modèle à deux faisceaux : résolution 1}
\sk
Le système d'équations~$[S^+]$ et~$[S^-]$ du modèle à deux faisceaux
est plus simple à résoudre si l'on considère les deux quantités~$\Sigma(\tau)=F^{+}(\tau)+F^{-}(\tau)$ et~$\Delta(\tau)=F^{+}(\tau)-F^{-}(\tau)$ car on obtient
\[
\ddf{\Sigma}{\tau} = \Delta(\tau) \quad [E_\Sigma] 
\qquad\qquad 
\ddf{\Delta}{\tau} = \Sigma(\tau) - 2\,\epsilon\,\sigma\,T(\tau)^4 \quad [E_\Delta]
\]
\noindent Ensuite la résolution impose d'expliciter les conditions aux limites
\begin{enumerate}[label=$\mathcal{C}_\arabic*$]
\item on se place à l'équilibre radiatif donc le flux net~$\Delta$ est constant à tout niveau : $\ddf{\Delta}{\tau}=0$
\item au sommet de l'atmosphère $F^+(\tau=0) = OLR$ (définition de $OLR$) et $F^-(\tau=0) = 0$ (contribution
incidente négligeable du Soleil dans l'infra-rouge), ce qui s'écrit encore~$\Delta(\tau=0)=\Sigma(\tau=0)=OLR$
\item à la surface de température~$T_s$ le bilan radiatif est le suivant : la surface reçoit l'intégralité du rayonnement
solaire incident~$(1-A\e{b}) \, \mathcal{F}\e{s}'$ (visible) plus du rayonnement de l'atmosphère située
juste au-dessus d'elle~$F^-(\tau=\tau_{\infty})$ (infra-rouge) ; de plus elle émet un rayonnement
$\epsilon\,\sigma\,T\e{s}^4$ dans l'infra-rouge vers l'atmosphère\footnote{On a supposé ici pour simplifier les calculs que l'émissivité
de la surface était similaire à l'émissivité de l'atmosphère}
\item on rappelle que selon la relation \emph{TOA}, nous avons $OLR = (1-A\e{b}) \, \mathcal{F}\e{s}'$
\end{enumerate}


\newpage
\section{Modèle à deux faisceaux : résolution 2}
\sk
\paragraph{Conséquence 1} Il est alors possible d'obtenir deux expressions différentes pour~$\Sigma(\tau)$.
Premièrement, en utilisant $[E_\Sigma]$ avec $\mathcal{C}_1$ et $\mathcal{C}_2$, on obtient~$\Sigma(\tau)=OLR \, (1+\tau)$.
Deuxièmement, en utilisant $[E_\Delta]$ avec $\mathcal{C}_1$, on obtient~$\Sigma(\tau)=2\,\epsilon\,\sigma\,T(\tau)^4$.
On obtient le \voc{profil radiatif}, 
c'est-à-dire le profil vertical de température\footnote{Suivant la géométrie
équivalente choisie pour le modèle plan-parallèle, le terme $1+\tau$
peut s'écrire un peu différemment, mais quoiqu'il en soit toujours sous une
forme~$a+b\,\tau$ avec $a,b$ constants. Les conclusions énoncées ici ne sont pas
modifiées.} imposé par les transferts radiatifs dans l'infrarouge
\[
T(\tau) = \sqrt[4]{\frac{OLR\,(1+\tau)}{2\,\sigma\,\epsilon}}
\]

\sk
\paragraph{Conséquence 2} Reste à calculer la température de surface avec ce modèle. D'après $\mathcal{C}_3$, le bilan
au sol s'écrit~$(1-A\e{b}) \, \mathcal{F}\e{s}' + F^-(\tau=\tau_{\infty}) = \epsilon\,\sigma\,T\e{s}^4$.
Il faut donc exprimer les flux ascendant et descendant dans l'infrarouge.
Du fait que $\mathcal{C}_1$ et $\mathcal{C}_2$ nous indiquent que~$\Delta=OLR$, on obtient aisément
\[
F^+(\tau) = \frac{\Sigma+\Delta}{2} = OLR \, (1+\frac{\tau}{2})
\qquad \qquad
F^-(\tau) = \frac{\Sigma-\Delta}{2} = OLR \, \frac{\tau}{2}
\]
\noindent On obtient alors l'expression liant $OLR$
et température de surface~$T\e{s}$
\[
\boxed{\epsilon\,\sigma\,T\e{s}^4 = OLR \, \left( 1 + \frac{\tau_{\infty}}{2} \right)}
\]
\noindent On obtient ainsi une définition quantitative de \voc{l'effet de serre}
\begin{citemize}
\item Dans l'infrarouge, le rayonnement sortant au sommet de l'atmosphère ($OLR$)
est inférieur au rayonnement émis par la surface ($\epsilon\,\sigma\,T\e{s}^4$).
Une partie du rayonnement émis par la surface reste donc piégée par la planète.
\item Avec un albédo et un rayonnement incident constant, donc à~$OLR$ constant (d'après $\mathcal{C}_4$),
augmenter la quantité de gaz à effet de serre (donc augmenter~$\tau_{\infty}$)
conduit à une augmentation de la température de surface~$T\e{s}$.
\end{citemize}

\sk
\paragraph{Conséquence 3} Il est alors instructif de s'intéresser à la température atmosphérique 
proche de la surface~$T(\tau_\infty)$
donnée par le profil radiatif. Cette température ne dépend que de~$OLR$
et s'obtient totalement indépendamment de la température de surface.
On peut alors montrer que
\[
T\e{s} = T(\tau_\infty) \, \sqrt[4]{\frac{2+\tau_\infty}{1+\tau_\infty}} > T(\tau_\infty)
\]
\noindent Tant que l'atmosphère n'est pas
optiquement épaisse dans l'infrarouge (donc tant que~$\tau_\infty$ reste fini),
il existe une \voc{discontinuité entre la surface et l'atmosphère}, la surface
étant toujours plus chaude que l'atmosphère. Cela implique que l'atmosphère
est instable proche de la surface, donc que du mélange turbulent / convectif
apparaît, donc que l'équilibre proche de la surface ne peut être simplement
radiatif mais \voc{radiatif-convectif}. Notons que dans le cas où l'atmosphère est optiquement épaisse,
$T\e{s} = T(\tau_\infty)$, ce qui est tout à fait représentatif des conditions sur Vénus.


%%% figure Salby


%\newpage
%\section{Structures thermiques}
%

Profil radiatif-convectif + profil radiatif + effet de la convection humide

%\figun{0.2}{0.1}{decouverte/pierrehumbert_pics/9780521865562c02_fig001.jpg}{R. Pierrehumbert, Principles of Planetary Climates, CUP, 2010}{fig:profearth}

\figun{0.65}{0.5}{decouverte/pierrehumbert_pics/9780521865562c02_fig002.jpg}{R. Pierrehumbert, Principles of Planetary Climates, CUP, 2010}{fig:profplanet}



\newpage
\section{Profil radiatif convectif (telluriques)}
\sk
Les conditions atmosphériques sont très instables proche d'une surface (en présence d'une telle surface). A cause de la discontinuité entre surface et atmosphère, sous l'action de la diffusion thermique, ou turbulente, entre la surface (chaude) et l'air immédiatement adjacent (plus froid) crée une couche d'air fine approximativement à la température de la surface ; les conditions de température étant plus froides au-dessus, les conditions atmosphériques sont très instables proche de la surface et des mouvements de convection vont se mettre en place pour mélanger l'air sur une certaine épaisseur atmosphérique. Un équilibre dit \voc{radiatif-convectif} prévaut, avec une structure thermique suivant le profil adiabatique, donnant naissance à une troposphère. Au-dessus de la limite radiative-convective (correspondant peu ou prou à la tropopause), les phénomènes radiatifs dominent et donnent naissance à une mésosphère -- ou une stratosphère si un absorbant visible y est présent en quantité suffisante, donnant naissance à une inversion stable à la tropopause.
%% on passait en troposphère dès que le gradient du profil radiatif dépassait celui du profil adiabatique (-g/cp)

\figun{0.4}{0.25}{decouverte/pierrehumbert_pics/9780521865562c03_fig014.jpg}{Figure tirée de R. Pierrehumbert, Principles of Planetary Climates, CUP, 2010}{fig:effetserre2}











\newpage
\section{Profil radiatif convectif (géantes)}
\sk
La présence d'une surface et une hypothèse d'équilibre radiatif imposent donc que le chauffage de la surface conduit inévitablement à de la convection. Que se passe-t-il sur les planètes géantes dépourvues de surface ? L'équilibre radiatif y prévaut également, car à partir d'une certaine profondeur, le gradient radiatif est instable -- et ce, même en l'absence d'une surface qui absorbe le rayonnement solaire. Pour formuler la stabilité de l'équilibre radiatif, on calcule le profil~$\dd T / \dd p$ et on le compare au gradient adiabatique sec ou humide dans l'atmosphère considérée. En dérivant le profil radiatif obtenu dans le cas du modèle à deux faisceaux
\[ T(\tau) = \sqrt[4]{\frac{OLR\,(1+\tau)}{2\,\sigma\,\epsilon}} \]
\noindent par rapport à l'épaisseur optique~$\tau$, nous obtenons
\[ 8 \, \sigma \, T^3 \, \ddf{T}{\tau} = OLR \]
\noindent soit, en utilisant~$\ddf{~}{p} = \ddf{\tau}{p} \ddf{~}{\tau}$
\[ \ddf{T}{\ln p} = \frac{1}{4\,(1+\tau)} \, p \, \ddf{\tau}{p} \]
Ainsi la stabilité de la couche s'écrit
\[ \frac{R}{c_p} \ge \frac{1}{4\,(1+\tau)} \, p \, \ddf{\tau}{p} \]
\noindent et dans le cas d'un coefficient d'absorption~$\kappa$ constant, nous pouvons même écrire la condition de stabilité
\[ \frac{R}{c_p} \ge \frac{\tau}{4\,(1+\tau)} \]
%% si l'absorption est constante p \, \ddf{\tau}{p} = - \kappa \, p / g costheta

\sk
Le terme en~$p$ dans ce qui précède guarantit (à moins d'une variation énorme de $\ddf{\tau}{p}$ en~$1/p$ ou plus rapide quand~$p \rightarrow 0$) que les hautes atmosphères planétaires sont toujours stables. De plus, les atmosphères optiquement fines sont toujours stables sur l'intégralité de leur épaisseur, puisque~$-p \, \ddf{\tau}{p} < \tau_\infty \ll 1$. Le critère de stabilité est en pratique un peu plus complexe qu'indiqué dans les atmosphères réelles. Bien sûr, $\tau$ et~$\kappa$ varient avec la longueur d'onde~$\lambda$ (limitation inhérente au modèle à deux faisceaux), mais surtout le coefficient d'absorption~$\kappa$ augmente avec la pression (donc la profondeur) en raison de l'élargissement collisionnel (\emph{collisional broadening}), efficace à partir de quelques bars. La loi de variation d'élargissement collisionnel peut s'écrire~$\kappa(p) = \kappa(p\e{s}) \, \frac{p}{p\e{s}}$. Les processus de changements d'état sont également susceptibles de rendre la situation plus complexe que le calcul proposé ici.


%\newpage
%\section{Effet de serre : déplacement d'équilibre}
%


%\figside{0.45}{0.3}{/home/aymeric/Big_Data/BOOKS/pierrehumbert_pics/9780521865562c03_fig005.jpg}{R. Pierrehumbert, Principles of Planetary Climates, CUP, 2010}{fig:effetserre1}

\sk
Nous présentons ici l'explication la plus simple (sans être simpliste) du déplacement
d'équilibre radiatif qu'induit l'augmentation de gaz à effet de serre.
%Le modèle à deux faisceaux ne nous aide pas énormément. diverge quand tau tend vers infini.

\sk
Il est possible de montrer par des calculs de transfert radiatif que le niveau
d'émission équivalent au sommet de l'atmosphère est tel que~$\tau = 1$.
Qualitativement, on comprend que les niveaux inférieurs sont optiquement
épais donc ne sont que marginalement ``vus'' depuis l'espace dans les longueurs d'onde infrarouges.
Ainsi dans l'équilibre TOA
\[ \TOA \] 
\noindent l'émission de rayonnement au niveau~$\tau=1$ 
à la température~$T(\tau=1)$ domine OLR.

\sk
Appelons~$P\e{rad}$ la pression du niveau~$\tau=1$. 
Pour relier
les deux quantités, on emploie la définition de l'épaisseur optique
$\EO$ que l'on combine
à l'équilibre hydrostatique pour obtenir
par intégration~$\tau = \kappa \frac{P}{g} \, q_X$,
avec $q_X$ le rapport de mélange massique 
de l'espèce~$X$ absorbante dans l'infrarouge.
Ainsi
\[ P\e{rad} = \frac{g}{\kappa \, q_X} \]

\figun{0.7}{0.25}{/home/aymeric/Big_Data/BOOKS/pierrehumbert_pics/9780521865562c03_fig006.jpg}{R. Pierrehumbert, Principles of Planetary Climates, CUP, 2010}{fig:effetserre2}

\sk
L'expression ci-dessus implique qu'une augmentation de
gaz à effet de serre ($q_X$ augmente) implique une 
élévation du niveau équivalent d'émission
($p\e{rad}$ diminue).
L'effet sur la température de surface se détermine alors
en écrivant la conservation de la température potentielle
dans la troposphère soumise à l'équilibre radiatif-convectif,
entre la surface et le niveau équivalent d'émission
\[ T_s = T\e{rad} \, \left( \frac{P\e{rad}}{P\e{s}} \right)^{-\frac{R}{c_p}} \]
\noindent où~$P_s$ est la pression de surface.
Une élévation du niveau équivalent d'émission
se traduit donc par une augmentation
de température (fournissant
un modèle à la fois simple et fidèle du 
changement climatique récent sur Terre, Figure~\ref{fig:effetserre2}). 
Approximativement, $OLR \sim \sigma \, T(P\e{rad})^4$
et~$P\e{rad}$ est alors défini par la 
condition TOA qui s'écrit~$OLR = (1-A\e{b}) \, \mathcal{F}\e{s}'$.
Le lien entre quantité de gaz à effet de serre~$q_X$
et température de surface~$T\e{s}$ peut ainsi s'écrire
\[ T\e{s} = \sqrt[4]{\frac{(1-A\e{b}) \, \mathcal{F}\e{s}'}{\sigma}} \, \left( \frac{\kappa \, q_X \,P\e{s}}{g} \right)^{\frac{R}{c_p}} \]

%
%\newpage
%\section{Effet de serre divergent}
%
%% http://www.skepticalscience.com/print.php?r=262

\sk
\paragraph{Approche rapide} Une augmentation de la température de surface~$T\e{s}$ 
est donc associée à une augmentation de la quantité de gaz à effet de serre~$q_X$.
Sur une planète pourvue d'océan,
une augmentation de la température de surface
provoque une augmentation de l'évaporation
donc de la quantité de vapeur d'eau dans l'atmosphère, 
par conséquent de l'effet de serre.
Il s'agit d'une \voc{rétroaction positive}~:
le système amplifie la perturbation initiale de température de surface.
La quantité~$q_X$ de vapeur d'eau dans l'atmosphère
peut donc virtuellement augmenter indéfiniment.
Néanmoins, la pression du niveau équivalent~$P\e{rad} = \frac{g}{\kappa \, q_X}$
ne peut diminuer indéfiniment, du moins continûment~:
lorsque le sommet de la couche radiative de l'atmosphère
est atteint $P\e{rad} \ll P\e{s}$ , la radiation sortante~OLR
atteint une valeur maximale~$OLR\e{max}$.

\sk
\paragraph{Approche plus subtile} On peut inverser le point de vue et se demander quel est la valeur
d'OLR qui correspond à une température de surface~$T\e{s}$.
D'après le modèle simplifié combinant la hauteur équivalente
d'émission et le profil radiatif-convectif dans la troposphère, nous avons
\[ OLR  = \textcolor{magenta}{\sigma \, T\e{s}^4} \textcolor{blue}{\left( \frac{g}{\kappa \, q_X \, P\e{s}} \right)^{\frac{4 \, R}{c_p}}} \]
Plaçons-nous toujours dans le cas d'une planète pourvue
d'océan en évaporation.
Pour les températures de surface relativement modérées,
les variations de quantité de vapeur d'eau~$q_X$
(et de pression de surface~$P\e{s}$)
sont modérées et les variations d'OLR suivent 
une loi en~$\sigma T\e{s}^4$ (terme en \textcolor{magenta}{magenta}).
Néanmoins, plus la température de surface~$T\e{s}$
augmente, plus la vapeur d'eau devient dominante
dans l'atmosphère en influençant~$q_X$, mais
surtout~$P\e{s}$ via la loi d'équilibre liquide-vapeur
de Clausius-Clapeyron
$  P\e{s}(T\e{s}) = P_0 \, \exp{ \left[ -\frac{\ell}{R\,T\e{s}} \right] } $
\noindent où~$\ell > 0$ est la chaleur latente de vaporisation.
Si $T\e{s}$ augmente, $P\e{s}(T\e{s})$ augmente, et de manière exponentielle.
Le terme en \textcolor{blue}{bleu}, qui décroît exponentiellement avec
la température~$T\e{s}$, influence de façon dominante
l'expression de l'OLR pour les température élevées.
L'effet combiné des deux termes 
(\textcolor{magenta}{magenta} et \textcolor{blue}{bleu})
impose donc que l'OLR atteint une valeur maximale~$OLR\e{max}$,
que l'on appelle limite de Komabayashi-Ingersoll
(du nom de deux auteurs d'articles indépendants parus à la fin des années 60).

%% EM: En revanche, l'asymptote n'est qu'approximativement horizontale, elle est légèrement décroissante en présence d'un gaz à effet de serre non condensable (typiquement CO2). Du coup, il vaut mieux distinguer le maximum (KI limit) et l'asymptote plus basse (limite de Nakajima, voir Fig. 3 de https://journals.ametsoc.org/doi/pdf/10.1175/1520-0469%281992%29049%3C2256%3AASOTGE%3E2.0.CO%3B2) 

\figside{0.4}{0.15}{/home/aymeric/Big_Data/BOOKS/pierrehumbert_pics/9780521865562c04_fig004.jpg}{R. Pierrehumbert, Principles of Planetary Climates, CUP, 2010}{fig:ki}

\sk
\paragraph{Effet de serre divergent} 
Quelle que soit l'approche adoptée pour définir~$OLR\e{max}$,
il existe cette limite lorsque toute l'atmosphère devient optiquement épaisse.
Si l'on se place dans un contexte de variation
(à l'échelle des temps géologiques) du flux incident
solaire~$(1-A\e{b}) \, \mathcal{F}\e{s}'$,
avec notamment une augmentation au cours du temps
étant donné l'activité radioactive du Soleil\footnote{En fait, le flux incident varie lorsque la pression atmosphérique devient conséquente à cause d'un effet de diffusion Rayleigh accru},
on constate que les valeurs du flux 
incident~$(1-A\e{b}) \, \mathcal{F}\e{s}'$ peuvent
dépasser~$OLR\e{max}$, ce qui signifie que
l'équilibre TOA ne peut être satisfait et que
l'atmosphère reçoit plus d'énergie qu'elle
n'en émet. La température de surface peut augmenter
de manière incontrôlée, au risque d'atteindre des valeurs
très élevées (plusieurs centaines de K, voire quelques milliers).
L'évaporation des océans peut alors
survenir de manière abrupte et rapide (Figure~\ref{fig:ki}),
dans ce que l'on appelle l'\voc{effet de serre divergent}
(\emph{runaway greenhouse}). De fait, 
l'équilibre radiatif type TOA peut n'être
récupéré que pour des températures de surface 
très élevées (valeurs de plus de~$1000-2000$~K,
pour lesquelles l'intégralité des océans a disparu
selon toute vraisemblance).
Au-delà de telles valeurs de température de surface~$T\e{s}$,
le flux sortant OLR se remet à augmenter avec~$T\e{s}$
en raison de la contribution grandissante de l'émission
thermique dans le visible (et de la moindre absorption
de la vapeur d'eau dans ces longueurs d'onde).


%%% KI : dépend de g


%% Fs + when Ts + because increased absorption of solar rad by water vapor
%% then - when Ts + because Rayleigh scattering



%%%%%%%%%%%%%%%%%%
%\newpage
%\section{Rétroactions}
%
Les processus de rétroactions climatiques peuvent amplifier (on parle alors de \voc{rétroaction positive}, en anglais \emph{positive feedback}) ou réduire (\voc{rétroaction négative}) la réponse à une perturbation initiale et sont donc centraux pour simuler correctement l’évolution du climat.

\begin{finger}
\item La rétroaction de Stefan-Boltzmann : Si la température augmente alors la perte par rayonnement augmente : feedback négatif très fort
\item La rétroaction de la glace et de l’albédo : Si la température augmente alors la glace diminue et donc le rayonnement solaire absorbé augmente ce qui augmente la température : feedback positif
\item Rétroaction des gaz à effet de serre au cours d’un cycle glaciaire-interglaciaire: une entrée en glaciation entraîne une baisse de la teneur en gaz à effet de serre (CO2, H2O vapeur et CH4) dans l'atmosphère par suite des modifications du climat (refroidissement de la surface terrestre et modification de la circulation océanique profonde); cette diminution atténue l'effet de serre initial et donc amplifie le refroidissement en cours. Inversement une déglaciation entraîne une augmentation des mêmes gaz à effet de serre, ce qui, cette fois, contribue à accentuer le réchauffement. Feedback positif
%\item La rétroaction des nuages: Si la température augmente et induit plus de nuages qui réfléchissent plus d’énergie solaire alors la température diminue. Cependant par effet de serre des nuages, la température augmente. Au contraire, si le climat se refroidit, la couverture neigeuse hivernale persistera plus longtemps. Or cette couverture blanche (d’albédo plus élevé que le sol) augmente la réflexion de l'énergie solaire et donc diminue le chauffage de la surface par le Soleil. Il en résulte un refroidissement de la surface qui amplifie le refroidissement climatique initial -- feedback positif si refroidissement, inconnu si réchauffement (a priori négatif pour les nuages bas).
\end{finger}

%\sk \subsection{Rétroactions} Préciser les rétroactions en jeu Indiquer le sens de ces rétroactions. Exemple terrestre~: \begin{itemize} \item Stefan-Boltzman : positive ou négative \item Glaces : positive ou négative \item Vapeur d'eau : positive ou négative \item Nuages : positive ou négative \end{itemize} 
%\visible<2->{\vskip 0.5cm\ebloc{}{Applications~:~changement climatique, paléo-climats, évolution des planètes du système solaire, climat des exoplanètes}}} \note{Peut être en général étendu aux autres planètes.\\ Les processus de rétroactions climatiques peuvent amplifier (on parle alors de rétroactions positives) ou réduire (rétroaction négative) la réponse à une perturbation initiale\\ SB : Si la température augmente alors la perte par rayonnement augmente : feedback négatif très fort\\ glace (albedo) : Si la température augmente alors la glace diminue et donc le rayonnement solaire absorbé augmente ce qui augmente la température feedback positif, mais attention au contrôle par la taille de la calotte polaire.\\ vapeur d’eau : l’augmentation de la température tend à favoriser l'évaporation car l'équilibre L-V est déplacé, de fait augmentation de l'humidité ie le contenu en vapeur d’eau de l’atmosphère, ce qui augmente l’effet de serre et donc la température de surface\\ nuages :  Si la température augmente et induit plus de nuages qui réfléchissent plus d’énergie solaire alors la température diminue. Cependant par effet de serre des nuages, la température augmente…. Au contraire, si le climat se refroidit, la couverture neigeuse hivernale persistera plus longtemps. Or cette couverture blanche (d’albédo plus élevé que le sol) augmente la réflexion de l'énergie solaire et donc diminue le chauffage de la surface par le Soleil. Il en résulte un refroidissement de la surface qui amplifie le refroidissement climatique initial feedback positif si refroidissement, inconnu si réchauffement (a priori négatif pour les nuages bas)} \note{[FACULTATIF] GES cycle glaciaire vs. interg.~: une entrée en glaciation entraîne une baisse de la teneur en gaz à effet de serre (CO2, H2O vapeur et CH4) dans l'atmosphère par suite des modifications du climat (refroidissement de la surface terrestre et modification de la circulation océanique profonde); cette diminution atténue l'effet de serre initial et donc amplifie le refroidissement en cours. Inversement une déglaciation entraîne une augmentation des mêmes gaz à effet de serre, ce qui, cette fois, contribue à accentuer le réchauffement feedbak positif}

%%%
%\newpage
%\section{Conduction}
%\sk
Les transferts thermiques par conduction se font par diffusion thermique. L'énergie est transférée via les collisions entre molécules. Ce type de transfert est dominant dans les intérieurs planétaires et dans les hautes atmosphères (thermosphère). Dans le dernier cas, le libre parcours moyen est si long que les atomes / molécules peuvent se mouvoir très rapidement d'une localisation à une autre, résultant en une conduction très efficace et un profil en général proche de l'isotherme.

\sk
Par analogie avec la diffusion moléculaire, pour définir la diffusion thermique, il s'agit de définir une loi phénoménologique (loi de Fourier, analogue de la loi de Fick) et une équation de conservation dans un volume de contrôle (conservation de l'énergie, analogue de la conservation de la matière). On définit ainsi pour le cas de la diffusion thermique uni-dimensionnelle selon~$x$
\begin{citemize}
\item \textit{Cause} inhomogénéité spatiale : Différence de température~$T(x,t)$
\item \textit{Conséquence} Densité de courant de chaleur~$\vec{J_Q}$ (W~m$^{-2}$)
\item \textit{\'Echange} Chaleur~$\delta Q = J_Q \, S \, \dd t$
\item \textit{Loi phénoménologique} Loi de Fourier~$J_Q = - \lambda\e{T} \, \Dp{T}{x}$
\item \textit{Conductivité thermique} en W~m$^{-1}$~K$^{-1}$~: $\lambda\e{T,roche} = 1-2$, $\lambda\e{T,eau} = 0.5$, $\lambda\e{T,air} = 0.02$.
\item \textit{Equation bilan locale} Conservation de l'énergie interne~$ \rho \, c_p \, \Dp{T}{t} + \Dp{J_Q}{x} = 0 $
\end{citemize}

\sk
Les équations tridimensionnelles sont
\[  
\textrm{Loi de Fourier} \quad \vec{J_Q} = - \lambda\e{T} \, \nabla T 
\qquad \qquad
\textrm{Conservation de l'énergie} \quad \rho \, c_p \, \Dp{T}{t} + \nabla \cdot \vec{J_Q} = 0
\]
\noindent L'équation de conservation de l'énergie n'est rien d'autre que le premier principe appliqué à un volume de contrôle~: la variation temporelle d'énergie interne est égale à la divergence du flux de chaleur (flux sortant moins flux entrant).

\sk
Combiner loi phénoménologique et équation de conservation permet d'obtenir ce qui est communément appelé l'équation de la chaleur, ou plus précisément l'équation de diffusion thermique
\[ \Dp{T}{t} = - D\e{T} \, \nabla^2 T \quad \textrm{[3D]} \qquad \qquad \Dp{T}{t} = - D\e{T} \, \DDp{T}{x} \quad \textrm{[1D]} \]
\noindent où~$D\e{T}$ est la diffusivité thermique notée
\[ D\e{T} = \frac{\lambda\e{T}}{\rho \, c_p} \]

\sk
Dans le cas unidimensionnel de la diffusion thermique dans un sol uniforme à la profondeur~$z$, en supposant un forçage périodique~$T(0,t) = T_0 + T_0' \, \cos \omega t$ ($\omega$ étant adapté au cas considéré selon si forçage diurne, saisonnier, \ldots), l'équation de diffusion thermique permet d'obtenir l'expression des variations spatiales et temporelles de~$T$
\[ T(x,t) = T_0 + T_0' \, e^{-\frac{z}{\delta}} \, \cos (\omega t - \frac{z}{\delta}) \]
\noindent où l'atténuation avec la profondeur est~$e^{-\frac{z}{\delta}}$ et le déphasage du maximum du forçage est~$\Delta t = \frac{z}{\omega \delta}$, avec~$\delta$ l'épaisseur de peau qui s'exprime
\[ \delta = \frac{2\,D\e{T}}{\omega} \]
\noindent En pratique, l'atténuation et le déphasage sont très marquées pour des profondeurs dans le sol même très modérées.







%%%%%%%%%%%%%%%%%%


\newpage
\section{Système et référentiel}
\sk
La position d'un point $M$ de l'atmosphère sera représentée dans un systèmes de coordonnées sphériques (figure~\ref{fig:repere}) par sa latitude $\varphi$, sa longitude $\lambda$, et son altitude~$z$ par rapport au niveau de la mer. Pour les déplacements horizontaux, on utilise le repère direct
$\left(M,\mathbf{i},\mathbf{j},\mathbf{k}\right)$ où $\mathbf{i}$ et $\mathbf{j}$ sont les vecteurs unitaires vers l'est et le nord, et $\mathbf{k}$ est dirigé suivant la verticale vers le haut. La direction définie par~$\mathbf{i}$ est souvent qualifiée de \voc{zonale}, celle définie par~$\mathbf{j}$ de \voc{méridienne}. 
%Pour des déplacements qui ne sont pas d'échelle planétaire, on utilisera également des distances horizontales vers l'est et le nord~$dx=a\, d\lambda\, \cos \varphi$ et~$dy=a\,d\varphi$ où~$a$ est le rayon de la Terre.
%
%\sk
On distingue deux référentiels pour l'étude des mouvements de l'air:
\begin{finger}
\item Un \voc{référentiel tournant} lié à la Terre, en rotation autour de l'axe des pôles avec la vitesse angulaire $\Omega$. La \voc{vitesse relative} est mesurée dans le référentiel tournant, par rapport à la surface de la Terre et a pour composantes~$u,v,w$ suivant \v i,\v j,\v k. Il s'agit de ce que l'on appelle communément le \voc{vent} avec le point de vue d'humain attaché à la surface de la Terre, c'est-à-dire au référentiel tournant. La composante horizontale du vecteur vitesse relative est donc~$\mathbf{V} = u \, \mathbf{i} + v \, \mathbf{j}$ et la composante verticale~$w \, \mathbf{k}$.
\item Un \voc{référentiel fixe} orienté suivant les directions de trois étoiles. La \voc{vitesse absolue} d'un point M est considérée dans le référentiel fixe et inclut donc le mouvement circulaire autour de l'axe des pôles. Ce référentiel peut être considéré comme galiléen. Il correspond à ce qu'on observerait depuis l'espace, lorsqu'on voit la Terre tourner au lieu d'être \ofg{attaché} à sa rotation.
\end{finger}

%\figun{0.4}{0.25}{\figfrancis/repere}{Schéma du système de coordonnées et du repère utilisés.}{fig:repere}
\figside{0.45}{0.22}{\figfrancis/repere}{Système de coordonnées et repère utilisés.}{fig:repere}




\newpage
\section{Changement de référentiel}
\sk
L'équation de base pour le mouvement de masses d'air est la relation fondamentale de la dynamique $\Sigma \v F=m \, \v a$ (seconde loi de Newton).  Cette relation est cependant valable dans un référentiel galiléen, tel le référentiel fixe. On s'intéresse plutôt au vent, c'est-à-dire que l'on souhaite considérer des mouvements atmosphériques par rapport à la surface de la Terre qui est en rotation autour de l'axe des pôles. On va donc dans un premier temps projeter l'accélération dans le référentiel tournant, puis étudier les principales forces horizontales. Autrement dit, on se donne pour objectif d'exprimer l'accélération dans le référentiel tournant, qu'on souhaite connaître, en fonction de l'accélération dans le référentiel fixe, qui est égale à la somme des forces.

\sk
La relation entre vitesse absolue~$\v V_a$ dans le référentiel fixe et vitesse relative~$\v V_r$ dans le référentiel tournant s'écrit, avec le vecteur de rotation~$\v \Omega$ de module~$\Omega$ dirigé selon l'axe des pôles~:
\[\v V_a = \v V_r + \vl{\Omega}\wedge\vl{CM}\]
Il s'agit de la relation de composition des vitesses pour un référentiel tournant. Le terme $\vl{\Omega}\wedge\vl{CM}$ est la vitesse d'un point fixe par rapport au sol ($\v V_r=0$), il est appelé \voc{vitesse d'entrainement}.
%La relation entre la dérivée temporelle d'un vecteur \v X dans le référentiel fixe (\emph{absolue}, $a$) et celle dans le référentiel tournant (\emph{relative}, $r$) s'écrit \[\frac{d\v X}{dt}_{|a}=\frac{d\v X}{dt}_{|r}+ \vl \Omega\wedge \v X\] En applicant au vecteur \vl{CM}, avec $\frac{d\vl{CM}}{dt}=\v V$, on a: \[\v V_a=\v V_r+\vl{\Omega}\wedge\vl{CM}\]
La relation entre accélération absolue~$\v a_a$, égale à la somme des forces, et accélération relative~$\v V_r$ dans le référentiel tournant s'écrit
\[ \v a_a=\Sigma\v F=\v a_r+2\vl{\Omega}\wedge\v V_r-\Omega^2\,\vl{HM} \]
Le premier terme est l'accélération relative~$\v a_r$, le deuxième l'\voc{accélération de Coriolis}~$\v a_c$, le troisième est l'\voc{accélération d'entrainement}~$\v a_e$. Les termes de Coriolis et d'entraînement induisent des \voc{forces apparentes}~$\v F_c = -m \, \v a_c$ et~$\v F_e = -m \, \v a_e$ dans le référentiel tournant. On parle de forces apparentes car du point de vue du référentiel fixe, ces termes n'apparaissent pas comme des forces~: ils ne sont que des termes d'accélération causés par le caractère non galiléen du référentiel tournant.
%En dérivant à nouveau $\v V_a$, on obtient: \[\v a_a=\left(\frac{d \v V_r}{dt}_{|r}+\vl{\Omega}\wedge\v V_r\right)+\vl{\Omega}\wedge \left(\v V_r+\vl{\Omega}\wedge\vl{CM}\right)\] soit en regroupant et avec $\vl{\Omega}\wedge(\vl{\Omega}\wedge\vl{CM})=-\Omega^2\cdot\vl{HM}$:




\newpage
\section{Accélération d'entraînement et pesanteur}
\sk
On considère un point M immobile par rapport à la surface de la Terre. Les forces (massiques) subies par M sont la force de gravitation \v G, dirigée vers le centre de la Terre, et \v R la réaction du sol dirigée perpendiculairement à la surface (figure \ref{fig:centrif}). Dans le référentiel fixe, l'accélération de M est celle du mouvement circulaire uniforme: $\v a_e = - \Omega^2 \, \vl{HM}$ (accélération d'entrainement). On doit donc avoir \[\v a_e=\v G+\v R\] 
C'est impossible si la Terre est sphérique (sauf au pôle et à l'équateur): on aurait alors \v R et \v G colinéaires mais pas dans la direction de $\v a_e$. La Terre a en fait pris une forme aplatie, où la surface n'est pas perpendiculaire à~$\v G$. En posant $\v g=\v G-\v a_e$, l'équilibre devient: \[\v g+\v R=\v 0\] On a donc une gravité apparente \v g dirigée localement vers le bas (perpendiculairement à la surface) mais pas exactement vers le centre de la Terre. La gravité réelle \v G a elle une faible composante horizontale. Dans ce qui suit, on considère que l'accélération d'entraînement est inclus dans le terme~$\v g$.

%\figside{0.55}{0.25}{\figfrancis/centrif}{Equilibre d'un point posé au sol. La forme réelle de la Terre est en trait continu, la sphère en pointillés.}{fig:centrif}
\figside{0.45}{0.2}{\figfrancis/centrif}{Equilibre d'un point posé au sol. La forme réelle de la Terre est en trait continu, la sphère en pointillés.}{fig:centrif}




\newpage
\section{Accélération de Coriolis et déviation du mouvement}
\sk
L'accélération de Coriolis peut être interprétée comme une force apparente massique $\v F_C = - 2 \, \v \Omega\wedge\v V_r$. Cette force apparente étant orthogonale à la vitesse à cause de la présence du produit vectoriel, sa puissance est nulle~: la \voc{force de Coriolis} va dévier le mouvement relatif mais ne peut pas modifier la vitesse du vent ou de courants. Pour des mouvements relatifs horizontaux à la vitesse \v V, le module de la force apparente de Coriolis est~$2 \, \Omega \, \sin \phi \, V$ qui change de signe lorsqu'on change d'hémisphère en fonction de~$\sin \phi$. Dans l'hémisphère nord, où $\sin \phi>0$, la force de Coriolis est dirigée à $90^{\circ}$ à droite du vent. 

\sk
Afin de bien comprendre l'effet de la force de Coriolis, il est profitable sur une planète comme la Terre d'utiliser la conservation du moment cinétique\footnote{
Puisque le moment cinétique~$\sigma$ se conserve on a \[ \ddf{\sigma}{t} = 0 = \ddf{r}{t} \, (\Omega \, r + u) + r \, \left( \Omega \ddf{r}{t}+\ddf{u}{t} \right) \qquad \Rightarrow \qquad \ddf{u}{t} = - \ddf{r}{t}  \, \left( 2\,\Omega + \frac{u}{r} \right) \] 
Le terme en $u/r$ est dû à la courbure de la surface, mais seule la vitesse relative intervient, pas la rotation de la Terre. En pratique, ce terme est négligeable sur Terre devant~$2 \, \Omega$. L'équation ci-dessus montre donc que raisonner avec la conservation du moment cinétique permet de comprendre l'effet sur les vents de la force de Coriolis.
}
(l'équivalent pour les systèmes en rotation de la quantité de mouvement pour les systèmes en translation). En effet, la somme des forces étant dirigée vers H, M conserve son \voc{moment cinétique}~$\sigma$ par rapport à l'axe des pôles, qui s'exprime
\[ \boxed{ \sigma = u_a \, r = (\Omega \, r + u) \, r } \]
où~$r$ est la distance entre le point considéré et l'axe de rotation qui passe par les deux pôles.
%\footnote{La conservation de $\sigma$ implique des variations de l'énergie cinétique $(\Omega r+u)^2$. C'est le travail de \v G (pour un mouvement sud-nord) qui en est l'origine.}. 

\sk
Pour illustrer les effets de cette force apparente de Coriolis, on considère une parcelle initialement au repos dans le référentiel tournant (c'est à dire~$u=0$ et~$v=0$ à~$t=0$) qui se déplacerait vers le Nord suivant l'axe~$\v j$. Elle se rapproche donc de l'axe des pôles et va voir sa vitesse absolue augmenter par conservation du moment cinétique: $\sigma$ est constant et~$r$ diminue, donc $u_a$ augmente. Dans le même temps, la vitesse d'entrainement locale~$u_e=\Omega \, r$ diminue sous l'effet de la diminution de la distance~$r$ à l'axe des pôles. La parcelle va donc acquérir une vitesse relative $u>0$ vers l'est\footnote{
En fait, l'expression ci-dessus permet même de calculer la variation de vitesse associée. Pour un mouvement sud-nord, la vitesse est $v=a \, \ddf{\phi}{t}$. D'autre part $r=a \, \cos \phi$ donc~$\ddf{r}{t}=-a \, \ddf{\phi}{t} \, \sin \phi = - v \, \sin \phi$. L'équation de conservation du moment cinétique devient 
\[ \ddf{u}{t} = v \, \sin \phi \, \left( 2 \, \Omega + \frac{u}{r} \right) \simeq 2 \, \Omega \, v \, \sin \phi \] 
La parcelle est bien déviée vers l'est pour un déplacement vers le nord tel que~$v>0$.
}
comme indiqué sur le schéma \ref{fig:coriolisns}. 

\figside{0.3}{0.2}{\figfrancis/coriolis_ns}{Déviation d'une parcelle se déplaçant vers le nord. Instant initial: vitesses d'entrainement $u_e$ et absolue $u_a$ égales. Instant final: vitesse d'entrainement $u_e'$ et absolue $u_a'$ augmentée par conservation du moment cinétique $\sigma$.}{fig:coriolisns}

%\subsubsection{Force de Coriolis: mouvement vers l'est} On considère un point M en mouvement par rapport à la surface de la Terre. On rappelle que pour un mouvement circulaire, on doit avoir une accélération normale égale à $V^2/R$ dirigée vers le centre du cercle. On suppose que les forces réelles s'exerçant sur M sont les mêmes que pour un point fixe: $\Sigma \vec F=\v a_e$. La composante de la vitesse relative vers l'est (suivant \v i) est $u$, et $\dot{r}$ dans la direction \vl{HM}. La vitesse absolue de M vers l'est est $u_a=\Omega r+u$. La relation $\v a=\Sigma\v F$ s'écrit dans la direction $\v e_r$: \[-\frac{(\Omega r+u)^2}{r}+\ddot{r}=a_e=-\Omega^2r\] soit en développant: \[\ddot{r}=u\cdot(2\Omega+\frac{u}{r})\] Pour un mouvement relatif vers l'est ($u>0$), la vitesse absolue est supérieure à la vitesse d'entrainement, et la somme des forces est insuffisante pour compenser $V_a^2/r$. La parcelle va donc s'éloigner de l'axe de rotation (figure \ref{fig:coriolisew}). Elle va au contraire se rapprocher pour $u<0$ (mouvement vers l'ouest). Pour trouver l'accélération relative dans la direction sud-nord, on projette $\v e_r$ sur \v j: $\dot{v}=-\ddot{r}\sin \phi$. \[\dot{v}=-u\sin \phi\cdot(2\Omega+\frac{u}{r})\] M est donc dévié vers le sud pour un déplacement relatif vers l'est.
%\begin{figure}[tbp] \begin{center} \includegraphics[width=12cm]{\figfrancis/coriolis_ew} \end{center} \caption{Déviation d'une parcelle ayant une vitesse relative initiale non nulle vers l'est (gauche) et l'ouest (droite). Un plan parallèle à l'équateur est représenté, vu depuis le pôle nord, l'axe de rotation est au centre. Les vitesse et accélération d'entrainement (égale à la somme des forces) sont en noir, la vitesse absolue en rouge. La trajectoire future de la parcelle est en pointillés.} \label{fig:coriolisew} \end{figure}


\newpage
\section{Cellules de Hadley et courants-jets}
\sk
La structure du vent zonal est dominée aux moyennes latitudes par la présence de deux \voc{jets}, c'est-à-dire de puissants courants atmosphériques, dits \voc{jets d'ouest} car ils soufflent de l'ouest vers l'est. Leur vitesse augmente sur la verticale entre la surface et un maximum au niveau de la tropopause, autour de 50~m~s$^{-1}$. Ce comportement peut être justifié en combinant l'équilibre géostrophique à l'équilibre hydrostatique (équation du vent thermique). Dans les tropiques, les vents moyens sont d'est, surtout dominants dans la basse troposphère, mais restent néanmoins moins forts que les vents d'ouest dans les moyennes latitudes. On les appelle les \voc{alizés}.

\sk
La circulation dans le plan méridien (sud-nord et verticale) est caractérisée par une série de cellules fermées. Le chauffage différentiel explique qu'une différence de pression naisse entre les tropiques et les moyennes latitudes, car la pression diminue plus vite avec l'altitude dans les couches d'air froid des moyennes latitudes que dans les couches d'air chaud des tropiques. Ceci donne naissance en altitude à des vents de l'équateur vers les pôles. Ces vents induisent un flux de masse atmosphérique vers les moyennes latitudes, donc, d'après l'équivalence entre pression et masse, une augmentation de la pression de surface aux moyennes latitudes par rapport aux tropiques. Ceci donne naissance proche de la surface à des vents des pôles vers l'équateur. Par continuité, dans les tropiques, l'air s'élève proche de l'équateur (suivant la saison, du côté de l'hémisphère d'été) et redescend au niveau des subtropiques. Ce système est appelé \voc{cellules de Hadley}. On observe également dans les moyennes latitudes des cellules moins intenses, contrôlées par les instabilités dans l'atmosphère, appelées cellules de Ferrel. 
%%(figure \ref{fig:MMC})

\sk
La structure en latitude des vents %décrite par la figure~\ref{fig:UTlatP}, 
avec des vents d'ouest aux moyennes latitudes et d'est sous les tropiques, est très liée à la circulation de Hadley.
% décrite par la figure~\ref{fig:MMC}.
Sous l'action de la force de Coriolis, les mouvements vers les pôles sont déviés vers l'est et les mouvements vers l'équateur sont déviés vers l'ouest. Les jets d'ouest des moyennes latitudes proviennent ainsi de la déviation vers l'est de la circulation vers les pôles dans la branche supérieure de la cellule de Hadley. Les vents d'est (alizés) sous les tropiques proviennent quant à eux de la déviation vers l'ouest de la circulation vers l'équateur dans la branche inférieure de la cellule de Hadley. Les vents de grande échelle comportent donc une composante vers l'équateur et l'ouest sous les tropiques, alors qu'aux moyennes latitudes, ils comportent une composante vers les pôles et l'est [la composante vers l'est domine cependant]. Une exception à cette image est observée dans les régions de ``mousson'' (sous-continent Indien, et dans une moindre mesure Afrique de l'ouest et Amérique centrale) où la direction du vent s'inverse entre l'été (vers le continent) et l'hiver (vers l'océan).

\figside{0.6}{0.3}{\figfrancis/WH_circ_scheme}{Schéma de la circulation atmosphérique: zone de convergence et alizés dans les tropiques; gradient de pression tropiques (H) -pôle (L), vents d'ouest et ondes aux moyennes latitudes. La position des jets d'ouest et l'extension des cellules de Hadley sont représentées à droite. Figure adaptée de Wallace and Hobbs, Atmospheric Science, 2006.}{fig:circscheme}


\newpage
\section{\'Equation fondamentale}
\sk
L'équation complète de la quantité de mouvement pour les mouvements atmosphériques, qui résulte de l'application de la seconde loi de Newton, s'écrit~:
%\begin{equation}
%\ddf{\v V_r}{t} + 2 \, \v \Omega\wedge\v V_r = \v g + \v F_P + \vl{Fr} 
\[   
\ddf{\v V_r}{t} = \v g + \v F_P + \v F_C + \vl{Fr}
\] %\frac{1}{\rho}\vl{grad}P  \]
%  \label{eq:qtemvt}
%\end{equation}
Le terme~$\vl{Fr}$ représente les forces de friction qui sont négligées sauf lorsqu'on se trouve à proximité de la surface.

\begin{detail}
%\newpage
\section{Forces de pression}
\sk
Les forces de pression horizontales se calculent comme la force de pression verticale dans la démonstration de l'équilibre hydrostatique. La force de pression s'exerçant sur une surface $S$ est normale à cette surface et vaut $P \, S$. Pour une parcelle d'air de volume $\delta x \, \delta y \, \delta z$ (figure \ref{fig:pres}), la force de pression totale dans la direction ($Ox$) vaut
\[ F_P^* = P(x) \, \delta y \, \delta z - P(x+\delta x) \, \delta y \, \delta z = - \frac{\partial P}{\partial x} \, \delta x \, \delta y \, \delta z \]
La force de pression {\em massique} est donc
\[F_P = \frac{F_P^*}{\rho \delta x \delta y \delta z}=-\frac{1}{\rho}\frac{\partial P}{\partial x}\]
On peut faire le même calcul sur ($Oy$). Finalement les deux composantes horizontales de la force de pression s'écrivent
\[\v F_P^H = -\frac{1}{\rho} \, \binom{\frac{\partial P}{\partial x}}{\frac{\partial P}{\partial y}}\] %  =-\frac{1}{\rho}\vl{grad}P\]

\sk
La force de pression est donc opposée aux variations horizontales de pression données par les dérivées partielles, ce qui lui confère des propriétés importantes.
\begin{citemize}
\item La force de pression est dirigée des hautes vers les basses pressions, perpendiculairement aux isobares.
\item La force de pression est inversement proportionelle à l'écartement des isobares.
\end{citemize}
Une région où la pression est particulièrement basse est appelée \voc{dépression}. Une région où la pression est particulièrement élevée est appelée \voc{anticyclone}.

\figside{0.6}{0.2}{\figfrancis/pressure}{Forces de pression (suivant ($Ox$)) s'exerçant sur une parcelle.}{fig:pres}

%\subsubsection{Équivalence avec le géopotentiel}
%L'équilibre hydrostatique fait que la pression décroit toujours avec
%l'altitude. Une pression localement élevée doit donc correspondre à une
%altitude élevée des surfaces isobares.
%\begin{figure}[htp]
%  \begin{center}
%    \includegraphics[width=\figwn]{\figfrancis/pres_geop}
%  \end{center}
%  \caption{Équivalence entre écarts de pression et d'altitude: les points A et
%  B sont à la même altitude, A et C à la même pression. La pression en B est
%  donc supérieure à celle en B.}
%  \label{fig:pres_geop}
%\end{figure}
%Sur la figure \ref{fig:pres_geop}, la force de pression horizonale dans la
%direction ($Ox$) est
%$F_P=-\frac{1}{\rho}\frac{P_B-P_A}{\delta x}$. Or $A$ et $C$ sont à la même
%pression, on a donc
%\[F_P=-\frac{1}{\rho}\frac{P_B-P_C}{\delta x}=-\frac{1}{\rho}\frac{P_B-P_C}{\delta z}\cdot\frac{\delta z}{\delta x}\]
%En utilisant
%\[\frac{P_B-P_C}{\delta z}=-\frac{\partial P}{\partial z}=\rho g\]
%on trouve 
%\[F_P=-g\left(\frac{\delta z}{\delta x}\right)_P\]
%On aurait une relation équivalente pour la direction ($Oy$), la
%force de pression horizontale vaut donc finalement
%\[\v F_P=-\frac{1}{\rho}\vl{grad}_Z(P)=-g\cdot\vl{grad}_P(Z)\]
%On utilise plutôt le gradient de pression horizontal avec la pression au
%niveau de la mer, et le gradient isobare de l'altitude $Z$ ou du
%{\em géopotentiel} $gZ$ dans l'atmosphère libre.
%Sur une carte d'une surface isobare, les lignes à $Z$ constant sont des
%{\em isohypses}. La force de pression est donc dirigée des hautes vers les
%basses valeurs de $Z$, perpendiculairement aux isohypses.

\sk
Les variations verticales de la pression sont données par l'équilibre hydrostatique comme indiqué dans les chapitres précédents. Cette propriété a deux conséquences importantes pour les variations de pression horizontales donc la force de pression horizontale. 
\begin{finger}
\item Une conséquence de cet équilibre est que la pression à une altitude $z$ est proportionelle à la masse de la colonne d'air située au dessus de $z$. Une diminution ou augmentation de cette masse dûe aux mouvements d'air horizontaux change donc la pression en dessous, en particulier à la surface.
\item D'autre part, même pour une masse d'air totale de la colonne constante, des écarts de température horizontaux peuvent créer des gradients de pression en changeant la répartition verticale de cette masse. L'équation hypsométrique donne l'épaisseur d'une colonne d'air de masse constante entre deux niveaux de pression donnés (voir chapitres précédents)~: la pression décroît plus vite dans une couche d'air froid que dans une couche d'air chaud. Une variation horizontale de température induit donc une force de pression horizontale selon ce principe.
\end{finger}

%\begin{equation}
%  g\cdot(Z_2-Z_1)=R<T>\ln{\frac{P_1}{P_2}}
%  \label{eq:hypso}
%\end{equation}
%La différence entre les forces de pressions aux niveaux 1 et 2 sera donc: \[\v F_{P_2}-\v F_{P_1}=-R\cdot\vl{grad}<T>\cdot\ln{\frac{P_1}{P_2}}\]



\end{detail}

\end{document}
