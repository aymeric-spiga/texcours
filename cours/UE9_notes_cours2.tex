\documentclass[a4paper,DIV16,10pt]{scrartcl}
%%%%%%%%%%%%%%%%%%%%%%%%%%%%%%%%%%%%%%%%%%%%%%%%%%%%%%%%%%%%%%%%%%%%%%%%%%%%%%%%%%%
\usepackage{texcours}
%%%%%%%%%%%%%%%%%%%%%%%%%%%%%%%%%%%%%%%%%%%%%%%%%%%%%%%%%%%%%%%%%%%%%%%%%%%%%%%%%%%
\newcommand{\zauthor}{Aymeric SPIGA}
\newcommand{\zaffil}{Laboratoire de Météorologie Dynamique}
\newcommand{\zemail}{aymeric.spiga@upmc.fr}
\newcommand{\zcourse}{Physique des atmosphères planétaires}
\newcommand{\zcode}{UE9}
\newcommand{\zuniversity}{UPMC}
\newcommand{\zlevel}{M2 Planétologie}
\newcommand{\zsubtitle}{Fiches complémentaires de cours}
\newcommand{\zlogo}{\includegraphics[height=1.5cm]{decouverte/cours_meteo/UPMC_cart-blanc-Q_7504-703-3.png}}
\newcommand{\zrights}{Copie et usage interdits sans autorisation explicite de l'auteur}
\newcommand{\zdate}{\today}
%%%%%%%%%%%%%%%%%%%%%%%%%%%%%%%%%%%%%%%%%%%%%%%%%%%%%%%%%%%%%%%%%%%%%%%%%%%%%%%%%%%
\begin{document} \inidoc
%%%%%%%%%%%%%%%%%%%%%%%%%%%%%%%%%%%%%%%%%%%%%%%%%%%%%%%%%%%%%%%%%%%%%%%%%%%%%%%%%%%

\newpage
\section{Profil radiatif convectif (telluriques)}
\sk
Les conditions atmosphériques sont très instables proche d'une surface (en présence d'une telle surface). A cause de la discontinuité entre surface et atmosphère, sous l'action de la diffusion thermique, ou turbulente, entre la surface (chaude) et l'air immédiatement adjacent (plus froid) crée une couche d'air fine approximativement à la température de la surface ; les conditions de température étant plus froides au-dessus, les conditions atmosphériques sont très instables proche de la surface et des mouvements de convection vont se mettre en place pour mélanger l'air sur une certaine épaisseur atmosphérique. Un équilibre dit \voc{radiatif-convectif} prévaut, avec une structure thermique suivant le profil adiabatique, donnant naissance à une troposphère. Au-dessus de la limite radiative-convective (correspondant peu ou prou à la tropopause), les phénomènes radiatifs dominent et donnent naissance à une mésosphère -- ou une stratosphère si un absorbant visible y est présent en quantité suffisante, donnant naissance à une inversion stable à la tropopause.
%% on passait en troposphère dès que le gradient du profil radiatif dépassait celui du profil adiabatique (-g/cp)

\figun{0.4}{0.25}{/home/aymeric/Big_Data/BOOKS/pierrehumbert_pics/9780521865562c03_fig014.jpg}{Figure tirée de R. Pierrehumbert, Principles of Planetary Climates, CUP, 2010}{fig:effetserre2}











\newpage
\section{Profil radiatif convectif (géantes)}
\sk
La présence d'une surface et une hypothèse d'équilibre radiatif imposent donc que le chauffage de la surface conduit inévitablement à de la convection. Que se passe-t-il sur les planètes géantes dépourvues de surface ? L'équilibre radiatif y prévaut également, car à partir d'une certaine profondeur, le gradient radiatif est instable -- et ce, même en l'absence d'une surface qui absorbe le rayonnement solaire. Pour formuler la stabilité de l'équilibre radiatif, on calcule le profil~$\dd T / \dd p$ et on le compare au gradient adiabatique sec ou humide dans l'atmosphère considérée. En dérivant le profil radiatif obtenu dans le cas du modèle à deux faisceaux
\[ T(\tau) = \sqrt[4]{\frac{OLR\,(1+\tau)}{2\,\sigma\,\epsilon}} \]
\noindent par rapport à l'épaisseur optique~$\tau$, nous obtenons
\[ 8 \, \sigma \, T^3 \, \ddf{T}{\tau} = OLR \]
\noindent soit, en utilisant~$\ddf{~}{p} = \ddf{\tau}{p} \ddf{~}{\tau}$
\[ \ddf{T}{\ln p} = \frac{1}{4\,(1+\tau)} \, p \, \ddf{\tau}{p} \]
Ainsi la stabilité de la couche s'écrit
\[ \frac{R}{c_p} \ge \frac{1}{4\,(1+\tau)} \, p \, \ddf{\tau}{p} \]
\noindent et dans le cas d'un coefficient d'absorption~$\kappa$ constant, nous pouvons même écrire la condition de stabilité
\[ \frac{R}{c_p} \ge \frac{\tau}{4\,(1+\tau)} \]
%% si l'absorption est constante p \, \ddf{\tau}{p} = - \kappa \, p / g costheta

\sk
Le terme en~$p$ dans ce qui précède guarantit (à moins d'une variation énorme de $\ddf{\tau}{p}$ en~$1/p$ ou plus rapide quand~$p \rightarrow 0$) que les hautes atmosphères planétaires sont toujours stables. De plus, les atmosphères optiquement fines sont toujours stables sur l'intégralité de leur épaisseur, puisque~$-p \, \ddf{\tau}{p} < \tau_\infty \ll 1$. Le critère de stabilité est en pratique un peu plus complexe qu'indiqué dans les atmosphères réelles. Bien sûr, $\tau$ et~$\kappa$ varient avec la longueur d'onde~$\lambda$ (limitation inhérente au modèle à deux faisceaux), mais surtout le coefficient d'absorption~$\kappa$ augmente avec la pression (donc la profondeur) en raison de l'élargissement collisionnel (\emph{collisional broadening}), efficace à partir de quelques bars. La loi de variation d'élargissement collisionnel peut s'écrire~$\kappa(p) = \kappa(p\e{s}) \, \frac{p}{p\e{s}}$. Les processus de changements d'état sont également susceptibles de rendre la situation plus complexe que le calcul proposé ici.


\newpage
\section{Effet de serre : déplacement d'équilibre}



%\figside{0.45}{0.3}{/home/aymeric/Big_Data/BOOKS/pierrehumbert_pics/9780521865562c03_fig005.jpg}{R. Pierrehumbert, Principles of Planetary Climates, CUP, 2010}{fig:effetserre1}

\sk
Nous présentons ici l'explication la plus simple (sans être simpliste) du déplacement
d'équilibre radiatif qu'induit l'augmentation de gaz à effet de serre.
%Le modèle à deux faisceaux ne nous aide pas énormément. diverge quand tau tend vers infini.

\sk
Il est possible de montrer par des calculs de transfert radiatif que le niveau
d'émission équivalent au sommet de l'atmosphère est tel que~$\tau = 1$.
Qualitativement, on comprend que les niveaux inférieurs sont optiquement
épais donc ne sont que marginalement ``vus'' depuis l'espace dans les longueurs d'onde infrarouges.
Ainsi dans l'équilibre TOA
\[ \TOA \] 
\noindent l'émission de rayonnement au niveau~$\tau=1$ 
à la température~$T(\tau=1)$ domine OLR.

\sk
Appelons~$P\e{rad}$ la pression du niveau~$\tau=1$. 
Pour relier
les deux quantités, on emploie la définition de l'épaisseur optique
$\EO$ que l'on combine
à l'équilibre hydrostatique pour obtenir
par intégration~$\tau = \kappa \frac{P}{g} \, q_X$,
avec $q_X$ le rapport de mélange massique 
de l'espèce~$X$ absorbante dans l'infrarouge.
Ainsi
\[ P\e{rad} = \frac{g}{\kappa \, q_X} \]

\figun{0.7}{0.25}{/home/aymeric/Big_Data/BOOKS/pierrehumbert_pics/9780521865562c03_fig006.jpg}{R. Pierrehumbert, Principles of Planetary Climates, CUP, 2010}{fig:effetserre2}

\sk
L'expression ci-dessus implique qu'une augmentation de
gaz à effet de serre ($q_X$ augmente) implique une 
élévation du niveau équivalent d'émission
($p\e{rad}$ diminue).
L'effet sur la température de surface se détermine alors
en écrivant la conservation de la température potentielle
dans la troposphère soumise à l'équilibre radiatif-convectif,
entre la surface et le niveau équivalent d'émission
\[ T_s = T\e{rad} \, \left( \frac{P\e{rad}}{P\e{s}} \right)^{-\frac{R}{c_p}} \]
\noindent où~$P_s$ est la pression de surface.
Une élévation du niveau équivalent d'émission
se traduit donc par une augmentation
de température (fournissant
un modèle à la fois simple et fidèle du 
changement climatique récent sur Terre, Figure~\ref{fig:effetserre2}). 
Approximativement, $OLR \sim \sigma \, T(P\e{rad})^4$
et~$P\e{rad}$ est alors défini par la 
condition TOA qui s'écrit~$OLR = (1-A\e{b}) \, \mathcal{F}\e{s}'$.
Le lien entre quantité de gaz à effet de serre~$q_X$
et température de surface~$T\e{s}$ peut ainsi s'écrire
\[ T\e{s} = \sqrt[4]{\frac{(1-A\e{b}) \, \mathcal{F}\e{s}'}{\sigma}} \, \left( \frac{\kappa \, q_X \,P\e{s}}{g} \right)^{\frac{R}{c_p}} \]


\newpage
\section{Effet de serre divergent}

%% http://www.skepticalscience.com/print.php?r=262

\sk
\paragraph{Approche rapide} Une augmentation de la température de surface~$T\e{s}$ 
est donc associée à une augmentation de la quantité de gaz à effet de serre~$q_X$.
Sur une planète pourvue d'océan,
une augmentation de la température de surface
provoque une augmentation de l'évaporation
donc de la quantité de vapeur d'eau dans l'atmosphère, 
par conséquent de l'effet de serre.
Il s'agit d'une \voc{rétroaction positive}~:
le système amplifie la perturbation initiale de température de surface.
La quantité~$q_X$ de vapeur d'eau dans l'atmosphère
peut donc virtuellement augmenter indéfiniment.
Néanmoins, la pression du niveau équivalent~$P\e{rad} = \frac{g}{\kappa \, q_X}$
ne peut diminuer indéfiniment, du moins continûment~:
lorsque le sommet de la couche radiative de l'atmosphère
est atteint $P\e{rad} \ll P\e{s}$ , la radiation sortante~OLR
atteint une valeur maximale~$OLR\e{max}$.

\sk
\paragraph{Approche plus subtile} On peut inverser le point de vue et se demander quel est la valeur
d'OLR qui correspond à une température de surface~$T\e{s}$.
D'après le modèle simplifié combinant la hauteur équivalente
d'émission et le profil radiatif-convectif dans la troposphère, nous avons
\[ OLR  = \textcolor{magenta}{\sigma \, T\e{s}^4} \textcolor{blue}{\left( \frac{g}{\kappa \, q_X \, P\e{s}} \right)^{\frac{4 \, R}{c_p}}} \]
Plaçons-nous toujours dans le cas d'une planète pourvue
d'océan en évaporation.
Pour les températures de surface relativement modérées,
les variations de quantité de vapeur d'eau~$q_X$
(et de pression de surface~$P\e{s}$)
sont modérées et les variations d'OLR suivent 
une loi en~$\sigma T\e{s}^4$ (terme en \textcolor{magenta}{magenta}).
Néanmoins, plus la température de surface~$T\e{s}$
augmente, plus la vapeur d'eau devient dominante
dans l'atmosphère en influençant~$q_X$, mais
surtout~$P\e{s}$ via la loi d'équilibre liquide-vapeur
de Clausius-Clapeyron
$  P\e{s}(T\e{s}) = P_0 \, \exp{ \left[ -\frac{\ell}{R\,T\e{s}} \right] } $
\noindent où~$\ell > 0$ est la chaleur latente de vaporisation.
Si $T\e{s}$ augmente, $P\e{s}(T\e{s})$ augmente, et de manière exponentielle.
Le terme en \textcolor{blue}{bleu}, qui décroît exponentiellement avec
la température~$T\e{s}$, influence de façon dominante
l'expression de l'OLR pour les température élevées.
L'effet combiné des deux termes 
(\textcolor{magenta}{magenta} et \textcolor{blue}{bleu})
impose donc que l'OLR atteint une valeur maximale~$OLR\e{max}$,
que l'on appelle limite de Komabayashi-Ingersoll
(du nom de deux auteurs d'articles indépendants parus à la fin des années 60).

\figside{0.4}{0.15}{/home/aymeric/Big_Data/BOOKS/pierrehumbert_pics/9780521865562c04_fig004.jpg}{R. Pierrehumbert, Principles of Planetary Climates, CUP, 2010}{fig:ki}

\sk
\paragraph{Effet de serre divergent} 
Quelle que soit l'approche adoptée pour définir~$OLR\e{max}$,
il existe cette limite lorsque toute l'atmosphère devient optiquement épaisse.
Si l'on se place dans un contexte de variation
(à l'échelle des temps géologiques) du flux incident
solaire~$(1-A\e{b}) \, \mathcal{F}\e{s}'$,
avec notamment une augmentation au cours du temps
étant donné l'activité radioactive du Soleil\footnote{En fait, le flux incident varie lorsque la pression atmosphérique devient conséquente à cause d'un effet de diffusion Rayleigh accru},
on constate que les valeurs du flux 
incident~$(1-A\e{b}) \, \mathcal{F}\e{s}'$ peuvent
dépasser~$OLR\e{max}$, ce qui signifie que
l'équilibre TOA ne peut être satisfait et que
l'atmosphère reçoit plus d'énergie qu'elle
n'en émet. La température de surface peut augmenter
de manière incontrôlée, au risque d'atteindre des valeurs
très élevées (plusieurs centaines de K, voire quelques milliers).
L'évaporation des océans peut alors
survenir de manière abrupte et rapide (Figure~\ref{fig:ki}),
dans ce que l'on appelle l'\voc{effet de serre divergent}
(\emph{runaway greenhouse}). De fait, 
l'équilibre radiatif type TOA peut n'être
récupéré que pour des températures de surface 
très élevées (valeurs de plus de~$1000-2000$~K,
pour lesquelles l'intégralité des océans a disparu
selon toute vraisemblance).
Au-delà de telles valeurs de température de surface~$T\e{s}$,
le flux sortant OLR se remet à augmenter avec~$T\e{s}$
en raison de la contribution grandissante de l'émission
thermique dans le visible (et de la moindre absorption
de la vapeur d'eau dans ces longueurs d'onde).


%%% KI : dépend de g


%% Fs + when Ts + because increased absorption of solar rad by water vapor
%% then - when Ts + because Rayleigh scattering



\newpage
\section{Transformations pseudo-adiabatiques}
\sk
On considère tout d'abord une parcelle d'air (contenant de la vapeur d'eau) en évolution isobare. Le premier principe appliqué à la parcelle indique donc
\[ \dd T = \frac{1}{c_p} \, \delta q \]
Lors de l'évaporation, les molécules d'eau liquide voient les liaisons hydrogène avec leurs proches voisins être brisées. Le passage de l'eau de la phase liquide à la phase vapeur consomme donc de l'énergie\footnote{On peut s'en convaincre en notant la sensation de froid immédiate que provoque la sortie d'un bain à cause de l'évaporation de l'eau liquide sur le corps mouillé~; ou en se souvenant que lorsque l'on souffle sur la soupe pour la refroidir, c'est précisément pour favoriser l'évaporation et la refroidir efficacement.}~: pour l'air qui compose la parcelle, $\delta q < 0$ et il y a refroidissement. 
A l'inverse, lors de la condensation, les molécules d'eau sous forme gazeuse créent des liaisons hydrogène avec les molécules d'eau de la phase liquide pour atteindre un état énergétique plus faible. Le passage de l'eau de la phase vapeur à la phase liquide libère donc de l'énergie~: pour l'air qui compose la parcelle, $\delta q > 0$ et il y a chauffage.

\sk
L'énergie~$\delta q$ consommée ou libérée par les changements d'état s'appelle~\voc{chaleur latente}, on la note~$\delta q\e{latent}$. Si une masse de vapeur~$\dd m\e{vapeur d'eau}$ est condensée ou évaporée, on a
\[ \delta q\e{latent} = \frac{- L \, \dd m\e{vapeur d'eau}}{m\e{air sec}} \qquad \Rightarrow \qquad \boxed{ \delta q\e{latent} = - L \, \dd r } \]
où~$L$ est la chaleur latente massique en~J~kg$^{-1}$. La formule ci-dessus comporte un signe négatif. La quantité~$\delta q\e{latent}$ est positive lorsqu'il y a condensation (le rapport de mélange en vapeur d'eau diminue $\dd r < 0$) et négative lorsqu'il y a évaporation (le rapport de mélange en vapeur d'eau augmente $\dd r > 0$).

\sk
On considère désormais une parcelle d'air en évolution adiabatique, à l'exception des échanges de chaleur latente~: $\delta q = \delta q\e{latent}$. On appelle une telle transformation \voc{pseudo-adiabatique} ou encore \voc{adiabatique saturée}. On fait l'approximation que la chaleur latente consommée ou dégagée est seulement échangée avec l'air sec~:
\begin{citemize}
\item La chaleur latente consommée/dégagée n'est pas utilisée pour refroidir/chauffer les gouttes d'eau présentes.
\item On néglige les pertes de masse par précipitation~: la masse d'air sec considérée est constante.
\end{citemize}
Pour une telle transformation, la variation de température s'écrit ainsi
\[ \dd T = \frac{R}{c_p} \, \frac{T}{P} \, \dd P - \frac{L}{c_p} \, \dd r \]
\noindent ou encore
\[ c_p \, \dd T + g \, \dd z + L \, \dd r = 0 \]


\newpage
\section{Gradient adiabatique humide}
\sk
Considérons une parcelle en ascension adiabatique saturée (et non plus sèche comme dans la section~\ref{adiabsec}). Pour une parcelle saturée, c'est-à-dire à l'équilibre liquide/vapeur, l'équation qui précède peut s'écrire, en utilisant l'équilibre hydrostatique
\[ C_P \, \dd T + g \, \dd z + L \, \dd r = 0 \]
Or, puisque la parcelle est saturée, on a~$r = r\e{sat}(T)$ et on peut écrire $\dd r\e{sat} = \ddf{r\e{sat}}{T} \, \dd T$. On a alors
\[ \left( C_P + L \, \ddf{r\e{sat}}{T} \right) \dd T + g \, \dd z = 0\]
Cette expression est similaire au cas sec, à l'exception notable du terme supplémentaire~$L \, \ddf{r\e{sat}}{T}$ lié aux échanges latents. On peut alors obtenir le profil vertical adopté dans l'atmosphère saturée par une parcelle ne subissant pas d'échange de chaleur avec l'extérieur autre que les échanges de chaleur latente
\[  \ddf{T}{z}  = \Gamma\e{saturé} \qquad \text{avec} \qquad \Gamma\e{saturé} = \frac{-g}{C_P+L \, \ddf{r\e{sat}}{T} } \]
On a vu que $\ddf{r\e{sat}}{T}$ est toujours positif, on en déduit donc
\[ \boxed{ \Gamma\e{saturé} > \Gamma\e{sec} \qquad \text{ou} \qquad |\Gamma\e{saturé}| < |\Gamma\e{sec}| } \]
A cause du dégagement de chaleur latente, la température diminue moins vite pour une parcelle saturée en ascension que pour une parcelle non saturée. Le calcul pour l'atmosphère terrestre montre que
\[ \Gamma\e{saturé} = -6.5 \, \text{K~km}^{-1} \] 
ce qui correspond à la valeur observée dans la troposphère sur Terre. %[Figure~\ref{fig:tempvert}].

\sk
La constatation que~$\Gamma\e{saturé}$ correspond au profil d'environnement effectivement mesuré dans la troposphère appelle un commentaire important. Les profils verticaux secs ou saturés sont ceux suivis par une parcelle en ascension~: autrement dit, ils donnent les variations de~$T\e{p}$ avec l'altitude~$z$. D'un point de vue instantané, ils ne correspondent pas aux profils d'environnement~$T\e{e}$ tels qu'ils peuvent être par exemple mesurés par des ballons-sonde lâchés dans l'atmosphère. La parcelle n'est pas nécessairement à l'équilibre thermique avec l'environnement. On peut néanmoins constater sur la figure~\ref{fig:tempvert} que la température de l'environnement diminue avec une pente très proche de~$\Gamma\e{saturé}$. Ceci s'explique par le fait que cette figure montre une moyenne sur tout le globe à toutes les saisons. La situation moyenne ainsi décrite correspond aux mouvements d'une multitude de parcelles en ascension qui finissent par définir l'environnement atmosphérique\footnote{Ce phénomène porte le nom d'ajustement convectif.}. Pour comprendre la formation des nuages, et plus généralement les mouvements atmosphériques, il faut néanmoins se placer dans le cas local où l'équilibre thermique n'est pas vérifié. C'est l'objet de la section suivante.
%Comme pour le cas adiabatique, on peut aussi intégrer l'équation pour obtenir:
%\begin{equation} e_h=C_PT+gz+Lr=cste \label{estath} \end{equation}  
%La quantité $e_h$ est appelée {\em énergie statique humide} et est conservée
%pour des mouvements adiabatiques ($r$ et $e_s$ sont séparément conservés) ou
%saturés (pseudo-adiabatiques).


\newpage
\section{Structures thermiques}


Profil radiatif-convectif + profil radiatif + effet de la convection humide

%\figun{0.2}{0.1}{decouverte/pierrehumbert_pics/9780521865562c02_fig001.jpg}{R. Pierrehumbert, Principles of Planetary Climates, CUP, 2010}{fig:profearth}

\figun{0.65}{0.5}{decouverte/pierrehumbert_pics/9780521865562c02_fig002.jpg}{R. Pierrehumbert, Principles of Planetary Climates, CUP, 2010}{fig:profplanet}



%\newpage
%\figun{0.99}{0.6}{decouverte/cours_meteo/sounding.pdf}{Radiosondage obtenu à la station de Trappes le 23 juin 2005 à 12:00. Les données sont projetées sur un émagramme avec la pression atmosphérique en ordonnée et la température atmosphérique en abscisse. Des cumulonimbus d'orages très violents se sont développés dans la région parisienne ce jour~: cet événement peut être interprété à l'aide de l'émagramme qui permet de déterminer la base du nuage, le niveau de convection libre et le sommet théorique du nuage.}{fig:sounding}

%\newpage
%\section{Formation d'un cumulonimbus sur Terre}
%\sk
On s'intéresse ici au développement des nuages cumuliformes, en particulier les cumulonimbus. Le cas d'étude donné dans le radiosonsage exemple permet de suivre graphiquement les concepts de cette partie. Le point de départ est une parcelle d'air non saturée, c'est-à-dire dont l'humidité est inférieure à~$1$, située proche de la surface. On suppose que son rapport de mélange en vapeur d'eau~$r$ est conservé au cours de l'ascension. 

\sk
En premier lieu, de nombreux phénomènes atmosphériques vont provoquer une élévation de la parcelle que l'on considère initialement proche de la surface.
\begin{citemize}
\item{\textbf{Soulèvement frontal}} Un front est une variation marquée et localisée de température. Lorsqu'un front se déplace horizontalement, l'air chaud passe au-dessus de l'air froid de densité moindre. 
\item{\textbf{Soulèvement orographique}} La présence d'un relief face au vent force les parcelles d'air à s'élever.
\item{\textbf{Convection sèche}} Un sol très chaud l'après-midi peut induire un profil de température de l'environnement très instable proche de la surface. Dans ce cas, les mouvements verticaux sont amplifiés proche de la surface par la poussée d'Archimède (voir chapitre précédent).
%%% circulation thermique. brise de terre et brise de mer.
\end{citemize}
Tous ces mécanismes expliquent que des nuages cumuliformes sont souvent trouvés au-dessus de régions soumises au passage de fronts, montagneuses ou dont la surface est particulièrement chaude. Ces nuages évoluent parfois vers un état de type cumulonimbus.

\sk
En second lieu, lorsqu'une parcelle d'air est soulevée vers les plus hauts niveaux de l'atmosphère par les phénomènes atmosphériques précités, elle subit un refroidissement par détente adiabatique. Le taux de refroidissement de la parcelle est~$|\Gamma\e{sec}|$. Sur un émagramme tel celui de la figure~\ref{fig:sounding}, la parcelle suit une \voc{courbe adiabatique sèche}. Cette décroissance de la température de la parcelle au cours de l'ascension a pour principale conséquence d'abaisser la valeur de $r\e{sat}$, de par les variations exponentielles de cette quantité avec la température. Il en résulte que l'humidité relative~$H = r / r\e{sat}$ augmente. Si la quantité de vapeur d'eau initiale~$r$ et/ou le soulèvement de la parcelle sont suffisants, $H$ peut atteindre~$1$ au cours de l'ascension~: la parcelle devient alors saturée. Des gouttelettes nuageuses apparaissent par condensation, autrement dit un nuage se forme. Le niveau d'altitude ou de pression auquel la condensation se produit suite à un refroidissement par soulèvement adiabatique s'appelle le \voc{niveau de condensation} ou la \voc{base du nuage}. A ce stade, le nuage n'est pas encore nécessairement cumuliforme.
%%Pour la convection. Les bases des nuages sont horizontales, leurs sommets évoluent en fonction de la température.

\sk
En troisième lieu, si la parcelle continue son ascension au-delà du niveau de condensation, sa température ne décroît plus d'un taux~$|\Gamma\e{sec}|$, mais d'un taux $|\Gamma\e{saturé}|$ plus faible, puisque la parcelle est désormais saturée (son humidité vaut~$1$ et son rapport de mélange~$r$ vaut~$r\e{sat}$). Sur l'émagramme, la parcelle suit une \voc{courbe adiabatique saturée}. Au cours de l'ascension, la parcelle reste saturée mais, puisque sa température continue de diminuer, $r\e{sat}$ diminue de concours, ce qui induit une diminution du rapport de mélange en vapeur d'eau~$r$ et une augmentation du rapport de mélange en eau liquide (qui prend la forme de gouttelettes nuageuses ou, si les conditions de croissance sont réunies, de précipitations pluvieuses).

\sk
En quatrième lieu, la forme du profil de température d'environnement détermine si, une fois le niveau de condensation atteint, le nuage va suivre ou non un développement vertical marqué. On rappelle que le profil d'environnement n'est pas celui suivi par la parcelle considérée, mais représente l'état atmosphérique tel qu'il peut être mesuré par un ballon-sonde météorologique par exemple.

\begin{finger}

\item Si les soulèvements initiaux de la parcelle ne l'amènent que dans des niveaux atmosphériques où sa température reste plus faible que celle de l'environnement, alors il n'y a pas de mouvements verticaux spontanés au sein du nuage. Le nuage formé est plutôt de type stratiforme (ou faiblement cumuliforme).

\item Si les soulèvements initiaux de la parcelle parviennent à la hisser à des niveaux atmosphériques où sa température devient plus élevée que celle de l'environnement, alors des mouvements verticaux spontanés apparaissent au sein du nuage. On parle de convection humide (ou convection profonde). Le nuage ainsi formé est cumuliforme. Le niveau atmosphérique à partir duquel la température de la parcelle en ascension adiabatique devient plus élevée que la température de l'environnement s'appelle le \voc{niveau de convection libre}. Le niveau atmosphérique à partir duquel la température de la parcelle redevient plus faible que l'environnement s'appelle le \voc{sommet théorique du nuage}. Si le sommet théorique du nuage est très élevé, la formation de cumulonimbus, donc d'orage, est très probable. 

\end{finger}



%
%\newpage
%\section{Formation d'un cumulonimbus sur Terre (compléments)}
%\figside{0.4}{0.2}{decouverte/cours_meteo/anvil.png}{Vue lointaine d'un cumulonimbus à un stade avancé de développement, où l'on peut observer la structure aplatie en forme d'enclume au sommet du nuage. Source~: Wallace and Hobbs, Atmospheric Science, 2006~; d'après une photographie du Bureau Australien de Météorologie.}{fig:enclume}

\sk
L'étude de la formation des cumulonimbus appelle deux remarques importantes qui illustrent les concepts de stabilité et instabilité atmosphérique.

\begin{finger}

\item Le sommet des cumulonimbus atteint très fréquemment la tropopause. Lorsque c'est le cas, ils prennent alors une apparence aplatie et la forme d'enclume comme présenté dans la figure~\ref{fig:enclume}. Cela provient du fait que la stratosphère voit la température de l'environnement augmenter avec l'altitude, contrairement à ce qui peut se passer dans la troposphère. Un tel profil de température est extrêmement stable, donc a tendance à inhiber les mouvements verticaux. Ainsi, le fort développement vertical des cumulonimbus est stoppé net lorsque les couches stables de la stratosphère sont atteintes. En conséquence, le nuage s'étale selon l'horizontale au voisinage de la tropopause. C'est la raison pour laquelle le sommet réel des nuages cumuliformes est le minimum du sommet théorique des nuages et de la hauteur de la tropopause. Plus généralement, des conditions stables peuvent conduire à la formation de nuages stratiformes, ce qui nuance un peu la distinction faite dans la section~\ref{classphys}.

\item On entrevoit par les développement précédents qu'il est possible qu'une couche atmosphérique donnée, dont la température suit le taux de décroissance~$|\Gamma\e{env}|$ selon l'altitude, apparaisse comme stable si l'on considère l'ascension d'une parcelle non saturée, mais instable si l'on considère l'ascension d'une parcelle saturée. Cette situation se présente lorsque 
\[ |\Gamma\e{saturée}| <  |\Gamma\e{env}|  < |\Gamma\e{sec}| \]
On parle alors d'\voc{instabilité conditionnelle}. Il s'agit de conditions où seule l'apparition d'un nuage peut donner lieu à une instabilité et au développement de mouvements verticaux potentiellement étendus. Dans ce cas de figure, les nuages qui se forment sont essentiellement cumuliformes.

\end{finger}



%
%\newpage
%\section{Rétroactions}
%
Les processus de rétroactions climatiques peuvent amplifier (on parle alors de \voc{rétroaction positive}, en anglais \emph{positive feedback}) ou réduire (\voc{rétroaction négative}) la réponse à une perturbation initiale et sont donc centraux pour simuler correctement l’évolution du climat.

\begin{finger}
\item La rétroaction de Stefan-Boltzmann : Si la température augmente alors la perte par rayonnement augmente : feedback négatif très fort
\item La rétroaction de la glace et de l’albédo : Si la température augmente alors la glace diminue et donc le rayonnement solaire absorbé augmente ce qui augmente la température : feedback positif
\item Rétroaction des gaz à effet de serre au cours d’un cycle glaciaire-interglaciaire: une entrée en glaciation entraîne une baisse de la teneur en gaz à effet de serre (CO2, H2O vapeur et CH4) dans l'atmosphère par suite des modifications du climat (refroidissement de la surface terrestre et modification de la circulation océanique profonde); cette diminution atténue l'effet de serre initial et donc amplifie le refroidissement en cours. Inversement une déglaciation entraîne une augmentation des mêmes gaz à effet de serre, ce qui, cette fois, contribue à accentuer le réchauffement. Feedback positif
%\item La rétroaction des nuages: Si la température augmente et induit plus de nuages qui réfléchissent plus d’énergie solaire alors la température diminue. Cependant par effet de serre des nuages, la température augmente. Au contraire, si le climat se refroidit, la couverture neigeuse hivernale persistera plus longtemps. Or cette couverture blanche (d’albédo plus élevé que le sol) augmente la réflexion de l'énergie solaire et donc diminue le chauffage de la surface par le Soleil. Il en résulte un refroidissement de la surface qui amplifie le refroidissement climatique initial -- feedback positif si refroidissement, inconnu si réchauffement (a priori négatif pour les nuages bas).
\end{finger}

%\sk \subsection{Rétroactions} Préciser les rétroactions en jeu Indiquer le sens de ces rétroactions. Exemple terrestre~: \begin{itemize} \item Stefan-Boltzman : positive ou négative \item Glaces : positive ou négative \item Vapeur d'eau : positive ou négative \item Nuages : positive ou négative \end{itemize} 
%\visible<2->{\vskip 0.5cm\ebloc{}{Applications~:~changement climatique, paléo-climats, évolution des planètes du système solaire, climat des exoplanètes}}} \note{Peut être en général étendu aux autres planètes.\\ Les processus de rétroactions climatiques peuvent amplifier (on parle alors de rétroactions positives) ou réduire (rétroaction négative) la réponse à une perturbation initiale\\ SB : Si la température augmente alors la perte par rayonnement augmente : feedback négatif très fort\\ glace (albedo) : Si la température augmente alors la glace diminue et donc le rayonnement solaire absorbé augmente ce qui augmente la température feedback positif, mais attention au contrôle par la taille de la calotte polaire.\\ vapeur d’eau : l’augmentation de la température tend à favoriser l'évaporation car l'équilibre L-V est déplacé, de fait augmentation de l'humidité ie le contenu en vapeur d’eau de l’atmosphère, ce qui augmente l’effet de serre et donc la température de surface\\ nuages :  Si la température augmente et induit plus de nuages qui réfléchissent plus d’énergie solaire alors la température diminue. Cependant par effet de serre des nuages, la température augmente…. Au contraire, si le climat se refroidit, la couverture neigeuse hivernale persistera plus longtemps. Or cette couverture blanche (d’albédo plus élevé que le sol) augmente la réflexion de l'énergie solaire et donc diminue le chauffage de la surface par le Soleil. Il en résulte un refroidissement de la surface qui amplifie le refroidissement climatique initial feedback positif si refroidissement, inconnu si réchauffement (a priori négatif pour les nuages bas)} \note{[FACULTATIF] GES cycle glaciaire vs. interg.~: une entrée en glaciation entraîne une baisse de la teneur en gaz à effet de serre (CO2, H2O vapeur et CH4) dans l'atmosphère par suite des modifications du climat (refroidissement de la surface terrestre et modification de la circulation océanique profonde); cette diminution atténue l'effet de serre initial et donc amplifie le refroidissement en cours. Inversement une déglaciation entraîne une augmentation des mêmes gaz à effet de serre, ce qui, cette fois, contribue à accentuer le réchauffement feedbak positif}


\newpage
\section{Conduction}
\sk
Les transferts thermiques par conduction se font par diffusion thermique. L'énergie est transférée via les collisions entre molécules. Ce type de transfert est dominant dans les intérieurs planétaires et dans les hautes atmosphères (thermosphère). Dans le dernier cas, le libre parcours moyen est si long que les atomes / molécules peuvent se mouvoir très rapidement d'une localisation à une autre, résultant en une conduction très efficace et un profil en général proche de l'isotherme.

\sk
Par analogie avec la diffusion moléculaire, pour définir la diffusion thermique, il s'agit de définir une loi phénoménologique (loi de Fourier, analogue de la loi de Fick) et une équation de conservation dans un volume de contrôle (conservation de l'énergie, analogue de la conservation de la matière). On définit ainsi pour le cas de la diffusion thermique uni-dimensionnelle selon~$x$
\begin{citemize}
\item \textit{Cause} inhomogénéité spatiale : Différence de température~$T(x,t)$
\item \textit{Conséquence} Densité de courant de chaleur~$\vec{J_Q}$ (W~m$^{-2}$)
\item \textit{\'Echange} Chaleur~$\delta Q = J_Q \, S \, \dd t$
\item \textit{Loi phénoménologique} Loi de Fourier~$J_Q = - \lambda\e{T} \, \Dp{T}{x}$
\item \textit{Conductivité thermique} en W~m$^{-1}$~K$^{-1}$~: $\lambda\e{T,roche} = 1-2$, $\lambda\e{T,eau} = 0.5$, $\lambda\e{T,air} = 0.02$.
\item \textit{Equation bilan locale} Conservation de l'énergie interne~$ \rho \, c_p \, \Dp{T}{t} + \Dp{J_Q}{x} = 0 $
\end{citemize}

\sk
Les équations tridimensionnelles sont
\[  
\textrm{Loi de Fourier} \quad \vec{J_Q} = - \lambda\e{T} \, \nabla T 
\qquad \qquad
\textrm{Conservation de l'énergie} \quad \rho \, c_p \, \Dp{T}{t} + \nabla \cdot \vec{J_Q} = 0
\]
\noindent L'équation de conservation de l'énergie n'est rien d'autre que le premier principe appliqué à un volume de contrôle~: la variation temporelle d'énergie interne est égale à la divergence du flux de chaleur (flux sortant moins flux entrant).

\sk
Combiner loi phénoménologique et équation de conservation permet d'obtenir ce qui est communément appelé l'équation de la chaleur, ou plus précisément l'équation de diffusion thermique
\[ \Dp{T}{t} = - D\e{T} \, \nabla^2 T \quad \textrm{[3D]} \qquad \qquad \Dp{T}{t} = - D\e{T} \, \DDp{T}{x} \quad \textrm{[1D]} \]
\noindent où~$D\e{T}$ est la diffusivité thermique notée
\[ D\e{T} = \frac{\lambda\e{T}}{\rho \, c_p} \]

\sk
Dans le cas unidimensionnel de la diffusion thermique dans un sol uniforme à la profondeur~$z$, en supposant un forçage périodique~$T(0,t) = T_0 + T_0' \, \cos \omega t$ ($\omega$ étant adapté au cas considéré selon si forçage diurne, saisonnier, \ldots), l'équation de diffusion thermique permet d'obtenir l'expression des variations spatiales et temporelles de~$T$
\[ T(x,t) = T_0 + T_0' \, e^{-\frac{z}{\delta}} \, \cos (\omega t - \frac{z}{\delta}) \]
\noindent où l'atténuation avec la profondeur est~$e^{-\frac{z}{\delta}}$ et le déphasage du maximum du forçage est~$\Delta t = \frac{z}{\omega \delta}$, avec~$\delta$ l'épaisseur de peau qui s'exprime
\[ \delta = \frac{2\,D\e{T}}{\omega} \]
\noindent En pratique, l'atténuation et le déphasage sont très marquées pour des profondeurs dans le sol même très modérées.








\newpage
\section{Inertie thermique}
L'inertie thermique $I$ mesure la résistance
thermique d'un milieu à un apport ou un
déficit de chaleur.
%
L'expression de $I$ (J~m$^{-2}$~s$^{-1/2}$~K$^{-1}$) 
s'obtient en déduisant d'une équation
simple de conduction thermique de Fourier,
par analyse dimensionnelle, l'épaisseur
de peau thermique $\delta$ 
%
\[
\rho \, c_p \, \Dp{T}{t} = \Dp{}{x} \left( \lambda \, \Dp{T}{x} \right)
\quad
\to
\quad
\delta = \sqrt{\f{\lambda \, \tau}{\rho \, c_p}}
\]
%
\noindent (où $\tau$ est une constante caractéristique de temps)
ce qui permet de mettre en évidence l'inertie
thermique dans le terme de flux de chaleur $\phi\e{c}$
à la surface
%
\[
\phi\e{c} = - \lambda \, \Dp{T}{x} = - \f{\lambda}{\delta} \, \Dp{T}{x'} 
= - \sqrt{\f{\lambda \, \rho \, c_p}{\tau}} \, \Dp{T}{x'}
\quad 
\textrm{avec}
\quad
x'=x/\delta 
\]
\noindent en ne retenant que les termes qui dépendent du milieu
dans la caractérisation du flux de chaleur :
%
%\[
$\boxed{
I = \sqrt{\lambda \, \rho \, c_p}
}$
%\]


Un milieu est donc de faible inertie thermique
lorsqu'il ne peut stocker que de petites quantités de chaleur
(faible capacité calorifique $c_p$)
et/ou qu'il ne peut transmettre cette chaleur que dans ses couches superficielles
(faible conductivité thermique $\lambda$).
%
Les océans terrestres constituent un exemple 
bien connu de milieu à très forte inertie thermique, 
de par leur grande capacité calorifique.
%
Autre exemple bien connu, 
l'inertie thermique des terrains rocheux
martiens est plus élevée que 
l'inertie thermique des terrains
poussiéreux, principalement
pour des raison de conductivité thermique.
%
L'inertie thermique
peut d'ailleurs permettre 
sous certaines conditions d'estimer
la taille des grains dans les sols
non consolidés.

Dépourvue d'océans, Mars forme
un gigantesque désert de faible inertie
thermique : $I$~dépasse 
rarement $400\U{J~m^{-2}~s^{-1/2}~K^{-1}}$
pour la plupart des sols martiens.
%
Pour qualifier les grands
ensembles sur le champ d'inertie
thermique planétaire,
le terme de \ofg{continents thermiques}
est parfois employé.
%
L'inertie thermique n'est pas une
quantité observable directement
et sa détermination requiert la combinaison
de mesures de température de surface et
d'un modèle simulant les variations thermiques du sol.


\newpage
\section{Constante de temps radiative}
A condition qu'elle abrite
des particules radiativement actives,
une atmosphère de pression
plus faible se caractérise par une constante
de relaxation radiative $\tau\e{R}$ plus
courte.
%
En effet, lorsque la pression
diminue, la densité diminue également
mais l'énergie radiative absorbée n'est
pas proportionnelle à la densité.
%
Nous pouvons illustrer ce point avec
un calcul simpl(ist)e sur une couche
atmosphérique d'épaisseur $e$, de densité $\rho$ 
et se comportant comme un corps noir
de température $T\e{e}$
($\sigma$ est la constante de Stefan-Boltzmann).
%
Le temps caractéristique $\tau\e{R}$ pour 
dissiper radiativement une
perturbation thermique $\Theta=\Delta T$
de l'équilibre radiatif de la couche
avec les couches environnantes est
donné par la conservation de l'énergie
%%
%Le temps de relaxation radiatif $\tau\e{R}$, 
%inverse de la constante
%d'amortissement radiatif, est le temps 
%nécessaire pour dissiper une perturbation
%thermique par des processus radiatifs.
%%
\[
\ddt{\Theta} + \f{\Theta}{\tau\e{R}} = 0
\quad \textrm{avec} \quad
\tau\e{R} = \f{c_p \, \rho \, e}{8 \, \sigma \, T\e{e}^3}
\]
%
\noindent L'expression de $\tau\e{R}$ traduit
bien le point mentionné précédemment.
%
Le rapport entre les constantes
de temps radiatives martiennes et terrestres
est donc principalement contrôlé
par la différence de densité.
%
Les valeurs planétaires donnent par exemple
%
\[
\f{ \tau\e{Mars} }{ \tau\e{Terre} }
=
\f{ \left( c_p \, \rho / T\e{e}^3 \right)\e{Mars} }{ \left( c_p \, \rho / T\e{e}^3 \right)\e{Terre} }
\sim 
\f{1}{40}
\]
%
\noindent soit un très fort amortissement
radiatif dans l'atmosphère martienne, deux ordres
de grandeur plus élevé que sur Terre.
%
Dans les conditions typiques pour la basse
atmosphère terrestre et martienne, 
$\tau\e{Mars}$ est de l'ordre de la journée
alors que $\tau\e{Terre}$ est de l'ordre du mois.
%
Sur Terre, les différences de constante radiative
entre la basse et la haute atmosphère sont expliquées
de la même façon.
%
L'estimation ci-dessus reste illustrative plus que quantitativement
valable.
%
Cependant, des calculs plus élaborés 
distinguant les molécules radiativement actives 
dans l'\IR~thermique sur Mars (\carb) et sur Terre (\eau),
donnent $\tau\e{Mars}/\tau\e{Terre}$
entre $1/5$ et $1/100$, ce qui
ne contredit pas l'ordre de grandeur
trouvé par le calcul simpliste
précédent. 
%
%
%%La grande concentration en \carb~et
%%la densité faible font que le temps
%%de relaxation radiatif de \lam~est
%%très faible.


\newpage
\section{\'Equation hypsométrique}
\sk
Dans l'équation de l'échelle de hauteur, faire l'hypothèse isotherme est très simpliste et rarement rencontré en pratique dans l'atmosphère. On se place dans le cas plus général, bien que toujours simplifié, de deux niveaux atmosphériques~$a$ et~$b$ entre lesquels la température ne varie pas trop brusquement avec l'altitude~$z$. On réalise alors l'intégration entre les deux niveaux~$a$ et~$b$
\[R \, T \, \frac{\dd P}{P} = - g \, \dd z \qquad \Rightarrow \qquad R \, \int_a^b T\, \frac{\dd P}{P} = - g \, \int_a^b dz\]
puis on définit la température moyenne de la couche atmosphérique entre~$a$ et~$b$ avec une moyenne pondérée
\[ \langle T \rangle = \frac{\int_a^b T \, \frac{\dd P}{P}}{\int_a^b \frac{\dd P}{P}} \]
pour obtenir finalement
\[R \, \langle T \rangle \, \int_a^b \frac{\dd P}{P} = - g \, \int_a^b dz
\qquad \Rightarrow \qquad \boxed{ g \, (z_a - z_b) = R \, \langle T \rangle \ln \left( \frac{P_b}{P_a} \right) } \]
Cette relation est appelée \voc{équation hypsométrique}. Elle correspond à une formulation utile en météorologie du principe que \ofg{l'air chaud se dilate}. Les conséquences de l'équation hypsométrique peuvent s'exprimer de diverses façons équivalentes.
\begin{citemize}
\item Pour une masse d'air donnée, une couche d'air chaud est plus épaisse.
\item La distance entre deux isobares est plus grande si l'air est chaud.
\item La pression diminue plus vite selon l'altitude dans une couche d'air froid.
\end{citemize}
En passant le résultat précédent au logarithme, on note que l'on retrouve toujours le fait que la pression diminue avec l'altitude selon une loi exponentielle. En notant l'échelle de hauteur moyenne~$\langle H \rangle$, on a
\[ P_b = P_a \, e^{ - \frac{z_b - z_a}{\langle H \rangle}} \qquad \text{avec} \qquad \langle H \rangle = \frac{R \, \langle T \rangle}{g} \]





\newpage
\section{Cellules de Hadley et Jets}
\sk
La structure du vent zonal est dominée aux moyennes latitudes sur Terre par la présence de deux \voc{jets}, c'est-à-dire de puissants courants atmosphériques, dits \voc{jets d'ouest} car ils soufflent de l'ouest vers l'est. Leur vitesse augmente sur la verticale entre la surface et un maximum au niveau de la tropopause, autour de 50~m~s$^{-1}$. Ce comportement peut être justifié en combinant l'équilibre géostrophique à l'équilibre hydrostatique (équation du vent thermique). Dans les tropiques, les vents moyens sont d'est, surtout dominants dans la basse troposphère, mais restent néanmoins moins forts que les vents d'ouest dans les moyennes latitudes. On les appelle les \voc{alizés}.

\sk
La structure en latitude des vents %décrite par la figure~\ref{fig:UTlatP}, 
avec des vents d'ouest aux moyennes latitudes et d'est sous les tropiques, est très liée à la circulation de Hadley.
% décrite par la figure~\ref{fig:MMC}.
Sous l'action de la force de Coriolis, les mouvements vers les pôles sont déviés vers l'est et les mouvements vers l'équateur sont déviés vers l'ouest. Les jets d'ouest des moyennes latitudes proviennent ainsi de la déviation vers l'est de la circulation vers les pôles dans la branche supérieure de la cellule de Hadley. Les vents d'est (alizés) sous les tropiques proviennent quant à eux de la déviation vers l'ouest de la circulation vers l'équateur dans la branche inférieure de la cellule de Hadley. Les vents de grande échelle comportent donc une composante vers l'équateur et l'ouest sous les tropiques, alors qu'aux moyennes latitudes, ils comportent une composante vers les pôles et l'est [la composante vers l'est domine cependant]. Une exception à cette image est observée dans les régions de ``mousson'' (sous-continent Indien, et dans une moindre mesure Afrique de l'ouest et Amérique centrale) où la direction du vent s'inverse entre l'été (vers le continent) et l'hiver (vers l'océan), sous l'effet local de circulations thermiques directes.

\sk
Un résultat similaire sur le lien entre cellules de Hadley et jets est obtenu par des arguments de conservation du moment cinétique. Le moment cinétique absolu $\mathcal{M}$ d'une particule d'air de masse $m$ par rapport à l'axe de rotation de la planète est en coordonnées sphériques
%
\[
\mathcal{M} = m \, (a \, \cos \varphi) (\Omega \, a \cos \varphi + u)
\]
%
\noindent pour une particule située à la latitude $\varphi$. 
%
\noindent La conservation de ce moment cinétique~$M$ impose à la particule d'air advectée vers les pôles, donc se rapprochant de son axe de rotation,
d'accélérer en un vent prograde d'altitude (jets d'ouest) et à la particule d'air ramenée vers l'équateur de décélérer en un vent rétrograde de surface (alizés).

\sk
Au-delà d'une certaine latitude, la conservation du moment cinétique autour de l'axe planétaire cesse d'être valable et les circulations non-axisymétriques (ondes de Rossby par exemple) prennent une part dominante dans le transport de moment cinétique. Dans ce cas, le vent ne peut plus être déterminé quantitativement par conservation du moment cinétique mais par son lien diagnostique à la structure thermique (équilibre du vent thermique).

\figside{0.6}{0.3}{\figfrancis/WH_circ_scheme}{Schéma de la circulation atmosphérique: zone de convergence et alizés dans les tropiques; gradient de pression tropiques (H) -pôle (L), vents d'ouest et ondes aux moyennes latitudes. La position des jets d'ouest et l'extension des cellules de Hadley sont représentées à droite. Figure adaptée de Wallace and Hobbs, Atmospheric Science, 2006.}{fig:circscheme}


%\newpage
%\section{Cellules de Hadley (un peu plus détaillé)}
%\sk
Sur Mars comme sur Terre, les forçages radiatifs
sont le moteur de la circulation de grande échelle.
%
L'énergie radiative absorbée dans le visible
par le système \ofg{surface - atmosphère}
subit de plus fortes variations latitudinales
que l'énergie radiative émise dans l'\IR.
%
En moyenne annuelle, il en résulte un chauffage net 
des régions équatoriales de la planète et
un refroidissement net de ses régions polaires.
%
Ce contraste thermique induit des contrastes latitudinaux d'échelle de hauteur $H$,
donc, d'après l'équation hypsométrique,
un gradient de pression latitudinal qui augmente avec l'altitude.
%
Une telle force méridienne entraîne une circulation en altitude
des régions équatoriales excédentaires en énergie
aux hautes latitudes en déficit d'énergie.
%
De ce transport de masse résulte une augmentation de la pression de surface
aux hautes latitudes et donc une circulation inversée proche du sol.
%
En moyenne zonale, ces mouvements sont conceptualisés par les
\ofg{cellules de Hadley}.


\sk
La circulation précitée prend place
dans un référentiel particulièrement non-galiléen :
la planète en rotation sur elle-même.
%
Le moment cinétique absolu $\mathcal{M}$ 
d'une particule d'air de masse $m$
par rapport à l'axe de rotation de la planète
est en coordonnées sphériques
%
\[
\mathcal{M} = m \, (a \, \cos \varphi) (\Omega \, a \cos \varphi + u)
\]
%
\noindent pour une particule située à la latitude $\varphi$. 
%
\noindent La conservation de ce moment cinétique
impose à la particule d'air advectée vers
les pôles, donc se rapprochant de son axe de rotation,
d'accélérer en un vent prograde d'altitude (jets d'ouest)
et à la particule d'air ramenée vers l'équateur de
décélérer en un vent rétrograde de surface (alizés).
%%les vents sont créés par transport de moment cinétique
%
La résultante $\vec{F\e{e}}$ des forces 
d'entraînement centrifuge et de Coriolis
%
\[
\vec{F\e{e}} = - m \, (2\,\Omega\,\sin\varphi\,u + \f{u^2\,\tan\varphi}{a}) \, \vec{y}
\]
%
\noindent est alors dirigée vers l'équateur
pour une particule animée d'un vent prograde
$u > 0$,
%tournant plus vite
%que la planète (i.e. animée d'un vent d'ouest ), 
et s'oppose au gradient de pression ayant donné
naissance à la circulation de Hadley, limitant son extension.
%
Au-delà d'une certaine latitude, la conservation du moment cinétique
autour de l'axe planétaire cesse d'être valable
et les circulations non-axisymétriques
prennent une part dominante dans le transport de moment cinétique.
%
%notamment les ondes stationnaires
%et les ondes baroclines.
%
Dans ce cas, le vent ne peut plus être déterminé
quantitativement par conservation du moment cinétique mais
par son lien diagnostique à la structure thermique
(équilibre du vent thermique)
%
\[
- \Dp{\vec{v_H}}{p} = \f{R}{p \, f} \: \vec{z} \wedge \vec{\nabla_p} T
\quad
\textrm{(coord. isobares)}
\]
%
\noindent qui combine l'équilibre vertical hydrostatique  
avec l'équilibre horizontal géostrophique aux moyennes latitudes.
%
Le modèle conceptuel axisymétrique simple de \textit{Held et Hou} [1980]\nocite{Held:80}
se base sur cette distinction entre deux régimes de vent pour en déduire
l'extension latitudinale de la cellule de Hadley $\mathcal{L}$
%et la vitesse
%du vent zonal maximal $\mathcal{U}$
%
%\quad \textrm{et} \quad \mathcal{U} = \f{ \Omega \, \mathcal{L}^2 }{ a }
%
\[
\mathcal{L} = \sqrt{ \f{ 5 \, \Delta \theta \, g \, H }{ 3 \, \Omega^2 \, \theta_0 } }
\]
%
\noindent En utilisant les constantes planétaires de la Terre et Mars,
et les contrastes thermiques typiques
$\Delta \theta\e{T} = 40\U{K}$
et $\Delta \theta\e{M} = 65\U{K}$,
nous obtenons 
$\mathcal{L}\e{M} \sim \mathcal{L}\e{T}$,
soit une cellule de Hadley
significativement plus étendue sur Mars
de rayon deux fois plus petit que celui
de la Terre.
%
Ce modèle reste néanmoins très simplifié et ne s'applique qu'aux 
moyennes annuelles principalement, ce qui en limite fortement
la portée sur des planètes aux saisons très marquées comme Mars.


%La vitesse maximale peut être déterminée
%à partir d'un arbitrage entre le vent
%limite au sens de la conservation du moment cinétique
%et le vent thermique; les valeurs 
%données par \numeroref{eq:heldhou} donnent
%ainsi $\mathcal{U}\e{T} \sim 55\U{m~s^{-1}}$
%et $\mathcal{U}\e{M} \sim 85\U{m~s^{-1}}$, en
%accord correct avec les résultats de modèles
%plus élaborés.




%%L'équilibre du vent thermique donne un 
%%sens des vents conforme à ce qu'on pourrait
%%déduire de la conservation du moment
%%cinétique, sur la base qu'une particule
%%se rapprochant de son axe de rotation
%%est accélérée dans le sens prograde.



\end{document}
