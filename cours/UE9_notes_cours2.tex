\documentclass[a4paper,DIV16,10pt]{scrartcl}
%%%%%%%%%%%%%%%%%%%%%%%%%%%%%%%%%%%%%%%%%%%%%%%%%%%%%%%%%%%%%%%%%%%%%%%%%%%%%%%%%%%
\usepackage{texcours}
%%%%%%%%%%%%%%%%%%%%%%%%%%%%%%%%%%%%%%%%%%%%%%%%%%%%%%%%%%%%%%%%%%%%%%%%%%%%%%%%%%%
\newcommand{\zauthor}{Aymeric SPIGA}
\newcommand{\zaffil}{Laboratoire de Météorologie Dynamique}
\newcommand{\zemail}{aymeric.spiga@upmc.fr}
\newcommand{\zcourse}{Physique des atmosphères planétaires}
\newcommand{\zcode}{UE9}
\newcommand{\zuniversity}{UPMC}
\newcommand{\zlevel}{M2 Planétologie}
\newcommand{\zsubtitle}{Fiches complémentaires du cours 2}
\newcommand{\zlogo}{\includegraphics[height=1.5cm]{decouverte/cours_meteo/UPMC_cart-blanc-Q_7504-703-3.png}}
\newcommand{\zrights}{Copie et usage interdits sans autorisation explicite de l'auteur}
\newcommand{\zdate}{\today}
%%%%%%%%%%%%%%%%%%%%%%%%%%%%%%%%%%%%%%%%%%%%%%%%%%%%%%%%%%%%%%%%%%%%%%%%%%%%%%%%%%%
\begin{document} \inidoc
%%%%%%%%%%%%%%%%%%%%%%%%%%%%%%%%%%%%%%%%%%%%%%%%%%%%%%%%%%%%%%%%%%%%%%%%%%%%%%%%%%%

%\newpage
%\section{Structures thermiques}
%

Profil radiatif-convectif + profil radiatif + effet de la convection humide

%\figun{0.2}{0.1}{decouverte/pierrehumbert_pics/9780521865562c02_fig001.jpg}{R. Pierrehumbert, Principles of Planetary Climates, CUP, 2010}{fig:profearth}

\figun{0.65}{0.5}{decouverte/pierrehumbert_pics/9780521865562c02_fig002.jpg}{R. Pierrehumbert, Principles of Planetary Climates, CUP, 2010}{fig:profplanet}



\newpage
\section{Profil radiatif convectif (telluriques)}
\sk
Les conditions atmosphériques sont très instables proche d'une surface (en présence d'une telle surface). A cause de la discontinuité entre surface et atmosphère, sous l'action de la diffusion thermique, ou turbulente, entre la surface (chaude) et l'air immédiatement adjacent (plus froid) crée une couche d'air fine approximativement à la température de la surface ; les conditions de température étant plus froides au-dessus, les conditions atmosphériques sont très instables proche de la surface et des mouvements de convection vont se mettre en place pour mélanger l'air sur une certaine épaisseur atmosphérique. Un équilibre dit \voc{radiatif-convectif} prévaut, avec une structure thermique suivant le profil adiabatique, donnant naissance à une troposphère. Au-dessus de la limite radiative-convective (correspondant peu ou prou à la tropopause), les phénomènes radiatifs dominent et donnent naissance à une mésosphère -- ou une stratosphère si un absorbant visible y est présent en quantité suffisante, donnant naissance à une inversion stable à la tropopause.
%% on passait en troposphère dès que le gradient du profil radiatif dépassait celui du profil adiabatique (-g/cp)

\figun{0.4}{0.25}{/home/aymeric/Big_Data/BOOKS/pierrehumbert_pics/9780521865562c03_fig014.jpg}{Figure tirée de R. Pierrehumbert, Principles of Planetary Climates, CUP, 2010}{fig:effetserre2}











\newpage
\section{Profil radiatif convectif (géantes)}
\sk
La présence d'une surface et une hypothèse d'équilibre radiatif imposent donc que le chauffage de la surface conduit inévitablement à de la convection. Que se passe-t-il sur les planètes géantes dépourvues de surface ? L'équilibre radiatif y prévaut également, car à partir d'une certaine profondeur, le gradient radiatif est instable -- et ce, même en l'absence d'une surface qui absorbe le rayonnement solaire. Pour formuler la stabilité de l'équilibre radiatif, on calcule le profil~$\dd T / \dd p$ et on le compare au gradient adiabatique sec ou humide dans l'atmosphère considérée. En dérivant le profil radiatif obtenu dans le cas du modèle à deux faisceaux
\[ T(\tau) = \sqrt[4]{\frac{OLR\,(1+\tau)}{2\,\sigma\,\epsilon}} \]
\noindent par rapport à l'épaisseur optique~$\tau$, nous obtenons
\[ 8 \, \sigma \, T^3 \, \ddf{T}{\tau} = OLR \]
\noindent soit, en utilisant~$\ddf{~}{p} = \ddf{\tau}{p} \ddf{~}{\tau}$
\[ \ddf{T}{\ln p} = \frac{1}{4\,(1+\tau)} \, p \, \ddf{\tau}{p} \]
Ainsi la stabilité de la couche s'écrit
\[ \frac{R}{c_p} \ge \frac{1}{4\,(1+\tau)} \, p \, \ddf{\tau}{p} \]
\noindent et dans le cas d'un coefficient d'absorption~$\kappa$ constant, nous pouvons même écrire la condition de stabilité
\[ \frac{R}{c_p} \ge \frac{\tau}{4\,(1+\tau)} \]
%% si l'absorption est constante p \, \ddf{\tau}{p} = - \kappa \, p / g costheta

\sk
Le terme en~$p$ dans ce qui précède guarantit (à moins d'une variation énorme de $\ddf{\tau}{p}$ en~$1/p$ ou plus rapide quand~$p \rightarrow 0$) que les hautes atmosphères planétaires sont toujours stables. De plus, les atmosphères optiquement fines sont toujours stables sur l'intégralité de leur épaisseur, puisque~$-p \, \ddf{\tau}{p} < \tau_\infty \ll 1$. Le critère de stabilité est en pratique un peu plus complexe qu'indiqué dans les atmosphères réelles. Bien sûr, $\tau$ et~$\kappa$ varient avec la longueur d'onde~$\lambda$ (limitation inhérente au modèle à deux faisceaux), mais surtout le coefficient d'absorption~$\kappa$ augmente avec la pression (donc la profondeur) en raison de l'élargissement collisionnel (\emph{collisional broadening}), efficace à partir de quelques bars. La loi de variation d'élargissement collisionnel peut s'écrire~$\kappa(p) = \kappa(p\e{s}) \, \frac{p}{p\e{s}}$. Les processus de changements d'état sont également susceptibles de rendre la situation plus complexe que le calcul proposé ici.


\newpage
\section{Effet de serre : déplacement d'équilibre}



%\figside{0.45}{0.3}{/home/aymeric/Big_Data/BOOKS/pierrehumbert_pics/9780521865562c03_fig005.jpg}{R. Pierrehumbert, Principles of Planetary Climates, CUP, 2010}{fig:effetserre1}

\sk
Nous présentons ici l'explication la plus simple (sans être simpliste) du déplacement
d'équilibre radiatif qu'induit l'augmentation de gaz à effet de serre.
%Le modèle à deux faisceaux ne nous aide pas énormément. diverge quand tau tend vers infini.

\sk
Il est possible de montrer par des calculs de transfert radiatif que le niveau
d'émission équivalent au sommet de l'atmosphère est tel que~$\tau = 1$.
Qualitativement, on comprend que les niveaux inférieurs sont optiquement
épais donc ne sont que marginalement ``vus'' depuis l'espace dans les longueurs d'onde infrarouges.
Ainsi dans l'équilibre TOA
\[ \TOA \] 
\noindent l'émission de rayonnement au niveau~$\tau=1$ 
à la température~$T(\tau=1)$ domine OLR.

\sk
Appelons~$P\e{rad}$ la pression du niveau~$\tau=1$. 
Pour relier
les deux quantités, on emploie la définition de l'épaisseur optique
$\EO$ que l'on combine
à l'équilibre hydrostatique pour obtenir
par intégration~$\tau = \kappa \frac{P}{g} \, q_X$,
avec $q_X$ le rapport de mélange massique 
de l'espèce~$X$ absorbante dans l'infrarouge.
Ainsi
\[ P\e{rad} = \frac{g}{\kappa \, q_X} \]

\figun{0.7}{0.25}{/home/aymeric/Big_Data/BOOKS/pierrehumbert_pics/9780521865562c03_fig006.jpg}{R. Pierrehumbert, Principles of Planetary Climates, CUP, 2010}{fig:effetserre2}

\sk
L'expression ci-dessus implique qu'une augmentation de
gaz à effet de serre ($q_X$ augmente) implique une 
élévation du niveau équivalent d'émission
($p\e{rad}$ diminue).
L'effet sur la température de surface se détermine alors
en écrivant la conservation de la température potentielle
dans la troposphère soumise à l'équilibre radiatif-convectif,
entre la surface et le niveau équivalent d'émission
\[ T_s = T\e{rad} \, \left( \frac{P\e{rad}}{P\e{s}} \right)^{-\frac{R}{c_p}} \]
\noindent où~$P_s$ est la pression de surface.
Une élévation du niveau équivalent d'émission
se traduit donc par une augmentation
de température (fournissant
un modèle à la fois simple et fidèle du 
changement climatique récent sur Terre, Figure~\ref{fig:effetserre2}). 
Approximativement, $OLR \sim \sigma \, T(P\e{rad})^4$
et~$P\e{rad}$ est alors défini par la 
condition TOA qui s'écrit~$OLR = (1-A\e{b}) \, \mathcal{F}\e{s}'$.
Le lien entre quantité de gaz à effet de serre~$q_X$
et température de surface~$T\e{s}$ peut ainsi s'écrire
\[ T\e{s} = \sqrt[4]{\frac{(1-A\e{b}) \, \mathcal{F}\e{s}'}{\sigma}} \, \left( \frac{\kappa \, q_X \,P\e{s}}{g} \right)^{\frac{R}{c_p}} \]


\newpage
\section{Effet de serre divergent}

%% http://www.skepticalscience.com/print.php?r=262

\sk
\paragraph{Approche rapide} Une augmentation de la température de surface~$T\e{s}$ 
est donc associée à une augmentation de la quantité de gaz à effet de serre~$q_X$.
Sur une planète pourvue d'océan,
une augmentation de la température de surface
provoque une augmentation de l'évaporation
donc de la quantité de vapeur d'eau dans l'atmosphère, 
par conséquent de l'effet de serre.
Il s'agit d'une \voc{rétroaction positive}~:
le système amplifie la perturbation initiale de température de surface.
La quantité~$q_X$ de vapeur d'eau dans l'atmosphère
peut donc virtuellement augmenter indéfiniment.
Néanmoins, la pression du niveau équivalent~$P\e{rad} = \frac{g}{\kappa \, q_X}$
ne peut diminuer indéfiniment, du moins continûment~:
lorsque le sommet de la couche radiative de l'atmosphère
est atteint $P\e{rad} \ll P\e{s}$ , la radiation sortante~OLR
atteint une valeur maximale~$OLR\e{max}$.

\sk
\paragraph{Approche plus subtile} On peut inverser le point de vue et se demander quel est la valeur
d'OLR qui correspond à une température de surface~$T\e{s}$.
D'après le modèle simplifié combinant la hauteur équivalente
d'émission et le profil radiatif-convectif dans la troposphère, nous avons
\[ OLR  = \textcolor{magenta}{\sigma \, T\e{s}^4} \textcolor{blue}{\left( \frac{g}{\kappa \, q_X \, P\e{s}} \right)^{\frac{4 \, R}{c_p}}} \]
Plaçons-nous toujours dans le cas d'une planète pourvue
d'océan en évaporation.
Pour les températures de surface relativement modérées,
les variations de quantité de vapeur d'eau~$q_X$
(et de pression de surface~$P\e{s}$)
sont modérées et les variations d'OLR suivent 
une loi en~$\sigma T\e{s}^4$ (terme en \textcolor{magenta}{magenta}).
Néanmoins, plus la température de surface~$T\e{s}$
augmente, plus la vapeur d'eau devient dominante
dans l'atmosphère en influençant~$q_X$, mais
surtout~$P\e{s}$ via la loi d'équilibre liquide-vapeur
de Clausius-Clapeyron
$  P\e{s}(T\e{s}) = P_0 \, \exp{ \left[ -\frac{\ell}{R\,T\e{s}} \right] } $
\noindent où~$\ell > 0$ est la chaleur latente de vaporisation.
Si $T\e{s}$ augmente, $P\e{s}(T\e{s})$ augmente, et de manière exponentielle.
Le terme en \textcolor{blue}{bleu}, qui décroît exponentiellement avec
la température~$T\e{s}$, influence de façon dominante
l'expression de l'OLR pour les température élevées.
L'effet combiné des deux termes 
(\textcolor{magenta}{magenta} et \textcolor{blue}{bleu})
impose donc que l'OLR atteint une valeur maximale~$OLR\e{max}$,
que l'on appelle limite de Komabayashi-Ingersoll
(du nom de deux auteurs d'articles indépendants parus à la fin des années 60).

\figside{0.4}{0.15}{/home/aymeric/Big_Data/BOOKS/pierrehumbert_pics/9780521865562c04_fig004.jpg}{R. Pierrehumbert, Principles of Planetary Climates, CUP, 2010}{fig:ki}

\sk
\paragraph{Effet de serre divergent} 
Quelle que soit l'approche adoptée pour définir~$OLR\e{max}$,
il existe cette limite lorsque toute l'atmosphère devient optiquement épaisse.
Si l'on se place dans un contexte de variation
(à l'échelle des temps géologiques) du flux incident
solaire~$(1-A\e{b}) \, \mathcal{F}\e{s}'$,
avec notamment une augmentation au cours du temps
étant donné l'activité radioactive du Soleil\footnote{En fait, le flux incident varie lorsque la pression atmosphérique devient conséquente à cause d'un effet de diffusion Rayleigh accru},
on constate que les valeurs du flux 
incident~$(1-A\e{b}) \, \mathcal{F}\e{s}'$ peuvent
dépasser~$OLR\e{max}$, ce qui signifie que
l'équilibre TOA ne peut être satisfait et que
l'atmosphère reçoit plus d'énergie qu'elle
n'en émet. La température de surface peut augmenter
de manière incontrôlée, au risque d'atteindre des valeurs
très élevées (plusieurs centaines de K, voire quelques milliers).
L'évaporation des océans peut alors
survenir de manière abrupte et rapide (Figure~\ref{fig:ki}),
dans ce que l'on appelle l'\voc{effet de serre divergent}
(\emph{runaway greenhouse}). De fait, 
l'équilibre radiatif type TOA peut n'être
récupéré que pour des températures de surface 
très élevées (valeurs de plus de~$1000-2000$~K,
pour lesquelles l'intégralité des océans a disparu
selon toute vraisemblance).
Au-delà de telles valeurs de température de surface~$T\e{s}$,
le flux sortant OLR se remet à augmenter avec~$T\e{s}$
en raison de la contribution grandissante de l'émission
thermique dans le visible (et de la moindre absorption
de la vapeur d'eau dans ces longueurs d'onde).


%%% KI : dépend de g


%% Fs + when Ts + because increased absorption of solar rad by water vapor
%% then - when Ts + because Rayleigh scattering



%\newpage
%\section{Rétroactions}
%
Les processus de rétroactions climatiques peuvent amplifier (on parle alors de \voc{rétroaction positive}, en anglais \emph{positive feedback}) ou réduire (\voc{rétroaction négative}) la réponse à une perturbation initiale et sont donc centraux pour simuler correctement l’évolution du climat.

\begin{finger}
\item La rétroaction de Stefan-Boltzmann : Si la température augmente alors la perte par rayonnement augmente : feedback négatif très fort
\item La rétroaction de la glace et de l’albédo : Si la température augmente alors la glace diminue et donc le rayonnement solaire absorbé augmente ce qui augmente la température : feedback positif
\item Rétroaction des gaz à effet de serre au cours d’un cycle glaciaire-interglaciaire: une entrée en glaciation entraîne une baisse de la teneur en gaz à effet de serre (CO2, H2O vapeur et CH4) dans l'atmosphère par suite des modifications du climat (refroidissement de la surface terrestre et modification de la circulation océanique profonde); cette diminution atténue l'effet de serre initial et donc amplifie le refroidissement en cours. Inversement une déglaciation entraîne une augmentation des mêmes gaz à effet de serre, ce qui, cette fois, contribue à accentuer le réchauffement. Feedback positif
%\item La rétroaction des nuages: Si la température augmente et induit plus de nuages qui réfléchissent plus d’énergie solaire alors la température diminue. Cependant par effet de serre des nuages, la température augmente. Au contraire, si le climat se refroidit, la couverture neigeuse hivernale persistera plus longtemps. Or cette couverture blanche (d’albédo plus élevé que le sol) augmente la réflexion de l'énergie solaire et donc diminue le chauffage de la surface par le Soleil. Il en résulte un refroidissement de la surface qui amplifie le refroidissement climatique initial -- feedback positif si refroidissement, inconnu si réchauffement (a priori négatif pour les nuages bas).
\end{finger}

%\sk \subsection{Rétroactions} Préciser les rétroactions en jeu Indiquer le sens de ces rétroactions. Exemple terrestre~: \begin{itemize} \item Stefan-Boltzman : positive ou négative \item Glaces : positive ou négative \item Vapeur d'eau : positive ou négative \item Nuages : positive ou négative \end{itemize} 
%\visible<2->{\vskip 0.5cm\ebloc{}{Applications~:~changement climatique, paléo-climats, évolution des planètes du système solaire, climat des exoplanètes}}} \note{Peut être en général étendu aux autres planètes.\\ Les processus de rétroactions climatiques peuvent amplifier (on parle alors de rétroactions positives) ou réduire (rétroaction négative) la réponse à une perturbation initiale\\ SB : Si la température augmente alors la perte par rayonnement augmente : feedback négatif très fort\\ glace (albedo) : Si la température augmente alors la glace diminue et donc le rayonnement solaire absorbé augmente ce qui augmente la température feedback positif, mais attention au contrôle par la taille de la calotte polaire.\\ vapeur d’eau : l’augmentation de la température tend à favoriser l'évaporation car l'équilibre L-V est déplacé, de fait augmentation de l'humidité ie le contenu en vapeur d’eau de l’atmosphère, ce qui augmente l’effet de serre et donc la température de surface\\ nuages :  Si la température augmente et induit plus de nuages qui réfléchissent plus d’énergie solaire alors la température diminue. Cependant par effet de serre des nuages, la température augmente…. Au contraire, si le climat se refroidit, la couverture neigeuse hivernale persistera plus longtemps. Or cette couverture blanche (d’albédo plus élevé que le sol) augmente la réflexion de l'énergie solaire et donc diminue le chauffage de la surface par le Soleil. Il en résulte un refroidissement de la surface qui amplifie le refroidissement climatique initial feedback positif si refroidissement, inconnu si réchauffement (a priori négatif pour les nuages bas)} \note{[FACULTATIF] GES cycle glaciaire vs. interg.~: une entrée en glaciation entraîne une baisse de la teneur en gaz à effet de serre (CO2, H2O vapeur et CH4) dans l'atmosphère par suite des modifications du climat (refroidissement de la surface terrestre et modification de la circulation océanique profonde); cette diminution atténue l'effet de serre initial et donc amplifie le refroidissement en cours. Inversement une déglaciation entraîne une augmentation des mêmes gaz à effet de serre, ce qui, cette fois, contribue à accentuer le réchauffement feedbak positif}


%\newpage
%\section{Conduction}
%\sk
Les transferts thermiques par conduction se font par diffusion thermique. L'énergie est transférée via les collisions entre molécules. Ce type de transfert est dominant dans les intérieurs planétaires et dans les hautes atmosphères (thermosphère). Dans le dernier cas, le libre parcours moyen est si long que les atomes / molécules peuvent se mouvoir très rapidement d'une localisation à une autre, résultant en une conduction très efficace et un profil en général proche de l'isotherme.

\sk
Par analogie avec la diffusion moléculaire, pour définir la diffusion thermique, il s'agit de définir une loi phénoménologique (loi de Fourier, analogue de la loi de Fick) et une équation de conservation dans un volume de contrôle (conservation de l'énergie, analogue de la conservation de la matière). On définit ainsi pour le cas de la diffusion thermique uni-dimensionnelle selon~$x$
\begin{citemize}
\item \textit{Cause} inhomogénéité spatiale : Différence de température~$T(x,t)$
\item \textit{Conséquence} Densité de courant de chaleur~$\vec{J_Q}$ (W~m$^{-2}$)
\item \textit{\'Echange} Chaleur~$\delta Q = J_Q \, S \, \dd t$
\item \textit{Loi phénoménologique} Loi de Fourier~$J_Q = - \lambda\e{T} \, \Dp{T}{x}$
\item \textit{Conductivité thermique} en W~m$^{-1}$~K$^{-1}$~: $\lambda\e{T,roche} = 1-2$, $\lambda\e{T,eau} = 0.5$, $\lambda\e{T,air} = 0.02$.
\item \textit{Equation bilan locale} Conservation de l'énergie interne~$ \rho \, c_p \, \Dp{T}{t} + \Dp{J_Q}{x} = 0 $
\end{citemize}

\sk
Les équations tridimensionnelles sont
\[  
\textrm{Loi de Fourier} \quad \vec{J_Q} = - \lambda\e{T} \, \nabla T 
\qquad \qquad
\textrm{Conservation de l'énergie} \quad \rho \, c_p \, \Dp{T}{t} + \nabla \cdot \vec{J_Q} = 0
\]
\noindent L'équation de conservation de l'énergie n'est rien d'autre que le premier principe appliqué à un volume de contrôle~: la variation temporelle d'énergie interne est égale à la divergence du flux de chaleur (flux sortant moins flux entrant).

\sk
Combiner loi phénoménologique et équation de conservation permet d'obtenir ce qui est communément appelé l'équation de la chaleur, ou plus précisément l'équation de diffusion thermique
\[ \Dp{T}{t} = - D\e{T} \, \nabla^2 T \quad \textrm{[3D]} \qquad \qquad \Dp{T}{t} = - D\e{T} \, \DDp{T}{x} \quad \textrm{[1D]} \]
\noindent où~$D\e{T}$ est la diffusivité thermique notée
\[ D\e{T} = \frac{\lambda\e{T}}{\rho \, c_p} \]

\sk
Dans le cas unidimensionnel de la diffusion thermique dans un sol uniforme à la profondeur~$z$, en supposant un forçage périodique~$T(0,t) = T_0 + T_0' \, \cos \omega t$ ($\omega$ étant adapté au cas considéré selon si forçage diurne, saisonnier, \ldots), l'équation de diffusion thermique permet d'obtenir l'expression des variations spatiales et temporelles de~$T$
\[ T(x,t) = T_0 + T_0' \, e^{-\frac{z}{\delta}} \, \cos (\omega t - \frac{z}{\delta}) \]
\noindent où l'atténuation avec la profondeur est~$e^{-\frac{z}{\delta}}$ et le déphasage du maximum du forçage est~$\Delta t = \frac{z}{\omega \delta}$, avec~$\delta$ l'épaisseur de peau qui s'exprime
\[ \delta = \frac{2\,D\e{T}}{\omega} \]
\noindent En pratique, l'atténuation et le déphasage sont très marquées pour des profondeurs dans le sol même très modérées.







%
%\newpage
%\section{Inertie thermique}
%L'inertie thermique $I$ mesure la résistance
thermique d'un milieu à un apport ou un
déficit de chaleur.
%
L'expression de $I$ (J~m$^{-2}$~s$^{-1/2}$~K$^{-1}$) 
s'obtient en déduisant d'une équation
simple de conduction thermique de Fourier,
par analyse dimensionnelle, l'épaisseur
de peau thermique $\delta$ 
%
\[
\rho \, c_p \, \Dp{T}{t} = \Dp{}{x} \left( \lambda \, \Dp{T}{x} \right)
\quad
\to
\quad
\delta = \sqrt{\f{\lambda \, \tau}{\rho \, c_p}}
\]
%
\noindent (où $\tau$ est une constante caractéristique de temps)
ce qui permet de mettre en évidence l'inertie
thermique dans le terme de flux de chaleur $\phi\e{c}$
à la surface
%
\[
\phi\e{c} = - \lambda \, \Dp{T}{x} = - \f{\lambda}{\delta} \, \Dp{T}{x'} 
= - \sqrt{\f{\lambda \, \rho \, c_p}{\tau}} \, \Dp{T}{x'}
\quad 
\textrm{avec}
\quad
x'=x/\delta 
\]
\noindent en ne retenant que les termes qui dépendent du milieu
dans la caractérisation du flux de chaleur :
%
%\[
$\boxed{
I = \sqrt{\lambda \, \rho \, c_p}
}$
%\]


Un milieu est donc de faible inertie thermique
lorsqu'il ne peut stocker que de petites quantités de chaleur
(faible capacité calorifique $c_p$)
et/ou qu'il ne peut transmettre cette chaleur que dans ses couches superficielles
(faible conductivité thermique $\lambda$).
%
Les océans terrestres constituent un exemple 
bien connu de milieu à très forte inertie thermique, 
de par leur grande capacité calorifique.
%
Autre exemple bien connu, 
l'inertie thermique des terrains rocheux
martiens est plus élevée que 
l'inertie thermique des terrains
poussiéreux, principalement
pour des raison de conductivité thermique.
%
L'inertie thermique
peut d'ailleurs permettre 
sous certaines conditions d'estimer
la taille des grains dans les sols
non consolidés.

Dépourvue d'océans, Mars forme
un gigantesque désert de faible inertie
thermique : $I$~dépasse 
rarement $400\U{J~m^{-2}~s^{-1/2}~K^{-1}}$
pour la plupart des sols martiens.
%
Pour qualifier les grands
ensembles sur le champ d'inertie
thermique planétaire,
le terme de \ofg{continents thermiques}
est parfois employé.
%
L'inertie thermique n'est pas une
quantité observable directement
et sa détermination requiert la combinaison
de mesures de température de surface et
d'un modèle simulant les variations thermiques du sol.

%
%\newpage
%\section{Constante de temps radiative}
%A condition qu'elle abrite
des particules radiativement actives,
une atmosphère de pression
plus faible se caractérise par une constante
de relaxation radiative $\tau\e{R}$ plus
courte.
%
En effet, lorsque la pression
diminue, la densité diminue également
mais l'énergie radiative absorbée n'est
pas proportionnelle à la densité.
%
Nous pouvons illustrer ce point avec
un calcul simpl(ist)e sur une couche
atmosphérique d'épaisseur $e$, de densité $\rho$ 
et se comportant comme un corps noir
de température $T\e{e}$
($\sigma$ est la constante de Stefan-Boltzmann).
%
Le temps caractéristique $\tau\e{R}$ pour 
dissiper radiativement une
perturbation thermique $\Theta=\Delta T$
de l'équilibre radiatif de la couche
avec les couches environnantes est
donné par la conservation de l'énergie
%%
%Le temps de relaxation radiatif $\tau\e{R}$, 
%inverse de la constante
%d'amortissement radiatif, est le temps 
%nécessaire pour dissiper une perturbation
%thermique par des processus radiatifs.
%%
\[
\ddt{\Theta} + \f{\Theta}{\tau\e{R}} = 0
\quad \textrm{avec} \quad
\tau\e{R} = \f{c_p \, \rho \, e}{8 \, \sigma \, T\e{e}^3}
\]
%
\noindent L'expression de $\tau\e{R}$ traduit
bien le point mentionné précédemment.
%
Le rapport entre les constantes
de temps radiatives martiennes et terrestres
est donc principalement contrôlé
par la différence de densité.
%
Les valeurs planétaires donnent par exemple
%
\[
\f{ \tau\e{Mars} }{ \tau\e{Terre} }
=
\f{ \left( c_p \, \rho / T\e{e}^3 \right)\e{Mars} }{ \left( c_p \, \rho / T\e{e}^3 \right)\e{Terre} }
\sim 
\f{1}{40}
\]
%
\noindent soit un très fort amortissement
radiatif dans l'atmosphère martienne, deux ordres
de grandeur plus élevé que sur Terre.
%
Dans les conditions typiques pour la basse
atmosphère terrestre et martienne, 
$\tau\e{Mars}$ est de l'ordre de la journée
alors que $\tau\e{Terre}$ est de l'ordre du mois.
%
Sur Terre, les différences de constante radiative
entre la basse et la haute atmosphère sont expliquées
de la même façon.
%
L'estimation ci-dessus reste illustrative plus que quantitativement
valable.
%
Cependant, des calculs plus élaborés 
distinguant les molécules radiativement actives 
dans l'\IR~thermique sur Mars (\carb) et sur Terre (\eau),
donnent $\tau\e{Mars}/\tau\e{Terre}$
entre $1/5$ et $1/100$, ce qui
ne contredit pas l'ordre de grandeur
trouvé par le calcul simpliste
précédent. 
%
%
%%La grande concentration en \carb~et
%%la densité faible font que le temps
%%de relaxation radiatif de \lam~est
%%très faible.


\newpage
\section{Système et référentiel}
\sk
La position d'un point $M$ de l'atmosphère sera représentée dans un systèmes de coordonnées sphériques (figure~\ref{fig:repere}) par sa latitude $\varphi$, sa longitude $\lambda$, et son altitude~$z$ par rapport au niveau de la mer. Pour les déplacements horizontaux, on utilise le repère direct
$\left(M,\mathbf{i},\mathbf{j},\mathbf{k}\right)$ où $\mathbf{i}$ et $\mathbf{j}$ sont les vecteurs unitaires vers l'est et le nord, et $\mathbf{k}$ est dirigé suivant la verticale vers le haut. La direction définie par~$\mathbf{i}$ est souvent qualifiée de \voc{zonale}, celle définie par~$\mathbf{j}$ de \voc{méridienne}. 
%Pour des déplacements qui ne sont pas d'échelle planétaire, on utilisera également des distances horizontales vers l'est et le nord~$dx=a\, d\lambda\, \cos \varphi$ et~$dy=a\,d\varphi$ où~$a$ est le rayon de la Terre.
%
%\sk
On distingue deux référentiels pour l'étude des mouvements de l'air:
\begin{finger}
\item Un \voc{référentiel tournant} lié à la Terre, en rotation autour de l'axe des pôles avec la vitesse angulaire $\Omega$. La \voc{vitesse relative} est mesurée dans le référentiel tournant, par rapport à la surface de la Terre et a pour composantes~$u,v,w$ suivant \v i,\v j,\v k. Il s'agit de ce que l'on appelle communément le \voc{vent} avec le point de vue d'humain attaché à la surface de la Terre, c'est-à-dire au référentiel tournant. La composante horizontale du vecteur vitesse relative est donc~$\mathbf{V} = u \, \mathbf{i} + v \, \mathbf{j}$ et la composante verticale~$w \, \mathbf{k}$.
\item Un \voc{référentiel fixe} orienté suivant les directions de trois étoiles. La \voc{vitesse absolue} d'un point M est considérée dans le référentiel fixe et inclut donc le mouvement circulaire autour de l'axe des pôles. Ce référentiel peut être considéré comme galiléen. Il correspond à ce qu'on observerait depuis l'espace, lorsqu'on voit la Terre tourner au lieu d'être \ofg{attaché} à sa rotation.
\end{finger}

%\figun{0.4}{0.25}{\figfrancis/repere}{Schéma du système de coordonnées et du repère utilisés.}{fig:repere}
\figside{0.45}{0.22}{\figfrancis/repere}{Système de coordonnées et repère utilisés.}{fig:repere}



\section{Changement de référentiel}
\sk
L'équation de base pour le mouvement de masses d'air est la relation fondamentale de la dynamique $\Sigma \v F=m \, \v a$ (seconde loi de Newton).  Cette relation est cependant valable dans un référentiel galiléen, tel le référentiel fixe. On s'intéresse plutôt au vent, c'est-à-dire que l'on souhaite considérer des mouvements atmosphériques par rapport à la surface de la Terre qui est en rotation autour de l'axe des pôles. On va donc dans un premier temps projeter l'accélération dans le référentiel tournant, puis étudier les principales forces horizontales. Autrement dit, on se donne pour objectif d'exprimer l'accélération dans le référentiel tournant, qu'on souhaite connaître, en fonction de l'accélération dans le référentiel fixe, qui est égale à la somme des forces.

\sk
La relation entre vitesse absolue~$\v V_a$ dans le référentiel fixe et vitesse relative~$\v V_r$ dans le référentiel tournant s'écrit\footnote{La relation entre la dérivée temporelle d'un vecteur \v X dans le référentiel fixe (\emph{absolue}, $a$) et celle dans le référentiel tournant (\emph{relative}, $r$) s'écrit \[ \left[ \frac{d\v X}{dt} \right]_{a} = \left[ \frac{d\v X}{dt} \right]_{r} + \vl \Omega\wedge \v X\] En applicant au vecteur $\vl{X} \equiv \vl{CM}$, avec $\frac{d\vl{CM}}{dt}=\v V$, on a: \[\v V_a=\v V_r+\vl{\Omega}\wedge\vl{CM}\]}, avec le vecteur de rotation~$\v \Omega$ de module~$\Omega$ dirigé selon l'axe des pôles~:
\[\v V_a = \v V_r + \vl{\Omega}\wedge\vl{CM}\]
Il s'agit de la relation de composition des vitesses pour un référentiel tournant. Le terme $\vl{\Omega}\wedge\vl{CM}$ est la vitesse d'un point fixe par rapport au sol ($\v V_r=0$), il est appelé \voc{vitesse d'entrainement}.

\sk
En dérivant d'un ordre supplémentaire par rapport au temps, la relation entre accélération absolue~$\v a_a$, égale à la somme des forces, et accélération relative~$\v V_r$ dans le référentiel tournant s'écrit\footnote{Cette fois on écrit la relation de dérivation dans le référentiel tournant à $\vl{X} \equiv \v V_a= \v V_r+\vl{\Omega}\wedge\vl{CM}$ pour obtenir \[\v a_a = \left[ \frac{d}{dt} \left( \v V_r+\vl{\Omega}\wedge\vl{CM} \right) \right]_{r} + \vl{\Omega}\wedge \left(\v V_r+\vl{\Omega}\wedge\vl{CM}\right)\] d'où, en regroupant les termes en $\vl{\Omega}\wedge\v V_r$, la relation \[ \v a_a=\v a_r + 2\vl{\Omega}\wedge\v V_r + \vl{\Omega}\wedge(\vl{\Omega}\wedge\vl{CM}) \] \noindent Le dernier terme s'écrit également plus simplement \[ \vl{\Omega}\wedge(\vl{\Omega}\wedge\vl{CM})=-\Omega^2\cdot\vl{HM} \] }
\[ \v a_a=\Sigma\v F=\v a_r+2\vl{\Omega}\wedge\v V_r-\Omega^2\,\vl{HM} \]
Le premier terme est l'accélération relative~$\v a_r$, le deuxième l'\voc{accélération de Coriolis}~$\v a_c$, le troisième est l'\voc{accélération d'entrainement}~$\v a_e$. Les termes de Coriolis et d'entraînement induisent des \voc{forces apparentes}~$\v F_c = -m \, \v a_c$ et~$\v F_e = -m \, \v a_e$ dans le référentiel tournant. On parle de forces apparentes car du point de vue du référentiel fixe, ces termes n'apparaissent pas comme des forces~: ils ne sont que des termes d'accélération causés par le caractère non galiléen du référentiel tournant.



\newpage
\section{Accélération d'entraînement et pesanteur}
\sk
On considère un point M immobile par rapport à la surface de la Terre. Les forces (massiques) subies par M sont la force de gravitation \v G, dirigée vers le centre de la Terre, et \v R la réaction du sol dirigée perpendiculairement à la surface (figure \ref{fig:centrif}). Dans le référentiel fixe, l'accélération de M est celle du mouvement circulaire uniforme: $\v a_e = - \Omega^2 \, \vl{HM}$ (accélération d'entrainement). On doit donc avoir \[\v a_e=\v G+\v R\] 
C'est impossible si la Terre est sphérique (sauf au pôle et à l'équateur): on aurait alors \v R et \v G colinéaires mais pas dans la direction de $\v a_e$. La Terre a en fait pris une forme aplatie, où la surface n'est pas perpendiculaire à~$\v G$. En posant $\v g=\v G-\v a_e$, l'équilibre devient: \[\v g+\v R=\v 0\] On a donc une gravité apparente \v g dirigée localement vers le bas (perpendiculairement à la surface) mais pas exactement vers le centre de la Terre. La gravité réelle \v G a elle une faible composante horizontale. Dans ce qui suit, on considère que l'accélération d'entraînement est inclus dans le terme~$\v g$.

%\figside{0.55}{0.25}{\figfrancis/centrif}{Equilibre d'un point posé au sol. La forme réelle de la Terre est en trait continu, la sphère en pointillés.}{fig:centrif}
\figside{0.45}{0.2}{\figfrancis/centrif}{Equilibre d'un point posé au sol. La forme réelle de la Terre est en trait continu, la sphère en pointillés.}{fig:centrif}




\newpage
\section{Accélération de Coriolis et déviation du mouvement}
\sk
L'accélération de Coriolis peut être interprétée comme une force apparente massique $\v F_C = - 2 \, \v \Omega\wedge\v V_r$. Cette force apparente étant orthogonale à la vitesse à cause de la présence du produit vectoriel, sa puissance est nulle~: la \voc{force de Coriolis} va dévier le mouvement relatif mais ne peut pas modifier la vitesse du vent ou de courants. Pour des mouvements relatifs horizontaux à la vitesse \v V, le module de la force apparente de Coriolis est~$2 \, \Omega \, \sin \phi \, V$ qui change de signe lorsqu'on change d'hémisphère en fonction de~$\sin \phi$. Dans l'hémisphère nord, où $\sin \phi>0$, la force de Coriolis est dirigée à $90^{\circ}$ à droite du vent. 

\sk
Afin de bien comprendre l'effet de la force de Coriolis, il est profitable sur une planète comme la Terre d'utiliser la conservation du moment cinétique\footnote{
Puisque le moment cinétique~$\sigma$ se conserve on a \[ \ddf{\sigma}{t} = 0 = \ddf{r}{t} \, (\Omega \, r + u) + r \, \left( \Omega \ddf{r}{t}+\ddf{u}{t} \right) \qquad \Rightarrow \qquad \ddf{u}{t} = - \ddf{r}{t}  \, \left( 2\,\Omega + \frac{u}{r} \right) \] 
Le terme en $u/r$ est dû à la courbure de la surface, mais seule la vitesse relative intervient, pas la rotation de la Terre. En pratique, ce terme est négligeable sur Terre devant~$2 \, \Omega$. L'équation ci-dessus montre donc que raisonner avec la conservation du moment cinétique permet de comprendre l'effet sur les vents de la force de Coriolis.
}
(l'équivalent pour les systèmes en rotation de la quantité de mouvement pour les systèmes en translation). En effet, la somme des forces étant dirigée vers H, M conserve son \voc{moment cinétique}~$\sigma$ par rapport à l'axe des pôles, qui s'exprime
\[ \boxed{ \sigma = u_a \, r = (\Omega \, r + u) \, r } \]
où~$r$ est la distance entre le point considéré et l'axe de rotation qui passe par les deux pôles.
%\footnote{La conservation de $\sigma$ implique des variations de l'énergie cinétique $(\Omega r+u)^2$. C'est le travail de \v G (pour un mouvement sud-nord) qui en est l'origine.}. 

\sk
Pour illustrer les effets de cette force apparente de Coriolis, on considère une parcelle initialement au repos dans le référentiel tournant (c'est à dire~$u=0$ et~$v=0$ à~$t=0$) qui se déplacerait vers le Nord suivant l'axe~$\v j$. Elle se rapproche donc de l'axe des pôles et va voir sa vitesse absolue augmenter par conservation du moment cinétique: $\sigma$ est constant et~$r$ diminue, donc $u_a$ augmente. Dans le même temps, la vitesse d'entrainement locale~$u_e=\Omega \, r$ diminue sous l'effet de la diminution de la distance~$r$ à l'axe des pôles. La parcelle va donc acquérir une vitesse relative $u>0$ vers l'est\footnote{
En fait, l'expression ci-dessus permet même de calculer la variation de vitesse associée. Pour un mouvement sud-nord, la vitesse est $v=a \, \ddf{\phi}{t}$. D'autre part $r=a \, \cos \phi$ donc~$\ddf{r}{t}=-a \, \ddf{\phi}{t} \, \sin \phi = - v \, \sin \phi$. L'équation de conservation du moment cinétique devient 
\[ \ddf{u}{t} = v \, \sin \phi \, \left( 2 \, \Omega + \frac{u}{r} \right) \simeq 2 \, \Omega \, v \, \sin \phi \] 
La parcelle est bien déviée vers l'est pour un déplacement vers le nord tel que~$v>0$.
}
comme indiqué sur le schéma \ref{fig:coriolisns}. 

\figside{0.3}{0.2}{\figfrancis/coriolis_ns}{Déviation d'une parcelle se déplaçant vers le nord. Instant initial: vitesses d'entrainement $u_e$ et absolue $u_a$ égales. Instant final: vitesse d'entrainement $u_e'$ et absolue $u_a'$ augmentée par conservation du moment cinétique $\sigma$.}{fig:coriolisns}

%\subsubsection{Force de Coriolis: mouvement vers l'est} On considère un point M en mouvement par rapport à la surface de la Terre. On rappelle que pour un mouvement circulaire, on doit avoir une accélération normale égale à $V^2/R$ dirigée vers le centre du cercle. On suppose que les forces réelles s'exerçant sur M sont les mêmes que pour un point fixe: $\Sigma \vec F=\v a_e$. La composante de la vitesse relative vers l'est (suivant \v i) est $u$, et $\dot{r}$ dans la direction \vl{HM}. La vitesse absolue de M vers l'est est $u_a=\Omega r+u$. La relation $\v a=\Sigma\v F$ s'écrit dans la direction $\v e_r$: \[-\frac{(\Omega r+u)^2}{r}+\ddot{r}=a_e=-\Omega^2r\] soit en développant: \[\ddot{r}=u\cdot(2\Omega+\frac{u}{r})\] Pour un mouvement relatif vers l'est ($u>0$), la vitesse absolue est supérieure à la vitesse d'entrainement, et la somme des forces est insuffisante pour compenser $V_a^2/r$. La parcelle va donc s'éloigner de l'axe de rotation (figure \ref{fig:coriolisew}). Elle va au contraire se rapprocher pour $u<0$ (mouvement vers l'ouest). Pour trouver l'accélération relative dans la direction sud-nord, on projette $\v e_r$ sur \v j: $\dot{v}=-\ddot{r}\sin \phi$. \[\dot{v}=-u\sin \phi\cdot(2\Omega+\frac{u}{r})\] M est donc dévié vers le sud pour un déplacement relatif vers l'est.
%\begin{figure}[tbp] \begin{center} \includegraphics[width=12cm]{\figfrancis/coriolis_ew} \end{center} \caption{Déviation d'une parcelle ayant une vitesse relative initiale non nulle vers l'est (gauche) et l'ouest (droite). Un plan parallèle à l'équateur est représenté, vu depuis le pôle nord, l'axe de rotation est au centre. Les vitesse et accélération d'entrainement (égale à la somme des forces) sont en noir, la vitesse absolue en rouge. La trajectoire future de la parcelle est en pointillés.} \label{fig:coriolisew} \end{figure}


\newpage
\section{Cellules de Hadley et courants-jets}
\sk
La structure du vent zonal est dominée aux moyennes latitudes sur Terre par la présence de deux \voc{jets}, c'est-à-dire de puissants courants atmosphériques, dits \voc{jets d'ouest} car ils soufflent de l'ouest vers l'est. Leur vitesse augmente sur la verticale entre la surface et un maximum au niveau de la tropopause, autour de 50~m~s$^{-1}$. Ce comportement peut être justifié en combinant l'équilibre géostrophique à l'équilibre hydrostatique (équation du vent thermique). Dans les tropiques, les vents moyens sont d'est, surtout dominants dans la basse troposphère, mais restent néanmoins moins forts que les vents d'ouest dans les moyennes latitudes. On les appelle les \voc{alizés}.

\sk
La structure en latitude des vents %décrite par la figure~\ref{fig:UTlatP}, 
avec des vents d'ouest aux moyennes latitudes et d'est sous les tropiques, est très liée à la circulation de Hadley.
% décrite par la figure~\ref{fig:MMC}.
Sous l'action de la force de Coriolis, les mouvements vers les pôles sont déviés vers l'est et les mouvements vers l'équateur sont déviés vers l'ouest. Les jets d'ouest des moyennes latitudes proviennent ainsi de la déviation vers l'est de la circulation vers les pôles dans la branche supérieure de la cellule de Hadley. Les vents d'est (alizés) sous les tropiques proviennent quant à eux de la déviation vers l'ouest de la circulation vers l'équateur dans la branche inférieure de la cellule de Hadley. Les vents de grande échelle comportent donc une composante vers l'équateur et l'ouest sous les tropiques, alors qu'aux moyennes latitudes, ils comportent une composante vers les pôles et l'est [la composante vers l'est domine cependant]. Une exception à cette image est observée dans les régions de ``mousson'' (sous-continent Indien, et dans une moindre mesure Afrique de l'ouest et Amérique centrale) où la direction du vent s'inverse entre l'été (vers le continent) et l'hiver (vers l'océan), sous l'effet local de circulations thermiques directes.

\sk
Un résultat similaire sur le lien entre cellules de Hadley et jets est obtenu par des arguments de conservation du moment cinétique. Le moment cinétique absolu $\mathcal{M}$ d'une particule d'air de masse $m$ par rapport à l'axe de rotation de la planète est en coordonnées sphériques
%
\[
\mathcal{M} = m \, (a \, \cos \varphi) (\Omega \, a \cos \varphi + u)
\]
%
\noindent pour une particule située à la latitude $\varphi$. 
%
\noindent La conservation de ce moment cinétique~$M$ impose à la particule d'air advectée vers les pôles, donc se rapprochant de son axe de rotation,
d'accélérer en un vent prograde d'altitude (jets d'ouest) et à la particule d'air ramenée vers l'équateur de décélérer en un vent rétrograde de surface (alizés).

\sk
Au-delà d'une certaine latitude, la conservation du moment cinétique autour de l'axe planétaire cesse d'être valable et les circulations non-axisymétriques (ondes de Rossby par exemple) prennent une part dominante dans le transport de moment cinétique. Dans ce cas, le vent ne peut plus être déterminé quantitativement par conservation du moment cinétique mais par son lien diagnostique à la structure thermique (équilibre du vent thermique).

\figside{0.6}{0.3}{\figfrancis/WH_circ_scheme}{Schéma de la circulation atmosphérique: zone de convergence et alizés dans les tropiques; gradient de pression tropiques (H) -pôle (L), vents d'ouest et ondes aux moyennes latitudes. La position des jets d'ouest et l'extension des cellules de Hadley sont représentées à droite. Figure adaptée de Wallace and Hobbs, Atmospheric Science, 2006.}{fig:circscheme}


\newpage
\section{Forces de pression}
\sk
Les forces de pression horizontales se calculent comme la force de pression verticale dans la démonstration de l'équilibre hydrostatique. La force de pression s'exerçant sur une surface $S$ est normale à cette surface et vaut $P \, S$. Pour une parcelle d'air de volume $\delta x \, \delta y \, \delta z$ (figure \ref{fig:pres}), la force de pression totale dans la direction ($Ox$) vaut
\[ F_P^* = P(x) \, \delta y \, \delta z - P(x+\delta x) \, \delta y \, \delta z = - \frac{\partial P}{\partial x} \, \delta x \, \delta y \, \delta z \]
La force de pression {\em massique} est donc
\[F_P = \frac{F_P^*}{\rho \delta x \delta y \delta z}=-\frac{1}{\rho}\frac{\partial P}{\partial x}\]
On peut faire le même calcul sur ($Oy$). Finalement les deux composantes horizontales de la force de pression s'écrivent
\[\v F_P^H = -\frac{1}{\rho} \, \binom{\frac{\partial P}{\partial x}}{\frac{\partial P}{\partial y}}\] %  =-\frac{1}{\rho}\vl{grad}P\]

\sk
La force de pression est donc opposée aux variations horizontales de pression données par les dérivées partielles, ce qui lui confère des propriétés importantes.
\begin{citemize}
\item La force de pression est dirigée des hautes vers les basses pressions, perpendiculairement aux isobares.
\item La force de pression est inversement proportionelle à l'écartement des isobares.
\end{citemize}
Une région où la pression est particulièrement basse est appelée \voc{dépression}. Une région où la pression est particulièrement élevée est appelée \voc{anticyclone}.

\figside{0.6}{0.2}{\figfrancis/pressure}{Forces de pression (suivant ($Ox$)) s'exerçant sur une parcelle.}{fig:pres}

%\subsubsection{Équivalence avec le géopotentiel}
%L'équilibre hydrostatique fait que la pression décroit toujours avec
%l'altitude. Une pression localement élevée doit donc correspondre à une
%altitude élevée des surfaces isobares.
%\begin{figure}[htp]
%  \begin{center}
%    \includegraphics[width=\figwn]{\figfrancis/pres_geop}
%  \end{center}
%  \caption{Équivalence entre écarts de pression et d'altitude: les points A et
%  B sont à la même altitude, A et C à la même pression. La pression en B est
%  donc supérieure à celle en B.}
%  \label{fig:pres_geop}
%\end{figure}
%Sur la figure \ref{fig:pres_geop}, la force de pression horizonale dans la
%direction ($Ox$) est
%$F_P=-\frac{1}{\rho}\frac{P_B-P_A}{\delta x}$. Or $A$ et $C$ sont à la même
%pression, on a donc
%\[F_P=-\frac{1}{\rho}\frac{P_B-P_C}{\delta x}=-\frac{1}{\rho}\frac{P_B-P_C}{\delta z}\cdot\frac{\delta z}{\delta x}\]
%En utilisant
%\[\frac{P_B-P_C}{\delta z}=-\frac{\partial P}{\partial z}=\rho g\]
%on trouve 
%\[F_P=-g\left(\frac{\delta z}{\delta x}\right)_P\]
%On aurait une relation équivalente pour la direction ($Oy$), la
%force de pression horizontale vaut donc finalement
%\[\v F_P=-\frac{1}{\rho}\vl{grad}_Z(P)=-g\cdot\vl{grad}_P(Z)\]
%On utilise plutôt le gradient de pression horizontal avec la pression au
%niveau de la mer, et le gradient isobare de l'altitude $Z$ ou du
%{\em géopotentiel} $gZ$ dans l'atmosphère libre.
%Sur une carte d'une surface isobare, les lignes à $Z$ constant sont des
%{\em isohypses}. La force de pression est donc dirigée des hautes vers les
%basses valeurs de $Z$, perpendiculairement aux isohypses.

\sk
Les variations verticales de la pression sont données par l'équilibre hydrostatique comme indiqué dans les chapitres précédents. Cette propriété a deux conséquences importantes pour les variations de pression horizontales donc la force de pression horizontale. 
\begin{finger}
\item Une conséquence de cet équilibre est que la pression à une altitude $z$ est proportionelle à la masse de la colonne d'air située au dessus de $z$. Une diminution ou augmentation de cette masse dûe aux mouvements d'air horizontaux change donc la pression en dessous, en particulier à la surface.
\item D'autre part, même pour une masse d'air totale de la colonne constante, des écarts de température horizontaux peuvent créer des gradients de pression en changeant la répartition verticale de cette masse. L'équation hypsométrique donne l'épaisseur d'une colonne d'air de masse constante entre deux niveaux de pression donnés (voir chapitres précédents)~: la pression décroît plus vite dans une couche d'air froid que dans une couche d'air chaud. Une variation horizontale de température induit donc une force de pression horizontale selon ce principe.
\end{finger}

%\begin{equation}
%  g\cdot(Z_2-Z_1)=R<T>\ln{\frac{P_1}{P_2}}
%  \label{eq:hypso}
%\end{equation}
%La différence entre les forces de pressions aux niveaux 1 et 2 sera donc: \[\v F_{P_2}-\v F_{P_1}=-R\cdot\vl{grad}<T>\cdot\ln{\frac{P_1}{P_2}}\]




\newpage
\section{\'Equation fondamentale}
\sk
L'équation complète de la quantité de mouvement pour les mouvements atmosphériques, qui résulte de l'application de la seconde loi de Newton, s'écrit~:
%\begin{equation}
%\ddf{\v V_r}{t} + 2 \, \v \Omega\wedge\v V_r = \v g + \v F_P + \vl{Fr} 
\[   
\ddf{\v V_r}{t} = \v g + \v F_P + \v F_C + \vl{Fr}
\] %\frac{1}{\rho}\vl{grad}P  \]
%  \label{eq:qtemvt}
%\end{equation}
Le terme~$\vl{Fr}$ représente les forces de friction qui sont négligées sauf lorsqu'on se trouve à proximité de la surface.

\section{Expression des forces en coordonnées sphériques}
\paragraph{L'équation fondamentale de la dynamique} dans le référentiel tournant
\[
\vec{\gamma}_r = -2 \, \vec{\Omega} \wedge \vec{U}_r + \vec{g} - \dfrac{\vec{\nabla}p}{\rho} + \vec{Fr}
\]

\paragraph{Composantes de l'acc\'el\'eration}
\[
\vec{\gamma}_r=\left[\begin{array}{ccc} u_t & - \dfrac{u\,v\tan\phi}{a} & + \dfrac{u\,w}{a}\\ & & \\ v_t & + \dfrac{u^2\tan\phi}{a} & + \dfrac{v\,w}{a}\\ & & \\ w_t & - \dfrac{u^2+v^2}{a} & \\ \end{array}\right] \qquad \text{notation} \quad u_t = \derd{u}{t}
\]

\paragraph{Composantes de la \ofg{force} de Coriolis}
\[
-2\,\vec{\Omega}\wedge\vec{U}_r=-2 \left[ \begin{array}{c} 0\\ \Omega\,\cos\phi\\ \Omega\,\sin\phi \end{array} \right] \wedge \left[ \begin{array}{c} u\\ v\\ w \end{array} \right] = \left[ \begin{array}{c} 2 \, \Omega \, ( v\sin\phi-w\cos\phi )\\ -2 \, \Omega \, u\sin\phi\\ 2 \, \Omega \, u\cos\phi \end{array} \right]
\]


\newpage
\section{Passage aux coordonnées sphériques}
\sk
Vitesse dans le référentiel tournant $\vec{U_r} = u\,\vec{i} + v\,\vec{j} + w\,\vec{k} $
\noindent Accélération (dérivée lagrangienne)
\[
\gamma_r = \derd{\vec{U_r}}{t}=\derd{u}{t}\,\vec{i}+\derd{v}{t}\,\vec{j}+\derd{w}{t}\,\vec{k} +u\,\derd{\vec{i}}{t}+v\,\derd{\vec{j}}{t}+w\,\derd{\vec{k}}{t}
\]

\figsup{0.35}{0.2}{decouverte/cours_dyn/didt1.png}{decouverte/cours_dyn/didt2.png}{Axe méridional, axe zonal}{fig:spher}

\sk
Décomposition sur les trois axes (zonal, méridien et vertical)
\[
\derd{\vec{i}}{t} = \left[ \derd{\vec{i}}{t} \right]_u + \left[ \derd{\vec{i}}{t} \right]_v + \left[ \derd{\vec{i}}{t} \right]_w
\]
\noindent Axe vertical
\[
\left[ \derd{\vec{i}}{t} \right]_w = \left[ \derd{\vec{j}}{t} \right]_w = \left[ \derd{\vec{k}}{t} \right]_w = \vec{0}
\]
\noindent Axe méridien \textcolor{magenta}{(aisé à expliquer)}
\[
\textcolor{magenta}{\left[ \derd{\vec{j}}{t} \right]_v = - \derd{\phi}{t} \, \vec{k} = - \dfrac{v}{a} \, \vec{k}}
\]
\[
\left[ \derd{\vec{i}}{t} \right]_v = \vec{0}
\qquad \qquad
\left[ \derd{\vec{k}}{t} \right]_v = + \dfrac{v}{a} \, \vec{j}
\]
\noindent Axe zonal
\[
\vec{j} = -\sin\phi \, \vec{m} + \cos\phi \, \vec{n} \qquad \vec{k} =  \cos\phi \, \vec{m} + \sin\phi \, \vec{n} \qquad \textrm{avec} \qquad \vec{m} = -\sin\phi \, \vec{j} + \cos\phi \, \vec{k}
\]
\[
\left[ \derd{\vec{n}}{t} \right]_u = \vec{0} \qquad \left[ \derd{\vec{m}}{t} \right]_u = \derd{\lambda}{t} \, \vec{i}
\]
\[
\left[ \derd{\vec{j}}{t} \right]_u = -\sin\phi \, \left[ \derd{\lambda}{t} \, \vec{i} \right] = \dfrac{-u}{a} \, \tan\phi \, \vec{i}
\]
\[
\left[ \derd{\vec{k}}{t} \right]_u =  \cos\phi \, \left[ \derd{\lambda}{t} \, \vec{i} \right] = \dfrac{u}{a} \, \vec{i}
\]
\[
\left[ \derd{\vec{i}}{t} \right]_u = - \derd{\lambda}{t} \, \vec{m} = \dfrac{u}{a} \tan\phi \, \vec{j} - \dfrac{u}{a} \,\vec{k}
\]



\newpage
\section{Eulérien vs. Lagrangien}
\sk
Comment caractériser un écoulement ?

\sk
\paragraph{Point de vue lagrangien} Le plus intuitif~:~Suivre les particules le long de leur trajectoire.
\paragraph{Point de vue eulérien} Le plus pratique~:~Suivre le courant depuis un point géométrique. Les points sont fixés ce qui est plus aisé en première approche pour modéliser l'écoulement sur une grille.
\centers{Variations lagrangiennes \quad = \quad Variations eulériennes \quad + \quad Terme d'advection}
Le terme d'advection transport concentre le caractère non-linéaire de la dynamique atmosphérique

\sk
On passe de l'un à l'autre des formalismes avec la formule de la dérivée d'une fonction composée~$\mathcal{F}[x(t)]$ où~$x$ est la position.
\[
\underbrace{\derd{\mathcal{F}}{t}}_{\text{En suivant la particule}}
= 
\underbrace{\Dp{\mathcal{F}}{t}}_{\text{En un point géométrique}} 
+ 
\underbrace{\left(\v U \cdot \v \nabla \right)\,\mathcal{F} }_{\text{Lié au déplacement de la particule}}
\]



\section{Equations complètes du mouvement}
\sk
L'équation fondamentale de la dynamique des fluides géophysiques 
en projection sur les coordonnées sphériques avec l'approximation de couche mince
s'écrit finalement

\begin{center}
\begin{tabular}{ccccccccc}
%%%%%%
\textcolor{blue}{$\ddf{u}{t}$} & 
\textcolor{brown}{$-\dfrac{uv\tan\phi}{a}$} & 
$+\dfrac{uw}{a}$ & 
= & 
\textcolor{red}{$2\Omega\sin\phi v$} & 
$-2\Omega \cos\phi w$ & 
\textcolor{green!75!black}{$-\dfrac{1}{\rho}\Dp{p}{x}$} & 
& 
$+Fr_x$\\
~\\
%%%%%%
\textcolor{blue}{$\ddf{v}{t}$} & 
\textcolor{brown}{$+\dfrac{u^2\tan\phi}{a}$} & 
$+\dfrac{vw}{a}$ & 
= & 
\textcolor{red}{$-2\Omega\sin\phi u$} &
&
\textcolor{green!75!black}{$-\dfrac{1}{\rho}\Dp{p}{y}$} &
&
$+Fr_y$\\
~\\
%%%%%%
$\ddf{w}{t}$ & 
$-\dfrac{u^2+v^2}{a}$ & 
&
=&
$2\Omega\cos\phi u$ & 
& 
\textcolor{green!75!black}{$-\dfrac{1}{\rho}\Dp{p}{z}$} & 
\textcolor{green!75!black}{$ -g$} & 
$+Fr_z$\\ 
\end{tabular}
\end{center}

\sk
Les termes s'interprètent comme suit
\begin{citemize}
\item{Pression} \textcolor{green!75!black}{$\bullet$} 
\item{Coriolis} \textcolor{red}{$\bullet$} où l'on définit~$f = 2 \, \Omega \, \sin\phi$
\item{Inertiels (sphéricité)} \textcolor{brown}{$\bullet$}
\item{Inertials (accélération)} \textcolor{blue}{$\bullet$}
\end{citemize}
Les termes en noir sont liés au déplacements verticaux et sont en général négligeables quand on considère la circulation générale de l'atmosphère et de l'océan.


\newpage 
\section{\'Echelles}
\sk
Tous les termes de l'équation du mouvement n'ont pas la même importance lorsqu'on considère des mouvements atmosphériques de grande échelle. On définit donc des échelles caractéristiques du mouvement étudié. Pour simplifier, on choisit des échelles qui sont des puissances de 10.
\begin{description}
\item[longueur] Les échelles de longueur sont $L$ sur l'horizontale, et $H$ sur la verticale. Pour des mouvements qui s'étendent sur la hauteur de la troposphère, $H\sim 10$~km. $L$ peut varier beaucoup, mais l'échelle dite synoptique $L=$1000~km, qui est celle des perturbations des latitudes moyennes, est d'un intérêt particulier. La dernière échelle de longueur est celle du rayon de la Terre~$a$, qui est de l'ordre de 10000~km. 
\item[vitesse] Les échelles de vitesse horizontale et verticale sont notées $U$ et $W$. On a typiquement $U$=10~m~s$^{-1}$ dans l'atmosphère. Le rapport d'aspect du mouvement impose d'autre part que $W\le UH/L$.
\item[temps] L'échelle de durée du mouvement est construite à partir de celles de vitesse et de longueur: $T=L/U$. L'autre échelle de temps est celle liée à la rotation de la Terre, qui apparait dans le terme de Coriolis.
\item[variables thermodynamiques] Les variations des variables thermodynamiques $P,T,\rho$ sur la verticale sont celles des profils moyens donnés en introduction. En un point donné, les variations à l'échelle synoptique $\delta P,\delta T,\delta\rho$ sont de l'ordre de 1\% de la valeur moyenne.
\end{description}




\newpage 
\section{\'Echelles (évaluation)}

	\sk \subsection{\'Echelles (mouvement vertical)}
	\sk
L'ordre de grandeur des termes de l'équation du mouvement 
%\ref{eq:qtemvt} 
projetée sur la verticale (dirigée suivant \v k) est indiqué dans la table \ref{tab:vqmouv}. On voit que l'équilibre hydrostatique est vérifié avec une très bonne approximation\footnote{On peut noter qu'on vérifie également l'équilibre hydrostatique entre des anomalies de densité et des anomalies de variations de pression sur la verticale. Les termes $\rho g$ et $\partial P/\partial z$ sont alors cent fois plus faibles que pour l'état moyen, mais toujours supérieurs aux autres termes de l'équation.}. Notamment la composante verticale de la force de Coriolis~$\v F_C$ est négligeable devant~\v g et les forces de pression. Le seul autre terme qui peut devenir important est l'accélération relative~$dw/dt$, lors de mouvements verticaux intenses à petite échelle, comme dans un nuage d'orage ou près de topographie raide.  
%\begin{equation}
%  \frac{\partial P}{\partial z}=-\rho  g
%  \label{eq:hydro}
%\end{equation}

\begin{table}
  \centering
  \begin{tabular}{ccccccc}
    \hline
    Équation & $dw/dt$ & $-2\Omega u\cos\phi$ & $-\left(u^2+v^2\right)/a$ & = &
    $-\rho^{-1}\partial P/\partial z$ & $-g$ \\
    Échelle & $UW/L$ & $fU$ & $U^2/a$ && $P_0/(\rho_0H)$ & $g$ \\
    m.s\md & 10$^{-7}$ & 10$^{-3}$ & 10$^{-5}$  && 10 & 10 \\ 
    \hline
  \end{tabular}
  \caption{\emph{Analyse d'échelle de l'équation du mouvement vertical (avec
  $L$=1000~km et $W$=1~cm.s\mo).}}
  \label{tab:vqmouv}
\end{table}




	\sk \subsection{\'Echelles (mouvement horizontal)}
	\sk
Le détail de l'équation horizontale projetée en coordonnées sphériques est donné dans la table \ref{tab:hqmouv} pour $L$=1000~km. Sur les composantes horizontales (\v i, \v j), l'expression de la force de Coriolis se réduit aux contributions des mouvements horizontaux dans la mesure où~$W<<U$ pour des mouvements d'échelle supérieure à 10~km. 
\[\v F_C = \binom{f \, v}{-f \, u} \qquad \textrm{ou} \qquad \v F_C = -f \, \v k \wedge \v V_h \]
où $\v V_h = u \v i + v \v j$ est la vitesse horizontale et 
\[ \boxed{ f = 2 \, \Omega \, \sin \phi } \]
est appelé \voc{facteur de Coriolis}. Aux moyennes latitudes ($\phi=45$\deg), la valeur de~$f$ est environ~$10^{-4}$~s$^{-1}$. 
%Les composantes de la force de Coriolis sont \[\v F_C=-2\Omega\left(\begin{array}{c}0\\\cos\phi\\\sin\phi\end{array}\right) \wedge\left(\begin{array}{c}u\\v\\w\end{array}\right) =-2\Omega\left(\begin{array}{c}w\cos\phi-v\sin \phi\\u\sin \phi\\-u\cos \phi\end{array}\right)\]
%\footnote{Pour des mouvements de
%type ``chute libre'', la vitesse verticale $w$ domine. On peut alors mettre en
%évidence une déviation vers l'est, mais qui reste très faible (de l'ordre de
%1cm pour 80m de chute).} 

\begin{table}
  \centering
  \begin{tabular}{cccccccc}
    \hline
    Équation-$x$ & $\frac{du}{dt}$ & $-2\Omega v\sin\phi$ & $+2\Omega
    w\cos\phi$ & $+\frac{uw}{a}$ & $-\frac{uv\tan\phi}{a}$ &=&
    $-\frac{1}{\rho}\frac{\partial P}{\partial x}$ \\
    Équation-$y$ & $\frac{dv}{dt}$ & $+2\Omega u\sin\phi$ &&         
                $+\frac{vw}{a}$ & $+\frac{u^2\tan\phi}{a}$ &=&
    $-\frac{1}{\rho}\frac{\partial P}{\partial y}$ \\
    Échelles & $U^2/L$ & $fU$ & $fW$ & $UW/a$ & $U^2/a$ && $\delta P/(\rho L)$
    \\
    m.s\md & 10$^{-4}$ & 10$^{-3}$ & 10$^{-6}$ & 10$^{-8}$ & 10$^{-5}$ &&
    10$^{-3}$ \\
    \hline
  \end{tabular}
  \caption{\emph{Analyse en ordre de grandeur de l'équation du mouvement
  horizontale.}}
  \label{tab:hqmouv}
\end{table}

\sk
Sur un plan horizontal, les termes restants de l'équation du mouvement sont ainsi:
%\begin{equation}
\[  \frac{d\v V_h}{dt}+f\v k\wedge\v V_h=\v F_P  \]
%  \label{eq:hqmouv}
%\end{equation}
avec $\v V_h$ la vitesse horizontale, et $\v F_P$ les forces de pression horizontales massiques. Pour évaluer lequel des deux termes à gauche domine, on définit le \voc{nombre de Rossby} $\mathcal{R}$, rapport entre accélération relative et de Coriolis
\[ \mathcal{R} = \frac{U^2/L}{f\,U} = \frac{U}{f\,L} \]
Avec $f$=10$^{-4}$~s$^{-1}$ aux moyennes latitudes et $U$=10~m~s$^{-1}$, on a $\mathcal{R}=0.1$ aux grandes échelles de la circulation terrestre ($L$=1000~km), donc Coriolis domine. Au contraire, à une échelle plus petite de $L$=10~km, $\mathcal{R}=10$ et Coriolis devient négligeable.




\newpage
\section{Grands équilibres}
\sk
Suivant les termes dominants, on peut définir un certain nombre d'équilibres (stationnaires) ou de modèles / équations (pouvant servir à la prédiction de l'écoulement au cours du temps):
\begin{description}
\item{Equilibre hydrostatique} \DDD{\bullet} 
\item{Equilibre g\'eostrophique} \DDD{\bullet}\AAA{\bullet}
\item{Equilibre cyclostrophique} \DDD{\bullet}\CCC{\bullet}
\item{Equilibre du vent gradient} \DDD{\bullet}\AAA{\bullet}\CCC{\bullet}
\item{Modèle quasi-g\'eostrophique} \DDD{\bullet}\AAA{\bullet}\BBB{\bullet}
\item{Equations primitives} \DDD{\bullet}\AAA{\bullet}\BBB{\bullet}\CCC{\bullet}
\end{description}

\mk
\paragraph{Nombre de Rossby} Le nombre de Rossby permet d'évaluer l'importance relative de l'accélération de Coriolis, impulsée par la rotation de la planète, par rapport aux autres mouvements de rotation. Il permet de savoir si l'on se trouve dans le domaine de validité de l'équilibre géostrophique ou de l'équilibre cyclostrophique
\[
R_o=\f{\text{accélération horizontale (inertielle + sphéricité)}}{\text{accélération de Coriolis}}\qquad\boxed{R_o=\frac{U}{L\,\Omega}}
\]
\begin{table}[h!]
\begin{tabular}{cccc}
$R_o \ll 1$ & \DDD{\bullet}\AAA{\bullet} & Equilibre g\'eostrophique & [Terre, Mars]\\
$R_o \gg 1$ & \DDD{\bullet}\CCC{\bullet} & Equilibre cyclostrophique & [Vénus, Titan]\\
$R_o$~tous & \DDD{\bullet}\AAA{\bullet}\CCC{\bullet} & Equilibre du vent gradient & [Toutes]\\
~ & & & \\
$R_o \ll 1$ & \DDD{\bullet}\AAA{\bullet}\BBB{\bullet} & Modèle quasi-g\'eostrophique & [Terre, Mars]\\
$R_o$~tous & \DDD{\bullet}\AAA{\bullet}\BBB{\bullet}\CCC{\bullet} & Equations primitives  & [Toutes]
\end{tabular}
\end{table}

\mk
Sur les planètes à rotation rapide, l'équilibre géostrophique est le développement des équations du mouvement à l'ordre 1 en le nombre de Rossby, qui décrit un écoulement bidimensionnel, stationnaire et non divergent. A un ordre supérieur en $\textrm{Ro}$, l'évolution lente de la fonction de courant géostrophique peut être diagnostiquée par un nouvel équilibre dit quasi-géostrophique (QG). Couplé à l'équation de conservation de la vorticité potentielle de Rossby, le modèle approché QG a permis à Charney dans les années 50 de faire fonctionner sur un ordinateur le premier modèle de prévision numérique du temps.


\end{document}
