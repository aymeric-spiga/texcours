\sk
Le Soleil qui se situe à une distance considérable dans le vide spatial nous procure une sensation de chaleur. De même, placer sa main sur le côté d'un radiateur en fonctionnement sans le toucher procure une sensation de chaleur instantanée qui ne peut être attribuée à un transfert convectif entre le radiateur et la main. Cet échange de chaleur est attribué au contraire à l'émission d'ondes électromagnétiques par la matière du fait de sa température; on parle d'émission de \voc{rayonnement thermique}. Tous les corps émettent du rayonnement thermique. La transmission de cette énergie entre une source et une cible ne nécessite pas la présence d'un milieu intermédiaire matériel. 
%Le but de cette section est d'en étudier les principales propriétés.
