\documentclass[
	a4paper,
	DIV16,
	10pt,
	cleardoublepage=empty,
        twoside=yes,
        BCOR=8.25mm
	]{scrbook}

%%% MOI
\usepackage{scrpage2}
\usepackage{my_latex}
\usepackage{patch_french}
\setcapindent{1em} %%% parce que sinon l'indentation "Figure ... -- texte" est trop grande

%%% BANDEAU
\deftripstyle{Vide}[0pt][.4pt]
{}{}{}
{}{}{}
\deftripstyle{Bandeau}[0pt][.4pt]
{Sciences de l'atmosphère et de l'océan}{}{UPMC}
{2C046 - année 2015-2016}{}{\pagemark} \pagestyle{scrheadings} \pagestyle{Bandeau}

%%% DEBUT
\begin{document}
\frontmatter

%%% PAGE DE GARDE
\subject{2C046 -- Sciences de l'atmosphère et de l'océan}
\title{Partie ``Rayonnement, thermodynamique \\ et dynamique de l'atmosphère''}
\subtitle{Cours niveau licence. Année universitaire 2015-2016.}
\author{~\\ ~\\
\Large Partie enseignée par Jean-Baptiste MADELEINE\\
\small Laboratoire de Météorologie Dynamique (Institut Pierre-Simon Laplace)
}
\date{}
\publishers{
\includegraphics[height=1.5cm]{decouverte/logos/UPMC_cart-blanc-Q_7504-703-3.png}\\
Contact: \email{jean-baptiste.madeleine@upmc.fr}
}

%%%%% decommenter pour le mode book
\dedication{Note de l'auteur: Les chapitres 1, 2, 3 et 5 consistent en de multiples 
réorganisations, ajouts et modifications sur des notes existantes de Francis Codron que je souhaite 
remercier pour son aide. Le chapitre 4 est entièrement original. Je remercie Jean-Baptiste Madeleine 
pour ses remarques constructives.\\ ~\\ ~\\ Dans l'éventualité où le lecteur trouverait des erreurs 
ou imprécisions dans ce cours, il est cordialement invité à les signaler à l'auteur Aymeric Spiga à l'adresse 
\url{aymeric.spiga@upmc.fr}} \maketitle \tableofcontents

%%%%%%%%%%%%%%%%%%%%%%%%%%%%%%%%%%%%%%%%%%%%%%%%%%%%%%%%%%%%%%%
\mainmatter
%%%%%%%%%%%%%%%%%%%%%%%%%%%%%%%%%%%%%%%%%%%%%%%%%%%%%%%%%%%%%%%

\figside{0.75}{0.35}{decouverte/cours_meteo/joussaume_pluri.png}{Schéma du système climatique présentant les différentes composantes du système : atmosphère, océans, cryosphère, biosphère et lithosphère, leurs constantes de temps et leurs interactions en termes d’échanges d’énergie, d’eau et de carbone. Source~:~S.~Joussaume \emph{in} Le Climat à Découvert, CNRS éditions, 2011}{fig:pluri}

\figsup{0.48}{0.35}{\figfrancis/WH_vert_struct}{\figpayan/LP211_Chap1_Page_03_Image_0001.png}{[Gauche] Structure verticale de la pression, la densité et du libre parcours moyen des molécules (distance moyenne parcourue par une molécule avant de subir un choc sur une autre molécule). Noter l'échelle logarithmique en abscisse : les variations des quantités selon le logarithme de~$z$ sont approximativement des droites, donc les variations avec $z$ sont proches d'exponentielles. [Droite] Plus haut dans l'atmosphère, la variation verticale de la pression est dépendante pour plusieurs ordres de grandeur avec l'activité solaire. Les courbes indiquées correspondent respectivement à une très faible activité solaire (température de la thermopause de 600 K) et une très forte activité solaire (température de la thermopause de 2000K).}{fig:presvert}

\figsup{0.49}{0.3}{\figfrancis/WH_stdatm}{\figpayan/LP211_Chap1_Page_05_Image_0001.png}{Structure verticale idéalisée de la température correspondant au profil moyenné global annuel. Voir figure~\ref{fig:presvert} pour la distinction entre figure de gauche et figure de droite}{fig:tempvert}

\figside{0.7}{0.35}{decouverte/cours_dyn/absorption.png}{Spectres d'absorption de l'atmosphère en fonction de la longueur d'onde. [Haut] Courbes d'émittance normalisée de corps noirs à 5780~K (rayonnement solaire) et 255~K (rayonnement terrestre). [Bas] Coefficients d'absorption (en~$\%$) entre le sommet de l'atmosphère et la surface. Les principaux gaz responsables de l'absorption à différentes longueurs d'onde sont indiqués en bas. Source: McBride and Gilmour, An Introduction to the Solar System, 2004 ; d'après Goody and Yung, Atmospheric radiation, 1989}{fig:atmspectrum}

\chapter{Rayonnement électromagnétique et émission thermique} \label{chap:rad} 
\dictum[The Beatles, 1970]{Here comes the sun, and I say it's allright}

		\bk
Comment déterminer les processus dynamiques, physiques, chimiques à l'oeuvre dans l'atmosphère ? Il faut commencer par faire le point sur les sources d'énergie pour l'atmosphère, les océans et la surface. La principale source d'énergie pour l'atmosphère et le système climatique de la Terre est le Soleil\footnote{Ce n'est pas le cas pour les géantes gazeuses Jupiter et Saturne où il existe un flux de chaleur interne significatif en regard du flux d'énergie reçu du Soleil. Ce flux est un reste de la contraction gravitationnelle au cours de la formation de ces géantes gazeuses.}. La figure~\ref{fig:flux} montre que d’autres sources existent mais en quantité réduite : l'énergie reçue par la géothermie, ou par les activités humaines, est~$4$ ordres de grandeur plus faible que la source solaire; celle reçue des étoiles~$8$ ordres de grandeur plus faible. L’énergie solaire est transmise principalement à la Terre au moyen du rayonnement électromagnétique~: on qualifie cette énergie de \voc{radiative}. %L'objet de ce chapitre est de s'intéresser au rayonnement électromagnétique et plus particulièrement au phénomène d'émission thermique.

\figside{0.75}{0.18}{decouverte/cours_meteo/fluxenergsurf.png}{Ordres de grandeur des flux énergétiques reçus à la surface de la Terre. Source~:~P.~von Balmoos \emph{in} Le Climat à Découvert, CNRS éditions, 2011}{fig:flux}



\mk \section{Description générale du rayonnement électromagnétique}

	\sk \subsection{Spectre électromagnétique}

		\sk
Les échanges d'\voc{énergie radiative} se font à distance par le biais du \voc{rayonnement électromagnétique}. Le rayonnement électromagnétique est composé d'une superposition d'ondes monochromatiques de longueurs d'onde~$\lambda$ se propageant à la vitesse de la lumière~$c$ (dans le vide~$c=3 \times 10^8$~m~s$^{-1}$). Le rayonnement électromagnétique parcourt la distance Terre-Soleil en $8$~minutes; à l'échelle des processus atmosphériques terrestres, la propagation des ondes électromagnétiques est si rapide qu'elle peut être considérée en première approximation comme immédiate. 

\sk
Les ondes composant le rayonnement électromagnétique peuvent être caractérisées indifféremment par leur \voc{longueur d'onde}~$\lambda$, leur \voc{fréquence}~$\nu = c / \lambda$ ou leur \voc{nombre d'onde}\footnote{Le nombre d'onde est souvent exprimé en cm$^{-1}$. Pour obtenir~$\overline{\nu}$ dans cette unité à partir de~$\lambda$ en microns, on utilise~$\overline{\nu} = 10^{4} / \lambda$.}~$\overline{\nu} = 1 / \lambda$. L'ensemble de ces ondes constitue le \voc{spectre} électromagnétique. Selon le principe de De Broglie, à chaque onde électromagnétique de fréquence~$\nu$ est associée une particule sans masse nommée \voc{photon} dont l'énergie est~$h \, \nu$ où $h = 6.63 \times 10^{-34}$~J~s est appelée la constante de Planck. Cette énergie est souvent exprimée en électron-volts eV ($1$~eV~$= 1.6 \times 10^{-19}$~J~s).

\sk
Le rayonnement visible occupe une bande très étroite du spectre aux longueurs d'ondes comprises entre 0.4 et 0.76~$\mu$m [figure~\ref{fig:spectrum}]. Lorsque l'on considère des longueurs d'ondes plus courtes (c'est-à-dire des fréquences plus élevées) que le rayonnement visible, on passe dans le domaine du rayonnement ultraviolet, puis celui des rayons X et gamma~$\gamma$. Lorsque l'on considère des longueurs d'ondes plus grandes (c'est-à-dire des fréquences plus faibles) que le rayonnement visible, on passe dans le domaine du rayonnement infrarouge, puis celui des micro-ondes et des ondes radio. Les photons les plus énergétiques correspondent aux rayons X; les moins énergétiques aux ondes radio.

\figsup{1}{0.1}{\figpayan/LP211_Chap2_Page_09_Image_0001.png}{\figpayan/LP211_Chap2_Page_09_Image_0002.png}{Classification du rayonnement électromagnétique en fonction de la longueur d'onde. On rappelle que 1~$\mu$m (micron) correspond à $10^{-6}$~m et 1~nm (nanomètre) correspond à $10^{-9}$~m.}{fig:spectrum}


	\sk \subsection{Mesures quantitatives~: grandeurs caractéristiques}

		\sk
La quantité de rayonnement émise par une source de rayonnement, ou reçue par une cible, dépend des paramètres~:
\begin{citemize}
\item longueur d'onde~$\lambda$ ;
\item temps d'exposition~$t$ ;
\item surface~$S$ de l'objet (source ou cible) ;
\item direction dans l'espace considérée, que l'on repère par l'angle~$\beta$ entre le rayonnement incident (ou émis) et la normale à la surface (appelé angle zénithal) ;
\item portion d'espace considérée, exprimée par un \voc{angle solide} $\omega$, l'équivalent bidimensionnel d'un angle\footnote{De même qu'un angle en radians est la longueur d'un arc de cercle divisée par le rayon, le stéradian est la surface d'une portion de sphère divisée par le rayon au carré. On a donc $4\pi$ stéradians sur tout l'espace: voir figure \ref{fig:radiance}} ;
\item propriétés physico-chimiques de l'objet, par exemple sa température (voir section~\ref{corpsnoir}).
\end{citemize}
On exprime cette quantité de rayonnement reçue ou émise sous la forme d'une énergie totale~$E\e{r}$ en Joules (J). Que l'on considère une source ou une cible, $E\e{r}$ est l'énergie transmise par le rayonnement (radiative).

\sk
L'énergie totale~$E\e{r}$ est une grandeur intégrée peu utilisée en pratique en sciences de l'atmosphère. On lui préfère les quantités décrites ci-dessous\footnote{Les noms anglais de ces quantités sont respectivement \emph{radiant flux} pour~$\Phi$, \emph{irradiance} pour~$F$, \emph{radiance} pour~$L$} qui décrivent la quantité de rayonnement émise ou reçue par unité de temps, surface, longueur d'onde, \ldots
\begin{finger}
\item \underline{unité de temps} Le \voc{flux énergétique}~$\Phi$ est l'énergie totale~$E\e{r}$ par unité de temps~$t$ (par seconde) $$ \Phi = \ddf{E\e{r}}{t} $$ C'est une puissance exprimée en Watts (W~$\equiv$~J~s$^{-1}$). Le flux énergétique~$\Phi$ est intégré sur toutes les longueurs d'onde, toutes les directions d'espace et sur l'intégralité de la surface de la source ou cible.
\item \underline{unité de temps + unité de surface} La \voc{densité de flux énergétique}~$F$ est le flux énergétique~$\Phi$ par unité de surface~$S$ de la source/cible $$ \boxed{ F = \ddf{\Phi}{S} } = \f{\dd^2 E\e{r}}{\dd S \, \dd t} $$ C'est un \voc{flux net} exprimé en~W~m$^{-2}$. On l'appelle également \voc{émittance}~$M$ pour une source et \voc{éclairement}~$E$ pour une cible. Cette quantité est intégrée sur toutes les longueurs d'onde et toutes les directions d'espace.
\item \underline{unité de temps + unité de surface + direction fixée} La \voc{luminance énergétique}~$L$ est la densité de flux énergétique dans une direction donnée de l'espace repérée par un angle~$\beta$ $$ L = \f{\dd F}{\cos\beta \, \dd \omega} = \f{\dd^2 \Phi}{\cos\beta \, \dd \omega \, \dd S} = \f{\dd^3 E\e{r}}{\cos\beta \, \dd \omega \, \dd S \, \dd t} $$ 
par unité d'angle solide~$\omega$ (figure \ref{fig:radiance}). C'est une quantité surtout utilisée pour les sources, parfois appelée radiance. La surface considérée~$\sigma$ est perpendiculaire à la direction d'émission~: on la relie à~$S$ par~$\dd \sigma = \cos\beta \, \dd S$. La luminance énergétique~$L$ en~W~m$^{-2}$~sr$^{-1}$ est intégrée sur toutes les longueurs d'onde.
\item \underline{longueur d'onde fixée (grandeurs spectrales)} Le flux énergétique \voc{spectral} ou monochromatique~$\Phi_{\lambda}$ est le flux énergétique~$\Phi$ par unité de longueur d'onde~$\lambda$ $$ \Phi_{\lambda} = \ddf{\Phi}{\lambda} = \f{\dd^2 E\e{r}}{\dd \lambda \, \dd t} $$ Cette quantité en~W~m$^{-1}$ est intégrée sur toutes les directions d'espace et sur toute la surface de la source ou cible. On peut également définir des équivalents spectraux~$F_{\lambda}$ et~$L_{\lambda}$ pour les quantités~$F$ et~$L$ $$ \boxed{F_{\lambda} = \ddf{F}{\lambda}} \qquad\qquad L_{\lambda} = \ddf{L}{\lambda} $$ et des quantités spectrales~$\Phi_{\nu}$, $F_{\nu}$ et $L_{\nu}$ à partir de la fréquence~$\nu$ $$ \Phi_{\nu} = \ddf{\Phi}{\nu} \qquad\qquad F_{\nu} = \ddf{F}{\nu} \qquad\qquad L_{\nu} = \ddf{L}{\nu} $$ Les quantités spectrales définies à partir de la fréquence sont parfois plus avantageuses, dans la mesure où la fréquence est indépendante du milieu matériel transparent où l'onde matérielle se propage\footnote{La longueur d'onde~$\lambda$ dépend de l'indice de réfraction~$n$ du milieu (pour l'air, $n$ est proche de~$1$) et de la longueur d'onde dans le vide~$\lambda_0$.}. Attention, les unités des quantités spectrales dépendent de la quantité référente : ainsi $L_{\lambda}$ est en~W~m$^{-3}$~sr$^{-1}$ et $L_{\lambda}$ est en~W~m$^{-2}$~sr$^{-1}$~s.
\end{finger}

\figun{0.6}{0.15}{\figfrancis/lum_emit.pdf}{Schéma montrant l'émittance~$M$ et la luminance~$L$ d'un élément de surface $dS$ d'une source. $M$ est l'intégrale du flux dans toutes les directions. $L$ est le flux émis dans une certaine direction par unité de surface perpendiculaire.}{fig:luminance}

\begin{figure} \begin{center} \input{\figfrancis/luminance.pdftex_t} \end{center} \caption{\emph{ Schéma en coordonnées sphériques de la luminance $L$ de l'élément de surface $dS$ d'une source située dans un plan $(Oxy)$. La luminance est définie pour chaque direction repérée par les angles $\theta$ et $\varphi$. L'angle solide élémentaire autour d'une direction donnée vaut $\dd \omega = \sin\theta \, \dd\theta \, \dd\varphi$ (rapport entre surface hachurée et $r^2$). 
%La relation avec le flux énérgétique~$\Phi$ émis par la source est $L = \dd^2 \Phi / \left( \dd\omega \, \dd S \, \cos\theta \right)$.
}} \label{fig:radiance} \end{figure}

\sk
\subsubsection{Exemples d'application}

\sk
Dans ce qui précède, on part de la quantité la plus intégrée possible, à savoir l'énergie totale~$E\e{r}$, pour parvenir par dérivation à des quantités moins complexes à appréhender en pratique. Le chemin inverse se fait par intégration (sommation continue). On considère ici quelques exemples illustratifs. 
\begin{finger}
\item Pour reprendre la situation de la figure~\ref{fig:luminance}, supposons que l'on dispose d'informations sur la luminance énergétique~$L$ d'une source plane de surface élémentaire~$\dd S$, c'est-à-dire la quantité de rayonnement émise dans chaque direction de l'espace. Afin de connaître le flux net~$F$ dans tout l'espace (que l'on peut noter également émittance~$M$ puisque l'on considère une source), il suffit de l'écrire comme une intégrale de la luminance sur toutes les directions d'un demi-espace: $$ F = \int_{2\,\pi} L \, \cos\beta \, \dd \omega $$ où $2\,\pi$ représente l'intégration sur un demi-espace. %Deux cas particuliers sont intéressants. Dans la limite d'un rayonnement rasant~$\beta=\pi/2$, la contribution au flux net est nulle. Par ailleurs, s
Si la luminance~$L$ est indépendante de la direction, c'est-à-dire que le rayonnement est \voc{isotrope}, l'intégration donne simplement~$F=\pi L$. Dans ce cas on parle d'une \voc{source lambertienne}.
%C'est le cas d'un réflecteur de Lambert.
%Jean-Henri Lambert (1728-1777) a observé que l’énergie émise par certaines sources (parmi toutes les types de sources à sa disposition) anisotropes diminue comme le cosinus de l'angle θ, autour de la direction perpendiculaire à la surface de la source. Cette variation de l'énergie émise est observée lorsque nous mesurons l'énergie thermique rayonnée par un orifice percé dans un four (ce qui nous ramène au corps noir défini plus loin), isolé thermiquement et dont la température interne est supérieure à la température externe. Dans ce contexte, l'orifice est appelé un émetteur Lambert et ne balaye un espace que de stéradian. Une source obéit à la loi de Lambert si l’énergie rayonnée depuis un point de cette source est la même dans toutes les directions (on dit que son intensité est isotrope et donc indépendante de l'angle d’où on observe cette source). Soit M la valeur de l’éclairement mesurée par un capteur. On peut facilement en déduire le flux énergétique de la source de surface S : Φ =S M
\item Suppons que l'on dispose cette fois d'informations sur la densité de flux énergétique~$F$ d'une source sphérique de rayon~$R$ (par exemple, le Soleil). Puisque l'on considère une source, $F$ peut également être appelée émittance et être notée~$M$. Si le rayonnement est \voc{uniforme}, c'est-à-dire qu'en chaque point la source émet le même flux énergétique~$\Phi$ par unité de surface~$\dd S$, alors on dispose de la relation suivante $$ \Phi = 4\,\pi\,R^2 \, M$$ et plus généralement pour une surface~$S$ qui est une source uniforme de rayonnement $$ \boxed{ \Phi = S \, M } $$ On notera qu'on fait les calculs en considérant seulement le côté extérieur de la surface, celui d'où nous regardons la source, car seule la moitié de l'énergie échangée par les points de la surface~$S$ est émise sous forme de rayonnement. L'autre moitié est échangée du côté intérieur de la surface avec le milieu constituant le corps. 
\item L'énergie transmise par le rayonnement, et toutes les grandeurs définies précédemment, varient généralement beaucoup avec la longueur d'onde étudiée. Supposons que l'on connaisse la luminance spectrale~$L_\lambda$ dans un petit intervalle~$\dd \lambda$ autour de la longueur d'onde~$\lambda$, et ce, pour toutes les longueurs d'onde~$\lambda$ du spectre électromagnétique. Par exemple, la luminance totale~$L$ est retrouvée par intégration des longueurs d'onde les plus courtes aux plus longues : $$L=\int_{\lambda} L_\lambda \, \dd\lambda = \int_\nu L_\nu \, \dd \nu $$ Il s'agit d'une des formes du \voc{principe de superposition}, qui indique que le rayonnement électromagnétiques se compose d'une superposition d'ondes monochromatiques. Une intégration similaire est effectuée par l'électronique embarquée dans un appareil photo qui traite les flux reçus par les capteurs dans une variété de longueurs d'onde en domaine visible, afin d'obtenir une image finale qui intègre toutes ces informations.
\end{finger}



\mk \section{Emission de rayonnement} \label{corpsnoir}

		\sk
Le Soleil qui se situe à une distance considérable dans le vide spatial nous procure une sensation de chaleur. De même, placer sa main sur le côté d'un radiateur en fonctionnement sans le toucher procure une sensation de chaleur instantanée qui ne peut être attribuée à un transfert convectif entre le radiateur et la main. Cet échange de chaleur est attribué au contraire à l'émission d'ondes électromagnétiques par la matière du fait de sa température; on parle d'émission de \voc{rayonnement thermique}. Tous les corps émettent du rayonnement thermique. La transmission de cette énergie entre une source et une cible ne nécessite pas la présence d'un milieu intermédiaire matériel. 
%Le but de cette section est d'en étudier les principales propriétés.


	\sk \subsection{Corps noir}

		\sk
On appelle \voc{corps noir} un objet dont la surface est idéale et satisfait les trois conditions suivantes~:
\begin{description}
\item[émetteur parfait] un corps noir rayonne plus d’énergie radiative à chaque température et pour chaque longueur d’onde que n'importe quelle autre surface,
\item[absorbant parfait] un corps noir absorbe complètement le rayonnement incident selon toutes les directions de l'espace et toutes les longueurs d'onde,
\item[source lambertienne] un corps noir émet du rayonnement de façon isotrope
\end{description}

\sk
Un corps noir est à l'équilibre thermodynamique avec son environnement. On peut montrer qu'un tel corps émet du rayonnement qui dépend seulement de sa température et non de sa nature. La définition du corps noir, et les développements théoriques qui l'accompagnent, sont partis du constat, fait notamment par les céramistes, qu'un objet placé dans un four à haute température devient rouge en même temps que les parois du four quelle que soit sa taille, sa forme ou le matériau qui le compose. Un exemple de source utilisée pour étudier expérimentalement le modèle du corps noir consiste à construire une enceinte chauffée, totalement hermétique, et y percer un trou pour y mesurer le flux énergétique émis [figure~\ref{fig:four}]

\figside{0.35}{0.15}{\figwallace/Radiation/radiation_Page_10_Image_0001.png}{L'énergie entrant par une petite fente dans une enceinte subit des réflexions sur la paroi jusqu'à ce qu'elle soit absorbée. L'ouverture dans la paroi d'une enceinte chauffée apparaît comme une source de type corps noir. Un absorbant presque parfait est aussi un émetteur presque parfait. Ce type de four a été employé au début du XXe siècle pour évaluer expérimentalement les prédictions théoriques de Planck. Source~: Wallace and Hobbs, Atmospheric Science, 2006.}{fig:four}


		\sk
L'émission de rayonnement par le corps noir est décrite par une luminance énergétique spectrale~$L_{\lambda}$, notée $B_\lambda$ dans ce qui suit\footnote{Correspond au nom anglais \emph{blackbody}}. La loi de variation de~$B_\lambda$ selon la température~$T$ est donnée par la \voc{loi de Planck}\footnote{La luminance spectrale $B_\nu$ est déterminée d'une façon similaire. La démonstration de la loi de Planck fait appel à des notions de quantification d'énergie et de thermodynamique statistique qui sont hors programme dans le cadre de ce cours.} $$ B_\lambda(T) = \frac{C_1 \, \lambda^{-5}}{\pi \, \left( e^{ C_2 / \lambda T}-1\right) } $$ où $C_1$ et $C_2$ sont des constantes. Comme le rayonnement du corps noir est isotrope, l'émittance spectrale du corps noir, obtenue par intégration sur toutes les directions de l'espace, vaut $ M_\lambda(T) = \pi \, B_\lambda(T) $. 

%\figun{0.5}{0.25}{\figfrancis/WH_BBrad}{Courbes de luminance spectrale d'un corps noir pour différentes températures. La courbe en pointillés indique la position du maximum en fonction de $T$.}{fig:BBrad} 
\figside{0.5}{0.25}{\figwallace/Radiation/radiation_Page_11_Image_0001.png}{Courbes de luminance spectrale d'un corps noir pour différentes températures. La courbe en pointillés indique la position du maximum en fonction de $T$. Source~: Wallace and Hobbs, Atmospheric Science, 2006.}{fig:BBrad} 

\sk
Les variations de la fonction~$B_\lambda$ sont illustrées sur la figure~\ref{fig:BBrad}. L'émission de rayonnement par le corps noir ne dépend que de la longueur d'onde~$\lambda$ et de la température~$T$ du corps. A une température donnée, le rayonnement émis est parfaitement déterminé pour chaque longueur d'onde; dans un domaine spectral particulier, le rayonnement émis ne dépend que de la température du corps noir.



		\paragraph{Variations selon la température} 

\begin{finger}
\item L'énergie émise dépend de la température du corps émetteur~: 
\begin{citemize}
\item quantitativement~: plus le corps est chaud, plus la quantité de rayonnement thermique est grande~: la luminance spectrale~$B_{\lambda}$ augmente avec la température $T$ quelle que soit la longueur d'onde.
\item qualitativement~: la \ofg{couleur} du corps dépend de sa température~: la longueur d'onde pour laquelle le rayonnement est maximal diminue quand la température augmente.
\end{citemize}
\item La dépendance en température de la forme des courbes sur la figure~\ref{fig:BBrad} est résumée par deux lois simples qui sont décrites à la section suivante~: la loi de Wien (position du maximum) et la loi de Stefan-Boltzmann (intégrale totale).  
\end{finger}

\paragraph{Variations selon la longueur d'onde} 

\begin{finger} 
\item Le rayonnement thermique est surtout significatif entre les longueurs d'onde~$0.1$ et~$100$~$\mu$m, soit le domaine visible et infrarouge. Pour le type de température usuellement rencontrées sur Terre, la contribution dans les longueurs d'onde visible est petite par rapport à la contribution dans l'infrarouge -- il faut atteindre des températures de plusieurs centaines de degrés Celsius pour qu'elle devienne significative, comme on peut le constater lorsqu'on porte à haute température un morceau de métal ou que l'on considère une coulée de lave fraîche.
\item La luminance énergétique~$B_{\lambda}$ tend vers 0 
\begin{citemize}
\item aux longueurs d'ondes très courtes, ce qui signifie que le rayonnement thermique comporte extrêmement peu des photons les plus énergétiques;
\item et aux longueurs d'onde très grandes, ce qui est attendu étant donné que l'énergie des photons tend vers~$0$ et que leur nombre n'est pas suffisant pour que la contribution énergétique soit significative.
\end{citemize}
\end{finger}



	\sk \subsection{Lois du corps noir}

		\sk \subsubsection{Loi de Wien : maximum d'émission thermique}

		\sk
On observe sur la figure \ref{fig:BBrad} que, lorsque~$T$ augmente, la maximum de la luminance spectrale~$B_\lambda$, appelé \voc{maximum d'émission}, se décale vers les longueurs d'onde courtes, c'est-à-dire correspond à des photons de plus en plus énergétiques. La loi exacte, appelée \voc{loi de déplacement de Wien}, s'obtient en dérivant $B_\lambda$ par rapport à $\lambda$, ce qui permet d'obtenir $$ \boxed{ \lambda\e{max} \, T = 2.898 \times 10^{-3} \, \textrm{(mètres~K)} } $$ où $\lambda_{max}$ est la longueur d'onde du maximum de luminance spectrale~$B_\lambda$. La longueur d'onde du maximum d'émission~$\lambda\e{max}$ est ainsi inversement proportionnelle à la température du corps émetteur. Une formulation alternative est que $\nu\e{max}$ est proportionelle à $T$.

\figun{0.6}{0.45}{/home/aymeric/Big_Data/BOOKS/pierrehumbert_pics/9780521865562c03_fig001.jpg}{Source~: R. Pierrehumbert, Principles of Planetary Climates, CUP, 2010.}{wvl} 



		\sk \subsubsection{Loi de Stefan-Boltzmann : flux net surfacique}

		\sk
La \voc{loi de Stefan-Boltzmann}\footnote{Joseph Stefan met expérimentalement en évidence en 1879 la dépendance de l'émittance en puissance quatrième de la température. Ludwig Boltzmann, à qui l'on doit également des résultats fondamentaux sur l'entropie et l'atomisme, prouve en 1884 le résultat par des arguments théoriques.} donne la valeur de l'intégrale sur toutes les longueurs d'ondes et dans tout l'espace\footnote{On entend par là toutes les directions du demi-espace extérieur au corps considéré.} de la courbe du corps noir, décrite par la loi de Planck et illustrée par les figures \ref{fig:BBrad} et \ref{fig:BBmax}. Cette loi donne donc l'expression d'une densité de flux énergétique~$F$ ou plus spécifiquement, puisque le corps noir est une source de rayonnement, d'une émittance totale~$M$. Cette dernière s'obtient tout d'abord avec une intégration par rapport à~$\lambda$ de la luminance énergétique spectrale~$B_\lambda$ donnée par la loi de Planck, afin d'obtenir la luminance énergétique~$B$. On déduit ensuite l'émittance totale~$M$ en intégrant selon toutes les directions de l'espace; comme le rayonnement du corps noir est isotrope, $M$ s'obtient à partir de~$B$ simplement en multipliant par~$\pi$. La loi de Stefan-Boltzmann établit que le flux net surfacique~$M$ émis par un corps noir ne dépend que de sa température par une dépendance type loi de puissance $$ \boxed{ M\e{corps noir} = \sigma \, T^4 } $$ avec~$\sigma=5.67 \times 10^{-8} \textrm{~W~m}^{-2}\textrm{~K}^{-4}$ appelée constante de Stefan-Boltzmann. La loi de Stefan-Boltzmann, comme la loi de Planck dont elle dérive, stipule que l'émittance~$M$ d'un corps pouvant être considéré en bonne approximation comme un corps noir ne dépend que de sa température et non de sa nature. Cette loi indique par ailleurs que l'émittance~$M$ augmente très rapidement avec la température -- de par la puissance quatrième impliquée.


	\sk \subsection{Lois des corps gris et émissivité}

		\sk
Le corps noir est un modèle idéal d'absorbant qu'en pratique on ne rencontre pas dans la nature. Par exemple, le charbon noir est un absorbant parfait, mais seulement dans les longueurs d'onde visible. La plupart des objets ressemblent néanmoins au corps noir, au moins à certaines températures et pour certaines longueurs d'onde considérées en pratique. Dans le cas d'un corps qui n'est pas un absorbant parfait, on parle d'un \voc{corps gris}. A température égale, un corps gris n'émet pas autant qu'un corps noir dans les mêmes conditions. Pour évaluer l'énergie émise par un corps gris par comparaison à celle qu'émettrait le corps noir dans les mêmes conditions, on définit un coefficient appelé \voc{émissivité} $\epsilon_\lambda$ compris entre~$0$ et~$1$ et égal au rapport entre la luminance spectrale du corps~$L_\lambda$ et celle du corps noir~$B_\lambda$: $ \epsilon_\lambda=L_\lambda / B_\lambda(T)$ En toute généralité, l'émissivité~$\epsilon_{\lambda}$ d'une surface à une longueur d'onde~$\lambda$ dépend de ses propriétés physico-chimiques, de sa température et de la direction d'émission\footnote{Par exemple, les métaux, matériaux conducteurs de l'électricité, ont une émissivité faible (sauf dans les directions rasantes) qui croît lentement avec la température et décroît avec la longueur d'onde ; au contraire, les diélectriques, matériaux isolant de l'électricité, ont une émissivité élevée qui augmente avec la longueur d'onde et se révèlent lambertiens sauf pour les directions rasantes où l'émissivité décroît significativement.}.

\sk
On peut définir une émissivité totale intégrée~$\epsilon$ qui permet d'exprimer l'émittance~$M$ d'un corps gris $$ \boxed{\SB} $$ Des valeurs de l'émissivité totale~$\epsilon$ pour certains matériaux sont données dans le tableau~\ref{tab:emiss}~: l'eau, la neige, les roches basaltiques ont des émissivités proches de~$1$ et sont donc des corps noirs en bonne approximation. 


\begin{table}[h!]
\label{tab:emiss}
\begin{center}
\footnotesize
\begin{tabular}{||c|c||c|c||c|c||}
\hline
Matériau & Emissivité~$\epsilon$ & Matériau & Emissivité~$\epsilon$ & Matériau & Emissivité~$\epsilon$ \\
\hline
Cuivre poli & 0.03 		& Cuivre oxydé & 0.5 		& Béton & 0.7 à 0.9 	\\
Carbone & 0.8 			& Lave (volcan actif) & 0.8 	& Suie & 0.95		\\
Ville & 0.85 			& Peinture blanche & 0.87 	& Peinture noire & 0.94 \\
Désert & 0.85 à 0.9 		& Herbe & 0.9 à 0.95		& Forêt & 0.95 		\\
Nuages cirrus & 0.10 à 0.90 	& Nuages cumulus & 0.25 à 0.99	& Eau & 0.92 à 0.97  	\\
Neige âgée & 0.8 		& Neige fraîche & 0.99		& &			\\
\hline
\end{tabular}
\normalsize
\caption{\emph{Quelques valeurs usuelles d'émissivité à la température ambiante (pour un rayonnement infrarouge). Source~: Hecht, Physique, 1999 -- avec quelques ajouts d'après site CNES}}
\end{center}
\end{table}



\mk \section{Energie reçue du Soleil}

	\sk \subsection{Caractéristiques et domaine de longueurs d'onde}

		\sk
Le Soleil peut être considéré en bonne approximation comme un corps noir car il absorbe tout le rayonnement incident. Sa \ofg{couleur} est dûe à du rayonnement émis et, plus précisément, correspond aux longueurs d'onde où le maximum de rayonnement est émis. D'après la loi de Wien, le Soleil, dont l'enveloppe externe a une température autour de~$6000$~K, a donc un maximum d'émission situé dans le visible à $\lambda\e{max} = 0.5 \mu$m, proche du maximum de sensibilité de l'oeil humain [figure~\ref{fig:BBmax} haut]. Au contraire, la surface terrestre, dont la température typique est d'environ~$288$~K, voit son maximum d'émission situé dans l'infrarouge vers 10~$\mu$m, alors que le rayonnement émis dans les longueurs d'ondes visible est négligeable [figure~\ref{fig:BBmax} bas]. Un raccourci usuel est donc de dire que \ofg{la Terre émet du rayonnement (thermique) dans l'infrarouge alors que le Soleil émet dans le visible}. En toute rigueur, cette affirmation ne parle que du voisinage du maximum d'émission, où la contribution au flux intégré selon toutes les longueurs d'onde est la plus significative. Il est ainsi plus exact de dire que, dans l'atmosphère, la région du spectre où~$\lambda$ est inférieure à environ 4~$\mu$m est dominée par le rayonnement d'origine solaire, alors qu'au-delà, le rayonnement est surtout d'origine terrestre. Il n’y a pratiquement pas de recouvrement entre la partie utile du spectre du rayonnement solaire et celui d’un corps de température ambiante; ce fait est d'une grande importance pour les phénomènes de type effet de serre, qui sont abordés plus loin dans ce cours. On désigne ainsi souvent le rayonnement d'origine solaire par le terme \voc{ondes courtes} et le rayonnement d'origine terrestre par le terme \voc{ondes longues}.

\figsup{0.65}{0.2}{decouverte/cours_meteo/6000K.jpg}{decouverte/cours_meteo/earth.jpg}{Courbes de luminance spectrale d'un corps noir pour différentes températures correspondant notamment au Soleil (haut) et à la Terre (bas). La quantité représentée ici est l'émittance spectrale~$M_\lambda = \pi \, B_\lambda$. Noter la différence d'indexation de l'abscisse et l'ordonnée sur les deux schémas. Le rayonnement thermique émis par la Terre est plusieurs ordres de grandeur moins énergétique que celui émis par le Soleil et le maximum d'émission se trouve à des longueurs d'onde plus grandes (infrarouge pour la Terre au lieu de visible pour le Soleil). Source : \url{http://hyperphysics.phy-astr.gsu.edu/hbase/bbrc.html}.}{fig:BBmax}


	\sk \subsection{Constante solaire}

		\sk
La distance Soleil-Terre est beaucoup plus grande que les rayons de la Terre et du Soleil. Ainsi, d'une part, le rayonnement solaire arrive au niveau de l'orbite terrestre en faisceaux pratiquement parallèles. D'autre part, la luminance en différents points de la Terre ne varie pas. On peut par conséquent définir une valeur moyenne de la densité de flux énergétique du rayonnement solaire au niveau de l'orbite terrestre, reçue par le système surface~+~atmosphère. Elle est désignée par le terme de \voc{constante solaire} notée~$\mathcal{F}\e{s}$. Les mesures indiquent que
\[ \mathcal{F}\e{s} = 1368 \text{~W~m}^{-2} \qquad \text{pour la Terre} \]

\sk
La constante solaire est une valeur instantanée côté jour~: le rayonnement solaire reçu au sommet de l'atmosphère en un point donné de l'orbite varie en fonction de l'heure de la journée et de la saison considérée (c'est-à-dire la position de la Terre au cours de sa révolution annuelle autour du Soleil)\footnote{En réalité, la constante solaire~$\mathcal{F}\e{s}$ varie elle-même d'environ~$3$~W~m$^{-2}$ en fonction des saisons à cause de l'excentricité de l'orbite terrestre, qui n'est pas exactement circulaire. De plus, elle peut varier évidemment en fonction des cycles solaires, néanmoins sans influence majeure sur la température des basses couches atmosphériques (troposphère et stratosphère).}. On peut donc définir un \voc{éclairement solaire moyen} noté~$\mathcal{F}\e{s}'$ reçu par la Terre qui intègre les effets diurnes et saisonniers. Autrement dit, $\mathcal{F}\e{s}$~est l'éclairement instantané reçu par un satellite en orbite autour de la Terre~; $\mathcal{F}\e{s}'$ est la valeur que l'on obtiendrait si l'on faisait la moyenne d'un grand nombre de mesures instantanées du satellite à diverses heures et saisons. 

\figside{0.5}{0.2}{decouverte/cours_dyn/incoming.png}{Energie reçue du Soleil par le système Terre. Source~: McBride and Gilmour, \emph{An Introduction to the Solar System}, CUP 2004.}{fig:eqrad}

\sk
On admet ici que~$\mathcal{F}\e{s}'$ peut être calculé en considérant que le flux total reçu du Soleil l'est à travers un disque de rayon le rayon~$R$ de la Terre (il s'agit de l'ombre projetée de la planète, voir Figure~\ref{fig:eqrad}). A cause de l'incidence parallèle, le flux énergétique intercepté par la Terre vaut donc~$\Phi = \pi \, R^2 \, \mathcal{F}\e{s}$. L'éclairement moyen à la surface de la Terre est alors $$\mathcal{F}\e{s}' = \frac{\Phi}{4 \, \pi \, R^2}$$ le dénominateur étant l'aire de la surface complète de la Terre. On obtient ainsi
\[ \boxed{ \mathcal{F}\e{s}' = \frac{\mathcal{F}\e{s}}{4} } \]

		\sk
La valeur de la constante solaire peut s'obtenir par le calcul. Le soleil est considéré en bonne approximation comme un corps noir de température~$T_{\sun} = 5780$~K. D'après la loi de Stefan-Boltzmann, son émittance est $M = \sigma \, T_{\sun}^4$ donc le flux énergétique~$\Phi_{\sun}$ émis par le Soleil de rayon~$R_{\sun} = 7 \times 10^5$~km est~$\Phi_{\sun} = 4 \, \pi \, R_{\sun}^2 \, \sigma \, T_{\sun}^4$. Ce flux énergétique est rayonné dans tout l'espace~: à une distance~$d$ du soleil il est réparti sur une sphère de centre le soleil et de rayon~$d$, donc de surface~$4 \, \pi \, d^2$. A cette distance, l'éclairement~$\mathcal{F}$, c'est-à-dire la densité de flux énergétique reçue en W~m$^{-2}$, s'écrit donc
\[ \mathcal{F} = \frac{\Phi_{\sun}}{4 \, \pi \, d^2} = \frac{4 \, \pi \, R_{\sun}^2 \, \sigma \, T_{\sun}^4}{4 \, \pi \, d^2} = \sigma \, T_{\sun}^4 \, \left( \frac{R_{\sun}}{d} \right)^2 \]
Si l'on prend~$d$ égal à la distance Terre-Soleil, $\mathcal{F}$ définit ainsi la constante solaire~$\mathcal{F}\e{s}$.
%\[ \mathcal{F}\e{s} = \frac{{\mathcal{F}\e{s}}^{\text{Terre}}}{d\e{soleil}^2} \]

%Variation de la constante solaire : Bien que l’intensité du soleil ait subit des variations depuis la formation de la Terre, on peut s’attendre à ce qu’elle soit stable sur une période de 1000 ans. On mesure mal la constante solaire, mais les mesures récentes, même avec leurs incertitudes, semblent indiquer que le soleil ne peut pas expliquer le réchauffement récent. Notons toutefois que les simulations actuelles ne tiennent pas compte des fluctuations possibles du rayonnement solaire (négligeable a priori).
%%%% pas sûr du dernier point.


\chapter{Bilan radiatif et effet de serre}

\dictum[Frederik van Eeden, 1887]{Le soleil accepte bien de passer par de petites fenêtres.}

\bk Les chapitres précédents n'ont été qu'un prélude pour bien comprendre les phénomènes qui déterminent la température à la surface de la Terre et dans son atmosphère. On s'attache dans le présent chapitre à effectuer des bilans d'énergie pour le système Terre et son atmosphère, définissant ainsi son \voc{bilan radiatif}.

\mk \section{Equilibre radiatif simple}

	\sk \subsection{Flux reçu et flux émis}

		\sk
Nous pouvons exprimer le rayonnement reçu du Soleil par la Terre par une densité de flux énergétique moyenne~$F\e{reçu}$ en W~m$^{-2}$ ou un flux énergétique~$\Phi\e{reçu}$ (en W)
\[ 
F\e{reçu} = (1-A\e{b}) \, \mathcal{F}\e{s}' 
\qquad \qquad
\Phi\e{reçu} = \pi \, R^2 \, (1-A\e{b}) \, \mathcal{F}\e{s}
\] 
La partie du rayonnement reçue du soleil qui est réfléchie vers l'espace est prise en compte via l'albédo planétaire noté~$A\e{b}$. On rappelle par ailleurs que~$\mathcal{F}\e{s}' = \mathcal{F}\e{s} / 4$ où $\mathcal{F}\e{s}$ est la constante solaire.


\sk
Par ailleurs, le système Terre émet également du rayonnement principalement dans les longueurs d'onde infrarouge [figure \ref{fig:eqrad2}]. 
Cette quantité de rayonnement émise au sommet de l'atmosphère radiative est notée $OLR$ pour \emph{Outgoing Longwave Radiation} en anglais.
A l'équilibre, la planète Terre doit émettre vers l'espace autant d'énergie qu'elle en reçoit du Soleil, donc
on obtient la relation générale appelée \emph{TOA} pour \emph{Top-Of-Atmosphere} en anglais, correspondant
au bilan radiatif au sommet de l'atmosphère
\[ \boxed{\TOA} \] 
La principale difficulté qui sous-tend les divers modèles pouvant être proposés réside dans l'expression du terme~$OLR$.




	\sk \subsection{Equilibre et température équivalente}

		\sk
Dans l'équilibre~\emph{TOA}, la manière la plus simple de définir~$OLR$ pour entamer un calcul préliminaire est comme suit. On fait l'hypothèse, assez réaliste en pratique, que la surface de la Terre est comme un corps noir, c'est-à-dire que son émissivité est très proche de~$1$ dans l'infrarouge où se trouve le maximum d'émission. D'après la loi de Stefan-Boltzmann, la densité de flux énergétique~$F\e{émis}$ émise par la Terre en W~m$^{-2}$ s'exprime
\[ F\e{émis} = \sigma \, {T\e{eq}}^4 \]
où~T\e{eq} est la \voc{température équivalente} du système Terre que l'on suppose uniforme sur toute la planète. Autrement dit, $T\e{eq}$ est la température équivalente d'un corps noir qui émettrait la quantité d'énergie~$F\e{émis}$. Le flux énergétique~$\Phi\e{émis}$ émis par la surface de la planète Terre s'exprime
\[ \Phi\e{émis} = 4 \, \pi \, R^2 \, F\e{émis} = 4 \, \pi \, R^2 \, \sigma \, {T\e{eq}}^4 \]
Contrairement au cas de l'énergie visible, il n'y a pas lieu de prendre en compte le contraste jour/nuit, car le rayonnement thermique émis par la Terre l'est à tout instant par l'intégralité de sa surface. La seule limite éventuellement discutable est l'uniformité de la température de la surface de la Terre, ce qui est irréaliste en pratique. On peut souligner cependant que, même dans le cas d'une planète n'ayant pas une température uniforme ou ne se comportant pas comme un corps noir, le rayonnement émis vers l'espace doit être égal en moyenne à $\sigma \, {T\e{eq}}^4$.
%% CHANGER LES SLIDES, ne pas utiliser P

\figsup{0.31}{0.17}{decouverte/cours_dyn/incoming.png}{decouverte/cours_dyn/emission.png}{Equilibre radiatif simple : à gauche, l'énergie reçue du Soleil par le système Terre ; à droite, l'énergie émise par le système Terre. Source~: McBride and Gilmour, \emph{An Introduction to the Solar System}, CUP 2004.}{fig:eqrad2}

\sk
A l'équilibre, la planète Terre doit émettre vers l'espace autant d'énergie qu'elle en reçoit du Soleil (équilibre \emph{TOA}). Ceci peut s'exprimer par unité de surface
\[ \boxed{ F\e{reçu} = F\e{émis} } \]
ou, pour un résultat similaire, en considérant l'intégralité de la surface planétaire
\[ \Phi\e{reçu} = \Phi\e{émis} \]
ce qui permet de déterminer la température équivalente en fonction des paramètres planétaires
\[ \boxed{
T\e{eq} = \bigg[ \frac{\mathcal{F}\e{s}'\,(1-A\e{b})}{\sigma} \bigg]^{\frac{1}{4}}
} \]


		\input{bilan_radiatif_temperature_equivalente_application.tex}

\mk \section{Modèles \ofg{à couches}} % dits aux puissances échangées

	\sk \subsection{Modèle à une couche et effet de serre}

		\sk
L'équilibre radiatif simple présenté à la section précédente souffre d'un problème majeur~: il suppose que l'atmosphère n'interagit pas avec les rayonnements incidents et émis, ce qui n'est pas le cas en réalité. On rappelle notamment avec la figure~\ref{fig:atmspectrum} deux points importants qui vont nous permettre de raffiner les calculs.
\begin{finger}
\item Au vu des températures typiques du Soleil et de la Terre, le rayonnement d'origine solaire est principalement émis dans les longueurs d'onde visible, alors que le rayonnement d'origine terrestre est principalement émis dans les longueurs d'onde infrarouge. Les fonctions de Planck normalisées montrées dans la figure~\ref{fig:atmspectrum} indiquent que les deux domaines d'émission ne se recoupent quasiment pas. On peut donc séparer les calculs selon le domaine visible (également appelé ondes courtes) pour tout ce qui concerne le rayonnement reçu du Soleil et le domaine infrarouge (également appelé ondes longues) pour tout ce qui concerne le rayonnement émis par la surface et l'atmosphère de la Terre. La figure~\ref{fig:modzero} reprend ainsi le calcul de l'équilibre radiatif simple en étant plus fidèle à cette distinction entre visible et infrarouge~; en l'absence d'atmosphère, la température de surface à l'équilibre~$T\e{s}$ est égale à~$T\e{eq}$.
%\figside{0.6}{0.3}{\figfrancis/WH_atmspectrum}{entre le sommet de l'atmosphère et 11~km.}{fig:atmspectrum}
\item L'équilibre radiatif simple néglige les propriétés d'absorption de l'atmosphère de la Terre. La figure~\ref{fig:atmspectrum} montre que cette approximation est relativement juste pour les longueurs d'onde visible, où l'atmosphère est assez transparente, mais très inexacte pour les longueurs d'onde infrarouge. On a vu dans les chapitres qui précèdent que, contrairement à ce qui prévaut dans les longueurs d'onde visible, l'atmosphère est très opaque, c'est-à-dire très absorbante, dans l'infrarouge à cause principalement des gaz à effet de serre (et des nuages). Comme décrit au chapitre précédent, et sur la figure~\ref{fig:atmspectrum}, les principaux gaz à effet de serre sont, par ordre d'importance dans le bilan radiatif de la Terre, H$_2$O, CO$_2$, CH$_4$, N$_2$O, O$_3$, auxquels il convient d'ajouter les gaz à effet de serre industriels, tels les halocarbures, notamment les chloro-fluoro carbures\footnote{Qui jouent par ailleurs un rôle dans la destruction de l'ozone stratosphérique}. On note au passage que certains gaz à effet de serre comme CO$_2$ et CH$_4$ sont à la fois controlés par des processus naturels et industriels. Le rayonnement émis par la surface terrestre, principalement dans l'infrarouge, est donc absorbé par ces espèces et réémis à la fois vers l'espace et vers la surface. Ainsi, contrairement à ce qui est supposé dans le cas de l'équilibre radiatif simple, une partie du rayonnement émis par la surface n'est pas évacuée vers l'espace et contribue à augmenter la température de la surface terrestre. Ainsi la température de surface à l'équilibre~$T\e{s}$ n'est pas égale à la température équivalente~$T\e{eq}$. La figure~\ref{fig:modun} résume cette situation qui permet d'obtenir par le calcul, présenté ci-dessous, une valeur pour~$T\e{s}$ plus proche de la température effectivement mesurée à la surface de la Terre. On parle de \voc{modèle à une couche}.
\end{finger}

\figside[page=1]{0.6}{0.25}{decouverte/cours_meteo/zero_couche.png}{Modèle à zéro couche~: schéma des flux nets échangés dans le visible et dans l'infrarouge pour une planète sans atmosphère (ou plus précisément dans laquelle l'atmosphère n'est active radiativement ni dans l'infrarouge ni dans le visible) de température de surface~$T\e{s}$. Il s'agit simplement d'une présentation alternative de l'équilibre radiatif simple décrit en figure~\ref{fig:eqrad2}, qui s'avère plus pratique pour prendre en compte la présence d'une atmosphère et effectuer des calculs plus proches de la réalité. Ce schéma est cependant plus précis que la figure~\ref{fig:eqrad2} dans la mesure où il précise dans quel domaine de longueur d'onde se font les échanges.}{fig:modzero}
%\figun{0.6}{0.2}{\figfrancis/GH_1lay_noatm}{Schéma des flux échangés dans le visible (jaune) et l'infrarouge (rouge) pour une planète sans atmosphère de température de surface $T_s$.}{fig:GH1laynoatm}

		\sk
Quelle température de surface est prédite par le modèle à une couche décrit par la figure~\ref{fig:modun} ? On considère toujours une planète d'albédo planétaire $A\e{b}$ recevant l'éclairement moyen $\mathcal{F}\e{s}'$ du Soleil. Ce bilan correspond à la partie visible de la figure~\ref{fig:modun}. L'atmosphère est considérée comme transparente dans ce domaine de longueur d'onde. Dans la partie infrarouge, au contraire on ne néglige plus l'absorption, par les gaz à effet de serre présents dans l'atmosphère, du rayonnement infrarouge émis par la surface de la planète à la température~$T\e{s}$~: on représente ainsi l'atmosphère par une couche isotherme de température~$T\e{a}$, parfaitement absorbante dans l'infrarouge. Le rayonnement infrarouge émis par la surface est complètement absorbé dans l'atmosphère, qui émet à son tour~$\sigma {T\e{a}}^4$ à la fois vers l'espace et vers la surface comme indiqué dans le domaine infrarouge de la figure~\ref{fig:modun}. Une partie du rayonnement infrarouge émis par la Terre n'est donc pas évacuée vers l'espace et reste \ofg{piégée} dans le système atmosphère~+~surface, contribuant ainsi à élever la température de la surface~$T\e{s}$.

\figside{0.6}{0.2}{decouverte/cours_meteo/une_couche.png}{Modèle à une couche~: schéma des flux échangés dans le visible et dans l'infrarouge pour une planète dont l'atmosphère de température~$T\e{a}$ est opaque dans l'infrarouge.}{fig:modun}

\sk
Il s'agit ensuite d'effectuer le bilan des flux reçus et cédés en chacune des interfaces en rassemblant les termes des deux domaines visible et infrarouge.
%%\footnote{Les modèles du type de celui présenté ici sont parfois également appelés modèles aux puissances échangées}
\begin{finger}
\item pour l'atmosphère
\[ \underbrace{\sigma \, {T\e{s}}^4}_{\text{bilan des flux reçus}} = \underbrace{\sigma \, {T\e{a}}^4 + \sigma \, {T\e{a}}^4}_{\text{bilan des flux cédés}} \] 
On note que le rayonnement visible reçu du Soleil n'intervient pas dans le bilan pour l'atmosphère, ce qui est normal puisque l'absorption est négligée. Ainsi, comme indiqué sur le schéma~\ref{fig:modun}, l'atmosphère reçoit un rayonnement~$\mathcal{F}\e{s}'$ dont la partie~$\mathcal{F}\e{s}'\,(1-A\e{b})$ qui n'est pas réfléchie/diffusée est entièrement transmise à la surface. Tout se passe comme si l'atmosphère recevait~$\mathcal{F}\e{s}'$ et cédait~$\mathcal{F}\e{s}'\,(1-A\e{b})$ à la surface et~$\mathcal{F}\e{s}'\,A\e{b}$ à l'espace~; son bilan d'énergie dans le visible est donc nul puisque tous ces termes se compensent.
\item pour la surface
\[ \underbrace{\mathcal{F}\e{s}'\,(1-A\e{b}) + \sigma \, {T\e{a}}^4}_{\text{bilan des flux reçus}} = \underbrace{\sigma \, {T\e{s}}^4}_{\text{bilan des flux cédés}} \]
\end{finger}
On dispose alors de deux équations qui permettent de déterminer les deux inconnues~$T\e{a}$ et~$T\e{s}$. Ainsi la température à la surface de la planète dans le modèle à une couche est 
\[ \boxed{ T\e{s} = \bigg[ \frac{ 2 \, \mathcal{F}\e{s}'\,(1-A\e{b}) }{ \sigma } \bigg]^{\frac{1}{4}} = \sqrt[4]{2} \, T\e{eq} } \]

\sk
Le calcul numérique donne une température de~$303$~K (environ~$30^{\circ}$C) pour la Terre, une valeur à la fois bien supérieure à~$T\e{eq}$, qui vaut~$255$~K, et plus proche de la température effectivement constatée à la surface, quoiqu'un peu surévaluée. Ainsi les gaz à effet de serre présents dans l'atmosphère contribuent à réchauffer significativement la surface d'une planète. Le modèle à une couche est le modèle le plus simple de l'effet de serre qui permet d'en rendre compte qualitativement et, dans une certaine mesure, quantitativement.

		\sk
La température de l'atmosphère dans le modèle à une couche est, d'après les équations qui précèdent,
\[ \boxed{ T\e{a} = \bigg[ \frac{ \mathcal{F}\e{s}'\,(1-A\e{b}) }{ \sigma } \bigg]^{\frac{1}{4}} = T\e{eq} } \]
Un résultat équivalent peut être obtenu en faisant un bilan des flux reçus/cédés pour l'interface \ofg{espace}
\[ \underbrace{\mathcal{F}\e{s}'\,A\e{b} + \sigma \, {T\e{a}}^4}_{\text{bilan des flux reçus}} = \underbrace{\mathcal{F}\e{s}'}_{\text{bilan des flux cédés}} \] 
Deux aspects de ce modèle simple de l'effet de serre sont importants:
\begin{enumerate}
\item Que l'on considère un équilibre radiatif simple [figures~\ref{fig:eqrad2}], ou un modèle à une couche [figures~\ref{fig:modun}], la température à laquelle est émise le rayonnement infrarouge sortant vers l'espace doit être (en moyenne) égale à $T\e{eq}$. Sans atmosphère, cette température est celle de la surface, avec une atmosphère opaque dans l'infrarouge, il s'agit de celle de l'atmosphère.
\item Il n'y a un effet de serre que si la température d'émission vers l'espace est inférieure à la température de la surface. On peut l'imaginer dans le cas où l'atmosphère est également opaque dans les longueurs d'onde visible~: la surface échange alors uniquement du rayonnement avec l'atmosphère, et est à la même température à l'équilibre~: $T\e{s}=T\e{a}=T\e{eq}$ [voir section suivante].
\end{enumerate}
Pour obtenir des températures atmosphériques plus en accord avec les variations verticales observées (qui servent à définir les différentes couches atmosphériques comme abordé au chapitre d'introduction), on peut adopter un modèle à~$2$, $3$, \ldots couches\footnote{Ce point est abordé en travaux dirigés.}.


	\sk \subsection{Modèle gris}

		\sk Le modèle à une couche peut être généralisé quelque peu en considérant deux raffinements.
\begin{finger}
\item L'atmosphère est absorbante dans le visible avec un coefficient d'absorption~$\alpha$. Plus~$\alpha$ est grand, plus le rayonnement incident dans les longueurs d'onde visible reçu par la surface terrestre est atténué. Ce phénomène porte le nom d'\voc{effet parasol} (ou parfois \ofg{anti-effet de serre}). En réalité cet effet est très modéré sur Terre. Certes, l'ozone stratosphérique absorbe complètement le rayonnement ultraviolet, mais ceci représente une contribution faible du flux total. Les aérosols, tels que les poussières désertiques ou les particules d'origine volcanique, absorbent dans le visible et peuvent contribuer lors d'événements particuliers, telles les éruptions volcaniques ou les tempêtes de poussière, à augmenter~$\alpha$.
\item L'atmosphère n'est pas tout à fait un corps noir~: son émissivité dans l'infrarouge est~$\epsilon$ comprise entre~$0$ et~$1$. D'après la seconde loi de Kirchhoff, ceci indique que la couche atmosphérique n'est pas parfaitement absorbante dans l'infrarouge~: elle absorbe une partie~$\epsilon \, F$ du flux incident~$F$ et en transmet une partie~$(1-\epsilon) \, F$. Reste qu'en pratique, comme mentionné dans la partie précédente, l'atmosphère se comporte comme un corps presque noir dans l'infrarouge et l'émissivité~$\epsilon$ est relativement proche de~$1$. 
\end{finger}
Un tel modèle porte le nom de \voc{modèle gris à une couche}. Comme dans la figure~\ref{fig:modun}, la méthode pour calculer les températures consiste à reporter tous les flux échangés entre chacune des couches, comme indiqué dans la figure~\ref{fig:modgris}, puis de faire le bilan des énergies reçues et cédées aux interfaces.
\[ \begin{aligned} & & \boxed{\text{bilan des flux reçus}} & = \boxed{\text{bilan des flux cédés}} \\ 
& \text{[espace]} & \mathcal{F}\e{s}'\,A\e{b} + (1-\epsilon) \, \sigma \, {T\e{s}}^4 + \epsilon \, \sigma \, {T\e{a}}^4 & = \mathcal{F}\e{s}' \\
& \text{[atmos.]} & \mathcal{F}\e{s}' + \sigma \, {T\e{s}}^4 & = \mathcal{F}\e{s}'\,A\e{b} + \mathcal{F}\e{s}'\,(1-A\e{b}-\alpha) + (1-\epsilon) \, \sigma \, {T\e{s}}^4 + \epsilon \, \sigma \, {T\e{a}}^4 + \epsilon \, \sigma \, {T\e{a}}^4 \\ 
& \text{[surface]} & \mathcal{F}\e{s}'\,(1-A\e{b}-\alpha) + \epsilon \, \sigma \, {T\e{a}}^4 & = \sigma \, {T\e{s}}^4 \\ \end{aligned} \]
%& \text{[surface]} & \mathcal{F}\e{s}'\,(1-A\e{b})\,(1-\alpha) + \epsilon \, \sigma \, {T\e{a}}^4 & = \sigma \, {T\e{s}}^4 \\ \end{aligned} \]

\sk
On obtient alors l'expression de la température de la surface de la planète dans le cadre du modèle gris en combinant les équations de bilan de deux des trois interfaces (par exemple espace et surface)
\[ \boxed{T\e{s} = \sqrt[4]{\frac{1 - \frac{\alpha^{\prime}}{2}}{1 - \frac{\epsilon}{2}}} \, T\e{eq}} \qquad  \text{avec} \quad \alpha^{\prime} = \frac{\alpha}{1-A\e{b}} \] 
\begin{citemize}
\item Dans le cas où l'atmosphère est un corps noir dans l'infrarouge~($\epsilon=1$) et qu'elle est transparente dans le visible~($\alpha=0$), on se retrouve dans la situation du modèle à une couche avec~$T\e{s} = \sqrt[4]{2} \, T\e{eq}$.
\item Dans le cas où l'atmosphère est opaque dans le visible~($\alpha^{\prime}=1$) et dans l'infrarouge~($\epsilon=1$), la surface échange alors uniquement du rayonnement avec l'atmosphère et~$T\e{s}=T\e{a}=T\e{eq}$.
\item Si~$\epsilon$ augmente, l'effet de serre augmente, donc la température de la surface de la planète augmente.
\item Si~$\alpha$ augmente, l'effet parasol augmente, donc la température de la surface de la planète diminue. C'est l'un des effets observés dans les mois qui suivent une éruption volcanique majeure.
\end{citemize}
%De façon plus générale, on a vu que le rayonnement sortant provenait majoritairement de la région de l'atmosphère autour d'une épaisseur optique de 1 à partir du sommet. Cette région dépend de la longueur d'onde: proche de la surface dans la fenêtre transparente, dans la haute troposphère dans les bandes d'absorption du CO$_2$, autour de 2~km dans celles de la vapeur d'eau. Comme la température décroit à partir de la surface, le rayonnement sortant est donc émis à des températures inférieures à $T_s$, et on peut écrire qu'il vaut \[IR_{sommet}=\sigma T_s^4 (1-\epsilon)=\sigma T_{eq}^4\] Où $\epsilon>0$ représente l'effet de serre. La valeur de $\epsilon$ augmente quand la température d'émission vers l'espace diminue par rapport à celle de surface, typiquement parce que l'altitude d'émission augmente.

\figside{0.7}{0.17}{decouverte/cours_meteo/une_couche_gris.png}{Modèle gris à une couche~: schéma des flux échangés dans le visible et dans l'infrarouge pour une planète comme la figure~\ref{fig:modun} sauf que l'atmosphère de température~$T\e{a}$ est opaque dans l'infrarouge, mais sans être un corps totalement noir, et est absorbante dans le visible avec un coefficient d'absorption~$\alpha$.}{fig:modgris}
%\figun{0.6}{0.2}{\figfrancis/GH_1lay_atm}{Comme la figure \ref{fig:GH1laynoatm} mais avec une atmosphère opaque dans l'infrarouge et de coefficient d'absorption $a$ dans le visible, de température $T_a$.}{fig:GH1layatm}


%%%%%%%%%%%%%%%%%%%%%%%%%%%%%%%%%%%%%%%%%%%%

\mk
\section{Description complète du bilan radiatif du système Terre}

\sk
\subsection{Mesures en moyenne dans le temps et dans l'espace}

\sk
Une représentation détaillée des différents flux échangés en moyenne temporelle et spatiale sur la Terre est présentée sur la figure~\ref{fig:bilflux}, qui est dérivée d'observations satellite les plus récentes. La figure est construite conformément à la séparation visible / infrarouge dictée par les résultats de la figure~\ref{fig:atmspectrum}. 

\sk
\subsubsection{Domaine visible}

\sk
Seulement~$50\%$ du rayonnement solaire incident dans les longueurs d'onde visible parviennent à la surface à cause, d'une part, de la réflexion/diffusion sur les molécules de l'air (diffusion Rayleigh dans toutes les directions, responsable de la couleur bleue du ciel), sur les gouttelettes nuageuses (diffusion de Mie) et sur la surface, et d'autre part, de l'absorption du rayonnement solaire incident par les molécules\footnote{Dans la mésosphère, c'est l'oxygène qui absorbe les radiations les plus énergétiques~; dans la stratosphère, l'absorption des radiations dans l’ultraviolet est assurée par différentes bandes d'absorption de l'ozone~; cette absorption peut avoir lieu dans certaines bandes jusque dans la troposphère.} et les aérosols composant l'atmosphère [relativement modérée dans les longueurs d'onde visible]. On note que la partie du rayonnement visible diffusée vers l'espace par les molécules de l'air, les nuages et la surface définit l'albédo planétaire mentionné précédemment~: un albédo élevé contribue à refroidir la surface et l'atmosphère. L'absorption de la lumière ultraviolet/visible, quant à elle, réchauffe directement l'atmosphère (notamment dans la stratosphère, car la troposphère n'est que très faiblement chauffée par les radiations solaires) et contribue à refroidir la surface par extinction du flux solaire incident. Dans le domaine visible, l'extinction est causée principalement par la diffusion et moins par l'absorption. La partie du rayonnement qui parvient à la surface est absorbée par la surface et convertie en énergie interne, c'est-à-dire contribue à élever sa température. On remarque que la surface terrestre ne peut être considérée tout à fait comme un corps noir puisqu'elle n'absorbe pas toute l'énergie incidente~: une petite partie du rayonnement incident est réfléchie par la surface. Cette composante réfléchie par la surface dépend fortement de la nature des sols (océans, forêts, déserts, glace, \ldots) et de leur répartition géographique.

\sk
\subsubsection{Domaine infrarouge}

\sk
Chauffée par l'absorption du rayonnement solaire incident, la surface terrestre se refroidit en émettant du rayonnement surtout dans l'infrarouge d'après la loi de Wien. La troposphère est ainsi principalement chauffée par l'absorption, par les gaz à effet de serre et les nuages, du rayonnement infrarouge émis par la surface. Dans l'infrarouge, à part quelques fenêtres atmosphériques à des longueurs d'onde bien précises, seule une petite partie du flux total émis par la surface s'échappe directement vers l'espace. A leur tour, les gaz à effet de serre émettent du rayonnement dans l'infrarouge, à la fois vers l'espace et vers la surface, ce qui refroidit localement l'atmosphère mais réchauffe la surface par effet de serre comme décrit précédemment avec le modèle à une couche [figure~\ref{fig:modun}]. L'atmosphère piège ainsi~$150$~W~m$^{-2}$ par effet de serre, puisque le rayonnement infrarouge sortant est~$240$~W~m$^{-2}$. On ajoute que la stratosphère est également refroidie par émission infrarouge du gaz carbonique, principalement dans la bande d'absorption à~$15$~$\mu$m. Du point de vue de l'atmosphère, émission infrarouge et refroidissement sont donc intimement liés.
%Émission nette par la vapeur d'eau, l'ozone, le CO2 et les autres gaz à effet de serre : Il s'agit du flux énergétique net émis sous forme de rayonnement énergétique (infrarouge) par l'ensemble des molécules de l'atmosphère. L'émission infrarouge est associée à un refroidissement local. Comme le Corps Noir, les molécules émettent un rayonnement pour se refroidir et équilibrer l'énergie absorbée. L'émission n'a lieu que dans les bandes d'absorption (ou d'émission). Il faut donc que la température locale soit celle du Corps Noir émettant à la longueur d'onde de la bande d'émission. Ainsi, plus on descend dans l'atmosphère plus l'émission se fera par les bandes centrées sur de faibles longueurs d'onde. Émission IR et refroidissement atmosphériques sont doncintimement liés. La stratosphère est principalement refroidie par l'émission IR du gaz carbonique. Ce refroidissement est associé à l'émission par la bande située à 15 μm. Dans la haute stratosphère, la bande d'émission de l'ozone à 9.6 μm permet l’émission IR et le refroidissement atmosphérique. Cependant l'ozone absorbe principalement les radiations solaires et ne peut être considérée comme un gaz à effet de serre (dans la stratosphère). La vapeur d'eau émet également dans la stratosphère dans la bande à 8 μm. La troposphère est principalement refroidie par l'émission de la vapeur d'eau dans la bande située à 6.3 micromètres.

\sk
\subsubsection{Autres échanges d'énergie}

\sk
Le bilan net en surface dans l'infrarouge de $65$~W~m$^{-2}$ est une petite différence entre le flux émis par la surface $\sigma \, {T\e{s}}^4$ et celui reçu depuis l'atmosphère. Si le bilan radiatif est bien équilibré au sommet de l'atmosphère, la surface gagne en moyenne de l'énergie et l'atmosphère en perd. En l'absence d'autres mécanismes de transfert d'énergie, cela conduirait à un refroidissement de l'atmosphère, et à une discontinuité de température à la surface entre le sol et l'air. En pratique, ce déséquilibre radiatif est compensé par des flux qui dépendent des mouvements et des changements de phase dans l'atmosphère
\begin{citemize}
\item de chaleur sensible (transport vertical de chaleur par la conduction et les mouvements de convection) 
\item de chaleur latente (évaporation depuis la surface et condensation dans l'atmosphère) 
\end{citemize}
depuis la surface vers l'atmosphère. Du fait que le transfert d'énergie du sol vers l'atmosphère se fait également sous forme d'un flux de chaleur sensible et latente, le sol n'émet donc que~$396$~W~m$^{-2}$ (au lieu de~$495$~W~m$^{-2}$) ce qui équivaut à une température de~$15^{\circ}$C, soit la température moyenne de la surface terrestre effectivement constatée. En l'absence de convection et de changements d'état dans l'atmosphère, la température de la surface et des basses couches atmosphériques serait beaucoup plus élevée. 
%%% les 240 W/m2 qui sortent sont les mêmes que dans la version sans atmosphère.
%%% noter la fenêtre atmosphérique dans l'infrarouge

%\figun{1.0}{0.4}{\figfrancis/bilan_rad_glob}{Schéma des flux moyens échangés entre la surface de la Terre, l'atmosphère, et l'espace: flux radiatifs ondes courtes (jaune) et infrarouge (rouge), et flux sensibles et latents (violet).}{fig:bilanrad}
\figun{1.0}{0.45}{decouverte/meteo_terre/bilanflux00004.png}{Bilan énergétique moyen de la Terre (en W~m$^{-2}$)~: flux échangés entre la surface de la Terre, l'atmosphère et l'espace. On distingue les flux radiatifs ondes courtes (rayonnement visible, en jaune) des flux radiatifs ondes longues (rayonnement infrarouge, en rouge). Noter les flux sensibles et latents qui ne sont pas relatifs au transfert radiatif. Source~: Planton CNRS editions 2011 ; adapté de Trenberth et al. BAMS 2009}{fig:bilflux}

%%% MANQUE UN TOPO SUR LE FORçAGE RADIATIF ????
%%% POUR REBRANCHER SUR LE CHANGEMENT CLIMATIQUE. VOIR PAYAN 10-12.

\sk
\subsection{Variations géographiques}

\sk
\subsubsection{Influence de la latitude}

\sk
Localement, l'éclairement varie suivant la latitude et la saison, en plus de l'alternance jour/nuit: il est proportionnel à $\cos\theta$ où $\theta$ est l'angle d'incidence avec la surface.
%[figure~\ref{fig:senslat1}]. 
En moyenne annuelle, le maximum d'ensoleillement est donc aux latitudes tropicales, mais il varie au cours de l'année et est même maximal aux pôles pendant l'été local [figure~\ref{fig:senslat2}]~: la durée du jour de 24h fait plus que compenser l'angle d'incidence réduit dû à la latitude élevée (ce qui peut paraître de prime abord contre-intuitif).
%\figside{0.3}{0.1}{\figfrancis/swcoslat.jpg}{Schéma de la relation entre densité de flux du rayonnement incident parallèle et éclairement de la surface suivant l'angle d'incidence.}{fig:senslat1}
\figside{0.65}{0.25}{\figfrancis/swtoaseas}{Cycle saisonnier de l'éclairement dû au rayonnement solaire incident au sommet de l'atmosphère.}{fig:senslat2}
%%%% http://www.energieplus-lesite.be/energieplus/page_16761.htm

\sk
\subsubsection{Rôle des nuages}

\sk
La présence de différents types de nuages est très variable, à la fois géographiquement et dans le temps. Ils ont pourtant une influence très grande sur le bilan radiatif, par deux mécanismes distincts [figure \ref{fig:schemacrf}].
\begin{finger}
\item Effet d'albédo~: les nuages réfléchissent une partie importante du rayonnement solaire incident (par rétro\-diffusion par les gouttes d'eau). Cet effet est d'autant plus fort que le nuage contient d'eau et que les gouttes sont fines. Un nuage très réfléchissant apparaitra sombre vu d'en dessous. Au total, les nuages sont responsables des 2/3 de l'albédo planétaire.
\item Effet de serre~: Les gouttes d'eau (ou la glace) des nuages sont d'excellents absorbants dans l'infrarouge. Un nuage même peu épais absorbe donc très rapidement tout le rayonnement infrarouge provenant des couches plus basses. Il émet lui même vers le haut du rayonnement suivant sa propre température: $\sigma T_N^4$ où $T_N$ est la température au sommet du nuage. Un nuage au sommet élevé (donc froid) aura donc un effet de serre très important.
\end{finger}
Au final, l'effet d'albédo l'emporte pour les nuages bas (type stratus), qui sont typiquement épais (albédo élevé) et dont le sommet est chaud. Au contraire, les fins nuages d'altitude (cirrus) ont un albédo faible mais un sommet très froid donc ont un effet net réchauffant. Pour les nuages de type orageux, qui sont épais avec un sommet froid, les deux effets tendent à se compenser.
\figside{0.6}{0.2}{\figfrancis/schema_crf}{Schema de l'influence des nuages sur le bilan radiatif: effet d'albédo dans le visible (jaune), et absorption et émission dans l'infrarouge (rouge). L'effet de serre vient du rayonnement émis vers l'espace plus faible que celui venant de la surface, qui est absorbé.}{fig:schemacrf}
%% nuages comme les lunettes dans la caméra infrarouge. faire également référence à la vidéo tirée du satellite.

\sk
\subsection{Moyennes annuelles~: cartes}

\sk
On présente dans cette section des cartes des différents termes du bilan radiatif de la terre, tels qu'observés par satellite depuis l'espace. On observe un bilan moyen sur une année qui est variable en fonction de la position géographique : si l'on a, en moyenne, égalité entre absorption du rayonnement solaire incident et émission de la Terre vers l’espace, ce n’est plus vrai si on considère une région donnée. 
\begin{finger}
\item
Le flux solaire absorbé (figure \ref{fig:swtoa}) montre essentiellement une dépendance en latitude. L'effet de l'ensoleil\-lement au sommet de l'atmosphère, plus fort dans les tropiques, est amplifié par un albédo élevé aux latitudes polaires à cause de la présence de neige et de glace au sol. En plus de ces variations en latitudes, on observe des différences locales dûes à l'albédo des régions nuageuses (zone de convergence intertropicale, bords est des océans) ou du sol (Sahara).
\item
Le flux infrarouge sortant au sommet de l'atmosphère (figure \ref{fig:olr}) a lui aussi une structure en latitude, mais moins marquée que pour les ondes courtes: les hautes latitudes, plus froides, émettent moins de rayonnement. On voit d'autre part nettement le flux sortant plus faible dans les régions humides des tropiques (continents et zone de convergence) où des nuages convectifs d'altitude élevée se forment.
\item
La signature des régions humides est nettement plus faible sur la carte du bilan net au sommet de l'atmosphère (figure \ref{fig:nettoa}); les effets de serre et d'albédo des nuages se compensant en grande partie. On retrouve par contre un bilan moins positif dans les régions où un albédo élevé provient du sol (Sahara) ou de nuages bas (Chili, Californie). D'autre part, on observe un gain net d'énergie dans les tropiques, et une perte dans les hautes latitudes; la distribution du bilan dans le visible qui est plus inégale que celle dans l'infrarouge détermine donc la structure globale. 
\end{finger}

\figside{0.65}{0.25}{\figfrancis/erbe_stoa_ann}{Rayonnement visible absorbé par la Terre, en moyenne annuelle (données ERBE).}{fig:swtoa}
\figside{0.65}{0.25}{\figfrancis/erbe_olr_ann}{Rayonnement infrarouge sortant au sommet de l'atmosphère, en moyenne annuelle.}{fig:olr}
\figside{0.65}{0.25}{\figfrancis/erbe_ntoa_ann}{Flux net absorbé par la Terre (visible  - infrarouge sortant) en moyenne annuelle.}{fig:nettoa}

%• Disparités régionales :
%o Sahara en été boréal : fort albédo + forte perte IR + capacité
%calorifique faible + atmosphère sèche
%o Océan : albédo faible compense la perte radiative IR + forte
%capacité calorifique

\sk
Ces excès et déficit d'énergie locaux doivent, en moyenne, être compensés par des transports d'énergie par les circulations atmosphérique et océanique. Ils fournissent la source d'énergie pour la \voc{dynamique atmosphérique} qui va contribuer à répartir l'énergie des régions excédentaires en énergie vers les régions déficitaires en énergie~[figure~\ref{fig:hadley}].

\figun{0.75}{0.35}{decouverte/cours_meteo/energiedyn.png}{Schéma représentant les latitudes où l'atmosphère est excédentaire ou, au contraire, déficitaire en énergie. La courbe bleue représente l'énergie radiative reçue du Soleil, principalement dans les courtes longueurs d'onde (noté \emph{shortwave} sur la figure, correspond au rayonnement visible et ultraviolet). La courbe rouge représente l'émission par la surface terrestre, principalement dans les longues longueurs d'onde (noté \emph{longwave} sur la figure, correspond au rayonnement infrarouge). Une circulation atmosphérique de grande échelle se met en place entre les régions excédentaires (équatoriales et tropicales) et déficitaires (hautes latitudes).}{fig:hadley}

%\end{document}

%\sk \subsection{Rétroactions} Préciser les rétroactions en jeu Indiquer le sens de ces rétroactions. Exemple terrestre~: \begin{itemize} \item Stefan-Boltzman : positive ou négative \item Glaces : positive ou négative \item Vapeur d'eau : positive ou négative \item Nuages : positive ou négative \end{itemize} 
%\visible<2->{\vskip 0.5cm\ebloc{}{Applications~:~changement climatique, paléo-climats, évolution des planètes du système solaire, climat des exoplanètes}}} \note{Peut être en général étendu aux autres planètes.\\ Les processus de rétroactions climatiques peuvent amplifier (on parle alors de rétroactions positives) ou réduire (rétroaction négative) la réponse à une perturbation initiale\\ SB : Si la température augmente alors la perte par rayonnement augmente : feedback négatif très fort\\ glace (albedo) : Si la température augmente alors la glace diminue et donc le rayonnement solaire absorbé augmente ce qui augmente la température feedback positif, mais attention au contrôle par la taille de la calotte polaire.\\ vapeur d’eau : l’augmentation de la température tend à favoriser l'évaporation car l'équilibre L-V est déplacé, de fait augmentation de l'humidité ie le contenu en vapeur d’eau de l’atmosphère, ce qui augmente l’effet de serre et donc la température de surface\\ nuages :  Si la température augmente et induit plus de nuages qui réfléchissent plus d’énergie solaire alors la température diminue. Cependant par effet de serre des nuages, la température augmente…. Au contraire, si le climat se refroidit, la couverture neigeuse hivernale persistera plus longtemps. Or cette couverture blanche (d’albédo plus élevé que le sol) augmente la réflexion de l'énergie solaire et donc diminue le chauffage de la surface par le Soleil. Il en résulte un refroidissement de la surface qui amplifie le refroidissement climatique initial feedback positif si refroidissement, inconnu si réchauffement (a priori négatif pour les nuages bas)} \note{[FACULTATIF] GES cycle glaciaire vs. interg.~: une entrée en glaciation entraîne une baisse de la teneur en gaz à effet de serre (CO2, H2O vapeur et CH4) dans l'atmosphère par suite des modifications du climat (refroidissement de la surface terrestre et modification de la circulation océanique profonde); cette diminution atténue l'effet de serre initial et donc amplifie le refroidissement en cours. Inversement une déglaciation entraîne une augmentation des mêmes gaz à effet de serre, ce qui, cette fois, contribue à accentuer le réchauffement feedbak positif}

%9. Perspectives
%9.1. Exemples de rétroactions
%Les processus de rétroactions climatiques peuvent amplifier (on parle alors de
%rétroactions positives) ou réduire (rétroaction négative) la réponse à une
%perturbation initiale et sont donc centraux pour simuler correctement l’évolution
%du climat.
%• La rétroaction de Stefan-Boltzmann :
%Si la température augmente alors la perte par rayonnement augmente :
%-- feedback négatif très fort
%• La rétroaction de la glace et de l’albédo :
%Si la température augmente alors la glace diminue et donc le rayonnement
%solaire absorbé augmente ce qui augmente la température
%-- feedback positif
%• La rétroaction des nuages
%Si la température augmente et induit plus de nuages qui réfléchissent plus
%d’énergie solaire alors la température diminue. Cependant par effet de serre des
%nuages, la température augmente…. Au contraire, si le climat se refroidit, la
%couverture neigeuse hivernale persistera plus longtemps. Or cette couverture
%blanche (d’albédo plus élevé que le sol) augmente la réflexion de l'énergie
%solaire et donc diminue le chauffage de la surface par le Soleil. Il en résulte un
%refroidissement de la surface qui amplifie le refroidissement climatique initial
%-- feedback positif si refroidissement, inconnu si réchauffement (a priori négatif
%pour les nuages bas).
%• Rétroaction des gaz à effet de serre au cours d’un cycle glaciaireinterglaciaire
%:
%une entrée en glaciation entraîne une baisse de la teneur en gaz à effet de serre
%(CO2, H2O vapeur et CH4) dans l'atmosphère par suite des modifications du
%climat (refroidissement de la surface terrestre et modification de la circulation
%océanique profonde); cette diminution atténue l'effet de serre initial et donc
%amplifie le refroidissement en cours. Inversement une déglaciation entraîne une
%augmentation des mêmes gaz à effet de serre, ce qui, cette fois, contribue à
%accentuer le réchauffement
%-- feedbak positif
%9.2. Convection et conduction
%Il faut tenir compte de la combinaison des effets radiatifs et convectifs pour
%comprendre la structure verticale de la température de l'atmosphère et la
%température de la surface.
%Les premiers modèles remontent au milieu des années 60 et ont depuis été
%remplacés par des formes très élaborées. Ils restent intéressants en étant d'une
%part la première étape vers les modèles à trois dimensions et d'autre part un outil
%intéressant pour l’estimation de l’impact du CO2 sur le climat de la Terre. Des
%versions planétaires ont été développées en astrophysique pour Mars etc …

\chapter{Bases de thermodynamique de l'atmosphère}

\dictum[Jean-Paul Sartre, 1948]{Torricelli a inventé la pesanteur de l'air, je dis qu'il l'a inventée plutôt que découverte, parce que, lorsqu'un objet est caché à tous les yeux, il faut l'inventer de toutes pièces pour pouvoir le découvrir.}

\bk
En filigrane des chapitres précédents sur le bilan radiatif, il apparaît que l'influence de la dynamique atmosphérique et des changements de phase ne peut être négligée. Avant de considérer ces phénomènes, il convient de se donner les outils conceptuels de thermodynamique pour caractériser l'état de l'atmosphère. Il ne s'agit pas de proposer un traité complet de thermodynamique, mais d'exprimer les équations et les concepts utiles pour décrire les phénomènes atmosphériques. 

%%% rappels GP, variables intensives, hauteur d'échelle

\mk
\section{Parcelle d'air}

\sk
\subsection{Définition et caractérisation}

\sk
On rappelle comme indiqué dans le chapitre d'introduction que l'atmosphère est composée d'un ensemble de molécules microscopiques et que l'on s'intéresse aux effets de comportement d'ensemble, qualifiés de \voc{macroscopiques}. Les variables thermodynamiques utilisées pour décrire l'atmosphère (pression~$P$, température~$T$, densité~$\rho$) sont des grandeurs macroscopiques \voc{intensives} dont la valeur ne dépend pas du volume d'air considéré. 
%Une façon d'y parvenir est d'utiliser des grandeurs volumiques ou massiques.

\sk
Le système que l'on étudie est appelé \voc{parcelle d'air}. Il s'agit d'un volume d'air dont les dimensions sont %à la fois
\begin{citemize}
\item assez grandes pour contenir un grand nombre de molécules et pouvoir moyenner leur comportement, afin d'exprimer un équilibre thermodynamique~;
\item assez petites par rapport au phénomène considéré, afin de décrire le fluide atmosphérique de façon continue~; la parcelle d'air peut donc être considérée comme un volume élémentaire.
\end{citemize}
On peut donc supposer que les variables macroscopiques d'intérêt sont quasiment constantes à l'échelle de la parcelle. Autrement dit, une parcelle est caractérisée par sa pression~$P$, sa température~$T$, sa densité~$\rho$. Les limites d'une parcelle sont arbitraires, mais ne correspondent pas en général à des barrières physiques. 

\sk
Tout le but de ce chapitre est de décrire les relations thermodynamiques qui lient les grandeurs intensives qui caractérisent l'état de la parcelle. Une première de ces relations a été obtenue en introduction~: il s'agit de l'équation des gaz parfaits pour l'air atmosphérique, qui relie les trois paramètres intensifs $P$, $T$ et $\rho$ 
\[ \boxed{ P = \rho \, R \,T } \] 
avec la \voc{constante de l'air sec} $R=\frac{R^*}{M}$ où~$R^*$ est la constante des gaz parfaits et~$M$ est la masse molaire de l'air. On rappelle que sur Terre~$R = 287$~J~K$^{-1}$~kg$^{-1}$. L'état thermodynamique d'une parcelle d'air est donc déterminé uniquement par deux paramètres sur les trois~$P$, $T$, $\rho$. Pour les applications météorologiques, on caractérise en général l'état de la parcelle par sa pression~$P$ et sa température~$T$, plus aisées à mesurer, par exemple via des ballons-sondes, que la densité~$\rho$.

\sk
\subsection{Parcelle et environnement} \label{parcenv}

\sk
Une parcelle est en \voc{équilibre mécanique} avec son environnement, c'est-à-dire que la pression de la parcelle~$P\e{p}$ et la pression de l'environnement~$P\e{e}$ dans lequel elle se trouve sont égales
\[ \boxed{P\e{p} = P\e{e}} \]
Néanmoins, une parcelle n'est pas en général en \voc{équilibre thermique} avec son environnement, c'est-à-dire que la température de la parcelle et la température de l'environnement dans lequel elle se trouve ne sont pas nécessairement égales
\[ \boxed{T\e{p} \neq T\e{e}} \]
Cette dernière propriété provient du fait que l'air est un très bon isolant thermique\footnote{Une telle propriété est utilisée dans le principe du double vitrage}.

\mk
\section{Équilibre hydrostatique}

\sk
Le chapitre d'introduction indique que la pression décroît exponentiellement avec l'altitude. On en donne ici une démonstration, en obtenant une loi dite hydrostatique dont les implications sont nombreuses. 

\sk
\subsection{Bilan des forces}

\sk
On considère une parcelle d'air cubique de dimensions élémentaires~$(\dd x,\dd y, \dd z)$, au repos et située à une altitude~$z$. La pression atmosphérique vaut~$P(z)$ sur la face du dessous et~$P(z+\dd z)$ sur la face du dessus. Pour le moment, on ne considère pas de variations de pression~$P$ selon l'horizontale\footnote{En pratique, les variations de pression selon l'horizontale sont effectivement négligeables par rapport aux variations de pression selon la verticale. On revient sur ce point dans le chapitre consacré à la dynamique}. Il y a équilibre des forces s'exerçant sur cette parcelle. On appelle \voc{équilibre hydrostatique} l'équilibre des forces selon la verticale, à savoir~:
\begin{citemize}
\item Son poids de module\footnote{On néglige les variations de~$g$ avec~$z$.}~$m \, g$ (où~$m = \rho \, \dd x \, \dd y \, \dd z$) dirigé vers le bas
\item La force de pression sur la face du dessous de module~$P(z) \, \dd x \, \dd y$ dirigée vers le haut
\item La force de pression sur la face du dessus de module~$P(z+\dd z) \, \dd x \, \dd y$ dirigée vers le bas
\item La force de viscosité qui est négligée
\end{citemize}
On note que, contrairement au poids qui s'applique de façon homogène sur tout le volume de la parcelle d'air, les forces de pression s'appliquent sur les surfaces frontières de la parcelle d'air. 
Pour une parcelle au repos, la résultante selon la verticale des forces de pression exercées par le fluide environnant (ici, l'air) n'est autre que la poussée d'Archimède.
%%Par ailleurs, l'équilibre hydrostatique suppose implicitement que la parcelle est à l'équilibre thermique avec son environnement~$T\e{p} = T\e{e}$ soit~$\rho\e{p} = \rho\e{e}$. On aborde le cas général où~$T\e{p} \neq T\e{e}$ dans le chapitre suivant pour définir les notions de stabilité.

\sk
L'équilibre hydrostatique de la parcelle s'écrit donc
\[ - \rho \, g \, \dd x \, \dd y \, \dd z + P(z) \, \dd x \, \dd y - P(z+\dd z) \, \dd x \, \dd y = 0 \qquad \Rightarrow \qquad - \rho \, g \, \dd z + P(z) - P(z+\dd z) = 0 \]
soit en introduisant la dérivée partielle suivant~$z$ de~$P$
\[ \frac{P(z+\dd z) - P(z)}{\dd z} = \boxed{ \Dp{P}{z} = - \rho \, g } \]
Cette relation est appelée \voc{équation hydrostatique} (ou relation de l'équilibre hydrostatique). Elle indique que, pour la parcelle considérée, la résultante des forces de pression selon la verticale équilibre la force de gravité. En principe, cette équation est valable pour une parcelle au repos. Par extension, elle est valable lorsque l'accélération verticale d'une parcelle est négligeable devant les autres forces. L'équation hydrostatique est donc approximativement valable pour les mouvements de grande échelle considérés dans toute la suite de ce cours. 

\sk
Si l'on intègre la relation hydrostatique entre deux niveaux~$z_1$ et~$z_2$ où la pression est~$P_1$ et~$P_2$, on obtient
\[ \Delta P = P_2 - P_1 = - g \, \int_{z_1}^{z_2} \rho \, \dd z \]
L'équilibre hydrostatique peut donc s'interpréter de la façon éclairante suivante~: la différence de pression entre deux niveaux verticaux est proportionnelle à la masse d'air (par unité de surface) entre ces niveaux. Une autre façon équivalente de formuler cela est de dire que la pression atmosphérique à une altitude~$z$ correspond au poids de la colonne d'air située au-dessus de l'altitude~$z$, exercé sur une surface unité de~$1$~m$^2$. Il s'ensuit que la pression atmosphérique~$P$ est décroissante selon l'altitude~$z$. Ainsi, la pression peut être utilisée pour repérer une position verticale à la place de l'altitude. En sciences de l'atmosphère, la pression atmosphérique est une coordonnée verticale plus naturelle que l'altitude~: non seulement elle est directement reliée à la masse atmosphérique par l'équilibre hydrostatique, mais elle est également plus aisée à mesurer.

\sk
\subsection{\'Equation hypsométrique}

\sk
\subsubsection{\'Echelle de hauteur}

\sk
En exprimant la densité~$\rho$ en fonction de l'équation des gaz parfaits, l'équilibre hydrostatique s'écrit
\[ \Dp{P}{z} = - g \, \frac{P}{RT} \]
On peut intégrer cette équation si on suppose que l'on connaît les variations de~$T$ en fonction de $P$ ou $z$. On considère que l'on peut négliger les variations de pression selon l'horizontale donc transformer les dérivées partielles~$\partial$ en dérivées simples~$\dd$. On effectue ensuite une séparation des variables
\[R \, T \, \frac{\dd P}{P} = - g \, \dd z\]

\sk
Cette équation peut s'écrire sous une forme dimensionnelle simple à retenir
\[ \boxed{ \frac{\dd P}{P} = - \frac{\dd z}{H(z)} \qquad \text{avec} \qquad H(z) = \frac{R \, T(z)}{g} } \]
La grandeur~$H$ se dénomme l'\voc{échelle de hauteur} et dépend des variations de la température~$T$ avec l'altitude~$z$. L'équation ci-dessus indique bien que la pression décroît avec l'altitude selon une loi exponentielle (voir Figure~\ref{fig:presvert}). Cette loi peut être plus ou moins complexe selon la fonction~$T(z)$. On peut néanmoins fournir une illustration simple du résultat de l'intégration dans le cas d'une atmosphère isotherme~$T(z)=T_0$
\[ P(z) = P(z=0) \, e^{-z/H} \qquad \text{avec} \qquad H = R \, T_0 / g \]

\sk
\subsubsection{\'Epaisseur des couches atmosphériques~: équation hypsométrique}

\sk
Le cas isotherme est simpliste et peu rencontré en pratique dans l'atmosphère. On se place dans le cas plus général, bien que toujours simplifié, de deux niveaux atmosphériques~$a$ et~$b$ entre lesquels la température ne varie pas trop brusquement avec l'altitude~$z$. On réalise alors l'intégration entre les deux niveaux~$a$ et~$b$
\[R \, T \, \frac{\dd P}{P} = - g \, \dd z \qquad \Rightarrow \qquad R \, \int_a^b T\, \frac{\dd P}{P} = - g \, \int_a^b dz\]
puis on définit la température moyenne de la couche atmosphérique entre~$a$ et~$b$ avec une moyenne pondérée
\[ \langle T \rangle = \frac{\int_a^b T \, \frac{\dd P}{P}}{\int_a^b \frac{\dd P}{P}} \]
pour obtenir finalement
\[R \, \langle T \rangle \, \int_a^b \frac{\dd P}{P} = - g \, \int_a^b dz
\qquad \Rightarrow \qquad \boxed{ g \, (z_a - z_b) = R \, \langle T \rangle \ln \left( \frac{P_b}{P_a} \right) } \]
Cette relation est appelée \voc{équation hypsométrique}. Elle correspond à une formulation utile en météorologie du principe que \ofg{l'air chaud se dilate}. Les conséquences de l'équation hypsométrique peuvent s'exprimer de diverses façons équivalentes.
\begin{citemize}
\item Pour une masse d'air donnée, une couche d'air chaud est plus épaisse.
\item La distance entre deux isobares est plus grande si l'air est chaud.
\item La pression diminue plus vite selon l'altitude dans une couche d'air froid.
\end{citemize}
En passant le résultat précédent au logarithme, on note que l'on retrouve toujours le fait que la pression diminue avec l'altitude selon une loi exponentielle. En notant l'échelle de hauteur moyenne~$\langle H \rangle$, on a
\[ P_b = P_a \, e^{ - \frac{z_b - z_a}{\langle H \rangle}} \qquad \text{avec} \qquad \langle H \rangle = \frac{R \, \langle T \rangle}{g} \]

\sk
\subsection{Applications pratiques}

\sk
\subsubsection{Expérience de Pascal}

\sk
Depuis son invention en 1643 par Torricelli, disciple de Galilée, le \voc{baromètre} est l'instrument de référence pour mesurer la pression atmosphérique à la surface de l'atmosphère terrestre\footnote{Le but initial de Torricelli était de parvenir le premier à maintenir artificiellement en laboratoire une chambre sous vide. Néanmoins, il est également reporté que l'invention du baromètre découle des réflexions de Torricelli autour de l'impossibilité, constatée en pratique, de pomper l'eau d'un fleuve au-dessus d'un certain niveau.}. Trois ans après son invention, le baromètre était déjà utilisé pour sa première application en sciences de l'atmosphère~: donner une preuve expérimentale de l'équilibre hydrostatique qui gouverne la stratification en pression de l'atmosphère. Autrement dit, le baromètre est un moyen indirect de mesurer la masse de l'atmosphère à travers la pression de surface. Blaise Pascal montra ainsi, par des mesures respectivement sur la Tour Saint-Jacques à Paris ($ \Delta z = 52 \U{m} $ au-dessus du sol) et sur le Puy-de-Dôme en Auvergne ($ \Delta z = 1000 \U{m} $ au-dessus du sol), que la pression atmosphérique varie avec l'altitude\footnote{Le texte original du traité de Pascal est disponible sur Gallica~: \url{http://gallica.bnf.fr/ark:/12148/bpt6k105083f}}.

\sk
L'équation hypsométrique permet de retrouver l'écart relatif en pression mesuré par Blaise Pascal entre le sol et le haut de la Tour Saint-Jacques. Comme les variations mesurées sont petites, on peut les assimiler aux différentielles et on peut négliger les variations de température avec l'altitude. Les variations relatives de pression mesurées par Pascal peuvent alors s'écrire
\[ \f{\Delta P}{P} \simeq \f{g}{R\,T} \, \Delta z \]
L'application numérique avec~$T = 288$~K donne une variation~$\Delta P / P = 0.62 \%$. La variation de pression détectée par Blaise Pascal\footnote{On note que, par une heureuse coincidence, la variation de pression entre le pied et le sommet de la Tour Saint-Jacques est de l'ordre de grandeur de la pression atmosphérique à la surface de Mars, ce qui permet de se représenter la finesse de l'atmosphère sur cette planète, comparé à notre Terre.} est donc d'environ~$6$~hPa (avec la valeur standard de la pression de surface~$P_0 = 1013$~hPa). Bien que cette baisse de pression soit détectable à l'aide du tube de Torricelli, Blaise Pascal a reproduit l'expérience au Puy-de-Dôme avec un écart~$\Delta z$ plus grand pour une meilleure précision quantitative. 

\sk
\subsubsection{Pression au niveau de la mer, altimétrie et cartes météorologiques}

\sk
Comme le montre la figure~\ref{fig:press} (haut), la pression~$P$ à la surface de la Terre est au premier ordre sensible à l'altimétrie (hauteur topographique), puisque la pression correspond au poids de la colonne d'air située au-dessus du point considéré. Pour produire des cartes de prévision du temps, on souhaite éliminer du champ de pression~$P$ cette composante topographique de premier ordre et mettre en évidence les variations de second ordre digne d'intérêt en météorologie.

\figsup{0.62}{0.32}{decouverte/cours_meteo/SURFPRESS/outputvar134_200.png}{decouverte/cours_meteo/SURFPRESS/outputvar151_200.png}{Champs de pression prédits au 01/09/2009 par les réanalyses ERA-INTERIM de l'ECMWF (Centre Européen de Prévision du Temps à Moyen Terme). La réprésentation graphique est basée sur une projection stéréographique polaire centrée sur le pôle Nord et les structures topographiques sont ajoutées dans le fond de la figure pour repérage. En haut, le champ de pression brut est tracé en hPa~: les valeurs de pression les plus basses correspondent aux reliefs topographiques les plus élevés. En bas, le champ de pression ramené au niveau de la mer est tracé en hPa~: la composante de premier ordre topographique sur le champ de pression a disparu pour laisser place aux composantes météorologiques de la pression~: dépressions (zones de basses pression) en bleu et anticyclones (zones de haute pression) en rouge. On peut d'ailleurs noter dans ce champ de pression normalisé l'activité ondulatoire de l'atmosphère aux moyennes latitudes. Les cartes de pression des bulletins météorologiques sont exclusivement des cartes de pression ramenées au niveau de la mer comme celle-ci.}{fig:press} 

\sk
Quand la pression de surface est mesurée à une station située à une altitude~$z \ll H$, on peut utiliser l'équation hypsométrique avec la température mesurée à la station pour déterminer une valeur approximative de la pression au niveau de la mer à~$z=0$. On suppose fréquemment que la température décroît linéairement avec l'altitude~$z$ selon un taux constant négatif~$\Gamma\e{e}$ en~$^{\circ}$C/km (ou K/km). On appelle d'ailleurs la loi~$T = T_0 + \Gamma\e{e} \, z$ le profil de l'atmosphère standard. En intégrant entre le niveau de la mer ($z=0, P=P_0$) et la station à ($z,P$), on obtient:
\[ \ln \left( \frac{P_0}{P} \right) = \frac{g}{R \, \Gamma\e{e}} \, \ln \left( \frac{T_0 + \Gamma\e{e} \, z}{T_0} \right) \]
\[ \Rightarrow \qquad P_0 = P \left( 1 + \frac{\Gamma\e{e} \, z}{T_0} \right)^{\frac{g}{R\,\Gamma\e{e}}} \]
La carte météorologique sur la Figure~\ref{fig:press} bas est obtenue en employant cette équation. L'équation qui précède est aussi utilisée avec $P_0=1013.25$~hPa par les altimètres des avions de ligne pour convertir~$P$ mesurée en~$z$.

\mk
\section{Premier principe et thermodynamique de l'air sec} 

\sk
\subsection{\'Energie interne}

\sk
Un système thermodynamique possède, en plus de son énergie d'ensemble (cinétique, potentielle), une \voc{énergie interne}~$U$. Comme la température~$T$, l'énergie interne~$U$ est une grandeur macroscopique qui représente les phénomènes microscopiques au sein d'un gaz. Le premier principe indique que les variations d'énergie interne sont égales à la somme du travail et de
la chaleur algébriquement reçus~:
\[ \dd U = \delta W + \delta Q\]

\sk
Dans le cas d'un gaz parfait, l'énergie potentielle d'interaction des molécules du gaz est négligeable, et l'énergie interne est égale à l'énergie cinétique des molécules, qui dépend seulement de la température. On peut montrer que $U = n \, \frac{\zeta \, R^* \, T}{2}$ où $\zeta$ est le nombre de degrés de liberté des molécules. Pour un gaz (principalement) diatomique comme l'air, $\zeta = 5$. 

\sk
Dans le cas de variations quasi-statiques d'un gaz, ce qui est supposé être le cas dans l'atmosphère, le travail s'exprime en fonction de la pression~$P$ du gaz et de la variation de volume~$\dd V$
\[ \delta W = - P \, \dd V \]

\sk
\subsection{Chaleurs molaires et enthalpie}

\sk
L'expérience montre que la quantité de chaleur échangée au cours d'une transformation à volume ou pression constant est proportionnelle à la variation de température du système~: $\delta Q = n \, C_V^* \, \dd T$ à volume constant, $\delta Q = n \, C_P^* \, \dd T$ à pression constante. $C_P^*, C_V^*$ sont les \voc{chaleurs molaires}, également appelées \voc{capacités calorifiques}. Il s'agit de l'énergie qu'il faut fournir à un gaz pour faire augmenter sa température de~$1$~K dans les conditions indiquées (à volume constant ou à pression constante). Pour une transformation à volume constant (isochore), $\dd U = \delta Q$ donc $C_V^*=\frac{\zeta \, R^*}{2}$.

\sk
Pour l'étude de l'atmosphère, toujours dans la logique de travailler sur des grandeurs intensives, il est  bien plus utile de s'intéresser aux variations de pression plutôt qu'à celles de volume. On utilise donc l'\voc{enthalpie}~$H = U + P \, V$. On a alors par dérivation $ \dd H = \dd U + \dd (P\,V) $ puis, en utilisant le premier principe
\[ \dd H = V \, \dd P + \delta Q \]
Pour une transformation à pression constante (isobare) on a $\dd H = \delta Q$. On en déduit pour une transformation quelconque\footnote{
D'autre part, en utilisant conjointement la dérivation de l'équation d'état du gaz parfait~$\dd (P\,V) = n \, R^* \, \dd T$ et l'expression de l'énergie interne~$U = n \, C_V^* \, \dd T$, on obtient $\dd H = n \, C_V^* \, \dd T + n \, R^* \, \dd T$ pour une transformation quelconque. On en déduit la relation de Mayer \[ C_P^* = C_V^* + R^* = \frac{(\zeta+2) \, R^*}{2}\]
} 
que $\dd H = n \, C_P^* \, dT$, ce qui permet d'écrire
\[ n \, C_P^* \, dT = V \, \dd P + \delta Q \]

\sk
\subsection{Transformations dans l'atmosphère~: cas général}

\sk
Afin de travailler sur des grandeurs intensives, on divise la relation précédente par la masse~$m$ de la parcelle pour obtenir
\[ C_P \, \dd T = \frac{\dd P}{\rho} + \delta q \]
où $\delta q$ est la chaleur massique reçue et $C_P = C_P^* / M$ est la \voc{chaleur massique de l'air} ($C_P$=1004 J~K$^{-1}$~kg$^{-1}$). Nous disposons alors d'une autre version du premier principe, très utile en météorologie et valable pour une transformation quelconque d'une parcelle d'air
\[ \boxed{ \underbrace{\textcolor{white}{\frac{R^2}{C_P}} \dd T \textcolor{white}{\frac{R}{C_P}}}_{\text{variation de température de la parcelle}} = \underbrace{\frac{R}{C_P} \, \frac{T}{P} \, \dd P}_{\text{travail expansion/compression}} + \underbrace{\frac{1}{C_P} \, \delta q}_{\text{chauffage diabatique}} } \]

\sk
Autrement dit, la température de la parcelle augmente si elle subit une compression ($\dd P > 0$) et/ou si on lui apporte de la chaleur ($\delta q > 0$). La température de la parcelle à l'inverse diminue si elle subit une détente ($\dd P < 0$) et/ou si elle cède de la chaleur à l'extérieur ($\delta q < 0$). Il est donc important de retenir que la température de la parcelle peut très bien varier quand bien même la parcelle n'échange aucune chaleur avec l'extérieur~: dans ce cas, $\delta q = 0$ et l'on parle de \voc{transformation adiabatique}. 

\sk
L'équation fondamentale ci-dessus est directement dérivée du premier principe, mais prend une forme plus pratique en sciences de l'atmosphère du fait que les transformations que subit une parcelle atmosphérique se réduisent en général aux transformations \voc{isobares} (à pression constante $\dd P = 0$) et aux transformations \voc{adiabatiques} (sans échanges de chaleur avec l'extérieur $\delta q = 0$). Les transformations isothermes, au cours de laquelle la température de la parcelle ne varie pas, sont très rarement rencontrées en sciences de l'atmosphère.

\sk
\subsection{Transformations non adiabatiques}

\sk
Dans le cas où la transformation n'est pas adiabatique, les échanges de chaleur~$\delta q$ d'une parcelle d'air avec son environnement sont non nuls et peuvent s'effectuer par~:
\begin{itemize}
\item Transfert radiatif~: l'atmosphère se refroidit en émettant dans l'infrarouge, ou se réchauffe en absorbant du rayonnement électromagnétique dans l'infrarouge [cas des gaz à effet de serre] ou dans le visible [cas de l'ozone dans la stratosphère].
%Ces échanges sont faibles et peuvent être négligés sauf à l'échelle de la circulation générale\footnote{Le refroidissement/réchauffement peut être localement élevé au sommet/à la base de nuages.}
\item Condensation ou évaporation d'eau~: ce point est abordé dans le chapitre suivant (ceci n'a lieu que lorsque l'air est à saturation).
\item Diffusion moléculaire (conduction thermique)~: ces transferts sont très négligeables sauf à quelques centimètres du sol.
\end{itemize}
Un cas notamment souvent cité en météorologie est celui d'une parcelle d'air située proche du sol, à la tombée de la nuit, qui subit peu de variations de pression ($\dd P \sim 0$) mais dont la température diminue sous l'effet du refroidissement radiatif ($\delta q < 0$). Ceci explique la présence de rosée sur le sol et de brouillard proche de la surface au petit matin, comme il est décrit plus en détail dans le chapitre suivant.

\sk
\subsection{Transformations adiabatiques}

\sk
Dans de nombreuses situations en sciences de l'atmosphère, on peut considérer que l'évolution de la parcelle est adiabatique et se fait sans échange de chaleur avec l'extérieur ($\delta q=0$). En vertu de l'équilibre hydrostatique qui relie pression~$P$ et altitude~$z$~:
\begin{citemize}
\item une parcelle dont l'altitude~$z$ augmente sans apport extérieur de chaleur, subit une \voc{ascendance} adiabatique, donc une détente telle que~$\dd P < 0$ et sa température diminue ;
\item inversement, une parcelle dont l'altitude~$z$ diminue sans apport extérieur de chaleur, subit une \voc{subsidence} adiabatique, donc une compression telle que~$\dd P > 0$ et sa température augmente. 
\end{citemize}

\sk
Dans le cas où la transformation est adiabatique, pression et température sont intimement liées en vetu du premier principe. La version du premier principe encadrée ci-dessus avec~$\delta q = 0$ indique
\[ \dd T = \frac{R}{C_P} \, \frac{T}{P} \, \dd P\]
\[ \Rightarrow \qquad \frac{\dd T}{T} - \frac{R}{C_P} \, \frac{\dd P}{P} = 0 \]
soit par intégration
\[ T \, P^{- \kappa} = \text{constante} \qquad \text{avec} \qquad \kappa = R / C_P \]
Autrement dit, dans le cas où une parcelle subit une transformation adiabatique, sa température varie proportionnellement à~$P^{\kappa}$. Il s'agit d'une version, avec les grandeurs intensives utiles en sciences de l'atmosphère, de l'équation~$P\,V^{\gamma}$, avec $\gamma = C_P / C_V$, vue dans les cours de thermodynamique générale pour les transformations adiabatiques.

%% La température ne nous donne pas des informations fiables sur les échanges de chaleur d'une parcelle avec l'extérieur. Pour ce faire, on se base sur la température potentielle.
%On définit la {\em température potentielle} $\Theta$ par:
%\begin{equation}
%  \Theta=T\cdot\left(\frac{P}{P_0}\right)^{-\kappa}
%  \label{theta}
%\end{equation}
%où $P_0$ est une pression de référence égale à 1000 hPa. $\Theta$ a donc la
%dimension d'une température (on l'exprime en Kelvin), et est conservée au
%cours de transformations adiabatiques. $\Theta$ est égale à la température
%d'une parcelle ramenée de façon adiabatique à une pression $P_0$.

\sk
\subsection{Gradient adiabatique sec} \label{adiabsec}

\sk
Il s'ensuit directement une loi simple des variations de température avec l'altitude d'une parcelle qui ne subit que des transformations adiabatiques. Considérons le cas d'une parcelle subissant un déplacement vertical quasi-statique et adiabatique tel que~$\delta q = 0$. Elle vérifie en première approximation l'équilibre hydrostatique~$\dd P\e{p} / \rho = - g \, \dd z$. La version du premier principe encadrée ci-dessus indique alors que
\[  \dd T\e{p}  = - \frac{g}{C_P} \, \dd z \]
d'où on tire le profil vertical adopté dans l'atmosphère sèche par une parcelle ne subissant pas d'échange de chaleur avec l'extérieur
\[  \boxed{ \ddf{T\e{p}}{z}  = \Gamma\e{sec} \qquad \text{avec} \qquad \Gamma\e{sec} = \frac{-g}{C_P} } \]
On note qu'il ne s'agit pas nécessairement du profil vertical suivi par l'environnement (voir section~\ref{parcenv}).

\sk
Le résultat trouvé ci-dessus revêt une importance particulière en sciences de l'atmosphère. La température d'une parcelle en ascension adiabatique décroît avec l'altitude selon un taux de variation constant, indépendamment des effets de pression. La constante~$\Gamma\e{sec}$ est appelée le \voc{gradient adiabatique sec} de température. Il n'est valable que pour une parcelle d'air non saturée en vapeur d'eau. Le calcul pour la Terre donne un refroidissement de l'ordre de~$10^{\circ}$C/km (ou K/km). Pourquoi cette valeur est-elle en désaccord avec la décroissance de~$6.5^{\circ}$C/km effectivement constatée dans l'atmosphère terrestre~? Le chapitre suivant apporte des éléments de réponse à ce paradoxe apparent.

%On définit également l'{\em énergie statique}  \begin{equation}  e_s=C_PT+gz=cste  \label{estat} \end{equation} L'énergie statique est la somme de l'enthalpie et de l'énergie potentielle de gravitation par unité de masse, et est conservée pour des transformations adiabatiques\footnote{L'énergie cinétique est négligeable. Typiquement, $\delta e_c$=50m$^2$s$^{-2}$ (variation de 10m.s\mo) et $\delta e_s$=10 000m$^2$s$^{-2}$ (variation de 1000m ou 10\deg).}. Les variations ou différences de $e_s$ et $\Theta$ sont reliées par: \[de_s=C_PTd\ln\Theta\]

%% pourquoi la température décroît avec l'altitude lorsqu'on monte une montagne: problème longtemps abordé avec des solutions erronées. y compris Fourier l'a mal interprété.

%% pierrehumbert: tropo means turning in greek, strato means stratification.

%%% CIRCULATIONS THERMIQUES ???

%%\footnote{Cette propriété est employée en pratique pour construire aisément les lignes \og adiabatiques sèches \fg dans un émagramme, comme il est décrit dans un chapitre ultérieur.}

\chapter{Changements de phase et (in)stabilité}

\dictum[René Char, 1934]{Il faut être l'homme de la pluie et l'enfant du beau temps.}

\bk
Le cycle de l'eau est une composante essentielle du système climatique terrestre. Bien que les quantités présentes dans l'atmosphère font de l'eau un composant minoritaire, son rôle climatique et météorologique est de première importance. Pour se préparer à l'étude de ces questions, notamment la formation des nuages abordée au chapitre suivant, on s'intéresse dans ce chapitre à l'évolution d'une parcelle d'air de manière plus approfondie qu'au chapitre précédent en introduisant d'une part les changements d'état de l'eau et d'autre part la notion de stabilité de la parcelle par rapport à son environnement.

\mk
\section{Air humide, air saturé}

\sk
L'eau est présente dans l'atmosphère sous trois phases différentes, de la moins à la plus ordonnée~: gazeuse (vapeur d'eau), liquide (fines gouttelettes en suspension formant les nuages, précipitations pluvieuses), solide (cristaux de glace dans les fins nuages de haute altitude, intempéries de type neige et grêle). On s'intéresse principalement aux phases liquide et gazeuse afin de préfigurer l'étude des nuages sur Terre. Des raisonnements similaires sont possibles, avec quelques subtilités, pour la phase solide afin de décrire les nuages formés de cristaux de glace lorsque la température de l'atmosphère est suffisamment basse.

\sk
\subsection{Quantification de la vapeur d'eau dans l'atmosphère}\label{rappmel}

\sk
Soit une parcelle contenant un mélange de gaz parfaits notés~$i$, dont un est la vapeur d'eau. On a défini la \voc{pression partielle}~$P_i$ et le \voc{rapport de mélange massique}~$r_i = \frac{m\e{gaz i}}{m\e{air}}$ dans le chapitre introductif. Ces deux quantités peuvent servir à définir la quantité de vapeur d'eau présente dans la parcelle d'air. Pour simplifier, on note
\[ P\e{vapeur d'eau} = e \qquad \text{et} \qquad r\e{vapeur d'eau} = \frac{m\e{vapeur d'eau}}{m\e{air}} = r \] 
%La quantité~$q$ est également appelée \voc{humidité spécifique}. 
Le rapport de mélange en vapeur d'eau~$r$ est conservé dans la parcelle si il n'y a pas de changement de phase.

\sk
La pression partielle de l'air sec est~$P - e$. Comme mentionné dans le chapitre d'introduction, la vapeur d'eau vérifie l'équation d'état des gaz parfaits tout comme l'air sec, mélange de gaz parfaits, d'où
\[  e \, V = \frac{m\e{vapeur d'eau}}{M\e{vapeur d'eau}} \, R^* \, T  \qquad \qquad \qquad (P-e) \, V = \frac{m\e{air sec}}{M\e{air sec}} \, R^* \, T  \]
On forme le rapport des deux expressions pour obtenir une expression en fonction de paramètres intensifs et ne dépendant pas de la température
\[ \frac{e}{P-e} = \frac{m\e{vapeur d'eau}}{m\e{air sec}} \, \frac{M\e{air sec}}{M\e{vapeur d'eau}} \]

\sk
L'expression ci-dessus peut être grandement simplifiée. L'eau est un composant minoritaire dans l'atmosphère terrestre~: l'ordre de grandeur de~$r$ est de l'ordre de~$0$ à~$20$~g~kg$^{-1}$. On a donc toujours~$r \ll 1$ et~$e \ll P$, soit~$P-e \simeq P$. Ainsi la masse d'air sec~$m\e{air sec}$ dans la parcelle est en très bonne approximation égale à la masse d'air~$m\e{air}$ dans la parcelle, ce qui vaut également pour la masse molaire. Le rapport de mélange en vapeur d'eau s'écrit alors~$r = \frac{m\e{vapeur d'eau}}{m\e{air}} \simeq \frac{m\e{vapeur d'eau}}{m\e{air sec}}$. L'expression ci-dessus se simplifie donc en
\[ r = \frac{M\e{vapeur d'eau}}{M\e{air}} \, \frac{e}{P} \qquad \Rightarrow \qquad \boxed{ r \simeq 0.622 \, \frac{e}{P} } \]
Cette équation signifie que, pour une pression~$P$ donnée, le rapport de mélange de vapeur d'eau~$r$ est en bonne approximation proportionnel à la pression partielle de vapeur d'eau~$e$.

\sk
%\subsection{Évaporation, Saturation}
\subsection{Equilibre liquide / vapeur}

\sk
L'\voc{évaporation} est l'échappement de molécules d'eau depuis une phase liquide vers une phase gazeuse. A l'interface liquide-gaz, sous l'effet de l'agitation thermique, certaines molécules d'eau dans le liquide vont voir les liaisons hydrogène rompues avec leurs plus proches voisins. L'échappement est ainsi plus facile pour des molécules ayant une énergie cinétique importante~: le taux d'évaporation~$\mathcal{E}$ à partir d'une surface dépend donc de la température de l'eau. 

\sk
La \voc{condensation} est le passage de molécules d'eau de la phase gazeuse à la phase liquide. A l'interface liquide-gaz, certaines molécules d'eau dans le gaz vont se lier à des molécules d'eau dans le liquide par le biais de liaisons hydrogène. Le taux de condensation~$\mathcal{C}$ dépend de la pression de la phase gazeuse, à savoir~$e$ dans le cas de la vapeur d'eau. 

\sk
Soit une enceinte remplie d'air totalement sec, c'est-à-dire qui ne contient aucune molécule d'eau sous forme vapeur. On introduit dans cette enceinte une quantité donnée d'eau liquide. Comme décrit ci-dessus, il va y avoir spontanément évaporation avec un taux d'évaporation~$\mathcal{E}$ (supposé constant) à la surface du liquide, d'autant plus que la température de l'eau est élevée. Des molécules d'eau s'échappent donc dans l'espace au-dessus du liquide et forment une phase gazeuse dont la pression partielle~$e$ augmente peu à peu. Des molécules de cette phase gazeuse subissent à leur tour un phénomène de condensation et repassent en phase liquide. Le taux de condensation~$\mathcal{C}$ est, au début de l'expérience, très petit devant~$\mathcal{E}$ car la pression partielle~$e$ est extrêmement faible. Puisque l'évaporation domine la condensation, le bilan est donc en faveur d'une augmentation des molécules sous forme gazeuse. Néanmoins, plus le nombre de molécules d'eau sous forme gazeuse augmente, plus la pression partielle~$e$ augmente, donc plus le taux de condensation~$\mathcal{C}$ augmente. Ce phénomène va continuer jusqu'à atteindre un équilibre stationnaire où les taux de condensation~$\mathcal{C}$ et~$\mathcal{E}$ se compensent. Cet équilibre est appelé \voc{équilibre liquide-vapeur}, on parle également souvent, par abus de langage, de \voc{\ofg{saturation}}.% ou de \voc{\ofg{conditions saturées}}. 

\sk
\subsection{Grandeurs saturantes}

\sk
\subsubsection{Définition}

\sk
La pression partielle~$e$ pour laquelle l'équilibre liquide-vapeur est atteint est appelée \voc{pression de vapeur saturante} que l'on note~$e\e{sat}$. Tant que~$e<e\e{sat}$, les échanges par évaporation dominent les échanges par condensation et $e$ augmente jusqu'à atteindre~$e\e{sat}$. Lorsque~$e=e\e{sat}$, la quantité de vapeur d'eau dans l'enceinte n'augmente plus\footnote{Il est important de noter que cet état stationnaire n'est pas dénué d'échanges entre les phases liquide et gaz par condensation et évaporation. Par analogie, on peut penser au remplissage d'une baignoire équipée d'un siphon~: le niveau de l'eau est constant à l'état stationnaire bien qu'il y ait en permanence un apport d'eau par le robinet et une perte d'eau par le siphon -- l'état stationnaire signifie juste que ces échanges se compensent.}. Ainsi, si l'on considère une enceinte avec de l'eau sous forme vapeur et liquide
\begin{citemize}
\item si la pression partielle de vapeur d'eau~$e$ dans l'enceinte est supérieure à la pression de vapeur saturante~$e\e{sat}$, il y a condensation jusqu'à ce que~$e=e\e{sat}$.
\item si la pression partielle de vapeur d'eau~$e$ dans l'enceinte est inférieure à la pression de vapeur saturante~$e\e{sat}$, il y a évaporation jusqu'à ce que~$e=e\e{sat}$.
\end{citemize}
Si l'on considère une enceinte contenant de l'eau sous forme vapeur uniquement, le premier point est toujours valable alors que le second point n'est pas vrai~: si la pression partielle de vapeur d'eau~$e$ dans l'enceinte est inférieure à la pression de vapeur saturante~$e\e{sat}$, rien ne se passe, car aucune phase liquide ne peut être évaporée. La pression partielle de vapeur d'eau~$e$ est donc toujours inférieure ou égale à la pression de vapeur saturante~$e\e{sat}$. 

\sk
Le rapport de mélange~$r\e{sat}$ correspondant à l'équilibre liquide/vapeur où $e=e\e{sat}$ est appelé \voc{rapport de mélange saturant}. D'après l'équation encadrée à la section précédente, on a 
\[ r\e{sat} \simeq 0.622 \, \frac{e\e{sat}}{P} \]
Les mêmes raisonnements qu'avec les pressions partielles~$e$ et~$e\e{sat}$ peuvent être faits avec les rapports de mélange~$r$ et~$r\e{sat}$. Ces quantités servent à définir l'\voc{humidité relative}~$H$
\[ \boxed{ H = \frac{e}{e\e{sat}} = \frac{r}{r\e{sat}} } \]
De ce qui précède, on déduit que l'humidité~$H$ est toujours inférieure à~$1$ ($100\%$) et que, lorsqu'il y a équilibre liquide/vapeur (\ofg{conditions saturées}), $H$ vaut~$1$ ($100\%$).

\sk
\subsubsection{Variation avec la température}

\sk
La pression de vapeur saturante~$e\e{sat}$ augmente exponentiellement avec la température~$T$ du gaz d'après la \voc{relation de Clausius-Clapeyron}\footnote{La pression de vapeur saturante est proportionnelle à la probabilité de rupture d'une liaison, elle-même variant exponentiellement suivant la température.}. Ainsi elle double pour une élévation de température de~$10$~K. Plus le gaz dans l'enceinte est chaud, plus la quantité de vapeur d'eau au terme de l'expérience est élevée. En pratique, le terme~$e\e{sat}$, qui varie exponentiellement avec la température~$T$, domine très fréquemment les variations de pression~$P$. Ainsi, en bonne approximation, le rapport de mélange saturant~$r\e{sat}$ varie également exponentiellement avec la température~$T$. 

\sk
La dépendance de~$e\e{sat}$ avec~$T$ permet par ailleurs de définir la \voc{température de rosée}~$T\e{rosée}$ associée à une valeur donnée de la pression partielle~$e$ de l'eau. Il s'agit de la température~$T\e{rosée}$ à laquelle la pression partielle~$e$ devient saturante, c'est-à-dire qui vérifie
\[ \boxed{ e\e{sat}(T\e{rosée}) = e } \]

%\subsection{Ébullition} L'ébullition est un cas particulier: des bulles de gaz se forment à l'{\em intérieur} du liquide bouillant. Dans le cas de l'eau, ce gaz est donc de la vapeur d'eau. La pression dans ces bulles est égale à celle du liquide, soit à peu près la pression atmosphérique si le liquide est en contact avec l'air. Les bulles sont d'autre part stables si leur pression est supérieure à la pression saturante. L'ébullition se produit donc à une température $T_b$ telle que \[e_{sat}(T_b)=P_{atm}\]

\sk
\subsection{Déplacement d'équilibre et application aux gouttelettes nuageuses}

\sk
On applique ici les raisonnements de la section précédente pour une interface plane liquide/vapeur à une goutte sphérique comme rencontrée dans les brouillards ou les nuages. La réalité est un peu plus complexe et fait intervenir les concepts de noyaux de condensation et de sursaturation, qui ne sont pas abordés dans ce cours. Les raisonnements présentés ci-dessous restent cependant valables au premier ordre.
%%Dans l'atmosphère, loin de la surface, il n'y a pas d'interface liquide/gaz permanente. Si $e<e_{sat}$, il n'y a ni condensation ni évaporation. Si $e$ devient supérieure à $e_{sat}$, il y a condensation sous forme de gouttes d'eau liquide (qui se forment plus vite qu'elles ne s'évaporent). Ces gouttes s'évaporent dès que $e<e_{sat}$

\sk
Soit une parcelle d'air à la température~$T_0$ et à la pression~$P$. Elle contient de la vapeur d'eau en équilibre avec des gouttelettes d'eau en suspension, en pratique cela correspond à une parcelle dans laquelle des gouttelettes nuageuses se sont formées. A l'équilibre liquide-vapeur, la pression partielle de vapeur d'eau dans la parcelle vaut~$e=e\e{sat}(T_0)$, le rapport de mélange de vapeur d'eau vaut~$r=r\e{sat}(T_0)$ et l'humidité~$H$ vaut~$100\%$. Si la température de la parcelle change, il y a déplacement de l'équilibre liquide/vapeur\footnote{Par abus de langage, on dit parfois que \ofg{l'air chaud peut contenir plus de vapeur d'eau que l'air froid}. Il est autorisé de garder cette phrase en tête en tant que moyen mnémotechnique, cependant elle est incorrecte physiquement car elle ne rend pas compte de l'équilibre liquide/vapeur.}.
\begin{finger}
\item
Supposons que l'on \underline{chauffe la parcelle} à une température~$T\e{c}>T_0$. Sa pression partielle en vapeur d'eau~$e$ est toujours proche de~$e\e{sat}(T_0)$, mais la pression de vapeur saturante~$e\e{sat}$ a augmenté de façon exponentielle de~$e\e{sat}(T_0)$ à~$e\e{sat}(T\e{c})$. On est alors dans la situation où~$e < e\e{sat}(T\e{c})$, donc~$H < 1$. Il y a alors évaporation d'eau liquide jusqu'à ce qu'un nouvel équilibre liquide/vapeur soit atteint, où~$e = e\e{sat}(T\e{c})$. Une façon équivalente de décrire ce déplacement d'équilibre est de dire que, lorsque la parcelle chauffe, la quantité de vapeur d'eau~$r=r\e{sat}(T_0)$ devient très inférieure à la quantité de vapeur d'eau à saturation~$r\e{sat}(T\e{c})$. De l'eau liquide doit passer sous forme gazeuse par évaporation pour compenser ce déséquilibre, de manière à ce que la quantité de vapeur d'eau~$r$ dans la parcelle augmente à~$r\e{sat}(T\e{c})$. Ainsi lorsque l'on chauffe la parcelle, des gouttelettes nuageuses disparaissent, le nuage se dissipe.
\item
Supposons à l'inverse que l'on \underline{refroidisse la parcelle} à une température~$T\e{f}<T_0$. La pression de vapeur saturante~$e\e{sat}$ a diminué de façon exponentielle de~$e\e{sat}(T_0)$ à~$e\e{sat}(T\e{f})$. On est alors dans la situation où~$e > e\e{sat}(T\e{f})$, donc~$H > 1$, qui est impossible. Il y a alors condensation d'eau liquide jusqu'à ce qu'un nouvel équilibre liquide/vapeur soit atteint, où~$e = e\e{sat}(T\e{f})$. Autrement dit, lorsque la parcelle refroidit, la quantité de vapeur d'eau~$r=r\e{sat}(T_0)$ devient très supérieure à la quantité de vapeur d'eau à saturation~$r\e{sat}(T\e{f})$. De l'eau sous forme gazeuse doit passer sous forme liquide par condensation pour compenser ce déséquilibre, de manière à ce que la quantité de vapeur d'eau~$r$ dans la parcelle diminue à~$r\e{sat}(T\e{f})$. Ainsi lorsque l'on refroidit la parcelle, de nouvelles gouttelettes nuageuses apparaissent, le nuage s'épaissit.
\end{finger}

\sk
Le second point s'applique également au cas d'une parcelle d'air ne contenant pas initialement de gouttelettes nuageuses. 

\sk
Le chapitre précédent a proposé une expression du premier principe qui distingue deux manières de faire varier la température d'une parcelle atmosphérique d'air sec~: transformations isobares et transformations adiabatiques. On peut désormais illustrer la formation de nuages associée à chacune des transformations appliquée à une parcelle de rapport de mélange en vapeur d'eau~$r \neq 0$ qui reste constant au cours de la transformation.
\begin{finger}
\item Lorsqu'une parcelle d'air proche de la surface subit un refroidissement isobare à la tombée de la nuit, sous l'influence du flux radiatif infrarouge, des gouttelettes nuageuses se forment car le rapport de mélange saturant~$r\e{sat}$ diminue jusqu'à devenir plus faible que~$r$. Il s'agit du brouillard nocturne~; la formation de rosée obéit à un principe similaire. La température de rosée~$T\e{rosée}$ peut ainsi être définie comme la température à laquelle la condensation se produit suite à un refroidissement isobare. 
\item Lorsqu'une parcelle d'air subit une élévation adiabatique, à cause par exemple de la présence d'une montagne, elle se refroidit et le rapport de mélange saturant~$r\e{sat}$ diminue. Le rapport de mélange~$r$ peut alors éventuellement devenir supérieur à~$r\e{sat}$ et des gouttelettes se forment pour que~$r=r\e{sat}$. Ceci explique par exemple que les montagnes soient souvent couvertes de nuages.
\end{finger}
Le chapitre suivant se propose de reprendre avec plus de précisions la formation des nuages.

\mk
\section{Evolution hors équilibre d'une parcelle d'air}

\sk
\subsection{Transformations pseudo-adiabatiques}

\sk
On considère tout d'abord une parcelle d'air (contenant de la vapeur d'eau) en évolution isobare. Le premier principe appliqué à la parcelle indique donc
\[ \dd T = \frac{1}{C_P} \, \delta q \]
Lors de l'évaporation, les molécules d'eau liquide voient les liaisons hydrogène avec leurs proches voisins être brisées. Le passage de l'eau de la phase liquide à la phase vapeur consomme donc de l'énergie\footnote{On peut s'en convaincre en notant la sensation de froid immédiate que provoque la sortie d'un bain à cause de l'évaporation de l'eau liquide sur le corps mouillé~; ou en se souvenant que lorsque l'on souffle sur la soupe pour la refroidir, c'est précisément pour favoriser l'évaporation et la refroidir efficacement.}~: pour l'air qui compose la parcelle, $\delta q < 0$ et il y a refroidissement. 
A l'inverse, lors de la condensation, les molécules d'eau sous forme gazeuse créent des liaisons hydrogène avec les molécules d'eau de la phase liquide pour atteindre un état énergétique plus faible. Le passage de l'eau de la phase vapeur à la phase liquide libère donc de l'énergie~: pour l'air qui compose la parcelle, $\delta q > 0$ et il y a chauffage.

\sk
L'énergie~$\delta q$ consommée ou libérée par les changements d'état s'appelle~\voc{chaleur latente}, on la note~$\delta q\e{latent}$. Si une masse de vapeur~$\dd m\e{vapeur d'eau}$ est condensée ou évaporée, on a
\[ \delta q\e{latent} = \frac{- L \, \dd m\e{vapeur d'eau}}{m\e{air sec}} \qquad \Rightarrow \qquad \boxed{ \delta q\e{latent} = - L \, \dd r } \]
où~$L$ est la chaleur latente massique en~J~kg$^{-1}$. La formule ci-dessus comporte un signe négatif. La quantité~$\delta q\e{latent}$ est positive lorsqu'il y a condensation (le rapport de mélange en vapeur d'eau diminue $\dd r < 0$) et négative lorsqu'il y a évaporation (le rapport de mélange en vapeur d'eau augmente $\dd r > 0$).

\sk
On considère désormais une parcelle d'air en évolution adiabatique, à l'exception des échanges de chaleur latente~: $\delta q = \delta q\e{latent}$. On appelle une telle transformation \voc{pseudo-adiabatique} ou encore \voc{adiabatique saturée}. On fait l'approximation que la chaleur latente consommée ou dégagée est seulement échangée avec l'air sec~:
\begin{citemize}
\item La chaleur latente consommée/dégagée n'est pas utilisée pour refroidir/chauffer les gouttes d'eau présentes.
\item On néglige les pertes de masse par précipitation~: la masse d'air sec considérée est constante.
\end{citemize}
Pour une telle transformation, la variation de température s'écrit ainsi
\[ \dd T = \frac{R}{C_P} \, \frac{T}{P} \, \dd P - \frac{L}{C_P} \, \dd r \]

\sk
\subsection{Profil vertical saturé}

\sk
Considérons une parcelle en ascension adiabatique saturée (et non plus sèche comme dans la section~\ref{adiabsec}). Pour une parcelle saturée, c'est-à-dire à l'équilibre liquide/vapeur, l'équation qui précède peut s'écrire, en utilisant l'équilibre hydrostatique
\[ C_P \, \dd T + g \, \dd z + L \, \dd r = 0 \]
Or, puisque la parcelle est saturée, on a~$r = r\e{sat}(T)$ et on peut écrire $\dd r\e{sat} = \ddf{r\e{sat}}{T} \, \dd T$. On a alors
\[ \left( C_P + L \, \ddf{r\e{sat}}{T} \right) \dd T + g \, \dd z = 0\]
Cette expression est similaire au cas sec, à l'exception notable du terme supplémentaire~$L \, \ddf{r\e{sat}}{T}$ lié aux échanges latents. On peut alors obtenir le profil vertical adopté dans l'atmosphère saturée par une parcelle ne subissant pas d'échange de chaleur avec l'extérieur autre que les échanges de chaleur latente
\[  \ddf{T}{z}  = \Gamma\e{saturé} \qquad \text{avec} \qquad \Gamma\e{saturé} = \frac{-g}{C_P+L \, \ddf{r\e{sat}}{T} } \]
On a vu que $\ddf{r\e{sat}}{T}$ est toujours positif, on en déduit donc
\[ \boxed{ \Gamma\e{saturé} > \Gamma\e{sec} \qquad \text{ou} \qquad |\Gamma\e{saturé}| < |\Gamma\e{sec}| } \]
A cause du dégagement de chaleur latente, la température diminue moins vite pour une parcelle saturée en ascension que pour une parcelle non saturée. Le calcul pour l'atmosphère terrestre montre que
\[ \Gamma\e{saturé} = -6.5 \, \text{K~km}^{-1} \] 
ce qui correspond à la valeur observée dans la troposphère [Figure~\ref{fig:tempvert}].

\sk
La constatation que~$\Gamma\e{saturé}$ correspond au profil d'environnement effectivement mesuré dans la troposphère appelle un commentaire important. Les profils verticaux secs ou saturés sont ceux suivis par une parcelle en ascension~: autrement dit, ils donnent les variations de~$T\e{p}$ avec l'altitude~$z$. D'un point de vue instantané, ils ne correspondent pas aux profils d'environnement~$T\e{e}$ tels qu'ils peuvent être par exemple mesurés par des ballons-sonde lâchés dans l'atmosphère. La parcelle n'est pas nécessairement à l'équilibre thermique avec l'environnement. On peut néanmoins constater sur la figure~\ref{fig:tempvert} que la température de l'environnement diminue avec une pente très proche de~$\Gamma\e{saturé}$. Ceci s'explique par le fait que cette figure montre une moyenne sur tout le globe à toutes les saisons. La situation moyenne ainsi décrite correspond aux mouvements d'une multitude de parcelles en ascension qui finissent par définir l'environnement atmosphérique\footnote{Ce phénomène porte le nom d'ajustement convectif.}. Pour comprendre la formation des nuages, et plus généralement les mouvements atmosphériques, il faut néanmoins se placer dans le cas local où l'équilibre thermique n'est pas vérifié. C'est l'objet de la section suivante.
%Comme pour le cas adiabatique, on peut aussi intégrer l'équation pour obtenir:
%\begin{equation} e_h=C_PT+gz+Lr=cste \label{estath} \end{equation}  
%La quantité $e_h$ est appelée {\em énergie statique humide} et est conservée
%pour des mouvements adiabatiques ($r$ et $e_s$ sont séparément conservés) ou
%saturés (pseudo-adiabatiques).

\mk
\section{Stabilité et instabilité verticale}

\sk
\subsection{Force de flottaison}

\sk
Soit une parcelle dont la température $T\e{p}$ n'est pas égale à celle de l'environnement~$T\e{e}$, que ce soit sous l'effet d'un chauffage diabatique (par exemple~: chaleur latente, effets radiatifs) ou d'une compression / détente adiabatique. On reprend le calcul réalisé précédemment pour l'équilibre hydrostatique, avec la différence notable que l'on n'est plus dans le cas statique~: on étudie le mouvement vertical d'une parcelle. 

\sk
La somme des forces massiques s'exerçant sur la parcelle suivant la verticale est
\[ - g  - \frac{1}{\rho\e{p}}  \, \Dp{P\e{e}}{z} \]
où~$\rho\e{p}$ est la masse volumique de la parcelle. L'environnement est à l'équilibre hydrostatique donc
\[ \Dp{P\e{e}}{z} = - \rho\e{e} \, g \]
Ainsi la résultante~$F_z$ des forces massiques qui s'exercent sur la parcelle selon la verticale vaut
\[ F_z = g \, \left( \frac{\rho\e{e}}{\rho\e{p}} - 1 \right) = g \, \frac{\rho\e{e}-\rho\e{p}}{\rho\e{p}} \]
En utilisant l'équation du gaz parfait pour la parcelle~$\rho\e{p}=P/RT\e{p}$ et l'environnement~$\rho\e{e}=P/RT\e{e}$, on a
\[ \boxed{ F_z = g \, \frac{T\e{p}-T\e{e}}{T\e{e}} } \]
La résultante des forces est donc dirigée vers le haut, donc la parcelle s'élève, si la parcelle est plus chaude (donc moins dense) que son environnement. 
Elle est dirigée vers le bas si la parcelle est plus froide (donc plus dense) que son environnement.
En d'autres termes, on écrit ici la version météorologique de la force ascendante ou descendante 
provoquée par la poussée d'Archimède, également appelée \voc{force de flottaison}.

\sk
\subsection{Stabilité et instabilité}

\sk
Ces considérations permettent de définir le concept de stabilité et instabilité verticale de l'atmosphère.
On considère l'atmosphère à un endroit donné de la planète, à une saison donnée, à une heure donnée de la journée.
On suppose que la température de l'environnement varie linéairement avec l'altitude
\[ \ddf{T\e{e}}{z} = \Gamma\e{env} \]
A une altitude~$z_0$ proche de la surface, la température de l'environnement est~$T\e{e}(z_0)=T_0$.

\sk
On considère une parcelle initialement à l'altitude~$z_0$ dont la température initiale~$T\e{p}(z_0)$ est également~$T_0$. On suppose que la parcelle subit une ascension verticale d'amplitude~$\delta z > 0$. Le profil de température suivi par la parcelle lors de son ascension est
\[ \ddf{T\e{p}}{z} = \Gamma\e{parcelle} \]
\begin{citemize}
\item Si la parcelle est non saturée, elle suit un profil adiabatique sec tel que $\Gamma\e{parcelle} = \Gamma\e{sec} \simeq - 10 \, \text{K/km}$.
\item Si elle est saturée, elle suit un profil adiabatique saturé tel que $\Gamma\e{parcelle} = \Gamma\e{saturé} \simeq - 6.5 \, \text{K/km}$. 
\end{citemize}
On rappelle qu'en général, à l'échelle où l'on étudie les mouvements de la parcelle
\[ \Gamma\e{parcelle} \neq \Gamma\e{env} \]

\sk
Quel est l'effet de la perturbation~$\delta z > 0$ sur le mouvement de la parcelle~? A l'altitude~$z_0 + \delta z$, les températures de la parcelle et de l'environnement sont respectivement
\[ T\e{p}(z_0 + \delta z) = T_0 + \Gamma\e{parcelle} \, \delta z 
\qquad \text{et} \qquad
T\e{e}(z_0 + \delta z) = T_0 + \Gamma\e{env} \, \delta z \]
\begin{finger}
\item Si $\Gamma\e{parcelle} > \Gamma\e{env}$, la température~$T\e{e}$ de l'environnement décroît plus vite que la température~$T\e{p}$ de la parcelle. Il en résulte que~$T\e{p}(z_0 + \delta z) > T\e{e}(z_0 + \delta z)$ et le mouvement de la parcelle est ascendant. La perturbation initiale est donc amplifiée par les forces de flottabilité. On parle de \voc{situation instable}. La situation est d'autant plus instable que la température de l'environnement décroît rapidement avec l'altitude. Lorsque la situation est instable, les mouvements verticaux sont amplifiés~: on parle parfois de \voc{situation convective}.
\item Si $\Gamma\e{parcelle} < \Gamma\e{env}$, la température~$T\e{e}$ de l'environnement décroît moins vite que la température~$T\e{p}$ de la parcelle. Il en résulte que~$T\e{p}(z_0 + \delta z) < T\e{e}(z_0 + \delta z)$ et le mouvement de la parcelle est descendant. La perturbation initiale n'est donc pas amplifiée et la parcelle revient à son état initial. On parle de \voc{situation stable}. La stabilité est d'autant plus grande que la température de l'environnement décroît lentement (ou augmente, dans le cas d'une inversion de température). Lorsque la situation est stable, les mouvements verticaux sont inhibés.
\end{finger}
La résultante des forces verticales s'exerçant sur la parcelle peut s'écrire en fonction des taux de variation~$\Gamma$ de la température
\[ F_z = g \, \frac{\Gamma\e{parcelle}-\Gamma\e{env}}{T\e{env}} \, \delta z \]

\sk
On peut illustrer la stabilité/instabilité atmosphérique dans le cas des polluants émis proche de la surface par les activités humaines [Figure~\ref{fig:pollution}]. Dans l'après-midi, du fait que le sol est chaud, le profil d'environnement est tel que la situation est très instable~: les mouvements verticaux qui transportent les polluants plus haut dans l'atmosphère sont encouragés et les polluants ne restent pas proches de la surface. A l'inverse, en soirée, du fait que le sol refroidit radiativement, le profil d'environnement est tel que la situation est stable~: les mouvements verticaux qui pourraient transporter les polluants plus haut dans l'atmosphère sont inhibés et les polluants sont confinés proche de la surface. Pour être moins exposé aux polluants dans les zones urbaines, il est donc préférable d'y effectuer son jogging en fin de matinée plutôt qu'en soirée !

\figside{0.65}{0.25}{decouverte/cours_meteo/inversion-temperature.png}{Stabilité et pollution atmosphérique. On notera que cette figure est très illustrative, mais présente une situation simplifiée. Le transport vertical de polluants dans l'atmosphère est en réalité inhibé dès que la couche atmosphérique est stable, ce qui est plus général que considérer uniquement une inversion thermique comme à droite de la figure. Source~: Airparif}{fig:pollution} 




\chapter{Dynamique et circulation générale}

\dictum[Livre d'Osée, 8e siècle avant JC]{Qui sème le vent récolte la tempête.}

\bk
Dans ce chapitre, on aborde de manière plus approfondie la circulation de l'atmosphère, c'est-à-dire les vents. On s'intéresse notamment à l'origine des mouvements horizontaux, en étudiant les diverses forces en présence, avant de donner quelques généralités sur les vents à grande échelle sur Terre (la \voc{circulation générale}). Dans tout le chapitre, les vecteurs sont notés \textbf{en gras}.

\mk 
\section{Equations de la dynamique et interprétation}

\sk
\subsection{Système de coordonnées et référentiel}
	\sk
La position d'un point $M$ de l'atmosphère sera représentée dans un systèmes de coordonnées sphériques (figure~\ref{fig:repere}) par sa latitude $\varphi$, sa longitude $\lambda$, et son altitude~$z$ par rapport au niveau de la mer. Pour les déplacements horizontaux, on utilise le repère direct
$\left(M,\mathbf{i},\mathbf{j},\mathbf{k}\right)$ où $\mathbf{i}$ et $\mathbf{j}$ sont les vecteurs unitaires vers l'est et le nord, et $\mathbf{k}$ est dirigé suivant la verticale vers le haut. La direction définie par~$\mathbf{i}$ est souvent qualifiée de \voc{zonale}, celle définie par~$\mathbf{j}$ de \voc{méridienne}. 
%Pour des déplacements qui ne sont pas d'échelle planétaire, on utilisera également des distances horizontales vers l'est et le nord~$dx=a\, d\lambda\, \cos \varphi$ et~$dy=a\,d\varphi$ où~$a$ est le rayon de la Terre.
%
%\sk
On distingue deux référentiels pour l'étude des mouvements de l'air:
\begin{finger}
\item Un \voc{référentiel tournant} lié à la Terre, en rotation autour de l'axe des pôles avec la vitesse angulaire $\Omega$. La \voc{vitesse relative} est mesurée dans le référentiel tournant, par rapport à la surface de la Terre et a pour composantes~$u,v,w$ suivant \v i,\v j,\v k. Il s'agit de ce que l'on appelle communément le \voc{vent} avec le point de vue d'humain attaché à la surface de la Terre, c'est-à-dire au référentiel tournant. La composante horizontale du vecteur vitesse relative est donc~$\mathbf{V} = u \, \mathbf{i} + v \, \mathbf{j}$ et la composante verticale~$w \, \mathbf{k}$.
\item Un \voc{référentiel fixe} orienté suivant les directions de trois étoiles. La \voc{vitesse absolue} d'un point M est considérée dans le référentiel fixe et inclut donc le mouvement circulaire autour de l'axe des pôles. Ce référentiel peut être considéré comme galiléen. Il correspond à ce qu'on observerait depuis l'espace, lorsqu'on voit la Terre tourner au lieu d'être \ofg{attaché} à sa rotation.
\end{finger}

%\figun{0.4}{0.25}{\figfrancis/repere}{Schéma du système de coordonnées et du repère utilisés.}{fig:repere}
\figside{0.45}{0.22}{\figfrancis/repere}{Système de coordonnées et repère utilisés.}{fig:repere}




\sk
\subsection{Equations du mouvement horizontal}
	\sk
L'équation de base pour le mouvement de masses d'air est la relation fondamentale de la dynamique $\Sigma \v F=m \, \v a$ (seconde loi de Newton).  Cette relation est cependant valable dans un référentiel galiléen, tel le référentiel fixe. On s'intéresse plutôt au vent, c'est-à-dire que l'on souhaite considérer des mouvements atmosphériques par rapport à la surface de la Terre qui est en rotation autour de l'axe des pôles. On va donc dans un premier temps projeter l'accélération dans le référentiel tournant, puis étudier les principales forces horizontales. Autrement dit, on se donne pour objectif d'exprimer l'accélération dans le référentiel tournant, qu'on souhaite connaître, en fonction de l'accélération dans le référentiel fixe, qui est égale à la somme des forces.

\sk
La relation entre vitesse absolue~$\v V_a$ dans le référentiel fixe et vitesse relative~$\v V_r$ dans le référentiel tournant s'écrit, avec le vecteur de rotation~$\v \Omega$ de module~$\Omega$ dirigé selon l'axe des pôles~:
\[\v V_a = \v V_r + \vl{\Omega}\wedge\vl{CM}\]
Il s'agit de la relation de composition des vitesses pour un référentiel tournant. Le terme $\vl{\Omega}\wedge\vl{CM}$ est la vitesse d'un point fixe par rapport au sol ($\v V_r=0$), il est appelé \voc{vitesse d'entrainement}.
%La relation entre la dérivée temporelle d'un vecteur \v X dans le référentiel fixe (\emph{absolue}, $a$) et celle dans le référentiel tournant (\emph{relative}, $r$) s'écrit \[\frac{d\v X}{dt}_{|a}=\frac{d\v X}{dt}_{|r}+ \vl \Omega\wedge \v X\] En applicant au vecteur \vl{CM}, avec $\frac{d\vl{CM}}{dt}=\v V$, on a: \[\v V_a=\v V_r+\vl{\Omega}\wedge\vl{CM}\]
La relation entre accélération absolue~$\v a_a$, égale à la somme des forces, et accélération relative~$\v V_r$ dans le référentiel tournant s'écrit
\[ \v a_a=\Sigma\v F=\v a_r+2\vl{\Omega}\wedge\v V_r-\Omega^2\,\vl{HM} \]
Le premier terme est l'accélération relative~$\v a_r$, le deuxième l'\voc{accélération de Coriolis}~$\v a_c$, le troisième est l'\voc{accélération d'entrainement}~$\v a_e$. Les termes de Coriolis et d'entraînement induisent des \voc{forces apparentes}~$\v F_c = -m \, \v a_c$ et~$\v F_e = -m \, \v a_e$ dans le référentiel tournant. On parle de forces apparentes car du point de vue du référentiel fixe, ces termes n'apparaissent pas comme des forces~: ils ne sont que des termes d'accélération causés par le caractère non galiléen du référentiel tournant.
%En dérivant à nouveau $\v V_a$, on obtient: \[\v a_a=\left(\frac{d \v V_r}{dt}_{|r}+\vl{\Omega}\wedge\v V_r\right)+\vl{\Omega}\wedge \left(\v V_r+\vl{\Omega}\wedge\vl{CM}\right)\] soit en regroupant et avec $\vl{\Omega}\wedge(\vl{\Omega}\wedge\vl{CM})=-\Omega^2\cdot\vl{HM}$:




\sk
\subsection{Action des forces apparentes}

\sk
On peut rapidement interpréter les deux termes liés aux forces apparentes dans le cadre du mouvement d'un point à la surface de la Terre.

\sk
\subsubsection{Accélération d'entraînement et pesanteur}
	\sk
On considère un point M immobile par rapport à la surface de la Terre. Les forces (massiques) subies par M sont la force de gravitation \v G, dirigée vers le centre de la Terre, et \v R la réaction du sol dirigée perpendiculairement à la surface (figure \ref{fig:centrif}). Dans le référentiel fixe, l'accélération de M est celle du mouvement circulaire uniforme: $\v a_e = - \Omega^2 \, \vl{HM}$ (accélération d'entrainement). On doit donc avoir \[\v a_e=\v G+\v R\] 
C'est impossible si la Terre est sphérique (sauf au pôle et à l'équateur): on aurait alors \v R et \v G colinéaires mais pas dans la direction de $\v a_e$. La Terre a en fait pris une forme aplatie, où la surface n'est pas perpendiculaire à~$\v G$. En posant $\v g=\v G-\v a_e$, l'équilibre devient: \[\v g+\v R=\v 0\] On a donc une gravité apparente \v g dirigée localement vers le bas (perpendiculairement à la surface) mais pas exactement vers le centre de la Terre. La gravité réelle \v G a elle une faible composante horizontale. Dans ce qui suit, on considère que l'accélération d'entraînement est inclus dans le terme~$\v g$.

%\figside{0.55}{0.25}{\figfrancis/centrif}{Equilibre d'un point posé au sol. La forme réelle de la Terre est en trait continu, la sphère en pointillés.}{fig:centrif}
\figside{0.45}{0.2}{\figfrancis/centrif}{Equilibre d'un point posé au sol. La forme réelle de la Terre est en trait continu, la sphère en pointillés.}{fig:centrif}




\sk
\subsubsection{Accélération de Coriolis et déviation du mouvement}
	\sk
L'accélération de Coriolis peut être interprétée comme une force apparente massique $\v F_C = - 2 \, \v \Omega\wedge\v V_r$. Cette force apparente étant orthogonale à la vitesse à cause de la présence du produit vectoriel, sa puissance est nulle~: la \voc{force de Coriolis} va dévier le mouvement relatif mais ne peut pas modifier la vitesse du vent ou de courants. Pour des mouvements relatifs horizontaux à la vitesse \v V, le module de la force apparente de Coriolis est~$2 \, \Omega \, \sin \phi \, V$ qui change de signe lorsqu'on change d'hémisphère en fonction de~$\sin \phi$. Dans l'hémisphère nord, où $\sin \phi>0$, la force de Coriolis est dirigée à $90^{\circ}$ à droite du vent. 

\sk
Afin de bien comprendre l'effet de la force de Coriolis, il est profitable sur une planète comme la Terre d'utiliser la conservation du moment cinétique\footnote{
Puisque le moment cinétique~$\sigma$ se conserve on a \[ \ddf{\sigma}{t} = 0 = \ddf{r}{t} \, (\Omega \, r + u) + r \, \left( \Omega \ddf{r}{t}+\ddf{u}{t} \right) \qquad \Rightarrow \qquad \ddf{u}{t} = - \ddf{r}{t}  \, \left( 2\,\Omega + \frac{u}{r} \right) \] 
Le terme en $u/r$ est dû à la courbure de la surface, mais seule la vitesse relative intervient, pas la rotation de la Terre. En pratique, ce terme est négligeable sur Terre devant~$2 \, \Omega$. L'équation ci-dessus montre donc que raisonner avec la conservation du moment cinétique permet de comprendre l'effet sur les vents de la force de Coriolis.
}
(l'équivalent pour les systèmes en rotation de la quantité de mouvement pour les systèmes en translation). En effet, la somme des forces étant dirigée vers H, M conserve son \voc{moment cinétique}~$\sigma$ par rapport à l'axe des pôles, qui s'exprime
\[ \boxed{ \sigma = u_a \, r = (\Omega \, r + u) \, r } \]
où~$r$ est la distance entre le point considéré et l'axe de rotation qui passe par les deux pôles.
%\footnote{La conservation de $\sigma$ implique des variations de l'énergie cinétique $(\Omega r+u)^2$. C'est le travail de \v G (pour un mouvement sud-nord) qui en est l'origine.}. 

\sk
Pour illustrer les effets de cette force apparente de Coriolis, on considère une parcelle initialement au repos dans le référentiel tournant (c'est à dire~$u=0$ et~$v=0$ à~$t=0$) qui se déplacerait vers le Nord suivant l'axe~$\v j$. Elle se rapproche donc de l'axe des pôles et va voir sa vitesse absolue augmenter par conservation du moment cinétique: $\sigma$ est constant et~$r$ diminue, donc $u_a$ augmente. Dans le même temps, la vitesse d'entrainement locale~$u_e=\Omega \, r$ diminue sous l'effet de la diminution de la distance~$r$ à l'axe des pôles. La parcelle va donc acquérir une vitesse relative $u>0$ vers l'est\footnote{
En fait, l'expression ci-dessus permet même de calculer la variation de vitesse associée. Pour un mouvement sud-nord, la vitesse est $v=a \, \ddf{\phi}{t}$. D'autre part $r=a \, \cos \phi$ donc~$\ddf{r}{t}=-a \, \ddf{\phi}{t} \, \sin \phi = - v \, \sin \phi$. L'équation de conservation du moment cinétique devient 
\[ \ddf{u}{t} = v \, \sin \phi \, \left( 2 \, \Omega + \frac{u}{r} \right) \simeq 2 \, \Omega \, v \, \sin \phi \] 
La parcelle est bien déviée vers l'est pour un déplacement vers le nord tel que~$v>0$.
}
comme indiqué sur le schéma \ref{fig:coriolisns}. 

\figside{0.3}{0.2}{\figfrancis/coriolis_ns}{Déviation d'une parcelle se déplaçant vers le nord. Instant initial: vitesses d'entrainement $u_e$ et absolue $u_a$ égales. Instant final: vitesse d'entrainement $u_e'$ et absolue $u_a'$ augmentée par conservation du moment cinétique $\sigma$.}{fig:coriolisns}

%\subsubsection{Force de Coriolis: mouvement vers l'est} On considère un point M en mouvement par rapport à la surface de la Terre. On rappelle que pour un mouvement circulaire, on doit avoir une accélération normale égale à $V^2/R$ dirigée vers le centre du cercle. On suppose que les forces réelles s'exerçant sur M sont les mêmes que pour un point fixe: $\Sigma \vec F=\v a_e$. La composante de la vitesse relative vers l'est (suivant \v i) est $u$, et $\dot{r}$ dans la direction \vl{HM}. La vitesse absolue de M vers l'est est $u_a=\Omega r+u$. La relation $\v a=\Sigma\v F$ s'écrit dans la direction $\v e_r$: \[-\frac{(\Omega r+u)^2}{r}+\ddot{r}=a_e=-\Omega^2r\] soit en développant: \[\ddot{r}=u\cdot(2\Omega+\frac{u}{r})\] Pour un mouvement relatif vers l'est ($u>0$), la vitesse absolue est supérieure à la vitesse d'entrainement, et la somme des forces est insuffisante pour compenser $V_a^2/r$. La parcelle va donc s'éloigner de l'axe de rotation (figure \ref{fig:coriolisew}). Elle va au contraire se rapprocher pour $u<0$ (mouvement vers l'ouest). Pour trouver l'accélération relative dans la direction sud-nord, on projette $\v e_r$ sur \v j: $\dot{v}=-\ddot{r}\sin \phi$. \[\dot{v}=-u\sin \phi\cdot(2\Omega+\frac{u}{r})\] M est donc dévié vers le sud pour un déplacement relatif vers l'est.
%\begin{figure}[tbp] \begin{center} \includegraphics[width=12cm]{\figfrancis/coriolis_ew} \end{center} \caption{Déviation d'une parcelle ayant une vitesse relative initiale non nulle vers l'est (gauche) et l'ouest (droite). Un plan parallèle à l'équateur est représenté, vu depuis le pôle nord, l'axe de rotation est au centre. Les vitesse et accélération d'entrainement (égale à la somme des forces) sont en noir, la vitesse absolue en rouge. La trajectoire future de la parcelle est en pointillés.} \label{fig:coriolisew} \end{figure}


\sk
\subsection{Forces de pression}
	\sk
Les forces de pression horizontales se calculent comme la force de pression verticale dans la démonstration de l'équilibre hydrostatique. La force de pression s'exerçant sur une surface $S$ est normale à cette surface et vaut $P \, S$. Pour une parcelle d'air de volume $\delta x \, \delta y \, \delta z$ (figure \ref{fig:pres}), la force de pression totale dans la direction ($Ox$) vaut
\[ F_P^* = P(x) \, \delta y \, \delta z - P(x+\delta x) \, \delta y \, \delta z = - \frac{\partial P}{\partial x} \, \delta x \, \delta y \, \delta z \]
La force de pression {\em massique} est donc
\[F_P = \frac{F_P^*}{\rho \delta x \delta y \delta z}=-\frac{1}{\rho}\frac{\partial P}{\partial x}\]
On peut faire le même calcul sur ($Oy$). Finalement les deux composantes horizontales de la force de pression s'écrivent
\[\v F_P^H = -\frac{1}{\rho} \, \binom{\frac{\partial P}{\partial x}}{\frac{\partial P}{\partial y}}\] %  =-\frac{1}{\rho}\vl{grad}P\]

\sk
La force de pression est donc opposée aux variations horizontales de pression données par les dérivées partielles, ce qui lui confère des propriétés importantes.
\begin{citemize}
\item La force de pression est dirigée des hautes vers les basses pressions, perpendiculairement aux isobares.
\item La force de pression est inversement proportionelle à l'écartement des isobares.
\end{citemize}
Une région où la pression est particulièrement basse est appelée \voc{dépression}. Une région où la pression est particulièrement élevée est appelée \voc{anticyclone}.

\figside{0.6}{0.2}{\figfrancis/pressure}{Forces de pression (suivant ($Ox$)) s'exerçant sur une parcelle.}{fig:pres}

%\subsubsection{Équivalence avec le géopotentiel}
%L'équilibre hydrostatique fait que la pression décroit toujours avec
%l'altitude. Une pression localement élevée doit donc correspondre à une
%altitude élevée des surfaces isobares.
%\begin{figure}[htp]
%  \begin{center}
%    \includegraphics[width=\figwn]{\figfrancis/pres_geop}
%  \end{center}
%  \caption{Équivalence entre écarts de pression et d'altitude: les points A et
%  B sont à la même altitude, A et C à la même pression. La pression en B est
%  donc supérieure à celle en B.}
%  \label{fig:pres_geop}
%\end{figure}
%Sur la figure \ref{fig:pres_geop}, la force de pression horizonale dans la
%direction ($Ox$) est
%$F_P=-\frac{1}{\rho}\frac{P_B-P_A}{\delta x}$. Or $A$ et $C$ sont à la même
%pression, on a donc
%\[F_P=-\frac{1}{\rho}\frac{P_B-P_C}{\delta x}=-\frac{1}{\rho}\frac{P_B-P_C}{\delta z}\cdot\frac{\delta z}{\delta x}\]
%En utilisant
%\[\frac{P_B-P_C}{\delta z}=-\frac{\partial P}{\partial z}=\rho g\]
%on trouve 
%\[F_P=-g\left(\frac{\delta z}{\delta x}\right)_P\]
%On aurait une relation équivalente pour la direction ($Oy$), la
%force de pression horizontale vaut donc finalement
%\[\v F_P=-\frac{1}{\rho}\vl{grad}_Z(P)=-g\cdot\vl{grad}_P(Z)\]
%On utilise plutôt le gradient de pression horizontal avec la pression au
%niveau de la mer, et le gradient isobare de l'altitude $Z$ ou du
%{\em géopotentiel} $gZ$ dans l'atmosphère libre.
%Sur une carte d'une surface isobare, les lignes à $Z$ constant sont des
%{\em isohypses}. La force de pression est donc dirigée des hautes vers les
%basses valeurs de $Z$, perpendiculairement aux isohypses.

\sk
Les variations verticales de la pression sont données par l'équilibre hydrostatique comme indiqué dans les chapitres précédents. Cette propriété a deux conséquences importantes pour les variations de pression horizontales donc la force de pression horizontale. 
\begin{finger}
\item Une conséquence de cet équilibre est que la pression à une altitude $z$ est proportionelle à la masse de la colonne d'air située au dessus de $z$. Une diminution ou augmentation de cette masse dûe aux mouvements d'air horizontaux change donc la pression en dessous, en particulier à la surface.
\item D'autre part, même pour une masse d'air totale de la colonne constante, des écarts de température horizontaux peuvent créer des gradients de pression en changeant la répartition verticale de cette masse. L'équation hypsométrique donne l'épaisseur d'une colonne d'air de masse constante entre deux niveaux de pression donnés (voir chapitres précédents)~: la pression décroît plus vite dans une couche d'air froid que dans une couche d'air chaud. Une variation horizontale de température induit donc une force de pression horizontale selon ce principe.
\end{finger}

%\begin{equation}
%  g\cdot(Z_2-Z_1)=R<T>\ln{\frac{P_1}{P_2}}
%  \label{eq:hypso}
%\end{equation}
%La différence entre les forces de pressions aux niveaux 1 et 2 sera donc: \[\v F_{P_2}-\v F_{P_1}=-R\cdot\vl{grad}<T>\cdot\ln{\frac{P_1}{P_2}}\]




\sk
\subsection{Équilibres dynamiques}
	\sk
L'équation complète de la quantité de mouvement pour les mouvements atmosphériques, qui résulte de l'application de la seconde loi de Newton, s'écrit~:
%\begin{equation}
%\ddf{\v V_r}{t} + 2 \, \v \Omega\wedge\v V_r = \v g + \v F_P + \vl{Fr} 
\[   
\ddf{\v V_r}{t} = \v g + \v F_P + \v F_C + \vl{Fr}
\] %\frac{1}{\rho}\vl{grad}P  \]
%  \label{eq:qtemvt}
%\end{equation}
Le terme~$\vl{Fr}$ représente les forces de friction qui sont négligées sauf lorsqu'on se trouve à proximité de la surface.

	\sk
Tous les termes de l'équation du mouvement n'ont pas la même importance lorsqu'on considère des mouvements atmosphériques de grande échelle. On définit donc des échelles caractéristiques du mouvement étudié. Pour simplifier, on choisit des échelles qui sont des puissances de 10.
\begin{description}
\item[longueur] Les échelles de longueur sont $L$ sur l'horizontale, et $H$ sur la verticale. Pour des mouvements qui s'étendent sur la hauteur de la troposphère, $H\sim 10$~km. $L$ peut varier beaucoup, mais l'échelle dite synoptique $L=$1000~km, qui est celle des perturbations des latitudes moyennes, est d'un intérêt particulier. La dernière échelle de longueur est celle du rayon de la Terre~$a$, qui est de l'ordre de 10000~km. 
\item[vitesse] Les échelles de vitesse horizontale et verticale sont notées $U$ et $W$. On a typiquement $U$=10~m~s$^{-1}$ dans l'atmosphère. Le rapport d'aspect du mouvement impose d'autre part que $W\le UH/L$.
\item[temps] L'échelle de durée du mouvement est construite à partir de celles de vitesse et de longueur: $T=L/U$. L'autre échelle de temps est celle liée à la rotation de la Terre, qui apparait dans le terme de Coriolis.
\item[variables thermodynamiques] Les variations des variables thermodynamiques $P,T,\rho$ sur la verticale sont celles des profils moyens donnés en introduction. En un point donné, les variations à l'échelle synoptique $\delta P,\delta T,\delta\rho$ sont de l'ordre de 1\% de la valeur moyenne.
\end{description}




\sk
\subsubsection{Mouvement vertical}
	\sk
L'ordre de grandeur des termes de l'équation du mouvement 
%\ref{eq:qtemvt} 
projetée sur la verticale (dirigée suivant \v k) est indiqué dans la table \ref{tab:vqmouv}. On voit que l'équilibre hydrostatique est vérifié avec une très bonne approximation\footnote{On peut noter qu'on vérifie également l'équilibre hydrostatique entre des anomalies de densité et des anomalies de variations de pression sur la verticale. Les termes $\rho g$ et $\partial P/\partial z$ sont alors cent fois plus faibles que pour l'état moyen, mais toujours supérieurs aux autres termes de l'équation.}. Notamment la composante verticale de la force de Coriolis~$\v F_C$ est négligeable devant~\v g et les forces de pression. Le seul autre terme qui peut devenir important est l'accélération relative~$dw/dt$, lors de mouvements verticaux intenses à petite échelle, comme dans un nuage d'orage ou près de topographie raide.  
%\begin{equation}
%  \frac{\partial P}{\partial z}=-\rho  g
%  \label{eq:hydro}
%\end{equation}

\begin{table}
  \centering
  \begin{tabular}{ccccccc}
    \hline
    Équation & $dw/dt$ & $-2\Omega u\cos\phi$ & $-\left(u^2+v^2\right)/a$ & = &
    $-\rho^{-1}\partial P/\partial z$ & $-g$ \\
    Échelle & $UW/L$ & $fU$ & $U^2/a$ && $P_0/(\rho_0H)$ & $g$ \\
    m.s\md & 10$^{-7}$ & 10$^{-3}$ & 10$^{-5}$  && 10 & 10 \\ 
    \hline
  \end{tabular}
  \caption{\emph{Analyse d'échelle de l'équation du mouvement vertical (avec
  $L$=1000~km et $W$=1~cm.s\mo).}}
  \label{tab:vqmouv}
\end{table}




\sk
\subsubsection{Mouvement horizontal}
	\sk
Le détail de l'équation horizontale projetée en coordonnées sphériques est donné dans la table \ref{tab:hqmouv} pour $L$=1000~km. Sur les composantes horizontales (\v i, \v j), l'expression de la force de Coriolis se réduit aux contributions des mouvements horizontaux dans la mesure où~$W<<U$ pour des mouvements d'échelle supérieure à 10~km. 
\[\v F_C = \binom{f \, v}{-f \, u} \qquad \textrm{ou} \qquad \v F_C = -f \, \v k \wedge \v V_h \]
où $\v V_h = u \v i + v \v j$ est la vitesse horizontale et 
\[ \boxed{ f = 2 \, \Omega \, \sin \phi } \]
est appelé \voc{facteur de Coriolis}. Aux moyennes latitudes ($\phi=45$\deg), la valeur de~$f$ est environ~$10^{-4}$~s$^{-1}$. 
%Les composantes de la force de Coriolis sont \[\v F_C=-2\Omega\left(\begin{array}{c}0\\\cos\phi\\\sin\phi\end{array}\right) \wedge\left(\begin{array}{c}u\\v\\w\end{array}\right) =-2\Omega\left(\begin{array}{c}w\cos\phi-v\sin \phi\\u\sin \phi\\-u\cos \phi\end{array}\right)\]
%\footnote{Pour des mouvements de
%type ``chute libre'', la vitesse verticale $w$ domine. On peut alors mettre en
%évidence une déviation vers l'est, mais qui reste très faible (de l'ordre de
%1cm pour 80m de chute).} 

\begin{table}
  \centering
  \begin{tabular}{cccccccc}
    \hline
    Équation-$x$ & $\frac{du}{dt}$ & $-2\Omega v\sin\phi$ & $+2\Omega
    w\cos\phi$ & $+\frac{uw}{a}$ & $-\frac{uv\tan\phi}{a}$ &=&
    $-\frac{1}{\rho}\frac{\partial P}{\partial x}$ \\
    Équation-$y$ & $\frac{dv}{dt}$ & $+2\Omega u\sin\phi$ &&         
                $+\frac{vw}{a}$ & $+\frac{u^2\tan\phi}{a}$ &=&
    $-\frac{1}{\rho}\frac{\partial P}{\partial y}$ \\
    Échelles & $U^2/L$ & $fU$ & $fW$ & $UW/a$ & $U^2/a$ && $\delta P/(\rho L)$
    \\
    m.s\md & 10$^{-4}$ & 10$^{-3}$ & 10$^{-6}$ & 10$^{-8}$ & 10$^{-5}$ &&
    10$^{-3}$ \\
    \hline
  \end{tabular}
  \caption{\emph{Analyse en ordre de grandeur de l'équation du mouvement
  horizontale.}}
  \label{tab:hqmouv}
\end{table}

\sk
Sur un plan horizontal, les termes restants de l'équation du mouvement sont ainsi:
%\begin{equation}
\[  \frac{d\v V_h}{dt}+f\v k\wedge\v V_h=\v F_P  \]
%  \label{eq:hqmouv}
%\end{equation}
avec $\v V_h$ la vitesse horizontale, et $\v F_P$ les forces de pression horizontales massiques. Pour évaluer lequel des deux termes à gauche domine, on définit le \voc{nombre de Rossby} $\mathcal{R}$, rapport entre accélération relative et de Coriolis
\[ \mathcal{R} = \frac{U^2/L}{f\,U} = \frac{U}{f\,L} \]
Avec $f$=10$^{-4}$~s$^{-1}$ aux moyennes latitudes et $U$=10~m~s$^{-1}$, on a $\mathcal{R}=0.1$ aux grandes échelles de la circulation terrestre ($L$=1000~km), donc Coriolis domine. Au contraire, à une échelle plus petite de $L$=10~km, $\mathcal{R}=10$ et Coriolis devient négligeable.




\sk
\subsubsection{Equilibre géostrophique}
	\sk
Dans le cas d'un nombre de Rossby petit (donc $L$>1000~km aux moyennes latitudes), on est proche d'un équilibre appelé \voc{équilibre géostrophique} entre les forces de Coriolis et de pression
\[ \boxed{ \v F_C+\v F_P=\v 0 } \]
qui s'écrit selon les deux composantes horizontales
\[ \boxed{ \binom{f \, v}{-f \, u} = \binom{\frac{1}{\rho} \,\frac{\partial P}{\partial x}}{\frac{1}{\rho} \,\frac{\partial P}{\partial y}} } \]
Le vent qui vérifie exactement cet équilibre est appelé \voc{vent géostrophique}~$\v V_g$. Sous forme vectorielle on a $f\v k\wedge\v V_g=\v F_P$ et sous forme projetée
\[ \v V_g = \binom{u}{v} = \binom{- \frac{1}{\rho \, f} \, \frac{\partial P}{\partial y}}{\frac{1}{\rho \, f} \, \frac{\partial P}{\partial x}} \]
%\begin{equation}
%  \v V_g=\frac{1}{\rho f}\v k\wedge\vl{grad}_z(P)=\frac{g}{f}\v k\wedge\vl{grad}_P(Z)
%  \label{eq:geost}
%\end{equation}

\figun{1.1}{0.3}{\figfrancis/geost}{Forces et vent dans l'équilibre géostrophique (hémisphère nord).}{fig:geost}

\sk
L'équilibre géostrophique peut s'illustrer graphiquement (voir figure~\ref{fig:geost}), formant ce que l'on appelle la loi de Buys-Ballot\footnote{Comme l'indique Buys-Ballot dans son article de 1857~: \emph{Note sur le rapport de l'intensité et de la direction du vent avec les écarts simultanés du baromètre ; [...] Ce n'est pas la girouette, mais c'est le baromètre d'après lequel on doit juger le vent [...] La grande force du vent est annoncée par une grande différence des écarts simultanés du baromètre dans les Pays-Bas [...] Pour un autre pays, on devra étudier les modifications.}}.
\begin{description}
\item[Direction] Comme la force de Coriolis est orthogonale au vecteur vitesse, et opposée à la force de pression, le vent géostrophique est lui-même orthogonal aux variations horizontales de pression donc parallèle aux isobares.
\item[Sens] Dans l'hémisphère nord, les basses pressions sont à gauche du vent, à droite dans l'hémisphère sud.
\item[Module] La vitesse du vent géostrophique est proportionnelle aux variations horizontales de pression~; autrement dit, plus les isobares sont resserrées, plus le vent est fort.
\end{description}
La carte de la pression et du vent en surface (figure \ref{fig:meteofrance}, voir aussi figure \ref{fig:SLPwind}) montre clairement que l'orientation et le module du vent sont dictés par l'équilibre géostrophique. Lorsque la friction est élevée proche de la surface, l'équilibre géostrophique est perturbé par la présence de la force de friction, ce qui a pour conséquence de donner un vent légèrement dévié vers l'intérieur des dépressions et vers l'extérieur des anticyclones. Quand le nombre de Rossby est grand (donc à petite échelle), l'équilibre géostrophique ne s'applique plus et le vent est accéléré des hautes vers les basses pressions. 
%% PARLER DU LAVABO

\figside
%{0.85}{0.6}
{0.65}{0.4}
{decouverte/cours_dyn/carte_france.jpg}{Carte météorologique Météo-France construite à partir des données relevées dans les stations météorologiques indiquées par des points. Les lignes isobares montrent qu'une forte dépression se situe au nord du Royaumu-Uni. Les \ofg{drapeaux} accolés aux points d'observations représentent les vents mesurés à la surface (le nombre de barres indique la force du vent). La direction du vent part du drapeau vers le point considéré. On remarque que le vent est approximativement parallèle aux isobares et tourne dans le sens inverse des aiguilles d'une montre autour de la dépression. Ce comportement est typique de celui déduit pour l'hémisphère Nord par l'équilibre géostrophique entre forces de pression et force de Coriolis (voir figure~\ref{fig:geost}). Le vent est légèrement rentrant vers l'intérieur de la dépression, sous l'influence de la force de friction qui vient d'ajouter aux deux forces précitées.}{fig:meteofrance}




\mk
\section{Circulation atmosphérique~: généralités}

\sk
\subsection{Structure en latitude}

\sk
Au premier ordre, les caractéristiques de l'atmosphère dépendent essentiellement de la latitude. On a en particulier un contraste entre les régions tropicales, comprises entre 30\deg sud et nord, et les latitudes moyennes (autour de 45\deg) et hautes latitudes (régions polaires). Ces variations apparaissent clairement en observant des moyennes sur toutes les longitudes, ou moyennes zonales (figure \ref{fig:UTlatP}). Ce contraste résulte à la fois de l'influence de la rotation de la planète et du chauffage différentiel qui rend les régions tropicales excédentaires en énergie alors que les moyennes et hautes latitudes sont déficitaires.

\figsup{0.48}{0.25}{\figfrancis/T_latP_jan}{\figfrancis/U_latP_jan}{Coupes latitude-pression de la température (haut) et du vent zonal (bas), en moyenne climatique et zonale, pour le mois de janvier. L'utilisation de la pression comme coordonnée verticale permet de se focaliser sur la troposphère.}{fig:UTlatP}

\sk
La température décroit partout sur la verticale jusqu'à un minimum à la tropopause, située entre 100~hPa dans les tropiques (où la température minimale est atteinte) et 300~hPa aux moyennes latitudes. Sur l'horizontale, la température est maximale et presque constante dans les tropiques, puis décroit très rapidement vers les pôles aux latitudes moyennes. La répartition de la vapeur d'eau (fig \ref{fig:humspec}) est très liée à la température: on observe un maximum dans les zones chaudes tropicales près de la surface, et peu d'eau en altitude ou aux latitudes polaires. La vapeur d'eau est également absente dans la stratosphère malgré la température élevée, à cause de l'absence de sources locales: la vapeur d'eau provient de l'évaporation en surface et ne peut franchir le piège froid à la tropopause.

\figside{0.5}{0.15}{\figfrancis/hum_spec}{Moyenne (zonale et temporelle) du rapport de mélange massique de vapeur d'eau (en g/kg d'air).}{fig:humspec}

\figun{0.98}{0.3}{\figfrancis/MMC}{Circulation moyenne (zonale et temporelle) dans le plan méridien. La circulation (schématisée par les flèches) est parallèle aux isolignes de la fonction de courant, et le flux de masse (débit) entre deux isolignes est constant. En vert, valeurs positives d'environ $1 \times 10^{11}$~kg~s$^{-1}$, qui correspondent à une rotation horaire. En marron, valeurs négatives d'environ $-1 \times 10^{11}$~kg~s$^{-1}$, qui correspondent à une rotation anti-horaire.}{fig:MMC} %%{\figfrancis/MMC_legende}

\sk
La structure du vent zonal est dominée aux moyennes latitudes par la présence de deux \voc{jets}, c'est-à-dire de puissants courants atmosphériques, dits \voc{jets d'ouest} car ils soufflent de l'ouest vers l'est. Leur vitesse augmente sur la verticale entre la surface et un maximum au niveau de la tropopause, autour de 50~m~s$^{-1}$. Ce comportement peut être justifié en combinant l'équilibre géostrophique à l'équilibre hydrostatique (équation du vent thermique). Dans les tropiques, les vents moyens sont d'est, surtout dominants dans la basse troposphère, mais restent néanmoins moins forts que les vents d'ouest dans les moyennes latitudes. On les appelle les \voc{alizés}.

\sk
La circulation dans le plan méridien (sud-nord et verticale) est caractérisée par une série de cellules fermées. Le chauffage différentiel explique qu'une différence de pression naisse entre les tropiques et les moyennes latitudes, car la pression diminue plus vite avec l'altitude dans les couches d'air froid des moyennes latitudes que dans les couches d'air chaud des tropiques. Ceci donne naissance en altitude à des vents de l'équateur vers les pôles. Ces vents induisent un flux de masse atmosphérique vers les moyennes latitudes, donc, d'après l'équivalence entre pression et masse, une augmentation de la pression de surface aux moyennes latitudes par rapport aux tropiques. Ceci donne naissance proche de la surface à des vents des pôles vers l'équateur. Par continuité, dans les tropiques, l'air s'élève proche de l'équateur (suivant la saison, du côté de l'hémisphère d'été) et redescend au niveau des subtropiques. Ce système est appelé \voc{cellules de Hadley}. On observe également dans les moyennes latitudes des cellules moins intenses, contrôlées par les instabilités dans l'atmosphère, appelées cellules de Ferrel. 
%%(figure \ref{fig:MMC})

\sk
La structure en latitude des vents %décrite par la figure~\ref{fig:UTlatP}, 
avec des vents d'ouest aux moyennes latitudes et d'est sous les tropiques, est très liée à la circulation de Hadley.
% décrite par la figure~\ref{fig:MMC}.
Sous l'action de la force de Coriolis, les mouvements vers les pôles sont déviés vers l'est et les mouvements vers l'équateur sont déviés vers l'ouest. Les jets d'ouest des moyennes latitudes proviennent ainsi de la déviation vers l'est de la circulation vers les pôles dans la branche supérieure de la cellule de Hadley. Les vents d'est (alizés) sous les tropiques proviennent quant à eux de la déviation vers l'ouest de la circulation vers l'équateur dans la branche inférieure de la cellule de Hadley. Les vents de grande échelle comportent donc une composante vers l'équateur et l'ouest sous les tropiques, alors qu'aux moyennes latitudes, ils comportent une composante vers les pôles et l'est [la composante vers l'est domine cependant]. Une exception à cette image est observée dans les régions de ``mousson'' (sous-continent Indien, et dans une moindre mesure Afrique de l'ouest et Amérique centrale) où la direction du vent s'inverse entre l'été (vers le continent) et l'hiver (vers l'océan).

\figside{0.6}{0.3}{\figfrancis/WH_circ_scheme}{Schéma de la circulation atmosphérique: zone de convergence et alizés dans les tropiques; gradient de pression tropiques (H) -pôle (L), vents d'ouest et ondes aux moyennes latitudes. La position des jets d'ouest et l'extension des cellules de Hadley sont représentées à droite. Figure adaptée de Wallace and Hobbs, Atmospheric Science, 2006.}{fig:circscheme}
 

\sk
\subsection{Structure en longitude}

\sk
Le champ de pression au niveau de la mer est relativement symétrique en longitude dans l'hémisphère sud, et varie peu suivant les saisons (figure \ref{fig:SLPwind}): on observe une ceinture de hautes pressions aux latitudes subtropicales (vers 30\deg), une pression un peu plus faible vers l'équateur, et une baisse rapide de la pression vers le pôle, avec un minimum autour de 60\deg. Dans l'hémisphère nord, des variations est-ouest liées aux contrastes continent-océan se rajoutent à cette structure en latitude. L'été, on observe des pressions relativement basses sur les continents chauds, et des hautes pressions sur les océans (anticyclones des açores dans l'Atlantique et d'Hawaï dans le Pacifique). L'hiver, ces anomalies s'inversent et on a des minimums de pression sur les océans (dépressions d'Islande et des Aléoutiennes) et des hautes pressions sur les continents froids (anticyclone de Sibérie).

\sk
On remarque que le vent a tendance à s'enrouler autour des extrema de pression isolés, dépressions et anticyclones. Il laisse les basses pressions à sa gauche dans l'hémisphère nord, et les hautes pressions à droite. Cette loi (de ``Buys-Ballot'') s'inverse dans l'hémisphère sud. On a vu à la section précédente qu'il s'agit d'une conséquence de l'équilibre géostrophique aux moyennes latitudes.

\figside{0.7}{0.35}{\figfrancis/WH_surfw_slp}{Vent et pression de surface observés. Le champ de pression est ramené au niveau de la mer afin de supprimer la composante permanent causée par les différences topographiques et d'obtenir une carte montrant uniquement les variations météorologiques de pression. Figure adaptée de Wallace and Hobbs, Atmospheric Science, 2006.}{fig:SLPwind}

\sk
La carte des précipitations moyennes (figure \ref{fig:seasprecip}) dans les tropiques une concentration dans une mince bande proche de l'équateur. Cette zone étroite correspond à la région de convergence des vents de surface, dirigés vers l'équateur sous l'effet des circulations type cellules de Hadley, d'où son nom de \voc{Zone de Convergence Intertropicale (ZCIT)}. On a également de fortes pluies un peu plus loin de l'équateur au cours des moussons d'été. Ces zones de pluies intenses sont aussi des régions d'ascendance à grande échelle. Au delà des subtropiques très sèches, dans lesquelles se trouvent la plupart des déserts de la planète, on retrouve d'autres régions de pluie sur les océans des latitudes moyennes. Ces pluies sont cette fois liées au passage des dépressions, et pas à une zone de convergence particulière.

\figun{0.65}{0.4}{\figfrancis/WH_precip_seas}{Précipitations moyennes saisonnières, en décembre (haut) et juillet (bas). Figure adaptée de Wallace and Hobbs, Atmospheric Science, 2006.}{fig:seasprecip}

\sk
\subsection{Circulations transitoires}

\sk
A la circulation moyenne décrite ci-dessus se superpose une circulation transitoire, qui varie d'un jour sur l'autre. La comparaison entre la vapeur d'eau instantanée et moyennée sur un mois (figure \ref{fig:wavevap}) montre la signature de cette circulation dans les basses couches de l'atmosphère: les variations horizontales de vapeur d'eau viennent du transport par la circulation. 

\figsup{0.7}{0.2}{\figfrancis/tcwv_month}{\figfrancis/tcwv_day}{Cartes de quantité de vapeur d'eau totale intégrée sur la verticale (kg~m$^{-2}$). Moyenne sur le mois de décembre 1999 (haut), et instantané au premier décembre 1999 (bas).}{fig:wavevap}

\sk
On reconnait dans la distribution instantanée les grandes régions sèches et humides des tropiques. A l'endroit de la transition vers les latitudes moyennes, on observe en revanche des filaments d'air qui s'enroulent, entrainés par une circulation tourbillonaire. Ces \voc{ondes baroclines} sont responsables des alternances fréquentes de temps sec et humide des régions tempérées. On observe environ 5 à 8 structures alternées sur un cercle complet de longitude, soit une longueur d'onde de quelques milliers de kilomètres (voir également la figure~\ref{fig:press}). Leur période est de quelques jours. Ces ondes sont également visibles dans la haute troposphère, toujours aux latitudes moyennes 
%(figure \ref{fig:wavepv}) 
autour de la position du jet. Elles ne pénètrent pas en revanche dans la stratosphère.

%\begin{figure}[tbp]
%  \begin{center}
%    \includegraphics[width=12cm]{\figfrancis/pv_month}
%    \\
%    \includegraphics[width=12cm]{\figfrancis/pv_day}
%  \end{center}
%  \caption{Cartes de la vorticité potentielle (un traceur dynamique conservé
%  au cours du mouvement) à 250~hPa: mois de décembre 1999 (haut) et premier
%  décembre (bas). Les valeurs absolues élevées aux pôles correspondent à de
%  l'air stratosphérique.}
%  \label{fig:wavepv}
%\end{figure}

\sk
\subsection{Résumé}

\sk
Les caractéristiques de la circulation atmosphérique sont donc très différentes dans deux zones qui couvrent chacune environ la moitié de la planète (figure \ref{fig:circscheme}):
\begin{description}
\item[Tropiques] Les tropiques sont marquées par des gradients horizontaux très faibles de température, mais des variations d'humidité marquées entre régions humides et sèches. Ces régions se déplacent à l'échelle saisonnière, mais restent stables à des périodes plus courtes (mais la précipitation dans les régions humides peut varier rapidement). La circulation est dominée par des cellules avec ascendance dans les zones de convergence, et subsidence au dessus des déserts. En surface, on a des vents d'est réguliers (alizés) qui convergent près de l'équateur.
\item[Moyennes latitudes] La région des latitudes moyennes est marquée au contraire par des gradients de température et de pression très forts. Les vents sont d'ouest en moyenne en surface et culminent avec un jet rapide au niveau de la tropopause. A cette circulation moyenne se rajoute une circulation horizontale intense de type ondulatoire à turbulente, qui, liée aux fortes variations horizontales, peut donner des variations très fortes et rapides de température ou d'humidité.
\end{description}
%%ces différences de comportement sont principalement dues à l'influence différente de la rotation de la Terre.



%\end{document}

%\subsection{Vent thermique}
%Le vent thermique décrit la variation verticale du vent géostrophique. Le nom
%vient du fait que les variations verticales de pression sont liées par
%l'équilibre hydrostatique à la température (équation \ref{eq:hypso}). Le
% résultat est plus simple à obtenir en utilisant la {\em pression} comme
%coordonnée verticale: en dérivant l'expression (\ref{eq:geost}) du vent
%géostrophique (version gradient du géopotentiel) par rapport à $P$, on
%obtient:
%\[\frac{\partial \v V_g}{\partial P}=\frac{g}{f}\v
%k\wedge\vl{grad}_P\left(\frac{\partial z}{\partial P}\right)\]
%On a utilisé la permutation des dérivations horizontale et verticale, et le
%fait que $g$ ne dépend pas de $P$. 
%Le second membre peut être transformé en utilisant l'équilibre hydrostatique
%(\ref{eq:hydro}):
%\[\frac{\partial z}{\partial P}=-\frac{1}{\rho g}=-\frac{RT}{gP}\]
%Le gradient horizontal étant pris à pression constante, seule $T$ varie. On
%obtient alors:
%\begin{equation}
%  \frac{\partial \v V_g}{\partial P}=-\frac{R}{fP}\v k\wedge\vl{grad}_P(T)
%  \label{eq:vth}
%\end{equation}
%Cette équation peut être intégrée (après avoir multiplié par $P$) entre deux
%niveaux de pression $P_1$ et $P_2$:
%\begin{equation}
%  \v V_{g2}-\v V_{g1}=\frac{R}{f}\ln{\frac{P_1}{P_2}}\v k\wedge\vl{grad}<T>
%  \label{eq:vthi}
%\end{equation}
%Où la température moyenne $<T>$ est définie comme pour l'équation
%hypsométrique (\ref{eq:hypso}). Cette forme intégrée peut être retrouvée
%directement à partir de (\ref{eq:hypso}) et (\ref{eq:geost}). La différence
%$\v V_{g2}-\v V_{g1}$ est appelée {\em vent thermique}, avec le niveau 2 situé
%à une altitude plus élevée que le niveau 1. Dans l'hémisphère nord, le vent
%thermique est dirigé parallèlement aux isothermes, avec les températures
%élevées à droite.
%
%La dérivée de $\v V_g$ par rapport à $z$ s'obtient en multipliant
%(\ref{eq:vth}) par $\partial P/\partial z=-\rho g$:
%\[\frac{\partial \v V_g}{\partial z}=\frac{g}{fT}\v k\wedge\vl{grad}_P(T)\]
%
%%\section{Mouvement inertiel}
%%\section{Rôle de la friction}
%%\section{Equilibre cyclostrophique}

	
\end{document}
