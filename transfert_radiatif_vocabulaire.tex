\sk
Un certain nombre de termes sont couramment utilisés pour désigner des situations particulières en terme d'absorptivité, de transmittivité, de réflectivité spectrale. Il est important de préciser à quelle longueur d'onde~$\lambda$ on se place lorsqu'on qualifie les propriétés d'un corps matériel (ce n'est pas toujours clair dans les ouvrages).
\begin{citemize}
\item Un corps \voc{transparent} à la longueur d'onde~$\lambda$ est tel que~: $\tau_\lambda = 1$, donc~$\rho_\lambda = \alpha_\lambda = 0$
\item Un corps \voc{opaque} à la longueur d'onde~$\lambda$ est tel que~: $\tau_\lambda = 0$, donc~$\rho_\lambda + \alpha_\lambda = 1$
\item Un corps \voc{brillant} à la longueur d'onde~$\lambda$ est tel que~: $\rho_\lambda = 1$, donc~$\tau_\lambda = \alpha_\lambda = 0$
\item Un corps \voc{sombre} à la longueur d'onde~$\lambda$ est tel que~: $\rho_\lambda = 0$, donc~$\tau_\lambda + \alpha_\lambda = 1$
\end{citemize}
Le corps noir étant un absorbant idéal, son coefficient spectral d'absorption~$\alpha_\lambda$ vaut 1 pour toutes les longueurs d'onde~$\lambda$. Il vérifie donc également~$\rho_\lambda = \tau_\lambda = 0$ pour toutes les longueurs d'onde. Par extension, un corps est qualifié de \voc{presque noir} à la longueur d'onde~$\lambda$ si~$\alpha_\lambda = 1$ à la longueur d'onde~$\lambda$. Un corps gris tel que défini au chapitre précédent est tel que son absorptivité~$\alpha_\lambda$ est la même selon toutes les longueurs d'onde.

\sk
On n'utilise le terme de~\ofg{complètement} noir, opaque, brillant, \ldots~que lorsque la propriété est vérifiée pour toutes les longueurs d'onde du spectre électromagnétique. Ce n'est en fait quasiment jamais le cas en pratique. La neige fraîche en est un excellent exemple~: elle apparaît brillante dans le visible car elle réfléchit le rayonnement incident ($\rho_{\lambda\e{VIS}}=1$). Il serait erroné de croire que c'est le cas dans toutes les longueurs d'ondes. La neige fraîche est presque noire dans l'infrarouge où elle est très absorbante ($\alpha_{\lambda\e{IR}}=1$) donc elle est sombre dans l'infrarouge ($\rho_{\lambda\e{IR}}=0$). Autre exemple, le verre est transparent pour les longueurs d'onde visible ($\tau_\lambda = 1$), mais relativement opaque pour le rayonnement infrarouge ($\tau_\lambda \rightarrow 1$).
%une surface \ofg{brillante} dans le visible ($A_{\lambda}$ proche de~$1$) peut s'avérer \ofg{sombre} dans l'infrarouge ($A_{\lambda} \ll 1$)
