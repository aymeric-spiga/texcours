
Basse atmosphère: \voc{équilibre thermodynamique local}. Ces conditions sont vérifiées 
si les collisions entre molécules 
sont plus fréquentes que 
l'absorption ou émission de rayonnement. 
Les molécules émettrices ont alors la même température que leur environnement

l'équilibre radiatif à l'ETL indique que le flux radiatif
dans ces couches de l'atmosphère est constant. ce n'est plus vrai.
l'atmosphère ne peut plus échanger de l'énergie par radiation.



%\slide[shrink=5]{%
Zones conductrices: thermosphères%
%}{
%%%%%%%%%%%%%%%%%%%%%%%%%%%%%%%%%%%%%%%%%%%%%%%%%%%%%%%%%%%%%%%
%%%%%%%%%%%%%%%%%%%%%%%%%%%%%%%%%%%%%%%%%%%%%%%%%%%%%%%%%%%%%%%
%{\footnotesize
%\begin{columns}
%\column{.8\textwidth}
\begin{itemize}
\item $dT/dz \gg 0$ dû à une source de chaleur à haute altitude
\item La chaleur est transportée par \emph{conduction} jusqu'à la
  mésosphère où elle peut être rayonnée.
\item En régime permanent, $K_c \frac{\partial T}{\partial z} =
  \int_{z}^{\infty} q(z) \, dz = Q(z)$.
\item En supposant une source à haute altitude $+Q$ à $z=z_1$ et un
  puits mésophérique $-Q$ à $z=z_0$ ($K_c = A T^s$), on obtient :
\begin{center}
%  \alert{
$T^{s+1} - T_0^{s+1} = \frac{(s+1)Q}{A} \left( z - z_0 \right)$
% }
\end{center}
\item On peut alors estimer $T_1$ température exosphérique.
\end{itemize}
%\column{.2\textwidth}
\includegraphics[width=0.3\textwidth]{decouverte/cours_dyn/thermosphere.png}
%\end{columns}
%\begin{columns}
%\begin{column}{.2\columnwidth}
\begin{tabular}{lc}
\hline
 & $T_1$ (K) \\
\hline
Vénus & $110\sim 300$\\
Terre & $1000$ \\
Mars & $200$ \\
Titan & $186$\\
\hline
\end{tabular}
%\end{column}
%\begin{column}{.05\columnwidth}
%\invisible{Yorgl}
%\end{column}
%\begin{column}{.75\columnwidth}
\begin{itemize}
\item {\bf Vénus \& Mars} : CO$_2 \rightarrow$ O$_2$ photolysé, mais CO$_2$
  bon radiateur même hors ETL 
\item {\bf Terre} : photodissociation UV de O$_2$, peu de CO$_2$ :
  thermosphère chaude !
\item {\bf Titan} : absorption EUV, mais bon refroidissement (HCN).
\end{itemize}
%\end{column}
%\end{columns}
%}
%%%%%%%%%%%%%%%%%%%%%%%%%%%%%%%%%%%%%%%%%%%%%%%%%%%%%%%%%%%%%%%
%%%%%%%%%%%%%%%%%%%%%%%%%%%%%%%%%%%%%%%%%%%%%%%%%%%%%%%%%%%%%%%
%\source{Transparent emprunté à E. Marcq}
