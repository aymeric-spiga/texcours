\sk
On donne ici quelques éléments éclairants sur les processus en jeu à l'échelle microscopique lors de l'absorption de rayonnement incident (photons) par les molécules qui composent l'atmosphère\footnote{Cette partie est inspirée d'éléments trouvés dans le cours de S. Jacquemoud de \emph{Méthodes physique en télédétection.}}. Cette absorption est en fait liée à leurs caractéristiques énergétiques. L'énergie d'une molécule ne peut prendre que des valeurs discrètes correspondant à des niveaux énergétiques. On les représente
souvent par un diagramme dans lequel chaque niveau est figuré par un trait horizontal. L'absorption permet à une molécule de passer d'un niveau d'énergie~$e_1$ à un niveau d'énergie supérieur~$e_2$. Le passage du niveau~$e_1$ au niveau~$e_2$ s'accompagne de l'absorption d'un rayonnement de fréquence~$\nu$ telle que~$\Delta e = e_2 - e_1 = h \, \nu$, soit l'énergie d'un photon de fréquence~$\nu$. Les molécules possèdent une énergie électronique~$e\e{e}$, quantifiée comme les atomes, mais aussi une énergie de vibration~$e\e{v}$ et une énergie de rotation~$e\e{r}$, elles aussi quantifiées. Une bonne approximation de l'énergie totale $e\e{t}$ est donnée par la relation~$e\e{t} = e\e{e} + e\e{v} + e\e{r}$. A chaque état électronique correspondent plusieurs états de vibration des noyaux et à chaque état vibrationnel correspondent plusieurs états de rotation.

\figsup{0.47}{0.17}{decouverte/cours_meteo/electronique.png}{decouverte/cours_meteo/vibre.png}{Niveaux électroniques définissant des états de molécules [gauche]. Effet de l'interaction entre rayonnement et molécules pour plusieurs longueurs d'onde. Figures extraites du cours de S. Jacquemoud de \emph{Méthodes physique en télédétection.}}{fig:electronique}

\sk
L'interaction entre le rayonnement et les molécules constituant le milieu (par exemple, l'atmosphère) se manifeste d'une façon différente selon l'énergie~$h\nu$ du photon incident. Ainsi, par ordre décroissant de l'énergie du photon incident, donc par ordre croissant de sa longueur d'onde~$\lambda$, le photon va provoquer sur les liaisons moléculaires des brisures, des réorganisations de nuage électronique, des vibrations, ou simplement des rotations.
\begin{finger}
\item dans l'ultraviolet : les molécules (O$_2$, O$_3$, NO$_2$, \ldots) sont dissociées. La photolyse ou \voc{photodissociation} d'une espèce est provoquée par l'absorption d'un photon possédant une énergie suffisante pour conduire cette espèce à un état électronique excité puis finalement à une rupture de liaison. Citons les cas classiques de la dissociation de l'oxygène [O$_2$~+~$h\nu$~$\rightarrow$~O~+~O] pour des longueurs d'onde inférieures à 246 nm, ou de l'ozone [O$_3$~+~$h\nu$~$\rightarrow$~O$_2$~+~O] pour des longueurs d'onde inférieures à 310 nm. La photodissociation de NO$_2$ produit les atomes d’oxygène nécessaires à la formation photochimique de l’ozone troposphérique [NO$_2$~+~$h\nu$~+O$_2$~$\rightarrow$~NO~+O$_3$]. Ces phénomènes de photolyse participent très fortement à la chimie de l'atmosphère.
\item dans le visible : les molécules changent de configuration électronique ; les électrons qui gravitent autour du noyau atomique peuvent changer d'orbite ou même d'atome. Les photons du domaine du visible ne sont presque pas absorbés par l'atmosphère (très légèrement par O$_2$ et O$_3$) et sont donc uniquement diffusés.
\item dans l'infrarouge moyen et thermique : les molécules (CO$_2$, H$_2$O, CH$_4$, N$_2$O, \ldots) vibrent dans l'axe de la liaison moléculaire (étirement) ou perpendiculairement à cet axe (pliage). Ces molécules sont appelées \voc{gaz à effet de serre} car elles absorbent le rayonnement infrarouge thermique émis par la Terre puis réemettent des photons à la même longueur d'onde.
\item dans le domaine des micro-ondes : les molécules tournent autour d'un de leurs axes.
\end{finger}
