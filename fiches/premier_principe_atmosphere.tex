\sk
Afin de travailler sur des grandeurs intensives, on divise la relation précédente par la masse~$m$ de la parcelle pour obtenir
\[ \EE \]
où $\delta q$ est la chaleur massique reçue et $C_P = C_P^* / M$ est la \voc{chaleur massique de l'air} ($C_P$=1004 J~K$^{-1}$~kg$^{-1}$). Nous disposons alors d'une autre version du premier principe, très utile en météorologie et valable pour une transformation quelconque d'une parcelle d'air
\[ \boxed{ \underbrace{\textcolor{white}{\frac{R^2}{C_P}} \dd T \textcolor{white}{\frac{R}{C_P}}}_{\text{variation de température de la parcelle}} = \underbrace{\frac{R}{C_P} \, \frac{T}{P} \, \dd P}_{\text{travail expansion/compression}} + \underbrace{\frac{1}{C_P} \, \delta q}_{\text{chauffage diabatique}} } \]

\sk
Autrement dit, la température de la parcelle augmente si elle subit une compression ($\dd P > 0$) et/ou si on lui apporte de la chaleur ($\delta q > 0$). La température de la parcelle à l'inverse diminue si elle subit une détente ($\dd P < 0$) et/ou si elle cède de la chaleur à l'extérieur ($\delta q < 0$). Il est donc important de retenir que la température de la parcelle peut très bien varier quand bien même la parcelle n'échange aucune chaleur avec l'extérieur~: dans ce cas, $\delta q = 0$ et l'on parle de \voc{transformation adiabatique}. 

\sk
L'équation fondamentale ci-dessus est directement dérivée du premier principe, mais prend une forme plus pratique en sciences de l'atmosphère du fait que les transformations que subit une parcelle atmosphérique se réduisent en général aux transformations \voc{isobares} (à pression constante $\dd P = 0$) et aux transformations \voc{adiabatiques} (sans échanges de chaleur avec l'extérieur $\delta q = 0$). Les transformations isothermes, au cours de laquelle la température de la parcelle ne varie pas, sont plus rarement rencontrées en sciences de l'atmosphère.


