\sk
Dans toutes les atmosphères connues, à grande échelle, un quasi-équilibre s'établit entre le champ de masse atmosphérique et la composante horizontale de la force centrifuge, liée au vent zonal (Figure~\ref{fig:vg}). Il s'agit de l'\voc{équilibre du vent gradient}, vrai en moyenne zonale. L'advection de l'air devient de moins en moins efficace quand on s'éloigne de l'équateur : finalement un quasi-équilibre s'établit entre le gradient de pression et la composante horizontale de la force centrifuge, prépondérantes dans l'équation du mouvement méridien. Plus la planète tourne vite, plus cet effet de frein est prépondérant. 
%\sk Le vent gradient est un équilibre diagnostic alors que les équilibres géostrophiques et cyclostrophiques sont des équations prognostiques

\[
\boxed{\dfrac{u^2\tan\phi}{a} + \fcoriolis u = -\dfrac{1}{\rho}\der{p}{y}}
\]

\figsup{0.5}{0.25}{decouverte/cours_dyn/td2_pression.png}{decouverte/cours_dyn/td2_centrifuge.png}{Gradient de pression (haut). Force centrifuge (bas).}{fig:vg}

