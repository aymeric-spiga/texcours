\sk
Toute différence de température entre deux régions (provoquée par exemple par un chauffage différentiel, ou par une différence des propriétés thermophysiques de la surface) est associée à des différences de pression, car d'après l'équation hypsométrique (équilibre hydrostatique + équation d'état du gaz parfait) la pression diminue plus vite avec l'altitude dans les couches d'air froid que dans les couches d'air chaud. Ceci donne naissance en altitude à un gradient de pression donc, en supposant que la force de pression est seule responsable de l'accélération du vent (vision à raffiner par la suite), des vents vont naître en altitude de la région chaude vers la régions froide. Ces vents induisent un flux de masse atmosphérique de la région chaude vers la région froide, donc causent, d'après l'équivalence entre pression et masse déduite de l'équilibre hydrostatique, une augmentation de la pression de surface dans la région froide par rapport à la région chaude. Ceci donne naissance proche de la surface à des vents de la région chaude vers la région froide. Par continuité, en considérant les convergences et divergences d'air proche du sol et en altitude, l'air s'élève dans les régions chaudes et redescend dans les régions froides.

\sk
Des exemples de circulations thermiques directes sont
\begin{finger}
\item les \voc{cellules de Hadley}, cellules fermées dans le plan méridien, sud-nord et verticale; sous les tropiques, l'air s'élève proche de l'équateur (suivant la saison, du côté de l'hémisphère d'été) et redescend au niveau des subtropiques.
\item les \voc{\og brises \fg~de mer et de terre} au bord de la mer sur Terre, naissant du contraste thermique entre continent et océan
\item les circulations atmosphériques sur Mars entre les régions polaires couvertes de glace et les régions de sol nu
\end{finger}

\sk
Les cellules fermées associées aux circulations thermiques directes \underline{ne sont pas des cellules de convection}. Elles résultent simplement de la déformation du champ de pression par des contrastes de température. Des cellules fermées non convectives peuvent également se développer dans le sens inverse de celui thermique direct (par exemple, les cellules de Ferrel sur Terre) : les mécanismes sont distincts des processus de circulation thermique directe et sont en général relatifs au forçage de l'écoulement moyen par les ondes atmosphériques résultant d'instabilités dans l'atmosphère.



