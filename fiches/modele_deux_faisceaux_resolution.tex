\sk
Le système d'équations~$[S^+]$ et~$[S^-]$ du modèle à deux faisceaux
est plus simple à résoudre si l'on considère les deux quantités~$\Sigma(\tau)=F^{+}(\tau)+F^{-}(\tau)$ et~$\Delta(\tau)=F^{+}(\tau)-F^{-}(\tau)$ car on obtient
\[
\ddf{\Sigma}{\tau} = \Delta(\tau) \quad [E_\Sigma] 
\qquad\qquad 
\ddf{\Delta}{\tau} = \Sigma(\tau) - 2\,\epsilon\,\sigma\,T(\tau)^4 \quad [E_\Delta]
\]
\noindent Ensuite la résolution impose d'expliciter les conditions aux limites
\begin{enumerate}[$\mathcal{C}_1$]
\item on se place à l'équilibre radiatif donc le flux net~$\Delta$ est constant à tout niveau : $\ddf{\Delta}{\tau}=0$
\item au sommet de l'atmosphère $F^+(\tau=0) = OLR$ (définition de $OLR$) et $F^-(\tau=0) = 0$ (contribution
incidente négligeable du Soleil dans l'infra-rouge), ce qui s'écrit encore~$\Delta(\tau=0)=\Sigma(\tau=0)=OLR$
\item à la surface de température~$T_s$ le bilan radiatif est le suivant : la surface reçoit l'intégralité du rayonnement
solaire incident~$(1-A\e{b}) \, \mathcal{F}\e{s}'$ (visible) plus du rayonnement de l'atmosphère située
juste au-dessus d'elle~$F^-(\tau=\tau_{\infty})$ (infra-rouge) ; de plus elle émet un rayonnement
$\epsilon\,\sigma\,T\e{s}^4$ dans l'infra-rouge vers l'atmosphère\footnote{On a supposé ici pour simplifier les calculs que l'émissivité
de la surface était similaire à l'émissivité de l'atmosphère}
\item on rappelle que selon la relation \emph{TOA}, nous avons $OLR = (1-A\e{b}) \, \mathcal{F}\e{s}'$
\end{enumerate}


\sk
Il est alors possible d'obtenir deux expressions différentes pour~$\Sigma(\tau)$.
Premièrement, en utilisant $[E_\Sigma]$ avec $\mathcal{C}_1$ et $\mathcal{C}_2$, on obtient~$\Sigma(\tau)=OLR \, (1+\tau)$.
Deuxièmement, en utilisant $[E_\Delta]$ avec $\mathcal{C}_1$, on obtient~$\Sigma(\tau)=2\,\epsilon\,\sigma\,T(\tau)^4$.
\textbf{Conclusion 1} : on obtient le \voc{profil radiatif}, 
c'est-à-dire le profil vertical de température\footnote{Suivant la géométrie
équivalente choisie pour le modèle plan-parallèle, le terme $1+\tau$
peut s'écrire un peu différemment, mais quoiqu'il en soit toujours sous une
forme~$a+b\,\tau$ avec $a,b$ constants. Les conclusions énoncées ici ne sont pas
modifiées.} imposé par les transferts radiatifs dans l'infrarouge
\[
T(\tau) = \sqrt[4]{\frac{OLR\,(1+\tau)}{2\,\sigma\,\epsilon}}
\]

\sk
Reste à calculer la température de surface avec ce modèle. D'après $\mathcal{C}_3$, le bilan
au sol s'écrit~$(1-A\e{b}) \, \mathcal{F}\e{s}' + F^-(\tau=\tau_{\infty}) = \epsilon\,\sigma\,T\e{s}^4$.
Il faut donc exprimer les flux ascendant et descendant dans l'infrarouge.
Du fait que $\mathcal{C}_1$ et $\mathcal{C}_2$ nous indiquent que~$\Delta=OLR$, on obtient aisément
\[
F^+(\tau) = \frac{\Sigma+\Delta}{2} = OLR \, (1+\frac{\tau}{2})
\qquad \qquad
F^-(\tau) = \frac{\Sigma-\Delta}{2} = OLR \, \frac{\tau}{2}
\]
\noindent On obtient alors l'expression liant $OLR$
et température de surface~$T\e{s}$
\[
\boxed{\epsilon\,\sigma\,T\e{s}^4 = OLR \, \left( 1 + \frac{\tau_{\infty}}{2} \right)}
\]
\textbf{Conclusion 2} : on obtient une définition quantitative de \voc{l'effet de serre}
\begin{citemize}
\item Dans l'infrarouge, le rayonnement sortant au sommet de l'atmosphère ($OLR$)
est inférieur au rayonnement émis par la surface ($\epsilon\,\sigma\,T\e{s}^4$).
Une partie du rayonnement émis par la surface reste donc piégée par la planète.
\item Avec un albédo et un rayonnement incident constant, donc à~$OLR$ constant (d'après $\mathcal{C}_4$),
augmenter la quantité de gaz à effet de serre (donc augmenter~$\tau_{\infty}$)
conduit à une augmentation de la température de surface~$T\e{s}$.
\end{citemize}

\sk
Il est alors instructif de s'intéresser à la température atmosphérique 
proche de la surface~$T(\tau_\infty)$
donnée par le profil radiatif. Cette température ne dépend que de~$OLR$
et s'obtient totalement indépendamment de la température de surface.
On peut alors montrer que
\[
T\e{s} = T(\tau_\infty) \, \sqrt[4]{\frac{2+\tau_\infty}{1+\tau_\infty}} > T(\tau_\infty)
\]
\noindent \textbf{Conclusion 3} : tant que l'atmosphère n'est pas
optiquement épaisse dans l'infrarouge (donc tant que~$\tau_\infty$ reste fini),
il existe une discontinuité entre la surface et l'atmosphère, la surface
étant toujours plus chaude que l'atmosphère. Cela implique que l'atmosphère
est instable proche de la surface, donc que du mélange turbulent / convectif
apparaît, donc que l'équilibre proche de la surface ne peut être simplement
radiatif mais \voc{radiatif-convectif}. Notons que dans le cas où l'atmosphère est optiquement épaisse,
$T\e{s} = T(\tau_\infty)$, ce qui est tout à fait représentatif des conditions sur Vénus.


%%% figure Salby
