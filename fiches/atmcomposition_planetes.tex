\figun{1}{0.35}{decouverte/cours_dyn/composition.png}{Objets du système solaire présentant une atmosphère substantielle et leurs caractéristiques.}{fig:composition}

\sk
Les compositions atmosphériques sont très distinctes selon les planètes du système solaire considérées [figure~\ref{fig:composition}]. L'atmosphère terrestre est unique parmi les atmosphères des autres planètes du système solaire. Elle est riche en azote et oxygène et pauvre en \carb, contrairement aux atmosphères de Venus et Mars, qui contiennent plus de~$90\%$ de \carb. On pourrait penser que la Terre a acquis son atmosphère pendant sa formation à partir des gaz présents dans la nébuleuse solaire. Une telle atmosphère serait alors primaire, et contiendrait des gaz de composition cosmique, c'est-à-dire similaire aux abondances chimiques du système solaire. Or, les gaz dominants dans le système solaire sont l'hydrogène et l'hélium. Ces gaz légers sont pratiquement absents dans notre atmosphère, car la gravitation terrestre est trop faible pour les retenir. Les planètes géantes comme Jupiter ou Saturne ont conservé ces gaz primordiaux dans leur atmosphère, au contraire des planètes internes du système solaire Venus, Terre et Mars qui ont des atmosphères de composition bien différente. Si la Terre a eu une telle atmosphère primaire pendant sa formation, elle l'a perdu rapidement. L'atmosphère actuelle doit donc être secondaire. L'évolution de sa composition résulte en partie de l'apparition de la vie [figure~\ref{fig:vie}].
%% parler des puits de carbone et des carbonates. comparaison entre Vénus et Mars.

\figun{0.75}{0.32}{\figpayan/vie.png}{Evolution parallèle de l’atmosphère et de la vie.}{fig:vie}
