\sk
L'atmosphère est composée d'un ensemble de molécules. Pour la description de la plupart des phénomènes étudiés, le suivi des comportements individuels de chacunes des molécules composant l'atmosphère est impossible. On s'intéresse donc aux effets de comportement d'ensemble, ou moyen. Les principales variables thermodynamiques utilisées pour décrire l'atmosphère sont donc des grandeurs \voc{intensives} dont la valeur ne dépend pas du volume d'air considéré.
\begin{finger}
\item La \voc{température} $T$ est exprimée en K (kelvin) dans le système international. Elle est un paramètre macroscopique qui représente l'agitation thermique des molécules microscopiques. Les mesures de température usuelles font parfois référence à des quantités en \deg C, auxquelles il faut ajouter la valeur $273.15$ pour convertir en kelvins.
\item La \voc{pression} $P$ est exprimée en Pa dans le système international. La pression fait référence à une force par unité de surface ($1$~Pa correspond à l'unité~N~m$^{-2}$). Paramètre macroscopique, elle est reliée à la quantité de mouvement des molécules microscopiques qui subissent des chocs sur une surface donnée. Les mesures et raisonnements météorologiques font souvent référence à des quantités en hPa ou en mbar. Ces deux unités sont équivalentes : 1~hPa correspond à~$10^{2}$~Pa, 1~mbar correspond à $10^{-3}$~bar, ce qui correspond bien à 1~hPa, puisque 1 bar est $10^{5}$~Pa. La pression atmosphérique vaut $1013.25$~hPa (ou mbar) en moyenne au niveau de la mer sur Terre. On utilise parfois l'unité d'$1$~atm (atmosphère) qui correspond à cette valeur de~$101325$~Pa.
\item La \voc{masse volumique} ou densité~$\rho$ est exprimée en~kg~m$^{-3}$ dans le système international. Elle représente une quantité de matière par unité de volume. Elle vaut environ $1.217$~kg~m$^{-3}$ proche de la surface sur Terre.
\end{finger}
