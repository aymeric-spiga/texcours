\sk
\paragraph{Principe} La dépendance de la force gravitationnelle en $r^{-2}$ explique qu'un corps attracteur puisse exercer une force gravitationnelle différente en divers points d'un objet. Typiquement la partie de l'objet la plus proche du corps attracteur est plus attirée par le corps attracteur que la partie lointaine. Ce phénomène est appelé \voc{marée}. 

\sk
\paragraph{Effet} L'effet de la force de marée sur un objet peut aller de la déformation simple (renflement dans la direction du corps attracteur), à la flexure produisant une chaleur interne, jusqu'à la destruction pure et simple de l'objet. Cette force est également responsable de certaines caractéristiques orbitales. L'action des marées explique que la Lune présente toujours la même face à la Terre, que la comète Shoemaker-Levy s'est détruite à l'approche de Jupiter, que Io soit le théâtre une activité volcanique soutenue, que les planètes extrasolaires proches de leur étoile voient leur orbite perturbée.  

\sk
\paragraph{Notations} Dans ce qui suit, on note~$r$ et~$m$ le rayon et la masse du corps cible, $R$ et~$M$ le rayon et la masse du corps attracteur, et~$a$ la distance entre corps cible et corps attracteur.












