\sk
Comme le montre la figure~\ref{fig:press} (haut), la pression~$P$ à la surface de la Terre est au premier ordre sensible à l'altimétrie (hauteur topographique), puisque la pression correspond au poids de la colonne d'air située au-dessus du point considéré. Pour produire des cartes de prévision du temps, on souhaite éliminer du champ de pression~$P$ cette composante topographique de premier ordre et mettre en évidence les variations de second ordre digne d'intérêt en météorologie.

\figsup{0.62}{0.32}{decouverte/cours_meteo/SURFPRESS/outputvar134_200.png}{decouverte/cours_meteo/SURFPRESS/outputvar151_200.png}{Champs de pression prédits au 01/09/2009 par les réanalyses ERA-INTERIM de l'ECMWF (Centre Européen de Prévision du Temps à Moyen Terme). La réprésentation graphique est basée sur une projection stéréographique polaire centrée sur le pôle Nord et les structures topographiques sont ajoutées dans le fond de la figure pour repérage. En haut, le champ de pression brut est tracé en hPa~: les valeurs de pression les plus basses correspondent aux reliefs topographiques les plus élevés. En bas, le champ de pression ramené au niveau de la mer est tracé en hPa~: la composante de premier ordre topographique sur le champ de pression a disparu pour laisser place aux composantes météorologiques de la pression~: dépressions (zones de basses pression) en bleu et anticyclones (zones de haute pression) en rouge. On peut d'ailleurs noter dans ce champ de pression normalisé l'activité ondulatoire de l'atmosphère aux moyennes latitudes. Les cartes de pression des bulletins météorologiques sont exclusivement des cartes de pression ramenées au niveau de la mer comme celle-ci.}{fig:press} 

\sk
Quand la pression de surface est mesurée à une station située à une altitude~$z \ll H$, on peut utiliser l'équation hypsométrique avec la température mesurée à la station pour déterminer une valeur approximative de la pression au niveau de la mer à~$z=0$. On suppose fréquemment que la température décroît linéairement avec l'altitude~$z$ selon un taux constant négatif~$\Gamma\e{e}$ en~$^{\circ}$C/km (ou K/km). On appelle d'ailleurs la loi~$T = T_0 + \Gamma\e{e} \, z$ le profil de l'atmosphère standard. En intégrant entre le niveau de la mer ($z=0, P=P_0$) et la station à ($z,P$), on obtient:
\[ \ln \left( \frac{P_0}{P} \right) = \frac{g}{R \, \Gamma\e{e}} \, \ln \left( \frac{T_0 + \Gamma\e{e} \, z}{T_0} \right) \]
\[ \Rightarrow \qquad P_0 = P \left( 1 + \frac{\Gamma\e{e} \, z}{T_0} \right)^{\frac{g}{R\,\Gamma\e{e}}} \]
La carte météorologique sur la Figure~\ref{fig:press} bas est obtenue en employant cette équation. L'équation qui précède est aussi utilisée avec $P_0=1013.25$~hPa par les altimètres des avions de ligne pour convertir~$P$ mesurée en~$z$.

