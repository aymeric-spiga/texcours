\sk
L'objectif des sciences de l'atmosphère est d'étudier la structure et l'évolution de l'atmosphère en caractérisant et en expliquant les phénomènes qui s'y déroulent. Les sciences de l'atmosphère font principalement appel à des notions de physique, chimie, et mécanique des fluides.
\begin{description}
\item[\voc{Atmosphère}] Ensemble de couches, principalement gazeuses, qui entourent la masse condensée, solide ou liquide, d'une planète (voir également citation de Laplace en en-tête).
\item[\voc{Air}] Mélange gazeux constituant l'atmosphère terrestre.
\item[\voc{Aéronomie}] Science dont l'objet est la connaissance de l'état physique de l'atmosphère terrestre et des lois qui la gouvernent.
\item[\voc{Météorologie}] Discipline ayant pour objet l'étude des phénomènes atmosphériques et de leurs variations, et qui a pour objectif de prévoir à court terme les variations du temps.
\item[\voc{Climat}] Ensemble des conditions atmosphériques et météorologiques d'un pays, d'une région. Le climat peut également être défini comme un système thermo-hydrodynamique non isolé dont les composantes sont les principales « enveloppes » externes de la Terre : on parle également de~\voc{système climatique} [figure~\ref{fig:pluri}]. 
\begin{citemize} \item L'atmosphère : l’air, les nuages, les aérosols, \ldots 
\item L’hydrosphère : les océans, les rivières, les précipitations, \ldots 
\item La lithosphère : les terres immergées, les sols, \ldots 
\item La cryosphère : glace, neige, banquise, glaciers, \ldots 
\item La biosphère : les organismes vivants, \ldots
\item L’anthroposphère : l’activité humaine, \ldots 
\end{citemize} 
\end{description}

\figside{0.75}{0.35}{decouverte/cours_meteo/joussaume_pluri.png}{Schéma du système climatique présentant les différentes composantes du système : atmosphère, océans, cryosphère, biosphère et lithosphère, leurs constantes de temps et leurs interactions en termes d’échanges d’énergie, d’eau et de carbone. Source~:~S.~Joussaume \emph{in} Le Climat à Découvert, CNRS éditions, 2011}{fig:pluri}
