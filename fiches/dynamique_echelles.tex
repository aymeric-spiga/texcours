\sk
Tous les termes de l'équation du mouvement n'ont pas la même importance lorsqu'on considère des mouvements atmosphériques de grande échelle. On définit donc des échelles caractéristiques du mouvement étudié. Pour simplifier, on choisit des échelles qui sont des puissances de 10.
\begin{description}
\item[longueur] Les échelles de longueur sont $L$ sur l'horizontale, et $H$ sur la verticale. Pour des mouvements qui s'étendent sur la hauteur de la troposphère, $H\sim 10$~km. $L$ peut varier beaucoup, mais l'échelle dite synoptique $L=$1000~km, qui est celle des perturbations des latitudes moyennes, est d'un intérêt particulier. La dernière échelle de longueur est celle du rayon de la Terre~$a$, qui est de l'ordre de 10000~km. 
\item[vitesse] Les échelles de vitesse horizontale et verticale sont notées $U$ et $W$. On a typiquement $U$=10~m~s$^{-1}$ dans l'atmosphère. Le rapport d'aspect du mouvement impose d'autre part que $W\le UH/L$.
\item[temps] L'échelle de durée du mouvement est construite à partir de celles de vitesse et de longueur: $T=L/U$. L'autre échelle de temps est celle liée à la rotation de la Terre, qui apparait dans le terme de Coriolis.
\item[variables thermodynamiques] Les variations des variables thermodynamiques $P,T,\rho$ sur la verticale sont celles des profils moyens donnés en introduction. En un point donné, les variations à l'échelle synoptique $\delta P,\delta T,\delta\rho$ sont de l'ordre de 1\% de la valeur moyenne.
\end{description}


