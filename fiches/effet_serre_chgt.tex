


%\figside{0.45}{0.3}{decouverte/pierrehumbert_pics/9780521865562c03_fig005.jpg}{R. Pierrehumbert, Principles of Planetary Climates, CUP, 2010}{fig:effetserre1}

\sk
Nous présentons ici, dans un cadre d'équilibre radiatif-convectif, l'explication la plus simple (sans être simpliste) du déplacement
d'équilibre radiatif qu'induit l'augmentation de gaz à effet de serre.
%Le modèle à deux faisceaux ne nous aide pas énormément. diverge quand tau tend vers infini.

\sk
Il est possible de montrer par des calculs de transfert radiatif que le niveau
d'émission équivalent au sommet de l'atmosphère est tel que~$\tau = 1$.
Qualitativement, on comprend que les niveaux inférieurs sont optiquement
épais donc ne sont que marginalement ``vus'' depuis l'espace dans les longueurs d'onde infrarouges.
Ainsi dans l'équilibre TOA
\[ \TOA \] 
\noindent l'émission de rayonnement au niveau~$\tau=1$ 
à la température~$T(\tau=1)$ domine OLR.

\sk
Appelons~$P\e{rad}$ la pression du niveau~$\tau=1$. 
Pour relier
les deux quantités, on emploie la définition de l'épaisseur optique
$\EO$ que l'on combine
à l'équilibre hydrostatique pour obtenir
par intégration~$\tau = \kappa \frac{P}{g} \, q_X$,
avec $q_X$ le rapport de mélange massique 
de l'espèce~$X$ absorbante dans l'infrarouge.
Ainsi
\[ \boxed{ P\e{rad} \simeq \frac{g}{\kappa \, q_X} } \]

\figun{0.7}{0.25}{decouverte/pierrehumbert_pics/9780521865562c03_fig006.jpg}{Déplacement d'équilibre correspondant à une augmentation de la concentration en gaz à effet de serre. Il convient de noter qu'entre les deux états schématisés, il existe un état transitoire correspondant à celui de gauche avec un flux émis~$\sigma T'^4 < (1-A\e{b}) \, \mathcal{F}\e{s}'$ à $P\e{rad,2}$, avant que l'équilibre TOA ne s'établisse à nouveau pour donner l'état de droite. Figure tirée de R. Pierrehumbert, Principles of Planetary Climates, CUP, 2010}{fig:effetserre2}

\sk
L'expression ci-dessus implique qu'une augmentation de
gaz à effet de serre ($q_X$ augmente) implique une 
élévation du niveau équivalent d'émission
($p\e{rad}$ diminue).
L'effet sur la température de surface se détermine alors
en écrivant la conservation de la température potentielle
dans la troposphère soumise à l'équilibre radiatif-convectif,
entre la surface et le niveau équivalent d'émission
\[ T_s = T\e{rad} \, \left( \frac{P\e{rad}}{P\e{s}} \right)^{-\frac{R}{c_p}} \]
\noindent où~$P_s$ est la pression de surface.
Une élévation du niveau équivalent d'émission
se traduit donc par une augmentation
de température de surface (fournissant
un modèle à la fois simple et fidèle du 
changement climatique récent sur Terre, Figure~\ref{fig:effetserre2}). 
Approximativement, $OLR \sim \sigma \, T(P\e{rad})^4$
et~$P\e{rad}$ est alors défini par la 
condition TOA qui s'écrit~$OLR = (1-A\e{b}) \, \mathcal{F}\e{s}'$.
Le lien entre quantité de gaz à effet de serre~$q_X$
et température de surface~$T\e{s}$ peut ainsi s'écrire
\[ T\e{s} = \sqrt[4]{\frac{(1-A\e{b}) \, \mathcal{F}\e{s}'}{\sigma}} \, \left( \frac{\kappa \, q_X \,P\e{s}}{g} \right)^{\frac{R}{c_p}} \]
