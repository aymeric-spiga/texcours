\bk
Comment déterminer les processus dynamiques, physiques, chimiques à l'oeuvre dans l'atmosphère ? Il faut commencer par faire le point sur les sources d'énergie pour l'atmosphère, les océans et la surface. La principale source d'énergie pour l'atmosphère et le système climatique de la Terre est le Soleil\footnote{Ce n'est pas le cas pour les géantes gazeuses Jupiter et Saturne où il existe un flux de chaleur interne significatif en regard du flux d'énergie reçu du Soleil. Ce flux est un reste de la contraction gravitationnelle au cours de la formation de ces géantes gazeuses.}. La figure~\ref{fig:flux} montre que d’autres sources existent mais en quantité réduite : l'énergie reçue par la géothermie, ou par les activités humaines, est~$4$ ordres de grandeur plus faible que la source solaire; celle reçue des étoiles~$8$ ordres de grandeur plus faible. L’énergie solaire est transmise principalement à la Terre au moyen du rayonnement électromagnétique~: on qualifie cette énergie de \voc{radiative}. %L'objet de ce chapitre est de s'intéresser au rayonnement électromagnétique et plus particulièrement au phénomène d'émission thermique.

\figside{0.75}{0.18}{decouverte/cours_meteo/fluxenergsurf.png}{Ordres de grandeur des flux énergétiques reçus à la surface de la Terre. Source~:~P.~von Balmoos \emph{in} Le Climat à Découvert, CNRS éditions, 2011}{fig:flux}

