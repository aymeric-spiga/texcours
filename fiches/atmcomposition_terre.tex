\begin{table}
\centering
\begin{tabular}{lcc}
%\toprule
%\makecell[l]{\textbf{Constituant}} &
%\makecell{\textbf{Masse}\\\textbf{Molaire}} &
%\makecell{\textbf{Rapport}\\\textbf{de Mélange}} \\
%\midrule
\hline
\textbf{Constituant} &
\textbf{Masse molaire} &
\textbf{Rapport de mélange} \\
\hline
Azote (N$_2$) & 28 & 78\% \\
Oxygène (O$_2$) & 32 & 21\% \\
Argon (Ar) & 40 & 0.93\% \\
\textbf{Vapeur d'eau (H$_2$O)} & 18 & 0-5\% \\
\textbf{Dioxyde de Carbone (CO$_2$)} & 44 & 380 ppmv \\
Néon (Ne) & 20 & 18 ppmv \\
Hélium (He) & 4 & 5 ppmv \\
\textbf{Méthane (CH$_4$)} & 16 & 1.75 ppmv \\
Krypton (Kr) & 84 & 1 ppmv \\
Hydrogène (H$_2$) & 2 & 0.5 ppmv \\
\textbf{Oxide nitreux (N$_2$O)} & 56 & 0.3 ppmv \\
\textbf{Ozone (O$_3$)} & 48 & 0-0.1 ppmv \\
%\bottomrule
\hline
\end{tabular}
\caption{\emph{Principaux composants de l'atmosphère. Les gaz à effet de serre sont indiqués en gras.}}
\label{tab:compos}
\end{table}


\sk
L'azote et l'oxygène dominent largement la composition de l'atmosphère terrestre (tableau \ref{tab:compos}), suivis par l'argon et d'autres gaz rares beaucoup moins abondants. Les rapports de mélange de vapeur d'eau et d'ozone sont très variables : la vapeur d'eau est présente surtout dans la troposphère, avec un maximum près de la surface et dans les tropiques, alors que l'ozone est principalement présente dans la stratosphère. Un certain nombre de gaz traces sont émis régulièrement au niveau de la surface, par des phénomènes naturels ou les activités humaines. Leur répartition dépend alors beaucoup de leur \voc{durée de vie} dans l'atmosphère. Le CO$_2$ qui est très stable est bien mélangé. Le méthane, qui a une durée de vie d'une dizaine d'années, est bien réparti dans la troposphère mais son rapport de mélange varie dans la stratosphère. Des polluants à durée de vie courte (quelques jours) comme l'ozone troposphérique, se retrouveront surtout au voisinage des sources. Les activités humaines ont également contribué à modifier le rapport de mélange de certains de ces gaz (par exemple, le~CO$_2$).

\sk
\begin{finger}
\item En employant les formules obtenues à la section précédente, et les informations ci-dessus sur la composition de l'atmosphère terrestre, il est possible de déterminer des valeurs numériques utiles
\begin{citemize}
\item La masse molaire de l'air est~$M = 28.966$~g~mol$^{-1}$ (on emploie souvent~$M \simeq 29$~g~mol$^{-1}$). 
\item La constante de l'air sec est~$R = 287$~J~K$^{-1}$~kg$^{-1}$.
\end{citemize}  
\end{finger}
