\sk
L'accélération de Coriolis peut être interprétée comme une force apparente massique $\v F_C = - 2 \, \v \Omega\wedge\v V_r$. Cette force apparente étant orthogonale à la vitesse à cause de la présence du produit vectoriel, sa puissance est nulle~: la \voc{force de Coriolis} va dévier le mouvement relatif mais ne peut pas modifier la vitesse du vent ou de courants. Pour des mouvements relatifs horizontaux à la vitesse \v V, le module de la force apparente de Coriolis est~$2 \, \Omega \, \sin \phi \, V$ qui change de signe lorsqu'on change d'hémisphère en fonction de~$\sin \phi$. Dans l'hémisphère nord, où $\sin \phi>0$, la force de Coriolis est dirigée à $90^{\circ}$ à droite du vent. 

\sk
Afin de bien comprendre l'effet de la force de Coriolis, il est profitable sur une planète comme la Terre d'utiliser la conservation du moment cinétique\footnote{
Puisque le moment cinétique~$\sigma$ se conserve on a \[ \ddf{\sigma}{t} = 0 = \ddf{r}{t} \, (\Omega \, r + u) + r \, \left( \Omega \ddf{r}{t}+\ddf{u}{t} \right) \qquad \Rightarrow \qquad \ddf{u}{t} = - \ddf{r}{t}  \, \left( 2\,\Omega + \frac{u}{r} \right) \] 
Le terme en $u/r$ est dû à la courbure de la surface, mais seule la vitesse relative intervient, pas la rotation de la Terre. En pratique, ce terme est négligeable sur Terre devant~$2 \, \Omega$. L'équation ci-dessus montre donc que raisonner avec la conservation du moment cinétique permet de comprendre l'effet sur les vents de la force de Coriolis.
}
(l'équivalent pour les systèmes en rotation de la quantité de mouvement pour les systèmes en translation). En effet, la somme des forces étant dirigée vers H, M conserve son \voc{moment cinétique}~$\sigma$ par rapport à l'axe des pôles, qui s'exprime
\[ \boxed{ \sigma = u_a \, r = (\Omega \, r + u) \, r } \]
où~$r$ est la distance entre le point considéré et l'axe de rotation qui passe par les deux pôles.
%\footnote{La conservation de $\sigma$ implique des variations de l'énergie cinétique $(\Omega r+u)^2$. C'est le travail de \v G (pour un mouvement sud-nord) qui en est l'origine.}. 

\sk
Pour illustrer les effets de cette force apparente de Coriolis, on considère une parcelle initialement au repos dans le référentiel tournant (c'est à dire~$u=0$ et~$v=0$ à~$t=0$) qui se déplacerait vers le Nord suivant l'axe~$\v j$. Elle se rapproche donc de l'axe des pôles et va voir sa vitesse absolue augmenter par conservation du moment cinétique: $\sigma$ est constant et~$r$ diminue, donc $u_a$ augmente. Dans le même temps, la vitesse d'entrainement locale~$u_e=\Omega \, r$ diminue sous l'effet de la diminution de la distance~$r$ à l'axe des pôles. La parcelle va donc acquérir une vitesse relative $u>0$ vers l'est\footnote{
En fait, l'expression ci-dessus permet même de calculer la variation de vitesse associée. Pour un mouvement sud-nord, la vitesse est $v=a \, \ddf{\phi}{t}$. D'autre part $r=a \, \cos \phi$ donc~$\ddf{r}{t}=-a \, \ddf{\phi}{t} \, \sin \phi = - v \, \sin \phi$. L'équation de conservation du moment cinétique devient 
\[ \ddf{u}{t} = v \, \sin \phi \, \left( 2 \, \Omega + \frac{u}{r} \right) \simeq 2 \, \Omega \, v \, \sin \phi \] 
La parcelle est bien déviée vers l'est pour un déplacement vers le nord tel que~$v>0$.
}
comme indiqué sur le schéma \ref{fig:coriolisns}. 

\figside{0.3}{0.2}{\figfrancis/coriolis_ns}{Déviation d'une parcelle se déplaçant vers le nord. Instant initial: vitesses d'entrainement $u_e$ et absolue $u_a$ égales. Instant final: vitesse d'entrainement $u_e'$ et absolue $u_a'$ augmentée par conservation du moment cinétique $\sigma$.}{fig:coriolisns}

%\subsubsection{Force de Coriolis: mouvement vers l'est} On considère un point M en mouvement par rapport à la surface de la Terre. On rappelle que pour un mouvement circulaire, on doit avoir une accélération normale égale à $V^2/R$ dirigée vers le centre du cercle. On suppose que les forces réelles s'exerçant sur M sont les mêmes que pour un point fixe: $\Sigma \vec F=\v a_e$. La composante de la vitesse relative vers l'est (suivant \v i) est $u$, et $\dot{r}$ dans la direction \vl{HM}. La vitesse absolue de M vers l'est est $u_a=\Omega r+u$. La relation $\v a=\Sigma\v F$ s'écrit dans la direction $\v e_r$: \[-\frac{(\Omega r+u)^2}{r}+\ddot{r}=a_e=-\Omega^2r\] soit en développant: \[\ddot{r}=u\cdot(2\Omega+\frac{u}{r})\] Pour un mouvement relatif vers l'est ($u>0$), la vitesse absolue est supérieure à la vitesse d'entrainement, et la somme des forces est insuffisante pour compenser $V_a^2/r$. La parcelle va donc s'éloigner de l'axe de rotation (figure \ref{fig:coriolisew}). Elle va au contraire se rapprocher pour $u<0$ (mouvement vers l'ouest). Pour trouver l'accélération relative dans la direction sud-nord, on projette $\v e_r$ sur \v j: $\dot{v}=-\ddot{r}\sin \phi$. \[\dot{v}=-u\sin \phi\cdot(2\Omega+\frac{u}{r})\] M est donc dévié vers le sud pour un déplacement relatif vers l'est.
%\begin{figure}[tbp] \begin{center} \includegraphics[width=12cm]{\figfrancis/coriolis_ew} \end{center} \caption{Déviation d'une parcelle ayant une vitesse relative initiale non nulle vers l'est (gauche) et l'ouest (droite). Un plan parallèle à l'équateur est représenté, vu depuis le pôle nord, l'axe de rotation est au centre. Les vitesse et accélération d'entrainement (égale à la somme des forces) sont en noir, la vitesse absolue en rouge. La trajectoire future de la parcelle est en pointillés.} \label{fig:coriolisew} \end{figure}
