\sk
On appelle \voc{gaz parfait} un gaz suffisament dilué pour que les interactions entre les molécules du gaz, autres que les chocs, soient négligeables. L'air composant l'atmosphère peut être considéré en bonne approximation comme un mélange de gaz parfaits\footnote{On peut en général considérer que le gaz est parfait si~$P < 1$~kbar. C'est le cas dans la plupart des atmosphères planétaires rencontrées. Il n'y a guère qu'au coeur des planètes géantes, où la pression dépasse cette limite, que l'approximation du gaz parfait doit être complètement abandonnée. Le domaine de validité de l'approximation du gaz parfait dépend tout de même fortement de la chimie du gaz considéré ; par exemple, dans l'atmosphère profonde de Vénus avec du CO$_2$ à 90 bars, des corrections de Van der Waals sont nécessaires.} notés~$i$, dont le nombre de moles est~$n_i$ pour un volume donné~$V$ d'air à la température~$T$. Chaque espèce gazeuse composant l'air est caractérisée par une \voc{pression partielle}~$P_i$ qui est définie comme la pression qu'aurait l'espèce gazeuse si elle occupait à elle seule le volume~$V$ à la température~$T$. Chacune de ces espèces gazeuses~$i$ se caractérise par une même température~$T$ et vérifie l'équation d'état du gaz parfait $$ P_i \, V = n_i \, R^* \, T $$ où~$R^*$=8.31 J~K\mo~mol\mo~est la constante des gaz parfaits (produit du nombre d'Avogadro et de la constante de Boltzmann). La pression totale de l'air~$P$ est, d'après la loi de Dalton, la somme des pressions partielles~$P_i$ des espèces gazeuses composant le mélange $P=\Sigma P_i$. En faisant la somme des lois du gaz parfait appliquées pour chacune des espèces gazeuses, on obtient $$ P \, V = \big( \Sigma n_i \big) \, R^* \, T $$ ce qui montre qu'un mélange de gaz parfaits est aussi un gaz parfait. Cette équation permet de relier la pression totale~$P$ à la température~$T$, mais présente l'inconvénient de contenir les grandeurs \voc{extensives}~$V$ et $n_i$ qui dépendent du volume d'air considéré. Il reste donc à donner une traduction intensive à la loi du gaz parfait pour un mélange de gaz. La masse totale contenue dans le volume~$V$ peut s'écrire $m=\Sigma n_iM_i$ où $M_i$ est la masse molaire du gaz $i$. En divisant l'équation précédente par $m$, et en utilisant la définition de la masse volumique~$\rho = m / V$, on obtient $$ \frac{P}{\rho} = \frac{\Sigma n_i}{\Sigma n_iM_i} \, R^* \, T $$ Or, d'une part, la \voc{masse molaire de l'air} composé d'un mélange de gaz~$i$ est $$ \boxed{ M=\frac{\Sigma n_iM_i}{\Sigma n_i} }$$ et d'autre part, on peut définir une \voc{constante de l'air sec} de la façon suivante $$ R=\frac{R^*}{M} $$ On a alors l'équation des gaz parfaits pour l'air atmosphérique qui permet de relier les trois paramètres intensifs importants : pression~$P$, température~$T$ et densité~$\rho$ $$ \boxed{\GP} $$ L'état thermodynamique d'un élément d'air est donc déterminé uniquement par deux paramètres sur les trois~$P$, $T$, $\rho$. En météorologie par exemple, on travaille principalement avec la pression et la température qui sont plus aisées à mesurer que la densité. Les valeurs numériques de~$M$, et donc~$R$, dépendent de la planète considérée et de sa composition atmosphérique. 

