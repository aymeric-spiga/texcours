\sk
L'ordre de grandeur des termes de l'équation du mouvement 
%\ref{eq:qtemvt} 
projetée sur la verticale (dirigée suivant \v k) est indiqué dans la table \ref{tab:vqmouv}. On voit que l'équilibre hydrostatique est vérifié avec une très bonne approximation\footnote{On peut noter qu'on vérifie également l'équilibre hydrostatique entre des anomalies de densité et des anomalies de variations de pression sur la verticale. Les termes $\rho g$ et $\partial P/\partial z$ sont alors cent fois plus faibles que pour l'état moyen, mais toujours supérieurs aux autres termes de l'équation.}. Notamment la composante verticale de la force de Coriolis~$\v F_C$ est négligeable devant~\v g et les forces de pression. Le seul autre terme qui peut devenir important est l'accélération relative~$dw/dt$, lors de mouvements verticaux intenses à petite échelle, comme dans un nuage d'orage ou près de topographie raide.  
%\begin{equation}
%  \frac{\partial P}{\partial z}=-\rho  g
%  \label{eq:hydro}
%\end{equation}

\begin{table}
  \centering
  \begin{tabular}{ccccccc}
    \hline
    Équation & $dw/dt$ & $-2\Omega u\cos\phi$ & $-\left(u^2+v^2\right)/a$ & = &
    $-\rho^{-1}\partial P/\partial z$ & $-g$ \\
    Échelle & $UW/L$ & $fU$ & $U^2/a$ && $P_0/(\rho_0H)$ & $g$ \\
    m.s\md & 10$^{-7}$ & 10$^{-3}$ & 10$^{-5}$  && 10 & 10 \\ 
    \hline
  \end{tabular}
  \caption{\emph{Analyse d'échelle de l'équation du mouvement vertical (avec
  $L$=1000~km et $W$=1~cm.s\mo).}}
  \label{tab:vqmouv}
\end{table}


