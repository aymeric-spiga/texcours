\sk
Les trois paramètres pression, température, densité varient en théorie selon les trois directions de l'espace. On constate cependant que, pour une unité de longueur donnée, leurs variations selon la verticale sont beaucoup plus significatives que leurs variations selon l'horizontale. On peut donc définir une structure moyenne en fonction de l'altitude dont sera toujours relativement proche la structure verticale en chaque jour et chaque région de la planète. 

\figsup{0.48}{0.35}{\figfrancis/WH_vert_struct}{\figpayan/LP211_Chap1_Page_03_Image_0001.png}{[Gauche] Structure verticale de la pression, la densité et du libre parcours moyen des molécules (distance moyenne parcourue par une molécule avant de subir un choc sur une autre molécule). Noter l'échelle logarithmique en abscisse : une droite sur ce schéma dénote donc une variation exponentielle des quantités avec l'altitude. [Droite] Plus haut dans l'atmosphère, la variation verticale de la pression est dépendante pour plusieurs ordres de grandeur avec l'activité solaire. Les courbes indiquées correspondent respectivement à une très faible activité solaire (température de la thermopause de 600 K) et une très forte activité solaire (température de la thermopause de 2000K).}{fig:presvert}

\sk
Pression et densité décroissent de façon approximativement exponentielle selon l'altitude~$z$ [figure \ref{fig:presvert}] $$ P\sim P_0 \, e^{-z/H} $$ où $H$ est appelée \voc{hauteur d'échelle} et~$P_0$ une valeur de pression de référence. Cette loi de variation découle du fait que la pression atmosphérique à une altitude~$z$ est due au poids de la colonne d'air située au-dessus de l'altitude~$z$. En pratique sur Terre, la pression est divisée par deux environ tous les 5 km. On évalue la masse de l'atmosphère terrestre à~$5 \times 10^{18}$~kg, soit environ un millionième de la masse de la Terre. La moitié de la masse de l'atmosphère se situe au dessous de~$5500$~m, les~$2/3$ au dessous de~$8400$~m, les~$3/4$ au dessous de~$10300$~m, les~9/10~au dessous de~$16100$~m. Si l'on considère que les neuf dixièmes de l’atmosphère sont situés dans les $16$~premiers kilomètres, l’atmosphère ne forme donc qu'une mince pellicule gazeuse en comparaison des $6367$~km du rayon terrestre. On dit que l'on peut faire l'\voc{approximation de la couche mince}.
