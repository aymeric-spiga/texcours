
%% peut-être détailler : Etot = Egrav + Ecin L0= m*v*a Et on a Egrav avec Fgrav=-grad(E)

\sk
\paragraph{Modification d'orbites (quantitatif)} Pour un objet de masse~$m$ suivant une orbite elliptique d'excentricité~$e$ sous l'effet de l'attraction d'un corps de masse~$M$, l'énergie totale~$E$ et le moment angulaire~$L_0$ sont conservés et s'écrivent\footnote{L'énergie totale~$E$ correspond à l'énergie cinétique
plus l'énergie potentielle de gravitation.
Le moment angulaire~$L_0$ se conserve car la force est centrale; 
il s'obtient simplement à l'apoastre (point 2) et périastre (point 1)
$L_0 = m v_1 r_1 = m v_2 r_2$, puis en utilisant
l'équation de l'énergie~$\left( v_1^2 - v_2^2 \right) = 2 \mathcal{G} M \left( \frac{1}{r_1} - \frac{1}{r_2} \right)$
on en déduit
\[ \frac{L_0^2}{2 m^2} = \frac{\mathcal{G}M}{(\frac{1}{r_1}-\frac{1}{r_2})} \]
puis le résultat~$L_0 = m \, \sqrt{\mathcal{G}\,M\,a\,(1-e^2)}$
étant donné que~$r_1 = a (1-e)$ et~$r_2 = a (1+e)$.}
\[  E = - \frac{\mathcal{G}\,m\,M}{2\,a} \qquad L_0 = m \, \sqrt{\mathcal{G}\,M\,a\,(1-e^2)} \] 
En éliminant~$m$ dans les deux équations qui précèdent, on obtient une relation de proportionalité entre~$E$ et~$L_0$
\[ E = - \frac{\mathcal{G}\,M}{2\,a\,\sqrt{\mathcal{G} \, M \, a \, (1-e^2)}} \,L_0 \qquad \Rightarrow \qquad \ddf{E}{L_0} = \frac{E}{L_0} = - \frac{1}{2} \sqrt{ \frac{\mathcal{G}\,M}{a^3 \, (1-e^2)} } \]
En introduisant~$\Gamma_m = \ddf{L_0}{t}$, le couple exercé par les forces de marée, on obtient la variation temporelle d'énergie orbitale par les forces de marée
\[ \ddf{E}{t} = \ddf{E}{L_0} \ddf{L_0}{t} = - \frac{\Gamma_m}{2} \, \sqrt{ \frac{\mathcal{G}\,M}{a^3 \, (1-e^2) } } \]
\noindent En exprimant par ailleurs la dérivée~$\ddf{E}{t}$
\[
\ddf{a}{t} \frac{\mathcal{G}\,m\,M}{2\,a^2} = \Gamma_m \, \sqrt{ \frac{\mathcal{G}\,M}{a^3 \, (1-e^2) } }
\qquad \Rightarrow \qquad
\ddf{a}{t} = \Gamma_m \, \frac{2\,\mathcal{G}^{-1/2}\,m^{-1/2}}{M \, a^{-1/2} \, (1-e^2) }
\]
\noindent En remplaçant par l'expression du couple des marées 
\[ \Gamma_m = \frac{3}{2} \, \frac{k_T}{Q} \, \frac{\mathcal{G}\,M^2\,R^5}{r^6} \, \textrm{sgn}(\omega-\Omega) \]
et en supposant une orbite de faible excentricité~$r \simeq a$ et~$e \ll 1$, le changement de rayon orbital (contraction ou expansion) s'écrit
\[
\ddf{a}{t} = 3 \, \frac{k_T}{Q} \, \frac{\mathcal{G}^{1/2}\,M\,R^5}{m^{1/2} \, a^{11/2} } \, \textrm{sgn}(\omega-\Omega)
\]
Ce type de formule permet de déterminer que la Lune s'éloigne de la Terre à une vitesse d'environ 4~cm~an$^{-1}$.

\sk
\paragraph{Circularisation} Les forces de marée ont également pour effet de rendre les orbites circulaires en faisant tendre l'excentricité~$e$ vers~$0$. Si la révolution du corps autour du corps attracteur est excentrique, en supposant la rotation synchrone, la différence de vitesse orbitale entre le passage au périastre plus rapide que le passage à l'apoastre fait qu'un couple de force va s'exercer pour repousser le corps au périastre et l'attirer à l'apoastre, rendant ainsi l'orbite plus circulaire. La littérature (e.g. Peale 2003) fournit une formule quantitative de cet effet
\[ \ddf{e}{t} = - \frac{21}{2} \, \frac{k_T}{Q} \, \frac{M}{m} \, \left( \frac{R}{a} \right)^5 \, e  \]
\noindent ce qui indique que l'échelle de temps caractéristique~$\tau = \frac{-e}{\ddf{e}{t}} $ de la circularisation varie en~$\left( \frac{a}{R} \right)^5$, donc très abruptement avec le demi-grand axe~$a$. Ceci explique que les orbites des satellites des planètes géantes ont une faible excentricité, et que la plupart des exoplanètes ayant un demi-grand axe inférieur à $0.07$~UA ont des orbites circulaires.
