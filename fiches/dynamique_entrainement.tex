\sk
On considère un point M immobile par rapport à la surface de la Terre. Les forces (massiques) subies par M sont la force de gravitation \v G, dirigée vers le centre de la Terre, et \v R la réaction du sol dirigée perpendiculairement à la surface (figure \ref{fig:centrif}). Dans le référentiel fixe, l'accélération de M est celle du mouvement circulaire uniforme: $\v a_e = - \Omega^2 \, \vl{HM}$ (accélération d'entrainement). On doit donc avoir \[\v a_e=\v G+\v R\] 
C'est impossible si la Terre est sphérique (sauf au pôle et à l'équateur): on aurait alors \v R et \v G colinéaires mais pas dans la direction de $\v a_e$. La Terre a en fait pris une forme aplatie, où la surface n'est pas perpendiculaire à~$\v G$. En posant $\v g=\v G-\v a_e$, l'équilibre devient: \[\v g+\v R=\v 0\] On a donc une gravité apparente \v g dirigée localement vers le bas (perpendiculairement à la surface) mais pas exactement vers le centre de la Terre. La gravité réelle \v G a elle une faible composante horizontale. Dans ce qui suit, on considère que l'accélération d'entraînement est inclus dans le terme~$\v g$.

%\figside{0.55}{0.25}{\figfrancis/centrif}{Equilibre d'un point posé au sol. La forme réelle de la Terre est en trait continu, la sphère en pointillés.}{fig:centrif}
\figside{0.45}{0.2}{\figfrancis/centrif}{Equilibre d'un point posé au sol. La forme réelle de la Terre est en trait continu, la sphère en pointillés.}{fig:centrif}


