\sk
Il existe une seconde loi de Kirchhoff, différente de celle précitée, qui stipule que l'émissivité spectrale doit être égale au coefficient d'absorption du corps $$ \epsilon_\lambda = \alpha_\lambda $$ pour des quantités intégrées selon toutes les directions de l'espace. Un corps ne peut émettre que les radiations qu'il est capable d'absorber. En d'autres termes, pour une température et une longueur d'onde donnée, un bon émetteur est souvent un bon absorbant (et vice versa). On retrouve par ce principe que le corps noir est le corps idéal qui rayonne un maximum d'énergie radiative à chaque température et pour chaque longueur d'onde. 
%Un corollaire est qu'un corps transparent ou réfléchissant à une certaine longueur d'onde émet peu de rayonnement thermique à cette même longueur d'onde. NON CAR IL SUFFIT QUE LA TEMPERATURE SOIT ELEVEE !

