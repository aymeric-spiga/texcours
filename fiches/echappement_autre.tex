\sk
\paragraph{Echappement non thermique} Le flux d'échappement de Jeans donne une limite inférieure au flux d'échappement de l'atmosphère d'un corps céleste. Le flux d'échappement peut en réalité être plus élevé du fait de phénomènes qualifiés de non-thermiques, ayant trait à la photochimie de la haute atmosphère et des phénomènes ioniques. Ces mécanismes peuvent se révéler importants pour les atomes plus lourds que le formalisme de Jeans n'aurait pas indiqué en échappement.
\begin{description}
\item{\emph{dissociation et recombinaison dissociative}} : une molécule est dissociée par les radiations UV ou un électron impactant, ou un ion se dissocie par recombinaison, formant des produits qui ont une énergie suffisante pour échapper à l'attraction gravitationnelle d'un corps
\item{\emph{réaction ion-neutre}} : une réaction entre un ion atomique et une molécule peut donner un ion moléculaire et un atome rapide
\item{\emph{échange de charge}} : un ion rapide peut échanger sa charge avec un neutre sans perdre son énergie cinétique, résultant en un neutre rapide pouvant s'échapper de l'attraction gravitationnelle du corps\footnote{Phénomène clé sur Io où des atomes de sodium rapides sont créés par échange de charge avec le plasma magnétosphérique.}
%\item{\emph{champs électriques}} les champs électriques dans la ionosphère accélèrent les ions, qui à leur tour peuvent accélérer les neutres par collision.
\item{\emph{bombardement (\emph{sputtering})}} : quand un atome, ou plus généralement un ion, rapide entre en collision avec un atome, ce dernier peut s'échapper ; ce mécanisme est important dans une variété d'atmosphères (épaisses ou ténues) et même sur les surfaces planétaires des corps dépourvus d'atmosphère.
\item{\emph{balayage \emph{sweeping} par le vent solaire}} sans champ magnétique planétaire, les particules chargées peuvent réagir directement avec le vent solaire et être emportées. %% voir SL p89
\end{description}

\sk
\paragraph{Echappement fluide} Ces mécanismes d'échappement -- de caractère plus extrême voire catastrophique -- ont pu être dominants par le passé pour les corps que nous connaissons.
\begin{description}
\item{\emph{échappement hydrodynamique}} : un courant (super)sonique formé des atomes les plus légers entraîne les atomes et molécules plus lourdes -- qui n'auraient pas subi d'échappement thermique au sens de Jeans ; une source d'énergie intense dans la haute atmosphère est requise pour que l'échappement hydrodynamique soit significatif : pour les atmosphères de Vénus, la Terre et Mars dans un lointain passé, cela a pu être possible en raison d'un flux extrême UV soutenu.
\item{\emph{érosion par impact}} : si l'impacteur est plus grand qu'une échelle de hauteur, une fraction significative du gaz chaud suite au choc peut s'échapper si la vitesse d'impact la vitesse d'échappement.
\end{description}

%% exercice: calculer la taille typique d'un impacteur
