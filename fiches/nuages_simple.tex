Un \voc{nuage} se définit comme un regroupement localisé de gouttelettes d'eau et/ou de cristaux de glace ou de neige en suspension dans l'atmosphère. 

\figside{0.5}{0.2}{decouverte/cours_meteo/gouttes.png}{Vue schématique des composants d'un nuage pluvieux chaud. Les tailles indicatives sont précisées afin de donner une idée des quelques ordres de grandeur en taille qui séparent une gouttelette nuageuse d'une goutte de pluie. La traduction des termes est la suivante: \emph{cloud-condensation nuclei} \donc~noyaux de condensation, \emph{moisture droplets} \donc~gouttelettes nuageuses, \emph{typical raindrop} \donc~goutte de pluie. Les diamètres sont indiqués en microns (10$^{-6}$~m). Source: à partir de MacDonald Adv. Geophys. 1958.}{fig:cloud}

\begin{finger}
\item Dans les nuages chauds, la vapeur d'eau condense en gouttelettes nuageuses dont la taille est de quelques microns. Les conditions locales de saturation (autrement dit, l'équilibre liquide-vapeur) déterminent le taux de condensation et de croissance des gouttes. Même si cela n'est pas l'objet du présent cours, il convient de noter que l'interface courbée des gouttes a un certain coût énergétique~: il est difficile de former des gouttes à moins d'atteindre une sursaturation très élevée (c'est-à-dire une humidité très supérieure à~$1$). La formation des gouttelettes nuageuses par condensation est par contre facilitée par la présence de \voc{noyaux de condensation} (par exemple, les poussières atmosphériques). Ensuite, les gouttelettes nuageuses peuvent, par collision ou coalescence, croître de plusieurs ordres de grandeurs en taille pour donner naissance à des gouttes de pluie de plusieurs millimètres de large qui donnent lieu à des précipitations. La figure~\ref{fig:cloud} donne un aperçu très schématique de l'intérieur d'un nuage pluvieux chaud.
\item Dans les nuages froids, i.e. ceux qui se trouvent dans une zone où la température est plus faible que 0 degrés celsius, règne un équilibre à trois phases (solide, liquide, gaz). Les gouttelettes nuageuses et cristaux de glace peuvent se former par condensation directe (des phénomènes de surfusion expliquent que les gouttelettes d'eau restent à l'état liquide). La neige se forme par agrégation de cristaux de glace. Par accrétion, plus précisément coalescence liquide sur glace, la grêle peut apparaître dans un nuage froid et exceptionnellement former des espèces précipitantes d'un diamètre important.
\end{finger}

\sk
La formation des gouttes et cristaux dans un nuage obéit à un ensemble de lois dites microphysiques dont la complexité dépasse le présent cours. Pour simplifier la description, on s'intéresse souvent aux nuages chauds uniquement. De plus, on se place dans le cas d'un équilibre thermodynamique liquide-vapeur pour une interface plane, pour laquelle la valeur maximum de l'humidité est~$1$ et suffit à déclencher la formation d'un nuage dans l'atmosphère.


