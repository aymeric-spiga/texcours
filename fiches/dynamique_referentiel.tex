\sk
La position d'un point $M$ de l'atmosphère sera représentée dans un systèmes de coordonnées sphériques (figure~\ref{fig:repere}) par sa latitude $\varphi$, sa longitude $\lambda$, et son altitude~$z$ par rapport au niveau de la mer. Pour les déplacements horizontaux, on utilise le repère direct
$\left(M,\mathbf{i},\mathbf{j},\mathbf{k}\right)$ où $\mathbf{i}$ et $\mathbf{j}$ sont les vecteurs unitaires vers l'est et le nord, et $\mathbf{k}$ est dirigé suivant la verticale vers le haut. La direction définie par~$\mathbf{i}$ est souvent qualifiée de \voc{zonale}, celle définie par~$\mathbf{j}$ de \voc{méridienne}. 
%Pour des déplacements qui ne sont pas d'échelle planétaire, on utilisera également des distances horizontales vers l'est et le nord~$dx=a\, d\lambda\, \cos \varphi$ et~$dy=a\,d\varphi$ où~$a$ est le rayon de la Terre.
%
%\sk
On distingue deux référentiels pour l'étude des mouvements de l'air:
\begin{finger}
\item Un \voc{référentiel tournant} lié à la Terre, en rotation autour de l'axe des pôles avec la vitesse angulaire $\Omega$. La \voc{vitesse relative} est mesurée dans le référentiel tournant, par rapport à la surface de la Terre et a pour composantes~$u,v,w$ suivant \v i,\v j,\v k. Il s'agit de ce que l'on appelle communément le \voc{vent} avec le point de vue d'humain attaché à la surface de la Terre, c'est-à-dire au référentiel tournant. La composante horizontale du vecteur vitesse relative est donc~$\mathbf{V} = u \, \mathbf{i} + v \, \mathbf{j}$ et la composante verticale~$w \, \mathbf{k}$.
\item Un \voc{référentiel fixe} orienté suivant les directions de trois étoiles. La \voc{vitesse absolue} d'un point M est considérée dans le référentiel fixe et inclut donc le mouvement circulaire autour de l'axe des pôles. Ce référentiel peut être considéré comme galiléen. Il correspond à ce qu'on observerait depuis l'espace, lorsqu'on voit la Terre tourner au lieu d'être \ofg{attaché} à sa rotation.
\end{finger}

%\figun{0.4}{0.25}{\figfrancis/repere}{Schéma du système de coordonnées et du repère utilisés.}{fig:repere}
\figside{0.4}{0.2}{\figfrancis/repere}{Système de coordonnées et repère utilisés.}{fig:repere}


