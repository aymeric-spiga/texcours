\sk Le modèle à une couche peut être généralisé quelque peu en considérant deux raffinements.
\begin{finger}
\item L'atmosphère est absorbante dans le visible avec un coefficient d'absorption~$\alpha$. Plus~$\alpha$ est grand, plus le rayonnement incident dans les longueurs d'onde visible reçu par la surface terrestre est atténué. Ce phénomène porte le nom d'\voc{effet parasol} (ou parfois \ofg{anti-effet de serre}). En réalité cet effet est très modéré sur Terre. Certes, l'ozone stratosphérique absorbe complètement le rayonnement ultraviolet, mais ceci représente une contribution faible du flux total. Les aérosols, tels que les poussières désertiques ou les particules d'origine volcanique, absorbent dans le visible et peuvent contribuer lors d'événements particuliers, telles les éruptions volcaniques ou les tempêtes de poussière, à augmenter~$\alpha$.
\item L'atmosphère n'est pas tout à fait un corps noir~: son émissivité dans l'infrarouge est~$\epsilon$ comprise entre~$0$ et~$1$. D'après la seconde loi de Kirchhoff, ceci indique que la couche atmosphérique n'est pas parfaitement absorbante dans l'infrarouge~: elle absorbe une partie~$\epsilon \, F$ du flux incident~$F$ et en transmet une partie~$(1-\epsilon) \, F$. Reste qu'en pratique, comme mentionné dans la partie précédente, l'atmosphère se comporte comme un corps presque noir dans l'infrarouge et l'émissivité~$\epsilon$ est relativement proche de~$1$. 
\end{finger}
Un tel modèle porte le nom de \voc{modèle gris à une couche}. Comme dans la figure~\ref{fig:modun}, la méthode pour calculer les températures consiste à reporter tous les flux échangés entre chacune des couches, comme indiqué dans la figure~\ref{fig:modgris}, puis de faire le bilan des énergies reçues et cédées aux interfaces.
\[ \begin{aligned} & & \boxed{\text{bilan des flux reçus}} & = \boxed{\text{bilan des flux cédés}} \\ 
& \text{[espace]} & \mathcal{F}\e{s}'\,A\e{b} + (1-\epsilon) \, \sigma \, {T\e{s}}^4 + \epsilon \, \sigma \, {T\e{a}}^4 & = \mathcal{F}\e{s}' \\
& \text{[atmos.]} & \mathcal{F}\e{s}' + \sigma \, {T\e{s}}^4 & = \mathcal{F}\e{s}'\,A\e{b} + \mathcal{F}\e{s}'\,(1-A\e{b}-\alpha) + (1-\epsilon) \, \sigma \, {T\e{s}}^4 + \epsilon \, \sigma \, {T\e{a}}^4 + \epsilon \, \sigma \, {T\e{a}}^4 \\ 
& \text{[surface]} & \mathcal{F}\e{s}'\,(1-A\e{b}-\alpha) + \epsilon \, \sigma \, {T\e{a}}^4 & = \sigma \, {T\e{s}}^4 \\ \end{aligned} \]
%& \text{[surface]} & \mathcal{F}\e{s}'\,(1-A\e{b})\,(1-\alpha) + \epsilon \, \sigma \, {T\e{a}}^4 & = \sigma \, {T\e{s}}^4 \\ \end{aligned} \]

\sk
On obtient alors l'expression de la température de la surface de la planète dans le cadre du modèle gris en combinant les équations de bilan de deux des trois interfaces (par exemple espace et surface)
\[ \boxed{T\e{s} = \sqrt[4]{\frac{1 - \frac{\alpha^{\prime}}{2}}{1 - \frac{\epsilon}{2}}} \, T\e{eq}} \qquad  \text{avec} \quad \alpha^{\prime} = \frac{\alpha}{1-A\e{b}} \] 
\begin{citemize}
\item Dans le cas où l'atmosphère est un corps noir dans l'infrarouge~($\epsilon=1$) et qu'elle est transparente dans le visible~($\alpha=0$), on se retrouve dans la situation du modèle à une couche avec~$T\e{s} = \sqrt[4]{2} \, T\e{eq}$.
\item Dans le cas où l'atmosphère est opaque dans le visible~($\alpha^{\prime}=1$) et dans l'infrarouge~($\epsilon=1$), la surface échange alors uniquement du rayonnement avec l'atmosphère et~$T\e{s}=T\e{a}=T\e{eq}$.
\item Si~$\epsilon$ augmente, l'effet de serre augmente, donc la température de la surface de la planète augmente.
\item Si~$\alpha$ augmente, l'effet parasol augmente, donc la température de la surface de la planète diminue. C'est l'un des effets observés dans les mois qui suivent une éruption volcanique majeure.
\end{citemize}
%De façon plus générale, on a vu que le rayonnement sortant provenait majoritairement de la région de l'atmosphère autour d'une épaisseur optique de 1 à partir du sommet. Cette région dépend de la longueur d'onde: proche de la surface dans la fenêtre transparente, dans la haute troposphère dans les bandes d'absorption du CO$_2$, autour de 2~km dans celles de la vapeur d'eau. Comme la température décroit à partir de la surface, le rayonnement sortant est donc émis à des températures inférieures à $T_s$, et on peut écrire qu'il vaut \[IR_{sommet}=\sigma T_s^4 (1-\epsilon)=\sigma T_{eq}^4\] Où $\epsilon>0$ représente l'effet de serre. La valeur de $\epsilon$ augmente quand la température d'émission vers l'espace diminue par rapport à celle de surface, typiquement parce que l'altitude d'émission augmente.

\figside{0.7}{0.17}{decouverte/cours_meteo/une_couche_gris.png}{Modèle gris à une couche~: schéma des flux échangés dans le visible et dans l'infrarouge pour une planète comme la figure~\ref{fig:modun} sauf que l'atmosphère de température~$T\e{a}$ est opaque dans l'infrarouge, mais sans être un corps totalement noir, et est absorbante dans le visible avec un coefficient d'absorption~$\alpha$.}{fig:modgris}
%\figun{0.6}{0.2}{\figfrancis/GH_1lay_atm}{Comme la figure \ref{fig:GH1laynoatm} mais avec une atmosphère opaque dans l'infrarouge et de coefficient d'absorption $a$ dans le visible, de température $T_a$.}{fig:GH1layatm}
