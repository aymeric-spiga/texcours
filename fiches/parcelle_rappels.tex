
\sk
Tout le but de ce chapitre est de décrire les relations thermodynamiques qui lient les grandeurs intensives qui caractérisent l'état de la parcelle. Une première de ces relations a été obtenue en introduction~: il s'agit de l'équation des gaz parfaits pour l'air atmosphérique, qui relie les trois paramètres intensifs $P$, $T$ et $\rho$ 
\[ \boxed{ P = \rho \, R \,T } \] 
avec la \voc{constante de l'air sec} $R=\frac{R^*}{M}$ où~$R^*$ est la constante des gaz parfaits et~$M$ est la masse molaire de l'air. On rappelle que sur Terre~$R = 287$~J~K$^{-1}$~kg$^{-1}$. L'état thermodynamique d'une parcelle d'air est donc déterminé uniquement par deux paramètres sur les trois~$P$, $T$, $\rho$. Pour les applications météorologiques, on caractérise en général l'état de la parcelle par sa pression~$P$ et sa température~$T$, plus aisées à mesurer, par exemple via des ballons-sondes, que la densité~$\rho$.

