\sk
L'équation du vent thermique indique que les jets d'altitude dans la branche descendante de la cellule de Hadley (aux moyennes latitudes) conduisent à un renforcement des gradients latitudinaux de température, qui ne peuvent être résorbés par la circulation de Hadley. Aux moyennes latitudes, les énergies cinétique et thermique sont plus facilement redistribuées par les instabilités non-axisymétriques que par la circulation zonale axisymétrique. L'écoulement zonal axisymétrique des moyennes latitudes terrestres et martiennes peut ainsi donner naissance à une circulation non axisymétrique par des instabilités barotropes et baroclines. A partir d'une certaine latitude, ces instabilités sont bien plus efficaces pour redistribuer l'énergie.

\sk
Les perturbations \voc{barotropes} de l'écoulement moyen se développent en extrayant de l'énergie cinétique au cisaillement horizontal de vent de cet écoulement moyen (ex: courant-jet de forte amplitude). Les tourbillons barotropes ont une structure verticale constante avec l'altitude et transportent de la quantité de mouvement selon la latitude, afin de réduire le cisaillement qui leur a donné naissance.

\sk
L'instabilité \voc{barocline} résulte au contraire des gradients latitudinaux de température aux moyennes latitudes, associés à un cisaillement vertical de vent par l'équilibre du vent thermique. Les ondes baroclines générées transportent de la chaleur (et un peu de quantité de mouvement) en latitude et en altitude pour réduire l'inclinaison des isentropes qui leur a donné naissance. Les perturbations baroclines se développent par conversion de l'énergie potentielle disponible de l'écoulement zonal moyen en énergie cinétique.

%%\note{Stabilité des états d'équilibre~\donc~Réponse à une perturbation~:~s'amplifie-t-elle ? A chaque équilibre son instabilité associée~:~ exemple équilibre hydrostatique et instabilités convectives} \note{Courants-Jet d'altitude branche Hadley descendante~$+$~Renforcement gradients latitudinaux de température~\donc~Instabilités aux moyennes latitudes, Notion d'Energie Potentielle Disponible de l'écoulement zonal moyen.}


