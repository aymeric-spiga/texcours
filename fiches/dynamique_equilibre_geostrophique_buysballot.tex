\sk
La carte de la pression et du vent en surface (figure \ref{fig:meteofrance}, voir aussi figure \ref{fig:SLPwind}) montre clairement que l'orientation et le module du vent sont dictés par l'équilibre géostrophique. Lorsque la friction est élevée proche de la surface, l'équilibre géostrophique est perturbé par la présence de la force de friction, ce qui a pour conséquence de donner un vent légèrement dévié vers l'intérieur des dépressions et vers l'extérieur des anticyclones. Quand le nombre de Rossby est grand (donc à petite échelle), l'équilibre géostrophique ne s'applique plus et le vent est accéléré des hautes vers les basses pressions. 
%% PARLER DU LAVABO

\figside
%{0.85}{0.6}
{0.65}{0.4}
{decouverte/cours_dyn/carte_france.jpg}{Carte météorologique Météo-France construite à partir des données relevées dans les stations météorologiques indiquées par des points. Les lignes isobares montrent qu'une forte dépression se situe au nord du Royaumu-Uni. Les \ofg{drapeaux} accolés aux points d'observations représentent les vents mesurés à la surface (le nombre de barres indique la force du vent). La direction du vent part du drapeau vers le point considéré. On remarque que le vent est approximativement parallèle aux isobares et tourne dans le sens inverse des aiguilles d'une montre autour de la dépression. Ce comportement est typique de celui déduit pour l'hémisphère Nord par l'équilibre géostrophique entre forces de pression et force de Coriolis (voir figure~\ref{fig:geost}). Le vent est légèrement rentrant vers l'intérieur de la dépression, sous l'influence de la force de friction qui vient d'ajouter aux deux forces précitées.}{fig:meteofrance}
