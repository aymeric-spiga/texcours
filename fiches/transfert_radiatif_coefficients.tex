\sk
Tout rayonnement se propageant dans un milieu matériel subit trois phénomènes~: réflexion, absorption, transmission. Autrement dit, tout corps cible irradié par une source voit le flux énergétique incident spectral~$\Phi_{\lambda}$ se répartir selon trois termes
\begin{citemize}
\item une partie~$\Phi_{\lambda}^r$ du flux incident est réfléchie ou diffusée;
\item une partie~$\Phi_{\lambda}^t$ du flux incident traverse le corps sans interactions;
\item une partie~$\Phi_{\lambda}^a$ du flux incident est absorbée, c'est-à-dire transformée en énergie interne.
\end{citemize}
Afin de définir les contributions respectives de ces trois phénomènes, on définit des coefficients spectraux de \voc{réflexion}~$\rho_{\lambda}$, de \voc{transmission}~$\tau_{\lambda}$, d'\voc{absorption}~$\alpha_{\lambda}$ compris entre~$0$ et~$1$
$$\Phi_{\lambda}^r = \rho_{\lambda} \, \Phi_{\lambda} \qquad\qquad \Phi_{\lambda}^t = \tau_{\lambda} \, \Phi_{\lambda} \qquad\qquad \Phi_{\lambda}^a = \alpha_{\lambda} \, \Phi_{\lambda}$$
Ces coefficients sont également appelés \voc{réflectivité}~$\rho_{\lambda}$, \voc{transmittivité}~$\tau_{\lambda}$, \voc{absorptivité}~$\alpha_{\lambda}$. Ils dépendent de la longueur d'onde~$\lambda$ du rayonnement incident, de l'angle d'incidence et des propriétés physiques et chimiques du corps récepteur (par exemple, température, composition). Lorsque le coefficient de réflexion~$\rho_{\lambda}$ ne dépend pas de l'angle d'incidence\footnote{L'énergie incidente à une surface pénètre dans celle-ci et est réfléchie aléatoirement à l'intérieur de l'objet par de microscopiques in-homogénéités du matériau. Au cours de ces multiples réflexions une partie de l'énergie incidente ressort de l'objet suivant une direction aléatoire. Bien souvent les réflexions multiples dans le matériau ne subissent aucune contrainte particulière, l'énergie est donc réfléchie de façon uniforme et isotrope par la surface. Le flux réfléchi est alors uniquement fonction de la quantité d'énergie incidente tombant sur la surface, qui s'exprime souvent simplement comme un cosinus de l'angle entre la normale à la surface et la direction de la source.}, on parle de l'objet cible comme d'un \voc{réflecteur lambertien}. 

%\sk
%\subsection{Rappel sur la distinction entre grandeurs intégrées et spectrales}
%\sk
%Dans le chapitre précédent, les grandeurs caractéristiques~$\Phi$,~$F$ ou~$L$, ainsi que les coefficients d'absorption~$\rho$, de transmission~$\tau$, de réflexion~$\rho$ (également appelé albédo~$A$), ont été décrits 
%\begin{citemize}
%\item soit d'une façon intégrée selon toutes les longueurs d'onde (par exemple l'émittance~$M$ dans la loi de Stefan-Boltzmann) ;
%\item soit en prenant en compte la dépendance spectrale, c'est-à-dire en considérant un petit intervalle~$\dd \lambda$ autour d'une longueur d'onde~$\lambda$ donnée (par exemple la luminance énergétique spectrale~$B_{\lambda}$ dans la loi de Planck). 
%\end{citemize}
%Dans le premier cas, on emploie simplement les symboles décrivant les grandeurs (exemple, luminance~$L$). Dans le second cas, on ajoute $\lambda$ en indice de ces symboles (exemple, luminance spectrale~$L_\lambda$). Ce qui est clair pour les variables l'est beaucoup moins dans le vocabulaire couramment utilisé, y compris dans certains ouvrages.
