\sk
La concentration, au sens où elle est définie dans la plupart des cours de physique / chimie, est une quantité très peu utilisée en sciences de l'atmosphère. La composition chimique de l'atmosphère s'exprime préférentiellement en utilisant le \voc{rapport de mélange volumique}, soit la proportion d'un volume d'air occupée par un gaz particulier. L'air étant un gaz parfait, ce rapport de mélange volumique est simplement égal au rapport du nombre de molécules/atomes du gaz sur le nombre total de molécules d'air $$ \boxed{ r_i=\frac{n_i}{\Sigma n_k} } \qquad \textrm{\footnotesize (parfois également noté $q_i$)} $$ D'après la loi de Dalton, il correspond également au rapport entre la pression partielle du gaz considéré avec la pression totale du mélange. Le rapport de mélange n'est exprimé en pourcentage que pour les composés les plus abondants. Pour les \voc{gaz traces}, moins abondants, on exprime le rapport de mélange en parties par million (1 ppmv = $10^{-6}$), ou par milliards (1 ppbv = $10^{-9}$), voire le pptv tel que~1 pptv = $10^{-12}$. Dire que le rapport de mélange de CO$_2$ au sol est d’environ 380 ppmv signifie que sur un million de molécules d’air, 380 sont des molécules de CO$_2$. %On utilise également parfois le rapport de mélange massique, soit la proportion d'une masse d'air occupée par un gaz particulier. 

