\sk
L'équilibre radiatif simple présenté à la section précédente souffre d'un problème majeur~: il suppose que l'atmosphère n'interagit pas avec les rayonnements incidents et émis, ce qui n'est pas le cas en réalité. On rappelle notamment avec la figure~\ref{fig:atmspectrum} deux points importants qui vont nous permettre de raffiner les calculs.
\begin{finger}
\item Au vu des températures typiques du Soleil et de la Terre, le rayonnement d'origine solaire est principalement émis dans les longueurs d'onde visible, alors que le rayonnement d'origine terrestre est principalement émis dans les longueurs d'onde infrarouge. Les fonctions de Planck normalisées montrées dans la figure~\ref{fig:atmspectrum} indiquent que les deux domaines d'émission ne se recoupent quasiment pas. On peut donc séparer les calculs selon le domaine visible (également appelé ondes courtes) pour tout ce qui concerne le rayonnement reçu du Soleil et le domaine infrarouge (également appelé ondes longues) pour tout ce qui concerne le rayonnement émis par la surface et l'atmosphère de la Terre. La figure~\ref{fig:modzero} reprend ainsi le calcul de l'équilibre radiatif simple en étant plus fidèle à cette distinction entre visible et infrarouge~; en l'absence d'atmosphère, la température de surface à l'équilibre~$T\e{s}$ est égale à~$T\e{eq}$.
%\figside{0.6}{0.3}{\figfrancis/WH_atmspectrum}{entre le sommet de l'atmosphère et 11~km.}{fig:atmspectrum}
\item L'équilibre radiatif simple néglige les propriétés d'absorption de l'atmosphère de la Terre. La figure~\ref{fig:atmspectrum} montre que cette approximation est relativement juste pour les longueurs d'onde visible, où l'atmosphère est assez transparente, mais très inexacte pour les longueurs d'onde infrarouge. On a vu dans les chapitres qui précèdent que, contrairement à ce qui prévaut dans les longueurs d'onde visible, l'atmosphère est très opaque, c'est-à-dire très absorbante, dans l'infrarouge à cause principalement des gaz à effet de serre (et des nuages). Comme décrit au chapitre précédent, et sur la figure~\ref{fig:atmspectrum}, les principaux gaz à effet de serre sont, par ordre d'importance dans le bilan radiatif de la Terre, H$_2$O, CO$_2$, CH$_4$, N$_2$O, O$_3$, auxquels il convient d'ajouter les gaz à effet de serre industriels, tels les halocarbures, notamment les chloro-fluoro carbures\footnote{Qui jouent par ailleurs un rôle dans la destruction de l'ozone stratosphérique}. On note au passage que certains gaz à effet de serre comme CO$_2$ et CH$_4$ sont à la fois controlés par des processus naturels et industriels. Le rayonnement émis par la surface terrestre, principalement dans l'infrarouge, est donc absorbé par ces espèces et réémis à la fois vers l'espace et vers la surface. Ainsi, contrairement à ce qui est supposé dans le cas de l'équilibre radiatif simple, une partie du rayonnement émis par la surface n'est pas évacuée vers l'espace et contribue à augmenter la température de la surface terrestre. Ainsi la température de surface à l'équilibre~$T\e{s}$ n'est pas égale à la température équivalente~$T\e{eq}$. La figure~\ref{fig:modun} résume cette situation qui permet d'obtenir par le calcul, présenté ci-dessous, une valeur pour~$T\e{s}$ plus proche de la température effectivement mesurée à la surface de la Terre. On parle de \voc{modèle à une couche}.
\end{finger}

\figside[page=1]{0.6}{0.25}{decouverte/cours_meteo/zero_couche.png}{Modèle à zéro couche~: schéma des flux nets échangés dans le visible et dans l'infrarouge pour une planète sans atmosphère (ou plus précisément dans laquelle l'atmosphère n'est active radiativement ni dans l'infrarouge ni dans le visible) de température de surface~$T\e{s}$. Il s'agit simplement d'une présentation alternative de l'équilibre radiatif simple décrit en figure~\ref{fig:eqrad2}, qui s'avère plus pratique pour prendre en compte la présence d'une atmosphère et effectuer des calculs plus proches de la réalité. Ce schéma est cependant plus précis que la figure~\ref{fig:eqrad2} dans la mesure où il précise dans quel domaine de longueur d'onde se font les échanges.}{fig:modzero}
%\figun{0.6}{0.2}{\figfrancis/GH_1lay_noatm}{Schéma des flux échangés dans le visible (jaune) et l'infrarouge (rouge) pour une planète sans atmosphère de température de surface $T_s$.}{fig:GH1laynoatm}
