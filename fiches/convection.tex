
\sk
Dans les atmosphères denses, les intérieurs en fusion, et même les disques protoplanétaires, le transport d'énergie est souvent le plus efficace via les transports par le fluide sur de grandes échelles spatiales, notamment par convection. La \voc{convection} est le mouvement dans un fluide causé par des gradients de densité, survenant en raison de contrastes de température.

\sk
Une parcelle atmosphérique plus chaude que son environnement ($T\e{p}>T\e{e}$) s'élève. Par équilibre mécanique, sa pression s'équilibre immédiatement avec l'environnement ($P\e{p}=P\e{e}$) donc diminue. En revanche, en raison de la mauvaise conductivité thermique de l'air, sa température~$T\e{p}$ suit des variations avec l'altitude (en conditions approximativement adiabatiques) distinctes de celles de l'environnement~$T\e{e}$. En fonction des variations de la température de l'environnement~$T\e{e}$ avec l'altitude, si celle-ci diminue suffisamment rapidement avec l'altitude, la parcelle atmosphérique va poursuivre son ascension. Il y aura un transport de chaleur vers le haut par convection.

\sk
Si le transport par convection domine une couche atmosphérique, sa structure thermique suit un gradient adiabatique. Autrement dit, l'action par convection d'une multitude de parcelles de température~$T\e{p} \neq T\e{e}$ finit par causer un profil d'environnement proche du profil adiabatique suivi par une parcelle individuelle.
