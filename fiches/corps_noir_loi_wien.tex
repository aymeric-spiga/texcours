\sk
On observe sur la figure \ref{fig:BBrad} que, lorsque~$T$ augmente, la maximum de la luminance spectrale~$B_\lambda$, appelé \voc{maximum d'émission}, se décale vers les longueurs d'onde courtes, c'est-à-dire correspond à des photons de plus en plus énergétiques. La loi exacte, appelée \voc{loi de déplacement de Wien}, s'obtient en dérivant $B_\lambda$ par rapport à $\lambda$, ce qui permet d'obtenir $$ \boxed{ \lambda\e{max} \, T = 2.898 \times 10^{-3} \, \textrm{(mètres~K)} } $$ où $\lambda_{max}$ est la longueur d'onde du maximum de luminance spectrale~$B_\lambda$. La longueur d'onde du maximum d'émission~$\lambda\e{max}$ est ainsi inversement proportionnelle à la température du corps émetteur. Une formulation alternative est que $\nu\e{max}$ est proportionelle à $T$.
