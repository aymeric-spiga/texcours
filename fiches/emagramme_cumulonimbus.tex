\sk
On s'intéresse ici au développement des nuages cumuliformes, en particulier les cumulonimbus. Le cas d'étude donné dans le radiosonsage exemple permet de suivre graphiquement les concepts de cette partie. Le point de départ est une parcelle d'air non saturée, c'est-à-dire dont l'humidité est inférieure à~$1$, située proche de la surface. On suppose que son rapport de mélange en vapeur d'eau~$r$ est conservé au cours de l'ascension. 

\sk
En premier lieu, de nombreux phénomènes atmosphériques vont provoquer une élévation de la parcelle que l'on considère initialement proche de la surface.
\begin{citemize}
\item{\textbf{Soulèvement frontal}} Un front est une variation marquée et localisée de température. Lorsqu'un front se déplace horizontalement, l'air chaud passe au-dessus de l'air froid de densité moindre. 
\item{\textbf{Soulèvement orographique}} La présence d'un relief face au vent force les parcelles d'air à s'élever.
\item{\textbf{Convection sèche}} Un sol très chaud l'après-midi peut induire un profil de température de l'environnement très instable proche de la surface. Dans ce cas, les mouvements verticaux sont amplifiés proche de la surface par la poussée d'Archimède (voir chapitre précédent).
%%% circulation thermique. brise de terre et brise de mer.
\end{citemize}
Tous ces mécanismes expliquent que des nuages cumuliformes sont souvent trouvés au-dessus de régions soumises au passage de fronts, montagneuses ou dont la surface est particulièrement chaude. Ces nuages évoluent parfois vers un état de type cumulonimbus.

\sk
En second lieu, lorsqu'une parcelle d'air est soulevée vers les plus hauts niveaux de l'atmosphère par les phénomènes atmosphériques précités, elle subit un refroidissement par détente adiabatique. Le taux de refroidissement de la parcelle est~$|\Gamma\e{sec}|$. Sur un émagramme tel celui de la figure~\ref{fig:sounding}, la parcelle suit une \voc{courbe adiabatique sèche}. Cette décroissance de la température de la parcelle au cours de l'ascension a pour principale conséquence d'abaisser la valeur de $r\e{sat}$, de par les variations exponentielles de cette quantité avec la température. Il en résulte que l'humidité relative~$H = r / r\e{sat}$ augmente. Si la quantité de vapeur d'eau initiale~$r$ et/ou le soulèvement de la parcelle sont suffisants, $H$ peut atteindre~$1$ au cours de l'ascension~: la parcelle devient alors saturée. Des gouttelettes nuageuses apparaissent par condensation, autrement dit un nuage se forme. Le niveau d'altitude ou de pression auquel la condensation se produit suite à un refroidissement par soulèvement adiabatique s'appelle le \voc{niveau de condensation} ou la \voc{base du nuage}. A ce stade, le nuage n'est pas encore nécessairement cumuliforme.
%%Pour la convection. Les bases des nuages sont horizontales, leurs sommets évoluent en fonction de la température.

\sk
En troisième lieu, si la parcelle continue son ascension au-delà du niveau de condensation, sa température ne décroît plus d'un taux~$|\Gamma\e{sec}|$, mais d'un taux $|\Gamma\e{saturé}|$ plus faible, puisque la parcelle est désormais saturée (son humidité vaut~$1$ et son rapport de mélange~$r$ vaut~$r\e{sat}$). Sur l'émagramme, la parcelle suit une \voc{courbe adiabatique saturée}. Au cours de l'ascension, la parcelle reste saturée mais, puisque sa température continue de diminuer, $r\e{sat}$ diminue de concours, ce qui induit une diminution du rapport de mélange en vapeur d'eau~$r$ et une augmentation du rapport de mélange en eau liquide (qui prend la forme de gouttelettes nuageuses ou, si les conditions de croissance sont réunies, de précipitations pluvieuses).

\sk
En quatrième lieu, la forme du profil de température d'environnement détermine si, une fois le niveau de condensation atteint, le nuage va suivre ou non un développement vertical marqué. On rappelle que le profil d'environnement n'est pas celui suivi par la parcelle considérée, mais représente l'état atmosphérique tel qu'il peut être mesuré par un ballon-sonde météorologique par exemple.

\begin{finger}

\item Si les soulèvements initiaux de la parcelle ne l'amènent que dans des niveaux atmosphériques où sa température reste plus faible que celle de l'environnement, alors il n'y a pas de mouvements verticaux spontanés au sein du nuage. Le nuage formé est plutôt de type stratiforme (ou faiblement cumuliforme).

\item Si les soulèvements initiaux de la parcelle parviennent à la hisser à des niveaux atmosphériques où sa température devient plus élevée que celle de l'environnement, alors des mouvements verticaux spontanés apparaissent au sein du nuage. On parle de convection humide (ou convection profonde). Le nuage ainsi formé est cumuliforme. Le niveau atmosphérique à partir duquel la température de la parcelle en ascension adiabatique devient plus élevée que la température de l'environnement s'appelle le \voc{niveau de convection libre}. Le niveau atmosphérique à partir duquel la température de la parcelle redevient plus faible que l'environnement s'appelle le \voc{sommet théorique du nuage}. Si le sommet théorique du nuage est très élevé, la formation de cumulonimbus, donc d'orage, est très probable. 

\end{finger}


