\sk
Dans le cas où la transformation n'est pas adiabatique, les échanges de chaleur~$\delta q$ d'une parcelle d'air avec son environnement sont non nuls et peuvent s'effectuer par~:
\begin{itemize}
\item Transfert radiatif~: l'atmosphère se refroidit en émettant dans l'infrarouge, ou se réchauffe en absorbant du rayonnement électromagnétique dans l'infrarouge [cas des gaz à effet de serre] ou dans le visible [cas de l'ozone dans la stratosphère].
%Ces échanges sont faibles et peuvent être négligés sauf à l'échelle de la circulation générale\footnote{Le refroidissement/réchauffement peut être localement élevé au sommet/à la base de nuages.}
\item Condensation ou évaporation d'eau~: le changement d'état consomme ou relâche de la chaleur (ceci n'a lieu que lorsque l'air est à saturation).
\item Diffusion moléculaire (conduction thermique)~: ces transferts sont très négligeables sauf à quelques centimètres du sol.
\end{itemize}
Un cas notamment souvent cité en météorologie est celui d'une parcelle d'air située proche du sol, à la tombée de la nuit, qui subit peu de variations de pression ($\dd P \sim 0$) mais dont la température diminue sous l'effet du refroidissement radiatif ($\delta q < 0$). Ceci explique la présence de rosée sur le sol et de brouillard proche de la surface au petit matin.


