\sk
L'atmosphère est composée d'un ensemble de molécules microscopiques et l'on s'intéresse aux effets de comportement d'ensemble, qualifiés de \voc{macroscopiques}. Les variables thermodynamiques utilisées pour décrire l'atmosphère (pression~$P$, température~$T$, densité~$\rho$) sont des grandeurs macroscopiques \voc{intensives} dont la valeur ne dépend pas du volume d'air considéré. 
%Une façon d'y parvenir est d'utiliser des grandeurs volumiques ou massiques.

\sk
Le système que l'on étudie est appelé \voc{parcelle d'air}. Il s'agit d'un volume d'air dont les dimensions sont %à la fois
\begin{citemize}
\item assez grandes pour contenir un grand nombre de molécules et pouvoir moyenner leur comportement, afin d'exprimer un équilibre thermodynamique~;
\item assez petites par rapport au phénomène considéré, afin de décrire le fluide atmosphérique de façon continue~; la parcelle d'air peut donc être considérée comme un volume élémentaire.
\end{citemize}
On peut donc supposer que les variables macroscopiques d'intérêt sont quasiment constantes à l'échelle de la parcelle. Autrement dit, une parcelle est caractérisée par sa pression~$P$, sa température~$T$, sa densité~$\rho$. Les limites d'une parcelle sont arbitraires, mais ne correspondent pas en général à des barrières physiques. 


