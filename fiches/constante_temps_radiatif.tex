A condition qu'elle abrite
des particules radiativement actives,
une atmosphère de pression
plus faible se caractérise par une constante
de relaxation radiative $\tau\e{R}$ plus
courte.
%
En effet, lorsque la pression
diminue, la densité diminue également
mais l'énergie radiative absorbée n'est
pas proportionnelle à la densité.
%
Nous pouvons illustrer ce point avec
un calcul simpl(ist)e sur une couche
atmosphérique d'épaisseur $e$, de densité $\rho$ 
et se comportant comme un corps noir
de température $T\e{e}$
($\sigma$ est la constante de Stefan-Boltzmann).
%
Le temps caractéristique $\tau\e{R}$ pour 
dissiper radiativement une
perturbation thermique $\Theta=\Delta T$
de l'équilibre radiatif de la couche
avec les couches environnantes est
donné par la conservation de l'énergie
%%
%Le temps de relaxation radiatif $\tau\e{R}$, 
%inverse de la constante
%d'amortissement radiatif, est le temps 
%nécessaire pour dissiper une perturbation
%thermique par des processus radiatifs.
%%

\[
c_p \, \rho \, e \, S \, \dd T = - S \, \sigma \, T^4 \qquad \textrm{avec} \qquad T = T\e{e} \, (1 + \epsilon) \quad \epsilon = \frac{\Theta}{T\e{e}} \ll 1
\]

\[
\ddt{\Theta} + \f{\Theta}{\tau\e{R}} = 0
\quad \textrm{avec} \quad
\tau\e{R} = \f{c_p \, \rho \, e}{8 \, \sigma \, T\e{e}^3}
\]
%
\noindent L'expression de $\tau\e{R}$ traduit
bien le point mentionné précédemment.
%
Le rapport entre les constantes
de temps radiatives martiennes et terrestres
est donc principalement contrôlé
par la différence de densité.
%%
%Les valeurs planétaires donnent par exemple
%%
\[
\f{ \tau\e{Mars} }{ \tau\e{Terre} }
=
\f{ \left( c_p \, \rho / T\e{e}^3 \right)\e{Mars} }{ \left( c_p \, \rho / T\e{e}^3 \right)\e{Terre} }
\sim 
\f{1}{40}
\]
%
\noindent soit un très fort amortissement
radiatif dans l'atmosphère martienne, deux ordres
de grandeur plus élevé que sur Terre.
%
Dans les conditions typiques pour la basse
atmosphère terrestre et martienne, 
$\tau\e{Mars}$ est de l'ordre de la journée
alors que $\tau\e{Terre}$ est de l'ordre du mois.
%
Sur Terre, les différences de constante radiative
entre la basse et la haute atmosphère sont expliquées
de la même façon.
%
L'estimation ci-dessus reste illustrative plus que quantitativement
valable.
%
Cependant, des calculs plus élaborés 
distinguant les molécules radiativement actives 
dans l'\IR~thermique sur Mars (\carb) et sur Terre (\eau),
donnent $\tau\e{Mars}/\tau\e{Terre}$
entre $1/5$ et $1/100$, ce qui
ne contredit pas l'ordre de grandeur
trouvé par le calcul simpliste
précédent. 
%
%
%%La grande concentration en \carb~et
%%la densité faible font que le temps
%%de relaxation radiatif de \lam~est
%%très faible.






%\begin{frame}
%\frametitle{Temps radiatif}
%\begin{itemize}
%\item Temps de résorption radiative $t_{rad}$ d'une anomalie de
%  température $\Delta T$ par rapport au profil d'équilibre :
%  $\frac{\partial \Delta T}{\partial t} = - \frac{\Delta T}{t_{rad}}$
%\begin{alertblock}{~}
%\alert{
%\begin{columns}
%\column{.4\textwidth}
%Atmosphère « mince » \\
%\vspace{.3cm}
%Atmosphère « épaisse »
%\column{.4\textwidth}
%$t_{rad} = \frac{C_p}{16 \alpha g \sigma T^3
%  d\tau/dP}$\\
% \vspace{.3cm}
%$t_{rad} = \frac{3 C_p P^2
%  d\tau/dP}{32g\sigma T^3}$
%\end{columns}
%} 
%\end{alertblock}
%\item Ordres de grandeur
%\begin{itemize}
%\item {\bf Vénus} (épais) : $t_{rad} \sim 30$ ans (sol), plus court dans les
%  nuages
%\item {\bf Terre} ($\alpha \simeq 0,1$) : $t_{rad} \sim 2$ mois : cycles saisonniers
%  marqués, diurnes faibles ($\sim 5 K$)
%\item {\bf Mars} ($\alpha = 1$) : $t_{rad} \sim 12$ heures : cycles diurnes et
%  saisonniers marqués
%\end{itemize}
%\end{itemize}
%\end{frame}

