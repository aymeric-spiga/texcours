\sk
On applique ici les raisonnements de la section précédente pour une interface plane liquide/vapeur à une goutte sphérique comme rencontrée dans les brouillards ou les nuages. La réalité est un peu plus complexe et fait intervenir les concepts de noyaux de condensation et de sursaturation, qui ne sont pas abordés dans ce cours. Les raisonnements présentés ci-dessous restent cependant valables au premier ordre.
%%Dans l'atmosphère, loin de la surface, il n'y a pas d'interface liquide/gaz permanente. Si $e<e_{sat}$, il n'y a ni condensation ni évaporation. Si $e$ devient supérieure à $e_{sat}$, il y a condensation sous forme de gouttes d'eau liquide (qui se forment plus vite qu'elles ne s'évaporent). Ces gouttes s'évaporent dès que $e<e_{sat}$

\sk
Soit une parcelle d'air à la température~$T_0$ et à la pression~$P$. Elle contient de la vapeur d'eau en équilibre avec des gouttelettes d'eau en suspension, en pratique cela correspond à une parcelle dans laquelle des gouttelettes nuageuses se sont formées. A l'équilibre liquide-vapeur, la pression partielle de vapeur d'eau dans la parcelle vaut~$e=e\e{sat}(T_0)$, le rapport de mélange de vapeur d'eau vaut~$r=r\e{sat}(T_0)$ et l'humidité~$H$ vaut~$100\%$. Si la température de la parcelle change, il y a déplacement de l'équilibre liquide/vapeur\footnote{Par abus de langage, on dit parfois que \ofg{l'air chaud peut contenir plus de vapeur d'eau que l'air froid}. Il est autorisé de garder cette phrase en tête en tant que moyen mnémotechnique, cependant elle est incorrecte physiquement car elle ne rend pas compte de l'équilibre liquide/vapeur.}.
\begin{finger}
\item
Supposons que l'on \underline{chauffe la parcelle} à une température~$T\e{c}>T_0$. Sa pression partielle en vapeur d'eau~$e$ est toujours proche de~$e\e{sat}(T_0)$, mais la pression de vapeur saturante~$e\e{sat}$ a augmenté de façon exponentielle de~$e\e{sat}(T_0)$ à~$e\e{sat}(T\e{c})$. On est alors dans la situation où~$e < e\e{sat}(T\e{c})$, donc~$H < 1$. Il y a alors évaporation d'eau liquide jusqu'à ce qu'un nouvel équilibre liquide/vapeur soit atteint, où~$e = e\e{sat}(T\e{c})$. Une façon équivalente de décrire ce déplacement d'équilibre est de dire que, lorsque la parcelle chauffe, la quantité de vapeur d'eau~$r=r\e{sat}(T_0)$ devient très inférieure à la quantité de vapeur d'eau à saturation~$r\e{sat}(T\e{c})$. De l'eau liquide doit passer sous forme gazeuse par évaporation pour compenser ce déséquilibre, de manière à ce que la quantité de vapeur d'eau~$r$ dans la parcelle augmente à~$r\e{sat}(T\e{c})$. Ainsi lorsque l'on chauffe la parcelle, des gouttelettes nuageuses disparaissent, le nuage se dissipe.
\item
Supposons à l'inverse que l'on \underline{refroidisse la parcelle} à une température~$T\e{f}<T_0$. La pression de vapeur saturante~$e\e{sat}$ a diminué de façon exponentielle de~$e\e{sat}(T_0)$ à~$e\e{sat}(T\e{f})$. On est alors dans la situation où~$e > e\e{sat}(T\e{f})$, donc~$H > 1$, qui est impossible. Il y a alors condensation d'eau liquide jusqu'à ce qu'un nouvel équilibre liquide/vapeur soit atteint, où~$e = e\e{sat}(T\e{f})$. Autrement dit, lorsque la parcelle refroidit, la quantité de vapeur d'eau~$r=r\e{sat}(T_0)$ devient très supérieure à la quantité de vapeur d'eau à saturation~$r\e{sat}(T\e{f})$. De l'eau sous forme gazeuse doit passer sous forme liquide par condensation pour compenser ce déséquilibre, de manière à ce que la quantité de vapeur d'eau~$r$ dans la parcelle diminue à~$r\e{sat}(T\e{f})$. Ainsi lorsque l'on refroidit la parcelle, de nouvelles gouttelettes nuageuses apparaissent, le nuage s'épaissit.
\end{finger}

\sk
Le second point s'applique également au cas d'une parcelle d'air ne contenant pas initialement de gouttelettes nuageuses. 

\sk
Le chapitre précédent a proposé une expression du premier principe qui distingue deux manières de faire varier la température d'une parcelle atmosphérique d'air sec~: transformations isobares et transformations adiabatiques. On peut désormais illustrer la formation de nuages associée à chacune des transformations appliquée à une parcelle de rapport de mélange en vapeur d'eau~$r \neq 0$ qui reste constant au cours de la transformation.
\begin{finger}
\item Lorsqu'une parcelle d'air proche de la surface subit un refroidissement isobare à la tombée de la nuit, sous l'influence du flux radiatif infrarouge, des gouttelettes nuageuses se forment car le rapport de mélange saturant~$r\e{sat}$ diminue jusqu'à devenir plus faible que~$r$. Il s'agit du brouillard nocturne~; la formation de rosée obéit à un principe similaire. La température de rosée~$T\e{rosée}$ peut ainsi être définie comme la température à laquelle la condensation se produit suite à un refroidissement isobare. 
\item Lorsqu'une parcelle d'air subit une élévation adiabatique, à cause par exemple de la présence d'une montagne, elle se refroidit et le rapport de mélange saturant~$r\e{sat}$ diminue. Le rapport de mélange~$r$ peut alors éventuellement devenir supérieur à~$r\e{sat}$ et des gouttelettes se forment pour que~$r=r\e{sat}$. Ceci explique par exemple que les montagnes soient souvent couvertes de nuages.
\end{finger}
Le chapitre suivant se propose de reprendre avec plus de précisions la formation des nuages.
