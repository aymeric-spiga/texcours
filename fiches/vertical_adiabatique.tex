\sk
Dans un point de vue lagrangien, on peut aisément déterminer
qu'un mouvement vertical ascendant ($w>0$) induit un chauffage adiabatique
et qu'un mouvement vertical descendant ($w<0$) induit un refroidissement adiabatique.
Il suffit de combiner l'équilibre hydrostatique,
ou plutôt sa variante, l'équation hypsométrique
\[ 
\f{\dd p}{p} = -\f{g \dd z}{R\,T} 
\qquad  
\Rightarrow
\qquad
\ddf{p}{t} = - \f{p}{R\,T} \, g \, w
\]
\noindent avec le premier principe dans le cas
adiabatique
\[
\ddf{\theta}{t} = 0
\]
\noindent pour obtenir
\[
\ddf{T}{t} = - \f{g}{c_p} \, w
\]

