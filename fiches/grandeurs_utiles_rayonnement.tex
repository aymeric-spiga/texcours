\sk
La quantité de rayonnement émise par une source de rayonnement, ou reçue par une cible, dépend des paramètres~:
\begin{citemize}
\item longueur d'onde~$\lambda$ ;
\item temps d'exposition~$t$ ;
\item surface~$S$ de l'objet (source ou cible) ;
\item direction dans l'espace considérée, que l'on repère par l'angle~$\beta$ entre le rayonnement incident (ou émis) et la normale à la surface (appelé angle zénithal) ;
\item portion d'espace considérée, exprimée par un \voc{angle solide} $\omega$, l'équivalent bidimensionnel d'un angle\footnote{De même qu'un angle en radians est la longueur d'un arc de cercle divisée par le rayon, le stéradian est la surface d'une portion de sphère divisée par le rayon au carré. On a donc $4\pi$ stéradians sur tout l'espace: voir figure \ref{fig:radiance}} ;
\item propriétés physico-chimiques de l'objet, par exemple sa température (voir section~\ref{corpsnoir}).
\end{citemize}
On exprime cette quantité de rayonnement reçue ou émise sous la forme d'une énergie totale~$E\e{r}$ en Joules (J). Que l'on considère une source ou une cible, $E\e{r}$ est l'énergie transmise par le rayonnement (radiative).

\sk
L'énergie totale~$E\e{r}$ est une grandeur intégrée peu utilisée en pratique en sciences de l'atmosphère. On lui préfère les quantités décrites ci-dessous\footnote{Les noms anglais de ces quantités sont respectivement \emph{radiant flux} pour~$\Phi$, \emph{irradiance} pour~$F$, \emph{radiance} pour~$L$} qui décrivent la quantité de rayonnement émise ou reçue par unité de temps, surface, longueur d'onde, \ldots
\begin{finger}
\item \underline{unité de temps} Le \voc{flux énergétique}~$\Phi$ est l'énergie totale~$E\e{r}$ par unité de temps~$t$ (par seconde) $$ \Phi = \ddf{E\e{r}}{t} $$ C'est une puissance exprimée en Watts (W~$\equiv$~J~s$^{-1}$). Le flux énergétique~$\Phi$ est intégré sur toutes les longueurs d'onde, toutes les directions d'espace et sur l'intégralité de la surface de la source ou cible.
\item \underline{unité de temps + unité de surface} La \voc{densité de flux énergétique}~$F$ est le flux énergétique~$\Phi$ par unité de surface~$S$ de la source/cible $$ \boxed{ F = \ddf{\Phi}{S} } = \f{\dd^2 E\e{r}}{\dd S \, \dd t} $$ C'est un \voc{flux net} exprimé en~W~m$^{-2}$. On l'appelle également \voc{émittance}~$M$ pour une source et \voc{éclairement}~$E$ pour une cible. Cette quantité est intégrée sur toutes les longueurs d'onde et toutes les directions d'espace.
\item \underline{unité de temps + unité de surface + direction fixée} La \voc{luminance énergétique}~$L$ est la densité de flux énergétique dans une direction donnée de l'espace repérée par un angle~$\beta$ $$ L = \f{\dd F}{\cos\beta \, \dd \omega} = \f{\dd^2 \Phi}{\cos\beta \, \dd \omega \, \dd S} = \f{\dd^3 E\e{r}}{\cos\beta \, \dd \omega \, \dd S \, \dd t} $$ 
par unité d'angle solide~$\omega$ (figure \ref{fig:radiance}). C'est une quantité surtout utilisée pour les sources, parfois appelée radiance. La surface considérée~$\sigma$ est perpendiculaire à la direction d'émission~: on la relie à~$S$ par~$\dd \sigma = \cos\beta \, \dd S$. La luminance énergétique~$L$ en~W~m$^{-2}$~sr$^{-1}$ est intégrée sur toutes les longueurs d'onde.
\item \underline{longueur d'onde fixée (grandeurs spectrales)} Le flux énergétique \voc{spectral} ou monochromatique~$\Phi_{\lambda}$ est le flux énergétique~$\Phi$ par unité de longueur d'onde~$\lambda$ $$ \Phi_{\lambda} = \ddf{\Phi}{\lambda} = \f{\dd^2 E\e{r}}{\dd \lambda \, \dd t} $$ Cette quantité en~W~m$^{-1}$ est intégrée sur toutes les directions d'espace et sur toute la surface de la source ou cible. On peut également définir des équivalents spectraux~$F_{\lambda}$ et~$L_{\lambda}$ pour les quantités~$F$ et~$L$ $$ \boxed{F_{\lambda} = \ddf{F}{\lambda}} \qquad\qquad L_{\lambda} = \ddf{L}{\lambda} $$ et des quantités spectrales~$\Phi_{\nu}$, $F_{\nu}$ et $L_{\nu}$ à partir de la fréquence~$\nu$ $$ \Phi_{\nu} = \ddf{\Phi}{\nu} \qquad\qquad F_{\nu} = \ddf{F}{\nu} \qquad\qquad L_{\nu} = \ddf{L}{\nu} $$ Les quantités spectrales définies à partir de la fréquence sont parfois plus avantageuses, dans la mesure où la fréquence est indépendante du milieu matériel transparent où l'onde matérielle se propage\footnote{La longueur d'onde~$\lambda$ dépend de l'indice de réfraction~$n$ du milieu (pour l'air, $n$ est proche de~$1$) et de la longueur d'onde dans le vide~$\lambda_0$.}. Attention, les unités des quantités spectrales dépendent de la quantité référente : ainsi $L_{\lambda}$ est en~W~m$^{-3}$~sr$^{-1}$ et $L_{\lambda}$ est en~W~m$^{-2}$~sr$^{-1}$~s.
\end{finger}

\figun{0.6}{0.15}{\figfrancis/lum_emit.pdf}{Schéma montrant l'émittance~$M$ et la luminance~$L$ d'un élément de surface $dS$ d'une source. $M$ est l'intégrale du flux dans toutes les directions. $L$ est le flux émis dans une certaine direction par unité de surface perpendiculaire.}{fig:luminance}

\begin{figure} \begin{center} \input{\figfrancis/luminance.pdftex_t} \end{center} \caption{\emph{ Schéma en coordonnées sphériques de la luminance $L$ de l'élément de surface $dS$ d'une source située dans un plan $(Oxy)$. La luminance est définie pour chaque direction repérée par les angles $\theta$ et $\varphi$. L'angle solide élémentaire autour d'une direction donnée vaut $\dd \omega = \sin\theta \, \dd\theta \, \dd\varphi$ (rapport entre surface hachurée et $r^2$). 
%La relation avec le flux énérgétique~$\Phi$ émis par la source est $L = \dd^2 \Phi / \left( \dd\omega \, \dd S \, \cos\theta \right)$.
}} \label{fig:radiance} \end{figure}

\sk
\subsubsection{Exemples d'application}

\sk
Dans ce qui précède, on part de la quantité la plus intégrée possible, à savoir l'énergie totale~$E\e{r}$, pour parvenir par dérivation à des quantités moins complexes à appréhender en pratique. Le chemin inverse se fait par intégration (sommation continue). On considère ici quelques exemples illustratifs. 
\begin{finger}
\item Pour reprendre la situation de la figure~\ref{fig:luminance}, supposons que l'on dispose d'informations sur la luminance énergétique~$L$ d'une source plane de surface élémentaire~$\dd S$, c'est-à-dire la quantité de rayonnement émise dans chaque direction de l'espace. Afin de connaître le flux net~$F$ dans tout l'espace (que l'on peut noter également émittance~$M$ puisque l'on considère une source), il suffit de l'écrire comme une intégrale de la luminance sur toutes les directions d'un demi-espace: $$ F = \int_{2\,\pi} L \, \cos\beta \, \dd \omega $$ où $2\,\pi$ représente l'intégration sur un demi-espace. %Deux cas particuliers sont intéressants. Dans la limite d'un rayonnement rasant~$\beta=\pi/2$, la contribution au flux net est nulle. Par ailleurs, s
Si la luminance~$L$ est indépendante de la direction, c'est-à-dire que le rayonnement est \voc{isotrope}, l'intégration donne simplement~$F=\pi L$. Dans ce cas on parle d'une \voc{source lambertienne}.
%C'est le cas d'un réflecteur de Lambert.
%Jean-Henri Lambert (1728-1777) a observé que l’énergie émise par certaines sources (parmi toutes les types de sources à sa disposition) anisotropes diminue comme le cosinus de l'angle θ, autour de la direction perpendiculaire à la surface de la source. Cette variation de l'énergie émise est observée lorsque nous mesurons l'énergie thermique rayonnée par un orifice percé dans un four (ce qui nous ramène au corps noir défini plus loin), isolé thermiquement et dont la température interne est supérieure à la température externe. Dans ce contexte, l'orifice est appelé un émetteur Lambert et ne balaye un espace que de stéradian. Une source obéit à la loi de Lambert si l’énergie rayonnée depuis un point de cette source est la même dans toutes les directions (on dit que son intensité est isotrope et donc indépendante de l'angle d’où on observe cette source). Soit M la valeur de l’éclairement mesurée par un capteur. On peut facilement en déduire le flux énergétique de la source de surface S : Φ =S M
\item Suppons que l'on dispose cette fois d'informations sur la densité de flux énergétique~$F$ d'une source sphérique de rayon~$R$ (par exemple, le Soleil). Puisque l'on considère une source, $F$ peut également être appelée émittance et être notée~$M$. Si le rayonnement est \voc{uniforme}, c'est-à-dire qu'en chaque point la source émet le même flux énergétique~$\Phi$ par unité de surface~$\dd S$, alors on dispose de la relation suivante $$ \Phi = 4\,\pi\,R^2 \, M$$ et plus généralement pour une surface~$S$ qui est une source uniforme de rayonnement $$ \boxed{ \Phi = S \, M } $$ On notera qu'on fait les calculs en considérant seulement le côté extérieur de la surface, celui d'où nous regardons la source, car seule la moitié de l'énergie échangée par les points de la surface~$S$ est émise sous forme de rayonnement. L'autre moitié est échangée du côté intérieur de la surface avec le milieu constituant le corps. 
\item L'énergie transmise par le rayonnement, et toutes les grandeurs définies précédemment, varient généralement beaucoup avec la longueur d'onde étudiée. Supposons que l'on connaisse la luminance spectrale~$L_\lambda$ dans un petit intervalle~$\dd \lambda$ autour de la longueur d'onde~$\lambda$, et ce, pour toutes les longueurs d'onde~$\lambda$ du spectre électromagnétique. Par exemple, la luminance totale~$L$ est retrouvée par intégration des longueurs d'onde les plus courtes aux plus longues : $$L=\int_{\lambda} L_\lambda \, \dd\lambda = \int_\nu L_\nu \, \dd \nu $$ Il s'agit d'une des formes du \voc{principe de superposition}, qui indique que le rayonnement électromagnétiques se compose d'une superposition d'ondes monochromatiques. Une intégration similaire est effectuée par l'électronique embarquée dans un appareil photo qui traite les flux reçus par les capteurs dans une variété de longueurs d'onde en domaine visible, afin d'obtenir une image finale qui intègre toutes ces informations.
\end{finger}

