\sk
D'après les seules équations thermodynamiques, on peut trouver une loi simple des variations de température avec l'altitude pour une parcelle qui ne subit que des transformations adiabatiques. Considérons le cas d'une parcelle subissant un déplacement vertical quasi-statique et adiabatique tel que~$\delta q = 0$. Elle vérifie en première approximation l'équilibre hydrostatique~$\dd P\e{p} / \rho = - g \, \dd z$. L'équation du premier principe modifiée pour le cas atmosphérique indique alors que
\[  \dd T\e{p}  = - \frac{g}{C_P} \, \dd z \]
d'où on tire le profil vertical adopté dans l'atmosphère sèche par une parcelle ne subissant pas d'échange de chaleur avec l'extérieur
\[  \boxed{ \ddf{T\e{p}}{z}  = \Gamma\e{sec} \qquad \text{avec} \qquad \Gamma\e{sec} = \frac{-g}{C_P} } \]
On note qu'il ne s'agit pas nécessairement du profil vertical suivi par l'environnement (voir section~\ref{parcenv}).

\sk
Le résultat trouvé ci-dessus revêt une importance particulière en sciences de l'atmosphère. La température d'une parcelle en ascension adiabatique décroît avec l'altitude selon un taux de variation constant, indépendamment des effets de pression. La constante~$\Gamma\e{sec}$ est appelée le \voc{gradient adiabatique sec} de température. Il n'est valable que pour une parcelle d'air non saturée en vapeur d'eau. Le calcul pour la Terre donne un refroidissement de l'ordre de~$10^{\circ}$C/km (ou K/km). 

\sk
Pourquoi cette valeur est-elle en désaccord avec la décroissance de~$6.5^{\circ}$C/km effectivement constatée dans l'atmosphère terrestre~? Cet écart est relatif aux processus humides qui ont une grande importance dans l'atmosphère terrestre.
%Le chapitre suivant apporte des éléments de réponse à ce paradoxe apparent.



