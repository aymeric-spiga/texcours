\sk
Depuis son invention en 1643 par Torricelli, disciple de Galilée, le \voc{baromètre} est l'instrument de référence pour mesurer la pression atmosphérique à la surface de l'atmosphère terrestre\footnote{Le but initial de Torricelli était de parvenir le premier à maintenir artificiellement en laboratoire une chambre sous vide. Néanmoins, il est également reporté que l'invention du baromètre découle des réflexions de Torricelli autour de l'impossibilité, constatée en pratique, de pomper l'eau d'un fleuve au-dessus d'un certain niveau.}. Trois ans après son invention, le baromètre était déjà utilisé pour sa première application en sciences de l'atmosphère~: donner une preuve expérimentale de l'équilibre hydrostatique qui gouverne la stratification en pression de l'atmosphère. Autrement dit, le baromètre est un moyen indirect de mesurer la masse de l'atmosphère à travers la pression de surface. Blaise Pascal montra ainsi, par des mesures respectivement sur la Tour Saint-Jacques à Paris ($ \Delta z = 52 \U{m} $ au-dessus du sol) et sur le Puy-de-Dôme en Auvergne ($ \Delta z = 1000 \U{m} $ au-dessus du sol), que la pression atmosphérique varie avec l'altitude\footnote{Le texte original du traité de Pascal est disponible sur Gallica~: \url{http://gallica.bnf.fr/ark:/12148/bpt6k105083f}}.

\sk
L'équation hypsométrique permet de retrouver l'écart relatif en pression mesuré par Blaise Pascal entre le sol et le haut de la Tour Saint-Jacques. Comme les variations mesurées sont petites, on peut les assimiler aux différentielles et on peut négliger les variations de température avec l'altitude. Les variations relatives de pression mesurées par Pascal peuvent alors s'écrire
\[ \f{\Delta P}{P} \simeq \f{g}{R\,T} \, \Delta z \]
L'application numérique avec~$T = 288$~K donne une variation~$\Delta P / P = 0.62 \%$. La variation de pression détectée par Blaise Pascal\footnote{On note que, par une heureuse coincidence, la variation de pression entre le pied et le sommet de la Tour Saint-Jacques est de l'ordre de grandeur de la pression atmosphérique à la surface de Mars, ce qui permet de se représenter la finesse de l'atmosphère sur cette planète, comparé à notre Terre.} est donc d'environ~$6$~hPa (avec la valeur standard de la pression de surface~$P_0 = 1013$~hPa). Bien que cette baisse de pression soit détectable à l'aide du tube de Torricelli, Blaise Pascal a reproduit l'expérience au Puy-de-Dôme avec un écart~$\Delta z$ plus grand pour une meilleure précision quantitative. 

