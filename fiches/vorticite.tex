\sk
La vorticité est définie généralement par le rotationnel du champ de vent
$\v \zeta = \v \nabla \wedge \v V$. La composante verticale de la vorticité~$\zeta$
quantifie la rotation du fluide dans le plan horizontal
\[ 
\zeta = \v k \cdot \left( \v \nabla \wedge \v u \right) 
\qquad\qquad \Rightarrow \qquad\qquad
\zeta = \Dp{v}{x} - \Dp{u}{y}
\]

\sk
Considérons les équations quasi-géostrophiques pour le mouvement horizontal,
en négligeant les termes de sphéricité pour plus de simplicité.
\[ \ddf{u}{t} - f\,v = -\frac{1}{\rho} \Dp{p}{x} \]
\[ \ddf{v}{t} + f\,u = -\frac{1}{\rho} \Dp{p}{y} \]
\noindent L'évolution 
de la vorticité relative 
selon la verticale
peut être obtenue simplement en
dérivant selon~$y$ l'équation du mouvement selon~$x$
(et vice versa) puis en soustrayant les deux termes.
L'on obtient alors l'\voc{équation de la vorticité}
\[
\ddf{\zeta}{t} + (\zeta+f) \, \v \nabla \cdot \vec v + \textcolor{red}{v \ddf{f}{y}}
= \frac{1}{\rho^2} \left( \Dp{\rho}{x} \Dp{p}{y} - \Dp{\rho}{y} \Dp{p}{x} \right)
\]
\noindent en notant que l'on a négligé les termes
$\Dp{w}{x}$ et $\Dp{w}{y}$ en se limitant au mouvement quasi-horizontal.
Le terme en \textcolor{red}{rouge} est en réalité égal à~$\ddf{f}{t}$
et correspond à l'\voc{effet~$\beta$} c'est-à-dire
l'effet sur l'écoulement de la variation avec la
latitude du paramètre de Coriolis.
Ainsi on obtient la notation plus compacte
\[
\ddf{(\zeta + f)}{t} + (\zeta+f) \, \v \nabla \cdot \vec v
= \frac{1}{\rho^2} \left( \Dp{\rho}{x} \Dp{p}{y} - \Dp{\rho}{y} \Dp{p}{x} \right)
\]
\noindent Le terme de droite est la projection verticale
du terme $\v \nabla p \wedge \v \nabla \rho$
appelé solénoïdal ou, plus classiquement, terme de production barocline.
Un fluide est appelé \voc{barocline} lorsque
ses surfaces d'égale masse volumique
ne sont pas alignées avec ses surfaces d'égale pression.
En pratique, un fluide est barocline
lorsqu'il y a des variations verticales de température
et de vent; sinon il est appelé \voc{barotrope}.


\sk
Un écoulement barotrope peut être considéré comme bi-dimensionnel (quasi-horizontal).
Alors l'équation de la vorticité s'écrit plus simplement
\[
\ddf{\xi}{t}
=
\Dp{\xi}{t} + u \Dp{\xi}{x} + v \Dp{\xi}{y}
=
- \xi \, \left( \Dp{u}{x} + \Dp{v}{y} \right)
\]
\noindent avec la vorticité absolue~$\xi = \zeta + f$
(rappelons qu'il s'agit de composantes verticales). 
Cette quantité se comprend comme une \og composée de vorticité \fg~entre
vorticité relative~$\zeta$ (calculée via le mouvement dans le référentiel tournant)
et vorticité planétaire~$f = 2\,\Omega\,\sin\phi$ (correspondant à la rotation de la planète).
Dans le cas d'un écoulement non-divergent, on obtient alors
\[
\ddf{\xi}{t} = 0
\qquad
\Rightarrow
\qquad
\Dp{\xi}{t} + u \Dp{\xi}{x} + v \Dp{\xi}{y}
=
0
\]
\noindent Il y a donc conservation de la vorticité totale~$\xi$
dans le cas d'un écoulement barotrope non-divergent.
Ce modèle très simplifié de l'évolution du fluide atmosphérique
est à la base des premiers modèles de prévision numérique du temps.
Par ailleurs, ce modèle très simple permet d'illustrer aisément
l'apparition d'ondes de Rossby dans l'écoulement 
sous l'effet de la conservation de la vorticité.

\sk
L'approche peut se généraliser au cas barocline, 
moyennant des calculs plus élaborés.
La quantité qui se conserve dans le cas général
est la \voc{vorticité potentielle d'Ertel} notée~$PV$
\[
PV = \left( \zeta_{\theta} + f \right) \left( -g \Dp{\theta}{p} \right)
\]
\noindent avec~$\zeta_{\theta}$ vorticité relative
calculée sur une surface isentrope (à~$\theta = \cte$).

