\sk
\subsubsection{Influence de la latitude}

\sk
Localement, l'éclairement varie suivant la latitude et la saison, en plus de l'alternance jour/nuit: il est proportionnel à $\cos\theta$ où $\theta$ est l'angle d'incidence avec la surface.
%[figure~\ref{fig:senslat1}]. 
En moyenne annuelle, le maximum d'ensoleillement est donc aux latitudes tropicales, mais il varie au cours de l'année et est même maximal aux pôles pendant l'été local [figure~\ref{fig:senslat2}]~: la durée du jour de 24h fait plus que compenser l'angle d'incidence réduit dû à la latitude élevée (ce qui peut paraître de prime abord contre-intuitif).
%\figside{0.3}{0.1}{\figfrancis/swcoslat.jpg}{Schéma de la relation entre densité de flux du rayonnement incident parallèle et éclairement de la surface suivant l'angle d'incidence.}{fig:senslat1}
\figside{0.65}{0.25}{\figfrancis/swtoaseas}{Cycle saisonnier de l'éclairement dû au rayonnement solaire incident au sommet de l'atmosphère.}{fig:senslat2}
%%%% http://www.energieplus-lesite.be/energieplus/page_16761.htm

\sk
\subsubsection{Rôle des nuages}

\sk
La présence de différents types de nuages est très variable, à la fois géographiquement et dans le temps. Ils ont pourtant une influence très grande sur le bilan radiatif, par deux mécanismes distincts [figure \ref{fig:schemacrf}].
\begin{finger}
\item Effet d'albédo~: les nuages réfléchissent une partie importante du rayonnement solaire incident (par rétro\-diffusion par les gouttes d'eau). Cet effet est d'autant plus fort que le nuage contient d'eau et que les gouttes sont fines. Un nuage très réfléchissant apparaitra sombre vu d'en dessous. Au total, les nuages sont responsables des 2/3 de l'albédo planétaire.
\item Effet de serre~: Les gouttes d'eau (ou la glace) des nuages sont d'excellents absorbants dans l'infrarouge. Un nuage même peu épais absorbe donc très rapidement tout le rayonnement infrarouge provenant des couches plus basses. Il émet lui même vers le haut du rayonnement suivant sa propre température: $\sigma T_N^4$ où $T_N$ est la température au sommet du nuage. Un nuage au sommet élevé (donc froid) aura donc un effet de serre très important.
\end{finger}
Au final, l'effet d'albédo l'emporte pour les nuages bas (type stratus), qui sont typiquement épais (albédo élevé) et dont le sommet est chaud. Au contraire, les fins nuages d'altitude (cirrus) ont un albédo faible mais un sommet très froid donc ont un effet net réchauffant. Pour les nuages de type orageux, qui sont épais avec un sommet froid, les deux effets tendent à se compenser.
\figside{0.6}{0.2}{\figfrancis/schema_crf}{Schema de l'influence des nuages sur le bilan radiatif: effet d'albédo dans le visible (jaune), et absorption et émission dans l'infrarouge (rouge). L'effet de serre vient du rayonnement émis vers l'espace plus faible que celui venant de la surface, qui est absorbé.}{fig:schemacrf}
%% nuages comme les lunettes dans la caméra infrarouge. faire également référence à la vidéo tirée du satellite.

