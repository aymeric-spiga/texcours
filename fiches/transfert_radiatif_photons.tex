\sk
Le rayonnement électromagnétique, lors de sa traversée de l'atmosphère terrestre, est perturbé par deux processus que l'on peut analyser séparément~: un processus d'\voc{absorption} par certains gaz atmosphériques (O$_2$, H$_2$O, O$_3$, etc.) et un processus de \voc{diffusion} par les molécules et les aérosols (poussières, cristaux de glace, gouttes nuageuses et gouttes de pluie). 
\begin{finger}
\item Dans le processus d'absorption, un certain nombre de photons disparaissent, une partie du rayonnement incident est convertie en énergie interne, et il y a une extinction du signal dans la direction de propagation. 
\item Au contraire, dans le processus de diffusion, les photons sont simplement redistribués dans toutes les directions avec une certaine probabilité définie par ce qu'on appelle la fonction de phase de diffusion~; on peut alors observer une extinction dans certaines directions et une augmentation dans d'autres. Lors de la diffusion, il n'y a pas de changement de longueur d'onde de l'onde incidente et de l'onde diffusée.
\end{finger}
%Ainsi la réflexion peut être \voc{diffuse} (dans toutes les directions), \voc{spéculaire} (dans la direction symétrique du rayonnement incident) ou quelconque. 
%Les deux effets peuvent être analysés séparément.
Autrement dit, on s'intéresse à l'\voc{extinction} progressive du rayonnement incident par absorption et diffusion. Deux phénomènes très importants sont mis de côté~: l'émission de rayonnement thermique, déjà abordée au chapitre précédent, et la diffusion multiple, que l'on néglige (rayonnement diffusé qui viendrait depuis d'autres directions).
