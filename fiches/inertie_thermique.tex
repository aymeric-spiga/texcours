L'inertie thermique $I$ mesure la résistance
thermique d'un milieu à un apport ou un
déficit de chaleur.
%
L'expression de $I$ (J~m$^{-2}$~s$^{-1/2}$~K$^{-1}$) 
s'obtient en déduisant d'une équation
simple de conduction thermique de Fourier,
par analyse dimensionnelle, l'épaisseur
de peau thermique $\delta$ 
%
\[
\rho \, c_p \, \Dp{T}{t} = \Dp{}{x} \left( \lambda \, \Dp{T}{x} \right)
\quad
\to
\quad
\delta = \sqrt{\f{\lambda \, \tau}{\rho \, c_p}}
\]
%
\noindent (où $\tau$ est une constante caractéristique de temps)
ce qui permet de mettre en évidence l'inertie
thermique dans le terme de flux de chaleur $\phi\e{c}$
à la surface
%
\[
\phi\e{c} = - \lambda \, \Dp{T}{x} = - \f{\lambda}{\delta} \, \Dp{T}{x'} 
= - \sqrt{\f{\lambda \, \rho \, c_p}{\tau}} \, \Dp{T}{x'}
\quad 
\textrm{avec}
\quad
x'=x/\delta 
\]
\noindent en ne retenant que les termes qui dépendent du milieu
dans la caractérisation du flux de chaleur :
%
\[
I = \sqrt{\lambda \, \rho \, c_p}
\]


Un milieu est donc de faible inertie thermique
lorsqu'il ne peut stocker que de petites quantités de chaleur
(faible capacité calorifique $c_p$)
et/ou qu'il ne peut transmettre cette chaleur que dans ses couches superficielles
(faible conductivité thermique $\lambda$).
%
Les océans terrestres constituent un exemple 
bien connu de milieu à très forte inertie thermique, 
de par leur grande capacité calorifique.
%
Autre exemple bien connu, 
l'inertie thermique des terrains rocheux
martiens est plus élevée que 
l'inertie thermique des terrains
poussiéreux, principalement
pour des raison de conductivité thermique.
%
L'inertie thermique
peut d'ailleurs permettre 
sous certaines conditions d'estimer
la taille des grains dans les sols
non consolidés.

Dépourvue d'océans, Mars forme
un gigantesque désert de faible inertie
thermique : $I$~dépasse 
rarement $400\U{J~m^{-2}~s^{-1/2}~K^{-1}}$
pour la plupart des sols martiens.
%
Pour qualifier les grands
ensembles sur le champ d'inertie
thermique planétaire,
le terme de \ofg{continents thermiques}
est parfois employé.
%
L'inertie thermique n'est pas une
quantité observable directement
et sa détermination requiert la combinaison
de mesures de température de surface et
d'un modèle simulant les variations thermiques du sol.
