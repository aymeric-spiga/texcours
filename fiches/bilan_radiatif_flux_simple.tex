\sk
Nous pouvons exprimer le rayonnement reçu du Soleil par la Terre par une densité de flux énergétique moyenne~$F\e{reçu}$ en W~m$^{-2}$ ou un flux énergétique~$\Phi\e{reçu}$ (en W)
\[ 
F\e{reçu} = (1-A\e{b}) \, \mathcal{F}\e{s}' 
\qquad \qquad
\Phi\e{reçu} = \pi \, R^2 \, (1-A\e{b}) \, \mathcal{F}\e{s}
\] 
La partie du rayonnement reçue du soleil qui est réfléchie vers l'espace est prise en compte via l'albédo planétaire noté~$A\e{b}$.

\sk
Par ailleurs, le système Terre émet également du rayonnement principalement dans les longueurs d'onde infrarouge [figure \ref{fig:eqrad2}]. 
Cette quantité de rayonnement émise au sommet de l'atmosphère radiative est notée $OLR$ pour \emph{Outgoing Longwave Radiation} en anglais.
A l'équilibre, la planète Terre doit émettre vers l'espace autant d'énergie qu'elle en reçoit du Soleil, donc
on obtient la relation générale appelée \emph{TOA} pour \emph{Top-Of-Atmosphere} en anglais, correspondant
au bilan radiatif au sommet de l'atmosphère
\[ 
\boxed{OLR = (1-A\e{b}) \, \mathcal{F}\e{s}'}
\] 
La principale difficulté qui sous-tend les divers modèles pouvant être proposés réside dans l'expression du terme~$OLR$.


