\sk
La présence d'une surface et une hypothèse d'équilibre radiatif imposent donc que le chauffage de la surface conduit inévitablement à de la convection. Que se passe-t-il sur les planètes géantes dépourvues de surface ? L'équilibre radiatif y prévaut également, car à partir d'une certaine profondeur, le gradient radiatif est instable -- et ce, même en l'absence d'une surface qui absorbe le rayonnement solaire. Pour formuler la stabilité de l'équilibre radiatif, on calcule le profil~$\dd T / \dd p$ et on le compare au gradient adiabatique sec ou humide dans l'atmosphère considérée. En dérivant le profil radiatif obtenu dans le cas du modèle à deux faisceaux
\[ T(\tau) = \sqrt[4]{\frac{OLR\,(1+\tau)}{2\,\sigma\,\epsilon}} \]
\noindent par rapport à l'épaisseur optique~$\tau$, nous obtenons
\[ 8 \, \sigma \, T^3 \, \ddf{T}{\tau} = OLR \]
\noindent soit, en utilisant~$\ddf{~}{p} = \ddf{\tau}{p} \ddf{~}{\tau}$
\[ \ddf{T}{\ln p} = \frac{1}{4\,(1+\tau)} \, p \, \ddf{\tau}{p} \]
Ainsi la stabilité de la couche s'écrit
\[ \frac{R}{c_p} \ge \frac{1}{4\,(1+\tau)} \, p \, \ddf{\tau}{p} \]

\sk
Le terme en~$p$ dans ce qui précède guarantit (à moins d'une variation énorme de $\ddf{\tau}{p}$ en~$1/p$ ou plus rapide quand~$p \rightarrow 0$) que les hautes atmosphères planétaires sont toujours stables. De plus, les atmosphères optiquement fines sont toujours stables sur l'intégralité de leur épaisseur, puisque~$-p \, \ddf{\tau}{p} < \tau_\infty \ll 1$. 

\sk
Dans le cas d'un coefficient d'absorption~$\kappa$ constant, nous pouvons même écrire la condition de stabilité
\[ \frac{R}{c_p} \ge \frac{\tau}{4\,(1+\tau)} \]
%% si l'absorption est constante p \, \ddf{\tau}{p} = - \kappa \, p / g costheta
\noindent Le critère de stabilité est en pratique un peu plus complexe dans les atmosphères réelles qu'indiqué ici. Bien sûr, $\tau$ et~$\kappa$ varient avec la longueur d'onde~$\lambda$ (limitation inhérente au modèle à deux faisceaux), mais surtout le coefficient d'absorption~$\kappa$ augmente avec la pression (donc la profondeur) en raison de l'élargissement collisionnel (\emph{collisional broadening}), efficace à partir de quelques bars. La loi de variation d'élargissement collisionnel peut s'écrire~$\kappa(p) = \kappa(p\e{s}) \, \frac{p}{p\e{s}}$. Les processus de changements d'état sont également susceptibles de rendre la situation plus complexe que le calcul proposé ici.
