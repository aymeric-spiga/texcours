\sk
L'équation de base pour le mouvement de masses d'air est la relation fondamentale de la dynamique $\Sigma \v F=m \, \v a$ (seconde loi de Newton).  Cette relation est cependant valable dans un référentiel galiléen, tel le référentiel fixe. On s'intéresse plutôt au vent, c'est-à-dire que l'on souhaite considérer des mouvements atmosphériques par rapport à la surface de la Terre qui est en rotation autour de l'axe des pôles. On va donc dans un premier temps projeter l'accélération dans le référentiel tournant, puis étudier les principales forces horizontales. Autrement dit, on se donne pour objectif d'exprimer l'accélération dans le référentiel tournant, qu'on souhaite connaître, en fonction de l'accélération dans le référentiel fixe, qui est égale à la somme des forces.

\sk
La relation entre vitesse absolue~$\v V_a$ dans le référentiel fixe et vitesse relative~$\v V_r$ dans le référentiel tournant s'écrit, avec le vecteur de rotation~$\v \Omega$ de module~$\Omega$ dirigé selon l'axe des pôles~:
\[\v V_a = \v V_r + \vl{\Omega}\wedge\vl{CM}\]
Il s'agit de la relation de composition des vitesses pour un référentiel tournant. Le terme $\vl{\Omega}\wedge\vl{CM}$ est la vitesse d'un point fixe par rapport au sol ($\v V_r=0$), il est appelé \voc{vitesse d'entrainement}.
%La relation entre la dérivée temporelle d'un vecteur \v X dans le référentiel fixe (\emph{absolue}, $a$) et celle dans le référentiel tournant (\emph{relative}, $r$) s'écrit \[\frac{d\v X}{dt}_{|a}=\frac{d\v X}{dt}_{|r}+ \vl \Omega\wedge \v X\] En applicant au vecteur \vl{CM}, avec $\frac{d\vl{CM}}{dt}=\v V$, on a: \[\v V_a=\v V_r+\vl{\Omega}\wedge\vl{CM}\]
La relation entre accélération absolue~$\v a_a$, égale à la somme des forces, et accélération relative~$\v V_r$ dans le référentiel tournant s'écrit
\[ \v a_a=\Sigma\v F=\v a_r+2\vl{\Omega}\wedge\v V_r-\Omega^2\,\vl{HM} \]
Le premier terme est l'accélération relative~$\v a_r$, le deuxième l'\voc{accélération de Coriolis}~$\v a_c$, le troisième est l'\voc{accélération d'entrainement}~$\v a_e$. Les termes de Coriolis et d'entraînement induisent des \voc{forces apparentes}~$\v F_c = -m \, \v a_c$ et~$\v F_e = -m \, \v a_e$ dans le référentiel tournant. On parle de forces apparentes car du point de vue du référentiel fixe, ces termes n'apparaissent pas comme des forces~: ils ne sont que des termes d'accélération causés par le caractère non galiléen du référentiel tournant.
%En dérivant à nouveau $\v V_a$, on obtient: \[\v a_a=\left(\frac{d \v V_r}{dt}_{|r}+\vl{\Omega}\wedge\v V_r\right)+\vl{\Omega}\wedge \left(\v V_r+\vl{\Omega}\wedge\vl{CM}\right)\] soit en regroupant et avec $\vl{\Omega}\wedge(\vl{\Omega}\wedge\vl{CM})=-\Omega^2\cdot\vl{HM}$:


