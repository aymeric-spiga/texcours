\sk
Le modèle à deux faisceaux est un bon compromis entre simplicité
et illustration de concepts importants. Il est une version simplifiée
de l'équation de Schwarzschild du transfert radiatif.
Ce modèle entend élucider
les transferts de rayonnement dans l'infrarouge entre
les couches qui composent la colonne atmosphérique. Les
hypothèses simplificatrices suivantes sont réalisées
\begin{citemize}
\item couches atmosphériques plan-parallèle (sphéricité négligée)
\item phénomènes d'absorption négligés dans le visible (transparence au visible)
\item phénomènes de diffusion (\emph{scattering}) négligés dans l'infrarouge
\item \emph{gray gas} dans l'infra-rouge : on considère que le coefficient d'absorption
du gaz est indépendant de la longueur d'onde~$\lambda$ ($k_{\lambda} = k$ pour tout~$\lambda$),
ce qui implique une hypothèse similaire pour l'épaisseur optique ($\tau_{\lambda} = \tau$ pour tout~$\lambda$).
\end{citemize}
En d'autres termes, on se cantonne dans ce modèle à deux types de phénomènes
\begin{enumerate}
\item Un faisceau de rayonnement infra-rouge de flux~$F$ traversant une couche 
atmosphérique donnée
subit une extinction à cause de l'absorption selon une loi de type Beer-Lambert
\[
\dd F = - F \dd \tau
\]
avec~$\tau$ l'épaisseur optique \emph{gray gas} 
dans l'infra-rouge.
\item Une couche atmosphérique émet un flux de rayonnement thermique~$M$ 
calculé par la loi intégrée de Stefan-Boltzmann ($\SB$)
puisque la majorité de l'émittance est émise dans l'infrarouge pour les températures considérées.
\end{enumerate}

\sk
L'épaisseur optique~$\tau$ peut servir de coordonnée verticale à la place de~$z$
en utilisant la relation entre les deux quantités. La couche atmosphérique
élémentaire considérée est ainsi d'épaisseur~$\dd\tau$ et située à une coordonnée
verticale~$\tau$ qui croît avec l'altitude. 

\sk
Si l'on considère un faisceau ascendant~$F^+(\tau)$ au bas de la couche considérée,
une fois la couche traversée son amplitude est
\[
F^+(\tau) - F^+(\tau) \dd\tau
\]
A ce flux au sommet de la couche, il convient d'ajouter
la contribution de l'émission thermique de la couche vers 
le haut, à savoir~$M\,\dd\tau$.
Le flux total ascendant au sommet de la couche est donc
\[
F^+(\tau+\dd\tau) = F^+(\tau) - F^+(\tau) \dd\tau + M\dd\tau
\]

\sk
Même raisonnement avec le flux descendant~$F^-(\tau+\dd\tau)$ au sommet de la couche considérée,
une fois la couche traversée son amplitude est~$F^-(\tau+\dd\tau) - F^-(\tau) \dd\tau$, où 
l'approximation du terme du second ordre~$F^-(\tau+\dd\tau) \dd\tau \simeq F^-(\tau) \dd\tau$
a été effectuée.
Le flux total descendant au bas de la couche est donc
\[
F^-(\tau+\dd\tau) = F^-(\tau) - F^-(\tau) \dd\tau + M\dd\tau
\]

\sk
Les deux résultats qui précèdent peuvent être transformés 
afin de faire apparaître une dérivée
en utilisant le théorème des accroissements finis
\[
\ddf{F^+}{\tau} = \frac{F^+(\tau+\dd\tau) - F^+(\tau)}{\dd\tau}
\]
\noindent ce qui permet d'obtenir au final
\[
\ddf{F^+}{\tau} = - F^+(\tau) + \epsilon\,\sigma\,T(\tau)^4 \quad [S^+]
\qquad\qquad 
\ddf{F^-}{\tau} = F^-(\tau) - \epsilon\,\sigma\,T(\tau)^4 \quad [S^-]
\]
\noindent $[S^+]$ et~$[S^-]$ sont parfois appelées les relations de Schwarzschild à deux faisceaux.
Il s'agit d'une version très simplifiée des équations de Schwarzschild du transfert radiatif.

\sk
Si l'on souhaite adopter la convention~$\tau=0$ au sommet de l'atmosphère,
et $\tau=\tau_{\infty}$ à la surface en $z=0$, donc adopter un axe
vertical d'épaisseur optique avec~$\tau$ croissant de haut en bas, il
suffit de remplacer~$\tau$ par~$\tau_{\infty}-\tau$ dans les équations précédentes pour obtenir
\[
\boxed{\ddf{F^+}{\tau} = F^+(\tau) - \epsilon\,\sigma\,T(\tau)^4 \quad [S^+]} 
\qquad\qquad 
\boxed{\ddf{F^-}{\tau} = -F^-(\tau) + \epsilon\,\sigma\,T(\tau)^4 \quad [S^-]}
\]







