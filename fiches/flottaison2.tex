\sk
La force de flottaison peut s'écrire de diverses manières,
(p désignant la parcelle et e l'environnement)
\[ F_z = g \, \frac{\rho\e{e}-\rho\e{p}}{\rho\e{p}} = g \, \frac{T\e{p}-T\e{e}}{T\e{e}}  \]
\noindent ou encore
\[ F_z = g \, \frac{\Gamma\e{p}-\Gamma\e{e}}{T\e{e}} \, \delta z \]

\sk
Si l'on se place dans un contexte d'une perturbation d'altitude~$z'$
pour la parcelle
\[ T' = T_0 - \Gamma_d \, z' \]
alors que pour l'environnement
\[ T = T_0 - \Gamma \, z' \qquad \textrm{avec} \qquad \Gamma = \ddf{T}{z} \]
\noindent D'après les expressions du paragraphe précédent,
on peut ainsi écrire l'effet de la force ascensionnelle
sur l'accélération comme
\[ \ddf{^2 z'}{t^2} = g \left( \f{T'}{T} - 1 \right) \]
\noindent ou encore
\[ 
\ddf{^2 z'}{t^2} + \left[ \f{g}{T} \, \left( \ddf{T}{z} + \f{g}{c_p} \right)  \right] \, z' = 0
\]
\noindent Nous avons donc un système qui
peut causer des oscillations de la parcelle sous l'effet de la force de flottaison.
Par analogie avec l'équation du second ordre d'un
oscillateur harmonique, on définit le terme en
facteur de $z'$ comme une fréquence au carré, 
nommée fréquence de Brunt-V{\"a}is{\"a}l{\"a}~$N$
exprimée comme
\[ N^2 = \f{g}{T} \, \left( \ddf{T}{z} + \f{g}{c_p} \right) \]
\noindent ou encore en utilisant la température potentielle
\[ N^2 = \f{g}{\theta} \, \ddf{\theta}{z} = g \, \ddf{\ln\theta}{z} \]
\noindent La fréquence de Brunt-V{\"a}is{\"a}l{\"a}~$N$
traduit l'instabilité convective si~$N<0$
et la stabilité convective si~$N>0$ -- avec dans ce cas,
les oscillations de la parcelle comme mécanisme de rappel.

\sk
Nous voyons ici un cas très particulier d'un effet plus général appelé onde de gravité.
