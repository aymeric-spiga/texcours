\sk
Dans de nombreuses situations en sciences de l'atmosphère, on peut considérer que l'évolution de la parcelle est \voc{adiabatique} et se fait sans échange de chaleur avec l'extérieur ($\delta q=0$). En vertu de l'équilibre hydrostatique qui relie pression~$P$ et altitude~$z$~:
\begin{citemize}
\item une parcelle dont l'altitude~$z$ augmente sans apport extérieur de chaleur, subit une \voc{ascendance} adiabatique, donc une détente telle que~$\dd P < 0$ et sa température diminue ;
\item inversement, une parcelle dont l'altitude~$z$ diminue sans apport extérieur de chaleur, subit une \voc{subsidence} adiabatique, donc une compression telle que~$\dd P > 0$ et sa température augmente. 
\end{citemize}

\sk
Dans le cas où la transformation est adiabatique, pression et température sont intimement liées en vetu du premier principe. La version du premier principe encadrée ci-dessus avec~$\delta q = 0$ indique
\[ \dd T = \frac{R}{C_P} \, \frac{T}{P} \, \dd P\]
\[ \Rightarrow \qquad \frac{\dd T}{T} - \frac{R}{C_P} \, \frac{\dd P}{P} = 0 \]
soit par intégration
\[ T \, P^{- \kappa} = \text{constante} \qquad \text{avec} \qquad \kappa = R / C_P \]
Autrement dit, dans le cas où une parcelle subit une transformation adiabatique, sa température varie proportionnellement à~$P^{\kappa}$. Il s'agit d'une version, avec les grandeurs intensives utiles en sciences de l'atmosphère, de l'équation~$P\,V^{\gamma}$, avec $\gamma = C_P / C_V$, vue dans les cours de thermodynamique générale pour les transformations adiabatiques.


