\sk
Dans de nombreuses situations en sciences de l'atmosphère, on peut considérer que l'évolution de la parcelle est \voc{adiabatique} et se fait sans échange de chaleur avec l'extérieur ($\delta q=0$). En vertu de l'équilibre hydrostatique qui relie pression~$P$ et altitude~$z$~:
\begin{citemize}
\item une parcelle dont l'altitude~$z$ augmente sans apport extérieur de chaleur, subit une \voc{ascendance} adiabatique, donc une détente telle que~$\dd P < 0$ et sa température diminue ;
\item inversement, une parcelle dont l'altitude~$z$ diminue sans apport extérieur de chaleur, subit une \voc{subsidence} adiabatique, donc une compression telle que~$\dd P > 0$ et sa température augmente. 
\end{citemize}

\sk
Dans le cas où la transformation est adiabatique, pression et température sont intimement liées en vetu du premier principe. La version du premier principe encadrée ci-dessus avec~$\delta q = 0$ indique
\[ \dd T = \frac{R}{c_p} \, \frac{T}{P} \, \dd P \qquad \Rightarrow \qquad \frac{\dd T}{T} - \frac{R}{c_p} \, \frac{\dd P}{P} = 0 \]
soit par intégration
\[ T \, P^{- \kappa} = \text{constante} \qquad \text{avec} \qquad \kappa = R / c_p \]
Autrement dit, dans le cas où une parcelle subit une transformation adiabatique, sa température varie proportionnellement à~$P^{\kappa}$. Il s'agit d'une version, avec les grandeurs intensives utiles en sciences de l'atmosphère, de l'équation~$P\,V^{\gamma}$, avec $\gamma = c_p / c_v$, vue dans les cours de thermodynamique générale pour les transformations adiabatiques.

\sk
En se basant sur les considérations précédentes, il est possible de définir une quantité nommée \voc{température potentielle}~$\theta$ (en Kelvin) qui se conserve au cours de transformations adiabatiques
\[ \theta  = T \, \Pi^{-1} \qquad \textrm{avec} \qquad \Pi = \left( \f{P}{P_0} \right)^{R/c_p} 
\qquad \qquad
\f{\dd \theta}{\theta} = \f{\dd T}{T} - \f{R}{c_p} \f{\dd P}{P} = 0
\]
\noindent avec $\Pi$ la fonction adimensionnelle d'Exner et $p_0$ est une valeur de référence pour la pression (par exemple, $1000$~hPa pour la Terre). La température potentielle est donc égale à la température d'une parcelle ramenée de façon adiabatique à une pression~$P_0$. Cette quantité donne des informations fiables sur les échanges de chaleur d'une parcelle avec l'extérieur, contrairement à la température.
