\sk
On cherche à relier simplement température et vent. Si l'on suppose l'équilibre du vent gradient vérifié (équilibre horizontal dans les moyennes latitudes), il donne déjà une relation entre champ de vent et champ de pression, donc seules quelques transformations de cette relation sont nécessaires, en faisant le lien entre champ de pression et champ de masse via l'équilibre hydrostatique (équilire vertical). Il est néanmoins nécessaire de se placer dans des coordonnées pression selon la verticale (surfaces isobares) car elles permettent d'utiliser deux propriétés du géopotentiel~$\Phi$~: 
\begin{enumerate}
\item sa dérivée verticale est l'inverse de la masse volumique~$\rho$ donc reliée directement au champ de masse
\[ \frac{\partial \Phi}{\partial P} = -\f{1}{\rho} \]
\item sa dérivée horizontale est reliée à la composante horizontale de la force de pression
\[ \frac{1}{\rho} \left[ \frac{\partial P}{\partial y} \right]_z
= \left[ \frac{\partial \Phi}{\partial y} \right]_P \]
\end{enumerate}

\sk
En utilisant l'expression de la force de pression en tant que dérivée horizontale du géopotentiel~$\Phi$, on peut alors modifier l'équilibre du vent gradient
\[ \dfrac{u^2\tan\phi}{a} + 2\Omega\sin\phi u = -\dfrac{1}{\rho}\Dp{P}{y} = - \frac{\partial \Phi}{\partial y} \]
\noindent que l'on peut ensuite dériver par rapport à la coordonnée verticale pression~$P$ 
\[ \left[ 2 \, u \, \dfrac{\tan\phi}{a} + 2\Omega\sin\phi \right] \Dp{u}{P} = - \Dp{~}{P} \left[ \frac{\partial \Phi}{\partial y} \right] \]
\noindent afin de pouvoir commuter les dérivées partielles puis utiliser la version de l'équilibre hydrostatique formulée ci-dessus avec le géopotentiel~$\Phi$
\[ \left[ 2 \, u \, \dfrac{\tan\phi}{a} + 2\Omega\sin\phi \right] \Dp{u}{P} = - \Dp{~}{y} \left[\frac{\partial \Phi}{\partial P} \right] = \Dp{~}{y} \left[ \frac{1}{\rho} \right] \]
\noindent Reste à employer l'équation d'état des gaz parfaits~$P=\rho\,R\,T$ pour faire apparaître la température
\[ \left[ 2 \, u \, \dfrac{\tan\phi}{a} + 2\Omega\sin\phi \right] \Dp{u}{P} = R \, \frac{\partial}{\partial y} \left[ \frac{T}{P} \right] 
\qquad \Rightarrow \qquad
\boxed{ \left[ \textcolor{brown}{2\,u\,\frac{\tan\phi}{a}} + \textcolor{red}{2\Omega\sin\phi} \right] \Dp{u}{P} = \frac{R}{P} \, \frac{\partial T}{\partial y} }
\]
\noindent L'expression encadrée de l'équilibre du vent thermique vient de la constatation finale que l'on peut sortir le terme en pression à l'intérieur de la dérivée à droite puisque~$P$ est une coordonnée supposée fixe par définition de la dérivée partielle suivant~$y$. Les termes sont colorés en fonction de l'équilibre dans lequel on se trouve : \textcolor{brown}{équilibre cyclostrophique} (exemple sur Vénus) ou \textcolor{red}{équilibre géostrophique} (exemple sur Mars ou la Terre). Dans le cas où la situation est ambigüe, il faut conserver les deux termes.

\sk
L'\voc{équilibre du vent thermique} exprime un lien diagnostique entre les variations verticales du vent zonal et les variations méridiennes de la température. Sur des planètes où la mesure de température est aisée à mesurer (e.g. par télédétection infrarouge) par rapport au vent, cet équilibre est employé pour calculer un champ de vent (appelé vent thermique) associé à un champ de température. Invariablement, l'équilibre du vent thermique peut permettre de déduire les variations de température associées à un vent donné. Il s'agit d'un équilibre entre deux champs~température$\leftrightarrow$vent, sans relation de causalité température$\rightarrow$vent ou vent$\rightarrow$température.

