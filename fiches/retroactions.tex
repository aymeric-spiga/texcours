
Les processus de rétroactions climatiques peuvent amplifier (on parle alors de \voc{rétroaction positive}, en anglais \emph{positive feedback}) ou réduire (\voc{rétroaction négative}) la réponse à une perturbation initiale et sont donc centraux pour simuler correctement l’évolution du climat.

\begin{finger}
\item La rétroaction de Stefan-Boltzmann : Si la température augmente alors la perte par rayonnement augmente : feedback négatif très fort
\item La rétroaction de la glace et de l’albédo : Si la température augmente alors la glace diminue et donc le rayonnement solaire absorbé augmente ce qui augmente la température : feedback positif
\item Rétroaction des gaz à effet de serre au cours d’un cycle glaciaire-interglaciaire: une entrée en glaciation entraîne une baisse de la teneur en gaz à effet de serre (CO2, H2O vapeur et CH4) dans l'atmosphère par suite des modifications du climat (refroidissement de la surface terrestre et modification de la circulation océanique profonde); cette diminution atténue l'effet de serre initial et donc amplifie le refroidissement en cours. Inversement une déglaciation entraîne une augmentation des mêmes gaz à effet de serre, ce qui, cette fois, contribue à accentuer le réchauffement. Feedback positif
%\item La rétroaction des nuages: Si la température augmente et induit plus de nuages qui réfléchissent plus d’énergie solaire alors la température diminue. Cependant par effet de serre des nuages, la température augmente. Au contraire, si le climat se refroidit, la couverture neigeuse hivernale persistera plus longtemps. Or cette couverture blanche (d’albédo plus élevé que le sol) augmente la réflexion de l'énergie solaire et donc diminue le chauffage de la surface par le Soleil. Il en résulte un refroidissement de la surface qui amplifie le refroidissement climatique initial -- feedback positif si refroidissement, inconnu si réchauffement (a priori négatif pour les nuages bas).
\end{finger}

%\sk \subsection{Rétroactions} Préciser les rétroactions en jeu Indiquer le sens de ces rétroactions. Exemple terrestre~: \begin{itemize} \item Stefan-Boltzman : positive ou négative \item Glaces : positive ou négative \item Vapeur d'eau : positive ou négative \item Nuages : positive ou négative \end{itemize} 
%\visible<2->{\vskip 0.5cm\ebloc{}{Applications~:~changement climatique, paléo-climats, évolution des planètes du système solaire, climat des exoplanètes}}} \note{Peut être en général étendu aux autres planètes.\\ Les processus de rétroactions climatiques peuvent amplifier (on parle alors de rétroactions positives) ou réduire (rétroaction négative) la réponse à une perturbation initiale\\ SB : Si la température augmente alors la perte par rayonnement augmente : feedback négatif très fort\\ glace (albedo) : Si la température augmente alors la glace diminue et donc le rayonnement solaire absorbé augmente ce qui augmente la température feedback positif, mais attention au contrôle par la taille de la calotte polaire.\\ vapeur d’eau : l’augmentation de la température tend à favoriser l'évaporation car l'équilibre L-V est déplacé, de fait augmentation de l'humidité ie le contenu en vapeur d’eau de l’atmosphère, ce qui augmente l’effet de serre et donc la température de surface\\ nuages :  Si la température augmente et induit plus de nuages qui réfléchissent plus d’énergie solaire alors la température diminue. Cependant par effet de serre des nuages, la température augmente…. Au contraire, si le climat se refroidit, la couverture neigeuse hivernale persistera plus longtemps. Or cette couverture blanche (d’albédo plus élevé que le sol) augmente la réflexion de l'énergie solaire et donc diminue le chauffage de la surface par le Soleil. Il en résulte un refroidissement de la surface qui amplifie le refroidissement climatique initial feedback positif si refroidissement, inconnu si réchauffement (a priori négatif pour les nuages bas)} \note{[FACULTATIF] GES cycle glaciaire vs. interg.~: une entrée en glaciation entraîne une baisse de la teneur en gaz à effet de serre (CO2, H2O vapeur et CH4) dans l'atmosphère par suite des modifications du climat (refroidissement de la surface terrestre et modification de la circulation océanique profonde); cette diminution atténue l'effet de serre initial et donc amplifie le refroidissement en cours. Inversement une déglaciation entraîne une augmentation des mêmes gaz à effet de serre, ce qui, cette fois, contribue à accentuer le réchauffement feedbak positif}
