\paragraph{Variations selon la température} 

\begin{finger}
\item L'énergie émise dépend de la température du corps émetteur~: 
\begin{citemize}
\item quantitativement~: plus le corps est chaud, plus la quantité de rayonnement thermique est grande~: la luminance spectrale~$B_{\lambda}$ augmente avec la température $T$ quelle que soit la longueur d'onde.
\item qualitativement~: la \ofg{couleur} du corps dépend de sa température~: la longueur d'onde pour laquelle le rayonnement est maximal diminue quand la température augmente.
\end{citemize}
\item La dépendance en température de la forme des courbes sur la figure~\ref{fig:BBrad} est résumée par deux lois simples qui sont décrites à la section suivante~: la loi de Wien (position du maximum) et la loi de Stefan-Boltzmann (intégrale totale).  
\end{finger}

\paragraph{Variations selon la longueur d'onde} 

\begin{finger} 
\item Le rayonnement thermique est surtout significatif entre les longueurs d'onde~$0.1$ et~$100$~$\mu$m, soit le domaine visible et infrarouge. Pour le type de température usuellement rencontrées sur Terre, la contribution dans les longueurs d'onde visible est petite par rapport à la contribution dans l'infrarouge -- il faut atteindre des températures de plusieurs centaines de degrés Celsius pour qu'elle devienne significative, comme on peut le constater lorsqu'on porte à haute température un morceau de métal ou que l'on considère une coulée de lave fraîche.
\item La luminance énergétique~$B_{\lambda}$ tend vers 0 
\begin{citemize}
\item aux longueurs d'ondes très courtes, ce qui signifie que le rayonnement thermique comporte extrêmement peu des photons les plus énergétiques;
\item et aux longueurs d'onde très grandes, ce qui est attendu étant donné que l'énergie des photons tend vers~$0$ et que leur nombre n'est pas suffisant pour que la contribution énergétique soit significative.
\end{citemize}
\end{finger}

