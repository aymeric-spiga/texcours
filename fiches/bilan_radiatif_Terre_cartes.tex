\sk
On présente dans cette section des cartes des différents termes du bilan radiatif de la terre, tels qu'observés par satellite depuis l'espace. On observe un bilan moyen sur une année qui est variable en fonction de la position géographique : si l'on a, en moyenne, égalité entre absorption du rayonnement solaire incident et émission de la Terre vers l’espace, ce n’est plus vrai si on considère une région donnée. 
\begin{finger}
\item
Le flux solaire absorbé (figure \ref{fig:swtoa}) montre essentiellement une dépendance en latitude. L'effet de l'ensoleil\-lement au sommet de l'atmosphère, plus fort dans les tropiques, est amplifié par un albédo élevé aux latitudes polaires à cause de la présence de neige et de glace au sol. En plus de ces variations en latitudes, on observe des différences locales dûes à l'albédo des régions nuageuses (zone de convergence intertropicale, bords est des océans) ou du sol (Sahara).
\item
Le flux infrarouge sortant au sommet de l'atmosphère (figure \ref{fig:olr}) a lui aussi une structure en latitude, mais moins marquée que pour les ondes courtes: les hautes latitudes, plus froides, émettent moins de rayonnement. On voit d'autre part nettement le flux sortant plus faible dans les régions humides des tropiques (continents et zone de convergence) où des nuages convectifs d'altitude élevée se forment.
\item
La signature des régions humides est nettement plus faible sur la carte du bilan net au sommet de l'atmosphère (figure \ref{fig:nettoa}); les effets de serre et d'albédo des nuages se compensant en grande partie. On retrouve par contre un bilan moins positif dans les régions où un albédo élevé provient du sol (Sahara) ou de nuages bas (Chili, Californie). D'autre part, on observe un gain net d'énergie dans les tropiques, et une perte dans les hautes latitudes; la distribution du bilan dans le visible qui est plus inégale que celle dans l'infrarouge détermine donc la structure globale. 
\end{finger}

\figside{0.65}{0.25}{\figfrancis/erbe_stoa_ann}{Rayonnement visible absorbé par la Terre, en moyenne annuelle (données ERBE).}{fig:swtoa}
\figside{0.65}{0.25}{\figfrancis/erbe_olr_ann}{Rayonnement infrarouge sortant au sommet de l'atmosphère, en moyenne annuelle.}{fig:olr}
\figside{0.65}{0.25}{\figfrancis/erbe_ntoa_ann}{Flux net absorbé par la Terre (visible  - infrarouge sortant) en moyenne annuelle.}{fig:nettoa}

%• Disparités régionales :
%o Sahara en été boréal : fort albédo + forte perte IR + capacité
%calorifique faible + atmosphère sèche
%o Océan : albédo faible compense la perte radiative IR + forte
%capacité calorifique

\sk
Ces excès et déficit d'énergie locaux doivent, en moyenne, être compensés par des transports d'énergie par les circulations atmosphérique et océanique. Ils fournissent la source d'énergie pour la \voc{dynamique atmosphérique} qui va contribuer à répartir l'énergie des régions excédentaires en énergie vers les régions déficitaires en énergie~[figure~\ref{fig:hadley}].

\figun{0.75}{0.35}{decouverte/cours_meteo/energiedyn.png}{Schéma représentant les latitudes où l'atmosphère est excédentaire ou, au contraire, déficitaire en énergie. La courbe bleue représente l'énergie radiative reçue du Soleil, principalement dans les courtes longueurs d'onde (noté \emph{shortwave} sur la figure, correspond au rayonnement visible et ultraviolet). La courbe rouge représente l'émission par la surface terrestre, principalement dans les longues longueurs d'onde (noté \emph{longwave} sur la figure, correspond au rayonnement infrarouge). Une circulation atmosphérique de grande échelle se met en place entre les régions excédentaires (équatoriales et tropicales) et déficitaires (hautes latitudes).}{fig:hadley}

