%%\footnote{Cette partie est inspirée d'éléments trouvés dans le cours de S. Jacquemoud de \emph{Méthodes physique en télédétection.}}

\sk
Les \voc{aérosols} sont constitués de petites particules solides ou liquides en suspension dans les basses couches de l’atmosphère. Environ trois milliards de tonnes de particules sont injectés chaque année dans l’atmosphère par les processus naturels ou les activités humaines. On distingue plusieurs types d'aérosols.
\begin{citemize}
\item[\emph{poussières d'origine désertique}] Il s'agit de la première source mondiale d’aérosols. Elles sont soulevées par des vents violents lors des tempêtes de sable. Les grosses particules retombent rapidement au sol alors que les plus petites forment un nuage sec qui peut s’élever jusqu’à 4-6 km d’altitude et s’étendre sur des milliers de kilomètres. On peut retrouver en Europe ou en Amérique des particules en provenance d'Afrique.
\item[\emph{aérosols solubles dans l’eau}] Ils peuvent être d'origine naturelle (substances organiques émises par la végétation) ou liés à l'activité industrielle (sulfates, nitrates).
\item[\emph{aérosols d'origine marine}] Ils sont formés à partir des bulles résultant du déferlement des vagues et des courants marins. Outre le dioxyde de carbone, les bulles transportent quantité de substances, notamment des particules de sel microscopiques (NaCl) qui participent à la formation de brumes. L'éclatement de ces bulles à la surface des océans donne naissance à un très grand nombre de gouttelettes (parfois une centaine pour une bulle d'un diamètre de l'ordre du millimètre) qui ne se brisent pas et sont à l'origine des aérosols marins (également appelés embruns).
\item[\emph{aérosols carbonés}] Ils sont présents dans les régions tropicales et boréales en raison de nombreux feux de forêt. Par exemple les brûlis de la végétation intertropicale en période sèche occasionnent chaque année des brumes sèches qui disparaissent une fois les pluies revenues.
\item[\emph{aérosols de sulfates}] Ils sont d'origine volcanique. Le dioxyde de soufre SO$_2$ émis lors des éruptions volcaniques produit ces fines particules d'acide sulfurique (SO$_2$ + H$_2$O~$\rightarrow$~H$_2$SO$_4$) qui s'entourent de glace et forment avec les cendres un écran empêchant le rayonnement solaire d'arriver jusqu'au sol.
\end{citemize}
La plupart des aérosols se trouvent dans la troposphère où ils résident en moyenne une semaine. En raison de leur petite taille, les aérosols peuvent être transportés sur de longues distances. Ils sont en général ramenés au sol par les précipitations (pluie, neige). Les aérosols de plus petite taille ($0.01-0.1$~$\mu$m) jouent un rôle important de \voc{noyaux de condensation} dans la formation des nuages en favorisant la condensation de vapeur d'eau en gouttelettes d'eau et/ou de cristaux de glace. Les aérosols de plus grande taille (0.1-1.0~$\mu$m), les plus nombreux, interceptent la lumière du Soleil. La stratosphère contient aussi des aérosols (principalement d'origine volcanique) jusqu'à 18-20 km d’altitude. Contrairement aux aérosols troposphériques, leur concentration est relativement uniforme et leur durée de vie beaucoup plus longue, de plusieurs mois à plusieurs années.  

\sk
Certaines molécules peuvent s’agréger pour former des particules liquides ou solides. Dans l'atmosphère terrestre, ceci concerne principalement l'eau à l'état liquide ou solide dans l'atmosphère qui participe à la formation d'\voc{hydrométéores}. Ce sont des particules d’eau liquide (gouttelettes d'eau) et/ou solide (cristaux de glace) suspendues dans l'atmosphère dont la taille varie de~$1$~$\mu$m à~$1$~cm. Les brumes et nuages sont formés de ces fines gouttelettes d'eau en suspension dans l'atmosphère, qui apparaissent dès que le seuil de saturation de l'air en vapeur d'eau est dépassé (ces mécanismes sont détaillés dans un chapitre ultérieur). Les nuages bas et intermédiaires sont constitués de gouttelettes d’eau liquide ; les nuages d’altitude de cristaux de glace de différentes formes géométriques. Certains sont accompagnés de précipitations lorsque les gouttes ou cristaux sont assez gros pour former de la pluie, neige, grêle ou verglas. %La figure~\ref{fig:droplet} donne un ordre d'idée des tailles respectives des noyaux de condensation, des gouttelettes nuageuses et des gouttes de pluie.
%\figside{0.4}{0.15}{decouverte/cours_meteo/gouttes.png}{Taille comparée des noyaux de condensation (\emph{cloud condensation nuclei}), des gouttelettes de brume ou de nuage (\emph{moisture droplets}), des gouttes de pluie (\emph{raindrop})}{fig:droplet}

