\sk
L’étude de l’absorption du rayonnement par un milieu repose sur deux relations établies au XVIIIe siècle qui comparent la luminance spectrale~$L_\lambda(0)$ d'un faisceau incident monochromatique pénétrant sous incidence normale dans un milieu matériel absorbant, à la luminance~$L_\lambda(\ell)$ après traversée du milieu de longueur~$\ell$.
\begin{citemize}
\item D'une part, le faisceau incident subit une extinction telle que le rapport~$\log \frac{L_\lambda(\ell)}{L_\lambda(0)}$ est proportionnel à la longueur~$\ell$ parcourue par le rayonnement dans le milieu absorbant (relation de Bouguer-Lambert).  
\item D'autre part, dans un milieu de concentration molaire effective~$[X]$ en espèce absorbante X, le faisceau incident subit une extinction telle que le rapport~$\log \frac{L_\lambda(\ell)}{L_\lambda(0)}$ est proportionnel à la concentration~$[X]$ (relation de Beer). Le facteur de proportionnalité dépend de l'épaisseur du milieu traversé, de sa nature, de sa composition, de la température et de la longueur d'onde du rayonnement incident.
\end{citemize}
Le logarithme indique que les variations relatives d'énergie radiative au cours de la propagation dans le milieu sont proportionnelles à la longueur parcourue et à la concentration d'espèces absorbantes.

\sk
Les deux lois historiques décrites ci-dessus décrivent exactement la situation de la section~\ref{sec:efficace} si l'on néglige la diffusion. Par rapport à la section~\ref{sec:efficace}, la seule hypothèse supplémentaire est que la section efficace d'absorption~$\Sigma_\lambda^a$ ne dépend que de la longueur d'onde et du matériau, elle est considérée comme uniforme sur toute la longueur traversée. Si l'on reprend la situation de la figure~\ref{fig:beer} avec cette hypothèse, on a 
\[ \dd L_\lambda = - L_\lambda(z) \, \Sigma_\lambda^a \, N_X \, \frac{1}{\cos\theta} \, \dd z \]
où $N_X$ est la concentration de l'espèce absorbante exprimée sous la forme d'un nombre de molécule par unité de volume (au lieu de $[X]$ en mol~L$^{-1}$).
La variation de flux au cours de la traversée du milieu matériel est donc
\[ \frac{\dd L_\lambda}{L_\lambda} = - \Sigma_\lambda^a \, N_X \, \frac{1}{\cos\theta} \, \dd z \]
Par intégration, si l'on suppose que~$z=0$ repère l'entrée dans le milieu matériel et~$z=\ell$ la sortie, on obtient
\[ \int_{z=0}^{z=\ell} \, \frac{\dd L_\lambda}{L_\lambda} = - \Sigma_\lambda^a \, N_X \, \frac{1}{\cos\theta} \, \int_{z=0}^{z=\ell} \dd z \]
d'où la valeur de la luminance spectrale à la sortie du milieu traversé, donnée par deux relations équivalentes
\[ \boxed{ \log \frac{L_\lambda(\ell)}{L_\lambda(0)} = - \zeta \, \ell } \qquad \textrm{et} \qquad \boxed{ L_\lambda(\ell) = L_\lambda(0) \, e^{- \zeta \, \ell} } \qquad \textrm{avec la constante en m}^{-1} \qquad \zeta = \Sigma_\lambda^a \, N_X \, \frac{1}{\cos\theta} \]
Cette loi reprend les résultats historiques présentés précédemment et porte le nom de \voc{loi de Beer-Lambert-Bouguer}. Elle indique la décroissance exponentielle du flux incident lors de sa traversée du milieu, d'autant plus importante que la longueur traversée~$\ell$ est grande et que la concentration en absorbant~$N_X$ du milieu est grande. La loi de Beer-Lambert-Bouguer peut également être appliquée sous la même forme avec l'éclairement. Le coefficient~$\zeta$ porte parfois le nom de coefficient d'extinction linéique. 
%Nous ferons l’hypothèse par la suite que cette dépendance est linéaire, ce qui est vrai pour l’air.
On donne quelques ordres de grandeur ci-dessous pour la valeur de~$\zeta$
\begin{citemize}
\item atmosphère pour un rayonnement visible dans le jaune $\zeta = 1 \times 10^{-5}$~m$^{-1}$ 
\item atmosphère pour un rayonnement visible dans le violet $\zeta = 4 \times 10^{-5}$~m$^{-1}$
\item verre $\zeta = 0.2$~m$^{-1}$
\item nuage bas $\zeta = 1 \times 10^{-3}$~m$^{-1}$
\end{citemize}
En remarquant que le rapport~$L_\lambda(\ell) / L_\lambda(0)$ définit justement la coefficient de transmission spectral~$\tau_\lambda$ \emph{en l'absence de diffusion}, on arrive à
\[ \tau_\lambda = e^{- \zeta \, \ell} \qquad \textrm{et} \qquad \alpha_\lambda = 1 - e^{- \zeta \, \ell} \]
On peut vérifier que l'expression est conforme à l'intuition : si la longueur de la traversée est particulièrement grande ($\ell \rightarrow \infty$), et/ou que l'espèce absorbante~X est très concentrée ($N_X \rightarrow \infty$), alors presque tout le rayonnement incident est absorbé~$\alpha_\lambda \rightarrow 1$ et une partie négligeable de ce rayonnement est transmise~$\tau_\lambda \rightarrow 0$. 

\sk
En sciences de l'atmosphère, on utilise souvent une forme plus générale de la loi de Beer-Lambert-Bouguer. On écrit la relation intermédiaire (non intégrée) qui conduit à cette loi sous la forme 
\[ \frac{\dd L_\lambda}{L_\lambda} = \, \rho_X \, k_\lambda \, \frac{1}{\cos\theta}  dz \]
où $k_\lambda$ est un coefficient d'absorption massique en m$^2$~kg$^{-1}$ et $\rho_X(z)$ est la densité d'absorbant~X, qui dépend de~$z$ comme ce peut être le cas dans l'atmosphère. Cette relation peut être intégrée sur une couche épaisse située entre les niveaux~$z_1$ et~$z_2$. On obtient 
\[ L_\lambda(z_1) = L_\lambda(z_2) \, e^{- \frac{t_\lambda}{\cos\theta}} \]
où 
\[ t_\lambda = \int_{z_1}^{z_2} \, k_\lambda \, \rho_X \, \dd z \]
est appelée l'épaisseur optique de la couche. Si l'extinction est uniquement due à de l'absorption, sans diffusion, on a une relation directe entre l'épaisseur optique et le coefficient d'absorption de la couche: 
\[\alpha_\lambda = 1 - e^{- \frac{t_\lambda}{\cos\theta}} \]
%Dans le cas particulier où la densité d'absorbant est de la forme
%\[\rho_a=\rho_a^0e^{-z/H_a}\]
%ce qui est le cas par exemple d'un gaz bien mélangé dans l'atmosphère, ou de
%la vapeur d'eau, on peut calculer l'altitude du taux d'extinction
%$dL_\lambda/dz$ maximum: on a alors également d'après la définition de
%$\tau_\lambda$
%\[\tau_\lambda=\tau_\lambda^0e^{-z/H_a}\]
%et $d\tau_\lambda/dz=-\tau_\lambda/H_a$. D'autre part, le taux d'extinction vaut 
%\[dL_\lambda/dz=-L_\lambda\mu d\tau_\lambda/dz=L_\lambda^\infty
%e^{-\mu\tau_\lambda}\mu\tau_\lambda/H_a\]
%Ce taux est maximal pour 
%\[d\left(\tau_\lambda\mu e^{-\tau_\lambda\mu}\right)=0\]
%soit pour $\mu\tau_\lambda=1$. On a donc un maximum d'extinction (absorption
%ou diffusion) du rayonnement incident pour une épaisseur optique de 1
%traversée à partir du sommet de l'atmosphère. Pour des épaisseurs optiques
%plus faibles, on a peu d'extinction car la densité d'absorbants est faible.
%Pour des épaisseurs optiques plus grandes, on a beaucoup d'absorbants mais la
%luminance résiduelle est petite (figure \ref{fig:absrate}).
% 
%\begin{figure}[tbp]
%  \begin{center}
%    \includegraphics{\figfrancis/WH_abs_max}
%  \end{center}
%  \caption{Comparaison des structures verticales de la densité de
%  l'atmosphère $\rho$, de la luminance d'un rayonnement incident $L_\lambda$
%  et de sa dérivée verticale. L'échelle horizontale est linéaire pour chaque
%  grandeur.}
%  \label{fig:absrate}
%\end{figure}

