\sk
Une quantité utile pour quantifier à l'ordre 0 l'amplitude des mouvements atmosphériques verticaux causés par l'instabilité convective est de considérer le travail de cette force de flottaison, que l'on appelle CAPE~$\mathcal{C}$ ou \voc{Convectively Available Potential Energy} et de calculer l'énergie cinétique verticale associée. Cela donne accès à une borne supérieure de la vitesse verticale~$w\e{max}$ atteinte dans l'ascendance car, en réalité, toute l'énergie potentielle n'est pas (loin de là) convertie en énergie cinétique.
\[ \mathcal{C} = \int_{\textrm{ascendance}}^{} g \, \frac{T\e{p}-T\e{e}}{T\e{e}} \, \dd z \]
\[ w\e{max} = \sqrt{2\,\mathcal{C}}  \]
