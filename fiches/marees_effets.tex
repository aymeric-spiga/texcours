
%% AlexBarbo: Un petit schéma ? C'est compréhensible, mais un schéma c'est encore mieux.

\figside{0.25}{0.15}{/home/aymeric/Work/decouverte/cours_geomorpho/Tidal_braking.jpg}{A diagram of the tidal braking effect by which the Moon slows the Earth's rotation. AndrewBuck, Wikipedia.}{fig:effetserre2}

\sk
\paragraph{Couple des marées} Si les planètes étaient parfaitement fluides, l'objet serait déformé par les forces de marée dans la direction donnée par le centre de masse de l'objet et celui du corps attracteur. En pratique, la déformation des forces de marée n'est pas instantanée et la direction de déformation (des points subplanétaire S et antiplanétaire A) forme un angle avec la direction donnée par le centre de masse de l'objet et celui du corps attracteur. Il s'exerce alors un couple de force en A et en S~: ce couple agit toujours dans le sens s'opposant à la rotation propre de l'objet~$\omega$. On définit 
\begin{citemize}
\item la déformation élastique de l'objet en réponse aux perturbations de marée, notée $k_T$ (appelée \emph{tidal Love number})
\item le facteur de qualité~$Q$ donnant le rapport entre l'énergie maximale stockée dans le bourrelet de marée et l'énergie dissipée en un cycle 
\end{citemize}
\noindent Le couple~$\Gamma_m$ exercé par les forces de marée dépend donc de la taille du bourrelet induit (selon une loi~$k_T \, M \, r^{-3}$), du décalage de phase entre l'orientation de ce bourrelet et l'axe des centres de masse (selon une loi~$Q^{-1}\,\textrm{signe}(\omega-\Omega)$), et s'exprime comme le couple d'une force de marée~$f_m$ donc proportionnel à~$\mathcal{G}\,M\,r^{-3}$. Finalement
\[ \Gamma_m = \frac{3}{2} \, \frac{k_T}{Q} \, \frac{\mathcal{G}\,M^2\,R^5}{r^6} \, \textrm{sgn}(\omega-\Omega) \]
\noindent Le couple des marées~$\Gamma_m$ permet de transférer du moment angulaire entre rotation propre et révolution orbitale.


\sk
\paragraph{Synchronisation et modification d'orbites} La Lune est en rotation synchrone autour de la Terre : elle lui présente toujours la même face car sa période de rotation (propre) est égale à sa période de révolution (orbitale). La synchronisation peut se décrire comme suit~: le point de départ est un objet en rotation asynchrone, dont le bourrelet induit par les forces de marée est \og en retard \fg~par rapport à la rotation de la planète. Le couple des marées décrit ci-dessus va toujours dans le sens d'une synchronisation. Par ce mécanisme, la Lune présente toujours la même face à la Terre -- et la Lune ralentit la rotation propre de la Terre (2 millisecondes en plus par siècle) pour tendre virtuellement (à très long terme \ldots) vers une synchronisation\footnote{La modification de la rotation propre d'un corps par les forces de marée ne conduit pas toujours à une rotation synchrone, en témoigne le cas de Mercure en résonance $3:2$ (3 rotations propres pour 2 révolutions orbitales) sous l'effet des forces de marée du Soleil.}.
%% Correia & Laskar Nature 2004 
%% http://www.nature.com/nature/journal/v429/n6994/abs/nature02609.html
Ce lent ralentissement de la rotation propre de la Terre implique donc que la Terre perd du moment angulaire, qui est transféré à la Lune. Comme cette dernière voit sa rotation synchronisée, l'ajustement se fait via une augmentation de la distance orbitale Terre-Lune\footnote{Autre exemple d'ajustement (inspiré par Tristan Guillot), lorsque le Soleil aura consommé son hydrogène dans~$\sim 5$~milliards d'années et sera devenu une géante rouge (pouvant s'étendre potentiellement jusque l'orbite de Mars). Le Soleil perdra de la masse, donc du moment angulaire, et comme la Lune par rapport à la Terre, la Terre s'éloignera du corps attracteur responsable des marées, ici le Soleil. A moins que les marées exercées par la Terre sur l'enveloppe du Soleil ne la conduisent au contraire à se rapprocher du Soleil.}. 
