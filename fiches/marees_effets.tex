
\figside{0.7}{0.1}{/home/aspiga/images/decouverte/cours_geomorpho/maree_1.png}{Mécanisme de l'effet de marée gravitationnelle. $\omega$ représente le sens de rotation de la planète. Crédits J. Laskar}{fig:couplemaree}

\sk
\paragraph{Couple des marées} Si les planètes étaient parfaitement fluides, l'objet serait déformé par les forces de marée dans la direction donnée par le centre de masse de l'objet et celui du corps attracteur, comme indiqué par le calcul des forces de marée. En pratique, puisque les objets ne sont pas parfaitement élastiques, la déformation des forces de marée n'est pas instantanée, et se réalise pendant un certain temps caractéristique. Ainsi la direction de déformation (des points subplanétaire S et antiplanétaire A) forme un angle (décalage angulaire~$\delta$ sur la figure~\ref{fig:couplemaree}) avec la direction donnée par le centre de masse de l'objet et celui du corps attracteur. 

\sk
Il s'exerce alors un couple de force de rappel en A et en S~: ce couple agit toujours dans le sens s'opposant à la rotation propre~$\omega$ de l'objet. On définit 
\begin{citemize}
\item la déformation élastique de l'objet en réponse aux perturbations de marée, notée $k_T$ (appelée \emph{tidal Love number} qui contient l'information sur la rhéologie du corps)
\item le facteur de qualité~$Q$ donnant en un cycle le rapport entre l'énergie maximale stockée dans le bourrelet de marée et l'énergie dissipée  
\end{citemize}
\noindent Le couple~$\Gamma_m$ exercé par les forces de marée dépend donc de la taille du bourrelet induit (selon une loi~$k_T \, M \, a^{-3}$), du décalage de phase entre l'orientation de ce bourrelet et l'axe des centres de masse (selon une loi~$Q^{-1}\,\textrm{signe}(\omega-\Omega)$), et s'exprime comme le couple d'une force de marée~$f_m$ donc proportionnel à~$\mathcal{G}\,M\,a^{-3}$. Finalement
\[ \Gamma_m = \frac{3}{2} \, \frac{k_T}{Q} \, \frac{\mathcal{G}\,M^2\,r^5}{a^6} \, \textrm{sgn}(\omega-\Omega) \]
\noindent Le couple des marées~$\Gamma_m$ permet de transférer du moment angulaire entre rotation propre et révolution orbitale.

\sk
\paragraph{Synchronisation et modification d'orbites: rotation propre~$\omega$} Le couple de marée cause au long terme une synchronisation des orbites, dans la plupart des cas\footnote{La modification de la rotation propre d'un corps par les forces de marée ne conduit pas toujours à une rotation synchrone, en témoigne le cas de Mercure en résonance $3:2$ (3 rotations propres pour 2 révolutions orbitales) sous l'effet des forces de marée du Soleil. Correia et Laskar dans un article Nature de 2004 ont montré que cela résulte de l'évolution chaotique (déterministe) de l'orbite de Mercure, et non de la friction noyau-manteau comme cela avait été proposé précédemment.}. La synchronisation peut se décrire ainsi~: le point de départ est un objet en rotation asynchrone, dont le bourrelet induit par les forces de marée est \og en retard \fg~par rapport à la rotation de la planète (figure~\ref{fig:couplemaree}). Le couple des marées décrit ci-dessus va toujours dans le sens d'une synchronisation, car il tend à s'opposer au sens de rotation de la planète de manière à faire tendre celle-ci vers la vitesse de révolution. Par ce mécanisme, la Lune est en rotation synchrone autour de la Terre : elle lui présente toujours la même face car sa période de rotation (propre) est égale à sa période de révolution (orbitale).

\sk
\paragraph{Temps de synchronisation} Le temps caractéristique~$\tau_s$ de synchronisation de l'orbite d'un objet autour d'un corps attracteur, dans le cas le plus simple, est
\[ \tau_s = \frac{\omega \, a^6 \, I \, Q}{3 \, \mathcal{G} \, M^2 \, k_T \, r^5} \]
\noindent avec I le moment d'inertie de l'objet cible ($I=\frac{2}{5} m r^2$ pour une sphère). Le temps de synchronisation est donc proportionnel à~$a^6 / M^2$. Avec ce type de loi, on comprend qu'une planète dans la Zone Habitable d'une Naine M de $M = 0.2$~masses solaires, de luminosité bien plus faible que le Soleil, est située bien plus près de son étoile que la Terre du Soleil. Bien que son étoile soit moins massive que le Soleil, la variation en~$a^6$ indique un temps de synchronisation de l'orbite de cette planète 5 ordres de grandeur plus rapide que la Terre située dans la Zone Habitable du Soleil\footnote{Exemple inspiré de la fiche en ligne de Martin Turbet \url{https://media4.obspm.fr/public/ressources_lu/pages_planetologie-habitabilite/torque.html}}. 
