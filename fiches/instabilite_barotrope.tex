\sk
Pour aborder l'instabilité barotrope de manière simplifiée,
on reprend l'équation de la vorticité barotrope 
linéarisée pour un écoulement quasi-horizontal,
employée pour décrire les ondes de Rossby barotropes
\[
\left( \Dp{}{t} + \moyenne{u} \Dp{}{x} \right) \zeta' + v' \Dp{f}{y} = 0
\]
\noindent L'unique différence est néanmoins que l'on considère ici
que le vent zonal de base~$\moyenne{u}$ a une dépendance latitudinale,
comme dans un courant-jet réel, ce qui conduit à une vorticité relative
de base non nulle~$\moyenne{\zeta}$, donc l'équation de la vorticité suivante
\[
\left( \Dp{}{t} + \moyenne{u} \Dp{}{x} \right) \zeta' + v' \Dp{(\moyenne{\zeta}+f)}{y} = 0
\]
En utilisant~$\Dp{\moyenne{\zeta}}{y} = -\DDp{u}{y}$,
et en définissant le potentiel~$\Psi'$ 
tel que~$u'=-\Dp{\Psi'}{y}$
et~$v'=\Dp{\Psi'}{x}$, on parvient à
\[
\left( \Dp{}{t} + \moyenne{u} \Dp{}{x} \right) \nabla^2 \Psi' + \left( \beta - \DDp{\moyenne{u}}{y} \right) \Dp{\Psi'}{x} = 0
\]

\sk
Considérons une solution monochromatique se propageant uniquement
selon la longitude (axe~$x$) et le temps
\[
\Psi'(x,y) = Re \left[ \hat{\Psi}(y) \, \exi{(k(x-c\,t)} \right]
\]
\noindent avec~$c$ une vitesse de phase complexe. Si la partie
imaginaire~$c_i$ de la vitesse de phase~$c$ est positive, l'onde
croît exponentiellement avec le temps, dénotant l'instabilité de l'écoulement.
On obtient alors une équation différentielle du second degré
\[
\left[ \moyenne{u} - c \right] \left( \ddf{^2\hat{\Psi}}{y^2} - k^2 \, \hat{\Psi} \right) + \left( \beta - \DDp{\moyenne{u}}{y} \right) \hat{\Psi} = 0
\]

\sk
Supposons que l'onde est confinée dans un canal entre $y=0$ et $y=L$ avec~$\hat{\Psi}=0$
(situation plutôt réaliste si l'on considère l'instabilité d'un jet).
En multipliant par le complexe conjugué~$\hat{\Psi}^*$ et en divisant par~$\moyenne{u}-c$,
la partie imaginaire de l'intégrale suivant~$y$ des termes ci-dessus
s'annule pour le premier terme, ce qui conduit à
\[
c_i \int_0^L \left( \beta - \DDp{\moyenne{u}}{y} \right) \f{|\hat{\Psi}|^2}{|\moyenne{u}-c|^2} \dd y = 0
\]
Une instabilité ne peut se développer ($c_i > 0$) que
si l'intégrale est nulle, ce qui n'est possible que si
\[
\boxed{
\beta - \DDp{\moyenne{u}}{y} \quad \textrm{change de signe dans le canal } [0,L]
}
\]
\noindent Le critère ci-dessus est le critère d'instabilité barotrope de Rayleigh-Kuo.
Un courant-jet barotrope peut ainsi devenir instable 
si sa courbure selon la latitude est particulièrement marquée.
L'instabilité barotrope peut survenir
sous les tropiques terrestres,
dans les moyennes latitudes sur Vénus,
ou le courant circumpolaire martien.
Dans les courants-jets des planètes géantes,
l'instabilité barotrope est également probable,
notamment dans les jets dirigés vers l'ouest.
Elle pourrait jouer un rôle également
dans le développement de la forme hexagonale de l'hexagone de Saturne.




