\sk
L'énergie gravitationnelle est accumulée au cours de la croissance de la planète
par accrétion de matériel la composant.
L'énergie potentielle gravitationnelle acquise par une sphère planétaire
de masse~$m(r)$ et de rayon~$r$ lorsqu'une mass infinitésimale $\dd m$
est ajoutée depuis une distance infinie
est
\[ \dd E\e{p} = - \mathcal{G} \, \frac{m(r) \, \dd m}{r}  \]

\sk
Supposons que le corps planétaire est formé en maintenant une densité constante~$\rho_0$
jusqu'à ce que son rayon soit~$R$ ; alors l'incrément infinitésimal de masse~$\dd m$
est~$\dd m = 4 \, \pi \, r^2 \, \rho_0 \, \dd r$, et~$E\e{p}$ peut être déterminée par
intégration
\[ E\e{p} = - \mathcal{G} \, \int_{0}^{R} \frac{m(r) \, \dd m}{r}  
= - 3 \, \mathcal{G} \, \left( \frac{4\,\pi}{3} \right)^2 \, \rho_0^2 \, \frac{R^5}{5}
= - \frac{3}{5} \, \frac{\mathcal{G}\,M^2}{R} \] 
\noindent Plus généralement, pour une distribution sphérique quelconque~$\rho(r)$,
une analyse dimensionnelle indique que~$E\e{p}$ est proportionnelle à~$-\mathcal{G}\,M^2/R$.
La formule simplifiée ci-dessus donne le bon résultat pour la Terre avec seulement~$10\%$ d'erreur.

\sk
Ensuite supposons simplement que cette énergie est convertie en énergie interne
(ne faisant qu'appliquer en cela le premier principe de la thermodynamique, puisque
l'énergie gravitationnelle~$E_p$ n'est autre que l'opposé du travail de la force gravitationnelle)
\[ M \, c_p \, \Delta T = E\e{p} \]
\noindent (où~$c_p$ est la capacité calorifique massique du matériau composant la planète)
ce qui fournit une estimation de l'augmentation de température interne de la planète
\[ \Delta T = \frac{3}{5} \, \frac{\mathcal{G}\,M}{R\,C_p} \]
\noindent Comme attendu, une contraction gravitationnelle 
(diminution du rayon planétaire~$R$ à masse totale~M constante)
entraîne une augmentation de la température interne du corps.

\sk
Une partie de cette énergie est perdue par radiation par la surface.
En fait, une large part est perdue pendant la formation de la planète,
phase pendant laquelle à la fois 
la conduction de chaleur dans la planète
et la radiation d'énergie vers l'espace
sont très efficaces (autrement dit, leur temps caractéristique
est initialement petit devant la durée de vie d'une planète).
Néanmoins, la poursuite de la contraction,
et la différentiation qui provoque la descente
des éléments lourds vers le centre de la planète,
contribue à réchauffer l'intérieur de la planète
après la formation.

\sk
La mesure du flux surfacique (en W~m$^{-2}$) 
rayonné par une planète peut mettre à jour, en retranchant 
le flux surfacique reçu du Soleil (équilibre TOA), une contribution 
du flux de chaleur interne provenant de la contraction
gravitationnelle initiale
\[ \frac{1}{4\,\pi\,R^2}\,\ddf{E\e{p}}{t} \]
\noindent Le flux de chaleur interne de Jupiter peut être
expliqué complètement par l'énergie interne accumulée lors de la phase
de contraction initiale, mais ce n'est pas le cas de Saturne
qui est plus ``brillante'' que ne semble indiquer son âge.
Un processus de différentiation pourrait expliquer ce phénomène
via un phénomène de pluie d'hélium (causé par l'immiscibilité
de ce dernier dans l'hydrogène) ; plus récemment, il a été
proposé que l'intérieur de Saturne refroidit plus lentement
à cause d'une convection en couches causée par les
gradients compositionnels.
Uranus a, comme les planètes telluriques,
quasiment perdu sa source de chaleur interne -- alors
que Neptune émet toujours une quantité importante
de chaleur qui peut être liée à une température d'accrétion initialement très élevée.







