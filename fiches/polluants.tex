\sk
On peut illustrer la stabilité/instabilité atmosphérique dans le cas des polluants émis proche de la surface par les activités humaines [Figure~\ref{fig:pollution}]. Dans l'après-midi, du fait que le sol est chaud, le profil d'environnement est tel que la situation est très instable~: les mouvements verticaux qui transportent les polluants plus haut dans l'atmosphère sont encouragés et les polluants ne restent pas proches de la surface. A l'inverse, en soirée, du fait que le sol refroidit radiativement, le profil d'environnement est tel que la situation est stable~: les mouvements verticaux qui pourraient transporter les polluants plus haut dans l'atmosphère sont inhibés et les polluants sont confinés proche de la surface. Pour être moins exposé aux polluants dans les zones urbaines, il est donc préférable d'y effectuer son jogging en fin de matinée plutôt qu'en soirée !

\figside{0.65}{0.25}{decouverte/cours_meteo/inversion-temperature.png}{Stabilité et pollution atmosphérique. On notera que cette figure est très illustrative, mais présente une situation simplifiée. Le transport vertical de polluants dans l'atmosphère est en réalité inhibé dès que la couche atmosphérique est stable, ce qui est plus général que considérer uniquement une inversion thermique comme à droite de la figure. Source~: Airparif}{fig:pollution} 
