\figside{0.4}{0.2}{decouverte/cours_meteo/anvil.png}{Vue lointaine d'un cumulonimbus à un stade avancé de développement, où l'on peut observer la structure aplatie en forme d'enclume au sommet du nuage. Source~: Wallace and Hobbs, Atmospheric Science, 2006~; d'après une photographie du Bureau Australien de Météorologie.}{fig:enclume}

\sk
L'étude de la formation des cumulonimbus appelle deux remarques importantes qui illustrent les concepts de stabilité et instabilité atmosphérique.

\begin{finger}

\item Le sommet des cumulonimbus atteint très fréquemment la tropopause. Lorsque c'est le cas, ils prennent alors une apparence aplatie et la forme d'enclume comme présenté dans la figure~\ref{fig:enclume}. Cela provient du fait que la stratosphère voit la température de l'environnement augmenter avec l'altitude, contrairement à ce qui peut se passer dans la troposphère. Un tel profil de température est extrêmement stable, donc a tendance à inhiber les mouvements verticaux. Ainsi, le fort développement vertical des cumulonimbus est stoppé net lorsque les couches stables de la stratosphère sont atteintes. En conséquence, le nuage s'étale selon l'horizontale au voisinage de la tropopause. C'est la raison pour laquelle le sommet réel des nuages cumuliformes est le minimum du sommet théorique des nuages et de la hauteur de la tropopause. Plus généralement, des conditions stables peuvent conduire à la formation de nuages stratiformes, ce qui nuance un peu la distinction faite dans la section~\ref{classphys}.

\item On entrevoit par les développement précédents qu'il est possible qu'une couche atmosphérique donnée, dont la température suit le taux de décroissance~$|\Gamma\e{env}|$ selon l'altitude, apparaisse comme stable si l'on considère l'ascension d'une parcelle non saturée, mais instable si l'on considère l'ascension d'une parcelle saturée. Cette situation se présente lorsque 
\[ |\Gamma\e{saturée}| <  |\Gamma\e{env}|  < |\Gamma\e{sec}| \]
On parle alors d'\voc{instabilité conditionnelle}. Il s'agit de conditions où seule l'apparition d'un nuage peut donner lieu à une instabilité et au développement de mouvements verticaux potentiellement étendus. Dans ce cas de figure, les nuages qui se forment sont essentiellement cumuliformes.

\end{finger}


