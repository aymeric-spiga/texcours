\sk
Sur Mars comme sur Terre, les forçages radiatifs
sont le moteur de la circulation de grande échelle.
%
L'énergie radiative absorbée dans le visible
par le système \ofg{surface - atmosphère}
subit de plus fortes variations latitudinales
que l'énergie radiative émise dans l'\IR.
%
En moyenne annuelle, il en résulte un chauffage net 
des régions équatoriales de la planète et
un refroidissement net de ses régions polaires.
%
Ce contraste thermique induit des contrastes latitudinaux d'échelle de hauteur $H$,
donc, d'après l'équation hypsométrique,
un gradient de pression latitudinal qui augmente avec l'altitude.
%
Une telle force méridienne entraîne une circulation en altitude
des régions équatoriales excédentaires en énergie
aux hautes latitudes en déficit d'énergie.
%
De ce transport de masse résulte une augmentation de la pression de surface
aux hautes latitudes et donc une circulation inversée proche du sol.
%
En moyenne zonale, ces mouvements sont conceptualisés par les
\ofg{cellules de Hadley}.


\sk
La circulation précitée prend place
dans un référentiel particulièrement non-galiléen :
la planète en rotation sur elle-même.
%
Le moment cinétique absolu $\mathcal{M}$ 
d'une particule d'air de masse $m$
par rapport à l'axe de rotation de la planète
est en coordonnées sphériques
%
\[
\mathcal{M} = m \, (a \, \cos \varphi) (\Omega \, a \cos \varphi + u)
\]
%
\noindent pour une particule située à la latitude $\varphi$. 
%
\noindent La conservation de ce moment cinétique
impose à la particule d'air advectée vers
les pôles, donc se rapprochant de son axe de rotation,
d'accélérer en un vent prograde d'altitude (jets d'ouest)
et à la particule d'air ramenée vers l'équateur de
décélérer en un vent rétrograde de surface (alizés).
%%les vents sont créés par transport de moment cinétique
%
La résultante $\vec{F\e{e}}$ des forces 
d'entraînement centrifuge et de Coriolis
%
\[
\vec{F\e{e}} = - m \, (2\,\Omega\,\sin\varphi\,u + \f{u^2\,\tan\varphi}{a}) \, \vec{y}
\]
%
\noindent est alors dirigée vers l'équateur
pour une particule animée d'un vent prograde
$u > 0$,
%tournant plus vite
%que la planète (i.e. animée d'un vent d'ouest ), 
et s'oppose au gradient de pression ayant donné
naissance à la circulation de Hadley, limitant son extension.
%
Au-delà d'une certaine latitude, la conservation du moment cinétique
autour de l'axe planétaire cesse d'être valable
et les circulations non-axisymétriques
prennent une part dominante dans le transport de moment cinétique.
%
%notamment les ondes stationnaires
%et les ondes baroclines.
%
Dans ce cas, le vent ne peut plus être déterminé
quantitativement par conservation du moment cinétique mais
par son lien diagnostique à la structure thermique
(équilibre du vent thermique)
%
\[
- \Dp{\vec{v_H}}{p} = \f{R}{p \, f} \: \vec{z} \wedge \vec{\nabla_p} T
\quad
\textrm{(coord. isobares)}
\]
%
\noindent qui combine l'équilibre vertical hydrostatique  
avec l'équilibre horizontal géostrophique aux moyennes latitudes.
%
Le modèle conceptuel axisymétrique simple de \textit{Held et Hou} [1980]\nocite{Held:80}
se base sur cette distinction entre deux régimes de vent pour en déduire
l'extension latitudinale de la cellule de Hadley $\mathcal{L}$
%et la vitesse
%du vent zonal maximal $\mathcal{U}$
%
%\quad \textrm{et} \quad \mathcal{U} = \f{ \Omega \, \mathcal{L}^2 }{ a }
%
\[
\mathcal{L} = \sqrt{ \f{ 5 \, \Delta \theta \, g \, H }{ 3 \, \Omega^2 \, \theta_0 } }
\]
%
\noindent En utilisant les constantes planétaires de la Terre et Mars,
et les contrastes thermiques typiques
$\Delta \theta\e{T} = 40\U{K}$
et $\Delta \theta\e{M} = 65\U{K}$,
nous obtenons 
$\mathcal{L}\e{M} \sim \mathcal{L}\e{T}$,
soit une cellule de Hadley
significativement plus étendue sur Mars
de rayon deux fois plus petit que celui
de la Terre.
%
Ce modèle reste néanmoins très simplifié et ne s'applique qu'aux 
moyennes annuelles principalement, ce qui en limite fortement
la portée sur des planètes aux saisons très marquées comme Mars.


%La vitesse maximale peut être déterminée
%à partir d'un arbitrage entre le vent
%limite au sens de la conservation du moment cinétique
%et le vent thermique; les valeurs 
%données par \numeroref{eq:heldhou} donnent
%ainsi $\mathcal{U}\e{T} \sim 55\U{m~s^{-1}}$
%et $\mathcal{U}\e{M} \sim 85\U{m~s^{-1}}$, en
%accord correct avec les résultats de modèles
%plus élaborés.




%%L'équilibre du vent thermique donne un 
%%sens des vents conforme à ce qu'on pourrait
%%déduire de la conservation du moment
%%cinétique, sur la base qu'une particule
%%se rapprochant de son axe de rotation
%%est accélérée dans le sens prograde.

