
\sk
Les ondes de gravité sont des oscillations atmosphériques
causés par le rappel de la force de flottaison.
%On se place pour simplifier dans un cas bidimensionnel~$(x,z)$.
L'approximation de Boussinesq est respectée.
L'écoulement est supposé adiabatique.
On réalise une analyse linéaire
\[
\rho =\rho_0 + \rho' 
\qquad
p = \moyenne{p}(z) + p' 
\qquad
\theta = \moyenne{\theta}(z) + \theta'
\qquad
u = \moyenne{u}(z) + u'
\qquad
v = \moyenne{v}(z) + v'
\qquad
w = w'
\]
\noindent où l'état de base vérifie les équations du mouvement.
L'écoulement moyen est supposé purement horizontal, avec~$\moyenne{w}=0$.

\sk
Les équations primitives se réduisent à
\[ 
\Dp{u'}{t} 
+ \moyenne{u} \Dp{u'}{x} 
+ \moyenne{v} \Dp{u'}{x} 
\textcolor{lightgray}{+ w' \Dp{\moyenne{u}}{z}}
- f v' 
+ \f{1}{\rho_0} \Dp{p'}{x}
= 0
\]
\[ 
\Dp{v'}{t} 
+ \moyenne{u} \Dp{v'}{x} 
+ \moyenne{v} \Dp{v'}{x} 
\textcolor{lightgray}{+ w' \Dp{\moyenne{v}}{z}}
+ f u' 
+ \f{1}{\rho_0} \Dp{p'}{y}
= 0
\]
\[ 
\Dp{w'}{t} 
+ \moyenne{u} \Dp{w'}{x} 
+ \moyenne{v} \Dp{w'}{x} 
+ \f{1}{\rho_0} \Dp{p'}{z}
- g \f{\theta'}{\moyenne{\theta}}
= 0
\]
\[ 
\Dp{u'}{x} 
+ \Dp{v'}{y}
+ \Dp{w'}{z}
= 0
\]
%% dernier: terme en D/Dt (rho'/rho_0) ??
\[ 
\Dp{\theta'}{t} 
+ \moyenne{u} \Dp{\theta'}{x} 
+ \moyenne{v} \Dp{\theta'}{x} 
+ w' \Dp{\moyenne{\theta}}{z}
= 0
\]
\noindent Les termes en grisé sont ici négligés pour simplifier. 
Ils représentent la possibilité d'avoir un vent d'environnement
variant avec l'altitude~$z$.


\sk
Pour déterminer les propriétés des ondes de gravité 
qui apparaissent dans l'atmosphère,
on poursuit l'analyse linéaire en substituant
aux perturbations une solution
de type onde monochromatique 
de vecteur d'onde 
$(k=\f{2\pi}{\lambda_x},l=\f{2\pi}{\lambda_y},m=\f{2\pi}{\lambda_z})$ 
et de fréquence absolue $\omega$, 
(autrement dit, un élément harmonique d'un développement de Fourier)
\[
\mathcal{F}'
= 
Re \left[ 
\hat{\mathcal{F}} \, \exi{(kx+ly+mz-\omega t)} 
\right]
\quad \textrm{avec} \quad 
\mathcal{F}' \equiv u', v', w', \f{p'}{\rho_0}, \f{\theta'}{\moyenne{\theta}}
\]
\noindent Ainsi les équations ci-dessus deviennent 
des équations dites de polarisation 
(5 équations pour 5 inconnues $\hat{u} \, \hat{v} \, \hat{w} \, \hat{p} \, \hat{\theta}$), 
où~$\tilde{\omega} = \omega - k \moyenne{u} - l \moyenne{v}$
est la \voc{fréquence intrinsèque} de l'onde considérée
dans un référentiel attaché à l'écoulement moyen
de composantes $\moyenne{u}$ et~$\moyenne{v}$

\begin{minipage}{.32\linewidth}
\begin{equation}\label{polau}
- \ir \, \tilde{\omega} \, \hat{u}
- f \hat{v}
+ \ir \, k \, \hat{p} 
= 0
\end{equation}
\begin{equation}\label{polav}
- \ir \, \tilde{\omega} \, \hat{v}
+ f \hat{u}
+ \ir \, l \, \hat{p} 
= 0
\end{equation}
\end{minipage}
\begin{minipage}{.32\linewidth}
\begin{equation}\label{polaw} 
- \ir \, \tilde{\omega} \, \hat{w}
+ \ir \, m \, \hat{p} 
- g \hat{\theta}
= 0
\end{equation}
\begin{equation}\label{poladiv}
\ir \, k \, \hat{u}
+ \ir \, l \, \hat{v}
+ \ir \, m \, \hat{w}
= 0
\end{equation}
\end{minipage}
\begin{minipage}{.32\linewidth}
\begin{equation}\label{polat}
- \ir \, \tilde{\omega} \, g \, \hat{\theta}
+ N^2 \, \hat{w}
= 0
\end{equation}
$$ \qquad \textrm{avec} \quad N^2 = \f{g}{\moyenne{\theta}} \Dp{\moyenne{\theta}}{z}$$ % = g \Dp{\ln\theta}{z} $$
\end{minipage}

\sk
$N^2$ est appelée la fréquence de Brunt-V{\"a}is{\"a}l{\"a}.

