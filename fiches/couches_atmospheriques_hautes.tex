\sk
\begin{description} 
\item[La \voc{thermosphère}] s'étend jusque des altitudes très élevées (800 km) et voit sa température contrôlée par l'absorption du rayonnement solaire ultraviolet. La température dans la thermosphère varie souvent d'un facteur deux suivant l'activité solaire et l'alternance jour-nuit. Les aurores surviennent dans cette couche atmosphérique. Les missions spatiales \ofg{basse orbite} telles que la Station Spatiale Internationale sont localisées au milieu de la thermosphère. \normalsize
\item[L'\voc{exosphère}] est située au-dessus de la thermosphère à partir d'une altitude d'environ~$800$~km sur Terre. Il s'agit de la zone où l'atmosphère subit un \voc{échappement} : les molécules peuvent s'échapper vers l'espace sans que des chocs avec d'autres molécules ne les renvoient dans l'atmosphère. L'exosphère constitue la dernière zone de transition entre l'atmosphère et l'espace. \normalsize
\end{description}

\figside{0.45}{0.3}{\figpayan/LP211_Chap1_Page_05_Image_0001.png}{Structure verticale idéalisée de la température étendue aux hautes atmosphères.}{fig:tempvert}
%% Voir figure~\ref{fig:presvert} pour la distinction entre figure de gauche et figure de droite

\sk
D'autres couches atmosphériques sont définies non pas à partir de la température mais à partir des propriétés électriques de l'atmosphère terrestre. On fait référence ici au vent solaire, qui est un flux de particules chargées (ions et électrons) formant un plasma qui s’échappe en permanence du Soleil vers l’espace interplanétaire
\begin{description}
\item[L'ionosphère] Comme son étymologie l'indique, l’ionosphère est une région de notre haute atmosphère contenant des ions et des électrons formés par photo-ionisation des molécules neutres qui s’y trouvent. C’est le Soleil, et plus particulièrement ses rayonnements énergétiques ultraviolets et X, mais aussi les particules du vent solaire et le rayonnement cosmique, qui sont à l’origine de cette ionisation de la haute atmosphère. L’ionosphère se situe entre~$50$ et~$1000$ kilomètres d’altitude (elle s'étend donc de la mésosphère à la thermosphère). L’ionosphère est habituellement divisée horizontalement en différentes couches, baptisées D, E et F dans lesquelles l’ionisation croît avec l’altitude. Ces couches proviennent des différences de pénétration dans l’atmosphère des rayonnements solaires selon leur énergie.
\item[La magnétosphère] Notre planète génère son propre champ magnétique, un peu à la manière d’une dynamo. C’est la différence de vitesse entre la rotation de la planète et de son coeur liquide qui, par induction, génère ce champ magnétique. Ce champ magnétique protège la Terre des agressions extérieures comme les rayons cosmiques et les particules énergétiques du vent solaire. Cette zone protégée s'appelle magnétosphère. Elle démarre au dessus de l’ionosphère, à plusieurs milliers de kilomètres de la surface du sol, et s’étend jusqu’à 70 000 kilomètres environ du côté du Soleil. Du côté opposé, la queue de la magnétosphère s’étire sur plusieurs millions de kilomètres. Les contours de la magnétosphère évoluent continuellement sous l’action du vent solaire et de sa variabilité.
\end{description}
\normalsize

