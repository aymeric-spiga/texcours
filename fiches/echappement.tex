\sk
La région de l'atmosphère d'une planète soumise
à l'échappement est appelée \voc{exosphère} et sa base
est l'exobase. Plus précisément, l'\voc{exobase} est
définie comme l'altitude à laquelle le libre
parcours moyen d'une particule (distance qu'elle peut parcourir sans
entrer en collision avec une autre particule)
est égal à une échelle de hauteur atmosphérique.
Sur Mercure ou la Lune, l'exobase est confondue avec la surface;
sur les planètes telluriques, elle est à quelques centaines de kilomètres de la surface.
La définition de la pression et de la température y sont sujettes
à caution.

\sk
Les vitesses~$v$ des molécules de masse~$m$ 
composant un gaz avec une densité particulaire~$N$,
en équilibre thermique à la température~$T$, 
suivent une distribution maxwellienne
\[ f(v) \dd v = N 
\, \left( \frac{2}{\pi} \right)^{1/2}
\, \left( \frac{m}{k_B \, T} \right)^{3/2}
v^2 \, e^{ \frac{-m\,v^2}{2\,k_B\,T}} \, \dd v \]
\noindent En dessous de l'exobase, les collisions entre molécules
assurent le caractère maxwellien de la distribution.
Au dessus de l'exobase, les collisions sont beaucoup
plus rares (trajectoires ballistiques) et les molécules 
dans la queue de la distribution
maxwellienne ayant une vitesse~$v > v_e$ peuvent s'échapper.
La distribution maxwellienne des vitesses s'étend
jusque l'infini, mais la distribution des molécules
est plutôt gaussienne, ce qui implique en pratique
que quasiment aucune molécule avec des vitesses
plus grandes qu'environ 4 fois la vitesse thermique moyenne
\textcolor{red}{$v_0=\sqrt{\frac{2 k_B T}{m}}$}.

\sk
Pour quantifier au premier ordre à quel point une planète
peut retenir son atmosphère, on définit le paramètre d'échappement~$\lambda_e$,
rapport entre
vitesse d'échappement~\textcolor{magenta}{$v\e{e} = \sqrt{\frac{2\,\mathcal{G}\,M\e{p}}{R+h}}$}
(vitesse d'une molécule permettant de se libérer 
de l'attraction gravitationnelle de la planète)
et vitesse thermique moyenne~$v_0$
\[ \lambda\e{e} = \left( \frac{v_{e}}{v_0} \right)^2 = \frac{\mathcal{G}\,M\e{p}\,m}{k_B\,T\,(R+z)} \] %= \frac{R+z}{H(z)}
\noindent Le paramètre d'échappement peut également être vu comme
le rapport entre énergie potentielle et cinétique.
Si le paramètre d'échappement~$\lambda\e{e}$ est significativement plus petit que 1 ($v_{e} < v_0$), 
les molécules considérées sont susceptibles de s'échapper en nombre.
Ainsi, seules les planètes froides et massives 
peuvent retenir les éléments légers comme l'hydrogène.

\sk
En intégrant le flux de molécules avec une distribution
maxwellienne des vitesses au-dessus de l'exobase,
avec l'approximation d'une atmosphère en équilibre hydrostatique,
on obtient le taux d'échappement thermique de Jeans (atomes par cm$^2$ par s)
\[ \Phi_J = \frac{N_{ex} \, v_0}{2\,\sqrt{\pi}} \, (1+\lambda_e) \, e^{-\lambda_e} \]
\noindent avec les indices $\emph{ex}$ pour l'exobase. Les paramètres typiques pour la Terre : $N\e{ex} = 10^5$~cm$^{-3}$, $T\e{ex} = 900$~K, $\lambda\e{e} \simeq 8$ pour l'hydrogène atomique, donnent~$\Phi\e{J} \simeq 6 \times 10^7$~cm$^{-2}$~s$^{-1}$.
%Le flux hors la planète ne peut dépasser le flux limite imposé par la diffusion.
%
Les éléments ou isotopes les plus légers sont perdus par échappement
à un taux bien plus élevé ; l'échappement de Jeans peut donc
générer un fractionnement isotopique significatif.
La présence ou non d'une atmosphère peut être prédite (au premier ordre)
en évaluant le taux d'échappement de Jeans.
Le temps caractéristique de stabilité d'une atmosphère peut
être obtenu par~$\tau\e{e} = \frac{H \, N}{\Phi\e{e}}$.


%Vitesse thermique $V_{th}$: Vitesse d'une molécule atmosphérique d'énergie cinétique~$\frac{3}{2}kT$
%$$V_{th}=\sqrt{\frac{3RT}{M}}$$
%\begin{itemize} \item \'Echappement thermique \begin{itemize} \item Négligeable uniquement si $\lambda = \left( \frac{V_{e}}{V_{th}}   \right) \gg 1$ \item Flux de Jeans : $\Phi_e = \frac{n_c}{2\sqrt{\pi}} V_{th} (1+\lambda) \exp (-\lambda)$ \item Inefficace sur Vénus \item Echappement de H, H$_2$ et D pour Mars, Titan et la Terre. \end{itemize} \item \'Echappement non thermique \begin{itemize} \item Processus majoritaires \item \'Echappement d'ions positifs excités. \item Excitations et ionisation par collisions (avec autres atomes ou électrons libres) et/ou photochimie. \end{itemize} \end{itemize}
%\begin{tabular}{|c|c|c|c|c|}
%\hline
%                & Vénus & Terre & Mars & Titan \\
%\hline
%$V_{e}$ [km/s]  & 10,3  & 11,2  & 5,0  & 2,6   \\
%\hline
%$V_{th}$ [km/s] & 2,2   & 4,1   & 2,4  & 1,8   \\
%\hline
%\end{tabular}


\figside{0.6}{0.15}{decouverte/cours_gqe/escape.png}{McBride and Gilmour, An Introduction to the Solar System, 2004}{echapplanete}


















%% voir slides Emmanuel


