\sk
Les variations verticales de température sont très différentes des variations de pression et de densité: la température décroît et augmente alternativement avec l'altitude%, de façon quasi-linéaire [figure \ref{fig:tempvert}], en restant comprise entre environ~$200$ et~$300$~K. 
Cette structure verticale de la température permet de diviser l'atmosphère en un certain nombre de couches aux propriétés différentes, dont les noms comportent le suffixe \emph{-sphère}. La limite supérieure d'une couche atmosphérique donnée porte un nom similaire, où le suffixe \emph{-sphère} est remplacé par le suffixe \emph{-pause}. Par exemple, la limite entre la troposphère et la stratosphère s'appelle la \voc{tropopause}. Les couches atmosphériques en partant de la surface vers l'espace sont détaillées ci-dessous.

\sk
\begin{description} 
\item[La \voc{troposphère}] \normalsize s'étend jusqu'à environ 11 km d'altitude et contient 80\% de la masse de l'atmosphère. La température y décroit en moyenne de 6.5\deg C par kilomètre (nous verrons pourquoi dans un chapitre ultérieur). La troposphère est une couche relativement bien mélangée sur la verticale (échelle de temps de quelques jours), sauf en certaines couches minces, appelées \voc{inversions}, où la température décroit peu ou même augmente avec l'altitude. La troposphère est la couche où ont lieu la plupart des phénomènes météorologiques acessibles à l'expérience humaine (par exemple, les nuages montrés en figure~\ref{fig:blue}). La partie inférieure de la troposphère contient la \voc{couche limite atmosphérique} située juste au dessus de la surface, d'épaisseur variant de quelques centaines de~m à 3 km et définie comme la partie de l'atmosphère influencée par la surface sur de courtes échelles de temps (typiquement un cycle diurne). La couche limite atmosphérique est le siège de mouvements turbulents organisés au cours de l'après-midi qui opèrent un mélange des espèces chimiques qui y sont émises. \normalsize
\item[La \voc{stratosphère}] \normalsize est située au dessus de la troposphère. L'altitude au-dessus du sol de la tropopause peut varier entre~$5$ et~$15$~km. Contrairement à la troposphère, la stratosphère contient très peu de vapeur d'eau (à cause des températures très basses rencontrées à la tropopause) mais la majorité de l'ozone~O$_3$. L'absorption par l'ozone du rayonnement solaire \voc{ultraviolet}, de longueur d'onde moindre que le rayonnement visible et plus énergétique, explique que la température dans la stratosphère est d'abord isotherme, puis augmente avec l'altitude jusqu'à un maximum à la stratopause. Cette structure verticale très stable inhibe fortement les mouvements verticaux, ce qui explique que la stratosphère soit organisée en couches horizontales (comme l'indique l'étymologie de son nom). Le temps de résidence de particules dans la stratosphère est très long à cause de l'absence de nuages et précipitations. \normalsize
\item[La \voc{mésosphère}] \normalsize voit sa température décroître selon la verticale. Contrairement à la troposphère, elle ne contient pas de vapeur d'eau et contrairement à la stratosphère, elle ne contient que peu d'ozone. Elle se situe sur Terre à des altitudes entre~$50$ et~$85$~km. La mésopause est souvent le point le plus froid de l'atmosphère terrestre, la température peut y atteindre des valeurs aussi basses que~$130$~K. \normalsize
\end{description}

\figside{0.45}{0.3}{\figfrancis/WH_stdatm}{Structure verticale idéalisée de la température correspondant au profil moyenné global annuel.}{fig:tempvert}
