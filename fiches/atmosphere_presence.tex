\sk
La figure~\ref{fig:blue} illustre la présence d'une atmosphère très active sur Terre par les nuages qui y prennent naissance. Il ne s'agit que d'un exemple parmi tant d'autres pour appréhender l'atmosphère. C'est le but de ce chapitre d'introduction d'évoquer la diversité des points de vue pouvant être adoptés pour étudier l'atmosphère, un système complexe où se mêlent processus physiques, dynamiques, chimiques, biologiques, et même sociétaux. Sont également abordées dans ce chapitre quelques notions de base nécessaires pour la suite du cours.

\figsup{0.45}{0.25}{decouverte/cours_meteo/blue_50.png}{decouverte/cours_meteo/blueclouds_50.png}{La planète Terre avec et sans les nuages de son atmosphère. Les nuages couvrent très souvent au moins la moitié du globe. Construit d'après une image \ofg{Blue Marble} NASA du projet \ofg{Visible Earth}. Des versions haute-résolution des images planes et des explications complètes peuvent être trouvées aux adresses suivantes \url{http://visibleearth.nasa.gov/view_rec.php?id=2430} et \url{http://visibleearth.nasa.gov/view_rec.php?id=2431}.}{fig:blue} %%%http://visibleearth.nasa.gov/view_rec.php?id=2429

