\sk
Les ondes de Rossby sont des ondes atmosphériques planétaires causées par 
le paramètre de Coriolis variant avec la latitude,
sous contrainte de conservation de la vorticité totale~$\xi = \zeta + f$
($f=2\Omega\sin\phi$ la vorticité planétaire).

%\sk
\paragraph{Approche qualitative} Considérons une parcelle d'air à l'équateur, de vorticité relative~$\zeta = 0$.
Admettons qu'une perturbation déplace légèrement cette parcelle vers le nord.
La vorticité planétaire~$f$ de cette parcelle augmente.
La conservation de~$\xi = \zeta + f$ impose donc
que la vorticité relative~$\zeta$ de la parcelle devienne négative,
ce qui impose un mouvement tournant de sens des aiguilles d'une montre
au voisinage de la parcelle.
A gauche de la parcelle, un déplacement vers le nord se fait
donc sentir, donnant naissance à une nouvelle perturbation~$\zeta < 0$ par conservation de~$\xi$.
A droite de la parcelle, un déplacement vers le sud se fait
donc sentir, donnant naissance à une nouvelle perturbation~$\zeta > 0$ par conservation de~$\xi$.
Il en résulte ainsi une propagation conjointe vers l'ouest 
des perturbations~$\zeta > 0$ et~$\zeta < 0$,
caractérisant les \voc{ondes de Rossby}.

%\sk
\paragraph{Approche quantitative}
Ce résultat qualitatif peut se retrouver quantitativement
dans un contexte barotrope quasi-horizontal non-divergent,
en utilisant la composante verticale de la vorticité absolue~$\xi = \zeta + f$
qui est conservée en suivant le mouvement horizontal
\[ 
\ddf{\xi}{t} = 0 
\qquad
\textrm{avec}
\qquad
\ddf{}{t} = \Dp{}{t} + u \Dp{}{x} + v \Dp{}{y}
\]
Pour illustrer aisément l'effet de~$\beta = \Dp{f}{y}$,
on se place dans un plan~$\beta$ tangent à la sphère localement
et pour lequel~$\beta = \cte$ (ainsi $f = f_0 + \beta\,y$).
On considère, pour réaliser une \voc{analyse linéaire},
un état de base~$\moyenne{\cdot}$ (écoulement zonal)
et une perturbation horizontale~$\cdot'$
\[
u = \moyenne{u} + u' 
\qquad 
v = v' 
\qquad 
\zeta = \Dp{v'}{x} - \Dp{u'}{y} = \zeta'
\]
\noindent L'équation barotrope de la vorticité s'écrit ainsi en version linéarisée
\[ 
\left( \Dp{}{t} + u \Dp{}{x} + v \Dp{}{y} \right) \zeta + \beta v = 0
\qquad
\Rightarrow
\qquad
\left( \Dp{}{t} + \moyenne{u} \Dp{}{x} \right) \zeta' + \beta v' = 0
\]

\sk
Dans le cas d'un écoulement bidimensionnel non divergent,
il est possible d'exprimer les coordonnées de la vitesse
à l'aide d'une \voc{fonction de courant}~$\Psi'$ telle que
%\[
%u' = - \Dp{\Psi'}{y} \qquad v' = \Dp{\Psi'}{x}
%\]
$u' = - \partial \Psi' / \partial y$ et $v' = \partial \Psi' / \partial x$
\noindent possédant la propriété intéressante suivante
$\zeta' = \nabla^2 \Psi'$ avec 
$\nabla^2 \equiv 
\partial^2 / \partial x^2 
+
\partial^2 / \partial y^2$
%\noindent 
ce qui permet d'exprimer l'équation linéarisée de la vorticité barotrope
en fonction de la seule variable~$\Psi'$
\[
\left( \Dp{}{t} + \moyenne{u} \Dp{}{x} \right) \nabla^2 \Psi' + \beta \Dp{\Psi'}{x} = 0
\]
\noindent On cherche des \voc{solutions harmoniques}
de type onde monochromatique 
de vecteur d'onde horizontal~$(k,l)$ 
et de fréquence absolue $\omega$
s'écrivant
\[
\Psi' = Re \left[ \hat{\Psi} \, \exi{(kx+ly-\omega t)} \right]
\]
\noindent D'après l'équation linéarisée, les ondes
qui se propagent vérifient ainsi
$ \left( - \omega + k \moyenne{u} \right) \left( - k^2 - l^2 \right) + k\,\beta = 0 $
et donc leur fréquence intrinsèque~$\tilde{\omega} = \omega - k\moyenne{u}$ 
dans le référentiel attaché à l'écoulement de base est
\[
\boxed{
\tilde{\omega} = - \beta \, \f{k}{k^2+l^2}
}
\]
\noindent La relation de dispersion ainsi obtenue pour
les ondes de Rossby traduit à la fois leur
propagation systématique vers l'ouest
et l'importance centrale de l'effet~$\beta$ 
de variation du paramètre de Coriolis avec la latitude.
%
Les ondes de Rossby se développent sur Terre
dans les moyennes latitudes avec un nombre d'onde~$4-6$.
Elles peuvent également être responsables de structures 
sur les planètes géantes, telles l'hexagone sur Saturne par exemple.
La forme en Y du nuage global de Vénus pourrait être liée
à une onde de Rossby, mais d'autres types d'ondes 
(par exemple, ondes de Kelvin) pourrait être impliquées.
