\sk
Dans le cas d'un nombre de Rossby petit (donc $L$>1000~km aux moyennes latitudes), on est proche d'un équilibre appelé \voc{équilibre géostrophique} entre les forces de Coriolis et de pression
\[ \boxed{ \v F_C+\v F_P=\v 0 } \]
qui s'écrit selon les deux composantes horizontales
\[ \boxed{ \binom{f \, v}{-f \, u} = \binom{\frac{1}{\rho} \,\frac{\partial P}{\partial x}}{\frac{1}{\rho} \,\frac{\partial P}{\partial y}} } \]
Le vent qui vérifie exactement cet équilibre est appelé \voc{vent géostrophique}~$\v V_g$. Sous forme vectorielle on a $f\v k\wedge\v V_g=\v F_P$ et sous forme projetée
\[ \v V_g = \binom{u}{v} = \binom{- \frac{1}{\rho \, f} \, \frac{\partial P}{\partial y}}{\frac{1}{\rho \, f} \, \frac{\partial P}{\partial x}} \]
%\begin{equation}
%  \v V_g=\frac{1}{\rho f}\v k\wedge\vl{grad}_z(P)=\frac{g}{f}\v k\wedge\vl{grad}_P(Z)
%  \label{eq:geost}
%\end{equation}

\figun{1.1}{0.3}{\figfrancis/geost}{Forces et vent dans l'équilibre géostrophique (hémisphère nord).}{fig:geost}

\sk
L'équilibre géostrophique peut s'illustrer graphiquement (voir figure~\ref{fig:geost}), formant ce que l'on appelle la loi de Buys-Ballot\footnote{Comme l'indique Buys-Ballot dans son article de 1857~: \emph{Note sur le rapport de l'intensité et de la direction du vent avec les écarts simultanés du baromètre ; [...] Ce n'est pas la girouette, mais c'est le baromètre d'après lequel on doit juger le vent [...] La grande force du vent est annoncée par une grande différence des écarts simultanés du baromètre dans les Pays-Bas [...] Pour un autre pays, on devra étudier les modifications.}}.
\begin{description}
\item[Direction] Comme la force de Coriolis est orthogonale au vecteur vitesse, et opposée à la force de pression, le vent géostrophique est lui-même orthogonal aux variations horizontales de pression donc parallèle aux isobares.
\item[Sens] Dans l'hémisphère nord, les basses pressions sont à gauche du vent, à droite dans l'hémisphère sud.
\item[Module] La vitesse du vent géostrophique est proportionnelle aux variations horizontales de pression~; autrement dit, plus les isobares sont resserrées, plus le vent est fort.
\end{description}

