\sk
\begin{finger}
\item Dans toute discussion de la composition atmosphérique, il est important de faire la distinction entre composés minoritaires et majoritaires [figure~\ref{fig:minor}]. Alors que les composés majoritaires suivent une distribution verticale en accord avec l'état énergétique et dynamique de l'atmosphère globale, les composés minoritaires peuvent avoir des comportements très différents qui dépendent à la fois des mécanismes photochimiques de production et de perte ainsi que des phénomènes de transport.
\item La composition de l'air donnée ici est valide sur les premiers~$80$ à~$100$ kilomètres d'altitude, à part quelques constituants mineurs. On appelle cette région l'\voc{homosphère}, elle correspond approximativement à la troposphère, la stratosphère et la mésosphère (Figure \ref{fig:tempvert}). Dans l'homosphère, l'atmosphère est un mélange homogène de différents gaz. Au dessus de cette altitude, le libre parcours moyen des molécules devient très grand et on a une \ofg{décantation} où les éléments plus légers dominent progressivement aux altitudes élevées. On parle alors d'\voc{hétérosphère}; elle regroupe approximativement la thermosphère et l'exosphère. 
\end{finger}
%Au niveau du sol, l'atmosphère standard sèche est caractérisée par une pression d’environ 1013 hPa et une concentration totale de 2,69 x 1019 molécule~cm$^{-3}$ lorsque la température est de 273~K.

\figun{0.9}{0.4}{\figpayan/LP211_Chap1_Page_06_Image_0001.png}{Composition de l’atmosphère~: des espèces en très faibles quantités jouent un rôle très important. Sur la figure sont données quelques mesures de constituants minoritaires dans l’homosphère. Les courbes en trait fin correspondent aux concentrations résultant de rapports de mélange volumiques constants de~$10^{-1}$ à~$10^{-13}$. (Source: Kockarts, Aéronomie, 2000).}{fig:minor}

