\sk
\begin{finger}
\item Dans toute discussion de la composition atmosphérique, il est important de faire la distinction entre composés minoritaires et majoritaires [figure~\ref{fig:minor}]. Alors que les composés majoritaires suivent une distribution verticale en accord avec l'état énergétique et dynamique de l'atmosphère globale, les composés minoritaires peuvent avoir des comportements très différents qui dépendent à la fois des mécanismes photochimiques de production et de perte ainsi que des phénomènes de transport.
\item La composition de l'air donnée ici est valide sur les premiers~$80$ à~$100$ kilomètres d'altitude, à part quelques constituants mineurs. On appelle cette région l'\voc{homosphère}, elle correspond approximativement à la troposphère, la stratosphère et la mésosphère (Figure \ref{fig:tempvert}). Dans l'homosphère, l'atmosphère est un mélange homogène de différents gaz, l'échelle de hauteur est la même pour tous les gaz. Au dessus de cette altitude, le libre parcours moyen des molécules devient très grand et on a une \ofg{décantation} où les éléments plus légers dominent progressivement aux altitudes élevées. Chaque composant suit sa propre échelle de hauteur. On parle alors d'\voc{hétérosphère}; elle regroupe approximativement la thermosphère et l'exosphère. 
\end{finger}
%Au niveau du sol, l'atmosphère standard sèche est caractérisée par une pression d’environ 1013 hPa et une concentration totale de 2,69 x 1019 molécule~cm$^{-3}$ lorsque la température est de 273~K.

\figun{0.7}{0.3}{\figpayan/LP211_Chap1_Page_06_Image_0001.png}{Composition de l’atmosphère~: des espèces en très faibles quantités jouent un rôle très important. Sur la figure sont données quelques mesures de constituants minoritaires dans l’homosphère. Les courbes en trait fin correspondent aux concentrations résultant de rapports de mélange volumiques constants de~$10^{-1}$ à~$10^{-13}$. (Source: Kockarts, Aéronomie, 2000).}{fig:minor}






\begin{itemize}
\item Fractionnement isotopique: Homosphère-hétérosphère
\begin{itemize}
\item A haute altitude $z>z_h$, la diffusion turbulente est moins efficace que la \voc{diffusion moléculaire} : \voc{hétérosphère}
\item Homopause : $z_h \sim 100\,km$ (telluriques), $\sim 750\,km$ (Titan)
\item Chaque espèce suit alors sa propre échelle de hauteur : les plus légères deviennent plus abondantes à haute altitude
\end{itemize}
\item 2 facteurs influencent le fractionnement
{\footnotesize \begin{description}
\item[\'Equilibre diffusif] : différence de composition atmosphérique entre l'homopause et l'exobase
\item[\'Echappement différentiel] : $V_{th}/V_e$ plus grand pour les isotopes légers.
\end{description} }
%\item Exemples
%\begin{itemize}
%\item {\bf D/H} (Vénus)
%\item {\bf $^{14}$N/$^{15}$N}, {\bf $^{16}$O/$^{18}$O} (Mars)
%\end{itemize}
\end{itemize}


Note: Diffusion turbulente, paramétrisée par un coefficient de diffusion turbulente qui le fait ressembler à un paramètre de diffusion moléculaire.
mais provient de la convection thermique (flottaison) dans la basse atmosphère et les ondes de gravité dans la haute atmosphère
