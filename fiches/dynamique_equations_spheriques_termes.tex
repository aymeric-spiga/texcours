\sk
L'équation fondamentale de la dynamique des fluides géophysiques 
en projection sur les coordonnées sphériques avec l'approximation de couche mince
s'écrit finalement

\begin{center}
\begin{tabular}{ccccccccc}
%%%%%%
\textcolor{blue}{$\ddf{u}{t}$} & 
\textcolor{brown}{$-\dfrac{uv\tan\phi}{a}$} & 
$+\dfrac{uw}{a}$ & 
= & 
\textcolor{red}{$2\Omega\sin\phi v$} & 
$-2\Omega \cos\phi w$ & 
\textcolor{green!75!black}{$-\dfrac{1}{\rho}\Dp{p}{x}$} & 
& 
$+Fr_x$\\
~\\
%%%%%%
\textcolor{blue}{$\ddf{v}{t}$} & 
\textcolor{brown}{$+\dfrac{u^2\tan\phi}{a}$} & 
$+\dfrac{vw}{a}$ & 
= & 
\textcolor{red}{$-2\Omega\sin\phi u$} &
&
\textcolor{green!75!black}{$-\dfrac{1}{\rho}\Dp{p}{y}$} &
&
$+Fr_y$\\
~\\
%%%%%%
$\ddf{w}{t}$ & 
$-\dfrac{u^2+v^2}{a}$ & 
&
=&
$2\Omega\cos\phi u$ & 
& 
\textcolor{green!75!black}{$-\dfrac{1}{\rho}\Dp{p}{z}$} & 
\textcolor{green!75!black}{$ -g$} & 
$+Fr_z$\\ 
\end{tabular}
\end{center}

\sk
Les termes s'interprètent comme suit
\begin{citemize}
\item{Pression} \textcolor{green!75!black}{$\bullet$} 
\item{Coriolis} \textcolor{red}{$\bullet$} où l'on définit~$f = 2 \, \Omega \, \sin\phi$
\item{Inertiels (sphéricité)} \textcolor{brown}{$\bullet$}
\item{Inertials (accélération)} \textcolor{blue}{$\bullet$}
\end{citemize}
Les termes en noir sont liés au déplacements verticaux et sont en général négligeables quand on considère la circulation générale de l'atmosphère et de l'océan.
