\sk
La contribution énergétique d'une interface obéit à l'équation de Kelvin.

\sk
Energie libre de Gibbs (à minimiser). Nucléation particule rayon~$R$, premier terme est le travail nécessaire pour former l'interface et second terme l'échange d'énergie associé aux molécules de vapeur allant vers la phase condensée
\[ 
\Delta G = 
\underbrace{\textcolor{white}{-\frac{4}{3}} 4 \, \pi \, R^2 \, \Sigma \textcolor{white}{\ln \frac{e}{e\e{sat}}}}_{\text{création de l'interface}} 
\quad
\underbrace{- \frac{4}{3} \, \pi \, R^3 \, n \, k \, T \, \ln \frac{e}{e\e{sat}}}_{\text{changement d'état}}
\]

L'équation montre que l'énergie libre de Gibbs dépend de l'humidité de l'atmosphère entourant la surface de la particule. En dessous de la saturation, le logarithme est négatif ou zéro et $\Delta G$ est une fonction croissante avec le rayon, mais pour des valeurs en sursaturation $e>e\e{sat}$ le logarithme est négatif et la fonction de Gibbs a un maximum.

Formule de Kelvin pour un rayon critique~$R\e{c}$
\[
R\e{c} = \frac{2 \, \Sigma}{n\,k\,T\,\ln \frac{e}{e\e{sat}}}
\]
noter la forte variabilité en l'humidité relative

La stabilité de petites gouttes requiert des conditions fortement sur-saturées ($e \gg e\e{sat}$).
