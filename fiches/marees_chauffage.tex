

\sk
Les forces de marées peuvent conduire à un chauffage interne des corps,
car elles produisent des déformations (flexures) à l'intérieur de ces corps
et les mouvements relatifs de ces masses produisent des frottements internes
qui induisent le chauffage de friction.
Autrement dit, via les forces de marée, l'énergie orbitale et de rotation propre sont dissipées par chauffage frictionnel dans l'intérieur du corps.
Lorsque la contraction gravitationnelle ou la radioactivité ne conduisent
pas à des flux dominants (comme c'est le cas respectivement sur les planètes géantes et sur Terre),
le chauffage par les forces de marées
peut être la source de chaleur principale à l'intérieur des corps.
Ainsi, le chauffage par les forces de marée de Jupiter et Saturne sur 
les lunes Io, Europe, Encelade est dominant.
Le taux~$\dot{E}$ de dissipation d'énergie par les marées peut s'écrire
selon la littérature (Showman et al. 1997, citant Peale et Cassen 1978)
\[ \mathcal{P} = \frac{21}{2} \, \frac{k_T}{Q} \, \frac{R^5\,\mathcal{G}M^2\,\Omega}{a^6} \, e^2 \]
\noindent Il convient de remarquer que cette formule est valable
pour les faibles excentricités.
%%%% Peale S.J. & Cassen P. "Melting of Io by tidal dissipation", Science 203, 892 (1979)
%Peale and Cassen \[ \ddf{E}{t} = \frac{36}{19} \, \frac{\pi\,\rho^2\,n^5\,R^7\,e^2}{\mu\,Q} \]
%%% chapitre 4 thèse Jérémy

\sk
Le chauffage par les forces de marée est un mécanisme central pour
expliquer un chauffage interne au sein de certains satellites
qui possède une atmosphère fine. L'exemple le plus frappant
est Io satellite de Jupiter dont l'activité volcanique
(source principale de sa fine atmosphère)
est causé par le chauffage interne par les forces de 
marées\footnote{Les fortes marées devraient également provoquer
une circularisation de l'orbite de Io, mais ce n'est pas le cas
en raison des perturbations causées par les autres satellites galiléens}.
Le chauffage par les marées est également la source d'énergie
principale d'Europe et Ganymède, satellite de Jupiter -- dans les
deux cas, il est responsable de la présence d'océans souterrains.
Enfin, le chauffage des marées est le meilleur candidate pour
expliquer les panaches observés sur Encelade (lune de Saturne)
et Triton (lune de Neptune).









