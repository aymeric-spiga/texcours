\sk
Les conditions atmosphériques sont très instables proche d'une surface (en présence d'une telle surface). A cause de la discontinuité entre surface et atmosphère, sous l'action de la diffusion thermique, ou turbulente, entre la surface (chaude) et l'air immédiatement adjacent (plus froid) crée une couche d'air fine approximativement à la température de la surface ; les conditions de température étant plus froides au-dessus, les conditions atmosphériques sont très instables proche de la surface et des mouvements de convection vont se mettre en place pour mélanger l'air sur une certaine épaisseur atmosphérique. Un équilibre dit \voc{radiatif-convectif} prévaut, avec une structure thermique suivant le profil adiabatique, donnant naissance à une troposphère. Au-dessus de la limite radiative-convective (correspondant peu ou prou à la tropopause), les phénomènes radiatifs dominent et donnent naissance à une mésosphère -- ou une stratosphère si un absorbant visible y est présent en quantité suffisante, donnant naissance à une inversion stable à la tropopause.
%% on passait en troposphère dès que le gradient du profil radiatif dépassait celui du profil adiabatique (-g/cp)

\figside{0.4}{0.2}{decouverte/pierrehumbert_pics/9780521865562c03_fig014.jpg}{Figure tirée de R. Pierrehumbert, Principles of Planetary Climates, CUP, 2010}{fig:effetserre2}









