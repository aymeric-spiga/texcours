\sk
Un fluide de Boussinesq est un fluide pour lequel les variations de densité peuvent être négligées 
sauf lorsqu'elles apparaissent dans des termes dont $g$, l'accélération de la gravité, est facteur.
La plupart des phénomènes atmosphériques, 
qu'ils soient de grande échelle (mouvement quasi-géostrophique),
de méso-échelle (fronts météorologiques, vents catabatiques, ondes de gravité)
ou 
de micro-échelle (convection de couche limite)
peuvent être considérés comme des mouvements
d'un fluide de Boussinesq, pour lequel $\rho=\cte$
sauf dans le terme de flottabilité verticale.
En d'autres termes, on ne retient que l'impact sous l'action de la gravité
de la stratification en densité du fluide sur les mouvements verticaux.
L'équation de continuité s'écrit alors
\[ \v \nabla \cdot \v v = 0 \]
\noindent Cette approximation est valable dans le cas de mouvements  
verticaux relativement confinés (d'extension verticale de l'ordre de $H$).

\sk
On se place dans le cadre d'une \voc{analyse linéaire} de perturbations
autour d'un état de base moyen qui vérifie l'équilibre hydrostatique.
\[ \rho = \rho_0 + \rho' \qquad p = \moyenne{p}(z) + p' \]
L'approximation de Boussinesq permet 
de remplacer~$\rho$ par~$\rho_0$ partout, sauf 
lorsqu'on considère les deux termes dominants 
de l'équation du mouvement vertical -- car ils font apparaître la force de flottaison.
\[ 
\f{1}{\rho} \Dp{p}{z} + g 
= 
\f{1}{\rho_0 + \rho'} \left( \Dp{\moyenne{p}}{z} + \Dp{p'}{z} \right) + g  
\]
\noindent Au premier ordre
\[ 
\f{1}{\rho_0 + \rho'} 
= 
\f{1}{\rho_0 \left( 1 + \f{\rho'}{\rho_0} \right) } 
\simeq 
\f{1}{\rho_0} \left( 1 - \f{\rho'}{\rho_0} \right)
\]
\noindent De plus, l'état de base vérifie l'équilibre hydrostatique
\[ 
\f{1}{\rho_0} \Dp{\moyenne{p}}{z} + g = 0
\]
\noindent Ainsi on parvient à
\[ 
\f{1}{\rho} \Dp{p}{z} + g 
= 
\f{1}{\rho_0} \Dp{p'}{z} + g \f{\rho'}{\rho_0}
\]
\noindent Le second terme est l'expression de la force de flottaison (la poussée d'Archimède).
En utilisant l'équation d'état, et en introduisant le température
potentielle~$\theta = \moyenne{\theta} + \theta'$, on a
\[
\rho' \simeq - \rho_0 \f{\theta'}{\moyenne{\theta}} + \f{p'}{c_s^2}
\]
\noindent avec~$c_s$ la célérité du son, bien plus rapide que
l'écrasante majorité des mouvements atmosphériques, ce qui donne
une autre expression de la force de flottaison
\[
-\f{\rho'}{\moyenne{\rho}} = \f{\theta'}{\moyenne{\theta}}
\]
\noindent Ce qui montre qu'on peut raisonner indifféremment
sur les perturbations de densité ou de température potentielle.
Les premières sont plutôt utilisées en océanographie,
les secondes en météorologie.

%force de flottaison est une perturbation de l'équilibre hydrostatique.
