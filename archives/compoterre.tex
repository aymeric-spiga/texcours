\begin{table}
\centering
\begin{tabular}{lcc}
%\toprule
%\makecell[l]{\textbf{Constituant}} &
%\makecell{\textbf{Masse}\\\textbf{Molaire}} &
%\makecell{\textbf{Rapport}\\\textbf{de Mélange}} \\
%\midrule
\hline
\textbf{Constituant} &
\textbf{Masse molaire} &
\textbf{Rapport de mélange} \\
\hline
Azote (N$_2$) & 28 & 78\% \\
Oxygène (O$_2$) & 32 & 21\% \\
Argon (Ar) & 40 & 0.93\% \\
\textbf{Vapeur d'eau (H$_2$O)} & 18 & 0-5\% \\
\textbf{Dioxyde de Carbone (CO$_2$)} & 44 & 380 ppmv \\
Néon (Ne) & 20 & 18 ppmv \\
Hélium (He) & 4 & 5 ppmv \\
\textbf{Méthane (CH$_4$)} & 16 & 1.75 ppmv \\
Krypton (Kr) & 84 & 1 ppmv \\
Hydrogène (H$_2$) & 2 & 0.5 ppmv \\
\textbf{Oxide nitreux (N$_2$O)} & 56 & 0.3 ppmv \\
\textbf{Ozone (O$_3$)} & 48 & 0-0.1 ppmv \\
%\bottomrule
\hline
\end{tabular}
\caption{\emph{Principaux composants de l'atmosphère. Les gaz à effet de serre sont indiqués en gras.}}
\label{tab:compos}
\end{table}
