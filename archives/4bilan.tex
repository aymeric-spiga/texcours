\chapter{Bilan radiatif et effet de serre}

\dictum[Frederik van Eeden, 1887]{Le soleil accepte bien de passer par de petites fenêtres.}

\bk Les chapitres précédents n'ont été qu'un prélude pour bien comprendre les phénomènes qui déterminent la température à la surface de la Terre et dans son atmosphère. On s'attache dans le présent chapitre à effectuer des bilans d'énergie pour le système Terre et son atmosphère, définissant ainsi son \voc{bilan radiatif}.

\mk \section{Equilibre radiatif simple}

	\sk \subsection{Flux reçu et flux émis}

		\sk
Nous pouvons exprimer le rayonnement reçu du Soleil par la Terre par une densité de flux énergétique moyenne~$F\e{reçu}$ en W~m$^{-2}$ ou un flux énergétique~$\Phi\e{reçu}$ (en W)
\[ 
F\e{reçu} = (1-A\e{b}) \, \mathcal{F}\e{s}' 
\qquad \qquad
\Phi\e{reçu} = \pi \, R^2 \, (1-A\e{b}) \, \mathcal{F}\e{s}
\] 
La partie du rayonnement reçue du soleil qui est réfléchie vers l'espace est prise en compte via l'albédo planétaire noté~$A\e{b}$. On rappelle par ailleurs que~$\mathcal{F}\e{s}' = \mathcal{F}\e{s} / 4$ où $\mathcal{F}\e{s}$ est la constante solaire.


\sk
Par ailleurs, le système Terre émet également du rayonnement principalement dans les longueurs d'onde infrarouge [figure \ref{fig:eqrad2}]. 
Cette quantité de rayonnement émise au sommet de l'atmosphère radiative est notée $OLR$ pour \emph{Outgoing Longwave Radiation} en anglais.
A l'équilibre, la planète Terre doit émettre vers l'espace autant d'énergie qu'elle en reçoit du Soleil, donc
on obtient la relation générale appelée \emph{TOA} pour \emph{Top-Of-Atmosphere} en anglais, correspondant
au bilan radiatif au sommet de l'atmosphère
\[ \boxed{\TOA} \] 
La principale difficulté qui sous-tend les divers modèles pouvant être proposés réside dans l'expression du terme~$OLR$.




	\sk \subsection{Equilibre et température équivalente}

		\sk
A l'équilibre, la planète Terre doit émettre vers l'espace autant d'énergie qu'elle en reçoit du Soleil. Ceci peut s'exprimer par unité de surface
\[ \boxed{ F\e{reçu} = F\e{émis} } \]
ou, pour un résultat similaire, en considérant l'intégralité de la surface planétaire
\[ \Phi\e{reçu} = \Phi\e{émis} \]
ce qui permet de déterminer la température équivalente en fonction des paramètres planétaires
\[ \boxed{
T\e{eq} = \bigg[ \frac{\mathcal{F}\e{s}'\,(1-A\e{b})}{\sigma} \bigg]^{\frac{1}{4}}
} \]


		Le calcul présenté ici porte le nom d'\voc{équilibre radiatif simple}. On y néglige les effets de l'atmosphère (sauf l'albédo) puisqu'on suppose que le rayonnement atteint la surface, ou est rayonné vers l'espace, sans être absorbé par l'atmosphère. La température équivalente est ainsi la température qu'aurait la Terre si l'on négligeait tout autre influence atmosphérique que la réflexion du rayonnement solaire incident. Les valeurs de $T\e{eq}$ pour quelques planètes telluriques sont données dans la table \ref{tab:planets}. On note que la température équivalente de Vénus est plus faible que celle de la Terre, bien qu'elle soit plus proche du Soleil, à cause de son fort albédo~; la formule indique bien que, plus le pouvoir réfléchissant d'une planète est grand, plus la température de sa surface est froide. Par ailleurs, comme indiqué par les calculs du tableau~\ref{tab:planets}, on remarque que la température équivalente, si elle peut renseigner sur le bilan énergétique simple de la planète, ne représente pas correctement la valeur de la température de surface. Par exemple, la température équivalente pour la Terre est~$T\e{eq} = 255 K = -18^{\circ}$C, bien trop faible par rapport à la température de surface effectivement mesurée. Il faut donc avoir recours à un modèle plus élaboré.

\begin{table}\label{tab:planets} \begin{center} \begin{tabular}{lccccc} &{\bf Mercure} &{\bf V\'enus}&{\bf Terre}&{\bf Mars} &{\bf Titan} \\ \hline $d\e{soleil}$ (UA) & 0.39 & 0.72 & 1 & 1.5 & 9.5 \\ $\mathcal{F}\e{s}\,$(W~m$^{-2}$) & $8994$ & $2614$ & $1367$ & $589$ & $15$ \\ $A\e{b}$ & $0.06$ & $0.75$ & $0.31$ & $0.25$ & $0.2$ \\ \textcolor{blue}{$T\e{surface}$ (K)} & \textcolor{blue}{$100/700$~K} & \textcolor{blue}{$730$} & \textcolor{blue}{$288$} & \textcolor{blue}{$220$} & \textcolor{blue}{$95$} \\ \hline $T\e{eq}$~(K) & $439$ & $232$ & $254$ & $210$ & $86$\\ \end{tabular} \caption{\emph{Comparaison des facteurs influençant la température équivalente du corps noir pour différentes planètes du système solaire.}} \end{center} \end{table}
%    Mercure & 0.39 & 8994 & 0.06 & 439 \\
%    Vénus & 0.72 & 2639 & 0.78 & 225 \\
%    Terre & 1 & 1368 & 0.30 & 255 \\
%    Mars & 1.52 & 592 & 0.17 & 216 \\


\mk \section{Modèles \ofg{à couches}} % dits aux puissances échangées

	\sk \subsection{Modèle à une couche et effet de serre}

		\sk
L'équilibre radiatif simple présenté à la section précédente souffre d'un problème majeur~: il suppose que l'atmosphère n'interagit pas avec les rayonnements incidents et émis, ce qui n'est pas le cas en réalité. On rappelle notamment avec la figure~\ref{fig:atmspectrum} deux points importants qui vont nous permettre de raffiner les calculs.
\begin{finger}
\item Au vu des températures typiques du Soleil et de la Terre, le rayonnement d'origine solaire est principalement émis dans les longueurs d'onde visible, alors que le rayonnement d'origine terrestre est principalement émis dans les longueurs d'onde infrarouge. Les fonctions de Planck normalisées montrées dans la figure~\ref{fig:atmspectrum} indiquent que les deux domaines d'émission ne se recoupent quasiment pas. On peut donc séparer les calculs selon le domaine visible (également appelé ondes courtes) pour tout ce qui concerne le rayonnement reçu du Soleil et le domaine infrarouge (également appelé ondes longues) pour tout ce qui concerne le rayonnement émis par la surface et l'atmosphère de la Terre. La figure~\ref{fig:modzero} reprend ainsi le calcul de l'équilibre radiatif simple en étant plus fidèle à cette distinction entre visible et infrarouge~; en l'absence d'atmosphère, la température de surface à l'équilibre~$T\e{s}$ est égale à~$T\e{eq}$.
%\figside{0.6}{0.3}{\figfrancis/WH_atmspectrum}{entre le sommet de l'atmosphère et 11~km.}{fig:atmspectrum}
\item L'équilibre radiatif simple néglige les propriétés d'absorption de l'atmosphère de la Terre. La figure~\ref{fig:atmspectrum} montre que cette approximation est relativement juste pour les longueurs d'onde visible, où l'atmosphère est assez transparente, mais très inexacte pour les longueurs d'onde infrarouge. On a vu dans les chapitres qui précèdent que, contrairement à ce qui prévaut dans les longueurs d'onde visible, l'atmosphère est très opaque, c'est-à-dire très absorbante, dans l'infrarouge à cause principalement des gaz à effet de serre (et des nuages). Comme décrit au chapitre précédent, et sur la figure~\ref{fig:atmspectrum}, les principaux gaz à effet de serre sont, par ordre d'importance dans le bilan radiatif de la Terre, H$_2$O, CO$_2$, CH$_4$, N$_2$O, O$_3$, auxquels il convient d'ajouter les gaz à effet de serre industriels, tels les halocarbures, notamment les chloro-fluoro carbures\footnote{Qui jouent par ailleurs un rôle dans la destruction de l'ozone stratosphérique}. On note au passage que certains gaz à effet de serre comme CO$_2$ et CH$_4$ sont à la fois controlés par des processus naturels et industriels. Le rayonnement émis par la surface terrestre, principalement dans l'infrarouge, est donc absorbé par ces espèces et réémis à la fois vers l'espace et vers la surface. Ainsi, contrairement à ce qui est supposé dans le cas de l'équilibre radiatif simple, une partie du rayonnement émis par la surface n'est pas évacuée vers l'espace et contribue à augmenter la température de la surface terrestre. Ainsi la température de surface à l'équilibre~$T\e{s}$ n'est pas égale à la température équivalente~$T\e{eq}$. La figure~\ref{fig:modun} résume cette situation qui permet d'obtenir par le calcul, présenté ci-dessous, une valeur pour~$T\e{s}$ plus proche de la température effectivement mesurée à la surface de la Terre. On parle de \voc{modèle à une couche}.
\end{finger}

\figside[page=1]{0.6}{0.25}{./figures.pdf}{Modèle à zéro couche~: schéma des flux nets échangés dans le visible et dans l'infrarouge pour une planète sans atmosphère (ou plus précisément dans laquelle l'atmosphère n'est active radiativement ni dans l'infrarouge ni dans le visible) de température de surface~$T\e{s}$. Il s'agit simplement d'une présentation alternative de l'équilibre radiatif simple décrit en figure~\ref{fig:eqrad2}, qui s'avère plus pratique pour prendre en compte la présence d'une atmosphère et effectuer des calculs plus proches de la réalité. Ce schéma est cependant plus précis que la figure~\ref{fig:eqrad2} dans la mesure où il précise dans quel domaine de longueur d'onde se font les échanges.}{fig:modzero}
%\figun{0.6}{0.2}{\figfrancis/GH_1lay_noatm}{Schéma des flux échangés dans le visible (jaune) et l'infrarouge (rouge) pour une planète sans atmosphère de température de surface $T_s$.}{fig:GH1laynoatm}

		\sk
Quelle température de surface est prédite par le modèle à une couche décrit par la figure~\ref{fig:modun} ? On considère toujours une planète d'albédo planétaire $A\e{b}$ recevant l'éclairement moyen $\mathcal{F}\e{s}'$ du Soleil. Ce bilan correspond à la partie visible de la figure~\ref{fig:modun}. L'atmosphère est considérée comme transparente dans ce domaine de longueur d'onde. Dans la partie infrarouge, au contraire on ne néglige plus l'absorption, par les gaz à effet de serre présents dans l'atmosphère, du rayonnement infrarouge émis par la surface de la planète à la température~$T\e{s}$~: on représente ainsi l'atmosphère par une couche isotherme de température~$T\e{a}$, parfaitement absorbante dans l'infrarouge. Le rayonnement infrarouge émis par la surface est complètement absorbé dans l'atmosphère, qui émet à son tour~$\sigma {T\e{a}}^4$ à la fois vers l'espace et vers la surface comme indiqué dans le domaine infrarouge de la figure~\ref{fig:modun}. Une partie du rayonnement infrarouge émis par la Terre n'est donc pas évacuée vers l'espace et reste \ofg{piégée} dans le système atmosphère~+~surface, contribuant ainsi à élever la température de la surface~$T\e{s}$.

\figside{0.6}{0.2}{decouverte/cours_meteo/une_couche.png}{Modèle à une couche~: schéma des flux échangés dans le visible et dans l'infrarouge pour une planète dont l'atmosphère de température~$T\e{a}$ est opaque dans l'infrarouge.}{fig:modun}

\sk
Il s'agit ensuite d'effectuer le bilan des flux reçus et cédés en chacune des interfaces en rassemblant les termes des deux domaines visible et infrarouge.
%%\footnote{Les modèles du type de celui présenté ici sont parfois également appelés modèles aux puissances échangées}
\begin{finger}
\item pour l'atmosphère
\[ \underbrace{\sigma \, {T\e{s}}^4}_{\text{bilan des flux reçus}} = \underbrace{\sigma \, {T\e{a}}^4 + \sigma \, {T\e{a}}^4}_{\text{bilan des flux cédés}} \] 
On note que le rayonnement visible reçu du Soleil n'intervient pas dans le bilan pour l'atmosphère, ce qui est normal puisque l'absorption est négligée. Ainsi, comme indiqué sur le schéma~\ref{fig:modun}, l'atmosphère reçoit un rayonnement~$\mathcal{F}\e{s}'$ dont la partie~$\mathcal{F}\e{s}'\,(1-A\e{b})$ qui n'est pas réfléchie/diffusée est entièrement transmise à la surface. Tout se passe comme si l'atmosphère recevait~$\mathcal{F}\e{s}'$ et cédait~$\mathcal{F}\e{s}'\,(1-A\e{b})$ à la surface et~$\mathcal{F}\e{s}'\,A\e{b}$ à l'espace~; son bilan d'énergie dans le visible est donc nul puisque tous ces termes se compensent.
\item pour la surface
\[ \underbrace{\mathcal{F}\e{s}'\,(1-A\e{b}) + \sigma \, {T\e{a}}^4}_{\text{bilan des flux reçus}} = \underbrace{\sigma \, {T\e{s}}^4}_{\text{bilan des flux cédés}} \]
\end{finger}
On dispose alors de deux équations qui permettent de déterminer les deux inconnues~$T\e{a}$ et~$T\e{s}$. Ainsi la température à la surface de la planète dans le modèle à une couche est 
\[ \boxed{ T\e{s} = \bigg[ \frac{ 2 \, \mathcal{F}\e{s}'\,(1-A\e{b}) }{ \sigma } \bigg]^{\frac{1}{4}} = \sqrt[4]{2} \, T\e{eq} } \]

\sk
Le calcul numérique donne une température de~$303$~K (environ~$30^{\circ}$C) pour la Terre, une valeur à la fois bien supérieure à~$T\e{eq}$, qui vaut~$255$~K, et plus proche de la température effectivement constatée à la surface, quoiqu'un peu surévaluée. Ainsi les gaz à effet de serre présents dans l'atmosphère contribuent à réchauffer significativement la surface d'une planète. Le modèle à une couche est le modèle le plus simple de l'effet de serre qui permet d'en rendre compte qualitativement et, dans une certaine mesure, quantitativement.

		\sk
La température de l'atmosphère dans le modèle à une couche est, d'après les équations qui précèdent,
\[ \boxed{ T\e{a} = \bigg[ \frac{ \mathcal{F}\e{s}'\,(1-A\e{b}) }{ \sigma } \bigg]^{\frac{1}{4}} = T\e{eq} } \]
Un résultat équivalent peut être obtenu en faisant un bilan des flux reçus/cédés pour l'interface \ofg{espace}
\[ \underbrace{\mathcal{F}\e{s}'\,A\e{b} + \sigma \, {T\e{a}}^4}_{\text{bilan des flux reçus}} = \underbrace{\mathcal{F}\e{s}'}_{\text{bilan des flux cédés}} \] 
Deux aspects de ce modèle simple de l'effet de serre sont importants:
\begin{enumerate}
\item Que l'on considère un équilibre radiatif simple [figures~\ref{fig:eqrad2}], ou un modèle à une couche [figures~\ref{fig:modun}], la température à laquelle est émise le rayonnement infrarouge sortant vers l'espace doit être (en moyenne) égale à $T\e{eq}$. Sans atmosphère, cette température est celle de la surface, avec une atmosphère opaque dans l'infrarouge, il s'agit de celle de l'atmosphère.
\item Il n'y a un effet de serre que si la température d'émission vers l'espace est inférieure à la température de la surface. On peut l'imaginer dans le cas où l'atmosphère est également opaque dans les longueurs d'onde visible~: la surface échange alors uniquement du rayonnement avec l'atmosphère, et est à la même température à l'équilibre~: $T\e{s}=T\e{a}=T\e{eq}$ [voir section suivante].
\end{enumerate}
Pour obtenir des températures atmosphériques plus en accord avec les variations verticales observées (qui servent à définir les différentes couches atmosphériques comme abordé au chapitre d'introduction), on peut adopter un modèle à~$2$, $3$, \ldots couches\footnote{Ce point est abordé en travaux dirigés.}.


	\sk \subsection{Modèle gris}

		\sk Le modèle à une couche peut être généralisé quelque peu en considérant deux raffinements.
\begin{finger}
\item L'atmosphère est absorbante dans le visible avec un coefficient d'absorption~$\alpha$. Plus~$\alpha$ est grand, plus le rayonnement incident dans les longueurs d'onde visible reçu par la surface terrestre est atténué. Ce phénomène porte le nom d'\voc{effet parasol} (ou parfois \ofg{anti-effet de serre}). En réalité cet effet est très modéré sur Terre. Certes, l'ozone stratosphérique absorbe complètement le rayonnement ultraviolet, mais ceci représente une contribution faible du flux total. Les aérosols, tels que les poussières désertiques ou les particules d'origine volcanique, absorbent dans le visible et peuvent contribuer lors d'événements particuliers, telles les éruptions volcaniques ou les tempêtes de poussière, à augmenter~$\alpha$.
\item L'atmosphère n'est pas tout à fait un corps noir~: son émissivité dans l'infrarouge est~$\epsilon$ comprise entre~$0$ et~$1$. D'après la seconde loi de Kirchhoff, ceci indique que la couche atmosphérique n'est pas parfaitement absorbante dans l'infrarouge~: elle absorbe une partie~$\epsilon \, F$ du flux incident~$F$ et en transmet une partie~$(1-\epsilon) \, F$. Reste qu'en pratique, comme mentionné dans la partie précédente, l'atmosphère se comporte comme un corps presque noir dans l'infrarouge et l'émissivité~$\epsilon$ est relativement proche de~$1$. 
\end{finger}
Un tel modèle porte le nom de \voc{modèle gris à une couche}. Comme dans la figure~\ref{fig:modun}, la méthode pour calculer les températures consiste à reporter tous les flux échangés entre chacune des couches, comme indiqué dans la figure~\ref{fig:modgris}, puis de faire le bilan des énergies reçues et cédées aux interfaces.
\[ \begin{aligned} & & \boxed{\text{bilan des flux reçus}} & = \boxed{\text{bilan des flux cédés}} \\ 
& \text{[espace]} & \mathcal{F}\e{s}'\,A\e{b} + (1-\epsilon) \, \sigma \, {T\e{s}}^4 + \epsilon \, \sigma \, {T\e{a}}^4 & = \mathcal{F}\e{s}' \\
& \text{[atmos.]} & \mathcal{F}\e{s}' + \sigma \, {T\e{s}}^4 & = \mathcal{F}\e{s}'\,A\e{b} + \mathcal{F}\e{s}'\,(1-A\e{b}-\alpha) + (1-\epsilon) \, \sigma \, {T\e{s}}^4 + \epsilon \, \sigma \, {T\e{a}}^4 + \epsilon \, \sigma \, {T\e{a}}^4 \\ 
& \text{[surface]} & \mathcal{F}\e{s}'\,(1-A\e{b}-\alpha) + \epsilon \, \sigma \, {T\e{a}}^4 & = \sigma \, {T\e{s}}^4 \\ \end{aligned} \]
%& \text{[surface]} & \mathcal{F}\e{s}'\,(1-A\e{b})\,(1-\alpha) + \epsilon \, \sigma \, {T\e{a}}^4 & = \sigma \, {T\e{s}}^4 \\ \end{aligned} \]

\sk
On obtient alors l'expression de la température de la surface de la planète dans le cadre du modèle gris en combinant les équations de bilan de deux des trois interfaces (par exemple espace et surface)
\[ \boxed{T\e{s} = \sqrt[4]{\frac{1 - \frac{\alpha^{\prime}}{2}}{1 - \frac{\epsilon}{2}}} \, T\e{eq}} \qquad  \text{avec} \quad \alpha^{\prime} = \frac{\alpha}{1-A\e{b}} \] 
\begin{citemize}
\item Dans le cas où l'atmosphère est un corps noir dans l'infrarouge~($\epsilon=1$) et qu'elle est transparente dans le visible~($\alpha=0$), on se retrouve dans la situation du modèle à une couche avec~$T\e{s} = \sqrt[4]{2} \, T\e{eq}$.
\item Dans le cas où l'atmosphère est opaque dans le visible~($\alpha^{\prime}=1$) et dans l'infrarouge~($\epsilon=1$), la surface échange alors uniquement du rayonnement avec l'atmosphère et~$T\e{s}=T\e{a}=T\e{eq}$.
\item Si~$\epsilon$ augmente, l'effet de serre augmente, donc la température de la surface de la planète augmente.
\item Si~$\alpha$ augmente, l'effet parasol augmente, donc la température de la surface de la planète diminue. C'est l'un des effets observés dans les mois qui suivent une éruption volcanique majeure.
\end{citemize}
%De façon plus générale, on a vu que le rayonnement sortant provenait majoritairement de la région de l'atmosphère autour d'une épaisseur optique de 1 à partir du sommet. Cette région dépend de la longueur d'onde: proche de la surface dans la fenêtre transparente, dans la haute troposphère dans les bandes d'absorption du CO$_2$, autour de 2~km dans celles de la vapeur d'eau. Comme la température décroit à partir de la surface, le rayonnement sortant est donc émis à des températures inférieures à $T_s$, et on peut écrire qu'il vaut \[IR_{sommet}=\sigma T_s^4 (1-\epsilon)=\sigma T_{eq}^4\] Où $\epsilon>0$ représente l'effet de serre. La valeur de $\epsilon$ augmente quand la température d'émission vers l'espace diminue par rapport à celle de surface, typiquement parce que l'altitude d'émission augmente.

\figside{0.7}{0.17}{decouverte/cours_meteo/une_couche_gris.png}{Modèle gris à une couche~: schéma des flux échangés dans le visible et dans l'infrarouge pour une planète comme la figure~\ref{fig:modun} sauf que l'atmosphère de température~$T\e{a}$ est opaque dans l'infrarouge, mais sans être un corps totalement noir, et est absorbante dans le visible avec un coefficient d'absorption~$\alpha$.}{fig:modgris}
%\figun{0.6}{0.2}{\figfrancis/GH_1lay_atm}{Comme la figure \ref{fig:GH1laynoatm} mais avec une atmosphère opaque dans l'infrarouge et de coefficient d'absorption $a$ dans le visible, de température $T_a$.}{fig:GH1layatm}


\mk \section{Description complète du bilan radiatif du système Terre}

	\sk \subsection{Mesures en moyenne dans le temps et dans l'espace}

		\sk
Une représentation détaillée des différents flux échangés en moyenne temporelle et spatiale sur la Terre est présentée sur la figure~\ref{fig:bilflux}, qui est dérivée d'observations satellite les plus récentes. La figure est construite conformément à la séparation visible / infrarouge dictée par les résultats de la figure~\ref{fig:atmspectrum}. 

\sk
\subsubsection{Domaine visible}

\sk
Seulement~$50\%$ du rayonnement solaire incident dans les longueurs d'onde visible parviennent à la surface à cause, d'une part, de la réflexion/diffusion sur les molécules de l'air (diffusion Rayleigh dans toutes les directions, responsable de la couleur bleue du ciel), sur les gouttelettes nuageuses (diffusion de Mie) et sur la surface, et d'autre part, de l'absorption du rayonnement solaire incident par les molécules\footnote{Dans la mésosphère, c'est l'oxygène qui absorbe les radiations les plus énergétiques~; dans la stratosphère, l'absorption des radiations dans l’ultraviolet est assurée par différentes bandes d'absorption de l'ozone~; cette absorption peut avoir lieu dans certaines bandes jusque dans la troposphère.} et les aérosols composant l'atmosphère [relativement modérée dans les longueurs d'onde visible]. On note que la partie du rayonnement visible diffusée vers l'espace par les molécules de l'air, les nuages et la surface définit l'albédo planétaire mentionné précédemment~: un albédo élevé contribue à refroidir la surface et l'atmosphère. L'absorption de la lumière ultraviolet/visible, quant à elle, réchauffe directement l'atmosphère (notamment dans la stratosphère, car la troposphère n'est que très faiblement chauffée par les radiations solaires) et contribue à refroidir la surface par extinction du flux solaire incident. Dans le domaine visible, l'extinction est causée principalement par la diffusion et moins par l'absorption. La partie du rayonnement qui parvient à la surface est absorbée par la surface et convertie en énergie interne, c'est-à-dire contribue à élever sa température. On remarque que la surface terrestre ne peut être considérée tout à fait comme un corps noir puisqu'elle n'absorbe pas toute l'énergie incidente~: une petite partie du rayonnement incident est réfléchie par la surface. Cette composante réfléchie par la surface dépend fortement de la nature des sols (océans, forêts, déserts, glace, \ldots) et de leur répartition géographique.

\sk
\subsubsection{Domaine infrarouge}

\sk
Chauffée par l'absorption du rayonnement solaire incident, la surface terrestre se refroidit en émettant du rayonnement surtout dans l'infrarouge d'après la loi de Wien. La troposphère est ainsi principalement chauffée par l'absorption, par les gaz à effet de serre et les nuages, du rayonnement infrarouge émis par la surface. Dans l'infrarouge, à part quelques fenêtres atmosphériques à des longueurs d'onde bien précises, seule une petite partie du flux total émis par la surface s'échappe directement vers l'espace. A leur tour, les gaz à effet de serre émettent du rayonnement dans l'infrarouge, à la fois vers l'espace et vers la surface, ce qui refroidit localement l'atmosphère mais réchauffe la surface par effet de serre comme décrit précédemment avec le modèle à une couche [figure~\ref{fig:modun}]. L'atmosphère piège ainsi~$150$~W~m$^{-2}$ par effet de serre, puisque le rayonnement infrarouge sortant est~$240$~W~m$^{-2}$. On ajoute que la stratosphère est également refroidie par émission infrarouge du gaz carbonique, principalement dans la bande d'absorption à~$15$~$\mu$m. Du point de vue de l'atmosphère, émission infrarouge et refroidissement sont donc intimement liés.
%Émission nette par la vapeur d'eau, l'ozone, le CO2 et les autres gaz à effet de serre : Il s'agit du flux énergétique net émis sous forme de rayonnement énergétique (infrarouge) par l'ensemble des molécules de l'atmosphère. L'émission infrarouge est associée à un refroidissement local. Comme le Corps Noir, les molécules émettent un rayonnement pour se refroidir et équilibrer l'énergie absorbée. L'émission n'a lieu que dans les bandes d'absorption (ou d'émission). Il faut donc que la température locale soit celle du Corps Noir émettant à la longueur d'onde de la bande d'émission. Ainsi, plus on descend dans l'atmosphère plus l'émission se fera par les bandes centrées sur de faibles longueurs d'onde. Émission IR et refroidissement atmosphériques sont doncintimement liés. La stratosphère est principalement refroidie par l'émission IR du gaz carbonique. Ce refroidissement est associé à l'émission par la bande située à 15 μm. Dans la haute stratosphère, la bande d'émission de l'ozone à 9.6 μm permet l’émission IR et le refroidissement atmosphérique. Cependant l'ozone absorbe principalement les radiations solaires et ne peut être considérée comme un gaz à effet de serre (dans la stratosphère). La vapeur d'eau émet également dans la stratosphère dans la bande à 8 μm. La troposphère est principalement refroidie par l'émission de la vapeur d'eau dans la bande située à 6.3 micromètres.

\sk
\subsubsection{Autres échanges d'énergie}

\sk
Le bilan net en surface dans l'infrarouge de $65$~W~m$^{-2}$ est une petite différence entre le flux émis par la surface $\sigma \, {T\e{s}}^4$ et celui reçu depuis l'atmosphère. Si le bilan radiatif est bien équilibré au sommet de l'atmosphère, la surface gagne en moyenne de l'énergie et l'atmosphère en perd. En l'absence d'autres mécanismes de transfert d'énergie, cela conduirait à un refroidissement de l'atmosphère, et à une discontinuité de température à la surface entre le sol et l'air. En pratique, ce déséquilibre radiatif est compensé par des flux qui dépendent des mouvements et des changements de phase dans l'atmosphère
\begin{citemize}
\item de chaleur sensible (transport vertical de chaleur par la conduction et les mouvements de convection) 
\item de chaleur latente (évaporation depuis la surface et condensation dans l'atmosphère) 
\end{citemize}
depuis la surface vers l'atmosphère. Du fait que le transfert d'énergie du sol vers l'atmosphère se fait également sous forme d'un flux de chaleur sensible et latente, le sol n'émet donc que~$396$~W~m$^{-2}$ (au lieu de~$495$~W~m$^{-2}$) ce qui équivaut à une température de~$15^{\circ}$C, soit la température moyenne de la surface terrestre effectivement constatée. En l'absence de convection et de changements d'état dans l'atmosphère, la température de la surface et des basses couches atmosphériques serait beaucoup plus élevée. 
%%% les 240 W/m2 qui sortent sont les mêmes que dans la version sans atmosphère.
%%% noter la fenêtre atmosphérique dans l'infrarouge

%\figun{1.0}{0.4}{\figfrancis/bilan_rad_glob}{Schéma des flux moyens échangés entre la surface de la Terre, l'atmosphère, et l'espace: flux radiatifs ondes courtes (jaune) et infrarouge (rouge), et flux sensibles et latents (violet).}{fig:bilanrad}
\figun{1.0}{0.45}{decouverte/meteo_terre/bilanflux00004.png}{Bilan énergétique moyen de la Terre (en W~m$^{-2}$)~: flux échangés entre la surface de la Terre, l'atmosphère et l'espace. On distingue les flux radiatifs ondes courtes (rayonnement visible, en jaune) des flux radiatifs ondes longues (rayonnement infrarouge, en rouge). Noter les flux sensibles et latents qui ne sont pas relatifs au transfert radiatif. Source~: Planton CNRS editions 2011 ; adapté de Trenberth et al. BAMS 2009}{fig:bilflux}

%%% MANQUE UN TOPO SUR LE FORçAGE RADIATIF ????
%%% POUR REBRANCHER SUR LE CHANGEMENT CLIMATIQUE. VOIR PAYAN 10-12.



	\sk \subsection{Variations géographiques}

		\sk
\subsubsection{Influence de la latitude}

\sk
Localement, l'éclairement varie suivant la latitude et la saison, en plus de l'alternance jour/nuit: il est proportionnel à $\cos\theta$ où $\theta$ est l'angle d'incidence avec la surface.
%[figure~\ref{fig:senslat1}]. 
En moyenne annuelle, le maximum d'ensoleillement est donc aux latitudes tropicales, mais il varie au cours de l'année et est même maximal aux pôles pendant l'été local [figure~\ref{fig:senslat2}]~: la durée du jour de 24h fait plus que compenser l'angle d'incidence réduit dû à la latitude élevée (ce qui peut paraître de prime abord contre-intuitif).
%\figside{0.3}{0.1}{\figfrancis/swcoslat.jpg}{Schéma de la relation entre densité de flux du rayonnement incident parallèle et éclairement de la surface suivant l'angle d'incidence.}{fig:senslat1}
\figside{0.65}{0.25}{\figfrancis/swtoaseas}{Cycle saisonnier de l'éclairement dû au rayonnement solaire incident au sommet de l'atmosphère.}{fig:senslat2}
%%%% http://www.energieplus-lesite.be/energieplus/page_16761.htm

\sk
\subsubsection{Rôle des nuages}

\sk
La présence de différents types de nuages est très variable, à la fois géographiquement et dans le temps. Ils ont pourtant une influence très grande sur le bilan radiatif, par deux mécanismes distincts [figure \ref{fig:schemacrf}].
\begin{finger}
\item Effet d'albédo~: les nuages réfléchissent une partie importante du rayonnement solaire incident (par rétro\-diffusion par les gouttes d'eau). Cet effet est d'autant plus fort que le nuage contient d'eau et que les gouttes sont fines. Un nuage très réfléchissant apparaitra sombre vu d'en dessous. Au total, les nuages sont responsables des 2/3 de l'albédo planétaire.
\item Effet de serre~: Les gouttes d'eau (ou la glace) des nuages sont d'excellents absorbants dans l'infrarouge. Un nuage même peu épais absorbe donc très rapidement tout le rayonnement infrarouge provenant des couches plus basses. Il émet lui même vers le haut du rayonnement suivant sa propre température: $\sigma T_N^4$ où $T_N$ est la température au sommet du nuage. Un nuage au sommet élevé (donc froid) aura donc un effet de serre très important.
\end{finger}
Au final, l'effet d'albédo l'emporte pour les nuages bas (type stratus), qui sont typiquement épais (albédo élevé) et dont le sommet est chaud. Au contraire, les fins nuages d'altitude (cirrus) ont un albédo faible mais un sommet très froid donc ont un effet net réchauffant. Pour les nuages de type orageux, qui sont épais avec un sommet froid, les deux effets tendent à se compenser.
\figside{0.6}{0.2}{\figfrancis/schema_crf}{Schema de l'influence des nuages sur le bilan radiatif: effet d'albédo dans le visible (jaune), et absorption et émission dans l'infrarouge (rouge). L'effet de serre vient du rayonnement émis vers l'espace plus faible que celui venant de la surface, qui est absorbé.}{fig:schemacrf}
%% nuages comme les lunettes dans la caméra infrarouge. faire également référence à la vidéo tirée du satellite.



	\sk \subsection{Moyennes annuelles~: cartes}

		\sk
\subsubsection{Influence de la latitude}

\sk
Localement, l'éclairement varie suivant la latitude et la saison, en plus de l'alternance jour/nuit: il est proportionnel à $\cos\theta$ où $\theta$ est l'angle d'incidence avec la surface.
%[figure~\ref{fig:senslat1}]. 
En moyenne annuelle, le maximum d'ensoleillement est donc aux latitudes tropicales, mais il varie au cours de l'année et est même maximal aux pôles pendant l'été local [figure~\ref{fig:senslat2}]~: la durée du jour de 24h fait plus que compenser l'angle d'incidence réduit dû à la latitude élevée (ce qui peut paraître de prime abord contre-intuitif).
%\figside{0.3}{0.1}{\figfrancis/swcoslat.jpg}{Schéma de la relation entre densité de flux du rayonnement incident parallèle et éclairement de la surface suivant l'angle d'incidence.}{fig:senslat1}
\figside{0.65}{0.25}{\figfrancis/swtoaseas}{Cycle saisonnier de l'éclairement dû au rayonnement solaire incident au sommet de l'atmosphère.}{fig:senslat2}
%%%% http://www.energieplus-lesite.be/energieplus/page_16761.htm

\sk
\subsubsection{Rôle des nuages}

\sk
La présence de différents types de nuages est très variable, à la fois géographiquement et dans le temps. Ils ont pourtant une influence très grande sur le bilan radiatif, par deux mécanismes distincts [figure \ref{fig:schemacrf}].
\begin{finger}
\item Effet d'albédo~: les nuages réfléchissent une partie importante du rayonnement solaire incident (par rétro\-diffusion par les gouttes d'eau). Cet effet est d'autant plus fort que le nuage contient d'eau et que les gouttes sont fines. Un nuage très réfléchissant apparaitra sombre vu d'en dessous. Au total, les nuages sont responsables des 2/3 de l'albédo planétaire.
\item Effet de serre~: Les gouttes d'eau (ou la glace) des nuages sont d'excellents absorbants dans l'infrarouge. Un nuage même peu épais absorbe donc très rapidement tout le rayonnement infrarouge provenant des couches plus basses. Il émet lui même vers le haut du rayonnement suivant sa propre température: $\sigma T_N^4$ où $T_N$ est la température au sommet du nuage. Un nuage au sommet élevé (donc froid) aura donc un effet de serre très important.
\end{finger}
Au final, l'effet d'albédo l'emporte pour les nuages bas (type stratus), qui sont typiquement épais (albédo élevé) et dont le sommet est chaud. Au contraire, les fins nuages d'altitude (cirrus) ont un albédo faible mais un sommet très froid donc ont un effet net réchauffant. Pour les nuages de type orageux, qui sont épais avec un sommet froid, les deux effets tendent à se compenser.
\figside{0.6}{0.2}{\figfrancis/schema_crf}{Schema de l'influence des nuages sur le bilan radiatif: effet d'albédo dans le visible (jaune), et absorption et émission dans l'infrarouge (rouge). L'effet de serre vient du rayonnement émis vers l'espace plus faible que celui venant de la surface, qui est absorbé.}{fig:schemacrf}
%% nuages comme les lunettes dans la caméra infrarouge. faire également référence à la vidéo tirée du satellite.



