\chapter{Transfert radiatif}

\dictum[Alberto Giacometti, 1949]{Le ciel n’est bleu que par convention, mais rouge en réalité.}

\bk Lorsqu'il traverse l'atmosphère terrestre, la totalité du rayonnement thermique émis par le Soleil n'est pas transmise à la surface. On appelle \voc{transfert radiatif} les phénomènes qui régissent l'interaction entre rayonnement incident et matière. Dans ce chapitre, on aborde des éléments de éléments de transfert radiatif qui sont nécessaires pour effectuer le bilan énergétique de l'atmosphère et du système climatique.

\mk \section{Réflexion, absorption, transmission}

	\sk \subsection{Coefficients d'interaction et lois de Kirchhoff}

		\sk
Tout rayonnement se propageant dans un milieu matériel subit trois phénomènes~: réflexion, absorption, transmission. Autrement dit, tout corps cible irradié par une source voit le flux énergétique incident spectral~$\Phi_{\lambda}$ se répartir selon trois termes
\begin{citemize}
\item une partie~$\Phi_{\lambda}^r$ du flux incident est réfléchie ou diffusée;
\item une partie~$\Phi_{\lambda}^t$ du flux incident traverse le corps sans interactions;
\item une partie~$\Phi_{\lambda}^a$ du flux incident est absorbée, c'est-à-dire transformée en énergie interne.
\end{citemize}
Afin de définir les contributions respectives de ces trois phénomènes, on définit des coefficients spectraux de \voc{réflexion}~$\rho_{\lambda}$, de \voc{transmission}~$\tau_{\lambda}$, d'\voc{absorption}~$\alpha_{\lambda}$ compris entre~$0$ et~$1$
$$\Phi_{\lambda}^r = \rho_{\lambda} \, \Phi_{\lambda} \qquad\qquad \Phi_{\lambda}^t = \tau_{\lambda} \, \Phi_{\lambda} \qquad\qquad \Phi_{\lambda}^a = \alpha_{\lambda} \, \Phi_{\lambda}$$
Ces coefficients sont également appelés \voc{réflectivité}~$\rho_{\lambda}$, \voc{transmittivité}~$\tau_{\lambda}$, \voc{absorptivité}~$\alpha_{\lambda}$. Ils dépendent de la longueur d'onde~$\lambda$ du rayonnement incident, de l'angle d'incidence et des propriétés physiques et chimiques du corps récepteur (par exemple, température, composition). Lorsque le coefficient de réflexion~$\rho_{\lambda}$ ne dépend pas de l'angle d'incidence\footnote{L'énergie incidente à une surface pénètre dans celle-ci et est réfléchie aléatoirement à l'intérieur de l'objet par de microscopiques in-homogénéités du matériau. Au cours de ces multiples réflexions une partie de l'énergie incidente ressort de l'objet suivant une direction aléatoire. Bien souvent les réflexions multiples dans le matériau ne subissent aucune contrainte particulière, l'énergie est donc réfléchie de façon uniforme et isotrope par la surface. Le flux réfléchi est alors uniquement fonction de la quantité d'énergie incidente tombant sur la surface, qui s'exprime souvent simplement comme un cosinus de l'angle entre la normale à la surface et la direction de la source.}, on parle de l'objet cible comme d'un \voc{réflecteur lambertien}. 

%\sk
%\subsection{Rappel sur la distinction entre grandeurs intégrées et spectrales}
%\sk
%Dans le chapitre précédent, les grandeurs caractéristiques~$\Phi$,~$F$ ou~$L$, ainsi que les coefficients d'absorption~$\rho$, de transmission~$\tau$, de réflexion~$\rho$ (également appelé albédo~$A$), ont été décrits 
%\begin{citemize}
%\item soit d'une façon intégrée selon toutes les longueurs d'onde (par exemple l'émittance~$M$ dans la loi de Stefan-Boltzmann) ;
%\item soit en prenant en compte la dépendance spectrale, c'est-à-dire en considérant un petit intervalle~$\dd \lambda$ autour d'une longueur d'onde~$\lambda$ donnée (par exemple la luminance énergétique spectrale~$B_{\lambda}$ dans la loi de Planck). 
%\end{citemize}
%Dans le premier cas, on emploie simplement les symboles décrivant les grandeurs (exemple, luminance~$L$). Dans le second cas, on ajoute $\lambda$ en indice de ces symboles (exemple, luminance spectrale~$L_\lambda$). Ce qui est clair pour les variables l'est beaucoup moins dans le vocabulaire couramment utilisé, y compris dans certains ouvrages.


		\sk
La loi de Kirchhoff traduit la conservation du flux incident~$\Phi_{\lambda} = \Phi_{\lambda}^r + \Phi_{\lambda}^t + \Phi_{\lambda}^a$ par une relation entre les coefficients spectraux de réflexion, transmission et absorption $$ \boxed{ \rho_{\lambda} + \tau_{\lambda} + \alpha_{\lambda} = 1 } $$ On peut également considérer le flux énergétique incident~$\Phi$ intégré selon toutes les longueurs d'onde. Les coefficients de réflexion~$\rho$, de transmission~$\tau$, d'absorption~$\alpha$ peuvent alors être définis par $$\Phi^r = \rho \, \Phi \qquad\qquad \Phi^t = \tau \, \Phi \qquad\qquad \Phi^a = \alpha \, \Phi $$ et la loi de Kirchhoff s'écrit alors $$ \boxed{ \rho + \tau + \alpha = 1 } $$ On note par ailleurs que les définitions des coefficients impliquent que $\rho \ne \int_{\lambda} \rho_\lambda \, \dd\lambda$


	\sk \subsection{Vocabulaire et mise en garde}

		\sk
Un certain nombre de termes sont couramment utilisés pour désigner des situations particulières en terme d'absorptivité, de transmittivité, de réflectivité spectrale. Il est important de préciser à quelle longueur d'onde~$\lambda$ on se place lorsqu'on qualifie les propriétés d'un corps matériel (ce n'est pas toujours clair dans les ouvrages).
\begin{citemize}
\item Un corps \voc{transparent} à la longueur d'onde~$\lambda$ est tel que~: $\tau_\lambda = 1$, donc~$\rho_\lambda = \alpha_\lambda = 0$
\item Un corps \voc{opaque} à la longueur d'onde~$\lambda$ est tel que~: $\tau_\lambda = 0$, donc~$\rho_\lambda + \alpha_\lambda = 1$
\item Un corps \voc{brillant} à la longueur d'onde~$\lambda$ est tel que~: $\rho_\lambda = 1$, donc~$\tau_\lambda = \alpha_\lambda = 0$
\item Un corps \voc{sombre} à la longueur d'onde~$\lambda$ est tel que~: $\rho_\lambda = 0$, donc~$\tau_\lambda + \alpha_\lambda = 1$
\end{citemize}
Le corps noir étant un absorbant idéal, son coefficient spectral d'absorption~$\alpha_\lambda$ vaut 1 pour toutes les longueurs d'onde~$\lambda$. Il vérifie donc également~$\rho_\lambda = \tau_\lambda = 0$ pour toutes les longueurs d'onde. Par extension, un corps est qualifié de \voc{presque noir} à la longueur d'onde~$\lambda$ si~$\alpha_\lambda = 1$ à la longueur d'onde~$\lambda$. Un corps gris tel que défini au chapitre précédent est tel que son absorptivité~$\alpha_\lambda$ est la même selon toutes les longueurs d'onde.

\sk
On n'utilise le terme de~\ofg{complètement} noir, opaque, brillant, \ldots~que lorsque la propriété est vérifiée pour toutes les longueurs d'onde du spectre électromagnétique. Ce n'est en fait quasiment jamais le cas en pratique. La neige fraîche en est un excellent exemple~: elle apparaît brillante dans le visible car elle réfléchit le rayonnement incident ($\rho_{\lambda\e{VIS}}=1$). Il serait erroné de croire que c'est le cas dans toutes les longueurs d'ondes. La neige fraîche est presque noire dans l'infrarouge où elle est très absorbante ($\alpha_{\lambda\e{IR}}=1$) donc elle est sombre dans l'infrarouge ($\rho_{\lambda\e{IR}}=0$). Autre exemple, le verre est transparent pour les longueurs d'onde visible ($\tau_\lambda = 1$), mais relativement opaque pour le rayonnement infrarouge ($\tau_\lambda \rightarrow 1$).
%une surface \ofg{brillante} dans le visible ($A_{\lambda}$ proche de~$1$) peut s'avérer \ofg{sombre} dans l'infrarouge ($A_{\lambda} \ll 1$)


	\sk \subsection{Albédo}

		\sk
En sciences de l'atmosphère, les coefficients de réflexion~$\rho$ et~$\rho_{\lambda}$ sont souvent désignés sous le nom respectivement d'\voc{albédo} noté~$A$ et d'albédo spectral noté~$A_{\lambda}$. Plus la surface réfléchit une grande partie du rayonnement électromagnétique incident, plus l'albédo est proche de~$1$. L'albédo spectral~$A_{\lambda}$ peut varier significativement en fonction de la longueur d'onde : voir l'exemple de la neige fraîche donné ci-dessus. 

\sk
De par la diversité des surfaces terrestres, et de la variabilité de la couverture nuageuse, les valeurs de l'albédo~$A$ varient fortement d'un point à l'autre du globe terrestre~: il est élevé pour de la neige fraîche et faible pour de la végétation et des roches sombres [table~\ref{tab:albedo}]. L'albédo de l'océan est faible, particulièrement pour des angles d'incidence rasants -- il dépend ainsi beaucoup de la distribution des vagues. 

\begin{table}\label{tab:albedo}
\begin{center}
\begin{tabular}{|c|c|c|c|}
\hline
Type & albédo~$A$ & Type & albédo~$A$ \\
\hline
Surface de lac & 0.02 à 0.04 & Surface de la mer & 0.05 à 0.15 \\
Asphalte & 0.07 & Mer calme (soleil au zenith) & 0.10 \\
Forêt équatoriale & 0.10 & Roches sombres, humus & 0.10 à 0.15 \\
Ville & 0.10 à 0.30 & Forêt de conifères & 0.12 \\
Cultures & 0.15 à 0.25 & Végétation basse, verte & 0.17 \\
Béton & 0.20 & Sable mouillé & 0.25 \\
Végétation sèche & 0.25 & Sable léger et sec & 0.25 à 0.45 \\
Forêt avec neige au sol & 0.25 & Glace & 0.30 à 0.40 \\
Neige tassée & 0.40 à 0.70 & Sommet de certains nuages & 0.70 \\
Neige fraîche & 0.75 à 0.95 & & \\
\hline
\end{tabular}
\caption{\emph{Quelques valeurs usuelles d'albédo (rayonnement visible). D'après mesures missions NASA et ESA.}}
\end{center}
\end{table}

\sk
L'\voc{albédo planétaire} est noté~$A\e{b}$ et défini comme la fraction moyenne de l'éclairement~$E$ au sommet de l'atmosphère (noté également~$\mathcal{F}\e{s}'$) qui est réfléchie vers l'espace~: il comprend donc la contribution des surfaces continentales, de l'océan et de l'atmosphère. Il vaut~$0.31$ pour la planète Terre~: une partie significative du rayonnement reçu du Soleil par la Terre est réfléchie vers l'espace\footnote{L'albédo planétaire est par exemple encore plus élevé sur Vénus ($0.75$) à cause de la couverture nuageuse permanente et très réfléchissante de cette planète.}. Ainsi le système Terre reçoit une densité de flux énergétique moyenne~$F\e{reçu}$ en W~m$^{-2}$ telle que
\[ F\e{reçu} = (1-A\e{b}) \, \mathcal{F}\e{s}' \] 
donc un flux énergétique~$\Phi\e{reçu}$ (en W) qui s'exprime
\[ \Phi\e{reçu} = \pi \, R^2 \, (1-A\e{b}) \, \mathcal{F}\e{s} \]
%L'albédo de Bond~ désigne l'albédo intégré sur toutes les longueurs d'onde et tous les angles d'incidence.

\sk
La valeur de~$30\%$ de l'albédo planétaire sur Terre est en fait majoritairement dû à l'atmosphère~:  seuls 4\% de l'énergie solaire incidente sont réfléchis par la surface terrestre comme indiqué sur la figure~\ref{fig:diffsep}. L'énergie réfléchie par l'atmosphère vers l'espace, responsable de plus de~$85\%$ de l'albedo planétaire, est diffusée par les molécules ou par des particules en suspension, gouttelettes nuageuses, gouttes de pluie ou aérosols.

\figside{0.4}{0.15}{\figpayan/LP211_Chap2_Page_27_Image_0001.png}{L'énergie solaire incidente est réfléchie vers l'espace par la surface et l'atmosphère d'une planète. La figure montre les différentes contributions à l'albédo planétaire total.}{fig:diffsep}


	\sk \subsection{Remarque sur l'émissivité}

		\sk
Il existe une seconde loi de Kirchhoff, différente de celle précitée, qui stipule que l'émissivité spectrale doit être égale au coefficient d'absorption du corps $$ \epsilon_\lambda = \alpha_\lambda $$ pour des quantités intégrées selon toutes les directions de l'espace. Un corps ne peut émettre que les radiations qu'il est capable d'absorber. En d'autres termes, pour une température et une longueur d'onde donnée, un bon émetteur est souvent un bon absorbant (et vice versa). On retrouve par ce principe que le corps noir est le corps idéal qui rayonne un maximum d'énergie radiative à chaque température et pour chaque longueur d'onde. 
%Un corollaire est qu'un corps transparent ou réfléchissant à une certaine longueur d'onde émet peu de rayonnement thermique à cette même longueur d'onde. NON CAR IL SUFFIT QUE LA TEMPERATURE SOIT ELEVEE !



\mk \section{Bases pour aborder l'interaction entre le rayonnement électromagnétique et l'atmosphère}

	\sk \subsection{Généralités}

		\sk
Le rayonnement électromagnétique, lors de sa traversée de l'atmosphère terrestre, est perturbé par deux processus que l'on peut analyser séparément~: un processus d'\voc{absorption} par certains gaz atmosphériques (O$_2$, H$_2$O, O$_3$, etc.) et un processus de \voc{diffusion} par les molécules et les aérosols (poussières, cristaux de glace, gouttes nuageuses et gouttes de pluie). 
\begin{finger}
\item Dans le processus d'absorption, un certain nombre de photons disparaissent, une partie du rayonnement incident est convertie en énergie interne, et il y a une extinction du signal dans la direction de propagation. 
\item Au contraire, dans le processus de diffusion, les photons sont simplement redistribués dans toutes les directions avec une certaine probabilité définie par ce qu'on appelle la fonction de phase de diffusion~; on peut alors observer une extinction dans certaines directions et une augmentation dans d'autres. Lors de la diffusion, il n'y a pas de changement de longueur d'onde de l'onde incidente et de l'onde diffusée.
\end{finger}
%Ainsi la réflexion peut être \voc{diffuse} (dans toutes les directions), \voc{spéculaire} (dans la direction symétrique du rayonnement incident) ou quelconque. 
%Les deux effets peuvent être analysés séparément.
Autrement dit, on s'intéresse à l'\voc{extinction} progressive du rayonnement incident par absorption et diffusion. Deux phénomènes très importants sont mis de côté~: l'émission de rayonnement thermique, déjà abordée au chapitre précédent, et la diffusion multiple, que l'on néglige (rayonnement diffusé qui viendrait depuis d'autres directions).


	\sk \subsection{Section efficace}\label{sec:efficace}

		\sk
Que l'on s'intéresse à la diffusion ou l'absorption, l’interaction entre le rayonnement et la matière dépend de la rencontre entre les photons et les éléments (atomes, molécules, particules) du milieu considéré. La diffusion et l'absorption dépendent donc de la probabilité que le rayonnement, c'est-à-dire les photons qui le constituent, rencontre les éléments constitutifs de la matière. 

\sk
De façon évidente, cette probabilité est liée 
\begin{citemize}
\item au flux de photons~: la probabilité est plus élevée s'il y a un plus grand nombre de photons incidents, ou de façon équivalente une plus grande énergie radiative incidente ;
\item au nombre d'éléments (atomes, molécules, particules) dans le milieu matériel~: la probabilité augmente avec le nombre d'éléments, autrement dit dans un milieu plus dense, le rayonnement aura une plus forte probabilité de rencontrer des éléments matériels avec lesquels interagir. 
\end{citemize}
Cependant, même si ces deux quantités sont élevées, la probabilité peut rester faible, car elle dépend également d'un paramètre qui traduit l'efficacité de la rencontre entre un photon de longueur d'onde donnée~$\lambda$ et l'espèce absorbante. Cette grandeur est appelée \voc{section efficace} et est décrite ci-dessous.

\figun{0.5}{0.2}{\figfrancis/Beer_upside}{Variation du rayonnement incident avec un angle $\theta$ sur une couche d'épaisseur $dz$}{fig:beer}

\sk
On considère la situation décrite dans la figure~\ref{fig:beer}. Soit une tranche d'atmosphère horizontale\footnote{On dit qu'on fait l'approximation plan-parallèle car on néglige la courbure de la Terre ainsi que les variations horizontales des paramètres géophysiques (température et profils de gaz).} d'épaisseur élémentaire~$\dd z$ qui reçoit un rayonnement monochromatique de longueur d'onde~$\lambda$ caractérisé par sa luminance énergétique spectrale~$L_\lambda$. Le rayonnement incident traverse la tranche d'atmosphère en faisant un angle~$\theta$ par rapport à la verticale. La distance parcourue par le rayonnement à travers la fine couche d'épaisseur~$\dd z$ vaut\footnote{On parle d'abscisse curviligne pour qualifier~$s$.} 
\[ \dd s = \frac{1}{\cos\theta} \, \dd z \]
%%%% OK avec Beer_upside
autrement dit l'inclinaison du rayonnement impose un chemin optique plus grand. A la sortie de la tranche d'atmosphère, le rayonnement a subi une extinction à cause des phénomènes de diffusion et absorption dans la tranche d'atmosphère. On caractérise alors l'extinction causée par la diffusion et l'absorption par une quantité appelée section efficace~$\Sigma_\lambda$ qui a la dimension d'une surface, exprimée en m$^2$. A la sortie de la tranche d'atmosphère, la luminance spectrale est~$L_\lambda + \dd L_\lambda$ avec 
\[ \dd L_\lambda = - L_\lambda(s) \, \Sigma_\lambda(s) \, N \, \dd s \qquad \textrm{ou de manière équivalente} \qquad \boxed{ \dd L_\lambda = - L_\lambda(z) \, \Sigma_\lambda(z) \, N \, \frac{1}{\cos\theta} \, \dd z } \]
où $N$ est le nombre de particules par unité de volume. La section efficace~$\Sigma_{\lambda}$ prend en compte l'extinction causée par les phénomènes d'absorption et de diffusion~: plus la section efficace est grande, plus l'extinction du rayonnement incident est élevée. Il est possible de séparer les deux contributions en définissant une section efficace d'absorption~$\Sigma_\lambda^a$ et une section efficace de diffusion/réflexion~$\Sigma_\lambda^r$ telles que~$\Sigma_{\lambda} = \Sigma_\lambda^a + \Sigma_\lambda^r$. La section efficace dépend de la longueur d'onde et de la nature physico-chimique du milieu absorbant~: par exemple, la composition de l'air dans le cas de l'atmosphère, ou la température dans le cas de la majorité des matériaux. Ainsi, en toute généralité, elle n'a pas de raison particulière de rester constante pour les différentes parties du milieu matériel traversé.

%Ce n'est pas le but de ce cours que de proposer une vision complète du transfert radiatif, mais le lecteur intéressé peut noter que l'équation complète du transfert radiatif dans l'atmosphère prend la forme indiquée précédemment avec cependant l'ajout des deux termes \ofg{sources} précités \[ \dd L_\lambda = - L_\lambda(s) \, \Sigma_\lambda(s) \, N \, \dd s + \textrm{émission thermique} + \textrm{diffusion multiple} \]



\mk \section{Diffusion}

	\sk
La diffusion est un phénomène macroscopique résultant de la réflexion, de la réfraction et de la diffraction du rayonnement incident, qui se produisent au niveau microscopique en raison des inhomogénéités du milieu matériel traversé. Par abus de langage, auquel les présentes notes n'échappent pas, on identifie souvent réflexion et diffusion. 

\sk
On distingue différents mécanismes de diffusion selon la taille relative des cibles (molécules ou particules) par rapport à la longueur d'onde du rayonnement électromagnétique incident [figure \ref{fig:diffsize}]. Les connaître permet de comprendre certains phénomènes atmosphériques perçus au quotidien [figure~\ref{fig:ciel}]. Les radiations solaires situées dans l'ultraviolet sont absorbées dans la haute atmosphère (notamment par l'ozone dans la stratosphère) si bien que l'on considère principalement les radiations visibles.
\begin{finger}
\item \underline{Taille des cibles petite devant la longueur d'onde du rayonnement incident} La \voc{diffusion Rayleigh} est la diffusion par les molécules\footnote{ou par des particules significativement petites devant la longueur d'onde, mais ce cas de figure est relativement rare en pratique} qui constituent l'atmosphère. La diffusion Rayleigh dépend fortement de la longueur d’onde incidente. Lord Rayleigh à démontré en 1873 que cette dépendance s’exprimait selon l'inverse de la puissance quatrième de la longueur d'onde
\[ \boxed{ \Sigma_\lambda^r \propto \lambda^{-4} } \]
\begin{citemize}
\item Cette dépendance en longueur d'onde a un effet très notable sur le rayonnement thermique reçu du Soleil, dont nous avons vu au précédent chapitre qu'il est maximum dans les longueurs d'onde visibles. Les molécules d'air de l'atmosphère diffusent plus les photons de courte longueur d'onde à cause de la loi en puissance quatrième de~$\lambda$~: ainsi, le violet et le bleu sont~$16$ fois plus diffusés que le rouge par le mécanisme de Rayleigh. C'est pour cette raison que l'on voit le ciel bleu depuis la surface~: il s'agit de la couleur émanant du rayonnement solaire incident diffusé en majorité par le mécanisme de Rayleigh\footnote{Le fait qu'on ne voit pas le ciel violet est dû à une moindre sensibilité de l'oeil à ces longueurs d'onde, ainsi qu'un moindre flux incident que dans le bleu d'après le spectre solaire.}. La figure~\ref{fig:diffsep} nous indique que~$6\%$ du rayonnement incident sont ainsi diffusés, soit une contribution d'environ~$20\%$ à l'albédo planétaire. 
\item La diffusion Rayleigh ne montre pas de direction préférentielle significative, à part une tendance légèrement supérieure à la diffusion vers l'arrière (rétrodiffusion) et vers l'avant [figure~\ref{fig:diffdir}]. Ceci explique que le ciel apparaisse bleu qu'on le regarde depuis la surface ou depuis un avion. Ceci explique également que la diffusion Rayleigh fasse apparaître le Soleil de la couleur la moins diffusée, à savoir jaune à rouge suivant l'importance de la diffusion, alors qu'il apparaîtrait blanc sans diffusion.
\item Au lever et au coucher du Soleil, lorsque la lumière solaire traverse une couche importante d'atmosphère, la diffusion Rayleigh est plus grande qu'en journée lorsque le Soleil est proche du zénith. La raison est purement géométrique, comme l'on peut s'en convaincre d'après la section~\ref{sec:efficace}~: si l'angle d'incidence~$\theta$ est plus grand, le facteur~$1/\cos\theta$ est plus grand, donc, pour une même section efficace de diffusion~$\Sigma_\lambda^r$, la variation du flux incident~$dL_\lambda$ est plus grande. Ainsi, le soleil est vu rouge le soir car plus de rayonnement incident dans les longueurs d'onde bleues est diffusé qu'en journée.
\item Au contraire du rayonnement thermique solaire, dont le maximum d'émission est dans le domaine visible, l'effet de la diffusion Rayleigh sur le rayonnement thermique émis par la Terre est négligeable, car ce dernier est situé dans l'infrarouge à des longueurs d'onde~$\lambda$ plus élevées pour lesquelles~$\Sigma_\lambda^r \sim 0$. 
\end{citemize}

\figside{0.6}{0.3}{\figfrancis/WH_diff_size}{Type de mécanisme de diffusion dominant en fonction de la longueur d'onde (en abscisse) et de la taille des particules (en ordonnée, l'unité est en $\mu$m). Figure adaptée de Wallace and Hobbs, Atmospheric Science, 2006.}{fig:diffsize}

\figsup{0.48}{0.2}{decouverte/cours_meteo/nuage_ciel.jpg}{decouverte/cours_meteo/ciel_rouge.jpg}{Ciel bleu et nuage blanc. Coucher de soleil rouge. Crédits photos: \url{http://www.meteofrance.com} et \url{http://www.exworld.fr}}{fig:ciel}

\item \underline{Taille des cibles grande devant la longueur d'onde du rayonnement incident} La diffusion par les particules les plus grosses, par exemple les gouttes de brume de quelques centaines de microns, les gouttes de pluie de l'ordre du mm, les cristaux de glace de quelques dizaines de microns, ou les poussières les plus grosses, peut être expliquée par les lois de l'\voc{optique géométrique}, les lois qui gouvernent le fonctionnement des lentilles convergentes/divergentes. Contrairement à la diffusion de Rayleigh, la diffusion est non sélective, c'est-à-dire qu'elle ne dépend pas de la longueur d'onde. Les gouttes d'eau de l'atmosphère diffusent toutes les longueurs d'onde de façon quasiment équivalente, ce qui produit un rayonnement blanc. Ceci explique pourquoi le brouillard et les nuages nous paraissent blancs. La réalité d'un nuage est parfois plus complexe~: ses propriétés radiatives dépendent de la taille des particules et leur nombre par unité de volume.
\item \underline{Taille des cibles grande devant la longueur d'onde du rayonnement incident} La diffusion par les particules de taille intermédiaire, par exemple les gouttes nuageuses ou les aérosols de plus petite taille (quelques microns), est plus délicate à étudier que les deux mécanismes précédemment cités. On parle de \voc{diffusion de Mie}. Les particules soulevées pendant une tempête de poussière sur Terre ou sur Mars causent par diffusion de Mie une couleur orangée au ciel. Suivant la taille et la nature de la particule interagissant avec le rayonnement, la diffusion de Mie peut avoir des caractéristiques très directionnelles [figure \ref{fig:diffdir}]. La section efficace~$\Sigma_\lambda^r$ de la diffusion de Mie suit une loi en l'inverse de~$\lambda^2$ avec la longueur d'onde~$\lambda$ du rayonnement incident. Ces variations sont donc moins sensibles à la longueur d'onde que dans le cas de la diffusion de Rayleigh.
\end{finger}

\figside{0.6}{0.25}{\figfrancis/WH_diff_dir}{Répartition de la probabilité de diffusion dans différentes directions, pour différents types de diffusion: (a) Rayleigh, (b) et (c) Mie avec une particule plus grande en (c).}{fig:diffdir}




\mk \section{Absorption} Au cours de leur pénétration dans l'atmosphère, les photons entrent en collision avec les molécules gazeuses et sont progressivement absorbés. L'étude de ces processus peut être extrêmement complexe, on donne donc ici uniquement les aspects les plus élémentaires pour comprendre les phénomènes en jeu.

	\sk \subsection{Aspect macroscopique~: loi de Beer-Lambert-Bouguer}
	
		\sk
L’étude de l’absorption du rayonnement par un milieu repose sur deux relations établies au XVIIIe siècle qui comparent la luminance spectrale~$L_\lambda(0)$ d'un faisceau incident monochromatique pénétrant sous incidence normale dans un milieu matériel absorbant, à la luminance~$L_\lambda(\ell)$ après traversée du milieu de longueur~$\ell$.
\begin{citemize}
\item D'une part, le faisceau incident subit une extinction telle que le rapport~$\log \frac{L_\lambda(\ell)}{L_\lambda(0)}$ est proportionnel à la longueur~$\ell$ parcourue par le rayonnement dans le milieu absorbant (relation de Bouguer-Lambert).  
\item D'autre part, dans un milieu de concentration molaire effective~$[X]$ en espèce absorbante X, le faisceau incident subit une extinction telle que le rapport~$\log \frac{L_\lambda(\ell)}{L_\lambda(0)}$ est proportionnel à la concentration~$[X]$ (relation de Beer). Le facteur de proportionnalité dépend de l'épaisseur du milieu traversé, de sa nature, de sa composition, de la température et de la longueur d'onde du rayonnement incident.
\end{citemize}
Le logarithme indique que les variations relatives d'énergie radiative au cours de la propagation dans le milieu sont proportionnelles à la longueur parcourue et à la concentration d'espèces absorbantes.

\sk
Les deux lois historiques décrites ci-dessus décrivent exactement la situation de la section~\ref{sec:efficace} si l'on néglige la diffusion. Par rapport à la section~\ref{sec:efficace}, la seule hypothèse supplémentaire est que la section efficace d'absorption~$\Sigma_\lambda^a$ ne dépend que de la longueur d'onde et du matériau, elle est considérée comme uniforme sur toute la longueur traversée. Si l'on reprend la situation de la figure~\ref{fig:beer} avec cette hypothèse, on a 
\[ \dd L_\lambda = - L_\lambda(z) \, \Sigma_\lambda^a \, N_X \, \frac{1}{\cos\theta} \, \dd z \]
où $N_X$ est la concentration de l'espèce absorbante exprimée sous la forme d'un nombre de molécule par unité de volume (au lieu de $[X]$ en mol~L$^{-1}$).
La variation de flux au cours de la traversée du milieu matériel est donc
\[ \frac{\dd L_\lambda}{L_\lambda} = - \Sigma_\lambda^a \, N_X \, \frac{1}{\cos\theta} \, \dd z \]
Par intégration, si l'on suppose que~$z=0$ repère l'entrée dans le milieu matériel et~$z=\ell$ la sortie, on obtient
\[ \int_{z=0}^{z=\ell} \, \frac{\dd L_\lambda}{L_\lambda} = - \Sigma_\lambda^a \, N_X \, \frac{1}{\cos\theta} \, \int_{z=0}^{z=\ell} \dd z \]
d'où la valeur de la luminance spectrale à la sortie du milieu traversé, donnée par deux relations équivalentes
\[ \boxed{ \log \frac{L_\lambda(\ell)}{L_\lambda(0)} = - \zeta \, \ell } \qquad \textrm{et} \qquad \boxed{ L_\lambda(\ell) = L_\lambda(0) \, e^{- \zeta \, \ell} } \qquad \textrm{avec la constante en m}^{-1} \qquad \zeta = \Sigma_\lambda^a \, N_X \, \frac{1}{\cos\theta} \]
Cette loi reprend les résultats historiques présentés précédemment et porte le nom de \voc{loi de Beer-Lambert-Bouguer}. Elle indique la décroissance exponentielle du flux incident lors de sa traversée du milieu, d'autant plus importante que la longueur traversée~$\ell$ est grande et que la concentration en absorbant~$N_X$ du milieu est grande. La loi de Beer-Lambert-Bouguer peut également être appliquée sous la même forme avec l'éclairement. Le coefficient~$\zeta$ porte parfois le nom de coefficient d'extinction linéique. 
%Nous ferons l’hypothèse par la suite que cette dépendance est linéaire, ce qui est vrai pour l’air.
On donne quelques ordres de grandeur ci-dessous pour la valeur de~$\zeta$
\begin{citemize}
\item atmosphère pour un rayonnement visible dans le jaune $\zeta = 1 \times 10^{-5}$~m$^{-1}$ 
\item atmosphère pour un rayonnement visible dans le violet $\zeta = 4 \times 10^{-5}$~m$^{-1}$
\item verre $\zeta = 0.2$~m$^{-1}$
\item nuage bas $\zeta = 1 \times 10^{-3}$~m$^{-1}$
\end{citemize}
En remarquant que le rapport~$L_\lambda(\ell) / L_\lambda(0)$ définit justement la coefficient de transmission spectral~$\tau_\lambda$ \emph{en l'absence de diffusion}, on arrive à
\[ \tau_\lambda = e^{- \zeta \, \ell} \qquad \textrm{et} \qquad \alpha_\lambda = 1 - e^{- \zeta \, \ell} \]
On peut vérifier que l'expression est conforme à l'intuition : si la longueur de la traversée est particulièrement grande ($\ell \rightarrow \infty$), et/ou que l'espèce absorbante~X est très concentrée ($N_X \rightarrow \infty$), alors presque tout le rayonnement incident est absorbé~$\alpha_\lambda \rightarrow 1$ et une partie négligeable de ce rayonnement est transmise~$\tau_\lambda \rightarrow 0$. 

\sk
En sciences de l'atmosphère, on utilise souvent une forme plus générale de la loi de Beer-Lambert-Bouguer. On écrit la relation intermédiaire (non intégrée) qui conduit à cette loi sous la forme 
\[ \frac{\dd L_\lambda}{L_\lambda} = \, \rho_X \, k_\lambda \, \frac{1}{\cos\theta}  dz \]
où $k_\lambda$ est un coefficient d'absorption massique en m$^2$~kg$^{-1}$ et $\rho_X(z)$ est la densité d'absorbant~X, qui dépend de~$z$ comme ce peut être le cas dans l'atmosphère. Cette relation peut être intégrée sur une couche épaisse située entre les niveaux~$z_1$ et~$z_2$. On obtient 
\[ L_\lambda(z_1) = L_\lambda(z_2) \, e^{- \frac{t_\lambda}{\cos\theta}} \]
où 
\[ t_\lambda = \int_{z_1}^{z_2} \, k_\lambda \, \rho_X \, \dd z \]
est appelée l'épaisseur optique de la couche. Si l'extinction est uniquement due à de l'absorption, sans diffusion, on a une relation directe entre l'épaisseur optique et le coefficient d'absorption de la couche: 
\[\alpha_\lambda = 1 - e^{- \frac{t_\lambda}{\cos\theta}} \]
%Dans le cas particulier où la densité d'absorbant est de la forme
%\[\rho_a=\rho_a^0e^{-z/H_a}\]
%ce qui est le cas par exemple d'un gaz bien mélangé dans l'atmosphère, ou de
%la vapeur d'eau, on peut calculer l'altitude du taux d'extinction
%$dL_\lambda/dz$ maximum: on a alors également d'après la définition de
%$\tau_\lambda$
%\[\tau_\lambda=\tau_\lambda^0e^{-z/H_a}\]
%et $d\tau_\lambda/dz=-\tau_\lambda/H_a$. D'autre part, le taux d'extinction vaut 
%\[dL_\lambda/dz=-L_\lambda\mu d\tau_\lambda/dz=L_\lambda^\infty
%e^{-\mu\tau_\lambda}\mu\tau_\lambda/H_a\]
%Ce taux est maximal pour 
%\[d\left(\tau_\lambda\mu e^{-\tau_\lambda\mu}\right)=0\]
%soit pour $\mu\tau_\lambda=1$. On a donc un maximum d'extinction (absorption
%ou diffusion) du rayonnement incident pour une épaisseur optique de 1
%traversée à partir du sommet de l'atmosphère. Pour des épaisseurs optiques
%plus faibles, on a peu d'extinction car la densité d'absorbants est faible.
%Pour des épaisseurs optiques plus grandes, on a beaucoup d'absorbants mais la
%luminance résiduelle est petite (figure \ref{fig:absrate}).
% 
%\begin{figure}[tbp]
%  \begin{center}
%    \includegraphics{\figfrancis/WH_abs_max}
%  \end{center}
%  \caption{Comparaison des structures verticales de la densité de
%  l'atmosphère $\rho$, de la luminance d'un rayonnement incident $L_\lambda$
%  et de sa dérivée verticale. L'échelle horizontale est linéaire pour chaque
%  grandeur.}
%  \label{fig:absrate}
%\end{figure}



	\sk \subsection{Aspect microscopique~: absorption par les gaz et liaisons moléculaires}

		\sk
On donne ici quelques éléments éclairants sur les processus en jeu à l'échelle microscopique lors de l'absorption de rayonnement incident (photons) par les molécules qui composent l'atmosphère\footnote{Cette partie est inspirée d'éléments trouvés dans le cours de S. Jacquemoud de \emph{Méthodes physique en télédétection.}}. Cette absorption est en fait liée à leurs caractéristiques énergétiques. L'énergie d'une molécule ne peut prendre que des valeurs discrètes correspondant à des niveaux énergétiques. On les représente
souvent par un diagramme dans lequel chaque niveau est figuré par un trait horizontal. L'absorption permet à une molécule de passer d'un niveau d'énergie~$e_1$ à un niveau d'énergie supérieur~$e_2$. Le passage du niveau~$e_1$ au niveau~$e_2$ s'accompagne de l'absorption d'un rayonnement de fréquence~$\nu$ telle que~$\Delta e = e_2 - e_1 = h \, \nu$, soit l'énergie d'un photon de fréquence~$\nu$. Les molécules possèdent une énergie électronique~$e\e{e}$, quantifiée comme les atomes, mais aussi une énergie de vibration~$e\e{v}$ et une énergie de rotation~$e\e{r}$, elles aussi quantifiées. Une bonne approximation de l'énergie totale $e\e{t}$ est donnée par la relation~$e\e{t} = e\e{e} + e\e{v} + e\e{r}$. A chaque état électronique correspondent plusieurs états de vibration des noyaux et à chaque état vibrationnel correspondent plusieurs états de rotation.

\figsup{0.47}{0.17}{decouverte/cours_meteo/electronique.png}{decouverte/cours_meteo/vibre.png}{Niveaux électroniques définissant des états de molécules [gauche]. Effet de l'interaction entre rayonnement et molécules pour plusieurs longueurs d'onde. Figures extraites du cours de S. Jacquemoud de \emph{Méthodes physique en télédétection.}}{fig:electronique}

\sk
L'interaction entre le rayonnement et les molécules constituant le milieu (par exemple, l'atmosphère) se manifeste d'une façon différente selon l'énergie~$h\nu$ du photon incident. Ainsi, par ordre décroissant de l'énergie du photon incident, donc par ordre croissant de sa longueur d'onde~$\lambda$, le photon va provoquer sur les liaisons moléculaires des brisures, des réorganisations de nuage électronique, des vibrations, ou simplement des rotations.
\begin{finger}
\item dans l'ultraviolet : les molécules (O$_2$, O$_3$, NO$_2$, \ldots) sont dissociées. La photolyse ou \voc{photodissociation} d'une espèce est provoquée par l'absorption d'un photon possédant une énergie suffisante pour conduire cette espèce à un état électronique excité puis finalement à une rupture de liaison. Citons les cas classiques de la dissociation de l'oxygène [O$_2$~+~$h\nu$~$\rightarrow$~O~+~O] pour des longueurs d'onde inférieures à 246 nm, ou de l'ozone [O$_3$~+~$h\nu$~$\rightarrow$~O$_2$~+~O] pour des longueurs d'onde inférieures à 310 nm. La photodissociation de NO$_2$ produit les atomes d’oxygène nécessaires à la formation photochimique de l’ozone troposphérique [NO$_2$~+~$h\nu$~+O$_2$~$\rightarrow$~NO~+O$_3$]. Ces phénomènes de photolyse participent très fortement à la chimie de l'atmosphère.
\item dans le visible : les molécules changent de configuration électronique ; les électrons qui gravitent autour du noyau atomique peuvent changer d'orbite ou même d'atome. Les photons du domaine du visible ne sont presque pas absorbés par l'atmosphère (très légèrement par O$_2$ et O$_3$) et sont donc uniquement diffusés.
\item dans l'infrarouge moyen et thermique : les molécules (CO$_2$, H$_2$O, CH$_4$, N$_2$O, \ldots) vibrent dans l'axe de la liaison moléculaire (étirement) ou perpendiculairement à cet axe (pliage). Ces molécules sont appelées \voc{gaz à effet de serre} car elles absorbent le rayonnement infrarouge thermique émis par la Terre puis réemettent des photons à la même longueur d'onde.
\item dans le domaine des micro-ondes : les molécules tournent autour d'un de leurs axes.
\end{finger}


		\sk
Les molécules de l'atmosphère absorbent donc le rayonnement à diverses longueurs d'onde. En conséquence, on comprend que les coefficients d'absorption des gaz qui composent l'atmosphère sont extrêmement variables en fonction de~$\lambda$ et présentent une structure très complexe. Un domaine limité de longueurs d'onde contigues où une certaine espèce atmosphérique est très absorbante est appelé \voc{bande d'absorption}. Certaines espèces possèdent des bandes d'absorption dans les longueurs d'onde visible, comme l'ozone~O$_3$, d'autres dans les longueurs d'onde infrarouge, comme les gaz à effet de serre CO$_2$ et H$_2$O [Figure~\ref{fig:atmspectrum} et table~\ref{tab:abs}]. Un domaine limité de longueurs d'onde contigues où les espèces principales qui composent une atmosphère ne sont pas (trop) absorbantes est appelée \voc{fenêtre atmosphérique}, car alors le coefficient de transmission atmosphérique est proche de~$1$.

\small
\begin{table}\label{tab:abs}
\begin{center}
\begin{tabular}{|c|c|}
\hline
Molécules & Principales bandes d'absorption (en $\mu$m) \\
\hline
O$_3$ & 0,242-0,31 (Hartley) / 0,31-0,4 (Huggins) / 0,4-0,85 (Chappuis) / 3,3 / 4,74 \\
O$_2$ & 0,175-0,2 (Schumann-Runge) / 0,2-0,26 (Herzberg) / 0,628 / 0,688 / 0,762 / 1,06 / 1,27 / 1,58 \\
CO$_2$ & 1,4 / 1,6 / 2,0 / 2,7 / 4,3 / >15 \\
H$_2$O & 0,72 / 0,82 / 0,94 / 1,1 / 1,38 / 1,87 / 2,7-3,2 / 6,25 / >14 \\
CH$_4$ & 1,66 / 2,2 / 2,3 / 2,37 / 3,26 / 3,31 / 3,53 / 3,83 / 3,55 / 7,65 \\
CO & 2,34 / 4,67 \\
N2O & 2,87 / 2,97 / 3,9 / 4,06 / 4,5 \\
\hline
\end{tabular}
\caption{\emph{Principales bandes d'absorption pour les gaz composant l'atmosphère terrestre. Voir la figure~\ref{fig:atmspectrum}}}
\end{center}
\end{table}
\normalsize

\figside{0.7}{0.35}{decouverte/cours_dyn/absorption.png}{Spectres d'absorption de l'atmosphère en fonction de la longueur d'onde. [Haut] Courbes d'émittance normalisée de corps noirs à 5780~K (rayonnement solaire) et 255~K (rayonnement terrestre). [Bas] Coefficients d'absorption (en~$\%$) entre le sommet de l'atmosphère et la surface. Les principaux gaz responsables de l'absorption à différentes longueurs d'onde sont indiqués en bas. Source: McBride and Gilmour, An Introduction to the Solar System, 2004 ; d'après Goody and Yung, Atmospheric radiation, 1989}{fig:atmspectrum}




\mk
\section{Une rapide synthèse~: spectre solaire à la surface de la Terre}

\sk
La figure~\ref{fig:radiation} compare le rayonnement solaire incident au sommet de l'atmosphère et à la surface, en fonction de la longueur d'onde. Dans les longueurs d'ondes visibles, où se situe la majorité du rayonnement solaire incident, une fraction du rayonnement, appelée albédo, est réfléchie vers l'espace~: les molécules constituant l'atmosphère, les aérosols et nuages contribuent tous à cet albedo planétaire (voir également figure~\ref{fig:diffsep}). En analysant conjointement la figure~\ref{fig:atmspectrum}, on voit que le rayonnement ultraviolet est complètement absorbé par l'ozone et l'oxygène. Restent que ces gaz diatomiques simples (comme c'est le cas de H$_2$, N$_2$) constituent un milieu presque complètement transparent à la fois au rayonnement solaire [surtout visible, ou \ofg{ondes courtes}] et au rayonnement terrestre [surtout infrarouge, ou ondes longues]. En revanche, certains gaz composés (en particulier CO$_2$, H$_2$O, N$_2$O, CH$_4$, CO \ldots) ne sont que partiellement transparents, car, s'ils sont transparents au rayonnement courte longueur d’onde (et donc à une grande partie du rayonnement solaire), ils absorbent plus ou moins fortement le rayonnement grande longueur d’onde, notamment le rayonnement émis par la Terre. Ainsi, contrairement à ce qui prévaut dans les longueurs d'onde visible, l'atmosphère est très opaque dans l'infrarouge (longueurs d'onde quelques microns). On distingue seulement une fenêtre ente 8 et 12 microns où le rayonnement émis par la surface s'échappe en grande partie vers l'espace. Les principaux absorbants sont la vapeur d'eau et le CO$_2$, d'autres gaz comme le méthane ou l'ozone ayant des contributions plus faibles. La plupart des gaz absorbent dans des bandes étroites. La vapeur d'eau au contraire absorbe dans un domaine spectral très large. Le spectre donné en figure~\ref{fig:atmspectrum} est valable en ciel clair et ne tient pas compte de la présence de nuages. L'eau liquide (ou glace) est un très fort absorbant dans l'infrarouge à toutes les longueurs d'ondes, et un nuage même peu épais absorbera donc rapidement la quasi-totalité du rayonnement incident. 

\figsup{0.49}{0.25}{\figpayan/LP211_Chap2b_Page_18_Image_0001.png}{\figpayan/LP211_Chap1_Page_07_Image_0004.png}{[Gauche] Spectre du rayonnement solaire (UV, Visible, IR) à l’extérieur de l’atmosphère terrestre et au niveau du sol (après traversée de l’atmosphère). Le rayonnement au sommet de l'atmosphère correspond à la courbe rouge. Le rayonnement diffusé par l'atmosphère correspond à la courbe bleu foncé. Le rayonnement absorbé par l'atmosphère correspond à la courbe bleu clair. Le rayonnement résultant au sol correspond à la courbe jaune. [Droite] Figure similaire, mais les espèces gazeuses responsables des bandes d'absorptions sont indiquées sur la figure.}{fig:radiation}

%\end{document}

%\mk \section{Synthèse : transfert radiatif dans l'infrarouge} On peut négliger la diffusion du rayonnement infrarouge dans l'atmosphère terrestre: la diffusion Rayleigh est très inefficace aux grandes longueurs d'onde, et les particules plus grosses (comme les gouttes d'eau) sont typiquement très absorbantes et donc peu diffusives.

%\begin{figure}[tbp]
%  \begin{center}
%    \includegraphics[width=\figw]{\figfrancis/Schwartzschild}
%  \end{center}
%  \caption{Variation du rayonnement infrarouge montant incident sur une couche
%  d'épaisseur $dz$: absorption et émission par la couche.}
%  \label{fig:schwartzschild}
%\end{figure}

%On doit par contre considérer en plus de l'absorption de rayonnement, l'émission dans l'infrarouge par l'atmosphère (figure \ref{fig:schwartzschild}). Dans les conditions appelées \emph{équilibre thermodynamique local\footnote{Ces conditions sont vérifiées si les collisions entre molécules sont plus fréquentes que l'absorption ou émission de rayonnement. Les molécules émettrices ont alors la même température que leur environnement}}, qui sont valables jusque vers 60~km d'altitude environ, le rayonnement émis (vers le haut et vers le bas) par une couche mince d'atmosphère dépend de sa température et de son coefficient d'absorption suivant la loi de Kirchoff. Pour un faisceau lumineux traversant une couche mince d'atmosphère, la variation de luminance vaut alors: \[dL_\lambda=\left(-L_\lambda+B_\lambda(T)\right)\mu d\tau_\lambda\] Le premier terme du second membre représente l'absorption du rayonnement incident, le deuxième l'émission par les gaz de la couche. La loi de Kirchoff fait qu'ils sont multipliés par le même coefficient $\mu\tau_\lambda$ qui donne le coefficient d'absorption et d'émission de la couche. Cette équation est appelée \emph{équation de Schwartzschild}. Son intégration entre une altitude $z_0$ et l'infini (espace) donne: \[L_\lambda(\infty)=L_\lambda(z_0)e^{-\mu\tau_\lambda(z_0,\infty)}+\int_{z_0}^\infty B_\lambda(T)e^{-\mu\tau_\lambda(z,\infty)}\mu\rho_ak_\lambda dz\] Le rayonnement sortant qu sommet de l'atmosphère est donc la somme du rayonnement présent en $z_0$ diminué de l'absorption entre $z_0$ et le sommet de l'atmosphère (premier terme), et de l'intégrale de la contribution du rayonnement émis par chaque couche au dessus de $z_0$. Comme pour le transfert dans le visible, on peut montrer (en supposant que $T$ varie peu) que la contribution maximale au rayonnement sortant à une longueur d'onde $\lambda$ provient d'une épaisseur optique de $\tau_\lambda=1$ à partir du sommet de l'atmosphère.
