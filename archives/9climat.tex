\chapter{Eléments sur le changement climatique récent}

%\dictum[Charles Baudelaire, 1857]{\scriptsize Quand le ciel bas et lourd pèse comme un couvercle\\ Sur l'esprit gémissant en proie aux longs ennuis,\\ Et que de l'horizon embrassant tout le cercle\\ Il nous verse un jour noir plus triste que les nuits ;}
\dictum[President Lyndon B. Johnson, 1965]{This generation has altered the composition of the atmosphere on a global scale through [\ldots] a steady increase in carbon dioxide from the burning of fossil fuels.}

\bk
Ce chapitre se propose de donner quelques éléments factuels sur le changement climatique récent. 

\mk
\section{Le système climatique}

\sk
Il est utile de commencer par rappeler la distinction entre météorologie et climatologie. La \voc{météorologie} est la discipline ayant pour objet l'étude des phénomènes atmosphériques et de leurs variations à court terme. La \voc{climatologie} est la discipline ayant pour objet l'étude des conditions atmosphériques moyennes d'un pays, d'une région, d'une planète, et de leurs variations à moyen ou long terme. Une définition distincte de la climatologie est qu'elle vise à étudier le \voc{système climatique}, un système thermo-hydrodynamique non isolé dont les composantes sont les principales « enveloppes » externes de la Terre~: l'atmosphère (l’air, les nuages, les aérosols), l’hydrosphère (les océans, les rivières, les précipitations), la lithosphère (les terres immergées, les sols), la cryosphère (glace, neige, banquise, glaciers), la biosphère (les organismes vivants, \ldots), l’anthroposphère (l’activité humaine). Cette définition est équivalente à la première dans la mesure où les variations de l'état de l'atmosphère au moyen ou long terme ne peut se penser sans l'étude des couplages avec les autres compartiments du système climatique (voir figure~\ref{fig:pluri}). 

\sk
Les diverses composantes du système climatique interagissent entre elles sur un large intervalle d'échelles d'espace et de temps (figure~\ref{fig:pluri}). De plus, le système climatique dépend de facteurs externes, tels le rayonnement électromagnétique reçu du soleil ou les variations des paramètres orbitaux de la Terre. Le système climatique et ses évolutions est donc étudié par un ensemble de communautés scientifiques (climatologues, hydrologues, glaciologues, géochimistes, géologues, astronomes, \ldots) qui collaborent et confrontent leur point de vue depuis plusieurs décennies et communiquent leurs résultats dans les publications scientifiques\footnote{Dans ce chapitre, nous reprenons beaucoup de graphiques tirés de rapport du GIEC (Groupe d'experts intergouvernemental sur l'évolution du climat -- IPCC en anglais). Ce rapport offre un compte-rendu synthétique à destination du grand public et des décideurs. Il se base pour cela sur de nombreuses publications scientifiques. Le document le plus récent (AR4 2007) récapitulant les bases physiques est disponible à cette adresse \url{http://www.ipcc.ch/publications_and_data/ar4/wg1/en/contents.html}, chaque point peut être approfondi avec les publications référencées.}. Ces travaux ont permis d'établir que la Terre subit actuellement un réchauffement de son climat causé pour sa plus grande partie par l'activité humaine (on parle de changement \voc{anthropique}). Ce consensus n'élude pas les questions qui restent à résoudre et sur lesquelles la communauté scientifique travaille activement. Dans la mesure où les impacts sociétaux sont majeurs, les changements du climat d'origine anthropique et ses conséquences doivent être connus le plus précisément et rigoureusement possible. 

\mk
\section{Variations récentes de température et de composition}

\sk
\subsection{Variations de température}

\figun{0.8}{0.65}{decouverte/cours_meteo/evolutemp20.png}{Moyenne annuelle globale et hémisphérique de la température à la surface de la Terre de 1850 à 2006 relative à la moyenne de 1961 à 1990. Les données combinent des mesures sur les continents (à proximité de la surface) et sur les océans (\emph{sea surface temperature}). Les barres d'erreurs (intervalle~$5-95\%$) sont reportées sur la figure. La figure est adaptée des travaux du projet HadCRUT3 (Brohan et al. 2006). La courbe lissée bleue représente les variations décennales. Source~:~IPCC Fourth Assessment Report: Climate Change 2007 (AR4).}{fig:evolutemp20}
%%Global and hemispheric annual combined land-surface air temperature and SST anomalies (°C) (red) for 1850 to 2006 relative to the 1961 to 1990 mean, along with 5 to 95% error bar ranges, from HadCRUT3 (adapted from Brohan et al., 2006). The smooth blue curves show decadal variations

\sk
Le climat peut varier aux cours du temps, à des degrés et des échelles de temps divers. Le but de ce chapitre n'est pas de traiter les nombreux changements du climat survenus par le passé, une science appelée \voc{paléoclimatologie}, mais de s'intéresser aux changements récents survenus au cours du XXe et du XXIe siècle. L'évolution de la température de surface mesurée depuis les~$140$ dernières années est reportée en figure~\ref{fig:evolutemp20}. On peut remarquer la variabilité significative entre les années (variabilité interannuelle) et entre les décennies (variabilité décennale); on observe également une tendance générale au réchauffement d'environ~$0.6^{\circ}$C sur la période considérée. Ces changements peuvent être mis en perspective avec les variations depuis quelques milliers d'années, obtenues par le croisement et la combinaison statistique d'indicateurs indépendants~: cernes des arbres, coraux, carottes glaciaires, relevés historiques (Figure~\ref{fig:multitemp}, on peut noter sur ces données l'apparition entre~$1400$ et~$1900$ environ d'une période de climat légèrement plus froide appelée le \og petit âge glaciaire \fg). L'augmentation de température constatée récemment apparaît ainsi comme particulièrement exceptionnelle par rapport aux évolutions passées. La décénnie 1990-2000 a été la plus chaude de la période instrumentale sur la figure~\ref{fig:evolutemp20} et probablement parmi les décennies les plus chaudes du dernier millénaire.

\figun{0.85}{0.25}{decouverte/cours_meteo/yiou_multiproxies_temp.png}{Pourcentage de recouvrement des reconstructions \og multi-sources \fg~de la température annuelle moyenne. Les sources des données passées sont les suivantes~:~cernes des arbres (dendrochronologie), coraux, carottes glaciaires, relevés historiques (par exemple, dates de vendanges). Plus le grisé est intense plus les reconstructions se recoupent. La ligne noire indique la température instrumentale (i.e. observée par un thermomètre selon la même méthode au cours des années) de Brohan et al (2006) (voir figure~\ref{fig:evolutemp20}). Les valeurs sont centrées autour de la moyenne de la période 1961-1990. Les séries sont lissées avec une fenêtre de 30 ans. Source~:~J. Guiot et P. Yiou \emph{in} Le Climat à Découvert, CNRS éditions, 2011. Voir aussi IPCC Fourth Assessment Report: Climate Change 2007 (AR4).}{fig:multitemp}

\sk
\subsection{Causes possibles de changements climatiques}

\sk
Quelles sont les causes possibles d'un changement du climat terrestre~? Elles peuvent être nombreuses.

\begin{finger}
\item Une première cause peut être l'augmentation de gaz à effet de serre (H$_2$O, CO$_2$, CH$_4$, N$_2$O) dont l'action sur la température de l'atmosphère et de la surface est primordiale. Cette raison ne peut être évoquée pour les réchauffements ou refroidissements plus anciens, par exemple au début du XXe siècle, puisque les variations significatives de gaz à effet de serre causées par l'activité humaine sont relativement récentes. Néanmoins, les évolutions climatiques très anciennes, à l'échelle de plusieurs centaines de millions d'années, ont pu être causées en partie par la modification de la concentration en gaz à effet de serre dans l'atmosphère.
\item Une deuxième cause peut être trouvée dans les variations d'activité volcanique, qui injectent de larges quantités de poussière et de gaz (tels SO$_2$) dans l'atmosphère qui peuvent conduire à la formation d'aérosols sulfatés, dont l'effet est notamment de réduire la quantité de rayonnement incident. Ceci peut provoquer un refroidissement, typiquement~de l'ordre de~$0.5^{\circ}$C en moyenne globale, sur une période d'environ deux ans suivant une éruption majeure (e.g. Pinatubo en 1991, voir l'effet détectable sur la figure~~\ref{fig:evolutemp20}).
\item Une troisième explication réside dans les variations dans l'intensité du rayonnement solaire reçu par la Terre. En premier lieu, des variations absolues peuvent provenir de l'activité solaire. Des mesures directes très précises de cette quantité ont été obtenues depuis l'orbite de la Terre au cours des trois dernières décennies~: des variations d'environ~$0.1\%$ de la constante solaire apparaissent au cours d'un cycle solaire. Pour les variations passées, des preuves indirectes de fluctuations du rayonnement solaire incident existent, notamment lors du Minimum de Maunder au XVIIe siècle, une période au cours de laquelle un nombre exceptionnellement bas de taches solaires a été enregistré. En second lieu, des variations relatives (suivant les hémisphères ou les saisons) d'intensité du rayonnement solaire reçu peuvent provenir des variations sur de grandes échelles de temps des paramètres orbitaux de la Terre (cycles de Milankovitch).
\item Une quatrième explication tient au fait que, même sans changement de facteur extérieur au système climatique, des variations du climat peuvent apparaître en raison de facteurs et couplages internes, en particulier les interactions entre l'atmosphère et l'océan. La bonne prise en compte de cette variabilité interne ne doit d'ailleurs pas être négligée, sous peine de sous-estimer notamment les variations climatiques induites par les changements orbitaux.
\end{finger}

\sk
Tous ces effets (gaz à effet de serre, aérosols volcaniques, variations de constante solaire, couplages internes) ont été inclus au sein des modèles du système climatique développés de façon indépendante par plusieurs équipes dans le monde (figure~\ref{fig:causewarming}). Les changements climatiques à l'échelle du millénaire peuvent être reproduits correctement. Sur les échelles de temps plus courtes, la variabilité interannuelle (qui dépend fortement des couplages océan-atmosphère) est reproduite, ainsi que l'effet des éruptions volcaniques. Néanmoins, le réchauffement le plus récent, observé au cours du XXe siècle, ne peut être expliqué par des causes naturelles comme l'indique la figure~\ref{fig:causewarming}~: l'influence de l'activité humaine, notamment l'augmentation de la concentration de gaz à effet de serre dans l'atmosphère, ne peut être négligée. De nombreux autres indices montrent que le réchauffement climatique entamé dans la seconde partie du XXe siècle et le début du XXIe siècle peut être principalement imputé à l'activité humaine et l'émission de gaz à effet de serre (voir aussi figure~\ref{fig:forcrad}).

\figside{0.55}{0.43}{decouverte/cours_meteo/climatadecouvert_119.png}{Comparaison entre les anomalies de temperature globale de surface ($^{\circ}$C) observée (en noir) et simulée par plusieurs modèles climatiques couplés globaux. En (a) les forçages anthropiques et naturels sont pris en compte, tandis qu’en (b), seuls les forçages naturels ont été considérés. Source~:~IPCC Fourth Assessment Report: Climate Change 2007 (AR4).}{fig:causewarming}

%\figun{0.8}{0.4}{decouverte/cours_meteo/figure-spm-4-l.png}{Comparaison des changements observés sur la température de surface aux échelles globale et continentale, avec les résultats de simulations de modèles climatiques utilisant les forçages naturels et anthropiques. Les moyennes décennales des observations sont montrées pour la période 1906–2005 (ligne noire) représentée selon le centre de la décennie et relative à la moyenne correspondante sur la période 1901–1950. Les lignes en pointillés sont utilisées quand la couverture spatiale est inférieure à 50\%. Les bandes bleues ombrées représentent l’intervalle de 5–95\% pour 19 simulations de cinq modèles climatiques qui n’utilisaient que les forçages naturels dus à l’activité solaire et aux volcans. Les bandes rouges ombrées représentent l’intervalle de 5–95\% pour 58 simulations provenant de 14 modèles climatiques utilisant des forçages à la fois naturels et anthropiques. Source~:~IPCC Fourth Assessment Report: Climate Change 2007 (AR4).}{fig:anthropcause}

\sk
\subsection{Changements de composition récents et cycles naturels}

\sk
Il faut d'ores et déjà distinguer l'\voc{empreinte climatique} de l'homme (liée à l'impact au long terme que peuvent avoir les activités humaines en modifiant la composition atmosphérique) de la \voc{pollution atmosphérique} qui traite de la qualité de l'air dans lequel l'homme évolue au quotidien. Un exemple d'empreinte climatique a été vu dans le chapitre de chimie atmosphérique lorsque l'action néfaste des CFC sur l'équilibre du cycle de l'ozone stratosphérique a été abordée. 

\sk
Un autre exemple d'empreinte climatique humaine, qui nous intéresse tout particulièrement ici, est l'augmenta\-tion de la concentration de gaz à effet de serre dans l'atmosphère terrestre. Le plus important des gaz à effet de serre rejeté dans l'atmosphère est le CO$_2$. Son augmentation en concentration par rapport à la période pré-industrielle est d'environ~$30\%$ (figure~\ref{fig:maunaloa}). Cette augmentation est principalement due à la combustion de carburants fossiles (comme l'ont montrés de multiples relevés d'isotopes du carbone) et à la déforestation. D'autres gaz à effet de serre ont augmenté significativement, comme le méthane (145\%) et l'oxyde nitreux (15\%) (figure~\ref{fig:giecrad}).

\figside{0.5}{0.25}{decouverte/cours_meteo/Mauna_Loa_Carbon_Dioxide.png}{Relevé des concentrations de dioxyde de carbone atmosphérique à l'observatoire de Mauna Loa à Hawaii. Les courbes obtenues portent le nom de \og courbes de Keeling \fg~du nom du scientifique qui a supervisé ses observations. D'autres jeux de données sont venus confirmer les tendances observées depuis l'observatoire de Mauna Loa; on montre celles-ci en exemple car il s'agit de l'enregistrement direct le plus étendu dans le temps de la concentration du CO$_2$ dans l'atmosphère. La courbe rouge montre les concentrations moyennées sur un mois et la courbe bleue montre le résultat d'une moyenne glissante sur~$12$ mois. Noter les fluctuations saisonnières de CO$_2$ (reportées dans le sous-graphique en bas à droite) qui correspondent à la consommation de dioxyde de carbone par la végétation~: ces fluctuations sont plus marquées dans l'hémisphère nord en raison de la couverture végétale (forêts) plus importante que dans l'hémisphère sud. Source~:~Robert A. Rohde \url{http://www.globalwarmingart.com} à partir de données publiées et référencées.}{fig:maunaloa}

\figside{0.6}{0.75}{decouverte/cours_meteo/ges.jpg}{Rapports de mélange du dioxyde de carbone~CO$_2$, méthane~CH$_4$ et oxyde nitreux~N$_2$O au cours des 10 000 dernières années (larges figures) et depuis 1750 (petits inserts). Les données indicatrices des changements de la composition de l’atmosphère au cours du dernier millénaire mettent en évidence l’augmentation rapide des gaz à effet de serre qui est imputable principalement à la croissance économique depuis 1750. Les gaz sont bien mélangés dans l’atmosphère et leurs concentrations depuis 1750 reflètent les émissions provenant des sources à travers le monde. Les mesures proviennent de carottes glaciaires (différentes couleurs sont utilisées pour les diverses études scientifiques utilisées) et de campagnes atmosphériques (lignes rouges). Les forçages radiatifs comparés à la valeur de 1750 pour chacun des gaz représentés sont reportés sur la droite des larges figures. Figure tirée du 4ème rapport du GIEC en 2007.}{fig:giecrad}

\sk
Dans la nature, le dioxyde de carbone~CO$_2$ est la principale espèce échangée entre les divers réservoirs qui participent au cycle du carbone~: atmosphère, océan, sol, biosphère. Les flux de carbone sous la forme de dioxyde de carbone qui entrent et sortent de l'atmosphère sont considérables~: à peu près un quart de la quantité contenue dans l'atmosphère est renouvelée par le cycle du carbone chaque année, la moitié échangée avec la biosphère continentale et l'autre moitié via les processus physico-chimiques avec la surface de l'océan. Sur des milliers d'années, avant que les activités humaines n'induisent une perturbation significative, la concentration atmosphérique de dioxyde de carbone s'est maintenue à une valeur constante, indiquant un état stationnaire du cycle du carbone et des échanges entre réservoirs (voir figure~\ref{fig:giecrad}). 

\sk
Pourquoi les émissions supplémentaires de CO$_2$ dans l'atmosphère induites par les activités humaines se traduisent-elles \emph{in fine} par une augmentation de la concentration du CO$_2$ dans l'atmosphère~? Les détails de sa distribution entre les réservoirs majeurs (océan, biosphère) peut être estimée à partir d'une combinaison de mesures des changements de concentrations atmosphériques de CO$_2$ et d'oxygène~: la capture du dioxyde de carbone par l'océan provoque peu de changements en oxygène atmosphérique, alors que la biosphère continentale relâche de l'oxygène et consomme du dioxyde de carbone dans l'atmosphère. Au cours des dernières décénnies néanmoins, une partie significative des émissions de CO$_2$ d'origine anthropique (environ~$40\%$) est restée dans l'atmosphère~: il apparaît que les puits de CO$_2$ (biosphère continentale et océan) ne peuvent résorber en quelques décennies l'augmentation des gaz à effet de serre provoquée par l'activité humaine (figure~\ref{fig:bilancarbone}). Les échanges entre les différents réservoirs se font en effet sur des échelles de temps larges, qui de plus diffèrent selon les réservoirs, de la décénnie pour la biosphère continentale et les compartiments supérieurs de l'océan, jusqu'au millénaire pour la partie profonde de l'océan. La combinaison de ces facteurs ne peut être résumée en une constante de temps unique, néanmoins il est certain que la perturbation anthropique de~CO$_2$ ne peut être résorbée en moins d'une centaine d'années. Ainsi, une partie substantielle du dioxyde de carbone relâché dans l'atmosphère par les activités humaines aujourd'hui va affecter la concentration globale pour (au moins) le siècle à suivre. 

\sk
Le problème se pose en des termes similaires pour l'oxyde nitreux dont la durée de vie dans l'atmosphère est similaire au dioxyde de carbone. Ce problème est moins critique pour le méthane dont la durée de vie, contrainte principalement par des constantes de réaction chimiques, est plutôt de l'ordre de la dizaine d'années. Le cas de la vapeur d'eau est quant à lui singulier~: il s'agit du gaz à effet de serre le plus puissant de l'atmosphère terrestre, un des principaux responsables de l'effet de serre naturel. Les activités humaines n'induisent cependant pas de modifications climatologiques de sa concentration dans l'atmosphère, dans la mesure où le temps caractéristique du cycle de l'eau est de l'ordre de quelques jours\footnote{Il est cependant fort probable que le changement climatique récent s'accompagne de modifications du cycle de l'eau, voir sections suivantes.}.
%La durée de vie des aérosols est quant à elle encore plus faible (quelques jours dans la troposphère, une poignée d'années dans la stratosphère).

\figside{0.6}{0.3}{decouverte/cours_meteo/bilancarbone.png}{Evolution annuelle depuis 1960 des émissions anthropiques de CO2 (combustibles fossiles et déforestation, en rouge), du taux d’accroissement de CO$_2$ atmosphérique (violet), et des puits océanique (bleu) et continentaux (vert). Cette figure montre que les puits de CO$_2$ ne sont pas suffisamment efficaces pour absorber les émissions. En conséquence, le contenu de CO$_2$ dans l’atmosphère augmente sous l'effet des émissions anthropiques. D’apres Le Quere et al. 2009. Source~:~N. Metzl et P. Ciais \emph{in} Le Climat à Découvert, CNRS éditions, 2011.}{fig:bilancarbone} %%% l'océan n'est pas partout un puits de CO2.

\sk
\section{Impacts radiatifs de l'augmentation des gaz à effet de serre}

\sk
Le mécanisme d'effet de serre est crucial pour comprendre le lien entre l'augmentation de la concentration de dioxyde de carbone dans l'atmosphère (figures~\ref{fig:maunaloa} et \ref{fig:giecrad}) et l'augmentation des températures dans la basse atmosphère (figures~\ref{fig:evolutemp20} et~\ref{fig:multitemp}). Il n'est pas suffisant de constater que l'augmentation des deux quantités est corrélée~; corrélation ne vaut pas causalité. Il faut un raisonnement physique pour établir la causalité.

\sk
\subsection{Naturel contre anthropique}

\sk
Rappelons que les gaz à effet de serre possèdent des bandes d'absorption très marquées dans les longueurs d'onde infrarouge du rayonnement électromagnétique; c'est principalement dans ce domaine de longueur d'onde que le rayonnement thermique est émis par la surface de la Terre. Les gaz à effet de serre vont donc absorber ce rayonnement, puis le réémettre vers la surface et vers l'espace. Ainsi, une partie du rayonnement émis par la Terre pour se refroidir ne pourra être évacuée vers l'espace et contribue à augmenter la température de la troposphère terrestre. La présence de gaz à effet de serre dans l'atmosphère terrestre aux concentrations historiques, avant l'activité industrielle humaine, contribue à un \voc{effet de serre naturel} d'environ~$30^{\circ}$C qui rend notre planète Terre habitable. Les gaz à effet concernés sont la vapeur d'eau~H$_2$O, le dioxyde de carbone~CO$_2$, le méthane~CH$_4$, l'oxyde nitreux~N$_2$O, l'ozone~O$_3$ (les nuages induisent également un léger effet de serre). %Le rayonnement émis par la surface terrestre, principalement dans l'infrarouge, est également absorbé par les espèces précitées (vapeur d'eau, CO$_2$, CH$_4$) et réémis à la fois vers l'espace et vers la surface. Ainsi, une partie du rayonnement émis par la surface est \ofg{piégée}, n'est pas évacuée vers l'espace et contribue à augmenter la température de la surface terrestre. Ce phénomène est désigné par le terme d'\voc{effet de serre} et les gaz qui en sont responsables s'appellent les \voc{gaz à effet de serre}.


\sk
Que se passe-t-il si la concentration en CO$_2$, CH$_4$, N$_2$O augmente dans l'atmosphère sous l'effet de l'activité humaine~? A l'effet de serre naturel vient s'ajouter un \voc{effet de serre d'origine anthropique}. Est-il significatif~? Pour répondre à cette question, on raisonne sur le CO$_2$ qui est tenu pour responsable des 2/3 du réchauffement climatique le plus récent. Supposons pour simplifier que la concentration de dioxyde de carbone dans l'atmosphère double par rapport à sa valeur pré-industrielle qui était~$280$~ppm. Cette valeur de~$560$~ppm n'est pas irréaliste~: la valeur en 2010 avoisine les~$390$~ppm~; si aucune action n'est prise pour limiter les émissions actuelles, la concentration en 2100 sera d'environ~$560$~ppm. Voyons quel effet peut avoir cette augmentation sur les flux radiatifs, la structure thermique de l'atmosphère et, finalement, la température de surface pour faire écho aux figures~\ref{fig:evolutemp20} et~\ref{fig:multitemp}.


%La bande d'absorption qui joue un rôle tout à fait central pour l'effet de serre du CO$_2$ est située à une longueur d'onde de~$15$~$\mu$m. Ainsi le rayonnement émis par le système [atmosphère + surface] vers l'espace est principalement déterminé par la réémission dans les longueurs d'onde autour de~$15$~$\mu$m du rayonnement incident de la surface terrestre, absorbée par le CO$_2$ atmosphérique. Si l'on se représente ce phénomène par la présence de plusieurs couches de CO$_2$ à une température donnée, la couche la plus proche de la surface va absorber le rayonnement venant de la surface et le réémettre vers la surface et la couche juste au-dessus; la couche juste au-dessus va recevoir ce rayonnement, l'absorber, puis le réémettre vers la couche encore au-dessus et la couche en-dessous, etc jusqu'au sommet de l'atmosphère. Il faut donc faire le bilan des échanges de rayonnement entre toutes les couches. Néanmoins, la contribution des couches atmosphériques n'est pas la même suivant la longueur d'onde considérée. Par exemple, supposons que l'on soit à une longueur d'onde correspondante à une bande d'absorption très marquée du CO$_2$. Dans les couches atmosphériques proches de la surface, la totalité du rayonnement incident de la surface est absorbée, puis réémise à une température légèrement inférieure. A son tour, cette couche atmosphérique voit son émission absorbée par la couche supérieure. Ainsi il existe pour chaque longueur d'onde une altitude \og équivalente \fg~qui représente la principale contribution (en terme de couche atmosphérique) au rayonnement émis vers l'espace au sommet de l'atmosphère. %dont la température est représentative de l'émission dans la bande d'absorption du CO$_2$ à 15~$\mu$m. 
%Pour illustrer ce concept, situons-nous tout d'abord à une longueur d'onde hors de la bande d'absorption du CO$_2$ à~$15$~$\mu$m. Le rayonnement infrarouge émis par la surface n'est pas absorbé par le CO$_2$ atmosphérique. L'altitude équivalente d'émission de l'atmosphère est donc très proche de la surface. Situons-nous ensuite sur les bords de la bande d'absorption à~$15$~$\mu$m du CO$_2$, où l'absorption est modérée, la contribution à la radiation quittant l'atmosphère se fera surtout depuis les basses couches de l'atmosphère. Situons-nous enfin dans la bande d'absorption à~$15$~$\mu$m du CO$_2$, l'absorption est très forte et donc l'altitude d'émission très élevée. Cette altitude d'émission dépend également de la concentration du composant~: une quantité négligeable de composant très absorbant donne le même type d'altitude d'émission qu'un composant très peu absorbant en grande quantité. Ceci explique que le CO$_2$ qui est très absorbant à~$15$~$\mu$m puisse avoir une grande opacité dans l'infrarouge.

\sk
\subsection{Epaisseur optique et hauteur équivalente d'émission}

\sk
Considérons une espèce~$X$ bien mélangée dans l'atmosphère, qui absorbe dans un intervalle de longueur d'onde donné. A la longueur d'onde~$\lambda$, son \voc{épaisseur optique}~$t_\lambda$ s'écrit
\[ \boxed{ t_\lambda = \int_{0}^{z\e{sommet}} \, k_\lambda \, \rho_X \, \dd z } \]
où $k_\lambda$ est un coefficient d'absorption massique en m$^2$~kg$^{-1}$ et $\rho_X$ est la densité d'absorbant~X. Le nom d'épaisseur optique est assez intuitif. Si un flux de rayonnement~$\Phi_\lambda$ à la longueur d'onde~$\lambda$ est émis à la base de l'atmosphère, le flux observé au sommet de l'atmosphère est d'autant plus réduit qu'à cette longueur d'onde l'épaisseur optique de l'atmosphère traversée est grande\footnote{Si l'extinction est uniquement due à de l'absorption, sans diffusion, on a une relation directe entre l'épaisseur optique et le coefficient d'absorption de la couche~: 
\[\alpha_\lambda = 1 - e^{- \frac{t_\lambda}{\cos\theta}} \] où~$\theta$ est l'angle d'incidence du rayonnement danns la couche.}. La formule ci-dessus ne fait qu'exprimer le fait que la réduction du flux (l'extinction) est plus d'autant plus marquée que 
\begin{citemize}
\item l'espèce considérée est très absorbante dans la longueur d'onde considérée ($k_\lambda$ grand)~;
\item l'espèce considérée est présente en grande quantité ($\rho_X$ grand).
\end{citemize}
Ainsi, le dioxyde de carbone~CO$_2$, bien qu'étant un composant minoritaire ($\rho$ faible), peut atteindre des épaisseurs optiques très grandes dans les intervalles de longueur d'onde où il est très fortement absorbant ($k_\lambda$ élevé), par exemple dans l'infrarouge autour de~$15$~$\mu$m. Autrement dit, un composant minoritaire en quantité peut avoir un rôle majoritaire radiativement.


\sk
Dans l'infrarouge, la situation est un peu plus complexe puisqu'au phénomène d'absorption s'ajoute le phénomène d'émission thermique pour les températures usuellement rencontrées à la surface et dans l'atmosphère terrestre. Si l'on découpe l'atmosphère en couches élémentaires, il convient de faire le bilan de ce que chaque couche reçoit et émet au voisinage d'une longueur d'onde~$\lambda$ dans l'infrarouge où le CO$_2$ est absorbant
\begin{citemize}
\item La surface émet vers l'atmosphère un rayonnement infrarouge à la température~$T_s$
\item La couche atmosphérique~$1$ proche de la surface reçoit ce rayonnement infrarouge de la surface~: une partie de ce rayonnement, d'autant plus grande que l'épaisseur optique de la couche est élevée, est absorbée et la partie restante est transmise à la couche atmosphérique~$2$ située au dessus. A noter que le même type de raisonnement doit être effectué pour le rayonnement reçu de la couche~$2$ au-dessus. La couche atmosphérique~$1$ émet quant à elle un rayonnement infrarouge à la température~$T_1$ déterminée par l'équilibre radiatif entre toutes ces contributions. Ce rayonnement est émis à la fois vers les couches inférieures et vers les couches supérieures.
\item Le raisonnement se poursuit de proche en proche jusqu'à pouvoir déterminer le rayonnement émis au sommet de l'atmosphère vers l'espace.
\end{citemize}

\sk
De ce raisonnement, on peut tirer deux observations
\begin{enumerate}
\item Si l'épaisseur optique de l'atmosphère dans l'infrarouge est nulle, le rayonnement émis au sommet de l'atmosphère est très proche de celui émis par la surface.
\item Si l'épaisseur optique de l'atmosphère dans l'infrarouge est très élevée, le rayonnement émis par la surface est entièrement absorbé par les couches atmosphériques proches de la surface~; le rayonnement émis au sommet de l'atmosphère est déterminé par l'état de l'atmosphère à une hauteur~$h$ intermédiaire entre la surface et le sommet de l'atmosphère\footnote{Cette hauteur reste inférieure au sommet de l'atmosphère car la densité de l'air, donc la densité d'absorbant~$\rho_X$, diminue exponentiellement avec~$z$ et, à partir d'une certaine altitude, devient si faible que~$k_\lambda \, \rho_X$ est négligeable même pour de grandes valeurs de~$k_\lambda$.}. 
\end{enumerate}
On peut en fait montrer que, pour des variations modérées de température atmosphérique, la contribution maximale au rayonnement à une longueur d'onde~$\lambda$ sortant au sommet de l'atmosphère provient d'une hauteur~$h$ telle que l'épaisseur optique soit~$t_\lambda = 1$. Cette hauteur~$h$ est dénommée \voc{hauteur équivalente d'émission}. Pour déterminer~$h$, il suffit donc de partir du sommet de l'atmosphère, où par définition~$t_\lambda = 0$, de descendre dans l'atmosphère et de déterminer quand la valeur de l'épaisseur optique~$t_\lambda$ atteint~$1$.
\begin{citemize}
\item Plus l'espèce est absorbante ($k_\lambda$ grand), moins bas on doit descendre pour atteindre~$t_\lambda = 1$.
\item Plus la concentration d'espèce absorbante est grande ($\rho_X$ grand), moins bas on doit descendre pour atteindre~$t_\lambda = 1$.
\end{citemize}

\sk
\subsection{Forçage radiatif induit par la variation de concentration en CO$_2$}

\sk
Ainsi, pour une espèce comme le CO$_2$ qui est très absorbante à une longueur d'onde de~$15$~$\mu$m, la hauteur équivalente d'émission~$h$ est assez élevée et se situe entre la haute troposphère et la stratosphère\footnote{Suivant que le centre de la bande d'absorption à~$15$~$\mu$m ou les ailes de cette bande sont considérés.}. Pour cette même espèce très absorbante dans l'infrarouge, un doublement de concentration va induire une augmentation de la hauteur équivalente d'émission de~$h$ à~$h'$. On peut montrer que cette augmentation est d'environ~$h'-h \simeq 3$~km. 

\sk
Or, nous avons vu dans les chapitres précédents que l'équilibre entre radiation (notamment l'effet de serre naturel) et convection (i.e. dynamique atmosphérique selon la verticale) impose que la température décroît selon l'altitude d'en moyenne~$6^{\circ}$C~par kilomètre d'ascension. La température d'émission à une hauteur~$h'$ est donc beaucoup plus faible qu'à une hauteur~$h$~: le rayonnement infrarouge sortant au sommet de l'atmosphère est donc également plus faible. La baisse de flux induit par ces changements dans des longueurs d'onde voisines de~$15$~$\mu$m est d'environ~$3$~W~m$^{-2}$ avec le raisonnement simplifié présenté ici. Des estimations beaucoup plus précises conduisent à~$4$~W~m$^{-2}$ en prenant en compte, dans un raisonnement plus complet que celui présenté ici, la structure détaillée de la bande d'absorption du dioxyde de carbone, les légers recouvrements avec les bandes d'absorption de la vapeur d'eau, la structure verticale de l'atmosphère et la couverture nuageuse. Ce changement dans la valeur du rayonnement émis au sommet de la troposphère est appelé~\voc{forçage radiatif}. 

\sk
Il est alors aisé de comprendre que l'augmentation de la quantité de CO$_2$ dans l'atmosphère conduit à une augmentation de la température de la surface. Au sommet de l'atmosphère (ou, de façon équivalente ici, au sommet de la troposphère), il y a un équilibre entre le flux solaire incident (longueurs d'onde visible) et le flux infrarouge sortant (longueurs d'onde infrarouge). Nous venons de voir qu'un doublement du CO$_2$ dans l'atmosphère conduit à une diminution du flux infrarouge sortant (émis vers l'espace). Ainsi il y a un excédent d'énergie qui conduit à l'augmentation des températures dans la troposphère et en particulier de la température de surface, afin que l'équilibre soit rétabli. Le calcul montre que l'augmentation correspondante est de~$1.2^{\circ}$C. 

\sk
L'effet des différents gaz à effet de serre (CO$2$, CH$_4$, N$_2$O, O$_3$) peut être comparé, entre eux et avec les autres sources de variabilité, via le forçage radiatif qu'ils induisent (figure~\ref{fig:forcrad}). On peut noter l'influence non négligeable du méthane qui est pourtant en quantité plus faible que le CO$_2$. Il faut d'ailleurs se garder d'additionner directement les différents forçages indiqués sur la figure~\ref{fig:forcrad}. Seule une combinaison prenant en compte les échelles d'impact spatiale (du local au global) et temporelle (de l'intermittence à la permanence) le permet, telle celle indiquée sur la figure~\ref{fig:forcrad}. Notamment, le forçage par le dioxyde de carbone ou celui induit par les changements de constante solaire sont globaux, alors que les forçages par les aérosols et les nuages, ou par l'ozone troposphérique, sont plus locaux. Actuellement, le forçage net d'origine anthropique est de~$1.6$~W~m$^{-2}$.

\figside{0.63}{0.33}{decouverte/cours_dyn/forcage_radiatif_IPCC.png}{Estimation des forçages radiatifs anthropiques ou naturels depuis le début de l’ère industrielle. Chaque rectangle représente l’estimation moyenne du forçage et les traits noirs indiquent l’incertitude associée. Les forçages positifs (en rouge) correspondent à un réchauffement du climat et ceux négatifs (en bleu) un refroidissement. La dernière colonne (Level Of Scientific Understanding) donne le niveau actuel de compréhension scientifique du forçage. Source~:~IPCC Fourth Assessment Report: Climate Change 2007 (AR4). Voir également P. Dubuisson \emph{in} Le Climat à Découvert, CNRS éditions, 2011}{fig:forcrad}

\sk
\subsection{Quelques subtilités supplémentaires}

\sk
La valeur de variation de température calculée précédemment est en fait sous-estimée. Elle suppose que l'augmentation de température causée par l'effet de serre anthropique ne s'accompagne pas de changement dans la structure thermique de l'atmosphère, dans la couverture nuageuse ou dans la quantité de vapeur d'eau dans l'atmosphère. Ces changements peuvent induire des \voc{rétroactions}, c'est-à-dire qu'ils ont tendance soit à renforcer la perturbation initiale qui leur a donné naissance (ici l'augmentation de température), dans ce cas il s'agit de \voc{rétroaction positive}, soit au contraire à la contrecarrer s'il s'agit de \voc{rétroaction négative}. Les quatre rétro-actions principales à considérer sont les suivantes~:
\begin{finger}
\item L'augmentation de la température favorise l'évaporation, donc la quantité de vapeur d'eau dans l'atmosphère augmente; la vapeur d'eau étant un gaz à effet de serre très efficace, la rétroaction est fortement positive.
\item Le changement de contenu en vapeur d'eau modifie la structure en température de l'atmosphère (le gradient adiabatique humide); l'effet moyen de cette rétroaction sur la température a tendance à être négative.
\item Si de la glace (albédo élevé) disparaît des surfaces continentales et des océans (albédo faible), la diminution résultante de l'albédo provoque un réchauffement de la surface; il s'agit donc d'une rétroaction positive.
\item Les nuages ont deux effets contraires sur la température. D'une part, ils réfléchissent le rayonnement solaire dans le domaine visible, donc réduisent la quantité de rayonnement solaire qui parvient à la surface. D'autre part, ils absorbent le rayonnement infrarouge émis par la surface de la Terre, donc réduisent la quantité de rayonnement infrarouge perdue vers l'espace. La rétroaction des nuages est la plus complexe des rétroactions (en témoigne la barre d'incertitude sur la figure~\ref{fig:forcrad}) et fait l'objet de recherches actives.
\end{finger}
S'ajoutent à ces rétroactions, les changements provoqués dans la circulation atmosphérique et océanique, également appelés rétroactions dynamiques, ainsi que les rétroactions biogéochimiques, qui font notamment intervenir l'évolution de la végétation. 

\sk
Ces rétroactions donnent tout le \og sel \fg~scientifique à la détermination précise des variations de température dans une situation d'augmentation anthropique des gaz à effet de serre. Les estimations actuelles s'accordent sur un réchauffement de la surface de~$2.5^{\circ}$C pour un doublement de la quantité de CO$_2$, mais l'intervalle des valeurs obtenues par les différentes équipes à l'international travaillant sur le sujet est~$1.5-4.5^{\circ}$C. Ainsi, si la communauté scientifique est unanime sur le réchauffement futur du climat et de la cause principale (l'augmentation des gaz à effet de serre), il reste à préciser son amplitude. Néanmoins, quel que soit le résultat exact, un réchauffement au moins supérieur ou égal à~$1^{\circ}$C est considérable~:
\begin{citemize}
\item Il est à comparer à la différence de $5-6^{\circ}$C entre les âges glaciaires et interglaciaires dans le passé.
\item Ce changement dans la température moyenne globale devrait se réaliser sur des échelles de temps de l'ordre d'un siècle, ce qui est significativement plus rapide que les changements climatiques survenus par le passé.
\end{citemize}

%\sk Le forçage radiatif causé par l'augmentation des aérosols à cause des sources anthropogéniques est principalement négatif, c'est-à-dire tend à faire baisser le température de la troposphère. Noter la grande incertitude. Une manière dont les particules dans l'atmosphère peuvent influencer le forçage radiatif apparaît via leur effect sur la formation de nuages (puisqu'ils forment, nous l'avons vu dans un chapitre précédent, des noyaux de condensation). Il s'agit d'un forçage radiatif indirect. Si les particules sont présentes en grand nombre lorsque les nuages se forment, le nuage résultant contient un grand nombre de petites gouttes, plus petites que si moins de noyaux de condensation sont disponibles. Le pouvoir réfléchissant (albédo) pour le rayonnement solaire incident d'un tel nuage est plus élevé que pour un nuage contenant une quantité plus petite de particules plus grosses. Cet effet d'albedo augmente la perte d'énergie.

%On doit par contre considérer en plus de l'absorption de rayonnement, l'émission dans l'infrarouge par l'atmosphère (figure \ref{fig:schwartzschild}). Dans les conditions appelées \emph{équilibre thermodynamique local\footnote{Ces conditions sont vérifiées si les collisions entre molécules sont plus fréquentes que l'absorption ou émission de rayonnement. Les molécules émettrices ont alors la même température que leur environnement}}, qui sont valables jusque vers 60~km d'altitude environ, le rayonnement émis (vers le haut et vers le bas) par une couche mince d'atmosphère dépend de sa température et de son coefficient d'absorption suivant la loi de Kirchoff. Pour un faisceau lumineux traversant une couche mince d'atmosphère, la variation de luminance vaut alors: \[dL_\lambda=\left(-L_\lambda+B_\lambda(T)\right)\mu d\tau_\lambda\] Le premier terme du second membre représente l'absorption du rayonnement incident, le deuxième l'émission par les gaz de la couche. La loi de Kirchoff fait qu'ils sont multipliés par le même coefficient $\mu\tau_\lambda$ qui donne le coefficient d'absorption et d'émission de la couche. Cette équation est appelée \emph{équation de Schwartzschild}. Son intégration entre une altitude $z_0$ et l'infini (espace) donne: \[L_\lambda(\infty)=L_\lambda(z_0)e^{-\mu\tau_\lambda(z_0,\infty)}+\int_{z_0}^\infty B_\lambda(T)e^{-\mu\tau_\lambda(z,\infty)}\mu\rho_ak_\lambda dz\] Le rayonnement sortant qu sommet de l'atmosphère est donc la somme du rayonnement présent en $z_0$ diminué de l'absorption entre $z_0$ et le sommet de l'atmosphère (premier terme), et de l'intégrale de la contribution du rayonnement émis par chaque couche au dessus de $z_0$. Comme pour le transfert dans le visible, on peut montrer (en supposant que $T$ varie peu) que la contribution maximale au rayonnement sortant à une longueur d'onde $\lambda$ provient d'une épaisseur optique de $\tau_\lambda=1$ à partir du sommet de l'atmosphère.
%De façon plus générale, on a vu que le rayonnement sortant provenait majoritairement de la région de l'atmosphère autour d'une épaisseur optique de 1 à partir du sommet. Cette région dépend de la longueur d'onde: proche de la surface dans la fenêtre transparente, dans la haute troposphère dans les bandes d'absorption du CO$_2$, autour de 2~km dans celles de la vapeur d'eau. Comme la température décroit à partir de la surface, le rayonnement sortant est donc émis à des températures inférieures à $T_s$, et on peut écrire qu'il vaut \[IR_{sommet}=\sigma T_s^4 (1-\epsilon)=\sigma T_{eq}^4\] Où $\epsilon>0$ représente l'effet de serre. La valeur de $\epsilon$ augmente quand la température d'émission vers l'espace diminue par rapport à celle de surface, typiquement parce que l'altitude d'émission augmente. 
%%Émission nette par la vapeur d'eau, l'ozone, le CO2 et les autres gaz à effet de serre : Il s'agit du flux énergétique net émis sous forme de rayonnement énergétique (infrarouge) par l'ensemble des molécules de l'atmosphère. L'émission infrarouge est associée à un refroidissement local. Comme le Corps Noir, les molécules émettent un rayonnement pour se refroidir et équilibrer l'énergie absorbée. L'émission n'a lieu que dans les bandes d'absorption (ou d'émission). Il faut donc que la température locale soit celle du Corps Noir émettant à la longueur d'onde de la bande d'émission. Ainsi, plus on descend dans l'atmosphère plus l'émission se fera par les bandes centrées sur de faibles longueurs d'onde. Émission IR et refroidissement atmosphériques sont doncintimement liés. La stratosphère est principalement refroidie par l'émission IR du gaz carbonique. Ce refroidissement est associé à l'émission par la bande située à 15 μm. Dans la haute stratosphère, la bande d'émission de l'ozone à 9.6 μm permet l’émission IR et le refroidissement atmosphérique. Cependant l'ozone absorbe principalement les radiations solaires et ne peut être considérée comme un gaz à effet de serre (dans la stratosphère). La vapeur d'eau émet également dans la stratosphère dans la bande à 8 μm. La troposphère est principalement refroidie par l'émission de la vapeur d'eau dans la bande située à 6.3 micromètres.

%rétroactions entre température atmosphérique et concentration de CO2 : le CO2 influence la température atmosphérique via l'effet de serre, mais la concentration de CO2 est elle-même influencée par des factors qui dépendent de la température atmosphérique. 

\figun{0.7}{0.35}{decouverte/cours_meteo/pred_clim.png}{Projections de la variation de la température globale de 1950 à 2100 selon trois scénarios (couleurs) et à partir de 14 modèles (spaghettis). La moyenne multimodèle pour chaque scénario est en trait plein de la couleur correspondante, et les observations (sur environ un demi-siècle) sont représentées en noir. Le scénario B1 suppose des émissions quasi-constantes à partir de 2000. Le scénario A1B suppose des émissions qui doublent d'ici 2050. Le scénario A2 suppose que les émissions sont multipliées par~4 d'ici 2100. Source~:~IPCC Fourth Assessment Report: Climate Change 2007 (AR4).}{fig:predclim} 

\sk
\section{Prédiction du changement climatique et impacts}

\sk
Les modèles climatiques couplent les divers compartiments indiqués sur la figure~\ref{fig:pluri} pour livrer une analyse la plus complète possible de l'intrication des différents phénomènes qui contribuent au changement récent du climat terrestre, l'augmentation des gaz à effet de serre en tête. Les premières prédictions de modèles couplés dans les années 80 pour les décennies suivantes se sont révélées conformes aux observations entre 1990 et 2010; entre temps, les modèles se sont améliorés en incluant des interactions plus complètes entre les différents compartiments du système climatique, qui ont permis de préciser les premiers diagnostics [Figure~\ref{fig:gcm}].
% Les résultats de simulations présentés dans ce chapitre proviennent de plusieurs modèles indépendants, validé pour le climat courant, autant en terme de quantités moyennes que des variabilité; ils ont été également validés pour la survenue d'anomalies climatiques ponctuelles, telles celles induites par des éruptions volcaniques ou des événements El Ni\~ no; et pour des cas test paléo-climatiques. 
Par ailleurs, ces modèles sont désormais un outil utile pour effectuer des projections afin de guider les politiques sur les actions à mener (figure~\ref{fig:predclim}). Ainsi, plusieurs scénarios d'augmentation des gaz à effet de serre (principalement CO$_2$) sont considérés, correspondant à des scénarios de projection géopolitique et économique. Les modèles du système climatique contribuent à prédire et comprendre quelle serait l'évolution de la température en fonction du scénario considéré. Ils peuvent également servir à comprendre les conséquences du changement climatique sur les précipitations (figure~\ref{fig:precip}) et sur la survenue d'événements extrêmes (canicules, précipitations intenses, sécheresses, \ldots). Sur ces points, suivant les régions considérées, les modèles ne s'accordent pas toujours. Les efforts de recherche entrepris doivent se poursuivre. Néanmoins, de grandes tendances commencent à se dégager, notamment sur la survenue plus fréquente d'épisodes caniculaires plus chauds.

\figside{0.6}{0.4}{decouverte/cours_meteo/giec2007_gcm.jpg}{Evolution des phénomènes et éléments du système climatique intégrés dans les modèles numériques de climat. FAR correspond à 1990. SAR correspond à 1995. TAR correspond à 2001. AR4 correspond à 2007. Figure tirée du 4ème rapport du GIEC en 2007.}{fig:gcm}

\figun{0.95}{0.25}{decouverte/cours_meteo/figure-spm-7-l.png}{Changements relatifs des précipitations (en pourcentages) pour la période 2090-2099 par rapport à 1980–1999. Les valeurs sont issues de moyennes sur de nombreux modèles basés sur le scénario A1B du RSSE pour les mois de décembre à février (à gauche) et de juin à août (à droite). Les aires blanches représentent les zones où moins de 66\% des modèles concordent sur le signe du changement, les zones hachurées correspondent aux cas où plus de 90\% des modèles concordent sur le signe du changement. Source~:~IPCC AR4 2007.}{fig:precip}

\sk
L'obtention de données reste de la première importance, à la fois pour vérifier que le comportement des modèles est proche du comportement observé, et pour étudier les impacts du réchauffement climatique récent. Un projet particulièrement important a été récemment mené pour mesurer les variations du niveau des océans, dont on peut prédire l'augmentation à cause de la hausse globale de température par dilatation et par fonte des glaciers continentaux (figure~\ref{fig:sealevel}). 

\sk
D'autres impacts que ceux précités sont également à l'étude
\begin{citemize}
\item Impacts sur les courants marins et donc sur les climats régionaux
\item Acidification des océans
\item Impacts sur les écosystèmes (affaiblissements, disparitions, déplacements)
\item Impacts directs sur la santé humaine (déplacement des zones endémiques pour les maladies, \ldots)
\item Risques géopolitiques (lié notamment au stress hydrique) 
\item Impacts économiques
\end{citemize}
%La liste peut encore s'allonger dans la mesure où l'action de l'homme provoque sur le climat des changements inédits, dont les conséquences sont tout autant imprévisibles. 
Les démarches que l'homme doit entreprendre pour comprendre et limiter son empreinte climatique est probablement un des enjeux majeurs du siècle qui débute.

\figun{0.7}{0.3}{decouverte/cours_meteo/cazenave_level.png}{Courbe d’évolution du niveau moyen global de la mer mesuré par altimétrie par 3 groupes différents. Droits de copyright de la figure~:~CLS, CNES, LEGOS. Source~:~A. Cazenave et M. Ablain \emph{in} Le Climat à Découvert, CNRS éditions, 2011}{fig:sealevel}

%%Extrait de la traduction du rapport IPCC AR4
%\begin{finger}
%\item A1. Le canevas et la famille de scénarios A1 décrivent un monde futur dans lequel la croissance économique sera très rapide, la population mondiale atteindra un maximum au milieu du siècle pour décliner ensuite et de nouvelles technologies plus efficaces seront introduites rapidement. Les principaux thèmes sous-jacents sont la convergence entre régions, le renforcement des capacités et des interactions culturelles et sociales accrues, avec une réduction substantielle des différences régionales dans le revenu par habitant. La famille de scénarios A1 se scinde en trois groupes qui décrivent des directions possibles de l’évolution technologique dans le système énergétique. Les trois groupes A1 se distinguent par leur accent technologique: forte intensité de combustibles fossiles (A1FI), sources d’énergie autres que fossiles (A1T) et équilibre entre les sources (A1B) (« équilibre » signifiant que l’on ne s’appuie pas excessivement sur une source d’énergie particulière, en supposant que des taux d’amélioration similaires s’appliquent à toutes les technologies de l’approvisionnement énergétique et des utilisations finales).
%\item A2. Le canevas et la famille de scénarios A2 décrivent un monde très hétérogène. Le thème sous-jacent est l’autosuffisance et la préservation des identités locales. Les schémas de fécondité entre régions convergent très lentement, avec pour résultat un accroissement continu de la population mondiale. Le développement économique a une orientation principalement régionale, et la croissance économique par habitant et l’évolution technologique sont plus fragmentées et plus lentes que dans les autres canevas.
%\item B1. Le canevas et la famille de scénarios B1 décrivent un monde convergent avec la même population mondiale culminant au milieu du siècle et déclinant ensuite, comme dans le canevas A1, mais avec des changements rapides dans les structures économiques vers une économie de services et d’information, avec des réductions dans l’intensité des matériaux et l’introduction de technologies propres et utilisant les ressources de manière efficiente. L’accent est placé sur des solutions mondiales orientées vers une viabilité économique, sociale et environnementale, y compris une meilleure équité, mais sans initiatives supplémentaires pour gérer le climat.
%\item B2. Le canevas et la famille de scénarios B2 décrivent un monde où l’accent est placé sur des solutions locales dans le sens de la viabilité économique, sociale et environnementale. La population mondiale s’accroît de manière continue mais à un rythme plus faible que dans A2, il y a des niveaux intermédiaires de développement économique et l’évolution technologique est moins rapide et plus diverse que dans les canevas et les familles de scénarios B1 et A1. Les scénarios sont également orientés vers la protection de l’environnement et l’équité sociale, mais ils sont axés sur des niveaux locaux et régionaux.
%\end{finger}
