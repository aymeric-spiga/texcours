\documentclass[
	a4paper,
	DIV16,
	10pt,
	cleardoublepage=empty,
        twoside=yes,
        BCOR=8.25mm
	]{scrbook}

%%% MOI
\usepackage{scrpage2}
\usepackage{my_latex}
\usepackage{patch_french}
\setcapindent{1em} %%% parce que sinon l'indentation "Figure ... -- texte" est trop grande

%%% BANDEAU
\deftripstyle{Vide}[0pt][.4pt]
{}{}{}
{}{}{}
\deftripstyle{Bandeau}[0pt][.4pt]
{LP211 - année 2013/2014}{Sciences de l'atmosphère}{Universit\'e Pierre et Marie Curie}
{Notes préparées par Aymeric Spiga (UPMC/LMD)}{}{\pagemark} \pagestyle{scrheadings} \pagestyle{Bandeau}

%%% DEBUT
\begin{document}
\frontmatter

%%% PAGE DE GARDE
\subject{LP211}
\title{Introduction aux sciences de l'atmosphère (Partie 2)}
\subtitle{Cours niveau licence. Année universitaire 2013-2014.}
\author{~\\ ~\\
Aymeric SPIGA\\
\small Maître de conférences à l'Université Pierre et Marie Curie\\
\small Chercheur au Laboratoire de Météorologie Dynamique (Institut Pierre-Simon Laplace)
}
\date{}
\publishers{
\includegraphics[height=2cm]{/home/aymeric/Images/Logo/UPMC_cart-blanc-Q_7504-703-3.png}\\
Contact: \email{aymeric.spiga@upmc.fr}\\Copie et usage interdits sans autorisation explicite de l'auteur.
}

%%%%% decommenter pour le mode book
\dedication{Les chapitres 1, 2, 4 consistent en de multiples réorganisations, ajouts et modifications sur des notes existantes de Francis Codron que je souhaite remercier pour son aide. Les chapitres 3 et 5 sont entièrement originaux. Je remercie Jean-Baptiste Madeleine pour ses remarques constructives.\\ ~\\ ~\\ Dans l'éventualité où le lecteur trouverait des erreurs ou imprécisions dans ce cours, il est cordialement invité à les signaler à l'auteur à l'adresse \url{aymeric.spiga@upmc.fr}} \maketitle \tableofcontents

%%%%%%%%%%%%%%%%%%%%%%%%%%%%%%%%%%%%%%%%%%%%%%%%%%%%%%%%%%%%%%%
\mainmatter
%%%%%%%%%%%%%%%%%%%%%%%%%%%%%%%%%%%%%%%%%%%%%%%%%%%%%%%%%%%%%%%

%%% CHAPITRE 1
%\def\thechapter{1} \title{Sciences de l'atmosphère\\ Chapitre \thechapter} \author{Aymeric SPIGA\\ ~\\ \small Modifié et augmenté à partir des notes de Francis CODRON et Sébastien PAYAN.} \maketitle  %% commenter la ligne en mode livre complet
%\chapter{Un survol de l'atmosphère} \label{chap:int}

\dictum[Pierre-Simon de Laplace, 1797]{Un fluide rare, transparent, compressible et élastique, qui environne un corps, en appuyant sur lui, est ce que l'on nomme son atmosphère.}

	\bk
	\sk
La figure~\ref{fig:blue} illustre la présence d'une atmosphère très active sur Terre par les nuages qui y prennent naissance. Il ne s'agit que d'un exemple parmi tant d'autres pour appréhender l'atmosphère. C'est le but de ce chapitre d'introduction d'évoquer la diversité des points de vue pouvant être adoptés pour étudier l'atmosphère, un système complexe où se mêlent processus physiques, dynamiques, chimiques, biologiques, et même sociétaux. Sont également abordées dans ce chapitre quelques notions de base nécessaires pour la suite du cours.

\figsup{0.45}{0.25}{decouverte/cours_meteo/blue_50.png}{decouverte/cours_meteo/blueclouds_50.png}{La planète Terre avec et sans les nuages de son atmosphère. Les nuages couvrent très souvent au moins la moitié du globe. Construit d'après une image \ofg{Blue Marble} NASA du projet \ofg{Visible Earth}. Des versions haute-résolution des images planes et des explications complètes peuvent être trouvées aux adresses suivantes \url{http://visibleearth.nasa.gov/view_rec.php?id=2430} et \url{http://visibleearth.nasa.gov/view_rec.php?id=2431}.}{fig:blue} %%%http://visibleearth.nasa.gov/view_rec.php?id=2429



\mk \section{Quelques définitions et généralités}

	\sk \subsection{Vocabulaire}
	\sk
L'objectif des sciences de l'atmosphère est d'étudier la structure et l'évolution de l'atmosphère en caractérisant et en expliquant les phénomènes qui s'y déroulent. Les sciences de l'atmosphère font principalement appel à des notions de physique, chimie, et mécanique des fluides.
\begin{description}
\item[\voc{Atmosphère}] Ensemble de couches, principalement gazeuses, qui entourent la masse condensée, solide ou liquide, d'une planète (voir également citation de Laplace en en-tête).
\item[\voc{Air}] Mélange gazeux constituant l'atmosphère terrestre.
\item[\voc{Aéronomie}] Science dont l'objet est la connaissance de l'état physique de l'atmosphère terrestre et des lois qui la gouvernent.
\item[\voc{Météorologie}] Discipline ayant pour objet l'étude des phénomènes atmosphériques et de leurs variations, et qui a pour objectif de prévoir à court terme les variations du temps.
\item[\voc{Climat}] Ensemble des conditions atmosphériques et météorologiques d'un pays, d'une région. Le climat peut également être défini comme un système thermo-hydrodynamique non isolé dont les composantes sont les principales « enveloppes » externes de la Terre : on parle également de~\voc{système climatique} [figure~\ref{fig:pluri}]. 
\begin{citemize} \item L'atmosphère : l’air, les nuages, les aérosols, \ldots 
\item L’hydrosphère : les océans, les rivières, les précipitations, \ldots 
\item La lithosphère : les terres immergées, les sols, \ldots 
\item La cryosphère : glace, neige, banquise, glaciers, \ldots 
\item La biosphère : les organismes vivants, \ldots
\item L’anthroposphère : l’activité humaine, \ldots 
\end{citemize} 
\end{description}

\figside{0.75}{0.35}{decouverte/cours_meteo/joussaume_pluri.png}{Schéma du système climatique présentant les différentes composantes du système : atmosphère, océans, cryosphère, biosphère et lithosphère, leurs constantes de temps et leurs interactions en termes d’échanges d’énergie, d’eau et de carbone. Source~:~S.~Joussaume \emph{in} Le Climat à Découvert, CNRS éditions, 2011}{fig:pluri}


	\sk \subsection{Grandeurs utiles}
	\sk
L'atmosphère est composée d'un ensemble de molécules. Pour la description de la plupart des phénomènes étudiés, le suivi des comportements individuels de chacunes des molécules composant l'atmosphère est impossible. On s'intéresse donc aux effets de comportement d'ensemble, ou moyen. Les principales variables thermodynamiques utilisées pour décrire l'atmosphère sont donc des grandeurs \voc{intensives} dont la valeur ne dépend pas du volume d'air considéré.
\begin{finger}
\item La \voc{température} $T$ est exprimée en K (kelvin) dans le système international. Elle est un paramètre macroscopique qui représente l'agitation thermique des molécules microscopiques. Les mesures de température usuelles font parfois référence à des quantités en \deg C, auxquelles il faut ajouter la valeur $273.15$ pour convertir en kelvins.
\item La \voc{pression} $P$ est exprimée en Pa dans le système international. La pression fait référence à une force par unité de surface ($1$~Pa correspond à l'unité~N~m$^{-2}$). Paramètre macroscopique, elle est reliée à la quantité de mouvement des molécules microscopiques qui subissent des chocs sur une surface donnée. Les mesures et raisonnements météorologiques font souvent référence à des quantités en hPa ou en mbar. Ces deux unités sont équivalentes : 1~hPa correspond à~$10^{2}$~Pa, 1~mbar correspond à $10^{-3}$~bar, ce qui correspond bien à 1~hPa, puisque 1 bar est $10^{5}$~Pa. La pression atmosphérique vaut $1013.25$~hPa (ou mbar) en moyenne au niveau de la mer sur Terre. On utilise parfois l'unité d'$1$~atm (atmosphère) qui correspond à cette valeur de~$101325$~Pa.
\item La \voc{masse volumique} ou densité~$\rho$ est exprimée en~kg~m$^{-3}$ dans le système international. Elle représente une quantité de matière par unité de volume. Elle vaut environ $1.217$~kg~m$^{-3}$ proche de la surface sur Terre.
\end{finger}

	\sk
Les trois paramètres pression, température, densité varient en théorie selon les trois directions de l'espace. On constate cependant que, pour une unité de longueur donnée, leurs variations selon la verticale sont beaucoup plus significatives que leurs variations selon l'horizontale. On peut donc définir une structure moyenne en fonction de l'altitude dont sera toujours relativement proche la structure verticale en chaque jour et chaque région de la planète. 

\figsup{0.48}{0.35}{\figfrancis/WH_vert_struct}{\figpayan/LP211_Chap1_Page_03_Image_0001.png}{[Gauche] Structure verticale de la pression, la densité et du libre parcours moyen des molécules (distance moyenne parcourue par une molécule avant de subir un choc sur une autre molécule). Noter l'échelle logarithmique en abscisse : une droite sur ce schéma dénote donc une variation exponentielle des quantités avec l'altitude. [Droite] Plus haut dans l'atmosphère, la variation verticale de la pression est dépendante pour plusieurs ordres de grandeur avec l'activité solaire. Les courbes indiquées correspondent respectivement à une très faible activité solaire (température de la thermopause de 600 K) et une très forte activité solaire (température de la thermopause de 2000K).}{fig:presvert}

\sk
Pression et densité décroissent de façon approximativement exponentielle selon l'altitude~$z$ [figure \ref{fig:presvert}] $$ P\sim P_0 \, e^{-z/H} $$ où $H$ est appelée \voc{hauteur d'échelle} et~$P_0$ une valeur de pression de référence. Cette loi de variation découle du fait que la pression atmosphérique à une altitude~$z$ est due au poids de la colonne d'air située au-dessus de l'altitude~$z$. En pratique sur Terre, la pression est divisée par deux environ tous les 5 km. On évalue la masse de l'atmosphère terrestre à~$5 \times 10^{18}$~kg, soit environ un millionième de la masse de la Terre. La moitié de la masse de l'atmosphère se situe au dessous de~$5500$~m, les~$2/3$ au dessous de~$8400$~m, les~$3/4$ au dessous de~$10300$~m, les~9/10~au dessous de~$16100$~m. Si l'on considère que les neuf dixièmes de l’atmosphère sont situés dans les $16$~premiers kilomètres, l’atmosphère ne forme donc qu'une mince pellicule gazeuse en comparaison des $6367$~km du rayon terrestre. On dit que l'on peut faire l'\voc{approximation de la couche mince}.


	\sk \subsection{Structure verticale : couches atmosphériques}
	\sk
Les variations verticales de température sont très différentes des variations de pression et de densité: la température décroît et augmente alternativement avec l'altitude%, de façon quasi-linéaire [figure \ref{fig:tempvert}], en restant comprise entre environ~$200$ et~$300$~K. 
Cette structure verticale de la température permet de diviser l'atmosphère en un certain nombre de couches aux propriétés différentes, dont les noms comportent le suffixe \emph{-sphère}. La limite supérieure d'une couche atmosphérique donnée porte un nom similaire, où le suffixe \emph{-sphère} est remplacé par le suffixe \emph{-pause}. Par exemple, la limite entre la troposphère et la stratosphère s'appelle la \voc{tropopause}. Les couches atmosphériques en partant de la surface vers l'espace sont détaillées ci-dessous.

\sk
\begin{description} 
\item[La \voc{troposphère}] \normalsize s'étend jusqu'à environ 11 km d'altitude et contient 80\% de la masse de l'atmosphère. La température y décroit en moyenne de 6.5\deg C par kilomètre (nous verrons pourquoi dans un chapitre ultérieur). La troposphère est une couche relativement bien mélangée sur la verticale (échelle de temps de quelques jours), sauf en certaines couches minces, appelées \voc{inversions}, où la température décroit peu ou même augmente avec l'altitude. La troposphère est la couche où ont lieu la plupart des phénomènes météorologiques acessibles à l'expérience humaine (par exemple, les nuages montrés en figure~\ref{fig:blue}). La partie inférieure de la troposphère contient la \voc{couche limite atmosphérique} située juste au dessus de la surface, d'épaisseur variant de quelques centaines de~m à 3 km et définie comme la partie de l'atmosphère influencée par la surface sur de courtes échelles de temps (typiquement un cycle diurne). La couche limite atmosphérique est le siège de mouvements turbulents organisés au cours de l'après-midi qui opèrent un mélange des espèces chimiques qui y sont émises. \normalsize
\item[La \voc{stratosphère}] \normalsize est située au dessus de la troposphère. L'altitude au-dessus du sol de la tropopause peut varier entre~$5$ et~$15$~km. Contrairement à la troposphère, la stratosphère contient très peu de vapeur d'eau (à cause des températures très basses rencontrées à la tropopause) mais la majorité de l'ozone~O$_3$. L'absorption par l'ozone du rayonnement solaire \voc{ultraviolet}, de longueur d'onde moindre que le rayonnement visible et plus énergétique, explique que la température dans la stratosphère est d'abord isotherme, puis augmente avec l'altitude jusqu'à un maximum à la stratopause. Cette structure verticale très stable inhibe fortement les mouvements verticaux, ce qui explique que la stratosphère soit organisée en couches horizontales (comme l'indique l'étymologie de son nom). Le temps de résidence de particules dans la stratosphère est très long à cause de l'absence de nuages et précipitations. \normalsize
\item[La \voc{mésosphère}] \normalsize voit sa température décroître selon la verticale. Contrairement à la troposphère, elle ne contient pas de vapeur d'eau et contrairement à la stratosphère, elle ne contient que peu d'ozone. Elle se situe sur Terre à des altitudes entre~$50$ et~$85$~km. La mésopause est souvent le point le plus froid de l'atmosphère terrestre, la température peut y atteindre des valeurs aussi basses que~$130$~K. \normalsize
\end{description}

\figside{0.45}{0.3}{\figfrancis/WH_stdatm}{Structure verticale idéalisée de la température correspondant au profil moyenné global annuel.}{fig:tempvert}

	\sk
\begin{description} 
\item[La \voc{thermosphère}] s'étend jusque des altitudes très élevées (800 km) et voit sa température contrôlée par l'absorption du rayonnement solaire ultraviolet. La température dans la thermosphère varie souvent d'un facteur deux suivant l'activité solaire et l'alternance jour-nuit. Les aurores surviennent dans cette couche atmosphérique. Les missions spatiales \ofg{basse orbite} telles que la Station Spatiale Internationale sont localisées au milieu de la thermosphère. \normalsize
\item[L'\voc{exosphère}] est située au-dessus de la thermosphère à partir d'une altitude d'environ~$800$~km sur Terre. Il s'agit de la zone où l'atmosphère subit un \voc{échappement} : les molécules peuvent s'échapper vers l'espace sans que des chocs avec d'autres molécules ne les renvoient dans l'atmosphère. L'exosphère constitue la dernière zone de transition entre l'atmosphère et l'espace. \normalsize
\end{description}

\figside{0.45}{0.3}{\figpayan/LP211_Chap1_Page_05_Image_0001.png}{Structure verticale idéalisée de la température étendue aux hautes atmosphères.}{fig:tempvert}
%% Voir figure~\ref{fig:presvert} pour la distinction entre figure de gauche et figure de droite

\sk
D'autres couches atmosphériques sont définies non pas à partir de la température mais à partir des propriétés électriques de l'atmosphère terrestre. On fait référence ici au vent solaire, qui est un flux de particules chargées (ions et électrons) formant un plasma qui s’échappe en permanence du Soleil vers l’espace interplanétaire
\begin{description}
\item[L'ionosphère] Comme son étymologie l'indique, l’ionosphère est une région de notre haute atmosphère contenant des ions et des électrons formés par photo-ionisation des molécules neutres qui s’y trouvent. C’est le Soleil, et plus particulièrement ses rayonnements énergétiques ultraviolets et X, mais aussi les particules du vent solaire et le rayonnement cosmique, qui sont à l’origine de cette ionisation de la haute atmosphère. L’ionosphère se situe entre~$50$ et~$1000$ kilomètres d’altitude (elle s'étend donc de la mésosphère à la thermosphère). L’ionosphère est habituellement divisée horizontalement en différentes couches, baptisées D, E et F dans lesquelles l’ionisation croît avec l’altitude. Ces couches proviennent des différences de pénétration dans l’atmosphère des rayonnements solaires selon leur énergie.
\item[La magnétosphère] Notre planète génère son propre champ magnétique, un peu à la manière d’une dynamo. C’est la différence de vitesse entre la rotation de la planète et de son coeur liquide qui, par induction, génère ce champ magnétique. Ce champ magnétique protège la Terre des agressions extérieures comme les rayons cosmiques et les particules énergétiques du vent solaire. Cette zone protégée s'appelle magnétosphère. Elle démarre au dessus de l’ionosphère, à plusieurs milliers de kilomètres de la surface du sol, et s’étend jusqu’à 70 000 kilomètres environ du côté du Soleil. Du côté opposé, la queue de la magnétosphère s’étire sur plusieurs millions de kilomètres. Les contours de la magnétosphère évoluent continuellement sous l’action du vent solaire et de sa variabilité.
\end{description}
\normalsize



\mk \section{Composition atmosphérique}
	
	\sk \subsection{Un mélange de gaz parfaits}
	\sk
On appelle \voc{gaz parfait} un gaz suffisament dilué pour que les interactions entre les molécules du gaz, autres que les chocs, soient négligeables. L'air composant l'atmosphère peut être considéré en bonne approximation comme un mélange de gaz parfaits\footnote{On peut en général considérer que le gaz est parfait si~$P < 1$~kbar. C'est le cas dans la plupart des atmosphères planétaires rencontrées. Il n'y a guère qu'au coeur des planètes géantes, où la pression dépasse cette limite, que l'approximation du gaz parfait doit être abandonnée.} notés~$i$, dont le nombre de moles est~$n_i$ pour un volume donné~$V$ d'air à la température~$T$. Chaque espèce gazeuse composant l'air est caractérisée par une \voc{pression partielle}~$P_i$ qui est définie comme la pression qu'aurait l'espèce gazeuse si elle occupait à elle seule le volume~$V$ à la température~$T$. Chacune de ces espèces gazeuses~$i$ se caractérise par une même température~$T$ et vérifie l'équation d'état du gaz parfait $$ P_i \, V = n_i \, R^* \, T $$ où~$R^*$=8.31 J~K\mo~mol\mo~est la constante des gaz parfaits (produit du nombre d'Avogadro et de la constante de Boltzmann). La pression totale de l'air~$P$ est, d'après la loi de Dalton, la somme des pressions partielles~$P_i$ des espèces gazeuses composant le mélange $P=\Sigma P_i$. En faisant la somme des lois du gaz parfait appliquées pour chacune des espèces gazeuses, on obtient $$ P \, V = \big( \Sigma n_i \big) \, R^* \, T $$ ce qui montre qu'un mélange de gaz parfaits est aussi un gaz parfait. Cette équation permet de relier la pression totale~$P$ à la température~$T$, mais présente l'inconvénient de contenir les grandeurs \voc{extensives}~$V$ et $n_i$ qui dépendent du volume d'air considéré. Il reste donc à donner une traduction intensive à la loi du gaz parfait pour un mélange de gaz. La masse totale contenue dans le volume~$V$ peut s'écrire $m=\Sigma n_iM_i$ où $M_i$ est la masse molaire du gaz $i$. En divisant l'équation précédente par $m$, et en utilisant la définition de la masse volumique~$\rho = m / V$, on obtient $$ \frac{P}{\rho} = \frac{\Sigma n_i}{\Sigma n_iM_i} \, R^* \, T $$ Or, d'une part, la \voc{masse molaire de l'air} composé d'un mélange de gaz~$i$ est $$ \boxed{ M=\frac{\Sigma n_iM_i}{\Sigma n_i} }$$ et d'autre part, on peut définir une \voc{constante de l'air sec} de la façon suivante $$ R=\frac{R^*}{M} $$ On a alors l'équation des gaz parfaits pour l'air atmosphérique qui permet de relier les trois paramètres intensifs importants : pression~$P$, température~$T$ et densité~$\rho$ $$ \boxed{ P = \rho \, R \,T } $$ L'état thermodynamique d'un élément d'air est donc déterminé uniquement par deux paramètres sur les trois~$P$, $T$, $\rho$. En météorologie par exemple, on travaille principalement avec la pression et la température qui sont plus aisées à mesurer que la densité. Les valeurs numériques de~$M$, et donc~$R$, dépendent de la planète considérée et de sa composition atmosphérique. 



	\sk \subsection{Composition moléculaire}
	\sk
La concentration, au sens où elle est définie dans la plupart des cours de physique / chimie, est une quantité très peu utilisée en sciences de l'atmosphère. La composition chimique de l'atmosphère s'exprime préférentiellement en utilisant le \voc{rapport de mélange volumique}, soit la proportion d'un volume d'air occupée par un gaz particulier. L'air étant un gaz parfait, ce rapport de mélange volumique est simplement égal au rapport du nombre de molécules/atomes du gaz sur le nombre total de molécules d'air $$ \boxed{ r_i=\frac{n_i}{\Sigma n_k} } \qquad \textrm{\footnotesize (parfois également noté $q_i$)} $$ D'après la loi de Dalton, il correspond également au rapport entre la pression partielle du gaz considéré avec la pression totale du mélange. Le rapport de mélange n'est exprimé en pourcentage que pour les composés les plus abondants. Pour les \voc{gaz traces}, moins abondants, on exprime le rapport de mélange en parties par million (1 ppmv = $10^{-6}$), ou par milliards (1 ppbv = $10^{-9}$), voire le pptv tel que~1 pptv = $10^{-12}$. Dire que le rapport de mélange de CO$_2$ au sol est d’environ 380 ppmv signifie que sur un million de molécules d’air, 380 sont des molécules de CO$_2$. %On utilise également parfois le rapport de mélange massique, soit la proportion d'une masse d'air occupée par un gaz particulier. 


	\begin{table}
\centering
\begin{tabular}{lcc}
%\toprule
%\makecell[l]{\textbf{Constituant}} &
%\makecell{\textbf{Masse}\\\textbf{Molaire}} &
%\makecell{\textbf{Rapport}\\\textbf{de Mélange}} \\
%\midrule
\hline
\textbf{Constituant} &
\textbf{Masse molaire} &
\textbf{Rapport de mélange} \\
\hline
Azote (N$_2$) & 28 & 78\% \\
Oxygène (O$_2$) & 32 & 21\% \\
Argon (Ar) & 40 & 0.93\% \\
\textbf{Vapeur d'eau (H$_2$O)} & 18 & 0-5\% \\
\textbf{Dioxyde de Carbone (CO$_2$)} & 44 & 380 ppmv \\
Néon (Ne) & 20 & 18 ppmv \\
Hélium (He) & 4 & 5 ppmv \\
\textbf{Méthane (CH$_4$)} & 16 & 1.75 ppmv \\
Krypton (Kr) & 84 & 1 ppmv \\
Hydrogène (H$_2$) & 2 & 0.5 ppmv \\
\textbf{Oxide nitreux (N$_2$O)} & 56 & 0.3 ppmv \\
\textbf{Ozone (O$_3$)} & 48 & 0-0.1 ppmv \\
%\bottomrule
\hline
\end{tabular}
\caption{\emph{Principaux composants de l'atmosphère. Les gaz à effet de serre sont indiqués en gras.}}
\label{tab:compos}
\end{table}


\sk
L'azote et l'oxygène dominent largement la composition de l'atmosphère terrestre (tableau \ref{tab:compos}), suivis par l'argon et d'autres gaz rares beaucoup moins abondants. Les rapports de mélange de vapeur d'eau et d'ozone sont très variables : la vapeur d'eau est présente surtout dans la troposphère, avec un maximum près de la surface et dans les tropiques, alors que l'ozone est principalement présente dans la stratosphère. Un certain nombre de gaz traces sont émis régulièrement au niveau de la surface, par des phénomènes naturels ou les activités humaines. Leur répartition dépend alors beaucoup de leur \voc{durée de vie} dans l'atmosphère. Le CO$_2$ qui est très stable est bien mélangé. Le méthane, qui a une durée de vie d'une dizaine d'années, est bien réparti dans la troposphère mais son rapport de mélange varie dans la stratosphère. Des polluants à durée de vie courte (quelques jours) comme l'ozone troposphérique, se retrouveront surtout au voisinage des sources. Les activités humaines ont également contribué à modifier le rapport de mélange de certains de ces gaz (par exemple, le~CO$_2$).

\sk
\begin{finger}
\item En employant les formules obtenues à la section précédente, et les informations ci-dessus sur la composition de l'atmosphère terrestre, il est possible de déterminer des valeurs numériques utiles
\begin{citemize}
\item La masse molaire de l'air est~$M = 28.966$~g~mol$^{-1}$ (on emploie souvent~$M \simeq 29$~g~mol$^{-1}$). 
\item La constante de l'air sec est~$R = 287$~J~K$^{-1}$~kg$^{-1}$.
\end{citemize}  
\end{finger}

	\sk
\begin{finger}
\item Dans toute discussion de la composition atmosphérique, il est important de faire la distinction entre composés minoritaires et majoritaires [figure~\ref{fig:minor}]. Alors que les composés majoritaires suivent une distribution verticale en accord avec l'état énergétique et dynamique de l'atmosphère globale, les composés minoritaires peuvent avoir des comportements très différents qui dépendent à la fois des mécanismes photochimiques de production et de perte ainsi que des phénomènes de transport.
\item La composition de l'air donnée ici est valide sur les premiers~$80$ à~$100$ kilomètres d'altitude, à part quelques constituants mineurs. On appelle cette région l'\voc{homosphère}, elle correspond approximativement à la troposphère, la stratosphère et la mésosphère (Figure \ref{fig:tempvert}). Dans l'homosphère, l'atmosphère est un mélange homogène de différents gaz, l'échelle de hauteur est la même pour tous les gaz. Au dessus de cette altitude, le libre parcours moyen des molécules devient très grand et on a une \ofg{décantation} où les éléments plus légers dominent progressivement aux altitudes élevées. Chaque composant suit sa propre échelle de hauteur. On parle alors d'\voc{hétérosphère}; elle regroupe approximativement la thermosphère et l'exosphère. 
\end{finger}
%Au niveau du sol, l'atmosphère standard sèche est caractérisée par une pression d’environ 1013 hPa et une concentration totale de 2,69 x 1019 molécule~cm$^{-3}$ lorsque la température est de 273~K.

\figun{0.7}{0.3}{\figpayan/LP211_Chap1_Page_06_Image_0001.png}{Composition de l’atmosphère~: des espèces en très faibles quantités jouent un rôle très important. Sur la figure sont données quelques mesures de constituants minoritaires dans l’homosphère. Les courbes en trait fin correspondent aux concentrations résultant de rapports de mélange volumiques constants de~$10^{-1}$ à~$10^{-13}$. (Source: Kockarts, Aéronomie, 2000).}{fig:minor}






\begin{itemize}
\item Fractionnement isotopique: Homosphère-hétérosphère
\begin{itemize}
\item A haute altitude $z>z_h$, la diffusion turbulente est moins efficace que la \voc{diffusion moléculaire} : \voc{hétérosphère}
\item Homopause : $z_h \sim 100\,km$ (telluriques), $\sim 750\,km$ (Titan)
\item Chaque espèce suit alors sa propre échelle de hauteur : les plus légères deviennent plus abondantes à haute altitude
\end{itemize}
\item 2 facteurs influencent le fractionnement
{\footnotesize \begin{description}
\item[\'Equilibre diffusif] : différence de composition atmosphérique entre l'homopause et l'exobase
\item[\'Echappement différentiel] : $V_{th}/V_e$ plus grand pour les isotopes légers.
\end{description} }
%\item Exemples
%\begin{itemize}
%\item {\bf D/H} (Vénus)
%\item {\bf $^{14}$N/$^{15}$N}, {\bf $^{16}$O/$^{18}$O} (Mars)
%\end{itemize}
\end{itemize}


Note: Diffusion turbulente, paramétrisée par un coefficient de diffusion turbulente qui le fait ressembler à un paramètre de diffusion moléculaire.
mais provient de la convection thermique (flottaison) dans la basse atmosphère et les ondes de gravité dans la haute atmosphère

	\figun{1}{0.35}{decouverte/cours_dyn/composition.png}{Objets du système solaire présentant une atmosphère substantielle et leurs caractéristiques.}{fig:composition}

\sk
Les compositions atmosphériques sont très distinctes selon les planètes du système solaire considérées [figure~\ref{fig:composition}]. L'atmosphère terrestre est unique parmi les atmosphères des autres planètes du système solaire. Elle est riche en azote et oxygène et pauvre en \carb, contrairement aux atmosphères de Venus et Mars, qui contiennent plus de~$90\%$ de \carb. On pourrait penser que la Terre a acquis son atmosphère pendant sa formation à partir des gaz présents dans la nébuleuse solaire. Une telle atmosphère serait alors primaire, et contiendrait des gaz de composition cosmique, c'est-à-dire similaire aux abondances chimiques du système solaire. Or, les gaz dominants dans le système solaire sont l'hydrogène et l'hélium. Ces gaz légers sont pratiquement absents dans notre atmosphère, car la gravitation terrestre est trop faible pour les retenir. Les planètes géantes comme Jupiter ou Saturne ont conservé ces gaz primordiaux dans leur atmosphère, au contraire des planètes internes du système solaire Venus, Terre et Mars qui ont des atmosphères de composition bien différente. Si la Terre a eu une telle atmosphère primaire pendant sa formation, elle l'a perdu rapidement. L'atmosphère actuelle doit donc être secondaire. L'évolution de sa composition résulte en partie de l'apparition de la vie [figure~\ref{fig:vie}].
%% parler des puits de carbone et des carbonates. comparaison entre Vénus et Mars.

\figun{0.75}{0.32}{\figpayan/vie.png}{Evolution parallèle de l’atmosphère et de la vie.}{fig:vie}


	\sk \subsection{Aérosols et hydrométéores}
	%%\footnote{Cette partie est inspirée d'éléments trouvés dans le cours de S. Jacquemoud de \emph{Méthodes physique en télédétection.}}

\sk
Les \voc{aérosols} sont constitués de petites particules solides ou liquides en suspension dans les basses couches de l’atmosphère. Environ trois milliards de tonnes de particules sont injectés chaque année dans l’atmosphère par les processus naturels ou les activités humaines. On distingue plusieurs types d'aérosols.
\begin{citemize}
\item[\emph{poussières d'origine désertique}] Il s'agit de la première source mondiale d’aérosols. Elles sont soulevées par des vents violents lors des tempêtes de sable. Les grosses particules retombent rapidement au sol alors que les plus petites forment un nuage sec qui peut s’élever jusqu’à 4-6 km d’altitude et s’étendre sur des milliers de kilomètres. On peut retrouver en Europe ou en Amérique des particules en provenance d'Afrique.
\item[\emph{aérosols solubles dans l’eau}] Ils peuvent être d'origine naturelle (substances organiques émises par la végétation) ou liés à l'activité industrielle (sulfates, nitrates).
\item[\emph{aérosols d'origine marine}] Ils sont formés à partir des bulles résultant du déferlement des vagues et des courants marins. Outre le dioxyde de carbone, les bulles transportent quantité de substances, notamment des particules de sel microscopiques (NaCl) qui participent à la formation de brumes. L'éclatement de ces bulles à la surface des océans donne naissance à un très grand nombre de gouttelettes (parfois une centaine pour une bulle d'un diamètre de l'ordre du millimètre) qui ne se brisent pas et sont à l'origine des aérosols marins (également appelés embruns).
\item[\emph{aérosols carbonés}] Ils sont présents dans les régions tropicales et boréales en raison de nombreux feux de forêt. Par exemple les brûlis de la végétation intertropicale en période sèche occasionnent chaque année des brumes sèches qui disparaissent une fois les pluies revenues.
\item[\emph{aérosols de sulfates}] Ils sont d'origine volcanique. Le dioxyde de soufre SO$_2$ émis lors des éruptions volcaniques produit ces fines particules d'acide sulfurique (SO$_2$ + H$_2$O~$\rightarrow$~H$_2$SO$_4$) qui s'entourent de glace et forment avec les cendres un écran empêchant le rayonnement solaire d'arriver jusqu'au sol.
\end{citemize}
La plupart des aérosols se trouvent dans la troposphère où ils résident en moyenne une semaine. En raison de leur petite taille, les aérosols peuvent être transportés sur de longues distances. Ils sont en général ramenés au sol par les précipitations (pluie, neige). Les aérosols de plus petite taille ($0.01-0.1$~$\mu$m) jouent un rôle important de \voc{noyaux de condensation} dans la formation des nuages en favorisant la condensation de vapeur d'eau en gouttelettes d'eau et/ou de cristaux de glace. Les aérosols de plus grande taille (0.1-1.0~$\mu$m), les plus nombreux, interceptent la lumière du Soleil. La stratosphère contient aussi des aérosols (principalement d'origine volcanique) jusqu'à 18-20 km d’altitude. Contrairement aux aérosols troposphériques, leur concentration est relativement uniforme et leur durée de vie beaucoup plus longue, de plusieurs mois à plusieurs années.  

\sk
Certaines molécules peuvent s’agréger pour former des particules liquides ou solides. Dans l'atmosphère terrestre, ceci concerne principalement l'eau à l'état liquide ou solide dans l'atmosphère qui participe à la formation d'\voc{hydrométéores}. Ce sont des particules d’eau liquide (gouttelettes d'eau) et/ou solide (cristaux de glace) suspendues dans l'atmosphère dont la taille varie de~$1$~$\mu$m à~$1$~cm. Les brumes et nuages sont formés de ces fines gouttelettes d'eau en suspension dans l'atmosphère, qui apparaissent dès que le seuil de saturation de l'air en vapeur d'eau est dépassé (ces mécanismes sont détaillés dans un chapitre ultérieur). Les nuages bas et intermédiaires sont constitués de gouttelettes d’eau liquide ; les nuages d’altitude de cristaux de glace de différentes formes géométriques. Certains sont accompagnés de précipitations lorsque les gouttes ou cristaux sont assez gros pour former de la pluie, neige, grêle ou verglas. %La figure~\ref{fig:droplet} donne un ordre d'idée des tailles respectives des noyaux de condensation, des gouttelettes nuageuses et des gouttes de pluie.
%\figside{0.4}{0.15}{decouverte/cours_meteo/gouttes.png}{Taille comparée des noyaux de condensation (\emph{cloud condensation nuclei}), des gouttelettes de brume ou de nuage (\emph{moisture droplets}), des gouttes de pluie (\emph{raindrop})}{fig:droplet}





\mk 
\section{Echanges énergétiques dans l'atmosphère}

\sk
L'atmosphère, et plus généralement le système climatique dont elle fait partie, échangent de l'énergie sous diverses formes. Ce sera le but des chapitres suivants que de détailler certains de ces échanges énergétiques. On donne ici un aperçu de la diversité des types d'énergie impliqués, ce qui permet d'illustrer la complexité du système atmosphérique (voir également les figures~\ref{fig:pluri} et~\ref{fig:flux}). 

%\figside{0.6}{0.4}{decouverte/cours_meteo/giec2007_gcm.jpg}{Evolution des phénomènes et éléments du système climatique intégrés dans les modèles numériques de climat. FAR correspond à 1990. SAR correspond à 1995. TAR correspond à 2001. AR4 correspond à 2007. Figure tirée du 4ème rapport du GIEC en 2007.}{fig:gcm}

\sk
\subsection{\'Energie radiative}\label{sec:energrad}

\sk
L'énergie \voc{radiative} est relative au rayonnement électromagnétique, dont les caractéristiques sont rappelées dans un chapitre ultérieur. L'atmosphère reçoit de l'énergie sous forme radiative principalement par le rayonnement lumineux reçu du soleil (maximum d'énergie émise dans les longueurs d'onde visible) et le rayonnement lumineux réémis par la surface terrestre chauffée par le Soleil (maximum d'énergie émise dans les longueurs d'onde \voc{infrarouge}, plus grandes que les longueurs d'onde visibles et moins énergétiques). Le flux d'énergie instantané, c'est-à-dire la puissance par unité de surface, reçu par la Terre (surface + atmosphère) du Soleil est~$1368$~W~m$^{-2}$. On appelle cette valeur la \voc{constante solaire}. Le rayonnement solaire reçu au sommet de l'atmosphère en un point donné de la surface varie en fonction de la latitude, de l'heure de la journée et de la saison considérée (c'est-à-dire la position de la Terre au cours de sa révolution annuelle autour du Soleil). %Le flux moyen reçu par le sommet de l'atmosphère est quatre fois plus faible que la constante solaire~:~$342$~W~m$^{-2}$. 

\sk
L'énergie radiative peut être absorbée par les espèces atmosphériques et convertie en énergie interne pour l'atmosphère. L’absorption du rayonnement solaire est dû à 
\begin{finger}
\item l’ozone (O$_3$), le dioxygène (O$_2$) et la vapeur d’eau (H$_2$O) dans les courtes longueurs d’onde (UV / visible),
\item la vapeur d’eau, le dioxyde de carbone (CO$_2$), le méthane (CH$_4$), l’oxyde nitreux (N$_2$O) et l’ozone (O$_3$) dans les longues longueurs d’ondes (infrarouge et micro-onde).
\end{finger}
Le rayonnement incident peut également être diffusé par les gaz et les aérosols présents dans l'atmosphère. Ce dernier phénomène prévaut pour les longueurs d'onde visible, puisque l'absorption du rayonnement solaire incident par l'atmosphère est relativement faible : on dit que l’atmosphère est relativement \voc{transparente} au rayonnement solaire dans le visible. L'ensemble de ces phénomènes sont responsables de la différence d'énergie radiative reçue au sommet de l'atmosphère et au niveau de la mer [figure~\ref{fig:radiation}]. 

%\figsup{0.45}{0.2}{\figpayan/LP211_Chap2b_Page_18_Image_0001.png}{\figpayan/LP211_Chap1_Page_07_Image_0004.png}{[Gauche] Spectre du rayonnement solaire (UV, Visible, IR) à l’extérieur de l’atmosphère terrestre et au niveau du sol (après traversée de l’atmosphère). Le rayonnement au sommet de l'atmosphère correspond à la courbe rouge. Le rayonnement diffusé par l'atmosphère correspond à la courbe bleu foncé. Le rayonnement absorbé par l'atmosphère correspond à la courbe bleu clair. Le rayonnement résultant au sol correspond à la courbe jaune. [Droite] Figure similaire, mais les espèces gazeuses responsables des bandes d'absorptions sont indiquées sur la figure.}{fig:radiation}

\sk
Le rayonnement émis par la surface terrestre, principalement dans l'infrarouge, est également absorbé par les espèces précitées (vapeur d'eau, CO$_2$, CH$_4$) et réémis à la fois vers l'espace et vers la surface. Ainsi, une partie du rayonnement émis par la surface est \ofg{piégée}, n'est pas évacuée vers l'espace et contribue à augmenter la température de la surface terrestre. Ce phénomène est désigné par le terme d'\voc{effet de serre} et les gaz qui en sont responsables s'appellent les \voc{gaz à effet de serre}. Les nuages peuvent également induire un effet de serre. Du point de vue radiatif, les nuages jouent d'ailleurs un double rôle dans le bilan énergétique de la Terre : ils réfléchissent le rayonnement du soleil vers l’espace dans le visible (donc refroidissent l'atmosphère) et exercent un effet de serre (et donc réchauffent l'atmosphère). 

\sk
L'activité humaine et les processus naturels modifient les rapports de mélange des espèces radiatives présentes dans l'atmosphère (gaz ou aérosols), en particulier les gaz à effet de serre. L'homme modifie ainsi le bilan énergétique de la planète, donc le fonctionnement du climat : on parle de \voc{changement climatique} [figure~\ref{fig:giecrad}].

%\figside{0.45}{0.45}{decouverte/cours_meteo/ges.jpg}{Rapports de mélange du dioxyde de carbone~CO$_2$, méthane~CH$_4$ et oxyde nitreux~N$_2$O au cours des 10 000 dernières années (larges figures) et depuis 1750 (petits inserts). Les données indicatrices des changements de la composition de l’atmosphère au cours du dernier millénaire mettent en évidence l’augmentation rapide des gaz à effet de serre qui est imputable principalement à la croissance économique depuis 1750. Les gaz sont bien mélangés dans l’atmosphère et leurs concentrations depuis 1750 reflètent les émissions provenant des sources à travers le monde. Les mesures proviennent de carottes glaciaires (différentes couleurs sont utilisées pour les diverses études scientifiques utilisées) et de campagnes atmosphériques (lignes rouges). Les forçages radiatifs comparés à la valeur de 1750 pour chacun des gaz représentés sont reportés sur la droite des larges figures. Figure tirée du 4ème rapport du GIEC en 2007.}{fig:giecrad}

\sk
\subsection{\'Energie mécanique}

La \voc{dynamique atmosphérique}, c'est-à-dire les vents, fait intervenir l'énergie cinétique et potentielle (dont la somme forme l'énergie mécanique). L'atmosphère ne reçoit pas partout la même quantité d'énergie~: de par sa sphéricité, la Terre reçoit plus d'énergie radiative du Soleil en chaque instant dans les régions équatoriales et tropicales qu'aux pôles. L'énergie émise par le système Terre (principalement dans les longueurs d'onde infrarouge) suit un comportement différent, où les différences selon la latitude sont moins marquées. Il en résulte un excédent d'énergie dans les basses latitudes et un déficit d'énergie dans les hautes latitudes. Cette différence provoque des courants atmosphériques de grande échelle qui ont tendance à répartir l'énergie des régions excédentaires en énergie vers les régions déficitaires en énergie~[figure~\ref{fig:hadley}]. De manière plus générale, la différence de température dans l'air est une des clés de tout mouvement dans l'atmosphère. Ce n'est cependant pas la seule : la rotation de la Terre est d'une importance cruciale pour expliquer comment les vents s'organisent sur Terre. Des développements plus approfondis sont proposés dans un chapitre ultérieur.

%\figside{0.5}{0.2}{decouverte/cours_meteo/energiedyn.png}{Schéma représentant les latitudes où l'atmosphère est excédentaire ou, au contraire, déficitaire en énergie. La courbe bleue représente l'énergie radiative reçue du Soleil, principalement dans les courtes longueurs d'onde (noté \emph{shortwave} sur la figure, correspond au rayonnement visible et ultraviolet). La courbe rouge représente l'émission par la surface terrestre, principalement dans les longues longueurs d'onde (noté \emph{longwave} sur la figure, correspond au rayonnement infrarouge). Une circulation atmosphérique de grande échelle se met en place entre les régions excédentaires (équatoriales et tropicales) et déficitaires (hautes latitudes).}{fig:hadley}

\sk
\subsection{\'Energie latente}

\sk
Une partie de l’énergie reçue par la Terre est transférée vers l’atmosphère par l’intermédiaire de processus qui ne sont ni radiatifs ni dynamiques. Les transferts de \voc{chaleur latente}, relatifs aux changements d'état de l'eau, jouent un rôle particulièrement important sur Terre. L'\voc{évaporation} (ou vaporisation) de l'eau consomme de la chaleur latente afin de briser les liaisons hydrogène qui existent à l'état liquide pour passer à l'état gazeux (ou vapeur). Ceci fait baisser l’énergie interne et contribue donc à un refroidissement de l'atmosphère. A l'inverse, lors de la \voc{condensation}, de la chaleur latente est dégagée et contribue à un chauffage de l'air. Le passage de l’état liquide à l’état solide, comme lors de la formation de neige ou de cristaux de glace dans les cirrus, s’accompagne également d’un dégagement de chaleur latente. En résumé
\begin{citemize}
\item consomment de l'énergie : fusion ($S \rightarrow L$), vaporisation ($L \rightarrow G$), sublimation ($S \rightarrow G$) ; 
\item dégagent de l'énergie : solidification ($L \rightarrow S$), condensation ($G \rightarrow L$), condensation solide ($G \rightarrow S$).
\end{citemize}

\sk
\subsection{Contribution de la chimie}

\sk
La chimie atmosphérique participe aux échanges d'énergie de façon directe par le caractère exothermique ou endothermique des réactions chimiques, mais également de façon indirecte dans la mesure où elle opère un couplage entre les diverses formes d'énergie dans l'atmosphère. Un exemple des couplages associés à la chimie atmosphérique est donné par la figure~\ref{fig:chimie}.
\begin{citemize}
\item Tout d'abord, la chimie contribue à la conversion de l'énergie radiative en énergie interne par le biais des processus \voc{photochimiques}, qui sont des réactions chimiques faisant intervenir le rayonnement par le biais des photons qui le constituent (d'où l'étymologie du nom). 
\item Ensuite, les réactions chimiques et photochimiques modifient la composition atmosphérique des espèces minoritaires dont certaines, comme l'ozone, ont un rôle radiatif central. 
\item Le fait que les espèces qui réagissent soient transportées par les vents introduit une complexité supplémentaire par l'intrication des constantes de temps chimiques, radiatives et dynamiques. 
\item Enfin, la cinétique chimique dépend de la température atmosphérique qui est influencée par les échanges d'énergie sous forme radiative, mécanique, latente. 
\end{citemize}

\figun{0.7}{0.45}{\figpayan/LP211_Chap1_Page_10_Image_0001.png}{Atmosphère et chimie, exemple de l’ozone stratosphérique. Figure inspirée de Brasseur, The role of the stratosphere in global change, 1993.}{fig:chimie}


%%% CHAPITRE 2
%\def\thechapter{2} \title{Sciences de l'atmosphère\\ Chapitre \thechapter} \author{Aymeric SPIGA\\ ~\\ \small Modifié et augmenté à partir des notes de Francis CODRON et Sébastien PAYAN.} \maketitle %% commenter la ligne en mode livre complet
%\chapter{Rayonnement électromagnétique et émission thermique} \label{chap:rad} 
\dictum[The Beatles, 1970]{Here comes the sun, and I say it's allright}

		\bk
Comment déterminer les processus dynamiques, physiques, chimiques à l'oeuvre dans l'atmosphère ? Il faut commencer par faire le point sur les sources d'énergie pour l'atmosphère, les océans et la surface. La principale source d'énergie pour l'atmosphère et le système climatique de la Terre est le Soleil\footnote{Ce n'est pas le cas pour les géantes gazeuses Jupiter et Saturne où il existe un flux de chaleur interne significatif en regard du flux d'énergie reçu du Soleil. Ce flux est un reste de la contraction gravitationnelle au cours de la formation de ces géantes gazeuses.}. La figure~\ref{fig:flux} montre que d’autres sources existent mais en quantité réduite : l'énergie reçue par la géothermie, ou par les activités humaines, est~$4$ ordres de grandeur plus faible que la source solaire; celle reçue des étoiles~$8$ ordres de grandeur plus faible. L’énergie solaire est transmise principalement à la Terre au moyen du rayonnement électromagnétique~: on qualifie cette énergie de \voc{radiative}. %L'objet de ce chapitre est de s'intéresser au rayonnement électromagnétique et plus particulièrement au phénomène d'émission thermique.

\figside{0.75}{0.18}{decouverte/cours_meteo/fluxenergsurf.png}{Ordres de grandeur des flux énergétiques reçus à la surface de la Terre. Source~:~P.~von Balmoos \emph{in} Le Climat à Découvert, CNRS éditions, 2011}{fig:flux}



\mk \section{Description générale du rayonnement électromagnétique}

	\sk \subsection{Spectre électromagnétique}

		\sk
Les échanges d'\voc{énergie radiative} se font à distance par le biais du \voc{rayonnement électromagnétique}. Le rayonnement électromagnétique est composé d'une superposition d'ondes monochromatiques de longueurs d'onde~$\lambda$ se propageant à la vitesse de la lumière~$c$ (dans le vide~$c=3 \times 10^8$~m~s$^{-1}$). Le rayonnement électromagnétique parcourt la distance Terre-Soleil en $8$~minutes; à l'échelle des processus atmosphériques terrestres, la propagation des ondes électromagnétiques est si rapide qu'elle peut être considérée en première approximation comme immédiate. 

\sk
Les ondes composant le rayonnement électromagnétique peuvent être caractérisées indifféremment par leur \voc{longueur d'onde}~$\lambda$, leur \voc{fréquence}~$\nu = c / \lambda$ ou leur \voc{nombre d'onde}\footnote{Le nombre d'onde est souvent exprimé en cm$^{-1}$. Pour obtenir~$\overline{\nu}$ dans cette unité à partir de~$\lambda$ en microns, on utilise~$\overline{\nu} = 10^{4} / \lambda$.}~$\overline{\nu} = 1 / \lambda$. L'ensemble de ces ondes constitue le \voc{spectre} électromagnétique. Selon le principe de De Broglie, à chaque onde électromagnétique de fréquence~$\nu$ est associée une particule sans masse nommée \voc{photon} dont l'énergie est~$h \, \nu$ où $h = 6.63 \times 10^{-34}$~J~s est appelée la constante de Planck. Cette énergie est souvent exprimée en électron-volts eV ($1$~eV~$= 1.6 \times 10^{-19}$~J~s).

\sk
Le rayonnement visible occupe une bande très étroite du spectre aux longueurs d'ondes comprises entre 0.4 et 0.76~$\mu$m [figure~\ref{fig:spectrum}]. Lorsque l'on considère des longueurs d'ondes plus courtes (c'est-à-dire des fréquences plus élevées) que le rayonnement visible, on passe dans le domaine du rayonnement ultraviolet, puis celui des rayons X et gamma~$\gamma$. Lorsque l'on considère des longueurs d'ondes plus grandes (c'est-à-dire des fréquences plus faibles) que le rayonnement visible, on passe dans le domaine du rayonnement infrarouge, puis celui des micro-ondes et des ondes radio. Les photons les plus énergétiques correspondent aux rayons X; les moins énergétiques aux ondes radio.

\figsup{1}{0.1}{\figpayan/LP211_Chap2_Page_09_Image_0001.png}{\figpayan/LP211_Chap2_Page_09_Image_0002.png}{Classification du rayonnement électromagnétique en fonction de la longueur d'onde. On rappelle que 1~$\mu$m (micron) correspond à $10^{-6}$~m et 1~nm (nanomètre) correspond à $10^{-9}$~m.}{fig:spectrum}


	\sk \subsection{Mesures quantitatives~: grandeurs caractéristiques}

		\sk
La quantité de rayonnement émise par une source de rayonnement, ou reçue par une cible, dépend des paramètres~:
\begin{citemize}
\item longueur d'onde~$\lambda$ ;
\item temps d'exposition~$t$ ;
\item surface~$S$ de l'objet (source ou cible) ;
\item direction dans l'espace considérée, que l'on repère par l'angle~$\beta$ entre le rayonnement incident (ou émis) et la normale à la surface (appelé angle zénithal) ;
\item portion d'espace considérée, exprimée par un \voc{angle solide} $\omega$, l'équivalent bidimensionnel d'un angle\footnote{De même qu'un angle en radians est la longueur d'un arc de cercle divisée par le rayon, le stéradian est la surface d'une portion de sphère divisée par le rayon au carré. On a donc $4\pi$ stéradians sur tout l'espace: voir figure \ref{fig:radiance}} ;
\item propriétés physico-chimiques de l'objet, par exemple sa température (voir section~\ref{corpsnoir}).
\end{citemize}
On exprime cette quantité de rayonnement reçue ou émise sous la forme d'une énergie totale~$E\e{r}$ en Joules (J). Que l'on considère une source ou une cible, $E\e{r}$ est l'énergie transmise par le rayonnement (radiative).

\sk
L'énergie totale~$E\e{r}$ est une grandeur intégrée peu utilisée en pratique en sciences de l'atmosphère. On lui préfère les quantités décrites ci-dessous\footnote{Les noms anglais de ces quantités sont respectivement \emph{radiant flux} pour~$\Phi$, \emph{irradiance} pour~$F$, \emph{radiance} pour~$L$} qui décrivent la quantité de rayonnement émise ou reçue par unité de temps, surface, longueur d'onde, \ldots
\begin{finger}
\item \underline{unité de temps} Le \voc{flux énergétique}~$\Phi$ est l'énergie totale~$E\e{r}$ par unité de temps~$t$ (par seconde) $$ \Phi = \ddf{E\e{r}}{t} $$ C'est une puissance exprimée en Watts (W~$\equiv$~J~s$^{-1}$). Le flux énergétique~$\Phi$ est intégré sur toutes les longueurs d'onde, toutes les directions d'espace et sur l'intégralité de la surface de la source ou cible.
\item \underline{unité de temps + unité de surface} La \voc{densité de flux énergétique}~$F$ est le flux énergétique~$\Phi$ par unité de surface~$S$ de la source/cible $$ \boxed{ F = \ddf{\Phi}{S} } = \f{\dd^2 E\e{r}}{\dd S \, \dd t} $$ C'est un \voc{flux net} exprimé en~W~m$^{-2}$. On l'appelle également \voc{émittance}~$M$ pour une source et \voc{éclairement}~$E$ pour une cible. Cette quantité est intégrée sur toutes les longueurs d'onde et toutes les directions d'espace.
\item \underline{unité de temps + unité de surface + direction fixée} La \voc{luminance énergétique}~$L$ est la densité de flux énergétique dans une direction donnée de l'espace repérée par un angle~$\beta$ $$ L = \f{\dd F}{\cos\beta \, \dd \omega} = \f{\dd^2 \Phi}{\cos\beta \, \dd \omega \, \dd S} = \f{\dd^3 E\e{r}}{\cos\beta \, \dd \omega \, \dd S \, \dd t} $$ 
par unité d'angle solide~$\omega$ (figure \ref{fig:radiance}). C'est une quantité surtout utilisée pour les sources, parfois appelée radiance. La surface considérée~$\sigma$ est perpendiculaire à la direction d'émission~: on la relie à~$S$ par~$\dd \sigma = \cos\beta \, \dd S$. La luminance énergétique~$L$ en~W~m$^{-2}$~sr$^{-1}$ est intégrée sur toutes les longueurs d'onde.
\item \underline{longueur d'onde fixée (grandeurs spectrales)} Le flux énergétique \voc{spectral} ou monochromatique~$\Phi_{\lambda}$ est le flux énergétique~$\Phi$ par unité de longueur d'onde~$\lambda$ $$ \Phi_{\lambda} = \ddf{\Phi}{\lambda} = \f{\dd^2 E\e{r}}{\dd \lambda \, \dd t} $$ Cette quantité en~W~m$^{-1}$ est intégrée sur toutes les directions d'espace et sur toute la surface de la source ou cible. On peut également définir des équivalents spectraux~$F_{\lambda}$ et~$L_{\lambda}$ pour les quantités~$F$ et~$L$ $$ \boxed{F_{\lambda} = \ddf{F}{\lambda}} \qquad\qquad L_{\lambda} = \ddf{L}{\lambda} $$ et des quantités spectrales~$\Phi_{\nu}$, $F_{\nu}$ et $L_{\nu}$ à partir de la fréquence~$\nu$ $$ \Phi_{\nu} = \ddf{\Phi}{\nu} \qquad\qquad F_{\nu} = \ddf{F}{\nu} \qquad\qquad L_{\nu} = \ddf{L}{\nu} $$ Les quantités spectrales définies à partir de la fréquence sont parfois plus avantageuses, dans la mesure où la fréquence est indépendante du milieu matériel transparent où l'onde matérielle se propage\footnote{La longueur d'onde~$\lambda$ dépend de l'indice de réfraction~$n$ du milieu (pour l'air, $n$ est proche de~$1$) et de la longueur d'onde dans le vide~$\lambda_0$.}. Attention, les unités des quantités spectrales dépendent de la quantité référente : ainsi $L_{\lambda}$ est en~W~m$^{-3}$~sr$^{-1}$ et $L_{\lambda}$ est en~W~m$^{-2}$~sr$^{-1}$~s.
\end{finger}

\figun{0.6}{0.15}{\figfrancis/lum_emit.pdf}{Schéma montrant l'émittance~$M$ et la luminance~$L$ d'un élément de surface $dS$ d'une source. $M$ est l'intégrale du flux dans toutes les directions. $L$ est le flux émis dans une certaine direction par unité de surface perpendiculaire.}{fig:luminance}

\begin{figure} \begin{center} \input{\figfrancis/luminance.pdftex_t} \end{center} \caption{\emph{ Schéma en coordonnées sphériques de la luminance $L$ de l'élément de surface $dS$ d'une source située dans un plan $(Oxy)$. La luminance est définie pour chaque direction repérée par les angles $\theta$ et $\varphi$. L'angle solide élémentaire autour d'une direction donnée vaut $\dd \omega = \sin\theta \, \dd\theta \, \dd\varphi$ (rapport entre surface hachurée et $r^2$). 
%La relation avec le flux énérgétique~$\Phi$ émis par la source est $L = \dd^2 \Phi / \left( \dd\omega \, \dd S \, \cos\theta \right)$.
}} \label{fig:radiance} \end{figure}

\sk
\subsubsection{Exemples d'application}

\sk
Dans ce qui précède, on part de la quantité la plus intégrée possible, à savoir l'énergie totale~$E\e{r}$, pour parvenir par dérivation à des quantités moins complexes à appréhender en pratique. Le chemin inverse se fait par intégration (sommation continue). On considère ici quelques exemples illustratifs. 
\begin{finger}
\item Pour reprendre la situation de la figure~\ref{fig:luminance}, supposons que l'on dispose d'informations sur la luminance énergétique~$L$ d'une source plane de surface élémentaire~$\dd S$, c'est-à-dire la quantité de rayonnement émise dans chaque direction de l'espace. Afin de connaître le flux net~$F$ dans tout l'espace (que l'on peut noter également émittance~$M$ puisque l'on considère une source), il suffit de l'écrire comme une intégrale de la luminance sur toutes les directions d'un demi-espace: $$ F = \int_{2\,\pi} L \, \cos\beta \, \dd \omega $$ où $2\,\pi$ représente l'intégration sur un demi-espace. %Deux cas particuliers sont intéressants. Dans la limite d'un rayonnement rasant~$\beta=\pi/2$, la contribution au flux net est nulle. Par ailleurs, s
Si la luminance~$L$ est indépendante de la direction, c'est-à-dire que le rayonnement est \voc{isotrope}, l'intégration donne simplement~$F=\pi L$. Dans ce cas on parle d'une \voc{source lambertienne}.
%C'est le cas d'un réflecteur de Lambert.
%Jean-Henri Lambert (1728-1777) a observé que l’énergie émise par certaines sources (parmi toutes les types de sources à sa disposition) anisotropes diminue comme le cosinus de l'angle θ, autour de la direction perpendiculaire à la surface de la source. Cette variation de l'énergie émise est observée lorsque nous mesurons l'énergie thermique rayonnée par un orifice percé dans un four (ce qui nous ramène au corps noir défini plus loin), isolé thermiquement et dont la température interne est supérieure à la température externe. Dans ce contexte, l'orifice est appelé un émetteur Lambert et ne balaye un espace que de stéradian. Une source obéit à la loi de Lambert si l’énergie rayonnée depuis un point de cette source est la même dans toutes les directions (on dit que son intensité est isotrope et donc indépendante de l'angle d’où on observe cette source). Soit M la valeur de l’éclairement mesurée par un capteur. On peut facilement en déduire le flux énergétique de la source de surface S : Φ =S M
\item Suppons que l'on dispose cette fois d'informations sur la densité de flux énergétique~$F$ d'une source sphérique de rayon~$R$ (par exemple, le Soleil). Puisque l'on considère une source, $F$ peut également être appelée émittance et être notée~$M$. Si le rayonnement est \voc{uniforme}, c'est-à-dire qu'en chaque point la source émet le même flux énergétique~$\Phi$ par unité de surface~$\dd S$, alors on dispose de la relation suivante $$ \Phi = 4\,\pi\,R^2 \, M$$ et plus généralement pour une surface~$S$ qui est une source uniforme de rayonnement $$ \boxed{ \Phi = S \, M } $$ On notera qu'on fait les calculs en considérant seulement le côté extérieur de la surface, celui d'où nous regardons la source, car seule la moitié de l'énergie échangée par les points de la surface~$S$ est émise sous forme de rayonnement. L'autre moitié est échangée du côté intérieur de la surface avec le milieu constituant le corps. 
\item L'énergie transmise par le rayonnement, et toutes les grandeurs définies précédemment, varient généralement beaucoup avec la longueur d'onde étudiée. Supposons que l'on connaisse la luminance spectrale~$L_\lambda$ dans un petit intervalle~$\dd \lambda$ autour de la longueur d'onde~$\lambda$, et ce, pour toutes les longueurs d'onde~$\lambda$ du spectre électromagnétique. Par exemple, la luminance totale~$L$ est retrouvée par intégration des longueurs d'onde les plus courtes aux plus longues : $$L=\int_{\lambda} L_\lambda \, \dd\lambda = \int_\nu L_\nu \, \dd \nu $$ Il s'agit d'une des formes du \voc{principe de superposition}, qui indique que le rayonnement électromagnétiques se compose d'une superposition d'ondes monochromatiques. Une intégration similaire est effectuée par l'électronique embarquée dans un appareil photo qui traite les flux reçus par les capteurs dans une variété de longueurs d'onde en domaine visible, afin d'obtenir une image finale qui intègre toutes ces informations.
\end{finger}



\mk \section{Emission de rayonnement} \label{corpsnoir}

		\sk
Le Soleil qui se situe à une distance considérable dans le vide spatial nous procure une sensation de chaleur. De même, placer sa main sur le côté d'un radiateur en fonctionnement sans le toucher procure une sensation de chaleur instantanée qui ne peut être attribuée à un transfert convectif entre le radiateur et la main. Cet échange de chaleur est attribué au contraire à l'émission d'ondes électromagnétiques par la matière du fait de sa température; on parle d'émission de \voc{rayonnement thermique}. Tous les corps émettent du rayonnement thermique. La transmission de cette énergie entre une source et une cible ne nécessite pas la présence d'un milieu intermédiaire matériel. 
%Le but de cette section est d'en étudier les principales propriétés.


	\sk \subsection{Corps noir}

		\sk
On appelle \voc{corps noir} un objet dont la surface est idéale et satisfait les trois conditions suivantes~:
\begin{description}
\item[émetteur parfait] un corps noir rayonne plus d’énergie radiative à chaque température et pour chaque longueur d’onde que n'importe quelle autre surface,
\item[absorbant parfait] un corps noir absorbe complètement le rayonnement incident selon toutes les directions de l'espace et toutes les longueurs d'onde,
\item[source lambertienne] un corps noir émet du rayonnement de façon isotrope
\end{description}

\sk
Un corps noir est à l'équilibre thermodynamique avec son environnement. On peut montrer qu'un tel corps émet du rayonnement qui dépend seulement de sa température et non de sa nature. La définition du corps noir, et les développements théoriques qui l'accompagnent, sont partis du constat, fait notamment par les céramistes, qu'un objet placé dans un four à haute température devient rouge en même temps que les parois du four quelle que soit sa taille, sa forme ou le matériau qui le compose. Un exemple de source utilisée pour étudier expérimentalement le modèle du corps noir consiste à construire une enceinte chauffée, totalement hermétique, et y percer un trou pour y mesurer le flux énergétique émis [figure~\ref{fig:four}]

\figside{0.35}{0.15}{\figwallace/Radiation/radiation_Page_10_Image_0001.png}{L'énergie entrant par une petite fente dans une enceinte subit des réflexions sur la paroi jusqu'à ce qu'elle soit absorbée. L'ouverture dans la paroi d'une enceinte chauffée apparaît comme une source de type corps noir. Un absorbant presque parfait est aussi un émetteur presque parfait. Ce type de four a été employé au début du XXe siècle pour évaluer expérimentalement les prédictions théoriques de Planck. Source~: Wallace and Hobbs, Atmospheric Science, 2006.}{fig:four}


		\sk
L'émission de rayonnement par le corps noir est décrite par une luminance énergétique spectrale~$L_{\lambda}$, notée $B_\lambda$ dans ce qui suit\footnote{Correspond au nom anglais \emph{blackbody}}. La loi de variation de~$B_\lambda$ selon la température~$T$ est donnée par la \voc{loi de Planck}\footnote{La luminance spectrale $B_\nu$ est déterminée d'une façon similaire. La démonstration de la loi de Planck fait appel à des notions de quantification d'énergie et de thermodynamique statistique qui sont hors programme dans le cadre de ce cours.} $$ B_\lambda(T) = \frac{C_1 \, \lambda^{-5}}{\pi \, \left( e^{ C_2 / \lambda T}-1\right) } $$ où $C_1$ et $C_2$ sont des constantes. Comme le rayonnement du corps noir est isotrope, l'émittance spectrale du corps noir, obtenue par intégration sur toutes les directions de l'espace, vaut $ M_\lambda(T) = \pi \, B_\lambda(T) $. 

%\figun{0.5}{0.25}{\figfrancis/WH_BBrad}{Courbes de luminance spectrale d'un corps noir pour différentes températures. La courbe en pointillés indique la position du maximum en fonction de $T$.}{fig:BBrad} 
\figside{0.5}{0.25}{\figwallace/Radiation/radiation_Page_11_Image_0001.png}{Courbes de luminance spectrale d'un corps noir pour différentes températures. La courbe en pointillés indique la position du maximum en fonction de $T$. Source~: Wallace and Hobbs, Atmospheric Science, 2006.}{fig:BBrad} 

\sk
Les variations de la fonction~$B_\lambda$ sont illustrées sur la figure~\ref{fig:BBrad}. L'émission de rayonnement par le corps noir ne dépend que de la longueur d'onde~$\lambda$ et de la température~$T$ du corps. A une température donnée, le rayonnement émis est parfaitement déterminé pour chaque longueur d'onde; dans un domaine spectral particulier, le rayonnement émis ne dépend que de la température du corps noir.



		\paragraph{Variations selon la température} 

\begin{finger}
\item L'énergie émise dépend de la température du corps émetteur~: 
\begin{citemize}
\item quantitativement~: plus le corps est chaud, plus la quantité de rayonnement thermique est grande~: la luminance spectrale~$B_{\lambda}$ augmente avec la température $T$ quelle que soit la longueur d'onde.
\item qualitativement~: la \ofg{couleur} du corps dépend de sa température~: la longueur d'onde pour laquelle le rayonnement est maximal diminue quand la température augmente.
\end{citemize}
\item La dépendance en température de la forme des courbes sur la figure~\ref{fig:BBrad} est résumée par deux lois simples qui sont décrites à la section suivante~: la loi de Wien (position du maximum) et la loi de Stefan-Boltzmann (intégrale totale).  
\end{finger}

\paragraph{Variations selon la longueur d'onde} 

\begin{finger} 
\item Le rayonnement thermique est surtout significatif entre les longueurs d'onde~$0.1$ et~$100$~$\mu$m, soit le domaine visible et infrarouge. Pour le type de température usuellement rencontrées sur Terre, la contribution dans les longueurs d'onde visible est petite par rapport à la contribution dans l'infrarouge -- il faut atteindre des températures de plusieurs centaines de degrés Celsius pour qu'elle devienne significative, comme on peut le constater lorsqu'on porte à haute température un morceau de métal ou que l'on considère une coulée de lave fraîche.
\item La luminance énergétique~$B_{\lambda}$ tend vers 0 
\begin{citemize}
\item aux longueurs d'ondes très courtes, ce qui signifie que le rayonnement thermique comporte extrêmement peu des photons les plus énergétiques;
\item et aux longueurs d'onde très grandes, ce qui est attendu étant donné que l'énergie des photons tend vers~$0$ et que leur nombre n'est pas suffisant pour que la contribution énergétique soit significative.
\end{citemize}
\end{finger}



	\sk \subsection{Lois du corps noir}

		\sk \subsubsection{Loi de Wien : maximum d'émission thermique}

		\sk
On observe sur la figure \ref{fig:BBrad} que, lorsque~$T$ augmente, la maximum de la luminance spectrale~$B_\lambda$, appelé \voc{maximum d'émission}, se décale vers les longueurs d'onde courtes, c'est-à-dire correspond à des photons de plus en plus énergétiques. La loi exacte, appelée \voc{loi de déplacement de Wien}, s'obtient en dérivant $B_\lambda$ par rapport à $\lambda$, ce qui permet d'obtenir $$ \boxed{ \lambda\e{max} \, T = 2.898 \times 10^{-3} \, \textrm{(mètres~K)} } $$ où $\lambda_{max}$ est la longueur d'onde du maximum de luminance spectrale~$B_\lambda$. La longueur d'onde du maximum d'émission~$\lambda\e{max}$ est ainsi inversement proportionnelle à la température du corps émetteur. Une formulation alternative est que $\nu\e{max}$ est proportionelle à $T$.

\figun{0.6}{0.45}{/home/aymeric/Big_Data/BOOKS/pierrehumbert_pics/9780521865562c03_fig001.jpg}{Source~: R. Pierrehumbert, Principles of Planetary Climates, CUP, 2010.}{wvl} 



		\sk \subsubsection{Loi de Stefan-Boltzmann : flux net surfacique}

		\sk
La \voc{loi de Stefan-Boltzmann}\footnote{Joseph Stefan met expérimentalement en évidence en 1879 la dépendance de l'émittance en puissance quatrième de la température. Ludwig Boltzmann, à qui l'on doit également des résultats fondamentaux sur l'entropie et l'atomisme, prouve en 1884 le résultat par des arguments théoriques.} donne la valeur de l'intégrale sur toutes les longueurs d'ondes et dans tout l'espace\footnote{On entend par là toutes les directions du demi-espace extérieur au corps considéré.} de la courbe du corps noir, décrite par la loi de Planck et illustrée par les figures \ref{fig:BBrad} et \ref{fig:BBmax}. Cette loi donne donc l'expression d'une densité de flux énergétique~$F$ ou plus spécifiquement, puisque le corps noir est une source de rayonnement, d'une émittance totale~$M$. Cette dernière s'obtient tout d'abord avec une intégration par rapport à~$\lambda$ de la luminance énergétique spectrale~$B_\lambda$ donnée par la loi de Planck, afin d'obtenir la luminance énergétique~$B$. On déduit ensuite l'émittance totale~$M$ en intégrant selon toutes les directions de l'espace; comme le rayonnement du corps noir est isotrope, $M$ s'obtient à partir de~$B$ simplement en multipliant par~$\pi$. La loi de Stefan-Boltzmann établit que le flux net surfacique~$M$ émis par un corps noir ne dépend que de sa température par une dépendance type loi de puissance $$ \boxed{ M\e{corps noir} = \sigma \, T^4 } $$ avec~$\sigma=5.67 \times 10^{-8} \textrm{~W~m}^{-2}\textrm{~K}^{-4}$ appelée constante de Stefan-Boltzmann. La loi de Stefan-Boltzmann, comme la loi de Planck dont elle dérive, stipule que l'émittance~$M$ d'un corps pouvant être considéré en bonne approximation comme un corps noir ne dépend que de sa température et non de sa nature. Cette loi indique par ailleurs que l'émittance~$M$ augmente très rapidement avec la température -- de par la puissance quatrième impliquée.


	\sk \subsection{Lois des corps gris et émissivité}

		\sk
Le corps noir est un modèle idéal d'absorbant qu'en pratique on ne rencontre pas dans la nature. Par exemple, le charbon noir est un absorbant parfait, mais seulement dans les longueurs d'onde visible. La plupart des objets ressemblent néanmoins au corps noir, au moins à certaines températures et pour certaines longueurs d'onde considérées en pratique. Dans le cas d'un corps qui n'est pas un absorbant parfait, on parle d'un \voc{corps gris}. A température égale, un corps gris n'émet pas autant qu'un corps noir dans les mêmes conditions. Pour évaluer l'énergie émise par un corps gris par comparaison à celle qu'émettrait le corps noir dans les mêmes conditions, on définit un coefficient appelé \voc{émissivité} $\epsilon_\lambda$ compris entre~$0$ et~$1$ et égal au rapport entre la luminance spectrale du corps~$L_\lambda$ et celle du corps noir~$B_\lambda$: $ \epsilon_\lambda=L_\lambda / B_\lambda(T)$ En toute généralité, l'émissivité~$\epsilon_{\lambda}$ d'une surface à une longueur d'onde~$\lambda$ dépend de ses propriétés physico-chimiques, de sa température et de la direction d'émission\footnote{Par exemple, les métaux, matériaux conducteurs de l'électricité, ont une émissivité faible (sauf dans les directions rasantes) qui croît lentement avec la température et décroît avec la longueur d'onde ; au contraire, les diélectriques, matériaux isolant de l'électricité, ont une émissivité élevée qui augmente avec la longueur d'onde et se révèlent lambertiens sauf pour les directions rasantes où l'émissivité décroît significativement.}.

\sk
On peut définir une émissivité totale intégrée~$\epsilon$ qui permet d'exprimer l'émittance~$M$ d'un corps gris $$ \boxed{\SB} $$ Des valeurs de l'émissivité totale~$\epsilon$ pour certains matériaux sont données dans le tableau~\ref{tab:emiss}~: l'eau, la neige, les roches basaltiques ont des émissivités proches de~$1$ et sont donc des corps noirs en bonne approximation. 


\begin{table}[h!]
\label{tab:emiss}
\begin{center}
\footnotesize
\begin{tabular}{||c|c||c|c||c|c||}
\hline
Matériau & Emissivité~$\epsilon$ & Matériau & Emissivité~$\epsilon$ & Matériau & Emissivité~$\epsilon$ \\
\hline
Cuivre poli & 0.03 		& Cuivre oxydé & 0.5 		& Béton & 0.7 à 0.9 	\\
Carbone & 0.8 			& Lave (volcan actif) & 0.8 	& Suie & 0.95		\\
Ville & 0.85 			& Peinture blanche & 0.87 	& Peinture noire & 0.94 \\
Désert & 0.85 à 0.9 		& Herbe & 0.9 à 0.95		& Forêt & 0.95 		\\
Nuages cirrus & 0.10 à 0.90 	& Nuages cumulus & 0.25 à 0.99	& Eau & 0.92 à 0.97  	\\
Neige âgée & 0.8 		& Neige fraîche & 0.99		& &			\\
\hline
\end{tabular}
\normalsize
\caption{\emph{Quelques valeurs usuelles d'émissivité à la température ambiante (pour un rayonnement infrarouge). Source~: Hecht, Physique, 1999 -- avec quelques ajouts d'après site CNES}}
\end{center}
\end{table}



\mk \section{Energie reçue du Soleil}

	\sk \subsection{Caractéristiques et domaine de longueurs d'onde}

		\sk
Le Soleil peut être considéré en bonne approximation comme un corps noir car il absorbe tout le rayonnement incident. Sa \ofg{couleur} est dûe à du rayonnement émis et, plus précisément, correspond aux longueurs d'onde où le maximum de rayonnement est émis. D'après la loi de Wien, le Soleil, dont l'enveloppe externe a une température autour de~$6000$~K, a donc un maximum d'émission situé dans le visible à $\lambda\e{max} = 0.5 \mu$m, proche du maximum de sensibilité de l'oeil humain [figure~\ref{fig:BBmax} haut]. Au contraire, la surface terrestre, dont la température typique est d'environ~$288$~K, voit son maximum d'émission situé dans l'infrarouge vers 10~$\mu$m, alors que le rayonnement émis dans les longueurs d'ondes visible est négligeable [figure~\ref{fig:BBmax} bas]. Un raccourci usuel est donc de dire que \ofg{la Terre émet du rayonnement (thermique) dans l'infrarouge alors que le Soleil émet dans le visible}. En toute rigueur, cette affirmation ne parle que du voisinage du maximum d'émission, où la contribution au flux intégré selon toutes les longueurs d'onde est la plus significative. Il est ainsi plus exact de dire que, dans l'atmosphère, la région du spectre où~$\lambda$ est inférieure à environ 4~$\mu$m est dominée par le rayonnement d'origine solaire, alors qu'au-delà, le rayonnement est surtout d'origine terrestre. Il n’y a pratiquement pas de recouvrement entre la partie utile du spectre du rayonnement solaire et celui d’un corps de température ambiante; ce fait est d'une grande importance pour les phénomènes de type effet de serre, qui sont abordés plus loin dans ce cours. On désigne ainsi souvent le rayonnement d'origine solaire par le terme \voc{ondes courtes} et le rayonnement d'origine terrestre par le terme \voc{ondes longues}.

\figsup{0.65}{0.2}{decouverte/cours_meteo/6000K.jpg}{decouverte/cours_meteo/earth.jpg}{Courbes de luminance spectrale d'un corps noir pour différentes températures correspondant notamment au Soleil (haut) et à la Terre (bas). La quantité représentée ici est l'émittance spectrale~$M_\lambda = \pi \, B_\lambda$. Noter la différence d'indexation de l'abscisse et l'ordonnée sur les deux schémas. Le rayonnement thermique émis par la Terre est plusieurs ordres de grandeur moins énergétique que celui émis par le Soleil et le maximum d'émission se trouve à des longueurs d'onde plus grandes (infrarouge pour la Terre au lieu de visible pour le Soleil). Source : \url{http://hyperphysics.phy-astr.gsu.edu/hbase/bbrc.html}.}{fig:BBmax}


	\sk \subsection{Constante solaire}

		\sk
La distance Soleil-Terre est beaucoup plus grande que les rayons de la Terre et du Soleil. Ainsi, d'une part, le rayonnement solaire arrive au niveau de l'orbite terrestre en faisceaux pratiquement parallèles. D'autre part, la luminance en différents points de la Terre ne varie pas. On peut par conséquent définir une valeur moyenne de la densité de flux énergétique du rayonnement solaire au niveau de l'orbite terrestre, reçue par le système surface~+~atmosphère. Elle est désignée par le terme de \voc{constante solaire} notée~$\mathcal{F}\e{s}$. Les mesures indiquent que
\[ \mathcal{F}\e{s} = 1368 \text{~W~m}^{-2} \qquad \text{pour la Terre} \]

\sk
La constante solaire est une valeur instantanée côté jour~: le rayonnement solaire reçu au sommet de l'atmosphère en un point donné de l'orbite varie en fonction de l'heure de la journée et de la saison considérée (c'est-à-dire la position de la Terre au cours de sa révolution annuelle autour du Soleil)\footnote{En réalité, la constante solaire~$\mathcal{F}\e{s}$ varie elle-même d'environ~$3$~W~m$^{-2}$ en fonction des saisons à cause de l'excentricité de l'orbite terrestre, qui n'est pas exactement circulaire. De plus, elle peut varier évidemment en fonction des cycles solaires, néanmoins sans influence majeure sur la température des basses couches atmosphériques (troposphère et stratosphère).}. On peut donc définir un \voc{éclairement solaire moyen} noté~$\mathcal{F}\e{s}'$ reçu par la Terre qui intègre les effets diurnes et saisonniers. Autrement dit, $\mathcal{F}\e{s}$~est l'éclairement instantané reçu par un satellite en orbite autour de la Terre~; $\mathcal{F}\e{s}'$ est la valeur que l'on obtiendrait si l'on faisait la moyenne d'un grand nombre de mesures instantanées du satellite à diverses heures et saisons. 

\figside{0.5}{0.2}{decouverte/cours_dyn/incoming.png}{Energie reçue du Soleil par le système Terre. Source~: McBride and Gilmour, \emph{An Introduction to the Solar System}, CUP 2004.}{fig:eqrad}

\sk
On admet ici que~$\mathcal{F}\e{s}'$ peut être calculé en considérant que le flux total reçu du Soleil l'est à travers un disque de rayon le rayon~$R$ de la Terre (il s'agit de l'ombre projetée de la planète, voir Figure~\ref{fig:eqrad}). A cause de l'incidence parallèle, le flux énergétique intercepté par la Terre vaut donc~$\Phi = \pi \, R^2 \, \mathcal{F}\e{s}$. L'éclairement moyen à la surface de la Terre est alors $$\mathcal{F}\e{s}' = \frac{\Phi}{4 \, \pi \, R^2}$$ le dénominateur étant l'aire de la surface complète de la Terre. On obtient ainsi
\[ \boxed{ \mathcal{F}\e{s}' = \frac{\mathcal{F}\e{s}}{4} } \]

		\sk
La valeur de la constante solaire peut s'obtenir par le calcul. Le soleil est considéré en bonne approximation comme un corps noir de température~$T_{\sun} = 5780$~K. D'après la loi de Stefan-Boltzmann, son émittance est $M = \sigma \, T_{\sun}^4$ donc le flux énergétique~$\Phi_{\sun}$ émis par le Soleil de rayon~$R_{\sun} = 7 \times 10^5$~km est~$\Phi_{\sun} = 4 \, \pi \, R_{\sun}^2 \, \sigma \, T_{\sun}^4$. Ce flux énergétique est rayonné dans tout l'espace~: à une distance~$d$ du soleil il est réparti sur une sphère de centre le soleil et de rayon~$d$, donc de surface~$4 \, \pi \, d^2$. A cette distance, l'éclairement~$\mathcal{F}$, c'est-à-dire la densité de flux énergétique reçue en W~m$^{-2}$, s'écrit donc
\[ \mathcal{F} = \frac{\Phi_{\sun}}{4 \, \pi \, d^2} = \frac{4 \, \pi \, R_{\sun}^2 \, \sigma \, T_{\sun}^4}{4 \, \pi \, d^2} = \sigma \, T_{\sun}^4 \, \left( \frac{R_{\sun}}{d} \right)^2 \]
Si l'on prend~$d$ égal à la distance Terre-Soleil, $\mathcal{F}$ définit ainsi la constante solaire~$\mathcal{F}\e{s}$.
%\[ \mathcal{F}\e{s} = \frac{{\mathcal{F}\e{s}}^{\text{Terre}}}{d\e{soleil}^2} \]

%Variation de la constante solaire : Bien que l’intensité du soleil ait subit des variations depuis la formation de la Terre, on peut s’attendre à ce qu’elle soit stable sur une période de 1000 ans. On mesure mal la constante solaire, mais les mesures récentes, même avec leurs incertitudes, semblent indiquer que le soleil ne peut pas expliquer le réchauffement récent. Notons toutefois que les simulations actuelles ne tiennent pas compte des fluctuations possibles du rayonnement solaire (négligeable a priori).
%%%% pas sûr du dernier point.



%%% CHAPITRE 3
%\def\thechapter{3} \title{Sciences de l'atmosphère\\ Chapitre \thechapter} \author{Aymeric SPIGA\\ ~\\ \small Modifié et augmenté à partir des notes de Francis CODRON et Sébastien PAYAN.} \maketitle %% commenter la ligne en mode livre complet
%\chapter{Transfert radiatif}

\dictum[Alberto Giacometti, 1949]{Le ciel n’est bleu que par convention, mais rouge en réalité.}

\bk Lorsqu'il traverse l'atmosphère terrestre, la totalité du rayonnement thermique émis par le Soleil n'est pas transmise à la surface. On appelle \voc{transfert radiatif} les phénomènes qui régissent l'interaction entre rayonnement incident et matière. Dans ce chapitre, on aborde des éléments de éléments de transfert radiatif qui sont nécessaires pour effectuer le bilan énergétique de l'atmosphère et du système climatique.

\mk \section{Réflexion, absorption, transmission}

	\sk \subsection{Coefficients d'interaction et lois de Kirchhoff}

		\sk
Tout rayonnement se propageant dans un milieu matériel subit trois phénomènes~: réflexion, absorption, transmission. Autrement dit, tout corps cible irradié par une source voit le flux énergétique incident spectral~$\Phi_{\lambda}$ se répartir selon trois termes
\begin{citemize}
\item une partie~$\Phi_{\lambda}^r$ du flux incident est réfléchie ou diffusée;
\item une partie~$\Phi_{\lambda}^t$ du flux incident traverse le corps sans interactions;
\item une partie~$\Phi_{\lambda}^a$ du flux incident est absorbée, c'est-à-dire transformée en énergie interne.
\end{citemize}
Afin de définir les contributions respectives de ces trois phénomènes, on définit des coefficients spectraux de \voc{réflexion}~$\rho_{\lambda}$, de \voc{transmission}~$\tau_{\lambda}$, d'\voc{absorption}~$\alpha_{\lambda}$ compris entre~$0$ et~$1$
$$\Phi_{\lambda}^r = \rho_{\lambda} \, \Phi_{\lambda} \qquad\qquad \Phi_{\lambda}^t = \tau_{\lambda} \, \Phi_{\lambda} \qquad\qquad \Phi_{\lambda}^a = \alpha_{\lambda} \, \Phi_{\lambda}$$
Ces coefficients sont également appelés \voc{réflectivité}~$\rho_{\lambda}$, \voc{transmittivité}~$\tau_{\lambda}$, \voc{absorptivité}~$\alpha_{\lambda}$. Ils dépendent de la longueur d'onde~$\lambda$ du rayonnement incident, de l'angle d'incidence et des propriétés physiques et chimiques du corps récepteur (par exemple, température, composition). Lorsque le coefficient de réflexion~$\rho_{\lambda}$ ne dépend pas de l'angle d'incidence\footnote{L'énergie incidente à une surface pénètre dans celle-ci et est réfléchie aléatoirement à l'intérieur de l'objet par de microscopiques in-homogénéités du matériau. Au cours de ces multiples réflexions une partie de l'énergie incidente ressort de l'objet suivant une direction aléatoire. Bien souvent les réflexions multiples dans le matériau ne subissent aucune contrainte particulière, l'énergie est donc réfléchie de façon uniforme et isotrope par la surface. Le flux réfléchi est alors uniquement fonction de la quantité d'énergie incidente tombant sur la surface, qui s'exprime souvent simplement comme un cosinus de l'angle entre la normale à la surface et la direction de la source.}, on parle de l'objet cible comme d'un \voc{réflecteur lambertien}. 

%\sk
%\subsection{Rappel sur la distinction entre grandeurs intégrées et spectrales}
%\sk
%Dans le chapitre précédent, les grandeurs caractéristiques~$\Phi$,~$F$ ou~$L$, ainsi que les coefficients d'absorption~$\rho$, de transmission~$\tau$, de réflexion~$\rho$ (également appelé albédo~$A$), ont été décrits 
%\begin{citemize}
%\item soit d'une façon intégrée selon toutes les longueurs d'onde (par exemple l'émittance~$M$ dans la loi de Stefan-Boltzmann) ;
%\item soit en prenant en compte la dépendance spectrale, c'est-à-dire en considérant un petit intervalle~$\dd \lambda$ autour d'une longueur d'onde~$\lambda$ donnée (par exemple la luminance énergétique spectrale~$B_{\lambda}$ dans la loi de Planck). 
%\end{citemize}
%Dans le premier cas, on emploie simplement les symboles décrivant les grandeurs (exemple, luminance~$L$). Dans le second cas, on ajoute $\lambda$ en indice de ces symboles (exemple, luminance spectrale~$L_\lambda$). Ce qui est clair pour les variables l'est beaucoup moins dans le vocabulaire couramment utilisé, y compris dans certains ouvrages.


		\sk
La loi de Kirchhoff traduit la conservation du flux incident~$\Phi_{\lambda} = \Phi_{\lambda}^r + \Phi_{\lambda}^t + \Phi_{\lambda}^a$ par une relation entre les coefficients spectraux de réflexion, transmission et absorption $$ \boxed{ \rho_{\lambda} + \tau_{\lambda} + \alpha_{\lambda} = 1 } $$ On peut également considérer le flux énergétique incident~$\Phi$ intégré selon toutes les longueurs d'onde. Les coefficients de réflexion~$\rho$, de transmission~$\tau$, d'absorption~$\alpha$ peuvent alors être définis par $$\Phi^r = \rho \, \Phi \qquad\qquad \Phi^t = \tau \, \Phi \qquad\qquad \Phi^a = \alpha \, \Phi $$ et la loi de Kirchhoff s'écrit alors $$ \boxed{ \rho + \tau + \alpha = 1 } $$ On note par ailleurs que les définitions des coefficients impliquent que $\rho \ne \int_{\lambda} \rho_\lambda \, \dd\lambda$


	\sk \subsection{Vocabulaire et mise en garde}

		\sk
Un certain nombre de termes sont couramment utilisés pour désigner des situations particulières en terme d'absorptivité, de transmittivité, de réflectivité spectrale. Il est important de préciser à quelle longueur d'onde~$\lambda$ on se place lorsqu'on qualifie les propriétés d'un corps matériel (ce n'est pas toujours clair dans les ouvrages).
\begin{citemize}
\item Un corps \voc{transparent} à la longueur d'onde~$\lambda$ est tel que~: $\tau_\lambda = 1$, donc~$\rho_\lambda = \alpha_\lambda = 0$
\item Un corps \voc{opaque} à la longueur d'onde~$\lambda$ est tel que~: $\tau_\lambda = 0$, donc~$\rho_\lambda + \alpha_\lambda = 1$
\item Un corps \voc{brillant} à la longueur d'onde~$\lambda$ est tel que~: $\rho_\lambda = 1$, donc~$\tau_\lambda = \alpha_\lambda = 0$
\item Un corps \voc{sombre} à la longueur d'onde~$\lambda$ est tel que~: $\rho_\lambda = 0$, donc~$\tau_\lambda + \alpha_\lambda = 1$
\end{citemize}
Le corps noir étant un absorbant idéal, son coefficient spectral d'absorption~$\alpha_\lambda$ vaut 1 pour toutes les longueurs d'onde~$\lambda$. Il vérifie donc également~$\rho_\lambda = \tau_\lambda = 0$ pour toutes les longueurs d'onde. Par extension, un corps est qualifié de \voc{presque noir} à la longueur d'onde~$\lambda$ si~$\alpha_\lambda = 1$ à la longueur d'onde~$\lambda$. Un corps gris tel que défini au chapitre précédent est tel que son absorptivité~$\alpha_\lambda$ est la même selon toutes les longueurs d'onde.

\sk
On n'utilise le terme de~\ofg{complètement} noir, opaque, brillant, \ldots~que lorsque la propriété est vérifiée pour toutes les longueurs d'onde du spectre électromagnétique. Ce n'est en fait quasiment jamais le cas en pratique. La neige fraîche en est un excellent exemple~: elle apparaît brillante dans le visible car elle réfléchit le rayonnement incident ($\rho_{\lambda\e{VIS}}=1$). Il serait erroné de croire que c'est le cas dans toutes les longueurs d'ondes. La neige fraîche est presque noire dans l'infrarouge où elle est très absorbante ($\alpha_{\lambda\e{IR}}=1$) donc elle est sombre dans l'infrarouge ($\rho_{\lambda\e{IR}}=0$). Autre exemple, le verre est transparent pour les longueurs d'onde visible ($\tau_\lambda = 1$), mais relativement opaque pour le rayonnement infrarouge ($\tau_\lambda \rightarrow 1$).
%une surface \ofg{brillante} dans le visible ($A_{\lambda}$ proche de~$1$) peut s'avérer \ofg{sombre} dans l'infrarouge ($A_{\lambda} \ll 1$)


	\sk \subsection{Albédo}

		\sk
En sciences de l'atmosphère, les coefficients de réflexion~$\rho$ et~$\rho_{\lambda}$ sont souvent désignés sous le nom respectivement d'\voc{albédo} noté~$A$ et d'albédo spectral noté~$A_{\lambda}$. Plus la surface réfléchit une grande partie du rayonnement électromagnétique incident, plus l'albédo est proche de~$1$. L'albédo spectral~$A_{\lambda}$ peut varier significativement en fonction de la longueur d'onde : voir l'exemple de la neige fraîche donné ci-dessus. 

\sk
De par la diversité des surfaces terrestres, et de la variabilité de la couverture nuageuse, les valeurs de l'albédo~$A$ varient fortement d'un point à l'autre du globe terrestre~: il est élevé pour de la neige fraîche et faible pour de la végétation et des roches sombres [table~\ref{tab:albedo}]. L'albédo de l'océan est faible, particulièrement pour des angles d'incidence rasants -- il dépend ainsi beaucoup de la distribution des vagues. 

\begin{table}\label{tab:albedo}
\begin{center}
\begin{tabular}{|c|c|c|c|}
\hline
Type & albédo~$A$ & Type & albédo~$A$ \\
\hline
Surface de lac & 0.02 à 0.04 & Surface de la mer & 0.05 à 0.15 \\
Asphalte & 0.07 & Mer calme (soleil au zenith) & 0.10 \\
Forêt équatoriale & 0.10 & Roches sombres, humus & 0.10 à 0.15 \\
Ville & 0.10 à 0.30 & Forêt de conifères & 0.12 \\
Cultures & 0.15 à 0.25 & Végétation basse, verte & 0.17 \\
Béton & 0.20 & Sable mouillé & 0.25 \\
Végétation sèche & 0.25 & Sable léger et sec & 0.25 à 0.45 \\
Forêt avec neige au sol & 0.25 & Glace & 0.30 à 0.40 \\
Neige tassée & 0.40 à 0.70 & Sommet de certains nuages & 0.70 \\
Neige fraîche & 0.75 à 0.95 & & \\
\hline
\end{tabular}
\caption{\emph{Quelques valeurs usuelles d'albédo (rayonnement visible). D'après mesures missions NASA et ESA.}}
\end{center}
\end{table}

\sk
L'\voc{albédo planétaire} est noté~$A\e{b}$ et défini comme la fraction moyenne de l'éclairement~$E$ au sommet de l'atmosphère (noté également~$\mathcal{F}\e{s}'$) qui est réfléchie vers l'espace~: il comprend donc la contribution des surfaces continentales, de l'océan et de l'atmosphère. Il vaut~$0.31$ pour la planète Terre~: une partie significative du rayonnement reçu du Soleil par la Terre est réfléchie vers l'espace\footnote{L'albédo planétaire est par exemple encore plus élevé sur Vénus ($0.75$) à cause de la couverture nuageuse permanente et très réfléchissante de cette planète.}. Ainsi le système Terre reçoit une densité de flux énergétique moyenne~$F\e{reçu}$ en W~m$^{-2}$ telle que
\[ F\e{reçu} = (1-A\e{b}) \, \mathcal{F}\e{s}' \] 
donc un flux énergétique~$\Phi\e{reçu}$ (en W) qui s'exprime
\[ \Phi\e{reçu} = \pi \, R^2 \, (1-A\e{b}) \, \mathcal{F}\e{s} \]
%L'albédo de Bond~ désigne l'albédo intégré sur toutes les longueurs d'onde et tous les angles d'incidence.

\sk
La valeur de~$30\%$ de l'albédo planétaire sur Terre est en fait majoritairement dû à l'atmosphère~:  seuls 4\% de l'énergie solaire incidente sont réfléchis par la surface terrestre comme indiqué sur la figure~\ref{fig:diffsep}. L'énergie réfléchie par l'atmosphère vers l'espace, responsable de plus de~$85\%$ de l'albedo planétaire, est diffusée par les molécules ou par des particules en suspension, gouttelettes nuageuses, gouttes de pluie ou aérosols.

\figside{0.4}{0.15}{\figpayan/LP211_Chap2_Page_27_Image_0001.png}{L'énergie solaire incidente est réfléchie vers l'espace par la surface et l'atmosphère d'une planète. La figure montre les différentes contributions à l'albédo planétaire total.}{fig:diffsep}


	\sk \subsection{Remarque sur l'émissivité}

		\sk
Il existe une seconde loi de Kirchhoff, différente de celle précitée, qui stipule que l'émissivité spectrale doit être égale au coefficient d'absorption du corps $$ \epsilon_\lambda = \alpha_\lambda $$ pour des quantités intégrées selon toutes les directions de l'espace. Un corps ne peut émettre que les radiations qu'il est capable d'absorber. En d'autres termes, pour une température et une longueur d'onde donnée, un bon émetteur est souvent un bon absorbant (et vice versa). On retrouve par ce principe que le corps noir est le corps idéal qui rayonne un maximum d'énergie radiative à chaque température et pour chaque longueur d'onde. 
%Un corollaire est qu'un corps transparent ou réfléchissant à une certaine longueur d'onde émet peu de rayonnement thermique à cette même longueur d'onde. NON CAR IL SUFFIT QUE LA TEMPERATURE SOIT ELEVEE !



\mk \section{Bases pour aborder l'interaction entre le rayonnement électromagnétique et l'atmosphère}

	\sk \subsection{Généralités}

		\sk
Le rayonnement électromagnétique, lors de sa traversée de l'atmosphère terrestre, est perturbé par deux processus que l'on peut analyser séparément~: un processus d'\voc{absorption} par certains gaz atmosphériques (O$_2$, H$_2$O, O$_3$, etc.) et un processus de \voc{diffusion} par les molécules et les aérosols (poussières, cristaux de glace, gouttes nuageuses et gouttes de pluie). 
\begin{finger}
\item Dans le processus d'absorption, un certain nombre de photons disparaissent, une partie du rayonnement incident est convertie en énergie interne, et il y a une extinction du signal dans la direction de propagation. 
\item Au contraire, dans le processus de diffusion, les photons sont simplement redistribués dans toutes les directions avec une certaine probabilité définie par ce qu'on appelle la fonction de phase de diffusion~; on peut alors observer une extinction dans certaines directions et une augmentation dans d'autres. Lors de la diffusion, il n'y a pas de changement de longueur d'onde de l'onde incidente et de l'onde diffusée.
\end{finger}
%Ainsi la réflexion peut être \voc{diffuse} (dans toutes les directions), \voc{spéculaire} (dans la direction symétrique du rayonnement incident) ou quelconque. 
%Les deux effets peuvent être analysés séparément.
Autrement dit, on s'intéresse à l'\voc{extinction} progressive du rayonnement incident par absorption et diffusion. Deux phénomènes très importants sont mis de côté~: l'émission de rayonnement thermique, déjà abordée au chapitre précédent, et la diffusion multiple, que l'on néglige (rayonnement diffusé qui viendrait depuis d'autres directions).


	\sk \subsection{Section efficace}\label{sec:efficace}

		\sk
Que l'on s'intéresse à la diffusion ou l'absorption, l’interaction entre le rayonnement et la matière dépend de la rencontre entre les photons et les éléments (atomes, molécules, particules) du milieu considéré. La diffusion et l'absorption dépendent donc de la probabilité que le rayonnement, c'est-à-dire les photons qui le constituent, rencontre les éléments constitutifs de la matière. 

\sk
De façon évidente, cette probabilité est liée 
\begin{citemize}
\item au flux de photons~: la probabilité est plus élevée s'il y a un plus grand nombre de photons incidents, ou de façon équivalente une plus grande énergie radiative incidente ;
\item au nombre d'éléments (atomes, molécules, particules) dans le milieu matériel~: la probabilité augmente avec le nombre d'éléments, autrement dit dans un milieu plus dense, le rayonnement aura une plus forte probabilité de rencontrer des éléments matériels avec lesquels interagir. 
\end{citemize}
Cependant, même si ces deux quantités sont élevées, la probabilité peut rester faible, car elle dépend également d'un paramètre qui traduit l'efficacité de la rencontre entre un photon de longueur d'onde donnée~$\lambda$ et l'espèce absorbante. Cette grandeur est appelée \voc{section efficace} et est décrite ci-dessous.

\figun{0.5}{0.2}{\figfrancis/Beer_upside}{Variation du rayonnement incident avec un angle $\theta$ sur une couche d'épaisseur $dz$}{fig:beer}

\sk
On considère la situation décrite dans la figure~\ref{fig:beer}. Soit une tranche d'atmosphère horizontale\footnote{On dit qu'on fait l'approximation plan-parallèle car on néglige la courbure de la Terre ainsi que les variations horizontales des paramètres géophysiques (température et profils de gaz).} d'épaisseur élémentaire~$\dd z$ qui reçoit un rayonnement monochromatique de longueur d'onde~$\lambda$ caractérisé par sa luminance énergétique spectrale~$L_\lambda$. Le rayonnement incident traverse la tranche d'atmosphère en faisant un angle~$\theta$ par rapport à la verticale. La distance parcourue par le rayonnement à travers la fine couche d'épaisseur~$\dd z$ vaut\footnote{On parle d'abscisse curviligne pour qualifier~$s$.} 
\[ \dd s = \frac{1}{\cos\theta} \, \dd z \]
%%%% OK avec Beer_upside
autrement dit l'inclinaison du rayonnement impose un chemin optique plus grand. A la sortie de la tranche d'atmosphère, le rayonnement a subi une extinction à cause des phénomènes de diffusion et absorption dans la tranche d'atmosphère. On caractérise alors l'extinction causée par la diffusion et l'absorption par une quantité appelée section efficace~$\Sigma_\lambda$ qui a la dimension d'une surface, exprimée en m$^2$. A la sortie de la tranche d'atmosphère, la luminance spectrale est~$L_\lambda + \dd L_\lambda$ avec 
\[ \dd L_\lambda = - L_\lambda(s) \, \Sigma_\lambda(s) \, N \, \dd s \qquad \textrm{ou de manière équivalente} \qquad \boxed{ \dd L_\lambda = - L_\lambda(z) \, \Sigma_\lambda(z) \, N \, \frac{1}{\cos\theta} \, \dd z } \]
où $N$ est le nombre de particules par unité de volume. La section efficace~$\Sigma_{\lambda}$ prend en compte l'extinction causée par les phénomènes d'absorption et de diffusion~: plus la section efficace est grande, plus l'extinction du rayonnement incident est élevée. Il est possible de séparer les deux contributions en définissant une section efficace d'absorption~$\Sigma_\lambda^a$ et une section efficace de diffusion/réflexion~$\Sigma_\lambda^r$ telles que~$\Sigma_{\lambda} = \Sigma_\lambda^a + \Sigma_\lambda^r$. La section efficace dépend de la longueur d'onde et de la nature physico-chimique du milieu absorbant~: par exemple, la composition de l'air dans le cas de l'atmosphère, ou la température dans le cas de la majorité des matériaux. Ainsi, en toute généralité, elle n'a pas de raison particulière de rester constante pour les différentes parties du milieu matériel traversé.

%Ce n'est pas le but de ce cours que de proposer une vision complète du transfert radiatif, mais le lecteur intéressé peut noter que l'équation complète du transfert radiatif dans l'atmosphère prend la forme indiquée précédemment avec cependant l'ajout des deux termes \ofg{sources} précités \[ \dd L_\lambda = - L_\lambda(s) \, \Sigma_\lambda(s) \, N \, \dd s + \textrm{émission thermique} + \textrm{diffusion multiple} \]



\mk \section{Diffusion}

	\sk
La diffusion est un phénomène macroscopique résultant de la réflexion, de la réfraction et de la diffraction du rayonnement incident, qui se produisent au niveau microscopique en raison des inhomogénéités du milieu matériel traversé. Par abus de langage, auquel les présentes notes n'échappent pas, on identifie souvent réflexion et diffusion. 

\sk
On distingue différents mécanismes de diffusion selon la taille relative des cibles (molécules ou particules) par rapport à la longueur d'onde du rayonnement électromagnétique incident [figure \ref{fig:diffsize}]. Les connaître permet de comprendre certains phénomènes atmosphériques perçus au quotidien [figure~\ref{fig:ciel}]. Les radiations solaires situées dans l'ultraviolet sont absorbées dans la haute atmosphère (notamment par l'ozone dans la stratosphère) si bien que l'on considère principalement les radiations visibles.
\begin{finger}
\item \underline{Taille des cibles petite devant la longueur d'onde du rayonnement incident} La \voc{diffusion Rayleigh} est la diffusion par les molécules\footnote{ou par des particules significativement petites devant la longueur d'onde, mais ce cas de figure est relativement rare en pratique} qui constituent l'atmosphère. La diffusion Rayleigh dépend fortement de la longueur d’onde incidente. Lord Rayleigh à démontré en 1873 que cette dépendance s’exprimait selon l'inverse de la puissance quatrième de la longueur d'onde
\[ \boxed{ \Sigma_\lambda^r \propto \lambda^{-4} } \]
\begin{citemize}
\item Cette dépendance en longueur d'onde a un effet très notable sur le rayonnement thermique reçu du Soleil, dont nous avons vu au précédent chapitre qu'il est maximum dans les longueurs d'onde visibles. Les molécules d'air de l'atmosphère diffusent plus les photons de courte longueur d'onde à cause de la loi en puissance quatrième de~$\lambda$~: ainsi, le violet et le bleu sont~$16$ fois plus diffusés que le rouge par le mécanisme de Rayleigh. C'est pour cette raison que l'on voit le ciel bleu depuis la surface~: il s'agit de la couleur émanant du rayonnement solaire incident diffusé en majorité par le mécanisme de Rayleigh\footnote{Le fait qu'on ne voit pas le ciel violet est dû à une moindre sensibilité de l'oeil à ces longueurs d'onde, ainsi qu'un moindre flux incident que dans le bleu d'après le spectre solaire.}. La figure~\ref{fig:diffsep} nous indique que~$6\%$ du rayonnement incident sont ainsi diffusés, soit une contribution d'environ~$20\%$ à l'albédo planétaire. 
\item La diffusion Rayleigh ne montre pas de direction préférentielle significative, à part une tendance légèrement supérieure à la diffusion vers l'arrière (rétrodiffusion) et vers l'avant [figure~\ref{fig:diffdir}]. Ceci explique que le ciel apparaisse bleu qu'on le regarde depuis la surface ou depuis un avion. Ceci explique également que la diffusion Rayleigh fasse apparaître le Soleil de la couleur la moins diffusée, à savoir jaune à rouge suivant l'importance de la diffusion, alors qu'il apparaîtrait blanc sans diffusion.
\item Au lever et au coucher du Soleil, lorsque la lumière solaire traverse une couche importante d'atmosphère, la diffusion Rayleigh est plus grande qu'en journée lorsque le Soleil est proche du zénith. La raison est purement géométrique, comme l'on peut s'en convaincre d'après la section~\ref{sec:efficace}~: si l'angle d'incidence~$\theta$ est plus grand, le facteur~$1/\cos\theta$ est plus grand, donc, pour une même section efficace de diffusion~$\Sigma_\lambda^r$, la variation du flux incident~$dL_\lambda$ est plus grande. Ainsi, le soleil est vu rouge le soir car plus de rayonnement incident dans les longueurs d'onde bleues est diffusé qu'en journée.
\item Au contraire du rayonnement thermique solaire, dont le maximum d'émission est dans le domaine visible, l'effet de la diffusion Rayleigh sur le rayonnement thermique émis par la Terre est négligeable, car ce dernier est situé dans l'infrarouge à des longueurs d'onde~$\lambda$ plus élevées pour lesquelles~$\Sigma_\lambda^r \sim 0$. 
\end{citemize}

\figside{0.6}{0.3}{\figfrancis/WH_diff_size}{Type de mécanisme de diffusion dominant en fonction de la longueur d'onde (en abscisse) et de la taille des particules (en ordonnée, l'unité est en $\mu$m). Figure adaptée de Wallace and Hobbs, Atmospheric Science, 2006.}{fig:diffsize}

\figsup{0.48}{0.2}{decouverte/cours_meteo/nuage_ciel.jpg}{decouverte/cours_meteo/ciel_rouge.jpg}{Ciel bleu et nuage blanc. Coucher de soleil rouge. Crédits photos: \url{http://www.meteofrance.com} et \url{http://www.exworld.fr}}{fig:ciel}

\item \underline{Taille des cibles grande devant la longueur d'onde du rayonnement incident} La diffusion par les particules les plus grosses, par exemple les gouttes de brume de quelques centaines de microns, les gouttes de pluie de l'ordre du mm, les cristaux de glace de quelques dizaines de microns, ou les poussières les plus grosses, peut être expliquée par les lois de l'\voc{optique géométrique}, les lois qui gouvernent le fonctionnement des lentilles convergentes/divergentes. Contrairement à la diffusion de Rayleigh, la diffusion est non sélective, c'est-à-dire qu'elle ne dépend pas de la longueur d'onde. Les gouttes d'eau de l'atmosphère diffusent toutes les longueurs d'onde de façon quasiment équivalente, ce qui produit un rayonnement blanc. Ceci explique pourquoi le brouillard et les nuages nous paraissent blancs. La réalité d'un nuage est parfois plus complexe~: ses propriétés radiatives dépendent de la taille des particules et leur nombre par unité de volume.
\item \underline{Taille des cibles grande devant la longueur d'onde du rayonnement incident} La diffusion par les particules de taille intermédiaire, par exemple les gouttes nuageuses ou les aérosols de plus petite taille (quelques microns), est plus délicate à étudier que les deux mécanismes précédemment cités. On parle de \voc{diffusion de Mie}. Les particules soulevées pendant une tempête de poussière sur Terre ou sur Mars causent par diffusion de Mie une couleur orangée au ciel. Suivant la taille et la nature de la particule interagissant avec le rayonnement, la diffusion de Mie peut avoir des caractéristiques très directionnelles [figure \ref{fig:diffdir}]. La section efficace~$\Sigma_\lambda^r$ de la diffusion de Mie suit une loi en l'inverse de~$\lambda^2$ avec la longueur d'onde~$\lambda$ du rayonnement incident. Ces variations sont donc moins sensibles à la longueur d'onde que dans le cas de la diffusion de Rayleigh.
\end{finger}

\figside{0.6}{0.25}{\figfrancis/WH_diff_dir}{Répartition de la probabilité de diffusion dans différentes directions, pour différents types de diffusion: (a) Rayleigh, (b) et (c) Mie avec une particule plus grande en (c).}{fig:diffdir}




\mk \section{Absorption} Au cours de leur pénétration dans l'atmosphère, les photons entrent en collision avec les molécules gazeuses et sont progressivement absorbés. L'étude de ces processus peut être extrêmement complexe, on donne donc ici uniquement les aspects les plus élémentaires pour comprendre les phénomènes en jeu.

	\sk \subsection{Aspect macroscopique~: loi de Beer-Lambert-Bouguer}
	
		\sk
L’étude de l’absorption du rayonnement par un milieu repose sur deux relations établies au XVIIIe siècle qui comparent la luminance spectrale~$L_\lambda(0)$ d'un faisceau incident monochromatique pénétrant sous incidence normale dans un milieu matériel absorbant, à la luminance~$L_\lambda(\ell)$ après traversée du milieu de longueur~$\ell$.
\begin{citemize}
\item D'une part, le faisceau incident subit une extinction telle que le rapport~$\log \frac{L_\lambda(\ell)}{L_\lambda(0)}$ est proportionnel à la longueur~$\ell$ parcourue par le rayonnement dans le milieu absorbant (relation de Bouguer-Lambert).  
\item D'autre part, dans un milieu de concentration molaire effective~$[X]$ en espèce absorbante X, le faisceau incident subit une extinction telle que le rapport~$\log \frac{L_\lambda(\ell)}{L_\lambda(0)}$ est proportionnel à la concentration~$[X]$ (relation de Beer). Le facteur de proportionnalité dépend de l'épaisseur du milieu traversé, de sa nature, de sa composition, de la température et de la longueur d'onde du rayonnement incident.
\end{citemize}
Le logarithme indique que les variations relatives d'énergie radiative au cours de la propagation dans le milieu sont proportionnelles à la longueur parcourue et à la concentration d'espèces absorbantes.

\sk
Les deux lois historiques décrites ci-dessus décrivent exactement la situation de la section~\ref{sec:efficace} si l'on néglige la diffusion. Par rapport à la section~\ref{sec:efficace}, la seule hypothèse supplémentaire est que la section efficace d'absorption~$\Sigma_\lambda^a$ ne dépend que de la longueur d'onde et du matériau, elle est considérée comme uniforme sur toute la longueur traversée. Si l'on reprend la situation de la figure~\ref{fig:beer} avec cette hypothèse, on a 
\[ \dd L_\lambda = - L_\lambda(z) \, \Sigma_\lambda^a \, N_X \, \frac{1}{\cos\theta} \, \dd z \]
où $N_X$ est la concentration de l'espèce absorbante exprimée sous la forme d'un nombre de molécule par unité de volume (au lieu de $[X]$ en mol~L$^{-1}$).
La variation de flux au cours de la traversée du milieu matériel est donc
\[ \frac{\dd L_\lambda}{L_\lambda} = - \Sigma_\lambda^a \, N_X \, \frac{1}{\cos\theta} \, \dd z \]
Par intégration, si l'on suppose que~$z=0$ repère l'entrée dans le milieu matériel et~$z=\ell$ la sortie, on obtient
\[ \int_{z=0}^{z=\ell} \, \frac{\dd L_\lambda}{L_\lambda} = - \Sigma_\lambda^a \, N_X \, \frac{1}{\cos\theta} \, \int_{z=0}^{z=\ell} \dd z \]
d'où la valeur de la luminance spectrale à la sortie du milieu traversé, donnée par deux relations équivalentes
\[ \boxed{ \log \frac{L_\lambda(\ell)}{L_\lambda(0)} = - \zeta \, \ell } \qquad \textrm{et} \qquad \boxed{ L_\lambda(\ell) = L_\lambda(0) \, e^{- \zeta \, \ell} } \qquad \textrm{avec la constante en m}^{-1} \qquad \zeta = \Sigma_\lambda^a \, N_X \, \frac{1}{\cos\theta} \]
Cette loi reprend les résultats historiques présentés précédemment et porte le nom de \voc{loi de Beer-Lambert-Bouguer}. Elle indique la décroissance exponentielle du flux incident lors de sa traversée du milieu, d'autant plus importante que la longueur traversée~$\ell$ est grande et que la concentration en absorbant~$N_X$ du milieu est grande. La loi de Beer-Lambert-Bouguer peut également être appliquée sous la même forme avec l'éclairement. Le coefficient~$\zeta$ porte parfois le nom de coefficient d'extinction linéique. 
%Nous ferons l’hypothèse par la suite que cette dépendance est linéaire, ce qui est vrai pour l’air.
On donne quelques ordres de grandeur ci-dessous pour la valeur de~$\zeta$
\begin{citemize}
\item atmosphère pour un rayonnement visible dans le jaune $\zeta = 1 \times 10^{-5}$~m$^{-1}$ 
\item atmosphère pour un rayonnement visible dans le violet $\zeta = 4 \times 10^{-5}$~m$^{-1}$
\item verre $\zeta = 0.2$~m$^{-1}$
\item nuage bas $\zeta = 1 \times 10^{-3}$~m$^{-1}$
\end{citemize}
En remarquant que le rapport~$L_\lambda(\ell) / L_\lambda(0)$ définit justement la coefficient de transmission spectral~$\tau_\lambda$ \emph{en l'absence de diffusion}, on arrive à
\[ \tau_\lambda = e^{- \zeta \, \ell} \qquad \textrm{et} \qquad \alpha_\lambda = 1 - e^{- \zeta \, \ell} \]
On peut vérifier que l'expression est conforme à l'intuition : si la longueur de la traversée est particulièrement grande ($\ell \rightarrow \infty$), et/ou que l'espèce absorbante~X est très concentrée ($N_X \rightarrow \infty$), alors presque tout le rayonnement incident est absorbé~$\alpha_\lambda \rightarrow 1$ et une partie négligeable de ce rayonnement est transmise~$\tau_\lambda \rightarrow 0$. 

\sk
En sciences de l'atmosphère, on utilise souvent une forme plus générale de la loi de Beer-Lambert-Bouguer. On écrit la relation intermédiaire (non intégrée) qui conduit à cette loi sous la forme 
\[ \frac{\dd L_\lambda}{L_\lambda} = \, \rho_X \, k_\lambda \, \frac{1}{\cos\theta}  dz \]
où $k_\lambda$ est un coefficient d'absorption massique en m$^2$~kg$^{-1}$ et $\rho_X(z)$ est la densité d'absorbant~X, qui dépend de~$z$ comme ce peut être le cas dans l'atmosphère. Cette relation peut être intégrée sur une couche épaisse située entre les niveaux~$z_1$ et~$z_2$. On obtient 
\[ L_\lambda(z_1) = L_\lambda(z_2) \, e^{- \frac{t_\lambda}{\cos\theta}} \]
où 
\[ t_\lambda = \int_{z_1}^{z_2} \, k_\lambda \, \rho_X \, \dd z \]
est appelée l'épaisseur optique de la couche. Si l'extinction est uniquement due à de l'absorption, sans diffusion, on a une relation directe entre l'épaisseur optique et le coefficient d'absorption de la couche: 
\[\alpha_\lambda = 1 - e^{- \frac{t_\lambda}{\cos\theta}} \]
%Dans le cas particulier où la densité d'absorbant est de la forme
%\[\rho_a=\rho_a^0e^{-z/H_a}\]
%ce qui est le cas par exemple d'un gaz bien mélangé dans l'atmosphère, ou de
%la vapeur d'eau, on peut calculer l'altitude du taux d'extinction
%$dL_\lambda/dz$ maximum: on a alors également d'après la définition de
%$\tau_\lambda$
%\[\tau_\lambda=\tau_\lambda^0e^{-z/H_a}\]
%et $d\tau_\lambda/dz=-\tau_\lambda/H_a$. D'autre part, le taux d'extinction vaut 
%\[dL_\lambda/dz=-L_\lambda\mu d\tau_\lambda/dz=L_\lambda^\infty
%e^{-\mu\tau_\lambda}\mu\tau_\lambda/H_a\]
%Ce taux est maximal pour 
%\[d\left(\tau_\lambda\mu e^{-\tau_\lambda\mu}\right)=0\]
%soit pour $\mu\tau_\lambda=1$. On a donc un maximum d'extinction (absorption
%ou diffusion) du rayonnement incident pour une épaisseur optique de 1
%traversée à partir du sommet de l'atmosphère. Pour des épaisseurs optiques
%plus faibles, on a peu d'extinction car la densité d'absorbants est faible.
%Pour des épaisseurs optiques plus grandes, on a beaucoup d'absorbants mais la
%luminance résiduelle est petite (figure \ref{fig:absrate}).
% 
%\begin{figure}[tbp]
%  \begin{center}
%    \includegraphics{\figfrancis/WH_abs_max}
%  \end{center}
%  \caption{Comparaison des structures verticales de la densité de
%  l'atmosphère $\rho$, de la luminance d'un rayonnement incident $L_\lambda$
%  et de sa dérivée verticale. L'échelle horizontale est linéaire pour chaque
%  grandeur.}
%  \label{fig:absrate}
%\end{figure}



	\sk \subsection{Aspect microscopique~: absorption par les gaz et liaisons moléculaires}

		\sk
On donne ici quelques éléments éclairants sur les processus en jeu à l'échelle microscopique lors de l'absorption de rayonnement incident (photons) par les molécules qui composent l'atmosphère\footnote{Cette partie est inspirée d'éléments trouvés dans le cours de S. Jacquemoud de \emph{Méthodes physique en télédétection.}}. Cette absorption est en fait liée à leurs caractéristiques énergétiques. L'énergie d'une molécule ne peut prendre que des valeurs discrètes correspondant à des niveaux énergétiques. On les représente
souvent par un diagramme dans lequel chaque niveau est figuré par un trait horizontal. L'absorption permet à une molécule de passer d'un niveau d'énergie~$e_1$ à un niveau d'énergie supérieur~$e_2$. Le passage du niveau~$e_1$ au niveau~$e_2$ s'accompagne de l'absorption d'un rayonnement de fréquence~$\nu$ telle que~$\Delta e = e_2 - e_1 = h \, \nu$, soit l'énergie d'un photon de fréquence~$\nu$. Les molécules possèdent une énergie électronique~$e\e{e}$, quantifiée comme les atomes, mais aussi une énergie de vibration~$e\e{v}$ et une énergie de rotation~$e\e{r}$, elles aussi quantifiées. Une bonne approximation de l'énergie totale $e\e{t}$ est donnée par la relation~$e\e{t} = e\e{e} + e\e{v} + e\e{r}$. A chaque état électronique correspondent plusieurs états de vibration des noyaux et à chaque état vibrationnel correspondent plusieurs états de rotation.

\figsup{0.47}{0.17}{decouverte/cours_meteo/electronique.png}{decouverte/cours_meteo/vibre.png}{Niveaux électroniques définissant des états de molécules [gauche]. Effet de l'interaction entre rayonnement et molécules pour plusieurs longueurs d'onde. Figures extraites du cours de S. Jacquemoud de \emph{Méthodes physique en télédétection.}}{fig:electronique}

\sk
L'interaction entre le rayonnement et les molécules constituant le milieu (par exemple, l'atmosphère) se manifeste d'une façon différente selon l'énergie~$h\nu$ du photon incident. Ainsi, par ordre décroissant de l'énergie du photon incident, donc par ordre croissant de sa longueur d'onde~$\lambda$, le photon va provoquer sur les liaisons moléculaires des brisures, des réorganisations de nuage électronique, des vibrations, ou simplement des rotations.
\begin{finger}
\item dans l'ultraviolet : les molécules (O$_2$, O$_3$, NO$_2$, \ldots) sont dissociées. La photolyse ou \voc{photodissociation} d'une espèce est provoquée par l'absorption d'un photon possédant une énergie suffisante pour conduire cette espèce à un état électronique excité puis finalement à une rupture de liaison. Citons les cas classiques de la dissociation de l'oxygène [O$_2$~+~$h\nu$~$\rightarrow$~O~+~O] pour des longueurs d'onde inférieures à 246 nm, ou de l'ozone [O$_3$~+~$h\nu$~$\rightarrow$~O$_2$~+~O] pour des longueurs d'onde inférieures à 310 nm. La photodissociation de NO$_2$ produit les atomes d’oxygène nécessaires à la formation photochimique de l’ozone troposphérique [NO$_2$~+~$h\nu$~+O$_2$~$\rightarrow$~NO~+O$_3$]. Ces phénomènes de photolyse participent très fortement à la chimie de l'atmosphère.
\item dans le visible : les molécules changent de configuration électronique ; les électrons qui gravitent autour du noyau atomique peuvent changer d'orbite ou même d'atome. Les photons du domaine du visible ne sont presque pas absorbés par l'atmosphère (très légèrement par O$_2$ et O$_3$) et sont donc uniquement diffusés.
\item dans l'infrarouge moyen et thermique : les molécules (CO$_2$, H$_2$O, CH$_4$, N$_2$O, \ldots) vibrent dans l'axe de la liaison moléculaire (étirement) ou perpendiculairement à cet axe (pliage). Ces molécules sont appelées \voc{gaz à effet de serre} car elles absorbent le rayonnement infrarouge thermique émis par la Terre puis réemettent des photons à la même longueur d'onde.
\item dans le domaine des micro-ondes : les molécules tournent autour d'un de leurs axes.
\end{finger}


		\sk
Les molécules de l'atmosphère absorbent donc le rayonnement à diverses longueurs d'onde. En conséquence, on comprend que les coefficients d'absorption des gaz qui composent l'atmosphère sont extrêmement variables en fonction de~$\lambda$ et présentent une structure très complexe. Un domaine limité de longueurs d'onde contigues où une certaine espèce atmosphérique est très absorbante est appelé \voc{bande d'absorption}. Certaines espèces possèdent des bandes d'absorption dans les longueurs d'onde visible, comme l'ozone~O$_3$, d'autres dans les longueurs d'onde infrarouge, comme les gaz à effet de serre CO$_2$ et H$_2$O [Figure~\ref{fig:atmspectrum} et table~\ref{tab:abs}]. Un domaine limité de longueurs d'onde contigues où les espèces principales qui composent une atmosphère ne sont pas (trop) absorbantes est appelée \voc{fenêtre atmosphérique}, car alors le coefficient de transmission atmosphérique est proche de~$1$.

\small
\begin{table}\label{tab:abs}
\begin{center}
\begin{tabular}{|c|c|}
\hline
Molécules & Principales bandes d'absorption (en $\mu$m) \\
\hline
O$_3$ & 0,242-0,31 (Hartley) / 0,31-0,4 (Huggins) / 0,4-0,85 (Chappuis) / 3,3 / 4,74 \\
O$_2$ & 0,175-0,2 (Schumann-Runge) / 0,2-0,26 (Herzberg) / 0,628 / 0,688 / 0,762 / 1,06 / 1,27 / 1,58 \\
CO$_2$ & 1,4 / 1,6 / 2,0 / 2,7 / 4,3 / >15 \\
H$_2$O & 0,72 / 0,82 / 0,94 / 1,1 / 1,38 / 1,87 / 2,7-3,2 / 6,25 / >14 \\
CH$_4$ & 1,66 / 2,2 / 2,3 / 2,37 / 3,26 / 3,31 / 3,53 / 3,83 / 3,55 / 7,65 \\
CO & 2,34 / 4,67 \\
N2O & 2,87 / 2,97 / 3,9 / 4,06 / 4,5 \\
\hline
\end{tabular}
\caption{\emph{Principales bandes d'absorption pour les gaz composant l'atmosphère terrestre. Voir la figure~\ref{fig:atmspectrum}}}
\end{center}
\end{table}
\normalsize

\figside{0.7}{0.35}{decouverte/cours_dyn/absorption.png}{Spectres d'absorption de l'atmosphère en fonction de la longueur d'onde. [Haut] Courbes d'émittance normalisée de corps noirs à 5780~K (rayonnement solaire) et 255~K (rayonnement terrestre). [Bas] Coefficients d'absorption (en~$\%$) entre le sommet de l'atmosphère et la surface. Les principaux gaz responsables de l'absorption à différentes longueurs d'onde sont indiqués en bas. Source: McBride and Gilmour, An Introduction to the Solar System, 2004 ; d'après Goody and Yung, Atmospheric radiation, 1989}{fig:atmspectrum}




\mk
\section{Une rapide synthèse~: spectre solaire à la surface de la Terre}

\sk
La figure~\ref{fig:radiation} compare le rayonnement solaire incident au sommet de l'atmosphère et à la surface, en fonction de la longueur d'onde. Dans les longueurs d'ondes visibles, où se situe la majorité du rayonnement solaire incident, une fraction du rayonnement, appelée albédo, est réfléchie vers l'espace~: les molécules constituant l'atmosphère, les aérosols et nuages contribuent tous à cet albedo planétaire (voir également figure~\ref{fig:diffsep}). En analysant conjointement la figure~\ref{fig:atmspectrum}, on voit que le rayonnement ultraviolet est complètement absorbé par l'ozone et l'oxygène. Restent que ces gaz diatomiques simples (comme c'est le cas de H$_2$, N$_2$) constituent un milieu presque complètement transparent à la fois au rayonnement solaire [surtout visible, ou \ofg{ondes courtes}] et au rayonnement terrestre [surtout infrarouge, ou ondes longues]. En revanche, certains gaz composés (en particulier CO$_2$, H$_2$O, N$_2$O, CH$_4$, CO \ldots) ne sont que partiellement transparents, car, s'ils sont transparents au rayonnement courte longueur d’onde (et donc à une grande partie du rayonnement solaire), ils absorbent plus ou moins fortement le rayonnement grande longueur d’onde, notamment le rayonnement émis par la Terre. Ainsi, contrairement à ce qui prévaut dans les longueurs d'onde visible, l'atmosphère est très opaque dans l'infrarouge (longueurs d'onde quelques microns). On distingue seulement une fenêtre ente 8 et 12 microns où le rayonnement émis par la surface s'échappe en grande partie vers l'espace. Les principaux absorbants sont la vapeur d'eau et le CO$_2$, d'autres gaz comme le méthane ou l'ozone ayant des contributions plus faibles. La plupart des gaz absorbent dans des bandes étroites. La vapeur d'eau au contraire absorbe dans un domaine spectral très large. Le spectre donné en figure~\ref{fig:atmspectrum} est valable en ciel clair et ne tient pas compte de la présence de nuages. L'eau liquide (ou glace) est un très fort absorbant dans l'infrarouge à toutes les longueurs d'ondes, et un nuage même peu épais absorbera donc rapidement la quasi-totalité du rayonnement incident. 

\figsup{0.49}{0.25}{\figpayan/LP211_Chap2b_Page_18_Image_0001.png}{\figpayan/LP211_Chap1_Page_07_Image_0004.png}{[Gauche] Spectre du rayonnement solaire (UV, Visible, IR) à l’extérieur de l’atmosphère terrestre et au niveau du sol (après traversée de l’atmosphère). Le rayonnement au sommet de l'atmosphère correspond à la courbe rouge. Le rayonnement diffusé par l'atmosphère correspond à la courbe bleu foncé. Le rayonnement absorbé par l'atmosphère correspond à la courbe bleu clair. Le rayonnement résultant au sol correspond à la courbe jaune. [Droite] Figure similaire, mais les espèces gazeuses responsables des bandes d'absorptions sont indiquées sur la figure.}{fig:radiation}

%\end{document}

%\mk \section{Synthèse : transfert radiatif dans l'infrarouge} On peut négliger la diffusion du rayonnement infrarouge dans l'atmosphère terrestre: la diffusion Rayleigh est très inefficace aux grandes longueurs d'onde, et les particules plus grosses (comme les gouttes d'eau) sont typiquement très absorbantes et donc peu diffusives.

%\begin{figure}[tbp]
%  \begin{center}
%    \includegraphics[width=\figw]{\figfrancis/Schwartzschild}
%  \end{center}
%  \caption{Variation du rayonnement infrarouge montant incident sur une couche
%  d'épaisseur $dz$: absorption et émission par la couche.}
%  \label{fig:schwartzschild}
%\end{figure}

%On doit par contre considérer en plus de l'absorption de rayonnement, l'émission dans l'infrarouge par l'atmosphère (figure \ref{fig:schwartzschild}). Dans les conditions appelées \emph{équilibre thermodynamique local\footnote{Ces conditions sont vérifiées si les collisions entre molécules sont plus fréquentes que l'absorption ou émission de rayonnement. Les molécules émettrices ont alors la même température que leur environnement}}, qui sont valables jusque vers 60~km d'altitude environ, le rayonnement émis (vers le haut et vers le bas) par une couche mince d'atmosphère dépend de sa température et de son coefficient d'absorption suivant la loi de Kirchoff. Pour un faisceau lumineux traversant une couche mince d'atmosphère, la variation de luminance vaut alors: \[dL_\lambda=\left(-L_\lambda+B_\lambda(T)\right)\mu d\tau_\lambda\] Le premier terme du second membre représente l'absorption du rayonnement incident, le deuxième l'émission par les gaz de la couche. La loi de Kirchoff fait qu'ils sont multipliés par le même coefficient $\mu\tau_\lambda$ qui donne le coefficient d'absorption et d'émission de la couche. Cette équation est appelée \emph{équation de Schwartzschild}. Son intégration entre une altitude $z_0$ et l'infini (espace) donne: \[L_\lambda(\infty)=L_\lambda(z_0)e^{-\mu\tau_\lambda(z_0,\infty)}+\int_{z_0}^\infty B_\lambda(T)e^{-\mu\tau_\lambda(z,\infty)}\mu\rho_ak_\lambda dz\] Le rayonnement sortant qu sommet de l'atmosphère est donc la somme du rayonnement présent en $z_0$ diminué de l'absorption entre $z_0$ et le sommet de l'atmosphère (premier terme), et de l'intégrale de la contribution du rayonnement émis par chaque couche au dessus de $z_0$. Comme pour le transfert dans le visible, on peut montrer (en supposant que $T$ varie peu) que la contribution maximale au rayonnement sortant à une longueur d'onde $\lambda$ provient d'une épaisseur optique de $\tau_\lambda=1$ à partir du sommet de l'atmosphère.


%%% CHAPITRE 4
%\def\thechapter{4} \title{Sciences de l'atmosphère\\ Chapitre \thechapter} \author{Aymeric SPIGA\\ ~\\ \small Modifié et augmenté à partir des notes de Francis CODRON et Sébastien PAYAN.} \maketitle %% commenter la ligne en mode livre complet
%\chapter{Bilan radiatif et effet de serre}

\dictum[Frederik van Eeden, 1887]{Le soleil accepte bien de passer par de petites fenêtres.}

\bk Les chapitres précédents n'ont été qu'un prélude pour bien comprendre les phénomènes qui déterminent la température à la surface de la Terre et dans son atmosphère. On s'attache dans le présent chapitre à effectuer des bilans d'énergie pour le système Terre et son atmosphère, définissant ainsi son \voc{bilan radiatif}.

\mk \section{Equilibre radiatif simple}

	\sk \subsection{Flux reçu et flux émis}

		\sk
Nous pouvons exprimer le rayonnement reçu du Soleil par la Terre par une densité de flux énergétique moyenne~$F\e{reçu}$ en W~m$^{-2}$ ou un flux énergétique~$\Phi\e{reçu}$ (en W)
\[ 
F\e{reçu} = (1-A\e{b}) \, \mathcal{F}\e{s}' 
\qquad \qquad
\Phi\e{reçu} = \pi \, R^2 \, (1-A\e{b}) \, \mathcal{F}\e{s}
\] 
La partie du rayonnement reçue du soleil qui est réfléchie vers l'espace est prise en compte via l'albédo planétaire noté~$A\e{b}$. On rappelle par ailleurs que~$\mathcal{F}\e{s}' = \mathcal{F}\e{s} / 4$ où $\mathcal{F}\e{s}$ est la constante solaire.


\sk
Par ailleurs, le système Terre émet également du rayonnement principalement dans les longueurs d'onde infrarouge [figure \ref{fig:eqrad2}]. 
Cette quantité de rayonnement émise au sommet de l'atmosphère radiative est notée $OLR$ pour \emph{Outgoing Longwave Radiation} en anglais.
A l'équilibre, la planète Terre doit émettre vers l'espace autant d'énergie qu'elle en reçoit du Soleil, donc
on obtient la relation générale appelée \emph{TOA} pour \emph{Top-Of-Atmosphere} en anglais, correspondant
au bilan radiatif au sommet de l'atmosphère
\[ \boxed{\TOA} \] 
La principale difficulté qui sous-tend les divers modèles pouvant être proposés réside dans l'expression du terme~$OLR$.




	\sk \subsection{Equilibre et température équivalente}

		\sk
Dans l'équilibre~\emph{TOA}, la manière la plus simple de définir~$OLR$ pour entamer un calcul préliminaire est comme suit. On fait l'hypothèse, assez réaliste en pratique, que la surface de la Terre est comme un corps noir, c'est-à-dire que son émissivité est très proche de~$1$ dans l'infrarouge où se trouve le maximum d'émission. D'après la loi de Stefan-Boltzmann, la densité de flux énergétique~$F\e{émis}$ émise par la Terre en W~m$^{-2}$ s'exprime
\[ F\e{émis} = \sigma \, {T\e{eq}}^4 \]
où~T\e{eq} est la \voc{température équivalente} du système Terre que l'on suppose uniforme sur toute la planète. Autrement dit, $T\e{eq}$ est la température équivalente d'un corps noir qui émettrait la quantité d'énergie~$F\e{émis}$. Le flux énergétique~$\Phi\e{émis}$ émis par la surface de la planète Terre s'exprime
\[ \Phi\e{émis} = 4 \, \pi \, R^2 \, F\e{émis} = 4 \, \pi \, R^2 \, \sigma \, {T\e{eq}}^4 \]
Contrairement au cas de l'énergie visible, il n'y a pas lieu de prendre en compte le contraste jour/nuit, car le rayonnement thermique émis par la Terre l'est à tout instant par l'intégralité de sa surface. La seule limite éventuellement discutable est l'uniformité de la température de la surface de la Terre, ce qui est irréaliste en pratique. On peut souligner cependant que, même dans le cas d'une planète n'ayant pas une température uniforme ou ne se comportant pas comme un corps noir, le rayonnement émis vers l'espace doit être égal en moyenne à $\sigma \, {T\e{eq}}^4$.
%% CHANGER LES SLIDES, ne pas utiliser P

\figsup{0.31}{0.17}{decouverte/cours_dyn/incoming.png}{decouverte/cours_dyn/emission.png}{Equilibre radiatif simple : à gauche, l'énergie reçue du Soleil par le système Terre ; à droite, l'énergie émise par le système Terre. Source~: McBride and Gilmour, \emph{An Introduction to the Solar System}, CUP 2004.}{fig:eqrad2}

\sk
A l'équilibre, la planète Terre doit émettre vers l'espace autant d'énergie qu'elle en reçoit du Soleil (équilibre \emph{TOA}). Ceci peut s'exprimer par unité de surface
\[ \boxed{ F\e{reçu} = F\e{émis} } \]
ou, pour un résultat similaire, en considérant l'intégralité de la surface planétaire
\[ \Phi\e{reçu} = \Phi\e{émis} \]
ce qui permet de déterminer la température équivalente en fonction des paramètres planétaires
\[ \boxed{
T\e{eq} = \bigg[ \frac{\mathcal{F}\e{s}'\,(1-A\e{b})}{\sigma} \bigg]^{\frac{1}{4}}
} \]


		\input{bilan_radiatif_temperature_equivalente_application.tex}

\mk \section{Modèles \ofg{à couches}} % dits aux puissances échangées

	\sk \subsection{Modèle à une couche et effet de serre}

		\sk
L'équilibre radiatif simple présenté à la section précédente souffre d'un problème majeur~: il suppose que l'atmosphère n'interagit pas avec les rayonnements incidents et émis, ce qui n'est pas le cas en réalité. On rappelle notamment avec la figure~\ref{fig:atmspectrum} deux points importants qui vont nous permettre de raffiner les calculs.
\begin{finger}
\item Au vu des températures typiques du Soleil et de la Terre, le rayonnement d'origine solaire est principalement émis dans les longueurs d'onde visible, alors que le rayonnement d'origine terrestre est principalement émis dans les longueurs d'onde infrarouge. Les fonctions de Planck normalisées montrées dans la figure~\ref{fig:atmspectrum} indiquent que les deux domaines d'émission ne se recoupent quasiment pas. On peut donc séparer les calculs selon le domaine visible (également appelé ondes courtes) pour tout ce qui concerne le rayonnement reçu du Soleil et le domaine infrarouge (également appelé ondes longues) pour tout ce qui concerne le rayonnement émis par la surface et l'atmosphère de la Terre. La figure~\ref{fig:modzero} reprend ainsi le calcul de l'équilibre radiatif simple en étant plus fidèle à cette distinction entre visible et infrarouge~; en l'absence d'atmosphère, la température de surface à l'équilibre~$T\e{s}$ est égale à~$T\e{eq}$.
%\figside{0.6}{0.3}{\figfrancis/WH_atmspectrum}{entre le sommet de l'atmosphère et 11~km.}{fig:atmspectrum}
\item L'équilibre radiatif simple néglige les propriétés d'absorption de l'atmosphère de la Terre. La figure~\ref{fig:atmspectrum} montre que cette approximation est relativement juste pour les longueurs d'onde visible, où l'atmosphère est assez transparente, mais très inexacte pour les longueurs d'onde infrarouge. On a vu dans les chapitres qui précèdent que, contrairement à ce qui prévaut dans les longueurs d'onde visible, l'atmosphère est très opaque, c'est-à-dire très absorbante, dans l'infrarouge à cause principalement des gaz à effet de serre (et des nuages). Comme décrit au chapitre précédent, et sur la figure~\ref{fig:atmspectrum}, les principaux gaz à effet de serre sont, par ordre d'importance dans le bilan radiatif de la Terre, H$_2$O, CO$_2$, CH$_4$, N$_2$O, O$_3$, auxquels il convient d'ajouter les gaz à effet de serre industriels, tels les halocarbures, notamment les chloro-fluoro carbures\footnote{Qui jouent par ailleurs un rôle dans la destruction de l'ozone stratosphérique}. On note au passage que certains gaz à effet de serre comme CO$_2$ et CH$_4$ sont à la fois controlés par des processus naturels et industriels. Le rayonnement émis par la surface terrestre, principalement dans l'infrarouge, est donc absorbé par ces espèces et réémis à la fois vers l'espace et vers la surface. Ainsi, contrairement à ce qui est supposé dans le cas de l'équilibre radiatif simple, une partie du rayonnement émis par la surface n'est pas évacuée vers l'espace et contribue à augmenter la température de la surface terrestre. Ainsi la température de surface à l'équilibre~$T\e{s}$ n'est pas égale à la température équivalente~$T\e{eq}$. La figure~\ref{fig:modun} résume cette situation qui permet d'obtenir par le calcul, présenté ci-dessous, une valeur pour~$T\e{s}$ plus proche de la température effectivement mesurée à la surface de la Terre. On parle de \voc{modèle à une couche}.
\end{finger}

\figside[page=1]{0.6}{0.25}{decouverte/cours_meteo/zero_couche.png}{Modèle à zéro couche~: schéma des flux nets échangés dans le visible et dans l'infrarouge pour une planète sans atmosphère (ou plus précisément dans laquelle l'atmosphère n'est active radiativement ni dans l'infrarouge ni dans le visible) de température de surface~$T\e{s}$. Il s'agit simplement d'une présentation alternative de l'équilibre radiatif simple décrit en figure~\ref{fig:eqrad2}, qui s'avère plus pratique pour prendre en compte la présence d'une atmosphère et effectuer des calculs plus proches de la réalité. Ce schéma est cependant plus précis que la figure~\ref{fig:eqrad2} dans la mesure où il précise dans quel domaine de longueur d'onde se font les échanges.}{fig:modzero}
%\figun{0.6}{0.2}{\figfrancis/GH_1lay_noatm}{Schéma des flux échangés dans le visible (jaune) et l'infrarouge (rouge) pour une planète sans atmosphère de température de surface $T_s$.}{fig:GH1laynoatm}

		\sk
Quelle température de surface est prédite par le modèle à une couche décrit par la figure~\ref{fig:modun} ? On considère toujours une planète d'albédo planétaire $A\e{b}$ recevant l'éclairement moyen $\mathcal{F}\e{s}'$ du Soleil. Ce bilan correspond à la partie visible de la figure~\ref{fig:modun}. L'atmosphère est considérée comme transparente dans ce domaine de longueur d'onde. Dans la partie infrarouge, au contraire on ne néglige plus l'absorption, par les gaz à effet de serre présents dans l'atmosphère, du rayonnement infrarouge émis par la surface de la planète à la température~$T\e{s}$~: on représente ainsi l'atmosphère par une couche isotherme de température~$T\e{a}$, parfaitement absorbante dans l'infrarouge. Le rayonnement infrarouge émis par la surface est complètement absorbé dans l'atmosphère, qui émet à son tour~$\sigma {T\e{a}}^4$ à la fois vers l'espace et vers la surface comme indiqué dans le domaine infrarouge de la figure~\ref{fig:modun}. Une partie du rayonnement infrarouge émis par la Terre n'est donc pas évacuée vers l'espace et reste \ofg{piégée} dans le système atmosphère~+~surface, contribuant ainsi à élever la température de la surface~$T\e{s}$.

\figside{0.6}{0.2}{decouverte/cours_meteo/une_couche.png}{Modèle à une couche~: schéma des flux échangés dans le visible et dans l'infrarouge pour une planète dont l'atmosphère de température~$T\e{a}$ est opaque dans l'infrarouge.}{fig:modun}

\sk
Il s'agit ensuite d'effectuer le bilan des flux reçus et cédés en chacune des interfaces en rassemblant les termes des deux domaines visible et infrarouge.
%%\footnote{Les modèles du type de celui présenté ici sont parfois également appelés modèles aux puissances échangées}
\begin{finger}
\item pour l'atmosphère
\[ \underbrace{\sigma \, {T\e{s}}^4}_{\text{bilan des flux reçus}} = \underbrace{\sigma \, {T\e{a}}^4 + \sigma \, {T\e{a}}^4}_{\text{bilan des flux cédés}} \] 
On note que le rayonnement visible reçu du Soleil n'intervient pas dans le bilan pour l'atmosphère, ce qui est normal puisque l'absorption est négligée. Ainsi, comme indiqué sur le schéma~\ref{fig:modun}, l'atmosphère reçoit un rayonnement~$\mathcal{F}\e{s}'$ dont la partie~$\mathcal{F}\e{s}'\,(1-A\e{b})$ qui n'est pas réfléchie/diffusée est entièrement transmise à la surface. Tout se passe comme si l'atmosphère recevait~$\mathcal{F}\e{s}'$ et cédait~$\mathcal{F}\e{s}'\,(1-A\e{b})$ à la surface et~$\mathcal{F}\e{s}'\,A\e{b}$ à l'espace~; son bilan d'énergie dans le visible est donc nul puisque tous ces termes se compensent.
\item pour la surface
\[ \underbrace{\mathcal{F}\e{s}'\,(1-A\e{b}) + \sigma \, {T\e{a}}^4}_{\text{bilan des flux reçus}} = \underbrace{\sigma \, {T\e{s}}^4}_{\text{bilan des flux cédés}} \]
\end{finger}
On dispose alors de deux équations qui permettent de déterminer les deux inconnues~$T\e{a}$ et~$T\e{s}$. Ainsi la température à la surface de la planète dans le modèle à une couche est 
\[ \boxed{ T\e{s} = \bigg[ \frac{ 2 \, \mathcal{F}\e{s}'\,(1-A\e{b}) }{ \sigma } \bigg]^{\frac{1}{4}} = \sqrt[4]{2} \, T\e{eq} } \]

\sk
Le calcul numérique donne une température de~$303$~K (environ~$30^{\circ}$C) pour la Terre, une valeur à la fois bien supérieure à~$T\e{eq}$, qui vaut~$255$~K, et plus proche de la température effectivement constatée à la surface, quoiqu'un peu surévaluée. Ainsi les gaz à effet de serre présents dans l'atmosphère contribuent à réchauffer significativement la surface d'une planète. Le modèle à une couche est le modèle le plus simple de l'effet de serre qui permet d'en rendre compte qualitativement et, dans une certaine mesure, quantitativement.

		\sk
La température de l'atmosphère dans le modèle à une couche est, d'après les équations qui précèdent,
\[ \boxed{ T\e{a} = \bigg[ \frac{ \mathcal{F}\e{s}'\,(1-A\e{b}) }{ \sigma } \bigg]^{\frac{1}{4}} = T\e{eq} } \]
Un résultat équivalent peut être obtenu en faisant un bilan des flux reçus/cédés pour l'interface \ofg{espace}
\[ \underbrace{\mathcal{F}\e{s}'\,A\e{b} + \sigma \, {T\e{a}}^4}_{\text{bilan des flux reçus}} = \underbrace{\mathcal{F}\e{s}'}_{\text{bilan des flux cédés}} \] 
Deux aspects de ce modèle simple de l'effet de serre sont importants:
\begin{enumerate}
\item Que l'on considère un équilibre radiatif simple [figures~\ref{fig:eqrad2}], ou un modèle à une couche [figures~\ref{fig:modun}], la température à laquelle est émise le rayonnement infrarouge sortant vers l'espace doit être (en moyenne) égale à $T\e{eq}$. Sans atmosphère, cette température est celle de la surface, avec une atmosphère opaque dans l'infrarouge, il s'agit de celle de l'atmosphère.
\item Il n'y a un effet de serre que si la température d'émission vers l'espace est inférieure à la température de la surface. On peut l'imaginer dans le cas où l'atmosphère est également opaque dans les longueurs d'onde visible~: la surface échange alors uniquement du rayonnement avec l'atmosphère, et est à la même température à l'équilibre~: $T\e{s}=T\e{a}=T\e{eq}$ [voir section suivante].
\end{enumerate}
Pour obtenir des températures atmosphériques plus en accord avec les variations verticales observées (qui servent à définir les différentes couches atmosphériques comme abordé au chapitre d'introduction), on peut adopter un modèle à~$2$, $3$, \ldots couches\footnote{Ce point est abordé en travaux dirigés.}.


	\sk \subsection{Modèle gris}

		\sk Le modèle à une couche peut être généralisé quelque peu en considérant deux raffinements.
\begin{finger}
\item L'atmosphère est absorbante dans le visible avec un coefficient d'absorption~$\alpha$. Plus~$\alpha$ est grand, plus le rayonnement incident dans les longueurs d'onde visible reçu par la surface terrestre est atténué. Ce phénomène porte le nom d'\voc{effet parasol} (ou parfois \ofg{anti-effet de serre}). En réalité cet effet est très modéré sur Terre. Certes, l'ozone stratosphérique absorbe complètement le rayonnement ultraviolet, mais ceci représente une contribution faible du flux total. Les aérosols, tels que les poussières désertiques ou les particules d'origine volcanique, absorbent dans le visible et peuvent contribuer lors d'événements particuliers, telles les éruptions volcaniques ou les tempêtes de poussière, à augmenter~$\alpha$.
\item L'atmosphère n'est pas tout à fait un corps noir~: son émissivité dans l'infrarouge est~$\epsilon$ comprise entre~$0$ et~$1$. D'après la seconde loi de Kirchhoff, ceci indique que la couche atmosphérique n'est pas parfaitement absorbante dans l'infrarouge~: elle absorbe une partie~$\epsilon \, F$ du flux incident~$F$ et en transmet une partie~$(1-\epsilon) \, F$. Reste qu'en pratique, comme mentionné dans la partie précédente, l'atmosphère se comporte comme un corps presque noir dans l'infrarouge et l'émissivité~$\epsilon$ est relativement proche de~$1$. 
\end{finger}
Un tel modèle porte le nom de \voc{modèle gris à une couche}. Comme dans la figure~\ref{fig:modun}, la méthode pour calculer les températures consiste à reporter tous les flux échangés entre chacune des couches, comme indiqué dans la figure~\ref{fig:modgris}, puis de faire le bilan des énergies reçues et cédées aux interfaces.
\[ \begin{aligned} & & \boxed{\text{bilan des flux reçus}} & = \boxed{\text{bilan des flux cédés}} \\ 
& \text{[espace]} & \mathcal{F}\e{s}'\,A\e{b} + (1-\epsilon) \, \sigma \, {T\e{s}}^4 + \epsilon \, \sigma \, {T\e{a}}^4 & = \mathcal{F}\e{s}' \\
& \text{[atmos.]} & \mathcal{F}\e{s}' + \sigma \, {T\e{s}}^4 & = \mathcal{F}\e{s}'\,A\e{b} + \mathcal{F}\e{s}'\,(1-A\e{b}-\alpha) + (1-\epsilon) \, \sigma \, {T\e{s}}^4 + \epsilon \, \sigma \, {T\e{a}}^4 + \epsilon \, \sigma \, {T\e{a}}^4 \\ 
& \text{[surface]} & \mathcal{F}\e{s}'\,(1-A\e{b}-\alpha) + \epsilon \, \sigma \, {T\e{a}}^4 & = \sigma \, {T\e{s}}^4 \\ \end{aligned} \]
%& \text{[surface]} & \mathcal{F}\e{s}'\,(1-A\e{b})\,(1-\alpha) + \epsilon \, \sigma \, {T\e{a}}^4 & = \sigma \, {T\e{s}}^4 \\ \end{aligned} \]

\sk
On obtient alors l'expression de la température de la surface de la planète dans le cadre du modèle gris en combinant les équations de bilan de deux des trois interfaces (par exemple espace et surface)
\[ \boxed{T\e{s} = \sqrt[4]{\frac{1 - \frac{\alpha^{\prime}}{2}}{1 - \frac{\epsilon}{2}}} \, T\e{eq}} \qquad  \text{avec} \quad \alpha^{\prime} = \frac{\alpha}{1-A\e{b}} \] 
\begin{citemize}
\item Dans le cas où l'atmosphère est un corps noir dans l'infrarouge~($\epsilon=1$) et qu'elle est transparente dans le visible~($\alpha=0$), on se retrouve dans la situation du modèle à une couche avec~$T\e{s} = \sqrt[4]{2} \, T\e{eq}$.
\item Dans le cas où l'atmosphère est opaque dans le visible~($\alpha^{\prime}=1$) et dans l'infrarouge~($\epsilon=1$), la surface échange alors uniquement du rayonnement avec l'atmosphère et~$T\e{s}=T\e{a}=T\e{eq}$.
\item Si~$\epsilon$ augmente, l'effet de serre augmente, donc la température de la surface de la planète augmente.
\item Si~$\alpha$ augmente, l'effet parasol augmente, donc la température de la surface de la planète diminue. C'est l'un des effets observés dans les mois qui suivent une éruption volcanique majeure.
\end{citemize}
%De façon plus générale, on a vu que le rayonnement sortant provenait majoritairement de la région de l'atmosphère autour d'une épaisseur optique de 1 à partir du sommet. Cette région dépend de la longueur d'onde: proche de la surface dans la fenêtre transparente, dans la haute troposphère dans les bandes d'absorption du CO$_2$, autour de 2~km dans celles de la vapeur d'eau. Comme la température décroit à partir de la surface, le rayonnement sortant est donc émis à des températures inférieures à $T_s$, et on peut écrire qu'il vaut \[IR_{sommet}=\sigma T_s^4 (1-\epsilon)=\sigma T_{eq}^4\] Où $\epsilon>0$ représente l'effet de serre. La valeur de $\epsilon$ augmente quand la température d'émission vers l'espace diminue par rapport à celle de surface, typiquement parce que l'altitude d'émission augmente.

\figside{0.7}{0.17}{decouverte/cours_meteo/une_couche_gris.png}{Modèle gris à une couche~: schéma des flux échangés dans le visible et dans l'infrarouge pour une planète comme la figure~\ref{fig:modun} sauf que l'atmosphère de température~$T\e{a}$ est opaque dans l'infrarouge, mais sans être un corps totalement noir, et est absorbante dans le visible avec un coefficient d'absorption~$\alpha$.}{fig:modgris}
%\figun{0.6}{0.2}{\figfrancis/GH_1lay_atm}{Comme la figure \ref{fig:GH1laynoatm} mais avec une atmosphère opaque dans l'infrarouge et de coefficient d'absorption $a$ dans le visible, de température $T_a$.}{fig:GH1layatm}


%%%%%%%%%%%%%%%%%%%%%%%%%%%%%%%%%%%%%%%%%%%%

\mk
\section{Description complète du bilan radiatif du système Terre}

\sk
\subsection{Mesures en moyenne dans le temps et dans l'espace}

\sk
Une représentation détaillée des différents flux échangés en moyenne temporelle et spatiale sur la Terre est présentée sur la figure~\ref{fig:bilflux}, qui est dérivée d'observations satellite les plus récentes. La figure est construite conformément à la séparation visible / infrarouge dictée par les résultats de la figure~\ref{fig:atmspectrum}. 

\sk
\subsubsection{Domaine visible}

\sk
Seulement~$50\%$ du rayonnement solaire incident dans les longueurs d'onde visible parviennent à la surface à cause, d'une part, de la réflexion/diffusion sur les molécules de l'air (diffusion Rayleigh dans toutes les directions, responsable de la couleur bleue du ciel), sur les gouttelettes nuageuses (diffusion de Mie) et sur la surface, et d'autre part, de l'absorption du rayonnement solaire incident par les molécules\footnote{Dans la mésosphère, c'est l'oxygène qui absorbe les radiations les plus énergétiques~; dans la stratosphère, l'absorption des radiations dans l’ultraviolet est assurée par différentes bandes d'absorption de l'ozone~; cette absorption peut avoir lieu dans certaines bandes jusque dans la troposphère.} et les aérosols composant l'atmosphère [relativement modérée dans les longueurs d'onde visible]. On note que la partie du rayonnement visible diffusée vers l'espace par les molécules de l'air, les nuages et la surface définit l'albédo planétaire mentionné précédemment~: un albédo élevé contribue à refroidir la surface et l'atmosphère. L'absorption de la lumière ultraviolet/visible, quant à elle, réchauffe directement l'atmosphère (notamment dans la stratosphère, car la troposphère n'est que très faiblement chauffée par les radiations solaires) et contribue à refroidir la surface par extinction du flux solaire incident. Dans le domaine visible, l'extinction est causée principalement par la diffusion et moins par l'absorption. La partie du rayonnement qui parvient à la surface est absorbée par la surface et convertie en énergie interne, c'est-à-dire contribue à élever sa température. On remarque que la surface terrestre ne peut être considérée tout à fait comme un corps noir puisqu'elle n'absorbe pas toute l'énergie incidente~: une petite partie du rayonnement incident est réfléchie par la surface. Cette composante réfléchie par la surface dépend fortement de la nature des sols (océans, forêts, déserts, glace, \ldots) et de leur répartition géographique.

\sk
\subsubsection{Domaine infrarouge}

\sk
Chauffée par l'absorption du rayonnement solaire incident, la surface terrestre se refroidit en émettant du rayonnement surtout dans l'infrarouge d'après la loi de Wien. La troposphère est ainsi principalement chauffée par l'absorption, par les gaz à effet de serre et les nuages, du rayonnement infrarouge émis par la surface. Dans l'infrarouge, à part quelques fenêtres atmosphériques à des longueurs d'onde bien précises, seule une petite partie du flux total émis par la surface s'échappe directement vers l'espace. A leur tour, les gaz à effet de serre émettent du rayonnement dans l'infrarouge, à la fois vers l'espace et vers la surface, ce qui refroidit localement l'atmosphère mais réchauffe la surface par effet de serre comme décrit précédemment avec le modèle à une couche [figure~\ref{fig:modun}]. L'atmosphère piège ainsi~$150$~W~m$^{-2}$ par effet de serre, puisque le rayonnement infrarouge sortant est~$240$~W~m$^{-2}$. On ajoute que la stratosphère est également refroidie par émission infrarouge du gaz carbonique, principalement dans la bande d'absorption à~$15$~$\mu$m. Du point de vue de l'atmosphère, émission infrarouge et refroidissement sont donc intimement liés.
%Émission nette par la vapeur d'eau, l'ozone, le CO2 et les autres gaz à effet de serre : Il s'agit du flux énergétique net émis sous forme de rayonnement énergétique (infrarouge) par l'ensemble des molécules de l'atmosphère. L'émission infrarouge est associée à un refroidissement local. Comme le Corps Noir, les molécules émettent un rayonnement pour se refroidir et équilibrer l'énergie absorbée. L'émission n'a lieu que dans les bandes d'absorption (ou d'émission). Il faut donc que la température locale soit celle du Corps Noir émettant à la longueur d'onde de la bande d'émission. Ainsi, plus on descend dans l'atmosphère plus l'émission se fera par les bandes centrées sur de faibles longueurs d'onde. Émission IR et refroidissement atmosphériques sont doncintimement liés. La stratosphère est principalement refroidie par l'émission IR du gaz carbonique. Ce refroidissement est associé à l'émission par la bande située à 15 μm. Dans la haute stratosphère, la bande d'émission de l'ozone à 9.6 μm permet l’émission IR et le refroidissement atmosphérique. Cependant l'ozone absorbe principalement les radiations solaires et ne peut être considérée comme un gaz à effet de serre (dans la stratosphère). La vapeur d'eau émet également dans la stratosphère dans la bande à 8 μm. La troposphère est principalement refroidie par l'émission de la vapeur d'eau dans la bande située à 6.3 micromètres.

\sk
\subsubsection{Autres échanges d'énergie}

\sk
Le bilan net en surface dans l'infrarouge de $65$~W~m$^{-2}$ est une petite différence entre le flux émis par la surface $\sigma \, {T\e{s}}^4$ et celui reçu depuis l'atmosphère. Si le bilan radiatif est bien équilibré au sommet de l'atmosphère, la surface gagne en moyenne de l'énergie et l'atmosphère en perd. En l'absence d'autres mécanismes de transfert d'énergie, cela conduirait à un refroidissement de l'atmosphère, et à une discontinuité de température à la surface entre le sol et l'air. En pratique, ce déséquilibre radiatif est compensé par des flux qui dépendent des mouvements et des changements de phase dans l'atmosphère
\begin{citemize}
\item de chaleur sensible (transport vertical de chaleur par la conduction et les mouvements de convection) 
\item de chaleur latente (évaporation depuis la surface et condensation dans l'atmosphère) 
\end{citemize}
depuis la surface vers l'atmosphère. Du fait que le transfert d'énergie du sol vers l'atmosphère se fait également sous forme d'un flux de chaleur sensible et latente, le sol n'émet donc que~$396$~W~m$^{-2}$ (au lieu de~$495$~W~m$^{-2}$) ce qui équivaut à une température de~$15^{\circ}$C, soit la température moyenne de la surface terrestre effectivement constatée. En l'absence de convection et de changements d'état dans l'atmosphère, la température de la surface et des basses couches atmosphériques serait beaucoup plus élevée. 
%%% les 240 W/m2 qui sortent sont les mêmes que dans la version sans atmosphère.
%%% noter la fenêtre atmosphérique dans l'infrarouge

%\figun{1.0}{0.4}{\figfrancis/bilan_rad_glob}{Schéma des flux moyens échangés entre la surface de la Terre, l'atmosphère, et l'espace: flux radiatifs ondes courtes (jaune) et infrarouge (rouge), et flux sensibles et latents (violet).}{fig:bilanrad}
\figun{1.0}{0.45}{decouverte/meteo_terre/bilanflux00004.png}{Bilan énergétique moyen de la Terre (en W~m$^{-2}$)~: flux échangés entre la surface de la Terre, l'atmosphère et l'espace. On distingue les flux radiatifs ondes courtes (rayonnement visible, en jaune) des flux radiatifs ondes longues (rayonnement infrarouge, en rouge). Noter les flux sensibles et latents qui ne sont pas relatifs au transfert radiatif. Source~: Planton CNRS editions 2011 ; adapté de Trenberth et al. BAMS 2009}{fig:bilflux}

%%% MANQUE UN TOPO SUR LE FORçAGE RADIATIF ????
%%% POUR REBRANCHER SUR LE CHANGEMENT CLIMATIQUE. VOIR PAYAN 10-12.

\sk
\subsection{Variations géographiques}

\sk
\subsubsection{Influence de la latitude}

\sk
Localement, l'éclairement varie suivant la latitude et la saison, en plus de l'alternance jour/nuit: il est proportionnel à $\cos\theta$ où $\theta$ est l'angle d'incidence avec la surface.
%[figure~\ref{fig:senslat1}]. 
En moyenne annuelle, le maximum d'ensoleillement est donc aux latitudes tropicales, mais il varie au cours de l'année et est même maximal aux pôles pendant l'été local [figure~\ref{fig:senslat2}]~: la durée du jour de 24h fait plus que compenser l'angle d'incidence réduit dû à la latitude élevée (ce qui peut paraître de prime abord contre-intuitif).
%\figside{0.3}{0.1}{\figfrancis/swcoslat.jpg}{Schéma de la relation entre densité de flux du rayonnement incident parallèle et éclairement de la surface suivant l'angle d'incidence.}{fig:senslat1}
\figside{0.65}{0.25}{\figfrancis/swtoaseas}{Cycle saisonnier de l'éclairement dû au rayonnement solaire incident au sommet de l'atmosphère.}{fig:senslat2}
%%%% http://www.energieplus-lesite.be/energieplus/page_16761.htm

\sk
\subsubsection{Rôle des nuages}

\sk
La présence de différents types de nuages est très variable, à la fois géographiquement et dans le temps. Ils ont pourtant une influence très grande sur le bilan radiatif, par deux mécanismes distincts [figure \ref{fig:schemacrf}].
\begin{finger}
\item Effet d'albédo~: les nuages réfléchissent une partie importante du rayonnement solaire incident (par rétro\-diffusion par les gouttes d'eau). Cet effet est d'autant plus fort que le nuage contient d'eau et que les gouttes sont fines. Un nuage très réfléchissant apparaitra sombre vu d'en dessous. Au total, les nuages sont responsables des 2/3 de l'albédo planétaire.
\item Effet de serre~: Les gouttes d'eau (ou la glace) des nuages sont d'excellents absorbants dans l'infrarouge. Un nuage même peu épais absorbe donc très rapidement tout le rayonnement infrarouge provenant des couches plus basses. Il émet lui même vers le haut du rayonnement suivant sa propre température: $\sigma T_N^4$ où $T_N$ est la température au sommet du nuage. Un nuage au sommet élevé (donc froid) aura donc un effet de serre très important.
\end{finger}
Au final, l'effet d'albédo l'emporte pour les nuages bas (type stratus), qui sont typiquement épais (albédo élevé) et dont le sommet est chaud. Au contraire, les fins nuages d'altitude (cirrus) ont un albédo faible mais un sommet très froid donc ont un effet net réchauffant. Pour les nuages de type orageux, qui sont épais avec un sommet froid, les deux effets tendent à se compenser.
\figside{0.6}{0.2}{\figfrancis/schema_crf}{Schema de l'influence des nuages sur le bilan radiatif: effet d'albédo dans le visible (jaune), et absorption et émission dans l'infrarouge (rouge). L'effet de serre vient du rayonnement émis vers l'espace plus faible que celui venant de la surface, qui est absorbé.}{fig:schemacrf}
%% nuages comme les lunettes dans la caméra infrarouge. faire également référence à la vidéo tirée du satellite.

\sk
\subsection{Moyennes annuelles~: cartes}

\sk
On présente dans cette section des cartes des différents termes du bilan radiatif de la terre, tels qu'observés par satellite depuis l'espace. On observe un bilan moyen sur une année qui est variable en fonction de la position géographique : si l'on a, en moyenne, égalité entre absorption du rayonnement solaire incident et émission de la Terre vers l’espace, ce n’est plus vrai si on considère une région donnée. 
\begin{finger}
\item
Le flux solaire absorbé (figure \ref{fig:swtoa}) montre essentiellement une dépendance en latitude. L'effet de l'ensoleil\-lement au sommet de l'atmosphère, plus fort dans les tropiques, est amplifié par un albédo élevé aux latitudes polaires à cause de la présence de neige et de glace au sol. En plus de ces variations en latitudes, on observe des différences locales dûes à l'albédo des régions nuageuses (zone de convergence intertropicale, bords est des océans) ou du sol (Sahara).
\item
Le flux infrarouge sortant au sommet de l'atmosphère (figure \ref{fig:olr}) a lui aussi une structure en latitude, mais moins marquée que pour les ondes courtes: les hautes latitudes, plus froides, émettent moins de rayonnement. On voit d'autre part nettement le flux sortant plus faible dans les régions humides des tropiques (continents et zone de convergence) où des nuages convectifs d'altitude élevée se forment.
\item
La signature des régions humides est nettement plus faible sur la carte du bilan net au sommet de l'atmosphère (figure \ref{fig:nettoa}); les effets de serre et d'albédo des nuages se compensant en grande partie. On retrouve par contre un bilan moins positif dans les régions où un albédo élevé provient du sol (Sahara) ou de nuages bas (Chili, Californie). D'autre part, on observe un gain net d'énergie dans les tropiques, et une perte dans les hautes latitudes; la distribution du bilan dans le visible qui est plus inégale que celle dans l'infrarouge détermine donc la structure globale. 
\end{finger}

\figside{0.65}{0.25}{\figfrancis/erbe_stoa_ann}{Rayonnement visible absorbé par la Terre, en moyenne annuelle (données ERBE).}{fig:swtoa}
\figside{0.65}{0.25}{\figfrancis/erbe_olr_ann}{Rayonnement infrarouge sortant au sommet de l'atmosphère, en moyenne annuelle.}{fig:olr}
\figside{0.65}{0.25}{\figfrancis/erbe_ntoa_ann}{Flux net absorbé par la Terre (visible  - infrarouge sortant) en moyenne annuelle.}{fig:nettoa}

%• Disparités régionales :
%o Sahara en été boréal : fort albédo + forte perte IR + capacité
%calorifique faible + atmosphère sèche
%o Océan : albédo faible compense la perte radiative IR + forte
%capacité calorifique

\sk
Ces excès et déficit d'énergie locaux doivent, en moyenne, être compensés par des transports d'énergie par les circulations atmosphérique et océanique. Ils fournissent la source d'énergie pour la \voc{dynamique atmosphérique} qui va contribuer à répartir l'énergie des régions excédentaires en énergie vers les régions déficitaires en énergie~[figure~\ref{fig:hadley}].

\figun{0.75}{0.35}{decouverte/cours_meteo/energiedyn.png}{Schéma représentant les latitudes où l'atmosphère est excédentaire ou, au contraire, déficitaire en énergie. La courbe bleue représente l'énergie radiative reçue du Soleil, principalement dans les courtes longueurs d'onde (noté \emph{shortwave} sur la figure, correspond au rayonnement visible et ultraviolet). La courbe rouge représente l'émission par la surface terrestre, principalement dans les longues longueurs d'onde (noté \emph{longwave} sur la figure, correspond au rayonnement infrarouge). Une circulation atmosphérique de grande échelle se met en place entre les régions excédentaires (équatoriales et tropicales) et déficitaires (hautes latitudes).}{fig:hadley}

%\end{document}

%\sk \subsection{Rétroactions} Préciser les rétroactions en jeu Indiquer le sens de ces rétroactions. Exemple terrestre~: \begin{itemize} \item Stefan-Boltzman : positive ou négative \item Glaces : positive ou négative \item Vapeur d'eau : positive ou négative \item Nuages : positive ou négative \end{itemize} 
%\visible<2->{\vskip 0.5cm\ebloc{}{Applications~:~changement climatique, paléo-climats, évolution des planètes du système solaire, climat des exoplanètes}}} \note{Peut être en général étendu aux autres planètes.\\ Les processus de rétroactions climatiques peuvent amplifier (on parle alors de rétroactions positives) ou réduire (rétroaction négative) la réponse à une perturbation initiale\\ SB : Si la température augmente alors la perte par rayonnement augmente : feedback négatif très fort\\ glace (albedo) : Si la température augmente alors la glace diminue et donc le rayonnement solaire absorbé augmente ce qui augmente la température feedback positif, mais attention au contrôle par la taille de la calotte polaire.\\ vapeur d’eau : l’augmentation de la température tend à favoriser l'évaporation car l'équilibre L-V est déplacé, de fait augmentation de l'humidité ie le contenu en vapeur d’eau de l’atmosphère, ce qui augmente l’effet de serre et donc la température de surface\\ nuages :  Si la température augmente et induit plus de nuages qui réfléchissent plus d’énergie solaire alors la température diminue. Cependant par effet de serre des nuages, la température augmente…. Au contraire, si le climat se refroidit, la couverture neigeuse hivernale persistera plus longtemps. Or cette couverture blanche (d’albédo plus élevé que le sol) augmente la réflexion de l'énergie solaire et donc diminue le chauffage de la surface par le Soleil. Il en résulte un refroidissement de la surface qui amplifie le refroidissement climatique initial feedback positif si refroidissement, inconnu si réchauffement (a priori négatif pour les nuages bas)} \note{[FACULTATIF] GES cycle glaciaire vs. interg.~: une entrée en glaciation entraîne une baisse de la teneur en gaz à effet de serre (CO2, H2O vapeur et CH4) dans l'atmosphère par suite des modifications du climat (refroidissement de la surface terrestre et modification de la circulation océanique profonde); cette diminution atténue l'effet de serre initial et donc amplifie le refroidissement en cours. Inversement une déglaciation entraîne une augmentation des mêmes gaz à effet de serre, ce qui, cette fois, contribue à accentuer le réchauffement feedbak positif}

%9. Perspectives
%9.1. Exemples de rétroactions
%Les processus de rétroactions climatiques peuvent amplifier (on parle alors de
%rétroactions positives) ou réduire (rétroaction négative) la réponse à une
%perturbation initiale et sont donc centraux pour simuler correctement l’évolution
%du climat.
%• La rétroaction de Stefan-Boltzmann :
%Si la température augmente alors la perte par rayonnement augmente :
%-- feedback négatif très fort
%• La rétroaction de la glace et de l’albédo :
%Si la température augmente alors la glace diminue et donc le rayonnement
%solaire absorbé augmente ce qui augmente la température
%-- feedback positif
%• La rétroaction des nuages
%Si la température augmente et induit plus de nuages qui réfléchissent plus
%d’énergie solaire alors la température diminue. Cependant par effet de serre des
%nuages, la température augmente…. Au contraire, si le climat se refroidit, la
%couverture neigeuse hivernale persistera plus longtemps. Or cette couverture
%blanche (d’albédo plus élevé que le sol) augmente la réflexion de l'énergie
%solaire et donc diminue le chauffage de la surface par le Soleil. Il en résulte un
%refroidissement de la surface qui amplifie le refroidissement climatique initial
%-- feedback positif si refroidissement, inconnu si réchauffement (a priori négatif
%pour les nuages bas).
%• Rétroaction des gaz à effet de serre au cours d’un cycle glaciaireinterglaciaire
%:
%une entrée en glaciation entraîne une baisse de la teneur en gaz à effet de serre
%(CO2, H2O vapeur et CH4) dans l'atmosphère par suite des modifications du
%climat (refroidissement de la surface terrestre et modification de la circulation
%océanique profonde); cette diminution atténue l'effet de serre initial et donc
%amplifie le refroidissement en cours. Inversement une déglaciation entraîne une
%augmentation des mêmes gaz à effet de serre, ce qui, cette fois, contribue à
%accentuer le réchauffement
%-- feedbak positif
%9.2. Convection et conduction
%Il faut tenir compte de la combinaison des effets radiatifs et convectifs pour
%comprendre la structure verticale de la température de l'atmosphère et la
%température de la surface.
%Les premiers modèles remontent au milieu des années 60 et ont depuis été
%remplacés par des formes très élaborées. Ils restent intéressants en étant d'une
%part la première étape vers les modèles à trois dimensions et d'autre part un outil
%intéressant pour l’estimation de l’impact du CO2 sur le climat de la Terre. Des
%versions planétaires ont été développées en astrophysique pour Mars etc …


\figside{0.75}{0.35}{decouverte/cours_meteo/joussaume_pluri.png}{Schéma du système climatique présentant les différentes composantes du système : atmosphère, océans, cryosphère, biosphère et lithosphère, leurs constantes de temps et leurs interactions en termes d’échanges d’énergie, d’eau et de carbone. Source~:~S.~Joussaume \emph{in} Le Climat à Découvert, CNRS éditions, 2011}{fig:pluri}

\figsup{0.48}{0.35}{\figfrancis/WH_vert_struct}{\figpayan/LP211_Chap1_Page_03_Image_0001.png}{[Gauche] Structure verticale de la pression, la densité et du libre parcours moyen des molécules (distance moyenne parcourue par une molécule avant de subir un choc sur une autre molécule). Noter l'échelle logarithmique en abscisse : les variations des quantités selon le logarithme de~$z$ sont approximativement des droites, donc les variations avec $z$ sont proches d'exponentielles. [Droite] Plus haut dans l'atmosphère, la variation verticale de la pression est dépendante pour plusieurs ordres de grandeur avec l'activité solaire. Les courbes indiquées correspondent respectivement à une très faible activité solaire (température de la thermopause de 600 K) et une très forte activité solaire (température de la thermopause de 2000K).}{fig:presvert}

\figsup{0.49}{0.3}{\figfrancis/WH_stdatm}{\figpayan/LP211_Chap1_Page_05_Image_0001.png}{Structure verticale idéalisée de la température correspondant au profil moyenné global annuel. Voir figure~\ref{fig:presvert} pour la distinction entre figure de gauche et figure de droite}{fig:tempvert}

%% CHAPITRE 5
%\def\thechapter{5} \title{Sciences de l'atmosphère\\ Chapitre \thechapter} \author{Aymeric SPIGA\\ ~\\ \small Modifié et augmenté à partir des notes de Francis CODRON.} \maketitle %% commenter la ligne en mode livre complet
\chapter{Bases de thermodynamique de l'atmosphère}

\dictum[Jean-Paul Sartre, 1948]{Torricelli a inventé la pesanteur de l'air, je dis qu'il l'a inventée plutôt que découverte, parce que, lorsqu'un objet est caché à tous les yeux, il faut l'inventer de toutes pièces pour pouvoir le découvrir.}

\bk
En filigrane des chapitres précédents sur le bilan radiatif, il apparaît que l'influence de la dynamique atmosphérique et des changements de phase ne peut être négligée. Avant de considérer ces phénomènes, il convient de se donner les outils conceptuels de thermodynamique pour caractériser l'état de l'atmosphère. Il ne s'agit pas de proposer un traité complet de thermodynamique, mais d'exprimer les équations et les concepts utiles pour décrire les phénomènes atmosphériques. 

%%% rappels GP, variables intensives, hauteur d'échelle

\mk
\section{Parcelle d'air}

\sk
\subsection{Définition et caractérisation}

\sk
On rappelle comme indiqué dans le chapitre d'introduction que l'atmosphère est composée d'un ensemble de molécules microscopiques et que l'on s'intéresse aux effets de comportement d'ensemble, qualifiés de \voc{macroscopiques}. Les variables thermodynamiques utilisées pour décrire l'atmosphère (pression~$P$, température~$T$, densité~$\rho$) sont des grandeurs macroscopiques \voc{intensives} dont la valeur ne dépend pas du volume d'air considéré. 
%Une façon d'y parvenir est d'utiliser des grandeurs volumiques ou massiques.

\sk
Le système que l'on étudie est appelé \voc{parcelle d'air}. Il s'agit d'un volume d'air dont les dimensions sont %à la fois
\begin{citemize}
\item assez grandes pour contenir un grand nombre de molécules et pouvoir moyenner leur comportement, afin d'exprimer un équilibre thermodynamique~;
\item assez petites par rapport au phénomène considéré, afin de décrire le fluide atmosphérique de façon continue~; la parcelle d'air peut donc être considérée comme un volume élémentaire.
\end{citemize}
On peut donc supposer que les variables macroscopiques d'intérêt sont quasiment constantes à l'échelle de la parcelle. Autrement dit, une parcelle est caractérisée par sa pression~$P$, sa température~$T$, sa densité~$\rho$. Les limites d'une parcelle sont arbitraires, mais ne correspondent pas en général à des barrières physiques. 

\sk
Tout le but de ce chapitre est de décrire les relations thermodynamiques qui lient les grandeurs intensives qui caractérisent l'état de la parcelle. Une première de ces relations a été obtenue en introduction~: il s'agit de l'équation des gaz parfaits pour l'air atmosphérique, qui relie les trois paramètres intensifs $P$, $T$ et $\rho$ 
\[ \boxed{ P = \rho \, R \,T } \] 
avec la \voc{constante de l'air sec} $R=\frac{R^*}{M}$ où~$R^*$ est la constante des gaz parfaits et~$M$ est la masse molaire de l'air. On rappelle que sur Terre~$R = 287$~J~K$^{-1}$~kg$^{-1}$. L'état thermodynamique d'une parcelle d'air est donc déterminé uniquement par deux paramètres sur les trois~$P$, $T$, $\rho$. Pour les applications météorologiques, on caractérise en général l'état de la parcelle par sa pression~$P$ et sa température~$T$, plus aisées à mesurer, par exemple via des ballons-sondes, que la densité~$\rho$.

\sk
\subsection{Parcelle et environnement} \label{parcenv}

\sk
Une parcelle est en \voc{équilibre mécanique} avec son environnement, c'est-à-dire que la pression de la parcelle~$P\e{p}$ et la pression de l'environnement~$P\e{e}$ dans lequel elle se trouve sont égales
\[ \boxed{P\e{p} = P\e{e}} \]
Néanmoins, une parcelle n'est pas en général en \voc{équilibre thermique} avec son environnement, c'est-à-dire que la température de la parcelle et la température de l'environnement dans lequel elle se trouve ne sont pas nécessairement égales
\[ \boxed{T\e{p} \neq T\e{e}} \]
Cette dernière propriété provient du fait que l'air est un très bon isolant thermique\footnote{Une telle propriété est utilisée dans le principe du double vitrage}.

\mk
\section{Équilibre hydrostatique}

\sk
Le chapitre d'introduction indique que la pression décroît exponentiellement avec l'altitude. On en donne ici une démonstration, en obtenant une loi dite hydrostatique dont les implications sont nombreuses. 

\sk
\subsection{Bilan des forces}

\sk
On considère une parcelle d'air cubique de dimensions élémentaires~$(\dd x,\dd y, \dd z)$, au repos et située à une altitude~$z$. La pression atmosphérique vaut~$P(z)$ sur la face du dessous et~$P(z+\dd z)$ sur la face du dessus. Pour le moment, on ne considère pas de variations de pression~$P$ selon l'horizontale\footnote{En pratique, les variations de pression selon l'horizontale sont effectivement négligeables par rapport aux variations de pression selon la verticale. On revient sur ce point dans le chapitre consacré à la dynamique}. Il y a équilibre des forces s'exerçant sur cette parcelle. On appelle \voc{équilibre hydrostatique} l'équilibre des forces selon la verticale, à savoir~:
\begin{citemize}
\item Son poids de module\footnote{On néglige les variations de~$g$ avec~$z$.}~$m \, g$ (où~$m = \rho \, \dd x \, \dd y \, \dd z$) dirigé vers le bas
\item La force de pression sur la face du dessous de module~$P(z) \, \dd x \, \dd y$ dirigée vers le haut
\item La force de pression sur la face du dessus de module~$P(z+\dd z) \, \dd x \, \dd y$ dirigée vers le bas
\item La force de viscosité qui est négligée
\end{citemize}
On note que, contrairement au poids qui s'applique de façon homogène sur tout le volume de la parcelle d'air, les forces de pression s'appliquent sur les surfaces frontières de la parcelle d'air. 
Pour une parcelle au repos, la résultante selon la verticale des forces de pression exercées par le fluide environnant (ici, l'air) n'est autre que la poussée d'Archimède.
%%Par ailleurs, l'équilibre hydrostatique suppose implicitement que la parcelle est à l'équilibre thermique avec son environnement~$T\e{p} = T\e{e}$ soit~$\rho\e{p} = \rho\e{e}$. On aborde le cas général où~$T\e{p} \neq T\e{e}$ dans le chapitre suivant pour définir les notions de stabilité.

\sk
L'équilibre hydrostatique de la parcelle s'écrit donc
\[ - \rho \, g \, \dd x \, \dd y \, \dd z + P(z) \, \dd x \, \dd y - P(z+\dd z) \, \dd x \, \dd y = 0 \qquad \Rightarrow \qquad - \rho \, g \, \dd z + P(z) - P(z+\dd z) = 0 \]
soit en introduisant la dérivée partielle suivant~$z$ de~$P$
\[ \frac{P(z+\dd z) - P(z)}{\dd z} = \boxed{ \Dp{P}{z} = - \rho \, g } \]
Cette relation est appelée \voc{équation hydrostatique} (ou relation de l'équilibre hydrostatique). Elle indique que, pour la parcelle considérée, la résultante des forces de pression selon la verticale équilibre la force de gravité. En principe, cette équation est valable pour une parcelle au repos. Par extension, elle est valable lorsque l'accélération verticale d'une parcelle est négligeable devant les autres forces. L'équation hydrostatique est donc approximativement valable pour les mouvements de grande échelle considérés dans toute la suite de ce cours. 

\sk
Si l'on intègre la relation hydrostatique entre deux niveaux~$z_1$ et~$z_2$ où la pression est~$P_1$ et~$P_2$, on obtient
\[ \Delta P = P_2 - P_1 = - g \, \int_{z_1}^{z_2} \rho \, \dd z \]
L'équilibre hydrostatique peut donc s'interpréter de la façon éclairante suivante~: la différence de pression entre deux niveaux verticaux est proportionnelle à la masse d'air (par unité de surface) entre ces niveaux. Une autre façon équivalente de formuler cela est de dire que la pression atmosphérique à une altitude~$z$ correspond au poids de la colonne d'air située au-dessus de l'altitude~$z$, exercé sur une surface unité de~$1$~m$^2$. Il s'ensuit que la pression atmosphérique~$P$ est décroissante selon l'altitude~$z$. Ainsi, la pression peut être utilisée pour repérer une position verticale à la place de l'altitude. En sciences de l'atmosphère, la pression atmosphérique est une coordonnée verticale plus naturelle que l'altitude~: non seulement elle est directement reliée à la masse atmosphérique par l'équilibre hydrostatique, mais elle est également plus aisée à mesurer.

\sk
\subsection{\'Equation hypsométrique}

\sk
\subsubsection{\'Echelle de hauteur}

\sk
En exprimant la densité~$\rho$ en fonction de l'équation des gaz parfaits, l'équilibre hydrostatique s'écrit
\[ \Dp{P}{z} = - g \, \frac{P}{RT} \]
On peut intégrer cette équation si on suppose que l'on connaît les variations de~$T$ en fonction de $P$ ou $z$. On considère que l'on peut négliger les variations de pression selon l'horizontale donc transformer les dérivées partielles~$\partial$ en dérivées simples~$\dd$. On effectue ensuite une séparation des variables
\[R \, T \, \frac{\dd P}{P} = - g \, \dd z\]

\sk
Cette équation peut s'écrire sous une forme dimensionnelle simple à retenir
\[ \boxed{ \frac{\dd P}{P} = - \frac{\dd z}{H(z)} \qquad \text{avec} \qquad H(z) = \frac{R \, T(z)}{g} } \]
La grandeur~$H$ se dénomme l'\voc{échelle de hauteur} et dépend des variations de la température~$T$ avec l'altitude~$z$. L'équation ci-dessus indique bien que la pression décroît avec l'altitude selon une loi exponentielle (voir Figure~\ref{fig:presvert}). Cette loi peut être plus ou moins complexe selon la fonction~$T(z)$. On peut néanmoins fournir une illustration simple du résultat de l'intégration dans le cas d'une atmosphère isotherme~$T(z)=T_0$
\[ P(z) = P(z=0) \, e^{-z/H} \qquad \text{avec} \qquad H = R \, T_0 / g \]

\sk
\subsubsection{\'Epaisseur des couches atmosphériques~: équation hypsométrique}

\sk
Le cas isotherme est simpliste et peu rencontré en pratique dans l'atmosphère. On se place dans le cas plus général, bien que toujours simplifié, de deux niveaux atmosphériques~$a$ et~$b$ entre lesquels la température ne varie pas trop brusquement avec l'altitude~$z$. On réalise alors l'intégration entre les deux niveaux~$a$ et~$b$
\[R \, T \, \frac{\dd P}{P} = - g \, \dd z \qquad \Rightarrow \qquad R \, \int_a^b T\, \frac{\dd P}{P} = - g \, \int_a^b dz\]
puis on définit la température moyenne de la couche atmosphérique entre~$a$ et~$b$ avec une moyenne pondérée
\[ \langle T \rangle = \frac{\int_a^b T \, \frac{\dd P}{P}}{\int_a^b \frac{\dd P}{P}} \]
pour obtenir finalement
\[R \, \langle T \rangle \, \int_a^b \frac{\dd P}{P} = - g \, \int_a^b dz
\qquad \Rightarrow \qquad \boxed{ g \, (z_a - z_b) = R \, \langle T \rangle \ln \left( \frac{P_b}{P_a} \right) } \]
Cette relation est appelée \voc{équation hypsométrique}. Elle correspond à une formulation utile en météorologie du principe que \ofg{l'air chaud se dilate}. Les conséquences de l'équation hypsométrique peuvent s'exprimer de diverses façons équivalentes.
\begin{citemize}
\item Pour une masse d'air donnée, une couche d'air chaud est plus épaisse.
\item La distance entre deux isobares est plus grande si l'air est chaud.
\item La pression diminue plus vite selon l'altitude dans une couche d'air froid.
\end{citemize}
En passant le résultat précédent au logarithme, on note que l'on retrouve toujours le fait que la pression diminue avec l'altitude selon une loi exponentielle. En notant l'échelle de hauteur moyenne~$\langle H \rangle$, on a
\[ P_b = P_a \, e^{ - \frac{z_b - z_a}{\langle H \rangle}} \qquad \text{avec} \qquad \langle H \rangle = \frac{R \, \langle T \rangle}{g} \]

\sk
\subsection{Applications pratiques}

\sk
\subsubsection{Expérience de Pascal}

\sk
Depuis son invention en 1643 par Torricelli, disciple de Galilée, le \voc{baromètre} est l'instrument de référence pour mesurer la pression atmosphérique à la surface de l'atmosphère terrestre\footnote{Le but initial de Torricelli était de parvenir le premier à maintenir artificiellement en laboratoire une chambre sous vide. Néanmoins, il est également reporté que l'invention du baromètre découle des réflexions de Torricelli autour de l'impossibilité, constatée en pratique, de pomper l'eau d'un fleuve au-dessus d'un certain niveau.}. Trois ans après son invention, le baromètre était déjà utilisé pour sa première application en sciences de l'atmosphère~: donner une preuve expérimentale de l'équilibre hydrostatique qui gouverne la stratification en pression de l'atmosphère. Autrement dit, le baromètre est un moyen indirect de mesurer la masse de l'atmosphère à travers la pression de surface. Blaise Pascal montra ainsi, par des mesures respectivement sur la Tour Saint-Jacques à Paris ($ \Delta z = 52 \U{m} $ au-dessus du sol) et sur le Puy-de-Dôme en Auvergne ($ \Delta z = 1000 \U{m} $ au-dessus du sol), que la pression atmosphérique varie avec l'altitude\footnote{Le texte original du traité de Pascal est disponible sur Gallica~: \url{http://gallica.bnf.fr/ark:/12148/bpt6k105083f}}.

\sk
L'équation hypsométrique permet de retrouver l'écart relatif en pression mesuré par Blaise Pascal entre le sol et le haut de la Tour Saint-Jacques. Comme les variations mesurées sont petites, on peut les assimiler aux différentielles et on peut négliger les variations de température avec l'altitude. Les variations relatives de pression mesurées par Pascal peuvent alors s'écrire
\[ \f{\Delta P}{P} \simeq \f{g}{R\,T} \, \Delta z \]
L'application numérique avec~$T = 288$~K donne une variation~$\Delta P / P = 0.62 \%$. La variation de pression détectée par Blaise Pascal\footnote{On note que, par une heureuse coincidence, la variation de pression entre le pied et le sommet de la Tour Saint-Jacques est de l'ordre de grandeur de la pression atmosphérique à la surface de Mars, ce qui permet de se représenter la finesse de l'atmosphère sur cette planète, comparé à notre Terre.} est donc d'environ~$6$~hPa (avec la valeur standard de la pression de surface~$P_0 = 1013$~hPa). Bien que cette baisse de pression soit détectable à l'aide du tube de Torricelli, Blaise Pascal a reproduit l'expérience au Puy-de-Dôme avec un écart~$\Delta z$ plus grand pour une meilleure précision quantitative. 

\sk
\subsubsection{Pression au niveau de la mer, altimétrie et cartes météorologiques}

\sk
Comme le montre la figure~\ref{fig:press} (haut), la pression~$P$ à la surface de la Terre est au premier ordre sensible à l'altimétrie (hauteur topographique), puisque la pression correspond au poids de la colonne d'air située au-dessus du point considéré. Pour produire des cartes de prévision du temps, on souhaite éliminer du champ de pression~$P$ cette composante topographique de premier ordre et mettre en évidence les variations de second ordre digne d'intérêt en météorologie.

\figsup{0.62}{0.32}{decouverte/cours_meteo/SURFPRESS/outputvar134_200.png}{decouverte/cours_meteo/SURFPRESS/outputvar151_200.png}{Champs de pression prédits au 01/09/2009 par les réanalyses ERA-INTERIM de l'ECMWF (Centre Européen de Prévision du Temps à Moyen Terme). La réprésentation graphique est basée sur une projection stéréographique polaire centrée sur le pôle Nord et les structures topographiques sont ajoutées dans le fond de la figure pour repérage. En haut, le champ de pression brut est tracé en hPa~: les valeurs de pression les plus basses correspondent aux reliefs topographiques les plus élevés. En bas, le champ de pression ramené au niveau de la mer est tracé en hPa~: la composante de premier ordre topographique sur le champ de pression a disparu pour laisser place aux composantes météorologiques de la pression~: dépressions (zones de basses pression) en bleu et anticyclones (zones de haute pression) en rouge. On peut d'ailleurs noter dans ce champ de pression normalisé l'activité ondulatoire de l'atmosphère aux moyennes latitudes. Les cartes de pression des bulletins météorologiques sont exclusivement des cartes de pression ramenées au niveau de la mer comme celle-ci.}{fig:press} 

\sk
Quand la pression de surface est mesurée à une station située à une altitude~$z \ll H$, on peut utiliser l'équation hypsométrique avec la température mesurée à la station pour déterminer une valeur approximative de la pression au niveau de la mer à~$z=0$. On suppose fréquemment que la température décroît linéairement avec l'altitude~$z$ selon un taux constant négatif~$\Gamma\e{e}$ en~$^{\circ}$C/km (ou K/km). On appelle d'ailleurs la loi~$T = T_0 + \Gamma\e{e} \, z$ le profil de l'atmosphère standard. En intégrant entre le niveau de la mer ($z=0, P=P_0$) et la station à ($z,P$), on obtient:
\[ \ln \left( \frac{P_0}{P} \right) = \frac{g}{R \, \Gamma\e{e}} \, \ln \left( \frac{T_0 + \Gamma\e{e} \, z}{T_0} \right) \]
\[ \Rightarrow \qquad P_0 = P \left( 1 + \frac{\Gamma\e{e} \, z}{T_0} \right)^{\frac{g}{R\,\Gamma\e{e}}} \]
La carte météorologique sur la Figure~\ref{fig:press} bas est obtenue en employant cette équation. L'équation qui précède est aussi utilisée avec $P_0=1013.25$~hPa par les altimètres des avions de ligne pour convertir~$P$ mesurée en~$z$.

\mk
\section{Premier principe et thermodynamique de l'air sec} 

\sk
\subsection{\'Energie interne}

\sk
Un système thermodynamique possède, en plus de son énergie d'ensemble (cinétique, potentielle), une \voc{énergie interne}~$U$. Comme la température~$T$, l'énergie interne~$U$ est une grandeur macroscopique qui représente les phénomènes microscopiques au sein d'un gaz. Le premier principe indique que les variations d'énergie interne sont égales à la somme du travail et de
la chaleur algébriquement reçus~:
\[ \dd U = \delta W + \delta Q\]

\sk
Dans le cas d'un gaz parfait, l'énergie potentielle d'interaction des molécules du gaz est négligeable, et l'énergie interne est égale à l'énergie cinétique des molécules, qui dépend seulement de la température. On peut montrer que $U = n \, \frac{\zeta \, R^* \, T}{2}$ où $\zeta$ est le nombre de degrés de liberté des molécules. Pour un gaz (principalement) diatomique comme l'air, $\zeta = 5$. 

\sk
Dans le cas de variations quasi-statiques d'un gaz, ce qui est supposé être le cas dans l'atmosphère, le travail s'exprime en fonction de la pression~$P$ du gaz et de la variation de volume~$\dd V$
\[ \delta W = - P \, \dd V \]

\sk
\subsection{Chaleurs molaires et enthalpie}

\sk
L'expérience montre que la quantité de chaleur échangée au cours d'une transformation à volume ou pression constant est proportionnelle à la variation de température du système~: $\delta Q = n \, C_V^* \, \dd T$ à volume constant, $\delta Q = n \, C_P^* \, \dd T$ à pression constante. $C_P^*, C_V^*$ sont les \voc{chaleurs molaires}, également appelées \voc{capacités calorifiques}. Il s'agit de l'énergie qu'il faut fournir à un gaz pour faire augmenter sa température de~$1$~K dans les conditions indiquées (à volume constant ou à pression constante). Pour une transformation à volume constant (isochore), $\dd U = \delta Q$ donc $C_V^*=\frac{\zeta \, R^*}{2}$.

\sk
Pour l'étude de l'atmosphère, toujours dans la logique de travailler sur des grandeurs intensives, il est  bien plus utile de s'intéresser aux variations de pression plutôt qu'à celles de volume. On utilise donc l'\voc{enthalpie}~$H = U + P \, V$. On a alors par dérivation $ \dd H = \dd U + \dd (P\,V) $ puis, en utilisant le premier principe
\[ \dd H = V \, \dd P + \delta Q \]
Pour une transformation à pression constante (isobare) on a $\dd H = \delta Q$. On en déduit pour une transformation quelconque\footnote{
D'autre part, en utilisant conjointement la dérivation de l'équation d'état du gaz parfait~$\dd (P\,V) = n \, R^* \, \dd T$ et l'expression de l'énergie interne~$U = n \, C_V^* \, \dd T$, on obtient $\dd H = n \, C_V^* \, \dd T + n \, R^* \, \dd T$ pour une transformation quelconque. On en déduit la relation de Mayer \[ C_P^* = C_V^* + R^* = \frac{(\zeta+2) \, R^*}{2}\]
} 
que $\dd H = n \, C_P^* \, dT$, ce qui permet d'écrire
\[ n \, C_P^* \, dT = V \, \dd P + \delta Q \]

\sk
\subsection{Transformations dans l'atmosphère~: cas général}

\sk
Afin de travailler sur des grandeurs intensives, on divise la relation précédente par la masse~$m$ de la parcelle pour obtenir
\[ C_P \, \dd T = \frac{\dd P}{\rho} + \delta q \]
où $\delta q$ est la chaleur massique reçue et $C_P = C_P^* / M$ est la \voc{chaleur massique de l'air} ($C_P$=1004 J~K$^{-1}$~kg$^{-1}$). Nous disposons alors d'une autre version du premier principe, très utile en météorologie et valable pour une transformation quelconque d'une parcelle d'air
\[ \boxed{ \underbrace{\textcolor{white}{\frac{R^2}{C_P}} \dd T \textcolor{white}{\frac{R}{C_P}}}_{\text{variation de température de la parcelle}} = \underbrace{\frac{R}{C_P} \, \frac{T}{P} \, \dd P}_{\text{travail expansion/compression}} + \underbrace{\frac{1}{C_P} \, \delta q}_{\text{chauffage diabatique}} } \]

\sk
Autrement dit, la température de la parcelle augmente si elle subit une compression ($\dd P > 0$) et/ou si on lui apporte de la chaleur ($\delta q > 0$). La température de la parcelle à l'inverse diminue si elle subit une détente ($\dd P < 0$) et/ou si elle cède de la chaleur à l'extérieur ($\delta q < 0$). Il est donc important de retenir que la température de la parcelle peut très bien varier quand bien même la parcelle n'échange aucune chaleur avec l'extérieur~: dans ce cas, $\delta q = 0$ et l'on parle de \voc{transformation adiabatique}. 

\sk
L'équation fondamentale ci-dessus est directement dérivée du premier principe, mais prend une forme plus pratique en sciences de l'atmosphère du fait que les transformations que subit une parcelle atmosphérique se réduisent en général aux transformations \voc{isobares} (à pression constante $\dd P = 0$) et aux transformations \voc{adiabatiques} (sans échanges de chaleur avec l'extérieur $\delta q = 0$). Les transformations isothermes, au cours de laquelle la température de la parcelle ne varie pas, sont très rarement rencontrées en sciences de l'atmosphère.

\sk
\subsection{Transformations non adiabatiques}

\sk
Dans le cas où la transformation n'est pas adiabatique, les échanges de chaleur~$\delta q$ d'une parcelle d'air avec son environnement sont non nuls et peuvent s'effectuer par~:
\begin{itemize}
\item Transfert radiatif~: l'atmosphère se refroidit en émettant dans l'infrarouge, ou se réchauffe en absorbant du rayonnement électromagnétique dans l'infrarouge [cas des gaz à effet de serre] ou dans le visible [cas de l'ozone dans la stratosphère].
%Ces échanges sont faibles et peuvent être négligés sauf à l'échelle de la circulation générale\footnote{Le refroidissement/réchauffement peut être localement élevé au sommet/à la base de nuages.}
\item Condensation ou évaporation d'eau~: ce point est abordé dans le chapitre suivant (ceci n'a lieu que lorsque l'air est à saturation).
\item Diffusion moléculaire (conduction thermique)~: ces transferts sont très négligeables sauf à quelques centimètres du sol.
\end{itemize}
Un cas notamment souvent cité en météorologie est celui d'une parcelle d'air située proche du sol, à la tombée de la nuit, qui subit peu de variations de pression ($\dd P \sim 0$) mais dont la température diminue sous l'effet du refroidissement radiatif ($\delta q < 0$). Ceci explique la présence de rosée sur le sol et de brouillard proche de la surface au petit matin, comme il est décrit plus en détail dans le chapitre suivant.

\sk
\subsection{Transformations adiabatiques}

\sk
Dans de nombreuses situations en sciences de l'atmosphère, on peut considérer que l'évolution de la parcelle est adiabatique et se fait sans échange de chaleur avec l'extérieur ($\delta q=0$). En vertu de l'équilibre hydrostatique qui relie pression~$P$ et altitude~$z$~:
\begin{citemize}
\item une parcelle dont l'altitude~$z$ augmente sans apport extérieur de chaleur, subit une \voc{ascendance} adiabatique, donc une détente telle que~$\dd P < 0$ et sa température diminue ;
\item inversement, une parcelle dont l'altitude~$z$ diminue sans apport extérieur de chaleur, subit une \voc{subsidence} adiabatique, donc une compression telle que~$\dd P > 0$ et sa température augmente. 
\end{citemize}

\sk
Dans le cas où la transformation est adiabatique, pression et température sont intimement liées en vetu du premier principe. La version du premier principe encadrée ci-dessus avec~$\delta q = 0$ indique
\[ \dd T = \frac{R}{C_P} \, \frac{T}{P} \, \dd P\]
\[ \Rightarrow \qquad \frac{\dd T}{T} - \frac{R}{C_P} \, \frac{\dd P}{P} = 0 \]
soit par intégration
\[ T \, P^{- \kappa} = \text{constante} \qquad \text{avec} \qquad \kappa = R / C_P \]
Autrement dit, dans le cas où une parcelle subit une transformation adiabatique, sa température varie proportionnellement à~$P^{\kappa}$. Il s'agit d'une version, avec les grandeurs intensives utiles en sciences de l'atmosphère, de l'équation~$P\,V^{\gamma}$, avec $\gamma = C_P / C_V$, vue dans les cours de thermodynamique générale pour les transformations adiabatiques.

%% La température ne nous donne pas des informations fiables sur les échanges de chaleur d'une parcelle avec l'extérieur. Pour ce faire, on se base sur la température potentielle.
%On définit la {\em température potentielle} $\Theta$ par:
%\begin{equation}
%  \Theta=T\cdot\left(\frac{P}{P_0}\right)^{-\kappa}
%  \label{theta}
%\end{equation}
%où $P_0$ est une pression de référence égale à 1000 hPa. $\Theta$ a donc la
%dimension d'une température (on l'exprime en Kelvin), et est conservée au
%cours de transformations adiabatiques. $\Theta$ est égale à la température
%d'une parcelle ramenée de façon adiabatique à une pression $P_0$.

\sk
\subsection{Gradient adiabatique sec} \label{adiabsec}

\sk
Il s'ensuit directement une loi simple des variations de température avec l'altitude d'une parcelle qui ne subit que des transformations adiabatiques. Considérons le cas d'une parcelle subissant un déplacement vertical quasi-statique et adiabatique tel que~$\delta q = 0$. Elle vérifie en première approximation l'équilibre hydrostatique~$\dd P\e{p} / \rho = - g \, \dd z$. La version du premier principe encadrée ci-dessus indique alors que
\[  \dd T\e{p}  = - \frac{g}{C_P} \, \dd z \]
d'où on tire le profil vertical adopté dans l'atmosphère sèche par une parcelle ne subissant pas d'échange de chaleur avec l'extérieur
\[  \boxed{ \ddf{T\e{p}}{z}  = \Gamma\e{sec} \qquad \text{avec} \qquad \Gamma\e{sec} = \frac{-g}{C_P} } \]
On note qu'il ne s'agit pas nécessairement du profil vertical suivi par l'environnement (voir section~\ref{parcenv}).

\sk
Le résultat trouvé ci-dessus revêt une importance particulière en sciences de l'atmosphère. La température d'une parcelle en ascension adiabatique décroît avec l'altitude selon un taux de variation constant, indépendamment des effets de pression. La constante~$\Gamma\e{sec}$ est appelée le \voc{gradient adiabatique sec} de température. Il n'est valable que pour une parcelle d'air non saturée en vapeur d'eau. Le calcul pour la Terre donne un refroidissement de l'ordre de~$10^{\circ}$C/km (ou K/km). Pourquoi cette valeur est-elle en désaccord avec la décroissance de~$6.5^{\circ}$C/km effectivement constatée dans l'atmosphère terrestre~? Le chapitre suivant apporte des éléments de réponse à ce paradoxe apparent.

%On définit également l'{\em énergie statique}  \begin{equation}  e_s=C_PT+gz=cste  \label{estat} \end{equation} L'énergie statique est la somme de l'enthalpie et de l'énergie potentielle de gravitation par unité de masse, et est conservée pour des transformations adiabatiques\footnote{L'énergie cinétique est négligeable. Typiquement, $\delta e_c$=50m$^2$s$^{-2}$ (variation de 10m.s\mo) et $\delta e_s$=10 000m$^2$s$^{-2}$ (variation de 1000m ou 10\deg).}. Les variations ou différences de $e_s$ et $\Theta$ sont reliées par: \[de_s=C_PTd\ln\Theta\]

%% pourquoi la température décroît avec l'altitude lorsqu'on monte une montagne: problème longtemps abordé avec des solutions erronées. y compris Fourier l'a mal interprété.

%% pierrehumbert: tropo means turning in greek, strato means stratification.

%%% CIRCULATIONS THERMIQUES ???

%%\footnote{Cette propriété est employée en pratique pour construire aisément les lignes \og adiabatiques sèches \fg dans un émagramme, comme il est décrit dans un chapitre ultérieur.}


%%%% CHAPITRE 6
%%\def\thechapter{6} \title{Sciences de l'atmosphère\\ Chapitre \thechapter} \author{Aymeric SPIGA\\ ~\\ \small Modifié et augmenté à partir des notes de Francis CODRON.} \maketitle %% commenter la ligne en mode livre complet
\chapter{Changements de phase et (in)stabilité}

\dictum[René Char, 1934]{Il faut être l'homme de la pluie et l'enfant du beau temps.}

\bk
Le cycle de l'eau est une composante essentielle du système climatique terrestre. Bien que les quantités présentes dans l'atmosphère font de l'eau un composant minoritaire, son rôle climatique et météorologique est de première importance. Pour se préparer à l'étude de ces questions, notamment la formation des nuages abordée au chapitre suivant, on s'intéresse dans ce chapitre à l'évolution d'une parcelle d'air de manière plus approfondie qu'au chapitre précédent en introduisant d'une part les changements d'état de l'eau et d'autre part la notion de stabilité de la parcelle par rapport à son environnement.

\mk
\section{Air humide, air saturé}

\sk
L'eau est présente dans l'atmosphère sous trois phases différentes, de la moins à la plus ordonnée~: gazeuse (vapeur d'eau), liquide (fines gouttelettes en suspension formant les nuages, précipitations pluvieuses), solide (cristaux de glace dans les fins nuages de haute altitude, intempéries de type neige et grêle). On s'intéresse principalement aux phases liquide et gazeuse afin de préfigurer l'étude des nuages sur Terre. Des raisonnements similaires sont possibles, avec quelques subtilités, pour la phase solide afin de décrire les nuages formés de cristaux de glace lorsque la température de l'atmosphère est suffisamment basse.

\sk
\subsection{Quantification de la vapeur d'eau dans l'atmosphère}\label{rappmel}

\sk
Soit une parcelle contenant un mélange de gaz parfaits notés~$i$, dont un est la vapeur d'eau. On a défini la \voc{pression partielle}~$P_i$ et le \voc{rapport de mélange massique}~$r_i = \frac{m\e{gaz i}}{m\e{air}}$ dans le chapitre introductif. Ces deux quantités peuvent servir à définir la quantité de vapeur d'eau présente dans la parcelle d'air. Pour simplifier, on note
\[ P\e{vapeur d'eau} = e \qquad \text{et} \qquad r\e{vapeur d'eau} = \frac{m\e{vapeur d'eau}}{m\e{air}} = r \] 
%La quantité~$q$ est également appelée \voc{humidité spécifique}. 
Le rapport de mélange en vapeur d'eau~$r$ est conservé dans la parcelle si il n'y a pas de changement de phase.

\sk
La pression partielle de l'air sec est~$P - e$. Comme mentionné dans le chapitre d'introduction, la vapeur d'eau vérifie l'équation d'état des gaz parfaits tout comme l'air sec, mélange de gaz parfaits, d'où
\[  e \, V = \frac{m\e{vapeur d'eau}}{M\e{vapeur d'eau}} \, R^* \, T  \qquad \qquad \qquad (P-e) \, V = \frac{m\e{air sec}}{M\e{air sec}} \, R^* \, T  \]
On forme le rapport des deux expressions pour obtenir une expression en fonction de paramètres intensifs et ne dépendant pas de la température
\[ \frac{e}{P-e} = \frac{m\e{vapeur d'eau}}{m\e{air sec}} \, \frac{M\e{air sec}}{M\e{vapeur d'eau}} \]

\sk
L'expression ci-dessus peut être grandement simplifiée. L'eau est un composant minoritaire dans l'atmosphère terrestre~: l'ordre de grandeur de~$r$ est de l'ordre de~$0$ à~$20$~g~kg$^{-1}$. On a donc toujours~$r \ll 1$ et~$e \ll P$, soit~$P-e \simeq P$. Ainsi la masse d'air sec~$m\e{air sec}$ dans la parcelle est en très bonne approximation égale à la masse d'air~$m\e{air}$ dans la parcelle, ce qui vaut également pour la masse molaire. Le rapport de mélange en vapeur d'eau s'écrit alors~$r = \frac{m\e{vapeur d'eau}}{m\e{air}} \simeq \frac{m\e{vapeur d'eau}}{m\e{air sec}}$. L'expression ci-dessus se simplifie donc en
\[ r = \frac{M\e{vapeur d'eau}}{M\e{air}} \, \frac{e}{P} \qquad \Rightarrow \qquad \boxed{ r \simeq 0.622 \, \frac{e}{P} } \]
Cette équation signifie que, pour une pression~$P$ donnée, le rapport de mélange de vapeur d'eau~$r$ est en bonne approximation proportionnel à la pression partielle de vapeur d'eau~$e$.

\sk
%\subsection{Évaporation, Saturation}
\subsection{Equilibre liquide / vapeur}

\sk
L'\voc{évaporation} est l'échappement de molécules d'eau depuis une phase liquide vers une phase gazeuse. A l'interface liquide-gaz, sous l'effet de l'agitation thermique, certaines molécules d'eau dans le liquide vont voir les liaisons hydrogène rompues avec leurs plus proches voisins. L'échappement est ainsi plus facile pour des molécules ayant une énergie cinétique importante~: le taux d'évaporation~$\mathcal{E}$ à partir d'une surface dépend donc de la température de l'eau. 

\sk
La \voc{condensation} est le passage de molécules d'eau de la phase gazeuse à la phase liquide. A l'interface liquide-gaz, certaines molécules d'eau dans le gaz vont se lier à des molécules d'eau dans le liquide par le biais de liaisons hydrogène. Le taux de condensation~$\mathcal{C}$ dépend de la pression de la phase gazeuse, à savoir~$e$ dans le cas de la vapeur d'eau. 

\sk
Soit une enceinte remplie d'air totalement sec, c'est-à-dire qui ne contient aucune molécule d'eau sous forme vapeur. On introduit dans cette enceinte une quantité donnée d'eau liquide. Comme décrit ci-dessus, il va y avoir spontanément évaporation avec un taux d'évaporation~$\mathcal{E}$ (supposé constant) à la surface du liquide, d'autant plus que la température de l'eau est élevée. Des molécules d'eau s'échappent donc dans l'espace au-dessus du liquide et forment une phase gazeuse dont la pression partielle~$e$ augmente peu à peu. Des molécules de cette phase gazeuse subissent à leur tour un phénomène de condensation et repassent en phase liquide. Le taux de condensation~$\mathcal{C}$ est, au début de l'expérience, très petit devant~$\mathcal{E}$ car la pression partielle~$e$ est extrêmement faible. Puisque l'évaporation domine la condensation, le bilan est donc en faveur d'une augmentation des molécules sous forme gazeuse. Néanmoins, plus le nombre de molécules d'eau sous forme gazeuse augmente, plus la pression partielle~$e$ augmente, donc plus le taux de condensation~$\mathcal{C}$ augmente. Ce phénomène va continuer jusqu'à atteindre un équilibre stationnaire où les taux de condensation~$\mathcal{C}$ et~$\mathcal{E}$ se compensent. Cet équilibre est appelé \voc{équilibre liquide-vapeur}, on parle également souvent, par abus de langage, de \voc{\ofg{saturation}}.% ou de \voc{\ofg{conditions saturées}}. 

\sk
\subsection{Grandeurs saturantes}

\sk
\subsubsection{Définition}

\sk
La pression partielle~$e$ pour laquelle l'équilibre liquide-vapeur est atteint est appelée \voc{pression de vapeur saturante} que l'on note~$e\e{sat}$. Tant que~$e<e\e{sat}$, les échanges par évaporation dominent les échanges par condensation et $e$ augmente jusqu'à atteindre~$e\e{sat}$. Lorsque~$e=e\e{sat}$, la quantité de vapeur d'eau dans l'enceinte n'augmente plus\footnote{Il est important de noter que cet état stationnaire n'est pas dénué d'échanges entre les phases liquide et gaz par condensation et évaporation. Par analogie, on peut penser au remplissage d'une baignoire équipée d'un siphon~: le niveau de l'eau est constant à l'état stationnaire bien qu'il y ait en permanence un apport d'eau par le robinet et une perte d'eau par le siphon -- l'état stationnaire signifie juste que ces échanges se compensent.}. Ainsi, si l'on considère une enceinte avec de l'eau sous forme vapeur et liquide
\begin{citemize}
\item si la pression partielle de vapeur d'eau~$e$ dans l'enceinte est supérieure à la pression de vapeur saturante~$e\e{sat}$, il y a condensation jusqu'à ce que~$e=e\e{sat}$.
\item si la pression partielle de vapeur d'eau~$e$ dans l'enceinte est inférieure à la pression de vapeur saturante~$e\e{sat}$, il y a évaporation jusqu'à ce que~$e=e\e{sat}$.
\end{citemize}
Si l'on considère une enceinte contenant de l'eau sous forme vapeur uniquement, le premier point est toujours valable alors que le second point n'est pas vrai~: si la pression partielle de vapeur d'eau~$e$ dans l'enceinte est inférieure à la pression de vapeur saturante~$e\e{sat}$, rien ne se passe, car aucune phase liquide ne peut être évaporée. La pression partielle de vapeur d'eau~$e$ est donc toujours inférieure ou égale à la pression de vapeur saturante~$e\e{sat}$. 

\sk
Le rapport de mélange~$r\e{sat}$ correspondant à l'équilibre liquide/vapeur où $e=e\e{sat}$ est appelé \voc{rapport de mélange saturant}. D'après l'équation encadrée à la section précédente, on a 
\[ r\e{sat} \simeq 0.622 \, \frac{e\e{sat}}{P} \]
Les mêmes raisonnements qu'avec les pressions partielles~$e$ et~$e\e{sat}$ peuvent être faits avec les rapports de mélange~$r$ et~$r\e{sat}$. Ces quantités servent à définir l'\voc{humidité relative}~$H$
\[ \boxed{ H = \frac{e}{e\e{sat}} = \frac{r}{r\e{sat}} } \]
De ce qui précède, on déduit que l'humidité~$H$ est toujours inférieure à~$1$ ($100\%$) et que, lorsqu'il y a équilibre liquide/vapeur (\ofg{conditions saturées}), $H$ vaut~$1$ ($100\%$).

\sk
\subsubsection{Variation avec la température}

\sk
La pression de vapeur saturante~$e\e{sat}$ augmente exponentiellement avec la température~$T$ du gaz d'après la \voc{relation de Clausius-Clapeyron}\footnote{La pression de vapeur saturante est proportionnelle à la probabilité de rupture d'une liaison, elle-même variant exponentiellement suivant la température.}. Ainsi elle double pour une élévation de température de~$10$~K. Plus le gaz dans l'enceinte est chaud, plus la quantité de vapeur d'eau au terme de l'expérience est élevée. En pratique, le terme~$e\e{sat}$, qui varie exponentiellement avec la température~$T$, domine très fréquemment les variations de pression~$P$. Ainsi, en bonne approximation, le rapport de mélange saturant~$r\e{sat}$ varie également exponentiellement avec la température~$T$. 

\sk
La dépendance de~$e\e{sat}$ avec~$T$ permet par ailleurs de définir la \voc{température de rosée}~$T\e{rosée}$ associée à une valeur donnée de la pression partielle~$e$ de l'eau. Il s'agit de la température~$T\e{rosée}$ à laquelle la pression partielle~$e$ devient saturante, c'est-à-dire qui vérifie
\[ \boxed{ e\e{sat}(T\e{rosée}) = e } \]

%\subsection{Ébullition} L'ébullition est un cas particulier: des bulles de gaz se forment à l'{\em intérieur} du liquide bouillant. Dans le cas de l'eau, ce gaz est donc de la vapeur d'eau. La pression dans ces bulles est égale à celle du liquide, soit à peu près la pression atmosphérique si le liquide est en contact avec l'air. Les bulles sont d'autre part stables si leur pression est supérieure à la pression saturante. L'ébullition se produit donc à une température $T_b$ telle que \[e_{sat}(T_b)=P_{atm}\]

\sk
\subsection{Déplacement d'équilibre et application aux gouttelettes nuageuses}

\sk
On applique ici les raisonnements de la section précédente pour une interface plane liquide/vapeur à une goutte sphérique comme rencontrée dans les brouillards ou les nuages. La réalité est un peu plus complexe et fait intervenir les concepts de noyaux de condensation et de sursaturation, qui ne sont pas abordés dans ce cours. Les raisonnements présentés ci-dessous restent cependant valables au premier ordre.
%%Dans l'atmosphère, loin de la surface, il n'y a pas d'interface liquide/gaz permanente. Si $e<e_{sat}$, il n'y a ni condensation ni évaporation. Si $e$ devient supérieure à $e_{sat}$, il y a condensation sous forme de gouttes d'eau liquide (qui se forment plus vite qu'elles ne s'évaporent). Ces gouttes s'évaporent dès que $e<e_{sat}$

\sk
Soit une parcelle d'air à la température~$T_0$ et à la pression~$P$. Elle contient de la vapeur d'eau en équilibre avec des gouttelettes d'eau en suspension, en pratique cela correspond à une parcelle dans laquelle des gouttelettes nuageuses se sont formées. A l'équilibre liquide-vapeur, la pression partielle de vapeur d'eau dans la parcelle vaut~$e=e\e{sat}(T_0)$, le rapport de mélange de vapeur d'eau vaut~$r=r\e{sat}(T_0)$ et l'humidité~$H$ vaut~$100\%$. Si la température de la parcelle change, il y a déplacement de l'équilibre liquide/vapeur\footnote{Par abus de langage, on dit parfois que \ofg{l'air chaud peut contenir plus de vapeur d'eau que l'air froid}. Il est autorisé de garder cette phrase en tête en tant que moyen mnémotechnique, cependant elle est incorrecte physiquement car elle ne rend pas compte de l'équilibre liquide/vapeur.}.
\begin{finger}
\item
Supposons que l'on \underline{chauffe la parcelle} à une température~$T\e{c}>T_0$. Sa pression partielle en vapeur d'eau~$e$ est toujours proche de~$e\e{sat}(T_0)$, mais la pression de vapeur saturante~$e\e{sat}$ a augmenté de façon exponentielle de~$e\e{sat}(T_0)$ à~$e\e{sat}(T\e{c})$. On est alors dans la situation où~$e < e\e{sat}(T\e{c})$, donc~$H < 1$. Il y a alors évaporation d'eau liquide jusqu'à ce qu'un nouvel équilibre liquide/vapeur soit atteint, où~$e = e\e{sat}(T\e{c})$. Une façon équivalente de décrire ce déplacement d'équilibre est de dire que, lorsque la parcelle chauffe, la quantité de vapeur d'eau~$r=r\e{sat}(T_0)$ devient très inférieure à la quantité de vapeur d'eau à saturation~$r\e{sat}(T\e{c})$. De l'eau liquide doit passer sous forme gazeuse par évaporation pour compenser ce déséquilibre, de manière à ce que la quantité de vapeur d'eau~$r$ dans la parcelle augmente à~$r\e{sat}(T\e{c})$. Ainsi lorsque l'on chauffe la parcelle, des gouttelettes nuageuses disparaissent, le nuage se dissipe.
\item
Supposons à l'inverse que l'on \underline{refroidisse la parcelle} à une température~$T\e{f}<T_0$. La pression de vapeur saturante~$e\e{sat}$ a diminué de façon exponentielle de~$e\e{sat}(T_0)$ à~$e\e{sat}(T\e{f})$. On est alors dans la situation où~$e > e\e{sat}(T\e{f})$, donc~$H > 1$, qui est impossible. Il y a alors condensation d'eau liquide jusqu'à ce qu'un nouvel équilibre liquide/vapeur soit atteint, où~$e = e\e{sat}(T\e{f})$. Autrement dit, lorsque la parcelle refroidit, la quantité de vapeur d'eau~$r=r\e{sat}(T_0)$ devient très supérieure à la quantité de vapeur d'eau à saturation~$r\e{sat}(T\e{f})$. De l'eau sous forme gazeuse doit passer sous forme liquide par condensation pour compenser ce déséquilibre, de manière à ce que la quantité de vapeur d'eau~$r$ dans la parcelle diminue à~$r\e{sat}(T\e{f})$. Ainsi lorsque l'on refroidit la parcelle, de nouvelles gouttelettes nuageuses apparaissent, le nuage s'épaissit.
\end{finger}

\sk
Le second point s'applique également au cas d'une parcelle d'air ne contenant pas initialement de gouttelettes nuageuses. 

\sk
Le chapitre précédent a proposé une expression du premier principe qui distingue deux manières de faire varier la température d'une parcelle atmosphérique d'air sec~: transformations isobares et transformations adiabatiques. On peut désormais illustrer la formation de nuages associée à chacune des transformations appliquée à une parcelle de rapport de mélange en vapeur d'eau~$r \neq 0$ qui reste constant au cours de la transformation.
\begin{finger}
\item Lorsqu'une parcelle d'air proche de la surface subit un refroidissement isobare à la tombée de la nuit, sous l'influence du flux radiatif infrarouge, des gouttelettes nuageuses se forment car le rapport de mélange saturant~$r\e{sat}$ diminue jusqu'à devenir plus faible que~$r$. Il s'agit du brouillard nocturne~; la formation de rosée obéit à un principe similaire. La température de rosée~$T\e{rosée}$ peut ainsi être définie comme la température à laquelle la condensation se produit suite à un refroidissement isobare. 
\item Lorsqu'une parcelle d'air subit une élévation adiabatique, à cause par exemple de la présence d'une montagne, elle se refroidit et le rapport de mélange saturant~$r\e{sat}$ diminue. Le rapport de mélange~$r$ peut alors éventuellement devenir supérieur à~$r\e{sat}$ et des gouttelettes se forment pour que~$r=r\e{sat}$. Ceci explique par exemple que les montagnes soient souvent couvertes de nuages.
\end{finger}
Le chapitre suivant se propose de reprendre avec plus de précisions la formation des nuages.

\mk
\section{Evolution hors équilibre d'une parcelle d'air}

\sk
\subsection{Transformations pseudo-adiabatiques}

\sk
On considère tout d'abord une parcelle d'air (contenant de la vapeur d'eau) en évolution isobare. Le premier principe appliqué à la parcelle indique donc
\[ \dd T = \frac{1}{C_P} \, \delta q \]
Lors de l'évaporation, les molécules d'eau liquide voient les liaisons hydrogène avec leurs proches voisins être brisées. Le passage de l'eau de la phase liquide à la phase vapeur consomme donc de l'énergie\footnote{On peut s'en convaincre en notant la sensation de froid immédiate que provoque la sortie d'un bain à cause de l'évaporation de l'eau liquide sur le corps mouillé~; ou en se souvenant que lorsque l'on souffle sur la soupe pour la refroidir, c'est précisément pour favoriser l'évaporation et la refroidir efficacement.}~: pour l'air qui compose la parcelle, $\delta q < 0$ et il y a refroidissement. 
A l'inverse, lors de la condensation, les molécules d'eau sous forme gazeuse créent des liaisons hydrogène avec les molécules d'eau de la phase liquide pour atteindre un état énergétique plus faible. Le passage de l'eau de la phase vapeur à la phase liquide libère donc de l'énergie~: pour l'air qui compose la parcelle, $\delta q > 0$ et il y a chauffage.

\sk
L'énergie~$\delta q$ consommée ou libérée par les changements d'état s'appelle~\voc{chaleur latente}, on la note~$\delta q\e{latent}$. Si une masse de vapeur~$\dd m\e{vapeur d'eau}$ est condensée ou évaporée, on a
\[ \delta q\e{latent} = \frac{- L \, \dd m\e{vapeur d'eau}}{m\e{air sec}} \qquad \Rightarrow \qquad \boxed{ \delta q\e{latent} = - L \, \dd r } \]
où~$L$ est la chaleur latente massique en~J~kg$^{-1}$. La formule ci-dessus comporte un signe négatif. La quantité~$\delta q\e{latent}$ est positive lorsqu'il y a condensation (le rapport de mélange en vapeur d'eau diminue $\dd r < 0$) et négative lorsqu'il y a évaporation (le rapport de mélange en vapeur d'eau augmente $\dd r > 0$).

\sk
On considère désormais une parcelle d'air en évolution adiabatique, à l'exception des échanges de chaleur latente~: $\delta q = \delta q\e{latent}$. On appelle une telle transformation \voc{pseudo-adiabatique} ou encore \voc{adiabatique saturée}. On fait l'approximation que la chaleur latente consommée ou dégagée est seulement échangée avec l'air sec~:
\begin{citemize}
\item La chaleur latente consommée/dégagée n'est pas utilisée pour refroidir/chauffer les gouttes d'eau présentes.
\item On néglige les pertes de masse par précipitation~: la masse d'air sec considérée est constante.
\end{citemize}
Pour une telle transformation, la variation de température s'écrit ainsi
\[ \dd T = \frac{R}{C_P} \, \frac{T}{P} \, \dd P - \frac{L}{C_P} \, \dd r \]

\sk
\subsection{Profil vertical saturé}

\sk
Considérons une parcelle en ascension adiabatique saturée (et non plus sèche comme dans la section~\ref{adiabsec}). Pour une parcelle saturée, c'est-à-dire à l'équilibre liquide/vapeur, l'équation qui précède peut s'écrire, en utilisant l'équilibre hydrostatique
\[ C_P \, \dd T + g \, \dd z + L \, \dd r = 0 \]
Or, puisque la parcelle est saturée, on a~$r = r\e{sat}(T)$ et on peut écrire $\dd r\e{sat} = \ddf{r\e{sat}}{T} \, \dd T$. On a alors
\[ \left( C_P + L \, \ddf{r\e{sat}}{T} \right) \dd T + g \, \dd z = 0\]
Cette expression est similaire au cas sec, à l'exception notable du terme supplémentaire~$L \, \ddf{r\e{sat}}{T}$ lié aux échanges latents. On peut alors obtenir le profil vertical adopté dans l'atmosphère saturée par une parcelle ne subissant pas d'échange de chaleur avec l'extérieur autre que les échanges de chaleur latente
\[  \ddf{T}{z}  = \Gamma\e{saturé} \qquad \text{avec} \qquad \Gamma\e{saturé} = \frac{-g}{C_P+L \, \ddf{r\e{sat}}{T} } \]
On a vu que $\ddf{r\e{sat}}{T}$ est toujours positif, on en déduit donc
\[ \boxed{ \Gamma\e{saturé} > \Gamma\e{sec} \qquad \text{ou} \qquad |\Gamma\e{saturé}| < |\Gamma\e{sec}| } \]
A cause du dégagement de chaleur latente, la température diminue moins vite pour une parcelle saturée en ascension que pour une parcelle non saturée. Le calcul pour l'atmosphère terrestre montre que
\[ \Gamma\e{saturé} = -6.5 \, \text{K~km}^{-1} \] 
ce qui correspond à la valeur observée dans la troposphère [Figure~\ref{fig:tempvert}].

\sk
La constatation que~$\Gamma\e{saturé}$ correspond au profil d'environnement effectivement mesuré dans la troposphère appelle un commentaire important. Les profils verticaux secs ou saturés sont ceux suivis par une parcelle en ascension~: autrement dit, ils donnent les variations de~$T\e{p}$ avec l'altitude~$z$. D'un point de vue instantané, ils ne correspondent pas aux profils d'environnement~$T\e{e}$ tels qu'ils peuvent être par exemple mesurés par des ballons-sonde lâchés dans l'atmosphère. La parcelle n'est pas nécessairement à l'équilibre thermique avec l'environnement. On peut néanmoins constater sur la figure~\ref{fig:tempvert} que la température de l'environnement diminue avec une pente très proche de~$\Gamma\e{saturé}$. Ceci s'explique par le fait que cette figure montre une moyenne sur tout le globe à toutes les saisons. La situation moyenne ainsi décrite correspond aux mouvements d'une multitude de parcelles en ascension qui finissent par définir l'environnement atmosphérique\footnote{Ce phénomène porte le nom d'ajustement convectif.}. Pour comprendre la formation des nuages, et plus généralement les mouvements atmosphériques, il faut néanmoins se placer dans le cas local où l'équilibre thermique n'est pas vérifié. C'est l'objet de la section suivante.
%Comme pour le cas adiabatique, on peut aussi intégrer l'équation pour obtenir:
%\begin{equation} e_h=C_PT+gz+Lr=cste \label{estath} \end{equation}  
%La quantité $e_h$ est appelée {\em énergie statique humide} et est conservée
%pour des mouvements adiabatiques ($r$ et $e_s$ sont séparément conservés) ou
%saturés (pseudo-adiabatiques).

\mk
\section{Stabilité et instabilité verticale}

\sk
\subsection{Force de flottaison}

\sk
Soit une parcelle dont la température $T\e{p}$ n'est pas égale à celle de l'environnement~$T\e{e}$, que ce soit sous l'effet d'un chauffage diabatique (par exemple~: chaleur latente, effets radiatifs) ou d'une compression / détente adiabatique. On reprend le calcul réalisé précédemment pour l'équilibre hydrostatique, avec la différence notable que l'on n'est plus dans le cas statique~: on étudie le mouvement vertical d'une parcelle. 

\sk
La somme des forces massiques s'exerçant sur la parcelle suivant la verticale est
\[ - g  - \frac{1}{\rho\e{p}}  \, \Dp{P\e{e}}{z} \]
où~$\rho\e{p}$ est la masse volumique de la parcelle. L'environnement est à l'équilibre hydrostatique donc
\[ \Dp{P\e{e}}{z} = - \rho\e{e} \, g \]
Ainsi la résultante~$F_z$ des forces massiques qui s'exercent sur la parcelle selon la verticale vaut
\[ F_z = g \, \left( \frac{\rho\e{e}}{\rho\e{p}} - 1 \right) = g \, \frac{\rho\e{e}-\rho\e{p}}{\rho\e{p}} \]
En utilisant l'équation du gaz parfait pour la parcelle~$\rho\e{p}=P/RT\e{p}$ et l'environnement~$\rho\e{e}=P/RT\e{e}$, on a
\[ \boxed{ F_z = g \, \frac{T\e{p}-T\e{e}}{T\e{e}} } \]
La résultante des forces est donc dirigée vers le haut, donc la parcelle s'élève, si la parcelle est plus chaude (donc moins dense) que son environnement. 
Elle est dirigée vers le bas si la parcelle est plus froide (donc plus dense) que son environnement.
En d'autres termes, on écrit ici la version météorologique de la force ascendante ou descendante 
provoquée par la poussée d'Archimède, également appelée \voc{force de flottaison}.

\sk
\subsection{Stabilité et instabilité}

\sk
Ces considérations permettent de définir le concept de stabilité et instabilité verticale de l'atmosphère.
On considère l'atmosphère à un endroit donné de la planète, à une saison donnée, à une heure donnée de la journée.
On suppose que la température de l'environnement varie linéairement avec l'altitude
\[ \ddf{T\e{e}}{z} = \Gamma\e{env} \]
A une altitude~$z_0$ proche de la surface, la température de l'environnement est~$T\e{e}(z_0)=T_0$.

\sk
On considère une parcelle initialement à l'altitude~$z_0$ dont la température initiale~$T\e{p}(z_0)$ est également~$T_0$. On suppose que la parcelle subit une ascension verticale d'amplitude~$\delta z > 0$. Le profil de température suivi par la parcelle lors de son ascension est
\[ \ddf{T\e{p}}{z} = \Gamma\e{parcelle} \]
\begin{citemize}
\item Si la parcelle est non saturée, elle suit un profil adiabatique sec tel que $\Gamma\e{parcelle} = \Gamma\e{sec} \simeq - 10 \, \text{K/km}$.
\item Si elle est saturée, elle suit un profil adiabatique saturé tel que $\Gamma\e{parcelle} = \Gamma\e{saturé} \simeq - 6.5 \, \text{K/km}$. 
\end{citemize}
On rappelle qu'en général, à l'échelle où l'on étudie les mouvements de la parcelle
\[ \Gamma\e{parcelle} \neq \Gamma\e{env} \]

\sk
Quel est l'effet de la perturbation~$\delta z > 0$ sur le mouvement de la parcelle~? A l'altitude~$z_0 + \delta z$, les températures de la parcelle et de l'environnement sont respectivement
\[ T\e{p}(z_0 + \delta z) = T_0 + \Gamma\e{parcelle} \, \delta z 
\qquad \text{et} \qquad
T\e{e}(z_0 + \delta z) = T_0 + \Gamma\e{env} \, \delta z \]
\begin{finger}
\item Si $\Gamma\e{parcelle} > \Gamma\e{env}$, la température~$T\e{e}$ de l'environnement décroît plus vite que la température~$T\e{p}$ de la parcelle. Il en résulte que~$T\e{p}(z_0 + \delta z) > T\e{e}(z_0 + \delta z)$ et le mouvement de la parcelle est ascendant. La perturbation initiale est donc amplifiée par les forces de flottabilité. On parle de \voc{situation instable}. La situation est d'autant plus instable que la température de l'environnement décroît rapidement avec l'altitude. Lorsque la situation est instable, les mouvements verticaux sont amplifiés~: on parle parfois de \voc{situation convective}.
\item Si $\Gamma\e{parcelle} < \Gamma\e{env}$, la température~$T\e{e}$ de l'environnement décroît moins vite que la température~$T\e{p}$ de la parcelle. Il en résulte que~$T\e{p}(z_0 + \delta z) < T\e{e}(z_0 + \delta z)$ et le mouvement de la parcelle est descendant. La perturbation initiale n'est donc pas amplifiée et la parcelle revient à son état initial. On parle de \voc{situation stable}. La stabilité est d'autant plus grande que la température de l'environnement décroît lentement (ou augmente, dans le cas d'une inversion de température). Lorsque la situation est stable, les mouvements verticaux sont inhibés.
\end{finger}
La résultante des forces verticales s'exerçant sur la parcelle peut s'écrire en fonction des taux de variation~$\Gamma$ de la température
\[ F_z = g \, \frac{\Gamma\e{parcelle}-\Gamma\e{env}}{T\e{env}} \, \delta z \]

\sk
On peut illustrer la stabilité/instabilité atmosphérique dans le cas des polluants émis proche de la surface par les activités humaines [Figure~\ref{fig:pollution}]. Dans l'après-midi, du fait que le sol est chaud, le profil d'environnement est tel que la situation est très instable~: les mouvements verticaux qui transportent les polluants plus haut dans l'atmosphère sont encouragés et les polluants ne restent pas proches de la surface. A l'inverse, en soirée, du fait que le sol refroidit radiativement, le profil d'environnement est tel que la situation est stable~: les mouvements verticaux qui pourraient transporter les polluants plus haut dans l'atmosphère sont inhibés et les polluants sont confinés proche de la surface. Pour être moins exposé aux polluants dans les zones urbaines, il est donc préférable d'y effectuer son jogging en fin de matinée plutôt qu'en soirée !

\figside{0.65}{0.25}{decouverte/cours_meteo/inversion-temperature.png}{Stabilité et pollution atmosphérique. On notera que cette figure est très illustrative, mais présente une situation simplifiée. Le transport vertical de polluants dans l'atmosphère est en réalité inhibé dès que la couche atmosphérique est stable, ce qui est plus général que considérer uniquement une inversion thermique comme à droite de la figure. Source~: Airparif}{fig:pollution} 





%%%% CHAPITRE 7
%%\def\thechapter{7} \title{Sciences de l'atmosphère\\ Chapitre \thechapter} \author{Aymeric SPIGA\\ ~\\} \maketitle %% commenter la ligne en mode livre complet
\chapter{Nuages}

\dictum[William Shakespeare, 1591]{Car tout nuage n’engendre pas tempête.}
%%Henry VI part 3 For every cloud engenders not a storm.

\bk
Lorsqu'il souhaite se représenter l'atmosphère, l'homme songe naturellement aux nuages, probablement influencé par ses observations quotidiennes et les allusions poétiques qu'elles ne manquent pas de provoquer. La physique offre un cadre pour comprendre les phénomènes nuageux et leurs implications pour le climat. Les chapitres précédents ont préparé certaines bases permettant d'aborder au premier ordre la formation des nuages.  

\mk
\section{Phénoménologie des nuages}

\sk
\subsection{Composition}
	Un \voc{nuage} se définit comme un regroupement localisé de gouttelettes d'eau et/ou de cristaux de glace ou de neige en suspension dans l'atmosphère. 

\figside{0.5}{0.2}{decouverte/cours_meteo/gouttes.png}{Vue schématique des composants d'un nuage pluvieux chaud. Les tailles indicatives sont précisées afin de donner une idée des quelques ordres de grandeur en taille qui séparent une gouttelette nuageuse d'une goutte de pluie. La traduction des termes est la suivante: \emph{cloud-condensation nuclei} \donc~noyaux de condensation, \emph{moisture droplets} \donc~gouttelettes nuageuses, \emph{typical raindrop} \donc~goutte de pluie. Les diamètres sont indiqués en microns (10$^{-6}$~m). Source: à partir de MacDonald Adv. Geophys. 1958.}{fig:cloud}

\begin{finger}
\item Dans les nuages chauds, la vapeur d'eau condense en gouttelettes nuageuses dont la taille est de quelques microns. Les conditions locales de saturation (autrement dit, l'équilibre liquide-vapeur) déterminent le taux de condensation et de croissance des gouttes. Même si cela n'est pas l'objet du présent cours, il convient de noter que l'interface courbée des gouttes a un certain coût énergétique~: il est difficile de former des gouttes à moins d'atteindre une sursaturation très élevée (c'est-à-dire une humidité très supérieure à~$1$). La formation des gouttelettes nuageuses par condensation est par contre facilitée par la présence de \voc{noyaux de condensation} (par exemple, les poussières atmosphériques). Ensuite, les gouttelettes nuageuses peuvent, par collision ou coalescence, croître de plusieurs ordres de grandeurs en taille pour donner naissance à des gouttes de pluie de plusieurs millimètres de large qui donnent lieu à des précipitations. La figure~\ref{fig:cloud} donne un aperçu très schématique de l'intérieur d'un nuage pluvieux chaud.
\item Dans les nuages froids, i.e. ceux qui se trouvent dans une zone où la température est plus faible que 0 degrés celsius, règne un équilibre à trois phases (solide, liquide, gaz). Les gouttelettes nuageuses et cristaux de glace peuvent se former par condensation directe (des phénomènes de surfusion expliquent que les gouttelettes d'eau restent à l'état liquide). La neige se forme par agrégation de cristaux de glace. Par accrétion, plus précisément coalescence liquide sur glace, la grêle peut apparaître dans un nuage froid et exceptionnellement former des espèces précipitantes d'un diamètre important.
\end{finger}

\sk
La formation des gouttes et cristaux dans un nuage obéit à un ensemble de lois dites microphysiques dont la complexité dépasse le présent cours. Pour simplifier la description, on s'intéresse souvent aux nuages chauds uniquement. De plus, on se place dans le cas d'un équilibre thermodynamique liquide-vapeur pour une interface plane, pour laquelle la valeur maximum de l'humidité est~$1$ et suffit à déclencher la formation d'un nuage dans l'atmosphère.




 
\sk
\subsection{Généralités à partir d'images satellite}

\sk
Les nuages couvrent très souvent au moins la moitié du globe et participent donc activement à l'albédo terrestre. La figure~\ref{fig:geostat} en atteste, ainsi que de la diversité des morphologies nuageuses. On peut y déceler les formes imposées par la dynamique atmosphérique, que l'on élucide en partie au chapitre suivant~:
\begin{citemize}
\item les vents qui se développent à l'échelle planétaire et convergent à proximité de l'équateur~;
\item les cyclones tropicaux dans l'Atlantique~;
\item les circulations de mousson dans la péninsule indienne~;
\item les fronts nuageux étirés dans l'hémisphère sud, avec parfois un enroulement à l'extrémité du front.
\end{citemize}

\figun{0.99}{0.31}{decouverte/cours_meteo/pentasat_IRcouleur_1800_220905.jpg}{Mosaïque IRC des géostationnaires (2 satellites américains, 2 satellites européens, 1 satellite japonais) pour le 22 septembre 2005 à 18h00 UTC. Source: base de données SATMOS \url{http://www.satmos.meteo.fr}.}{fig:geostat}

\sk
Par ailleurs, l'altitude des nuages qui se forment dans l'atmosphère peut varier significativement. L'image reproduite en figure~\ref{fig:geostat} provient de canaux infrarouge des instruments du satellite qui reçoivent le rayonnement thermique émis par les nuages (et non la diffusion du rayonnement solaire incident dans le visible). Les nuages dont le sommet est élevé, donc particulièrement froid, sont indiqués en blanc; les nuages de sommet plus chaud car moins élevés sont indiqués en gris. Les nuances entre les diverses altitudes atteintes par les nuages peuvent être notamment appréciées sur la figure pour le continent eurasien et les hautes latitudes sud.

\sk
\subsection{Classification par l'observation au sol}

\sk
Historiquement, les nuages ont été classés et nommés avant que leurs mécanismes de formation ne soient compris. C'est la raison pour laquelle les (nombreux) noms décrivant les nuages se réfèrent principalement à leur apparence et leur propension à donner lieu à des précipitations. Le classement phénoménologique, c'est-à-dire guidé par l'observation des phénomènes, le plus aisé à prendre en considération pour les nuages, fait référence à leur altitude de formation donc \emph{in fine} leur composition (Figure~\ref{fig:phenomclouds}).

\figun{0.6}{0.37}{decouverte/cours_meteo/nuages.jpg}{Classification phénoménologique des nuages par leur altitude de formation. Source: Météo-France.}{fig:phenomclouds}

\begin{description}

\item{\textbf{Nuages supérieurs}} Les \voc{cirrus} sont des nuages blancs organisés en filaments ou bandes étroites. Ils se forment dans la haute troposphère ($6-12$~km d'altitude) et sont composés de cristaux de glace dispersés. Lorsqu'ils s'organisent en bancs d'élements un peu plus épais et d'aspect ondulé, on parle de cirrocumulus. Lorsqu'ils s'organisent en un voile blanchâtre transparent qui couvre au moins partiellement le ciel, on parle de cirrostratus. Ces nuages ne sont pas accompagnés de précipitations.

\item{\textbf{Nuages moyens}} Les altocumulus, altostratus et nimbostratus se forment dans la troposphère moyenne entre~$2$ et~$5$~km d'altitude. Tous ces nuages moyens sont constitués de gouttelettes d'eau (parfois surfondues) et de cristaux de glace ou de neige. Les altocumulus prennent l'apparence de bancs d'éléments blancs ou gris de petite largeur. Les altostratus forment une couche grisâtre uniforme peu épaisse couvrant au moins partiellement le ciel. Le nimbostratus est l'archétype du nuage de mauvais temps~: il consiste en une couche nuageuse grise et sombre, parfois rendue floue par les précipitations de pluie ou de neige et souvent assez épaisse pour masquer le soleil. Les altostratus et nimbostratus peuvent conduire à des précipitations de pluie ou de neige, plutôt intenses dans le deuxième cas.

\item{\textbf{Nuages inférieurs}} Les nuages inférieurs se forment à une altitude au-dessus de la surface inférieure à~$2$~km. Ils sont composés majoritairement de gouttelettes d'eau. Le \voc{stratus} est une couche nuageuse grise, dense et uniforme, qui peut donner lieu à de la bruine ou du brouillard. Les \voc{cumulus} sont des nuages bien délimités, blancs, assez denses et présentant parfois un aspect bourgeonnant qui correspond à une certaine extension verticale (on parle dans ce cas de cumulus congestus, qui peuvent donner lieu à des cumulonimbus, voir ci-dessous). S'ils prennent la forme de petits éléments blancs ou gris organisés en bancs, on parle de stratocumulus. Les cumulus ne donnent pas toujours lieu à des précipitations et, le cas échéant, il s'agit surtout d'averses.

\end{description}

Il existe un type de nuage très important qui échappe à cette classification~: le \voc{cumulonimbus}. Il est aussi qualifié de nuage d'orage. Ce nuage montre un développement vertical considérable (on parle souvent de \og tour convective \fg), qui part de la basse troposphère pour se prolonger jusque dans la haute troposphère. La partie supérieure a souvent une apparence étalée par rapport au reste du nuage (\og enclume \fg). La partie inférieure est très sombre en raison de l'atténuation du rayonnement solaire incident liée à l'extension verticale importante du nuage. Les précipitations associées aux cumulonimbus sont de fortes averses de pluie, de neige ou de grêle.

\mk
\section{Quelques éléments de physique des nuages}

\sk
\subsection{Classification physique des nuages} \label{classphys}
	\sk
La cause générale de la formation d'un nuage est le refroidissement d'une masse d'air. Dans la grande majorité des cas, les transformations qui provoquent ce refroidissement sont soit isobares, soit adiabatiques. Ceci est illustré par la formulation du premier principe adoptée dans les chapitre précédents.
%\[ \underbrace{\textcolor{white}{\frac{R^2}{C_P}} \dd T \textcolor{white}{\frac{R}{C_P}}}_{\text{variation de température de la parcelle}} = \underbrace{\frac{R}{C_P} \, \frac{T}{P} \, \dd P}_{\text{travail expansion/compression}} + \underbrace{\frac{1}{C_P} \, \delta q}_{\text{chauffage diabatique}} \]

\sk
Suivant la transformation qui va donner naissance au nuage, la morphologie de ce dernier est très différente. Par opposition à la classification phénoménologique donnée en début de chapitre, on peut alors établir une classification physique des nuages en fonction de la transformation thermodynamique qui leur donne naissance.
\begin{finger}
\item Si la transformation est isobare, cela signifie que le nuage se forme sans que la pression ne varie significativement dans la parcelle d'air considérée. D'après l'équilibre hydrostatique, et le fait que les variations de pression selon l'horizontale sont négligeables par rapport aux variations de pression selon la verticale, un tel phénomène ne peut exister que si l'altitude de la parcelle varie peu. La parcelle subit un refroidissement diabatique, qui peut être provoqué par exemple par les pertes radiatives (comme le brouillard nocturne cité au chapitre précédent) ou par un déplacement horizontal par les vents vers une région plus froide de l'atmosphère. Les nuages obtenus ne présentent pas de développement selon la verticale et sont même souvent particulièrement étendus selon l'horizontale. Il s'agit des nuages de type cirrus et stratus. Ces nuages étant organisés en strates, puisque non étendus selon la verticale, on dit qu'ils sont \voc{stratiformes}.
\item Si la transformation est adiabatique, le nuage se forme par le biais de parcelles d'air qui n'échangent pas de chaleur avec l'air environnant (ou, du moins, pour lesquelles les échanges diabatiques sont négligeables par rapport au terme de travail d'expansion/compression). Cette condition n'est réalisée que pour des parcelles subissant une variation de pression significative donc, selon l'équilibre hydrostatique, une variation d'altitude importante. Les nuages obtenus ne présentent donc pas d'étendue horizontale et sont au contraire très développés selon la verticale. Il s'agit des nuages de type cumulus et cumulonimbus. On les qualifie de nuages \voc{cumuliformes}.
\end{finger}
%%Ces nuages sont gouvernés par la convection~: la parcelle d'air plus chaude que son environnement a plus de facilité à refroidir en s'élevant qu'en échangeant de la chaleur avec son environnement.

\sk
La plupart des nuages se forment ainsi par refroidissement (isobare ou adiabatique) d'une masse d'air. Il est cependant à noter que le brassage d'une masse d'air chaude d'humidité voisine de~$100\%$ avec une masse d'air froide relativement sèche peut également donner naissance à des nuages de type stratiforme. %% avions. %% brise de terre.

%Quelques remarques sur les processus de formation: la plupart des nuages stratiforme se forment aussi au départ par refroidissement adiabatique -- même si le chauffage / refroidissement radiatif joue ensuite un rôle important dans le cycle de vie, en particulier dans les transitions type stratus -> strato-cumulus.
%Pour les nuages de couche limite, c'est le refroidissement lors du transport vertical turbulent ; pour les nuages type cirrus / altostratus / nimbi-stratus c'est le soulèvement en bloc de couches, dans les zones frontales par exemple. En fait, il n'y a guère que le brouillard radiatif ou d'advection qui soit vraiment d'origine adiabatique...
%Sinon, il y a un type un peu particulier qu'on appelle nuage de mélange, qu'on voit par exemple dans l'haleine qui condense en hiver. C'est ici le mélange de 2 masses d'air humides (mais non saturées) qui donne de l'air à une température intermédiaire mais saturé cette fois à cause de la forme de la courbe rsat(T).


\sk
\subsection{Développement d'un nuage cumuliforme}
	\figun{0.99}{0.6}{decouverte/cours_meteo/sounding.pdf}{Radiosondage obtenu à la station de Trappes le 23 juin 2005 à 12:00. Les données sont projetées sur un émagramme avec la pression atmosphérique en ordonnée et la température atmosphérique en abscisse. Des cumulonimbus d'orages très violents se sont développés dans la région parisienne ce jour~: cet événement peut être interprété à l'aide de l'émagramme qui permet de déterminer la base du nuage, le niveau de convection libre et le sommet théorique du nuage.}{fig:sounding}

	\sk
On s'intéresse ici au développement des nuages cumuliformes, en particulier les cumulonimbus. Le cas d'étude donné dans le radiosonsage exemple permet de suivre graphiquement les concepts de cette partie. Le point de départ est une parcelle d'air non saturée, c'est-à-dire dont l'humidité est inférieure à~$1$, située proche de la surface. On suppose que son rapport de mélange en vapeur d'eau~$r$ est conservé au cours de l'ascension. 

\sk
En premier lieu, de nombreux phénomènes atmosphériques vont provoquer une élévation de la parcelle que l'on considère initialement proche de la surface.
\begin{citemize}
\item{\textbf{Soulèvement frontal}} Un front est une variation marquée et localisée de température. Lorsqu'un front se déplace horizontalement, l'air chaud passe au-dessus de l'air froid de densité moindre. 
\item{\textbf{Soulèvement orographique}} La présence d'un relief face au vent force les parcelles d'air à s'élever.
\item{\textbf{Convection sèche}} Un sol très chaud l'après-midi peut induire un profil de température de l'environnement très instable proche de la surface. Dans ce cas, les mouvements verticaux sont amplifiés proche de la surface par la poussée d'Archimède (voir chapitre précédent).
%%% circulation thermique. brise de terre et brise de mer.
\end{citemize}
Tous ces mécanismes expliquent que des nuages cumuliformes sont souvent trouvés au-dessus de régions soumises au passage de fronts, montagneuses ou dont la surface est particulièrement chaude. Ces nuages évoluent parfois vers un état de type cumulonimbus.

\sk
En second lieu, lorsqu'une parcelle d'air est soulevée vers les plus hauts niveaux de l'atmosphère par les phénomènes atmosphériques précités, elle subit un refroidissement par détente adiabatique. Le taux de refroidissement de la parcelle est~$|\Gamma\e{sec}|$. Sur un émagramme tel celui de la figure~\ref{fig:sounding}, la parcelle suit une \voc{courbe adiabatique sèche}. Cette décroissance de la température de la parcelle au cours de l'ascension a pour principale conséquence d'abaisser la valeur de $r\e{sat}$, de par les variations exponentielles de cette quantité avec la température. Il en résulte que l'humidité relative~$H = r / r\e{sat}$ augmente. Si la quantité de vapeur d'eau initiale~$r$ et/ou le soulèvement de la parcelle sont suffisants, $H$ peut atteindre~$1$ au cours de l'ascension~: la parcelle devient alors saturée. Des gouttelettes nuageuses apparaissent par condensation, autrement dit un nuage se forme. Le niveau d'altitude ou de pression auquel la condensation se produit suite à un refroidissement par soulèvement adiabatique s'appelle le \voc{niveau de condensation} ou la \voc{base du nuage}. A ce stade, le nuage n'est pas encore nécessairement cumuliforme.
%%Pour la convection. Les bases des nuages sont horizontales, leurs sommets évoluent en fonction de la température.

\sk
En troisième lieu, si la parcelle continue son ascension au-delà du niveau de condensation, sa température ne décroît plus d'un taux~$|\Gamma\e{sec}|$, mais d'un taux $|\Gamma\e{saturé}|$ plus faible, puisque la parcelle est désormais saturée (son humidité vaut~$1$ et son rapport de mélange~$r$ vaut~$r\e{sat}$). Sur l'émagramme, la parcelle suit une \voc{courbe adiabatique saturée}. Au cours de l'ascension, la parcelle reste saturée mais, puisque sa température continue de diminuer, $r\e{sat}$ diminue de concours, ce qui induit une diminution du rapport de mélange en vapeur d'eau~$r$ et une augmentation du rapport de mélange en eau liquide (qui prend la forme de gouttelettes nuageuses ou, si les conditions de croissance sont réunies, de précipitations pluvieuses).

\sk
En quatrième lieu, la forme du profil de température d'environnement détermine si, une fois le niveau de condensation atteint, le nuage va suivre ou non un développement vertical marqué. On rappelle que le profil d'environnement n'est pas celui suivi par la parcelle considérée, mais représente l'état atmosphérique tel qu'il peut être mesuré par un ballon-sonde météorologique par exemple.

\begin{finger}

\item Si les soulèvements initiaux de la parcelle ne l'amènent que dans des niveaux atmosphériques où sa température reste plus faible que celle de l'environnement, alors il n'y a pas de mouvements verticaux spontanés au sein du nuage. Le nuage formé est plutôt de type stratiforme (ou faiblement cumuliforme).

\item Si les soulèvements initiaux de la parcelle parviennent à la hisser à des niveaux atmosphériques où sa température devient plus élevée que celle de l'environnement, alors des mouvements verticaux spontanés apparaissent au sein du nuage. On parle de convection humide (ou convection profonde). Le nuage ainsi formé est cumuliforme. Le niveau atmosphérique à partir duquel la température de la parcelle en ascension adiabatique devient plus élevée que la température de l'environnement s'appelle le \voc{niveau de convection libre}. Le niveau atmosphérique à partir duquel la température de la parcelle redevient plus faible que l'environnement s'appelle le \voc{sommet théorique du nuage}. Si le sommet théorique du nuage est très élevé, la formation de cumulonimbus, donc d'orage, est très probable. 

\end{finger}





%%%% CHAPITRE 8
%%\def\thechapter{8} \title{Sciences de l'atmosphère\\ Chapitre \thechapter} \author{Aymeric SPIGA\\ ~\\ \small Modifié et augmenté à partir des notes de Francis CODRON.} \maketitle %% commenter la ligne en mode livre complet
\chapter{Dynamique et circulation générale}

\dictum[Livre d'Osée, 8e siècle avant JC]{Qui sème le vent récolte la tempête.}

\bk
Dans ce chapitre, on aborde de manière plus approfondie la circulation de l'atmosphère, c'est-à-dire les vents. On s'intéresse notamment à l'origine des mouvements horizontaux, en étudiant les diverses forces en présence, avant de donner quelques généralités sur les vents à grande échelle sur Terre (la \voc{circulation générale}). Dans tout le chapitre, les vecteurs sont notés \textbf{en gras}.

\mk 
\section{Equations de la dynamique et interprétation}

\sk
\subsection{Système de coordonnées et référentiel}
	\sk
La position d'un point $M$ de l'atmosphère sera représentée dans un systèmes de coordonnées sphériques (figure~\ref{fig:repere}) par sa latitude $\varphi$, sa longitude $\lambda$, et son altitude~$z$ par rapport au niveau de la mer. Pour les déplacements horizontaux, on utilise le repère direct
$\left(M,\mathbf{i},\mathbf{j},\mathbf{k}\right)$ où $\mathbf{i}$ et $\mathbf{j}$ sont les vecteurs unitaires vers l'est et le nord, et $\mathbf{k}$ est dirigé suivant la verticale vers le haut. La direction définie par~$\mathbf{i}$ est souvent qualifiée de \voc{zonale}, celle définie par~$\mathbf{j}$ de \voc{méridienne}. 
%Pour des déplacements qui ne sont pas d'échelle planétaire, on utilisera également des distances horizontales vers l'est et le nord~$dx=a\, d\lambda\, \cos \varphi$ et~$dy=a\,d\varphi$ où~$a$ est le rayon de la Terre.
%
%\sk
On distingue deux référentiels pour l'étude des mouvements de l'air:
\begin{finger}
\item Un \voc{référentiel tournant} lié à la Terre, en rotation autour de l'axe des pôles avec la vitesse angulaire $\Omega$. La \voc{vitesse relative} est mesurée dans le référentiel tournant, par rapport à la surface de la Terre et a pour composantes~$u,v,w$ suivant \v i,\v j,\v k. Il s'agit de ce que l'on appelle communément le \voc{vent} avec le point de vue d'humain attaché à la surface de la Terre, c'est-à-dire au référentiel tournant. La composante horizontale du vecteur vitesse relative est donc~$\mathbf{V} = u \, \mathbf{i} + v \, \mathbf{j}$ et la composante verticale~$w \, \mathbf{k}$.
\item Un \voc{référentiel fixe} orienté suivant les directions de trois étoiles. La \voc{vitesse absolue} d'un point M est considérée dans le référentiel fixe et inclut donc le mouvement circulaire autour de l'axe des pôles. Ce référentiel peut être considéré comme galiléen. Il correspond à ce qu'on observerait depuis l'espace, lorsqu'on voit la Terre tourner au lieu d'être \ofg{attaché} à sa rotation.
\end{finger}

%\figun{0.4}{0.25}{\figfrancis/repere}{Schéma du système de coordonnées et du repère utilisés.}{fig:repere}
\figside{0.45}{0.22}{\figfrancis/repere}{Système de coordonnées et repère utilisés.}{fig:repere}




\sk
\subsection{Equations du mouvement horizontal}
	\sk
L'équation de base pour le mouvement de masses d'air est la relation fondamentale de la dynamique $\Sigma \v F=m \, \v a$ (seconde loi de Newton).  Cette relation est cependant valable dans un référentiel galiléen, tel le référentiel fixe. On s'intéresse plutôt au vent, c'est-à-dire que l'on souhaite considérer des mouvements atmosphériques par rapport à la surface de la Terre qui est en rotation autour de l'axe des pôles. On va donc dans un premier temps projeter l'accélération dans le référentiel tournant, puis étudier les principales forces horizontales. Autrement dit, on se donne pour objectif d'exprimer l'accélération dans le référentiel tournant, qu'on souhaite connaître, en fonction de l'accélération dans le référentiel fixe, qui est égale à la somme des forces.

\sk
La relation entre vitesse absolue~$\v V_a$ dans le référentiel fixe et vitesse relative~$\v V_r$ dans le référentiel tournant s'écrit, avec le vecteur de rotation~$\v \Omega$ de module~$\Omega$ dirigé selon l'axe des pôles~:
\[\v V_a = \v V_r + \vl{\Omega}\wedge\vl{CM}\]
Il s'agit de la relation de composition des vitesses pour un référentiel tournant. Le terme $\vl{\Omega}\wedge\vl{CM}$ est la vitesse d'un point fixe par rapport au sol ($\v V_r=0$), il est appelé \voc{vitesse d'entrainement}.
%La relation entre la dérivée temporelle d'un vecteur \v X dans le référentiel fixe (\emph{absolue}, $a$) et celle dans le référentiel tournant (\emph{relative}, $r$) s'écrit \[\frac{d\v X}{dt}_{|a}=\frac{d\v X}{dt}_{|r}+ \vl \Omega\wedge \v X\] En applicant au vecteur \vl{CM}, avec $\frac{d\vl{CM}}{dt}=\v V$, on a: \[\v V_a=\v V_r+\vl{\Omega}\wedge\vl{CM}\]
La relation entre accélération absolue~$\v a_a$, égale à la somme des forces, et accélération relative~$\v V_r$ dans le référentiel tournant s'écrit
\[ \v a_a=\Sigma\v F=\v a_r+2\vl{\Omega}\wedge\v V_r-\Omega^2\,\vl{HM} \]
Le premier terme est l'accélération relative~$\v a_r$, le deuxième l'\voc{accélération de Coriolis}~$\v a_c$, le troisième est l'\voc{accélération d'entrainement}~$\v a_e$. Les termes de Coriolis et d'entraînement induisent des \voc{forces apparentes}~$\v F_c = -m \, \v a_c$ et~$\v F_e = -m \, \v a_e$ dans le référentiel tournant. On parle de forces apparentes car du point de vue du référentiel fixe, ces termes n'apparaissent pas comme des forces~: ils ne sont que des termes d'accélération causés par le caractère non galiléen du référentiel tournant.
%En dérivant à nouveau $\v V_a$, on obtient: \[\v a_a=\left(\frac{d \v V_r}{dt}_{|r}+\vl{\Omega}\wedge\v V_r\right)+\vl{\Omega}\wedge \left(\v V_r+\vl{\Omega}\wedge\vl{CM}\right)\] soit en regroupant et avec $\vl{\Omega}\wedge(\vl{\Omega}\wedge\vl{CM})=-\Omega^2\cdot\vl{HM}$:




\sk
\subsection{Action des forces apparentes}

\sk
On peut rapidement interpréter les deux termes liés aux forces apparentes dans le cadre du mouvement d'un point à la surface de la Terre.

\sk
\subsubsection{Accélération d'entraînement et pesanteur}
	\sk
On considère un point M immobile par rapport à la surface de la Terre. Les forces (massiques) subies par M sont la force de gravitation \v G, dirigée vers le centre de la Terre, et \v R la réaction du sol dirigée perpendiculairement à la surface (figure \ref{fig:centrif}). Dans le référentiel fixe, l'accélération de M est celle du mouvement circulaire uniforme: $\v a_e = - \Omega^2 \, \vl{HM}$ (accélération d'entrainement). On doit donc avoir \[\v a_e=\v G+\v R\] 
C'est impossible si la Terre est sphérique (sauf au pôle et à l'équateur): on aurait alors \v R et \v G colinéaires mais pas dans la direction de $\v a_e$. La Terre a en fait pris une forme aplatie, où la surface n'est pas perpendiculaire à~$\v G$. En posant $\v g=\v G-\v a_e$, l'équilibre devient: \[\v g+\v R=\v 0\] On a donc une gravité apparente \v g dirigée localement vers le bas (perpendiculairement à la surface) mais pas exactement vers le centre de la Terre. La gravité réelle \v G a elle une faible composante horizontale. Dans ce qui suit, on considère que l'accélération d'entraînement est inclus dans le terme~$\v g$.

%\figside{0.55}{0.25}{\figfrancis/centrif}{Equilibre d'un point posé au sol. La forme réelle de la Terre est en trait continu, la sphère en pointillés.}{fig:centrif}
\figside{0.45}{0.2}{\figfrancis/centrif}{Equilibre d'un point posé au sol. La forme réelle de la Terre est en trait continu, la sphère en pointillés.}{fig:centrif}




\sk
\subsubsection{Accélération de Coriolis et déviation du mouvement}
	\sk
L'accélération de Coriolis peut être interprétée comme une force apparente massique $\v F_C = - 2 \, \v \Omega\wedge\v V_r$. Cette force apparente étant orthogonale à la vitesse à cause de la présence du produit vectoriel, sa puissance est nulle~: la \voc{force de Coriolis} va dévier le mouvement relatif mais ne peut pas modifier la vitesse du vent ou de courants. Pour des mouvements relatifs horizontaux à la vitesse \v V, le module de la force apparente de Coriolis est~$2 \, \Omega \, \sin \phi \, V$ qui change de signe lorsqu'on change d'hémisphère en fonction de~$\sin \phi$. Dans l'hémisphère nord, où $\sin \phi>0$, la force de Coriolis est dirigée à $90^{\circ}$ à droite du vent. 

\sk
Afin de bien comprendre l'effet de la force de Coriolis, il est profitable sur une planète comme la Terre d'utiliser la conservation du moment cinétique\footnote{
Puisque le moment cinétique~$\sigma$ se conserve on a \[ \ddf{\sigma}{t} = 0 = \ddf{r}{t} \, (\Omega \, r + u) + r \, \left( \Omega \ddf{r}{t}+\ddf{u}{t} \right) \qquad \Rightarrow \qquad \ddf{u}{t} = - \ddf{r}{t}  \, \left( 2\,\Omega + \frac{u}{r} \right) \] 
Le terme en $u/r$ est dû à la courbure de la surface, mais seule la vitesse relative intervient, pas la rotation de la Terre. En pratique, ce terme est négligeable sur Terre devant~$2 \, \Omega$. L'équation ci-dessus montre donc que raisonner avec la conservation du moment cinétique permet de comprendre l'effet sur les vents de la force de Coriolis.
}
(l'équivalent pour les systèmes en rotation de la quantité de mouvement pour les systèmes en translation). En effet, la somme des forces étant dirigée vers H, M conserve son \voc{moment cinétique}~$\sigma$ par rapport à l'axe des pôles, qui s'exprime
\[ \boxed{ \sigma = u_a \, r = (\Omega \, r + u) \, r } \]
où~$r$ est la distance entre le point considéré et l'axe de rotation qui passe par les deux pôles.
%\footnote{La conservation de $\sigma$ implique des variations de l'énergie cinétique $(\Omega r+u)^2$. C'est le travail de \v G (pour un mouvement sud-nord) qui en est l'origine.}. 

\sk
Pour illustrer les effets de cette force apparente de Coriolis, on considère une parcelle initialement au repos dans le référentiel tournant (c'est à dire~$u=0$ et~$v=0$ à~$t=0$) qui se déplacerait vers le Nord suivant l'axe~$\v j$. Elle se rapproche donc de l'axe des pôles et va voir sa vitesse absolue augmenter par conservation du moment cinétique: $\sigma$ est constant et~$r$ diminue, donc $u_a$ augmente. Dans le même temps, la vitesse d'entrainement locale~$u_e=\Omega \, r$ diminue sous l'effet de la diminution de la distance~$r$ à l'axe des pôles. La parcelle va donc acquérir une vitesse relative $u>0$ vers l'est\footnote{
En fait, l'expression ci-dessus permet même de calculer la variation de vitesse associée. Pour un mouvement sud-nord, la vitesse est $v=a \, \ddf{\phi}{t}$. D'autre part $r=a \, \cos \phi$ donc~$\ddf{r}{t}=-a \, \ddf{\phi}{t} \, \sin \phi = - v \, \sin \phi$. L'équation de conservation du moment cinétique devient 
\[ \ddf{u}{t} = v \, \sin \phi \, \left( 2 \, \Omega + \frac{u}{r} \right) \simeq 2 \, \Omega \, v \, \sin \phi \] 
La parcelle est bien déviée vers l'est pour un déplacement vers le nord tel que~$v>0$.
}
comme indiqué sur le schéma \ref{fig:coriolisns}. 

\figside{0.3}{0.2}{\figfrancis/coriolis_ns}{Déviation d'une parcelle se déplaçant vers le nord. Instant initial: vitesses d'entrainement $u_e$ et absolue $u_a$ égales. Instant final: vitesse d'entrainement $u_e'$ et absolue $u_a'$ augmentée par conservation du moment cinétique $\sigma$.}{fig:coriolisns}

%\subsubsection{Force de Coriolis: mouvement vers l'est} On considère un point M en mouvement par rapport à la surface de la Terre. On rappelle que pour un mouvement circulaire, on doit avoir une accélération normale égale à $V^2/R$ dirigée vers le centre du cercle. On suppose que les forces réelles s'exerçant sur M sont les mêmes que pour un point fixe: $\Sigma \vec F=\v a_e$. La composante de la vitesse relative vers l'est (suivant \v i) est $u$, et $\dot{r}$ dans la direction \vl{HM}. La vitesse absolue de M vers l'est est $u_a=\Omega r+u$. La relation $\v a=\Sigma\v F$ s'écrit dans la direction $\v e_r$: \[-\frac{(\Omega r+u)^2}{r}+\ddot{r}=a_e=-\Omega^2r\] soit en développant: \[\ddot{r}=u\cdot(2\Omega+\frac{u}{r})\] Pour un mouvement relatif vers l'est ($u>0$), la vitesse absolue est supérieure à la vitesse d'entrainement, et la somme des forces est insuffisante pour compenser $V_a^2/r$. La parcelle va donc s'éloigner de l'axe de rotation (figure \ref{fig:coriolisew}). Elle va au contraire se rapprocher pour $u<0$ (mouvement vers l'ouest). Pour trouver l'accélération relative dans la direction sud-nord, on projette $\v e_r$ sur \v j: $\dot{v}=-\ddot{r}\sin \phi$. \[\dot{v}=-u\sin \phi\cdot(2\Omega+\frac{u}{r})\] M est donc dévié vers le sud pour un déplacement relatif vers l'est.
%\begin{figure}[tbp] \begin{center} \includegraphics[width=12cm]{\figfrancis/coriolis_ew} \end{center} \caption{Déviation d'une parcelle ayant une vitesse relative initiale non nulle vers l'est (gauche) et l'ouest (droite). Un plan parallèle à l'équateur est représenté, vu depuis le pôle nord, l'axe de rotation est au centre. Les vitesse et accélération d'entrainement (égale à la somme des forces) sont en noir, la vitesse absolue en rouge. La trajectoire future de la parcelle est en pointillés.} \label{fig:coriolisew} \end{figure}


\sk
\subsection{Forces de pression}
	\sk
Les forces de pression horizontales se calculent comme la force de pression verticale dans la démonstration de l'équilibre hydrostatique. La force de pression s'exerçant sur une surface $S$ est normale à cette surface et vaut $P \, S$. Pour une parcelle d'air de volume $\delta x \, \delta y \, \delta z$ (figure \ref{fig:pres}), la force de pression totale dans la direction ($Ox$) vaut
\[ F_P^* = P(x) \, \delta y \, \delta z - P(x+\delta x) \, \delta y \, \delta z = - \frac{\partial P}{\partial x} \, \delta x \, \delta y \, \delta z \]
La force de pression {\em massique} est donc
\[F_P = \frac{F_P^*}{\rho \delta x \delta y \delta z}=-\frac{1}{\rho}\frac{\partial P}{\partial x}\]
On peut faire le même calcul sur ($Oy$). Finalement les deux composantes horizontales de la force de pression s'écrivent
\[\v F_P^H = -\frac{1}{\rho} \, \binom{\frac{\partial P}{\partial x}}{\frac{\partial P}{\partial y}}\] %  =-\frac{1}{\rho}\vl{grad}P\]

\sk
La force de pression est donc opposée aux variations horizontales de pression données par les dérivées partielles, ce qui lui confère des propriétés importantes.
\begin{citemize}
\item La force de pression est dirigée des hautes vers les basses pressions, perpendiculairement aux isobares.
\item La force de pression est inversement proportionelle à l'écartement des isobares.
\end{citemize}
Une région où la pression est particulièrement basse est appelée \voc{dépression}. Une région où la pression est particulièrement élevée est appelée \voc{anticyclone}.

\figside{0.6}{0.2}{\figfrancis/pressure}{Forces de pression (suivant ($Ox$)) s'exerçant sur une parcelle.}{fig:pres}

%\subsubsection{Équivalence avec le géopotentiel}
%L'équilibre hydrostatique fait que la pression décroit toujours avec
%l'altitude. Une pression localement élevée doit donc correspondre à une
%altitude élevée des surfaces isobares.
%\begin{figure}[htp]
%  \begin{center}
%    \includegraphics[width=\figwn]{\figfrancis/pres_geop}
%  \end{center}
%  \caption{Équivalence entre écarts de pression et d'altitude: les points A et
%  B sont à la même altitude, A et C à la même pression. La pression en B est
%  donc supérieure à celle en B.}
%  \label{fig:pres_geop}
%\end{figure}
%Sur la figure \ref{fig:pres_geop}, la force de pression horizonale dans la
%direction ($Ox$) est
%$F_P=-\frac{1}{\rho}\frac{P_B-P_A}{\delta x}$. Or $A$ et $C$ sont à la même
%pression, on a donc
%\[F_P=-\frac{1}{\rho}\frac{P_B-P_C}{\delta x}=-\frac{1}{\rho}\frac{P_B-P_C}{\delta z}\cdot\frac{\delta z}{\delta x}\]
%En utilisant
%\[\frac{P_B-P_C}{\delta z}=-\frac{\partial P}{\partial z}=\rho g\]
%on trouve 
%\[F_P=-g\left(\frac{\delta z}{\delta x}\right)_P\]
%On aurait une relation équivalente pour la direction ($Oy$), la
%force de pression horizontale vaut donc finalement
%\[\v F_P=-\frac{1}{\rho}\vl{grad}_Z(P)=-g\cdot\vl{grad}_P(Z)\]
%On utilise plutôt le gradient de pression horizontal avec la pression au
%niveau de la mer, et le gradient isobare de l'altitude $Z$ ou du
%{\em géopotentiel} $gZ$ dans l'atmosphère libre.
%Sur une carte d'une surface isobare, les lignes à $Z$ constant sont des
%{\em isohypses}. La force de pression est donc dirigée des hautes vers les
%basses valeurs de $Z$, perpendiculairement aux isohypses.

\sk
Les variations verticales de la pression sont données par l'équilibre hydrostatique comme indiqué dans les chapitres précédents. Cette propriété a deux conséquences importantes pour les variations de pression horizontales donc la force de pression horizontale. 
\begin{finger}
\item Une conséquence de cet équilibre est que la pression à une altitude $z$ est proportionelle à la masse de la colonne d'air située au dessus de $z$. Une diminution ou augmentation de cette masse dûe aux mouvements d'air horizontaux change donc la pression en dessous, en particulier à la surface.
\item D'autre part, même pour une masse d'air totale de la colonne constante, des écarts de température horizontaux peuvent créer des gradients de pression en changeant la répartition verticale de cette masse. L'équation hypsométrique donne l'épaisseur d'une colonne d'air de masse constante entre deux niveaux de pression donnés (voir chapitres précédents)~: la pression décroît plus vite dans une couche d'air froid que dans une couche d'air chaud. Une variation horizontale de température induit donc une force de pression horizontale selon ce principe.
\end{finger}

%\begin{equation}
%  g\cdot(Z_2-Z_1)=R<T>\ln{\frac{P_1}{P_2}}
%  \label{eq:hypso}
%\end{equation}
%La différence entre les forces de pressions aux niveaux 1 et 2 sera donc: \[\v F_{P_2}-\v F_{P_1}=-R\cdot\vl{grad}<T>\cdot\ln{\frac{P_1}{P_2}}\]




\sk
\subsection{Équilibres dynamiques}
	\sk
L'équation complète de la quantité de mouvement pour les mouvements atmosphériques, qui résulte de l'application de la seconde loi de Newton, s'écrit~:
%\begin{equation}
%\ddf{\v V_r}{t} + 2 \, \v \Omega\wedge\v V_r = \v g + \v F_P + \vl{Fr} 
\[   
\ddf{\v V_r}{t} = \v g + \v F_P + \v F_C + \vl{Fr}
\] %\frac{1}{\rho}\vl{grad}P  \]
%  \label{eq:qtemvt}
%\end{equation}
Le terme~$\vl{Fr}$ représente les forces de friction qui sont négligées sauf lorsqu'on se trouve à proximité de la surface.

	\sk
Tous les termes de l'équation du mouvement n'ont pas la même importance lorsqu'on considère des mouvements atmosphériques de grande échelle. On définit donc des échelles caractéristiques du mouvement étudié. Pour simplifier, on choisit des échelles qui sont des puissances de 10.
\begin{description}
\item[longueur] Les échelles de longueur sont $L$ sur l'horizontale, et $H$ sur la verticale. Pour des mouvements qui s'étendent sur la hauteur de la troposphère, $H\sim 10$~km. $L$ peut varier beaucoup, mais l'échelle dite synoptique $L=$1000~km, qui est celle des perturbations des latitudes moyennes, est d'un intérêt particulier. La dernière échelle de longueur est celle du rayon de la Terre~$a$, qui est de l'ordre de 10000~km. 
\item[vitesse] Les échelles de vitesse horizontale et verticale sont notées $U$ et $W$. On a typiquement $U$=10~m~s$^{-1}$ dans l'atmosphère. Le rapport d'aspect du mouvement impose d'autre part que $W\le UH/L$.
\item[temps] L'échelle de durée du mouvement est construite à partir de celles de vitesse et de longueur: $T=L/U$. L'autre échelle de temps est celle liée à la rotation de la Terre, qui apparait dans le terme de Coriolis.
\item[variables thermodynamiques] Les variations des variables thermodynamiques $P,T,\rho$ sur la verticale sont celles des profils moyens donnés en introduction. En un point donné, les variations à l'échelle synoptique $\delta P,\delta T,\delta\rho$ sont de l'ordre de 1\% de la valeur moyenne.
\end{description}




\sk
\subsubsection{Mouvement vertical}
	\sk
L'ordre de grandeur des termes de l'équation du mouvement 
%\ref{eq:qtemvt} 
projetée sur la verticale (dirigée suivant \v k) est indiqué dans la table \ref{tab:vqmouv}. On voit que l'équilibre hydrostatique est vérifié avec une très bonne approximation\footnote{On peut noter qu'on vérifie également l'équilibre hydrostatique entre des anomalies de densité et des anomalies de variations de pression sur la verticale. Les termes $\rho g$ et $\partial P/\partial z$ sont alors cent fois plus faibles que pour l'état moyen, mais toujours supérieurs aux autres termes de l'équation.}. Notamment la composante verticale de la force de Coriolis~$\v F_C$ est négligeable devant~\v g et les forces de pression. Le seul autre terme qui peut devenir important est l'accélération relative~$dw/dt$, lors de mouvements verticaux intenses à petite échelle, comme dans un nuage d'orage ou près de topographie raide.  
%\begin{equation}
%  \frac{\partial P}{\partial z}=-\rho  g
%  \label{eq:hydro}
%\end{equation}

\begin{table}
  \centering
  \begin{tabular}{ccccccc}
    \hline
    Équation & $dw/dt$ & $-2\Omega u\cos\phi$ & $-\left(u^2+v^2\right)/a$ & = &
    $-\rho^{-1}\partial P/\partial z$ & $-g$ \\
    Échelle & $UW/L$ & $fU$ & $U^2/a$ && $P_0/(\rho_0H)$ & $g$ \\
    m.s\md & 10$^{-7}$ & 10$^{-3}$ & 10$^{-5}$  && 10 & 10 \\ 
    \hline
  \end{tabular}
  \caption{\emph{Analyse d'échelle de l'équation du mouvement vertical (avec
  $L$=1000~km et $W$=1~cm.s\mo).}}
  \label{tab:vqmouv}
\end{table}




\sk
\subsubsection{Mouvement horizontal}
	\sk
Le détail de l'équation horizontale projetée en coordonnées sphériques est donné dans la table \ref{tab:hqmouv} pour $L$=1000~km. Sur les composantes horizontales (\v i, \v j), l'expression de la force de Coriolis se réduit aux contributions des mouvements horizontaux dans la mesure où~$W<<U$ pour des mouvements d'échelle supérieure à 10~km. 
\[\v F_C = \binom{f \, v}{-f \, u} \qquad \textrm{ou} \qquad \v F_C = -f \, \v k \wedge \v V_h \]
où $\v V_h = u \v i + v \v j$ est la vitesse horizontale et 
\[ \boxed{ f = 2 \, \Omega \, \sin \phi } \]
est appelé \voc{facteur de Coriolis}. Aux moyennes latitudes ($\phi=45$\deg), la valeur de~$f$ est environ~$10^{-4}$~s$^{-1}$. 
%Les composantes de la force de Coriolis sont \[\v F_C=-2\Omega\left(\begin{array}{c}0\\\cos\phi\\\sin\phi\end{array}\right) \wedge\left(\begin{array}{c}u\\v\\w\end{array}\right) =-2\Omega\left(\begin{array}{c}w\cos\phi-v\sin \phi\\u\sin \phi\\-u\cos \phi\end{array}\right)\]
%\footnote{Pour des mouvements de
%type ``chute libre'', la vitesse verticale $w$ domine. On peut alors mettre en
%évidence une déviation vers l'est, mais qui reste très faible (de l'ordre de
%1cm pour 80m de chute).} 

\begin{table}
  \centering
  \begin{tabular}{cccccccc}
    \hline
    Équation-$x$ & $\frac{du}{dt}$ & $-2\Omega v\sin\phi$ & $+2\Omega
    w\cos\phi$ & $+\frac{uw}{a}$ & $-\frac{uv\tan\phi}{a}$ &=&
    $-\frac{1}{\rho}\frac{\partial P}{\partial x}$ \\
    Équation-$y$ & $\frac{dv}{dt}$ & $+2\Omega u\sin\phi$ &&         
                $+\frac{vw}{a}$ & $+\frac{u^2\tan\phi}{a}$ &=&
    $-\frac{1}{\rho}\frac{\partial P}{\partial y}$ \\
    Échelles & $U^2/L$ & $fU$ & $fW$ & $UW/a$ & $U^2/a$ && $\delta P/(\rho L)$
    \\
    m.s\md & 10$^{-4}$ & 10$^{-3}$ & 10$^{-6}$ & 10$^{-8}$ & 10$^{-5}$ &&
    10$^{-3}$ \\
    \hline
  \end{tabular}
  \caption{\emph{Analyse en ordre de grandeur de l'équation du mouvement
  horizontale.}}
  \label{tab:hqmouv}
\end{table}

\sk
Sur un plan horizontal, les termes restants de l'équation du mouvement sont ainsi:
%\begin{equation}
\[  \frac{d\v V_h}{dt}+f\v k\wedge\v V_h=\v F_P  \]
%  \label{eq:hqmouv}
%\end{equation}
avec $\v V_h$ la vitesse horizontale, et $\v F_P$ les forces de pression horizontales massiques. Pour évaluer lequel des deux termes à gauche domine, on définit le \voc{nombre de Rossby} $\mathcal{R}$, rapport entre accélération relative et de Coriolis
\[ \mathcal{R} = \frac{U^2/L}{f\,U} = \frac{U}{f\,L} \]
Avec $f$=10$^{-4}$~s$^{-1}$ aux moyennes latitudes et $U$=10~m~s$^{-1}$, on a $\mathcal{R}=0.1$ aux grandes échelles de la circulation terrestre ($L$=1000~km), donc Coriolis domine. Au contraire, à une échelle plus petite de $L$=10~km, $\mathcal{R}=10$ et Coriolis devient négligeable.




\sk
\subsubsection{Equilibre géostrophique}
	\sk
Dans le cas d'un nombre de Rossby petit (donc $L$>1000~km aux moyennes latitudes), on est proche d'un équilibre appelé \voc{équilibre géostrophique} entre les forces de Coriolis et de pression
\[ \boxed{ \v F_C+\v F_P=\v 0 } \]
qui s'écrit selon les deux composantes horizontales
\[ \boxed{ \binom{f \, v}{-f \, u} = \binom{\frac{1}{\rho} \,\frac{\partial P}{\partial x}}{\frac{1}{\rho} \,\frac{\partial P}{\partial y}} } \]
Le vent qui vérifie exactement cet équilibre est appelé \voc{vent géostrophique}~$\v V_g$. Sous forme vectorielle on a $f\v k\wedge\v V_g=\v F_P$ et sous forme projetée
\[ \v V_g = \binom{u}{v} = \binom{- \frac{1}{\rho \, f} \, \frac{\partial P}{\partial y}}{\frac{1}{\rho \, f} \, \frac{\partial P}{\partial x}} \]
%\begin{equation}
%  \v V_g=\frac{1}{\rho f}\v k\wedge\vl{grad}_z(P)=\frac{g}{f}\v k\wedge\vl{grad}_P(Z)
%  \label{eq:geost}
%\end{equation}

\figun{1.1}{0.3}{\figfrancis/geost}{Forces et vent dans l'équilibre géostrophique (hémisphère nord).}{fig:geost}

\sk
L'équilibre géostrophique peut s'illustrer graphiquement (voir figure~\ref{fig:geost}), formant ce que l'on appelle la loi de Buys-Ballot\footnote{Comme l'indique Buys-Ballot dans son article de 1857~: \emph{Note sur le rapport de l'intensité et de la direction du vent avec les écarts simultanés du baromètre ; [...] Ce n'est pas la girouette, mais c'est le baromètre d'après lequel on doit juger le vent [...] La grande force du vent est annoncée par une grande différence des écarts simultanés du baromètre dans les Pays-Bas [...] Pour un autre pays, on devra étudier les modifications.}}.
\begin{description}
\item[Direction] Comme la force de Coriolis est orthogonale au vecteur vitesse, et opposée à la force de pression, le vent géostrophique est lui-même orthogonal aux variations horizontales de pression donc parallèle aux isobares.
\item[Sens] Dans l'hémisphère nord, les basses pressions sont à gauche du vent, à droite dans l'hémisphère sud.
\item[Module] La vitesse du vent géostrophique est proportionnelle aux variations horizontales de pression~; autrement dit, plus les isobares sont resserrées, plus le vent est fort.
\end{description}
La carte de la pression et du vent en surface (figure \ref{fig:meteofrance}, voir aussi figure \ref{fig:SLPwind}) montre clairement que l'orientation et le module du vent sont dictés par l'équilibre géostrophique. Lorsque la friction est élevée proche de la surface, l'équilibre géostrophique est perturbé par la présence de la force de friction, ce qui a pour conséquence de donner un vent légèrement dévié vers l'intérieur des dépressions et vers l'extérieur des anticyclones. Quand le nombre de Rossby est grand (donc à petite échelle), l'équilibre géostrophique ne s'applique plus et le vent est accéléré des hautes vers les basses pressions. 
%% PARLER DU LAVABO

\figside
%{0.85}{0.6}
{0.65}{0.4}
{decouverte/cours_dyn/carte_france.jpg}{Carte météorologique Météo-France construite à partir des données relevées dans les stations météorologiques indiquées par des points. Les lignes isobares montrent qu'une forte dépression se situe au nord du Royaumu-Uni. Les \ofg{drapeaux} accolés aux points d'observations représentent les vents mesurés à la surface (le nombre de barres indique la force du vent). La direction du vent part du drapeau vers le point considéré. On remarque que le vent est approximativement parallèle aux isobares et tourne dans le sens inverse des aiguilles d'une montre autour de la dépression. Ce comportement est typique de celui déduit pour l'hémisphère Nord par l'équilibre géostrophique entre forces de pression et force de Coriolis (voir figure~\ref{fig:geost}). Le vent est légèrement rentrant vers l'intérieur de la dépression, sous l'influence de la force de friction qui vient d'ajouter aux deux forces précitées.}{fig:meteofrance}




\mk
\section{Circulation atmosphérique~: généralités}

\sk
\subsection{Structure en latitude}

\sk
Au premier ordre, les caractéristiques de l'atmosphère dépendent essentiellement de la latitude. On a en particulier un contraste entre les régions tropicales, comprises entre 30\deg sud et nord, et les latitudes moyennes (autour de 45\deg) et hautes latitudes (régions polaires). Ces variations apparaissent clairement en observant des moyennes sur toutes les longitudes, ou moyennes zonales (figure \ref{fig:UTlatP}). Ce contraste résulte à la fois de l'influence de la rotation de la planète et du chauffage différentiel qui rend les régions tropicales excédentaires en énergie alors que les moyennes et hautes latitudes sont déficitaires.

\figsup{0.48}{0.25}{\figfrancis/T_latP_jan}{\figfrancis/U_latP_jan}{Coupes latitude-pression de la température (haut) et du vent zonal (bas), en moyenne climatique et zonale, pour le mois de janvier. L'utilisation de la pression comme coordonnée verticale permet de se focaliser sur la troposphère.}{fig:UTlatP}

\sk
La température décroit partout sur la verticale jusqu'à un minimum à la tropopause, située entre 100~hPa dans les tropiques (où la température minimale est atteinte) et 300~hPa aux moyennes latitudes. Sur l'horizontale, la température est maximale et presque constante dans les tropiques, puis décroit très rapidement vers les pôles aux latitudes moyennes. La répartition de la vapeur d'eau (fig \ref{fig:humspec}) est très liée à la température: on observe un maximum dans les zones chaudes tropicales près de la surface, et peu d'eau en altitude ou aux latitudes polaires. La vapeur d'eau est également absente dans la stratosphère malgré la température élevée, à cause de l'absence de sources locales: la vapeur d'eau provient de l'évaporation en surface et ne peut franchir le piège froid à la tropopause.

\figside{0.5}{0.15}{\figfrancis/hum_spec}{Moyenne (zonale et temporelle) du rapport de mélange massique de vapeur d'eau (en g/kg d'air).}{fig:humspec}

\figun{0.98}{0.3}{\figfrancis/MMC}{Circulation moyenne (zonale et temporelle) dans le plan méridien. La circulation (schématisée par les flèches) est parallèle aux isolignes de la fonction de courant, et le flux de masse (débit) entre deux isolignes est constant. En vert, valeurs positives d'environ $1 \times 10^{11}$~kg~s$^{-1}$, qui correspondent à une rotation horaire. En marron, valeurs négatives d'environ $-1 \times 10^{11}$~kg~s$^{-1}$, qui correspondent à une rotation anti-horaire.}{fig:MMC} %%{\figfrancis/MMC_legende}

\sk
La structure du vent zonal est dominée aux moyennes latitudes par la présence de deux \voc{jets}, c'est-à-dire de puissants courants atmosphériques, dits \voc{jets d'ouest} car ils soufflent de l'ouest vers l'est. Leur vitesse augmente sur la verticale entre la surface et un maximum au niveau de la tropopause, autour de 50~m~s$^{-1}$. Ce comportement peut être justifié en combinant l'équilibre géostrophique à l'équilibre hydrostatique (équation du vent thermique). Dans les tropiques, les vents moyens sont d'est, surtout dominants dans la basse troposphère, mais restent néanmoins moins forts que les vents d'ouest dans les moyennes latitudes. On les appelle les \voc{alizés}.

\sk
La circulation dans le plan méridien (sud-nord et verticale) est caractérisée par une série de cellules fermées. Le chauffage différentiel explique qu'une différence de pression naisse entre les tropiques et les moyennes latitudes, car la pression diminue plus vite avec l'altitude dans les couches d'air froid des moyennes latitudes que dans les couches d'air chaud des tropiques. Ceci donne naissance en altitude à des vents de l'équateur vers les pôles. Ces vents induisent un flux de masse atmosphérique vers les moyennes latitudes, donc, d'après l'équivalence entre pression et masse, une augmentation de la pression de surface aux moyennes latitudes par rapport aux tropiques. Ceci donne naissance proche de la surface à des vents des pôles vers l'équateur. Par continuité, dans les tropiques, l'air s'élève proche de l'équateur (suivant la saison, du côté de l'hémisphère d'été) et redescend au niveau des subtropiques. Ce système est appelé \voc{cellules de Hadley}. On observe également dans les moyennes latitudes des cellules moins intenses, contrôlées par les instabilités dans l'atmosphère, appelées cellules de Ferrel. 
%%(figure \ref{fig:MMC})

\sk
La structure en latitude des vents %décrite par la figure~\ref{fig:UTlatP}, 
avec des vents d'ouest aux moyennes latitudes et d'est sous les tropiques, est très liée à la circulation de Hadley.
% décrite par la figure~\ref{fig:MMC}.
Sous l'action de la force de Coriolis, les mouvements vers les pôles sont déviés vers l'est et les mouvements vers l'équateur sont déviés vers l'ouest. Les jets d'ouest des moyennes latitudes proviennent ainsi de la déviation vers l'est de la circulation vers les pôles dans la branche supérieure de la cellule de Hadley. Les vents d'est (alizés) sous les tropiques proviennent quant à eux de la déviation vers l'ouest de la circulation vers l'équateur dans la branche inférieure de la cellule de Hadley. Les vents de grande échelle comportent donc une composante vers l'équateur et l'ouest sous les tropiques, alors qu'aux moyennes latitudes, ils comportent une composante vers les pôles et l'est [la composante vers l'est domine cependant]. Une exception à cette image est observée dans les régions de ``mousson'' (sous-continent Indien, et dans une moindre mesure Afrique de l'ouest et Amérique centrale) où la direction du vent s'inverse entre l'été (vers le continent) et l'hiver (vers l'océan).

\figside{0.6}{0.3}{\figfrancis/WH_circ_scheme}{Schéma de la circulation atmosphérique: zone de convergence et alizés dans les tropiques; gradient de pression tropiques (H) -pôle (L), vents d'ouest et ondes aux moyennes latitudes. La position des jets d'ouest et l'extension des cellules de Hadley sont représentées à droite. Figure adaptée de Wallace and Hobbs, Atmospheric Science, 2006.}{fig:circscheme}
 

\sk
\subsection{Structure en longitude}

\sk
Le champ de pression au niveau de la mer est relativement symétrique en longitude dans l'hémisphère sud, et varie peu suivant les saisons (figure \ref{fig:SLPwind}): on observe une ceinture de hautes pressions aux latitudes subtropicales (vers 30\deg), une pression un peu plus faible vers l'équateur, et une baisse rapide de la pression vers le pôle, avec un minimum autour de 60\deg. Dans l'hémisphère nord, des variations est-ouest liées aux contrastes continent-océan se rajoutent à cette structure en latitude. L'été, on observe des pressions relativement basses sur les continents chauds, et des hautes pressions sur les océans (anticyclones des açores dans l'Atlantique et d'Hawaï dans le Pacifique). L'hiver, ces anomalies s'inversent et on a des minimums de pression sur les océans (dépressions d'Islande et des Aléoutiennes) et des hautes pressions sur les continents froids (anticyclone de Sibérie).

\sk
On remarque que le vent a tendance à s'enrouler autour des extrema de pression isolés, dépressions et anticyclones. Il laisse les basses pressions à sa gauche dans l'hémisphère nord, et les hautes pressions à droite. Cette loi (de ``Buys-Ballot'') s'inverse dans l'hémisphère sud. On a vu à la section précédente qu'il s'agit d'une conséquence de l'équilibre géostrophique aux moyennes latitudes.

\figside{0.7}{0.35}{\figfrancis/WH_surfw_slp}{Vent et pression de surface observés. Le champ de pression est ramené au niveau de la mer afin de supprimer la composante permanent causée par les différences topographiques et d'obtenir une carte montrant uniquement les variations météorologiques de pression. Figure adaptée de Wallace and Hobbs, Atmospheric Science, 2006.}{fig:SLPwind}

\sk
La carte des précipitations moyennes (figure \ref{fig:seasprecip}) dans les tropiques une concentration dans une mince bande proche de l'équateur. Cette zone étroite correspond à la région de convergence des vents de surface, dirigés vers l'équateur sous l'effet des circulations type cellules de Hadley, d'où son nom de \voc{Zone de Convergence Intertropicale (ZCIT)}. On a également de fortes pluies un peu plus loin de l'équateur au cours des moussons d'été. Ces zones de pluies intenses sont aussi des régions d'ascendance à grande échelle. Au delà des subtropiques très sèches, dans lesquelles se trouvent la plupart des déserts de la planète, on retrouve d'autres régions de pluie sur les océans des latitudes moyennes. Ces pluies sont cette fois liées au passage des dépressions, et pas à une zone de convergence particulière.

\figun{0.65}{0.4}{\figfrancis/WH_precip_seas}{Précipitations moyennes saisonnières, en décembre (haut) et juillet (bas). Figure adaptée de Wallace and Hobbs, Atmospheric Science, 2006.}{fig:seasprecip}

\sk
\subsection{Circulations transitoires}

\sk
A la circulation moyenne décrite ci-dessus se superpose une circulation transitoire, qui varie d'un jour sur l'autre. La comparaison entre la vapeur d'eau instantanée et moyennée sur un mois (figure \ref{fig:wavevap}) montre la signature de cette circulation dans les basses couches de l'atmosphère: les variations horizontales de vapeur d'eau viennent du transport par la circulation. 

\figsup{0.7}{0.2}{\figfrancis/tcwv_month}{\figfrancis/tcwv_day}{Cartes de quantité de vapeur d'eau totale intégrée sur la verticale (kg~m$^{-2}$). Moyenne sur le mois de décembre 1999 (haut), et instantané au premier décembre 1999 (bas).}{fig:wavevap}

\sk
On reconnait dans la distribution instantanée les grandes régions sèches et humides des tropiques. A l'endroit de la transition vers les latitudes moyennes, on observe en revanche des filaments d'air qui s'enroulent, entrainés par une circulation tourbillonaire. Ces \voc{ondes baroclines} sont responsables des alternances fréquentes de temps sec et humide des régions tempérées. On observe environ 5 à 8 structures alternées sur un cercle complet de longitude, soit une longueur d'onde de quelques milliers de kilomètres (voir également la figure~\ref{fig:press}). Leur période est de quelques jours. Ces ondes sont également visibles dans la haute troposphère, toujours aux latitudes moyennes 
%(figure \ref{fig:wavepv}) 
autour de la position du jet. Elles ne pénètrent pas en revanche dans la stratosphère.

%\begin{figure}[tbp]
%  \begin{center}
%    \includegraphics[width=12cm]{\figfrancis/pv_month}
%    \\
%    \includegraphics[width=12cm]{\figfrancis/pv_day}
%  \end{center}
%  \caption{Cartes de la vorticité potentielle (un traceur dynamique conservé
%  au cours du mouvement) à 250~hPa: mois de décembre 1999 (haut) et premier
%  décembre (bas). Les valeurs absolues élevées aux pôles correspondent à de
%  l'air stratosphérique.}
%  \label{fig:wavepv}
%\end{figure}

\sk
\subsection{Résumé}

\sk
Les caractéristiques de la circulation atmosphérique sont donc très différentes dans deux zones qui couvrent chacune environ la moitié de la planète (figure \ref{fig:circscheme}):
\begin{description}
\item[Tropiques] Les tropiques sont marquées par des gradients horizontaux très faibles de température, mais des variations d'humidité marquées entre régions humides et sèches. Ces régions se déplacent à l'échelle saisonnière, mais restent stables à des périodes plus courtes (mais la précipitation dans les régions humides peut varier rapidement). La circulation est dominée par des cellules avec ascendance dans les zones de convergence, et subsidence au dessus des déserts. En surface, on a des vents d'est réguliers (alizés) qui convergent près de l'équateur.
\item[Moyennes latitudes] La région des latitudes moyennes est marquée au contraire par des gradients de température et de pression très forts. Les vents sont d'ouest en moyenne en surface et culminent avec un jet rapide au niveau de la tropopause. A cette circulation moyenne se rajoute une circulation horizontale intense de type ondulatoire à turbulente, qui, liée aux fortes variations horizontales, peut donner des variations très fortes et rapides de température ou d'humidité.
\end{description}
%%ces différences de comportement sont principalement dues à l'influence différente de la rotation de la Terre.



%\end{document}

%\subsection{Vent thermique}
%Le vent thermique décrit la variation verticale du vent géostrophique. Le nom
%vient du fait que les variations verticales de pression sont liées par
%l'équilibre hydrostatique à la température (équation \ref{eq:hypso}). Le
% résultat est plus simple à obtenir en utilisant la {\em pression} comme
%coordonnée verticale: en dérivant l'expression (\ref{eq:geost}) du vent
%géostrophique (version gradient du géopotentiel) par rapport à $P$, on
%obtient:
%\[\frac{\partial \v V_g}{\partial P}=\frac{g}{f}\v
%k\wedge\vl{grad}_P\left(\frac{\partial z}{\partial P}\right)\]
%On a utilisé la permutation des dérivations horizontale et verticale, et le
%fait que $g$ ne dépend pas de $P$. 
%Le second membre peut être transformé en utilisant l'équilibre hydrostatique
%(\ref{eq:hydro}):
%\[\frac{\partial z}{\partial P}=-\frac{1}{\rho g}=-\frac{RT}{gP}\]
%Le gradient horizontal étant pris à pression constante, seule $T$ varie. On
%obtient alors:
%\begin{equation}
%  \frac{\partial \v V_g}{\partial P}=-\frac{R}{fP}\v k\wedge\vl{grad}_P(T)
%  \label{eq:vth}
%\end{equation}
%Cette équation peut être intégrée (après avoir multiplié par $P$) entre deux
%niveaux de pression $P_1$ et $P_2$:
%\begin{equation}
%  \v V_{g2}-\v V_{g1}=\frac{R}{f}\ln{\frac{P_1}{P_2}}\v k\wedge\vl{grad}<T>
%  \label{eq:vthi}
%\end{equation}
%Où la température moyenne $<T>$ est définie comme pour l'équation
%hypsométrique (\ref{eq:hypso}). Cette forme intégrée peut être retrouvée
%directement à partir de (\ref{eq:hypso}) et (\ref{eq:geost}). La différence
%$\v V_{g2}-\v V_{g1}$ est appelée {\em vent thermique}, avec le niveau 2 situé
%à une altitude plus élevée que le niveau 1. Dans l'hémisphère nord, le vent
%thermique est dirigé parallèlement aux isothermes, avec les températures
%élevées à droite.
%
%La dérivée de $\v V_g$ par rapport à $z$ s'obtient en multipliant
%(\ref{eq:vth}) par $\partial P/\partial z=-\rho g$:
%\[\frac{\partial \v V_g}{\partial z}=\frac{g}{fT}\v k\wedge\vl{grad}_P(T)\]
%
%%\section{Mouvement inertiel}
%%\section{Rôle de la friction}
%%\section{Equilibre cyclostrophique}

	
%%%% CHAPITRE 9
%%\def\thechapter{9} \title{Sciences de l'atmosphère\\ Chapitre \thechapter} \author{Aymeric SPIGA\\ ~\\} \maketitle %% commenter la ligne en mode livre complet
\chapter{Eléments sur le changement climatique récent}

%\dictum[Charles Baudelaire, 1857]{\scriptsize Quand le ciel bas et lourd pèse comme un couvercle\\ Sur l'esprit gémissant en proie aux longs ennuis,\\ Et que de l'horizon embrassant tout le cercle\\ Il nous verse un jour noir plus triste que les nuits ;}
\dictum[President Lyndon B. Johnson, 1965]{This generation has altered the composition of the atmosphere on a global scale through [\ldots] a steady increase in carbon dioxide from the burning of fossil fuels.}

\bk
Ce chapitre se propose de donner quelques éléments factuels sur le changement climatique récent. 

\mk
\section{Le système climatique}

\sk
Il est utile de commencer par rappeler la distinction entre météorologie et climatologie. La \voc{météorologie} est la discipline ayant pour objet l'étude des phénomènes atmosphériques et de leurs variations à court terme. La \voc{climatologie} est la discipline ayant pour objet l'étude des conditions atmosphériques moyennes d'un pays, d'une région, d'une planète, et de leurs variations à moyen ou long terme. Une définition distincte de la climatologie est qu'elle vise à étudier le \voc{système climatique}, un système thermo-hydrodynamique non isolé dont les composantes sont les principales « enveloppes » externes de la Terre~: l'atmosphère (l’air, les nuages, les aérosols), l’hydrosphère (les océans, les rivières, les précipitations), la lithosphère (les terres immergées, les sols), la cryosphère (glace, neige, banquise, glaciers), la biosphère (les organismes vivants, \ldots), l’anthroposphère (l’activité humaine). Cette définition est équivalente à la première dans la mesure où les variations de l'état de l'atmosphère au moyen ou long terme ne peut se penser sans l'étude des couplages avec les autres compartiments du système climatique (voir figure~\ref{fig:pluri}). 

\sk
Les diverses composantes du système climatique interagissent entre elles sur un large intervalle d'échelles d'espace et de temps (figure~\ref{fig:pluri}). De plus, le système climatique dépend de facteurs externes, tels le rayonnement électromagnétique reçu du soleil ou les variations des paramètres orbitaux de la Terre. Le système climatique et ses évolutions est donc étudié par un ensemble de communautés scientifiques (climatologues, hydrologues, glaciologues, géochimistes, géologues, astronomes, \ldots) qui collaborent et confrontent leur point de vue depuis plusieurs décennies et communiquent leurs résultats dans les publications scientifiques\footnote{Dans ce chapitre, nous reprenons beaucoup de graphiques tirés de rapport du GIEC (Groupe d'experts intergouvernemental sur l'évolution du climat -- IPCC en anglais). Ce rapport offre un compte-rendu synthétique à destination du grand public et des décideurs. Il se base pour cela sur de nombreuses publications scientifiques. Le document le plus récent (AR4 2007) récapitulant les bases physiques est disponible à cette adresse \url{http://www.ipcc.ch/publications_and_data/ar4/wg1/en/contents.html}, chaque point peut être approfondi avec les publications référencées.}. Ces travaux ont permis d'établir que la Terre subit actuellement un réchauffement de son climat causé pour sa plus grande partie par l'activité humaine (on parle de changement \voc{anthropique}). Ce consensus n'élude pas les questions qui restent à résoudre et sur lesquelles la communauté scientifique travaille activement. Dans la mesure où les impacts sociétaux sont majeurs, les changements du climat d'origine anthropique et ses conséquences doivent être connus le plus précisément et rigoureusement possible. 

\mk
\section{Variations récentes de température et de composition}

\sk
\subsection{Variations de température}

\figun{0.8}{0.65}{decouverte/cours_meteo/evolutemp20.png}{Moyenne annuelle globale et hémisphérique de la température à la surface de la Terre de 1850 à 2006 relative à la moyenne de 1961 à 1990. Les données combinent des mesures sur les continents (à proximité de la surface) et sur les océans (\emph{sea surface temperature}). Les barres d'erreurs (intervalle~$5-95\%$) sont reportées sur la figure. La figure est adaptée des travaux du projet HadCRUT3 (Brohan et al. 2006). La courbe lissée bleue représente les variations décennales. Source~:~IPCC Fourth Assessment Report: Climate Change 2007 (AR4).}{fig:evolutemp20}
%%Global and hemispheric annual combined land-surface air temperature and SST anomalies (°C) (red) for 1850 to 2006 relative to the 1961 to 1990 mean, along with 5 to 95% error bar ranges, from HadCRUT3 (adapted from Brohan et al., 2006). The smooth blue curves show decadal variations

\sk
Le climat peut varier aux cours du temps, à des degrés et des échelles de temps divers. Le but de ce chapitre n'est pas de traiter les nombreux changements du climat survenus par le passé, une science appelée \voc{paléoclimatologie}, mais de s'intéresser aux changements récents survenus au cours du XXe et du XXIe siècle. L'évolution de la température de surface mesurée depuis les~$140$ dernières années est reportée en figure~\ref{fig:evolutemp20}. On peut remarquer la variabilité significative entre les années (variabilité interannuelle) et entre les décennies (variabilité décennale); on observe également une tendance générale au réchauffement d'environ~$0.6^{\circ}$C sur la période considérée. Ces changements peuvent être mis en perspective avec les variations depuis quelques milliers d'années, obtenues par le croisement et la combinaison statistique d'indicateurs indépendants~: cernes des arbres, coraux, carottes glaciaires, relevés historiques (Figure~\ref{fig:multitemp}, on peut noter sur ces données l'apparition entre~$1400$ et~$1900$ environ d'une période de climat légèrement plus froide appelée le \og petit âge glaciaire \fg). L'augmentation de température constatée récemment apparaît ainsi comme particulièrement exceptionnelle par rapport aux évolutions passées. La décénnie 1990-2000 a été la plus chaude de la période instrumentale sur la figure~\ref{fig:evolutemp20} et probablement parmi les décennies les plus chaudes du dernier millénaire.

\figun{0.85}{0.25}{decouverte/cours_meteo/yiou_multiproxies_temp.png}{Pourcentage de recouvrement des reconstructions \og multi-sources \fg~de la température annuelle moyenne. Les sources des données passées sont les suivantes~:~cernes des arbres (dendrochronologie), coraux, carottes glaciaires, relevés historiques (par exemple, dates de vendanges). Plus le grisé est intense plus les reconstructions se recoupent. La ligne noire indique la température instrumentale (i.e. observée par un thermomètre selon la même méthode au cours des années) de Brohan et al (2006) (voir figure~\ref{fig:evolutemp20}). Les valeurs sont centrées autour de la moyenne de la période 1961-1990. Les séries sont lissées avec une fenêtre de 30 ans. Source~:~J. Guiot et P. Yiou \emph{in} Le Climat à Découvert, CNRS éditions, 2011. Voir aussi IPCC Fourth Assessment Report: Climate Change 2007 (AR4).}{fig:multitemp}

\sk
\subsection{Causes possibles de changements climatiques}

\sk
Quelles sont les causes possibles d'un changement du climat terrestre~? Elles peuvent être nombreuses.

\begin{finger}
\item Une première cause peut être l'augmentation de gaz à effet de serre (H$_2$O, CO$_2$, CH$_4$, N$_2$O) dont l'action sur la température de l'atmosphère et de la surface est primordiale. Cette raison ne peut être évoquée pour les réchauffements ou refroidissements plus anciens, par exemple au début du XXe siècle, puisque les variations significatives de gaz à effet de serre causées par l'activité humaine sont relativement récentes. Néanmoins, les évolutions climatiques très anciennes, à l'échelle de plusieurs centaines de millions d'années, ont pu être causées en partie par la modification de la concentration en gaz à effet de serre dans l'atmosphère.
\item Une deuxième cause peut être trouvée dans les variations d'activité volcanique, qui injectent de larges quantités de poussière et de gaz (tels SO$_2$) dans l'atmosphère qui peuvent conduire à la formation d'aérosols sulfatés, dont l'effet est notamment de réduire la quantité de rayonnement incident. Ceci peut provoquer un refroidissement, typiquement~de l'ordre de~$0.5^{\circ}$C en moyenne globale, sur une période d'environ deux ans suivant une éruption majeure (e.g. Pinatubo en 1991, voir l'effet détectable sur la figure~~\ref{fig:evolutemp20}).
\item Une troisième explication réside dans les variations dans l'intensité du rayonnement solaire reçu par la Terre. En premier lieu, des variations absolues peuvent provenir de l'activité solaire. Des mesures directes très précises de cette quantité ont été obtenues depuis l'orbite de la Terre au cours des trois dernières décennies~: des variations d'environ~$0.1\%$ de la constante solaire apparaissent au cours d'un cycle solaire. Pour les variations passées, des preuves indirectes de fluctuations du rayonnement solaire incident existent, notamment lors du Minimum de Maunder au XVIIe siècle, une période au cours de laquelle un nombre exceptionnellement bas de taches solaires a été enregistré. En second lieu, des variations relatives (suivant les hémisphères ou les saisons) d'intensité du rayonnement solaire reçu peuvent provenir des variations sur de grandes échelles de temps des paramètres orbitaux de la Terre (cycles de Milankovitch).
\item Une quatrième explication tient au fait que, même sans changement de facteur extérieur au système climatique, des variations du climat peuvent apparaître en raison de facteurs et couplages internes, en particulier les interactions entre l'atmosphère et l'océan. La bonne prise en compte de cette variabilité interne ne doit d'ailleurs pas être négligée, sous peine de sous-estimer notamment les variations climatiques induites par les changements orbitaux.
\end{finger}

\sk
Tous ces effets (gaz à effet de serre, aérosols volcaniques, variations de constante solaire, couplages internes) ont été inclus au sein des modèles du système climatique développés de façon indépendante par plusieurs équipes dans le monde (figure~\ref{fig:causewarming}). Les changements climatiques à l'échelle du millénaire peuvent être reproduits correctement. Sur les échelles de temps plus courtes, la variabilité interannuelle (qui dépend fortement des couplages océan-atmosphère) est reproduite, ainsi que l'effet des éruptions volcaniques. Néanmoins, le réchauffement le plus récent, observé au cours du XXe siècle, ne peut être expliqué par des causes naturelles comme l'indique la figure~\ref{fig:causewarming}~: l'influence de l'activité humaine, notamment l'augmentation de la concentration de gaz à effet de serre dans l'atmosphère, ne peut être négligée. De nombreux autres indices montrent que le réchauffement climatique entamé dans la seconde partie du XXe siècle et le début du XXIe siècle peut être principalement imputé à l'activité humaine et l'émission de gaz à effet de serre (voir aussi figure~\ref{fig:forcrad}).

\figside{0.55}{0.43}{decouverte/cours_meteo/climatadecouvert_119.png}{Comparaison entre les anomalies de temperature globale de surface ($^{\circ}$C) observée (en noir) et simulée par plusieurs modèles climatiques couplés globaux. En (a) les forçages anthropiques et naturels sont pris en compte, tandis qu’en (b), seuls les forçages naturels ont été considérés. Source~:~IPCC Fourth Assessment Report: Climate Change 2007 (AR4).}{fig:causewarming}

%\figun{0.8}{0.4}{decouverte/cours_meteo/figure-spm-4-l.png}{Comparaison des changements observés sur la température de surface aux échelles globale et continentale, avec les résultats de simulations de modèles climatiques utilisant les forçages naturels et anthropiques. Les moyennes décennales des observations sont montrées pour la période 1906–2005 (ligne noire) représentée selon le centre de la décennie et relative à la moyenne correspondante sur la période 1901–1950. Les lignes en pointillés sont utilisées quand la couverture spatiale est inférieure à 50\%. Les bandes bleues ombrées représentent l’intervalle de 5–95\% pour 19 simulations de cinq modèles climatiques qui n’utilisaient que les forçages naturels dus à l’activité solaire et aux volcans. Les bandes rouges ombrées représentent l’intervalle de 5–95\% pour 58 simulations provenant de 14 modèles climatiques utilisant des forçages à la fois naturels et anthropiques. Source~:~IPCC Fourth Assessment Report: Climate Change 2007 (AR4).}{fig:anthropcause}

\sk
\subsection{Changements de composition récents et cycles naturels}

\sk
Il faut d'ores et déjà distinguer l'\voc{empreinte climatique} de l'homme (liée à l'impact au long terme que peuvent avoir les activités humaines en modifiant la composition atmosphérique) de la \voc{pollution atmosphérique} qui traite de la qualité de l'air dans lequel l'homme évolue au quotidien. Un exemple d'empreinte climatique a été vu dans le chapitre de chimie atmosphérique lorsque l'action néfaste des CFC sur l'équilibre du cycle de l'ozone stratosphérique a été abordée. 

\sk
Un autre exemple d'empreinte climatique humaine, qui nous intéresse tout particulièrement ici, est l'augmenta\-tion de la concentration de gaz à effet de serre dans l'atmosphère terrestre. Le plus important des gaz à effet de serre rejeté dans l'atmosphère est le CO$_2$. Son augmentation en concentration par rapport à la période pré-industrielle est d'environ~$30\%$ (figure~\ref{fig:maunaloa}). Cette augmentation est principalement due à la combustion de carburants fossiles (comme l'ont montrés de multiples relevés d'isotopes du carbone) et à la déforestation. D'autres gaz à effet de serre ont augmenté significativement, comme le méthane (145\%) et l'oxyde nitreux (15\%) (figure~\ref{fig:giecrad}).

\figside{0.5}{0.25}{decouverte/cours_meteo/Mauna_Loa_Carbon_Dioxide.png}{Relevé des concentrations de dioxyde de carbone atmosphérique à l'observatoire de Mauna Loa à Hawaii. Les courbes obtenues portent le nom de \og courbes de Keeling \fg~du nom du scientifique qui a supervisé ses observations. D'autres jeux de données sont venus confirmer les tendances observées depuis l'observatoire de Mauna Loa; on montre celles-ci en exemple car il s'agit de l'enregistrement direct le plus étendu dans le temps de la concentration du CO$_2$ dans l'atmosphère. La courbe rouge montre les concentrations moyennées sur un mois et la courbe bleue montre le résultat d'une moyenne glissante sur~$12$ mois. Noter les fluctuations saisonnières de CO$_2$ (reportées dans le sous-graphique en bas à droite) qui correspondent à la consommation de dioxyde de carbone par la végétation~: ces fluctuations sont plus marquées dans l'hémisphère nord en raison de la couverture végétale (forêts) plus importante que dans l'hémisphère sud. Source~:~Robert A. Rohde \url{http://www.globalwarmingart.com} à partir de données publiées et référencées.}{fig:maunaloa}

\figside{0.6}{0.75}{decouverte/cours_meteo/ges.jpg}{Rapports de mélange du dioxyde de carbone~CO$_2$, méthane~CH$_4$ et oxyde nitreux~N$_2$O au cours des 10 000 dernières années (larges figures) et depuis 1750 (petits inserts). Les données indicatrices des changements de la composition de l’atmosphère au cours du dernier millénaire mettent en évidence l’augmentation rapide des gaz à effet de serre qui est imputable principalement à la croissance économique depuis 1750. Les gaz sont bien mélangés dans l’atmosphère et leurs concentrations depuis 1750 reflètent les émissions provenant des sources à travers le monde. Les mesures proviennent de carottes glaciaires (différentes couleurs sont utilisées pour les diverses études scientifiques utilisées) et de campagnes atmosphériques (lignes rouges). Les forçages radiatifs comparés à la valeur de 1750 pour chacun des gaz représentés sont reportés sur la droite des larges figures. Figure tirée du 4ème rapport du GIEC en 2007.}{fig:giecrad}

\sk
Dans la nature, le dioxyde de carbone~CO$_2$ est la principale espèce échangée entre les divers réservoirs qui participent au cycle du carbone~: atmosphère, océan, sol, biosphère. Les flux de carbone sous la forme de dioxyde de carbone qui entrent et sortent de l'atmosphère sont considérables~: à peu près un quart de la quantité contenue dans l'atmosphère est renouvelée par le cycle du carbone chaque année, la moitié échangée avec la biosphère continentale et l'autre moitié via les processus physico-chimiques avec la surface de l'océan. Sur des milliers d'années, avant que les activités humaines n'induisent une perturbation significative, la concentration atmosphérique de dioxyde de carbone s'est maintenue à une valeur constante, indiquant un état stationnaire du cycle du carbone et des échanges entre réservoirs (voir figure~\ref{fig:giecrad}). 

\sk
Pourquoi les émissions supplémentaires de CO$_2$ dans l'atmosphère induites par les activités humaines se traduisent-elles \emph{in fine} par une augmentation de la concentration du CO$_2$ dans l'atmosphère~? Les détails de sa distribution entre les réservoirs majeurs (océan, biosphère) peut être estimée à partir d'une combinaison de mesures des changements de concentrations atmosphériques de CO$_2$ et d'oxygène~: la capture du dioxyde de carbone par l'océan provoque peu de changements en oxygène atmosphérique, alors que la biosphère continentale relâche de l'oxygène et consomme du dioxyde de carbone dans l'atmosphère. Au cours des dernières décénnies néanmoins, une partie significative des émissions de CO$_2$ d'origine anthropique (environ~$40\%$) est restée dans l'atmosphère~: il apparaît que les puits de CO$_2$ (biosphère continentale et océan) ne peuvent résorber en quelques décennies l'augmentation des gaz à effet de serre provoquée par l'activité humaine (figure~\ref{fig:bilancarbone}). Les échanges entre les différents réservoirs se font en effet sur des échelles de temps larges, qui de plus diffèrent selon les réservoirs, de la décénnie pour la biosphère continentale et les compartiments supérieurs de l'océan, jusqu'au millénaire pour la partie profonde de l'océan. La combinaison de ces facteurs ne peut être résumée en une constante de temps unique, néanmoins il est certain que la perturbation anthropique de~CO$_2$ ne peut être résorbée en moins d'une centaine d'années. Ainsi, une partie substantielle du dioxyde de carbone relâché dans l'atmosphère par les activités humaines aujourd'hui va affecter la concentration globale pour (au moins) le siècle à suivre. 

\sk
Le problème se pose en des termes similaires pour l'oxyde nitreux dont la durée de vie dans l'atmosphère est similaire au dioxyde de carbone. Ce problème est moins critique pour le méthane dont la durée de vie, contrainte principalement par des constantes de réaction chimiques, est plutôt de l'ordre de la dizaine d'années. Le cas de la vapeur d'eau est quant à lui singulier~: il s'agit du gaz à effet de serre le plus puissant de l'atmosphère terrestre, un des principaux responsables de l'effet de serre naturel. Les activités humaines n'induisent cependant pas de modifications climatologiques de sa concentration dans l'atmosphère, dans la mesure où le temps caractéristique du cycle de l'eau est de l'ordre de quelques jours\footnote{Il est cependant fort probable que le changement climatique récent s'accompagne de modifications du cycle de l'eau, voir sections suivantes.}.
%La durée de vie des aérosols est quant à elle encore plus faible (quelques jours dans la troposphère, une poignée d'années dans la stratosphère).

\figside{0.6}{0.3}{decouverte/cours_meteo/bilancarbone.png}{Evolution annuelle depuis 1960 des émissions anthropiques de CO2 (combustibles fossiles et déforestation, en rouge), du taux d’accroissement de CO$_2$ atmosphérique (violet), et des puits océanique (bleu) et continentaux (vert). Cette figure montre que les puits de CO$_2$ ne sont pas suffisamment efficaces pour absorber les émissions. En conséquence, le contenu de CO$_2$ dans l’atmosphère augmente sous l'effet des émissions anthropiques. D’apres Le Quere et al. 2009. Source~:~N. Metzl et P. Ciais \emph{in} Le Climat à Découvert, CNRS éditions, 2011.}{fig:bilancarbone} %%% l'océan n'est pas partout un puits de CO2.

\sk
\section{Impacts radiatifs de l'augmentation des gaz à effet de serre}

\sk
Le mécanisme d'effet de serre est crucial pour comprendre le lien entre l'augmentation de la concentration de dioxyde de carbone dans l'atmosphère (figures~\ref{fig:maunaloa} et \ref{fig:giecrad}) et l'augmentation des températures dans la basse atmosphère (figures~\ref{fig:evolutemp20} et~\ref{fig:multitemp}). Il n'est pas suffisant de constater que l'augmentation des deux quantités est corrélée~; corrélation ne vaut pas causalité. Il faut un raisonnement physique pour établir la causalité.

\sk
\subsection{Naturel contre anthropique}

\sk
Rappelons que les gaz à effet de serre possèdent des bandes d'absorption très marquées dans les longueurs d'onde infrarouge du rayonnement électromagnétique; c'est principalement dans ce domaine de longueur d'onde que le rayonnement thermique est émis par la surface de la Terre. Les gaz à effet de serre vont donc absorber ce rayonnement, puis le réémettre vers la surface et vers l'espace. Ainsi, une partie du rayonnement émis par la Terre pour se refroidir ne pourra être évacuée vers l'espace et contribue à augmenter la température de la troposphère terrestre. La présence de gaz à effet de serre dans l'atmosphère terrestre aux concentrations historiques, avant l'activité industrielle humaine, contribue à un \voc{effet de serre naturel} d'environ~$30^{\circ}$C qui rend notre planète Terre habitable. Les gaz à effet concernés sont la vapeur d'eau~H$_2$O, le dioxyde de carbone~CO$_2$, le méthane~CH$_4$, l'oxyde nitreux~N$_2$O, l'ozone~O$_3$ (les nuages induisent également un léger effet de serre). %Le rayonnement émis par la surface terrestre, principalement dans l'infrarouge, est également absorbé par les espèces précitées (vapeur d'eau, CO$_2$, CH$_4$) et réémis à la fois vers l'espace et vers la surface. Ainsi, une partie du rayonnement émis par la surface est \ofg{piégée}, n'est pas évacuée vers l'espace et contribue à augmenter la température de la surface terrestre. Ce phénomène est désigné par le terme d'\voc{effet de serre} et les gaz qui en sont responsables s'appellent les \voc{gaz à effet de serre}.


\sk
Que se passe-t-il si la concentration en CO$_2$, CH$_4$, N$_2$O augmente dans l'atmosphère sous l'effet de l'activité humaine~? A l'effet de serre naturel vient s'ajouter un \voc{effet de serre d'origine anthropique}. Est-il significatif~? Pour répondre à cette question, on raisonne sur le CO$_2$ qui est tenu pour responsable des 2/3 du réchauffement climatique le plus récent. Supposons pour simplifier que la concentration de dioxyde de carbone dans l'atmosphère double par rapport à sa valeur pré-industrielle qui était~$280$~ppm. Cette valeur de~$560$~ppm n'est pas irréaliste~: la valeur en 2010 avoisine les~$390$~ppm~; si aucune action n'est prise pour limiter les émissions actuelles, la concentration en 2100 sera d'environ~$560$~ppm. Voyons quel effet peut avoir cette augmentation sur les flux radiatifs, la structure thermique de l'atmosphère et, finalement, la température de surface pour faire écho aux figures~\ref{fig:evolutemp20} et~\ref{fig:multitemp}.


%La bande d'absorption qui joue un rôle tout à fait central pour l'effet de serre du CO$_2$ est située à une longueur d'onde de~$15$~$\mu$m. Ainsi le rayonnement émis par le système [atmosphère + surface] vers l'espace est principalement déterminé par la réémission dans les longueurs d'onde autour de~$15$~$\mu$m du rayonnement incident de la surface terrestre, absorbée par le CO$_2$ atmosphérique. Si l'on se représente ce phénomène par la présence de plusieurs couches de CO$_2$ à une température donnée, la couche la plus proche de la surface va absorber le rayonnement venant de la surface et le réémettre vers la surface et la couche juste au-dessus; la couche juste au-dessus va recevoir ce rayonnement, l'absorber, puis le réémettre vers la couche encore au-dessus et la couche en-dessous, etc jusqu'au sommet de l'atmosphère. Il faut donc faire le bilan des échanges de rayonnement entre toutes les couches. Néanmoins, la contribution des couches atmosphériques n'est pas la même suivant la longueur d'onde considérée. Par exemple, supposons que l'on soit à une longueur d'onde correspondante à une bande d'absorption très marquée du CO$_2$. Dans les couches atmosphériques proches de la surface, la totalité du rayonnement incident de la surface est absorbée, puis réémise à une température légèrement inférieure. A son tour, cette couche atmosphérique voit son émission absorbée par la couche supérieure. Ainsi il existe pour chaque longueur d'onde une altitude \og équivalente \fg~qui représente la principale contribution (en terme de couche atmosphérique) au rayonnement émis vers l'espace au sommet de l'atmosphère. %dont la température est représentative de l'émission dans la bande d'absorption du CO$_2$ à 15~$\mu$m. 
%Pour illustrer ce concept, situons-nous tout d'abord à une longueur d'onde hors de la bande d'absorption du CO$_2$ à~$15$~$\mu$m. Le rayonnement infrarouge émis par la surface n'est pas absorbé par le CO$_2$ atmosphérique. L'altitude équivalente d'émission de l'atmosphère est donc très proche de la surface. Situons-nous ensuite sur les bords de la bande d'absorption à~$15$~$\mu$m du CO$_2$, où l'absorption est modérée, la contribution à la radiation quittant l'atmosphère se fera surtout depuis les basses couches de l'atmosphère. Situons-nous enfin dans la bande d'absorption à~$15$~$\mu$m du CO$_2$, l'absorption est très forte et donc l'altitude d'émission très élevée. Cette altitude d'émission dépend également de la concentration du composant~: une quantité négligeable de composant très absorbant donne le même type d'altitude d'émission qu'un composant très peu absorbant en grande quantité. Ceci explique que le CO$_2$ qui est très absorbant à~$15$~$\mu$m puisse avoir une grande opacité dans l'infrarouge.

\sk
\subsection{Epaisseur optique et hauteur équivalente d'émission}

\sk
Considérons une espèce~$X$ bien mélangée dans l'atmosphère, qui absorbe dans un intervalle de longueur d'onde donné. A la longueur d'onde~$\lambda$, son \voc{épaisseur optique}~$t_\lambda$ s'écrit
\[ \boxed{ t_\lambda = \int_{0}^{z\e{sommet}} \, k_\lambda \, \rho_X \, \dd z } \]
où $k_\lambda$ est un coefficient d'absorption massique en m$^2$~kg$^{-1}$ et $\rho_X$ est la densité d'absorbant~X. Le nom d'épaisseur optique est assez intuitif. Si un flux de rayonnement~$\Phi_\lambda$ à la longueur d'onde~$\lambda$ est émis à la base de l'atmosphère, le flux observé au sommet de l'atmosphère est d'autant plus réduit qu'à cette longueur d'onde l'épaisseur optique de l'atmosphère traversée est grande\footnote{Si l'extinction est uniquement due à de l'absorption, sans diffusion, on a une relation directe entre l'épaisseur optique et le coefficient d'absorption de la couche~: 
\[\alpha_\lambda = 1 - e^{- \frac{t_\lambda}{\cos\theta}} \] où~$\theta$ est l'angle d'incidence du rayonnement danns la couche.}. La formule ci-dessus ne fait qu'exprimer le fait que la réduction du flux (l'extinction) est plus d'autant plus marquée que 
\begin{citemize}
\item l'espèce considérée est très absorbante dans la longueur d'onde considérée ($k_\lambda$ grand)~;
\item l'espèce considérée est présente en grande quantité ($\rho_X$ grand).
\end{citemize}
Ainsi, le dioxyde de carbone~CO$_2$, bien qu'étant un composant minoritaire ($\rho$ faible), peut atteindre des épaisseurs optiques très grandes dans les intervalles de longueur d'onde où il est très fortement absorbant ($k_\lambda$ élevé), par exemple dans l'infrarouge autour de~$15$~$\mu$m. Autrement dit, un composant minoritaire en quantité peut avoir un rôle majoritaire radiativement.


\sk
Dans l'infrarouge, la situation est un peu plus complexe puisqu'au phénomène d'absorption s'ajoute le phénomène d'émission thermique pour les températures usuellement rencontrées à la surface et dans l'atmosphère terrestre. Si l'on découpe l'atmosphère en couches élémentaires, il convient de faire le bilan de ce que chaque couche reçoit et émet au voisinage d'une longueur d'onde~$\lambda$ dans l'infrarouge où le CO$_2$ est absorbant
\begin{citemize}
\item La surface émet vers l'atmosphère un rayonnement infrarouge à la température~$T_s$
\item La couche atmosphérique~$1$ proche de la surface reçoit ce rayonnement infrarouge de la surface~: une partie de ce rayonnement, d'autant plus grande que l'épaisseur optique de la couche est élevée, est absorbée et la partie restante est transmise à la couche atmosphérique~$2$ située au dessus. A noter que le même type de raisonnement doit être effectué pour le rayonnement reçu de la couche~$2$ au-dessus. La couche atmosphérique~$1$ émet quant à elle un rayonnement infrarouge à la température~$T_1$ déterminée par l'équilibre radiatif entre toutes ces contributions. Ce rayonnement est émis à la fois vers les couches inférieures et vers les couches supérieures.
\item Le raisonnement se poursuit de proche en proche jusqu'à pouvoir déterminer le rayonnement émis au sommet de l'atmosphère vers l'espace.
\end{citemize}

\sk
De ce raisonnement, on peut tirer deux observations
\begin{enumerate}
\item Si l'épaisseur optique de l'atmosphère dans l'infrarouge est nulle, le rayonnement émis au sommet de l'atmosphère est très proche de celui émis par la surface.
\item Si l'épaisseur optique de l'atmosphère dans l'infrarouge est très élevée, le rayonnement émis par la surface est entièrement absorbé par les couches atmosphériques proches de la surface~; le rayonnement émis au sommet de l'atmosphère est déterminé par l'état de l'atmosphère à une hauteur~$h$ intermédiaire entre la surface et le sommet de l'atmosphère\footnote{Cette hauteur reste inférieure au sommet de l'atmosphère car la densité de l'air, donc la densité d'absorbant~$\rho_X$, diminue exponentiellement avec~$z$ et, à partir d'une certaine altitude, devient si faible que~$k_\lambda \, \rho_X$ est négligeable même pour de grandes valeurs de~$k_\lambda$.}. 
\end{enumerate}
On peut en fait montrer que, pour des variations modérées de température atmosphérique, la contribution maximale au rayonnement à une longueur d'onde~$\lambda$ sortant au sommet de l'atmosphère provient d'une hauteur~$h$ telle que l'épaisseur optique soit~$t_\lambda = 1$. Cette hauteur~$h$ est dénommée \voc{hauteur équivalente d'émission}. Pour déterminer~$h$, il suffit donc de partir du sommet de l'atmosphère, où par définition~$t_\lambda = 0$, de descendre dans l'atmosphère et de déterminer quand la valeur de l'épaisseur optique~$t_\lambda$ atteint~$1$.
\begin{citemize}
\item Plus l'espèce est absorbante ($k_\lambda$ grand), moins bas on doit descendre pour atteindre~$t_\lambda = 1$.
\item Plus la concentration d'espèce absorbante est grande ($\rho_X$ grand), moins bas on doit descendre pour atteindre~$t_\lambda = 1$.
\end{citemize}

\sk
\subsection{Forçage radiatif induit par la variation de concentration en CO$_2$}

\sk
Ainsi, pour une espèce comme le CO$_2$ qui est très absorbante à une longueur d'onde de~$15$~$\mu$m, la hauteur équivalente d'émission~$h$ est assez élevée et se situe entre la haute troposphère et la stratosphère\footnote{Suivant que le centre de la bande d'absorption à~$15$~$\mu$m ou les ailes de cette bande sont considérés.}. Pour cette même espèce très absorbante dans l'infrarouge, un doublement de concentration va induire une augmentation de la hauteur équivalente d'émission de~$h$ à~$h'$. On peut montrer que cette augmentation est d'environ~$h'-h \simeq 3$~km. 

\sk
Or, nous avons vu dans les chapitres précédents que l'équilibre entre radiation (notamment l'effet de serre naturel) et convection (i.e. dynamique atmosphérique selon la verticale) impose que la température décroît selon l'altitude d'en moyenne~$6^{\circ}$C~par kilomètre d'ascension. La température d'émission à une hauteur~$h'$ est donc beaucoup plus faible qu'à une hauteur~$h$~: le rayonnement infrarouge sortant au sommet de l'atmosphère est donc également plus faible. La baisse de flux induit par ces changements dans des longueurs d'onde voisines de~$15$~$\mu$m est d'environ~$3$~W~m$^{-2}$ avec le raisonnement simplifié présenté ici. Des estimations beaucoup plus précises conduisent à~$4$~W~m$^{-2}$ en prenant en compte, dans un raisonnement plus complet que celui présenté ici, la structure détaillée de la bande d'absorption du dioxyde de carbone, les légers recouvrements avec les bandes d'absorption de la vapeur d'eau, la structure verticale de l'atmosphère et la couverture nuageuse. Ce changement dans la valeur du rayonnement émis au sommet de la troposphère est appelé~\voc{forçage radiatif}. 

\sk
Il est alors aisé de comprendre que l'augmentation de la quantité de CO$_2$ dans l'atmosphère conduit à une augmentation de la température de la surface. Au sommet de l'atmosphère (ou, de façon équivalente ici, au sommet de la troposphère), il y a un équilibre entre le flux solaire incident (longueurs d'onde visible) et le flux infrarouge sortant (longueurs d'onde infrarouge). Nous venons de voir qu'un doublement du CO$_2$ dans l'atmosphère conduit à une diminution du flux infrarouge sortant (émis vers l'espace). Ainsi il y a un excédent d'énergie qui conduit à l'augmentation des températures dans la troposphère et en particulier de la température de surface, afin que l'équilibre soit rétabli. Le calcul montre que l'augmentation correspondante est de~$1.2^{\circ}$C. 

\sk
L'effet des différents gaz à effet de serre (CO$2$, CH$_4$, N$_2$O, O$_3$) peut être comparé, entre eux et avec les autres sources de variabilité, via le forçage radiatif qu'ils induisent (figure~\ref{fig:forcrad}). On peut noter l'influence non négligeable du méthane qui est pourtant en quantité plus faible que le CO$_2$. Il faut d'ailleurs se garder d'additionner directement les différents forçages indiqués sur la figure~\ref{fig:forcrad}. Seule une combinaison prenant en compte les échelles d'impact spatiale (du local au global) et temporelle (de l'intermittence à la permanence) le permet, telle celle indiquée sur la figure~\ref{fig:forcrad}. Notamment, le forçage par le dioxyde de carbone ou celui induit par les changements de constante solaire sont globaux, alors que les forçages par les aérosols et les nuages, ou par l'ozone troposphérique, sont plus locaux. Actuellement, le forçage net d'origine anthropique est de~$1.6$~W~m$^{-2}$.

\figside{0.63}{0.33}{decouverte/cours_dyn/forcage_radiatif_IPCC.png}{Estimation des forçages radiatifs anthropiques ou naturels depuis le début de l’ère industrielle. Chaque rectangle représente l’estimation moyenne du forçage et les traits noirs indiquent l’incertitude associée. Les forçages positifs (en rouge) correspondent à un réchauffement du climat et ceux négatifs (en bleu) un refroidissement. La dernière colonne (Level Of Scientific Understanding) donne le niveau actuel de compréhension scientifique du forçage. Source~:~IPCC Fourth Assessment Report: Climate Change 2007 (AR4). Voir également P. Dubuisson \emph{in} Le Climat à Découvert, CNRS éditions, 2011}{fig:forcrad}

\sk
\subsection{Quelques subtilités supplémentaires}

\sk
La valeur de variation de température calculée précédemment est en fait sous-estimée. Elle suppose que l'augmentation de température causée par l'effet de serre anthropique ne s'accompagne pas de changement dans la structure thermique de l'atmosphère, dans la couverture nuageuse ou dans la quantité de vapeur d'eau dans l'atmosphère. Ces changements peuvent induire des \voc{rétroactions}, c'est-à-dire qu'ils ont tendance soit à renforcer la perturbation initiale qui leur a donné naissance (ici l'augmentation de température), dans ce cas il s'agit de \voc{rétroaction positive}, soit au contraire à la contrecarrer s'il s'agit de \voc{rétroaction négative}. Les quatre rétro-actions principales à considérer sont les suivantes~:
\begin{finger}
\item L'augmentation de la température favorise l'évaporation, donc la quantité de vapeur d'eau dans l'atmosphère augmente; la vapeur d'eau étant un gaz à effet de serre très efficace, la rétroaction est fortement positive.
\item Le changement de contenu en vapeur d'eau modifie la structure en température de l'atmosphère (le gradient adiabatique humide); l'effet moyen de cette rétroaction sur la température a tendance à être négative.
\item Si de la glace (albédo élevé) disparaît des surfaces continentales et des océans (albédo faible), la diminution résultante de l'albédo provoque un réchauffement de la surface; il s'agit donc d'une rétroaction positive.
\item Les nuages ont deux effets contraires sur la température. D'une part, ils réfléchissent le rayonnement solaire dans le domaine visible, donc réduisent la quantité de rayonnement solaire qui parvient à la surface. D'autre part, ils absorbent le rayonnement infrarouge émis par la surface de la Terre, donc réduisent la quantité de rayonnement infrarouge perdue vers l'espace. La rétroaction des nuages est la plus complexe des rétroactions (en témoigne la barre d'incertitude sur la figure~\ref{fig:forcrad}) et fait l'objet de recherches actives.
\end{finger}
S'ajoutent à ces rétroactions, les changements provoqués dans la circulation atmosphérique et océanique, également appelés rétroactions dynamiques, ainsi que les rétroactions biogéochimiques, qui font notamment intervenir l'évolution de la végétation. 

\sk
Ces rétroactions donnent tout le \og sel \fg~scientifique à la détermination précise des variations de température dans une situation d'augmentation anthropique des gaz à effet de serre. Les estimations actuelles s'accordent sur un réchauffement de la surface de~$2.5^{\circ}$C pour un doublement de la quantité de CO$_2$, mais l'intervalle des valeurs obtenues par les différentes équipes à l'international travaillant sur le sujet est~$1.5-4.5^{\circ}$C. Ainsi, si la communauté scientifique est unanime sur le réchauffement futur du climat et de la cause principale (l'augmentation des gaz à effet de serre), il reste à préciser son amplitude. Néanmoins, quel que soit le résultat exact, un réchauffement au moins supérieur ou égal à~$1^{\circ}$C est considérable~:
\begin{citemize}
\item Il est à comparer à la différence de $5-6^{\circ}$C entre les âges glaciaires et interglaciaires dans le passé.
\item Ce changement dans la température moyenne globale devrait se réaliser sur des échelles de temps de l'ordre d'un siècle, ce qui est significativement plus rapide que les changements climatiques survenus par le passé.
\end{citemize}

%\sk Le forçage radiatif causé par l'augmentation des aérosols à cause des sources anthropogéniques est principalement négatif, c'est-à-dire tend à faire baisser le température de la troposphère. Noter la grande incertitude. Une manière dont les particules dans l'atmosphère peuvent influencer le forçage radiatif apparaît via leur effect sur la formation de nuages (puisqu'ils forment, nous l'avons vu dans un chapitre précédent, des noyaux de condensation). Il s'agit d'un forçage radiatif indirect. Si les particules sont présentes en grand nombre lorsque les nuages se forment, le nuage résultant contient un grand nombre de petites gouttes, plus petites que si moins de noyaux de condensation sont disponibles. Le pouvoir réfléchissant (albédo) pour le rayonnement solaire incident d'un tel nuage est plus élevé que pour un nuage contenant une quantité plus petite de particules plus grosses. Cet effet d'albedo augmente la perte d'énergie.

%On doit par contre considérer en plus de l'absorption de rayonnement, l'émission dans l'infrarouge par l'atmosphère (figure \ref{fig:schwartzschild}). Dans les conditions appelées \emph{équilibre thermodynamique local\footnote{Ces conditions sont vérifiées si les collisions entre molécules sont plus fréquentes que l'absorption ou émission de rayonnement. Les molécules émettrices ont alors la même température que leur environnement}}, qui sont valables jusque vers 60~km d'altitude environ, le rayonnement émis (vers le haut et vers le bas) par une couche mince d'atmosphère dépend de sa température et de son coefficient d'absorption suivant la loi de Kirchoff. Pour un faisceau lumineux traversant une couche mince d'atmosphère, la variation de luminance vaut alors: \[dL_\lambda=\left(-L_\lambda+B_\lambda(T)\right)\mu d\tau_\lambda\] Le premier terme du second membre représente l'absorption du rayonnement incident, le deuxième l'émission par les gaz de la couche. La loi de Kirchoff fait qu'ils sont multipliés par le même coefficient $\mu\tau_\lambda$ qui donne le coefficient d'absorption et d'émission de la couche. Cette équation est appelée \emph{équation de Schwartzschild}. Son intégration entre une altitude $z_0$ et l'infini (espace) donne: \[L_\lambda(\infty)=L_\lambda(z_0)e^{-\mu\tau_\lambda(z_0,\infty)}+\int_{z_0}^\infty B_\lambda(T)e^{-\mu\tau_\lambda(z,\infty)}\mu\rho_ak_\lambda dz\] Le rayonnement sortant qu sommet de l'atmosphère est donc la somme du rayonnement présent en $z_0$ diminué de l'absorption entre $z_0$ et le sommet de l'atmosphère (premier terme), et de l'intégrale de la contribution du rayonnement émis par chaque couche au dessus de $z_0$. Comme pour le transfert dans le visible, on peut montrer (en supposant que $T$ varie peu) que la contribution maximale au rayonnement sortant à une longueur d'onde $\lambda$ provient d'une épaisseur optique de $\tau_\lambda=1$ à partir du sommet de l'atmosphère.
%De façon plus générale, on a vu que le rayonnement sortant provenait majoritairement de la région de l'atmosphère autour d'une épaisseur optique de 1 à partir du sommet. Cette région dépend de la longueur d'onde: proche de la surface dans la fenêtre transparente, dans la haute troposphère dans les bandes d'absorption du CO$_2$, autour de 2~km dans celles de la vapeur d'eau. Comme la température décroit à partir de la surface, le rayonnement sortant est donc émis à des températures inférieures à $T_s$, et on peut écrire qu'il vaut \[IR_{sommet}=\sigma T_s^4 (1-\epsilon)=\sigma T_{eq}^4\] Où $\epsilon>0$ représente l'effet de serre. La valeur de $\epsilon$ augmente quand la température d'émission vers l'espace diminue par rapport à celle de surface, typiquement parce que l'altitude d'émission augmente. 
%%Émission nette par la vapeur d'eau, l'ozone, le CO2 et les autres gaz à effet de serre : Il s'agit du flux énergétique net émis sous forme de rayonnement énergétique (infrarouge) par l'ensemble des molécules de l'atmosphère. L'émission infrarouge est associée à un refroidissement local. Comme le Corps Noir, les molécules émettent un rayonnement pour se refroidir et équilibrer l'énergie absorbée. L'émission n'a lieu que dans les bandes d'absorption (ou d'émission). Il faut donc que la température locale soit celle du Corps Noir émettant à la longueur d'onde de la bande d'émission. Ainsi, plus on descend dans l'atmosphère plus l'émission se fera par les bandes centrées sur de faibles longueurs d'onde. Émission IR et refroidissement atmosphériques sont doncintimement liés. La stratosphère est principalement refroidie par l'émission IR du gaz carbonique. Ce refroidissement est associé à l'émission par la bande située à 15 μm. Dans la haute stratosphère, la bande d'émission de l'ozone à 9.6 μm permet l’émission IR et le refroidissement atmosphérique. Cependant l'ozone absorbe principalement les radiations solaires et ne peut être considérée comme un gaz à effet de serre (dans la stratosphère). La vapeur d'eau émet également dans la stratosphère dans la bande à 8 μm. La troposphère est principalement refroidie par l'émission de la vapeur d'eau dans la bande située à 6.3 micromètres.

%rétroactions entre température atmosphérique et concentration de CO2 : le CO2 influence la température atmosphérique via l'effet de serre, mais la concentration de CO2 est elle-même influencée par des factors qui dépendent de la température atmosphérique. 

\figun{0.7}{0.35}{decouverte/cours_meteo/pred_clim.png}{Projections de la variation de la température globale de 1950 à 2100 selon trois scénarios (couleurs) et à partir de 14 modèles (spaghettis). La moyenne multimodèle pour chaque scénario est en trait plein de la couleur correspondante, et les observations (sur environ un demi-siècle) sont représentées en noir. Le scénario B1 suppose des émissions quasi-constantes à partir de 2000. Le scénario A1B suppose des émissions qui doublent d'ici 2050. Le scénario A2 suppose que les émissions sont multipliées par~4 d'ici 2100. Source~:~IPCC Fourth Assessment Report: Climate Change 2007 (AR4).}{fig:predclim} 

\sk
\section{Prédiction du changement climatique et impacts}

\sk
Les modèles climatiques couplent les divers compartiments indiqués sur la figure~\ref{fig:pluri} pour livrer une analyse la plus complète possible de l'intrication des différents phénomènes qui contribuent au changement récent du climat terrestre, l'augmentation des gaz à effet de serre en tête. Les premières prédictions de modèles couplés dans les années 80 pour les décennies suivantes se sont révélées conformes aux observations entre 1990 et 2010; entre temps, les modèles se sont améliorés en incluant des interactions plus complètes entre les différents compartiments du système climatique, qui ont permis de préciser les premiers diagnostics [Figure~\ref{fig:gcm}].
% Les résultats de simulations présentés dans ce chapitre proviennent de plusieurs modèles indépendants, validé pour le climat courant, autant en terme de quantités moyennes que des variabilité; ils ont été également validés pour la survenue d'anomalies climatiques ponctuelles, telles celles induites par des éruptions volcaniques ou des événements El Ni\~ no; et pour des cas test paléo-climatiques. 
Par ailleurs, ces modèles sont désormais un outil utile pour effectuer des projections afin de guider les politiques sur les actions à mener (figure~\ref{fig:predclim}). Ainsi, plusieurs scénarios d'augmentation des gaz à effet de serre (principalement CO$_2$) sont considérés, correspondant à des scénarios de projection géopolitique et économique. Les modèles du système climatique contribuent à prédire et comprendre quelle serait l'évolution de la température en fonction du scénario considéré. Ils peuvent également servir à comprendre les conséquences du changement climatique sur les précipitations (figure~\ref{fig:precip}) et sur la survenue d'événements extrêmes (canicules, précipitations intenses, sécheresses, \ldots). Sur ces points, suivant les régions considérées, les modèles ne s'accordent pas toujours. Les efforts de recherche entrepris doivent se poursuivre. Néanmoins, de grandes tendances commencent à se dégager, notamment sur la survenue plus fréquente d'épisodes caniculaires plus chauds.

\figside{0.6}{0.4}{decouverte/cours_meteo/giec2007_gcm.jpg}{Evolution des phénomènes et éléments du système climatique intégrés dans les modèles numériques de climat. FAR correspond à 1990. SAR correspond à 1995. TAR correspond à 2001. AR4 correspond à 2007. Figure tirée du 4ème rapport du GIEC en 2007.}{fig:gcm}

\figun{0.95}{0.25}{decouverte/cours_meteo/figure-spm-7-l.png}{Changements relatifs des précipitations (en pourcentages) pour la période 2090-2099 par rapport à 1980–1999. Les valeurs sont issues de moyennes sur de nombreux modèles basés sur le scénario A1B du RSSE pour les mois de décembre à février (à gauche) et de juin à août (à droite). Les aires blanches représentent les zones où moins de 66\% des modèles concordent sur le signe du changement, les zones hachurées correspondent aux cas où plus de 90\% des modèles concordent sur le signe du changement. Source~:~IPCC AR4 2007.}{fig:precip}

\sk
L'obtention de données reste de la première importance, à la fois pour vérifier que le comportement des modèles est proche du comportement observé, et pour étudier les impacts du réchauffement climatique récent. Un projet particulièrement important a été récemment mené pour mesurer les variations du niveau des océans, dont on peut prédire l'augmentation à cause de la hausse globale de température par dilatation et par fonte des glaciers continentaux (figure~\ref{fig:sealevel}). 

\sk
D'autres impacts que ceux précités sont également à l'étude
\begin{citemize}
\item Impacts sur les courants marins et donc sur les climats régionaux
\item Acidification des océans
\item Impacts sur les écosystèmes (affaiblissements, disparitions, déplacements)
\item Impacts directs sur la santé humaine (déplacement des zones endémiques pour les maladies, \ldots)
\item Risques géopolitiques (lié notamment au stress hydrique) 
\item Impacts économiques
\end{citemize}
%La liste peut encore s'allonger dans la mesure où l'action de l'homme provoque sur le climat des changements inédits, dont les conséquences sont tout autant imprévisibles. 
Les démarches que l'homme doit entreprendre pour comprendre et limiter son empreinte climatique est probablement un des enjeux majeurs du siècle qui débute.

\figun{0.7}{0.3}{decouverte/cours_meteo/cazenave_level.png}{Courbe d’évolution du niveau moyen global de la mer mesuré par altimétrie par 3 groupes différents. Droits de copyright de la figure~:~CLS, CNES, LEGOS. Source~:~A. Cazenave et M. Ablain \emph{in} Le Climat à Découvert, CNRS éditions, 2011}{fig:sealevel}

%%Extrait de la traduction du rapport IPCC AR4
%\begin{finger}
%\item A1. Le canevas et la famille de scénarios A1 décrivent un monde futur dans lequel la croissance économique sera très rapide, la population mondiale atteindra un maximum au milieu du siècle pour décliner ensuite et de nouvelles technologies plus efficaces seront introduites rapidement. Les principaux thèmes sous-jacents sont la convergence entre régions, le renforcement des capacités et des interactions culturelles et sociales accrues, avec une réduction substantielle des différences régionales dans le revenu par habitant. La famille de scénarios A1 se scinde en trois groupes qui décrivent des directions possibles de l’évolution technologique dans le système énergétique. Les trois groupes A1 se distinguent par leur accent technologique: forte intensité de combustibles fossiles (A1FI), sources d’énergie autres que fossiles (A1T) et équilibre entre les sources (A1B) (« équilibre » signifiant que l’on ne s’appuie pas excessivement sur une source d’énergie particulière, en supposant que des taux d’amélioration similaires s’appliquent à toutes les technologies de l’approvisionnement énergétique et des utilisations finales).
%\item A2. Le canevas et la famille de scénarios A2 décrivent un monde très hétérogène. Le thème sous-jacent est l’autosuffisance et la préservation des identités locales. Les schémas de fécondité entre régions convergent très lentement, avec pour résultat un accroissement continu de la population mondiale. Le développement économique a une orientation principalement régionale, et la croissance économique par habitant et l’évolution technologique sont plus fragmentées et plus lentes que dans les autres canevas.
%\item B1. Le canevas et la famille de scénarios B1 décrivent un monde convergent avec la même population mondiale culminant au milieu du siècle et déclinant ensuite, comme dans le canevas A1, mais avec des changements rapides dans les structures économiques vers une économie de services et d’information, avec des réductions dans l’intensité des matériaux et l’introduction de technologies propres et utilisant les ressources de manière efficiente. L’accent est placé sur des solutions mondiales orientées vers une viabilité économique, sociale et environnementale, y compris une meilleure équité, mais sans initiatives supplémentaires pour gérer le climat.
%\item B2. Le canevas et la famille de scénarios B2 décrivent un monde où l’accent est placé sur des solutions locales dans le sens de la viabilité économique, sociale et environnementale. La population mondiale s’accroît de manière continue mais à un rythme plus faible que dans A2, il y a des niveaux intermédiaires de développement économique et l’évolution technologique est moins rapide et plus diverse que dans les canevas et les familles de scénarios B1 et A1. Les scénarios sont également orientés vers la protection de l’environnement et l’équité sociale, mais ils sont axés sur des niveaux locaux et régionaux.
%\end{finger}


\end{document}
