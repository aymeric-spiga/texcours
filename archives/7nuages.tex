\chapter{Nuages}

\dictum[William Shakespeare, 1591]{Car tout nuage n’engendre pas tempête.}
%%Henry VI part 3 For every cloud engenders not a storm.

\bk
Lorsqu'il souhaite se représenter l'atmosphère, l'homme songe naturellement aux nuages, probablement influencé par ses observations quotidiennes et les allusions poétiques qu'elles ne manquent pas de provoquer. La physique offre un cadre pour comprendre les phénomènes nuageux et leurs implications pour le climat. Les chapitres précédents ont préparé certaines bases permettant d'aborder au premier ordre la formation des nuages.  

\mk
\section{Phénoménologie des nuages}

\sk
\subsection{Composition}
	Un \voc{nuage} se définit comme un regroupement localisé de gouttelettes d'eau et/ou de cristaux de glace ou de neige en suspension dans l'atmosphère. 

\figside{0.5}{0.2}{decouverte/cours_meteo/gouttes.png}{Vue schématique des composants d'un nuage pluvieux chaud. Les tailles indicatives sont précisées afin de donner une idée des quelques ordres de grandeur en taille qui séparent une gouttelette nuageuse d'une goutte de pluie. La traduction des termes est la suivante: \emph{cloud-condensation nuclei} \donc~noyaux de condensation, \emph{moisture droplets} \donc~gouttelettes nuageuses, \emph{typical raindrop} \donc~goutte de pluie. Les diamètres sont indiqués en microns (10$^{-6}$~m). Source: à partir de MacDonald Adv. Geophys. 1958.}{fig:cloud}

\begin{finger}
\item Dans les nuages chauds, la vapeur d'eau condense en gouttelettes nuageuses dont la taille est de quelques microns. Les conditions locales de saturation (autrement dit, l'équilibre liquide-vapeur) déterminent le taux de condensation et de croissance des gouttes. Même si cela n'est pas l'objet du présent cours, il convient de noter que l'interface courbée des gouttes a un certain coût énergétique~: il est difficile de former des gouttes à moins d'atteindre une sursaturation très élevée (c'est-à-dire une humidité très supérieure à~$1$). La formation des gouttelettes nuageuses par condensation est par contre facilitée par la présence de \voc{noyaux de condensation} (par exemple, les poussières atmosphériques). Ensuite, les gouttelettes nuageuses peuvent, par collision ou coalescence, croître de plusieurs ordres de grandeurs en taille pour donner naissance à des gouttes de pluie de plusieurs millimètres de large qui donnent lieu à des précipitations. La figure~\ref{fig:cloud} donne un aperçu très schématique de l'intérieur d'un nuage pluvieux chaud.
\item Dans les nuages froids, i.e. ceux qui se trouvent dans une zone où la température est plus faible que 0 degrés celsius, règne un équilibre à trois phases (solide, liquide, gaz). Les gouttelettes nuageuses et cristaux de glace peuvent se former par condensation directe (des phénomènes de surfusion expliquent que les gouttelettes d'eau restent à l'état liquide). La neige se forme par agrégation de cristaux de glace. Par accrétion, plus précisément coalescence liquide sur glace, la grêle peut apparaître dans un nuage froid et exceptionnellement former des espèces précipitantes d'un diamètre important.
\end{finger}

\sk
La formation des gouttes et cristaux dans un nuage obéit à un ensemble de lois dites microphysiques dont la complexité dépasse le présent cours. Pour simplifier la description, on s'intéresse souvent aux nuages chauds uniquement. De plus, on se place dans le cas d'un équilibre thermodynamique liquide-vapeur pour une interface plane, pour laquelle la valeur maximum de l'humidité est~$1$ et suffit à déclencher la formation d'un nuage dans l'atmosphère.




 
\sk
\subsection{Généralités à partir d'images satellite}

\sk
Les nuages couvrent très souvent au moins la moitié du globe et participent donc activement à l'albédo terrestre. La figure~\ref{fig:geostat} en atteste, ainsi que de la diversité des morphologies nuageuses. On peut y déceler les formes imposées par la dynamique atmosphérique, que l'on élucide en partie au chapitre suivant~:
\begin{citemize}
\item les vents qui se développent à l'échelle planétaire et convergent à proximité de l'équateur~;
\item les cyclones tropicaux dans l'Atlantique~;
\item les circulations de mousson dans la péninsule indienne~;
\item les fronts nuageux étirés dans l'hémisphère sud, avec parfois un enroulement à l'extrémité du front.
\end{citemize}

\figun{0.99}{0.31}{decouverte/cours_meteo/pentasat_IRcouleur_1800_220905.jpg}{Mosaïque IRC des géostationnaires (2 satellites américains, 2 satellites européens, 1 satellite japonais) pour le 22 septembre 2005 à 18h00 UTC. Source: base de données SATMOS \url{http://www.satmos.meteo.fr}.}{fig:geostat}

\sk
Par ailleurs, l'altitude des nuages qui se forment dans l'atmosphère peut varier significativement. L'image reproduite en figure~\ref{fig:geostat} provient de canaux infrarouge des instruments du satellite qui reçoivent le rayonnement thermique émis par les nuages (et non la diffusion du rayonnement solaire incident dans le visible). Les nuages dont le sommet est élevé, donc particulièrement froid, sont indiqués en blanc; les nuages de sommet plus chaud car moins élevés sont indiqués en gris. Les nuances entre les diverses altitudes atteintes par les nuages peuvent être notamment appréciées sur la figure pour le continent eurasien et les hautes latitudes sud.

\sk
\subsection{Classification par l'observation au sol}

\sk
Historiquement, les nuages ont été classés et nommés avant que leurs mécanismes de formation ne soient compris. C'est la raison pour laquelle les (nombreux) noms décrivant les nuages se réfèrent principalement à leur apparence et leur propension à donner lieu à des précipitations. Le classement phénoménologique, c'est-à-dire guidé par l'observation des phénomènes, le plus aisé à prendre en considération pour les nuages, fait référence à leur altitude de formation donc \emph{in fine} leur composition (Figure~\ref{fig:phenomclouds}).

\figun{0.6}{0.37}{decouverte/cours_meteo/nuages.jpg}{Classification phénoménologique des nuages par leur altitude de formation. Source: Météo-France.}{fig:phenomclouds}

\begin{description}

\item{\textbf{Nuages supérieurs}} Les \voc{cirrus} sont des nuages blancs organisés en filaments ou bandes étroites. Ils se forment dans la haute troposphère ($6-12$~km d'altitude) et sont composés de cristaux de glace dispersés. Lorsqu'ils s'organisent en bancs d'élements un peu plus épais et d'aspect ondulé, on parle de cirrocumulus. Lorsqu'ils s'organisent en un voile blanchâtre transparent qui couvre au moins partiellement le ciel, on parle de cirrostratus. Ces nuages ne sont pas accompagnés de précipitations.

\item{\textbf{Nuages moyens}} Les altocumulus, altostratus et nimbostratus se forment dans la troposphère moyenne entre~$2$ et~$5$~km d'altitude. Tous ces nuages moyens sont constitués de gouttelettes d'eau (parfois surfondues) et de cristaux de glace ou de neige. Les altocumulus prennent l'apparence de bancs d'éléments blancs ou gris de petite largeur. Les altostratus forment une couche grisâtre uniforme peu épaisse couvrant au moins partiellement le ciel. Le nimbostratus est l'archétype du nuage de mauvais temps~: il consiste en une couche nuageuse grise et sombre, parfois rendue floue par les précipitations de pluie ou de neige et souvent assez épaisse pour masquer le soleil. Les altostratus et nimbostratus peuvent conduire à des précipitations de pluie ou de neige, plutôt intenses dans le deuxième cas.

\item{\textbf{Nuages inférieurs}} Les nuages inférieurs se forment à une altitude au-dessus de la surface inférieure à~$2$~km. Ils sont composés majoritairement de gouttelettes d'eau. Le \voc{stratus} est une couche nuageuse grise, dense et uniforme, qui peut donner lieu à de la bruine ou du brouillard. Les \voc{cumulus} sont des nuages bien délimités, blancs, assez denses et présentant parfois un aspect bourgeonnant qui correspond à une certaine extension verticale (on parle dans ce cas de cumulus congestus, qui peuvent donner lieu à des cumulonimbus, voir ci-dessous). S'ils prennent la forme de petits éléments blancs ou gris organisés en bancs, on parle de stratocumulus. Les cumulus ne donnent pas toujours lieu à des précipitations et, le cas échéant, il s'agit surtout d'averses.

\end{description}

Il existe un type de nuage très important qui échappe à cette classification~: le \voc{cumulonimbus}. Il est aussi qualifié de nuage d'orage. Ce nuage montre un développement vertical considérable (on parle souvent de \og tour convective \fg), qui part de la basse troposphère pour se prolonger jusque dans la haute troposphère. La partie supérieure a souvent une apparence étalée par rapport au reste du nuage (\og enclume \fg). La partie inférieure est très sombre en raison de l'atténuation du rayonnement solaire incident liée à l'extension verticale importante du nuage. Les précipitations associées aux cumulonimbus sont de fortes averses de pluie, de neige ou de grêle.

\mk
\section{Quelques éléments de physique des nuages}

\sk
\subsection{Classification physique des nuages} \label{classphys}
	\sk
La cause générale de la formation d'un nuage est le refroidissement d'une masse d'air. Dans la grande majorité des cas, les transformations qui provoquent ce refroidissement sont soit isobares, soit adiabatiques. Ceci est illustré par la formulation du premier principe adoptée dans les chapitre précédents.
%\[ \underbrace{\textcolor{white}{\frac{R^2}{C_P}} \dd T \textcolor{white}{\frac{R}{C_P}}}_{\text{variation de température de la parcelle}} = \underbrace{\frac{R}{C_P} \, \frac{T}{P} \, \dd P}_{\text{travail expansion/compression}} + \underbrace{\frac{1}{C_P} \, \delta q}_{\text{chauffage diabatique}} \]

\sk
Suivant la transformation qui va donner naissance au nuage, la morphologie de ce dernier est très différente. Par opposition à la classification phénoménologique donnée en début de chapitre, on peut alors établir une classification physique des nuages en fonction de la transformation thermodynamique qui leur donne naissance.
\begin{finger}
\item Si la transformation est isobare, cela signifie que le nuage se forme sans que la pression ne varie significativement dans la parcelle d'air considérée. D'après l'équilibre hydrostatique, et le fait que les variations de pression selon l'horizontale sont négligeables par rapport aux variations de pression selon la verticale, un tel phénomène ne peut exister que si l'altitude de la parcelle varie peu. La parcelle subit un refroidissement diabatique, qui peut être provoqué par exemple par les pertes radiatives (comme le brouillard nocturne cité au chapitre précédent) ou par un déplacement horizontal par les vents vers une région plus froide de l'atmosphère. Les nuages obtenus ne présentent pas de développement selon la verticale et sont même souvent particulièrement étendus selon l'horizontale. Il s'agit des nuages de type cirrus et stratus. Ces nuages étant organisés en strates, puisque non étendus selon la verticale, on dit qu'ils sont \voc{stratiformes}.
\item Si la transformation est adiabatique, le nuage se forme par le biais de parcelles d'air qui n'échangent pas de chaleur avec l'air environnant (ou, du moins, pour lesquelles les échanges diabatiques sont négligeables par rapport au terme de travail d'expansion/compression). Cette condition n'est réalisée que pour des parcelles subissant une variation de pression significative donc, selon l'équilibre hydrostatique, une variation d'altitude importante. Les nuages obtenus ne présentent donc pas d'étendue horizontale et sont au contraire très développés selon la verticale. Il s'agit des nuages de type cumulus et cumulonimbus. On les qualifie de nuages \voc{cumuliformes}.
\end{finger}
%%Ces nuages sont gouvernés par la convection~: la parcelle d'air plus chaude que son environnement a plus de facilité à refroidir en s'élevant qu'en échangeant de la chaleur avec son environnement.

\sk
La plupart des nuages se forment ainsi par refroidissement (isobare ou adiabatique) d'une masse d'air. Il est cependant à noter que le brassage d'une masse d'air chaude d'humidité voisine de~$100\%$ avec une masse d'air froide relativement sèche peut également donner naissance à des nuages de type stratiforme. %% avions. %% brise de terre.

%Quelques remarques sur les processus de formation: la plupart des nuages stratiforme se forment aussi au départ par refroidissement adiabatique -- même si le chauffage / refroidissement radiatif joue ensuite un rôle important dans le cycle de vie, en particulier dans les transitions type stratus -> strato-cumulus.
%Pour les nuages de couche limite, c'est le refroidissement lors du transport vertical turbulent ; pour les nuages type cirrus / altostratus / nimbi-stratus c'est le soulèvement en bloc de couches, dans les zones frontales par exemple. En fait, il n'y a guère que le brouillard radiatif ou d'advection qui soit vraiment d'origine adiabatique...
%Sinon, il y a un type un peu particulier qu'on appelle nuage de mélange, qu'on voit par exemple dans l'haleine qui condense en hiver. C'est ici le mélange de 2 masses d'air humides (mais non saturées) qui donne de l'air à une température intermédiaire mais saturé cette fois à cause de la forme de la courbe rsat(T).


\sk
\subsection{Développement d'un nuage cumuliforme}
	\figun{0.99}{0.6}{decouverte/cours_meteo/sounding.pdf}{Radiosondage obtenu à la station de Trappes le 23 juin 2005 à 12:00. Les données sont projetées sur un émagramme avec la pression atmosphérique en ordonnée et la température atmosphérique en abscisse. Des cumulonimbus d'orages très violents se sont développés dans la région parisienne ce jour~: cet événement peut être interprété à l'aide de l'émagramme qui permet de déterminer la base du nuage, le niveau de convection libre et le sommet théorique du nuage.}{fig:sounding}

	\sk
On s'intéresse ici au développement des nuages cumuliformes, en particulier les cumulonimbus. Le cas d'étude donné dans le radiosonsage exemple permet de suivre graphiquement les concepts de cette partie. Le point de départ est une parcelle d'air non saturée, c'est-à-dire dont l'humidité est inférieure à~$1$, située proche de la surface. On suppose que son rapport de mélange en vapeur d'eau~$r$ est conservé au cours de l'ascension. 

\sk
En premier lieu, de nombreux phénomènes atmosphériques vont provoquer une élévation de la parcelle que l'on considère initialement proche de la surface.
\begin{citemize}
\item{\textbf{Soulèvement frontal}} Un front est une variation marquée et localisée de température. Lorsqu'un front se déplace horizontalement, l'air chaud passe au-dessus de l'air froid de densité moindre. 
\item{\textbf{Soulèvement orographique}} La présence d'un relief face au vent force les parcelles d'air à s'élever.
\item{\textbf{Convection sèche}} Un sol très chaud l'après-midi peut induire un profil de température de l'environnement très instable proche de la surface. Dans ce cas, les mouvements verticaux sont amplifiés proche de la surface par la poussée d'Archimède (voir chapitre précédent).
%%% circulation thermique. brise de terre et brise de mer.
\end{citemize}
Tous ces mécanismes expliquent que des nuages cumuliformes sont souvent trouvés au-dessus de régions soumises au passage de fronts, montagneuses ou dont la surface est particulièrement chaude. Ces nuages évoluent parfois vers un état de type cumulonimbus.

\sk
En second lieu, lorsqu'une parcelle d'air est soulevée vers les plus hauts niveaux de l'atmosphère par les phénomènes atmosphériques précités, elle subit un refroidissement par détente adiabatique. Le taux de refroidissement de la parcelle est~$|\Gamma\e{sec}|$. Sur un émagramme tel celui de la figure~\ref{fig:sounding}, la parcelle suit une \voc{courbe adiabatique sèche}. Cette décroissance de la température de la parcelle au cours de l'ascension a pour principale conséquence d'abaisser la valeur de $r\e{sat}$, de par les variations exponentielles de cette quantité avec la température. Il en résulte que l'humidité relative~$H = r / r\e{sat}$ augmente. Si la quantité de vapeur d'eau initiale~$r$ et/ou le soulèvement de la parcelle sont suffisants, $H$ peut atteindre~$1$ au cours de l'ascension~: la parcelle devient alors saturée. Des gouttelettes nuageuses apparaissent par condensation, autrement dit un nuage se forme. Le niveau d'altitude ou de pression auquel la condensation se produit suite à un refroidissement par soulèvement adiabatique s'appelle le \voc{niveau de condensation} ou la \voc{base du nuage}. A ce stade, le nuage n'est pas encore nécessairement cumuliforme.
%%Pour la convection. Les bases des nuages sont horizontales, leurs sommets évoluent en fonction de la température.

\sk
En troisième lieu, si la parcelle continue son ascension au-delà du niveau de condensation, sa température ne décroît plus d'un taux~$|\Gamma\e{sec}|$, mais d'un taux $|\Gamma\e{saturé}|$ plus faible, puisque la parcelle est désormais saturée (son humidité vaut~$1$ et son rapport de mélange~$r$ vaut~$r\e{sat}$). Sur l'émagramme, la parcelle suit une \voc{courbe adiabatique saturée}. Au cours de l'ascension, la parcelle reste saturée mais, puisque sa température continue de diminuer, $r\e{sat}$ diminue de concours, ce qui induit une diminution du rapport de mélange en vapeur d'eau~$r$ et une augmentation du rapport de mélange en eau liquide (qui prend la forme de gouttelettes nuageuses ou, si les conditions de croissance sont réunies, de précipitations pluvieuses).

\sk
En quatrième lieu, la forme du profil de température d'environnement détermine si, une fois le niveau de condensation atteint, le nuage va suivre ou non un développement vertical marqué. On rappelle que le profil d'environnement n'est pas celui suivi par la parcelle considérée, mais représente l'état atmosphérique tel qu'il peut être mesuré par un ballon-sonde météorologique par exemple.

\begin{finger}

\item Si les soulèvements initiaux de la parcelle ne l'amènent que dans des niveaux atmosphériques où sa température reste plus faible que celle de l'environnement, alors il n'y a pas de mouvements verticaux spontanés au sein du nuage. Le nuage formé est plutôt de type stratiforme (ou faiblement cumuliforme).

\item Si les soulèvements initiaux de la parcelle parviennent à la hisser à des niveaux atmosphériques où sa température devient plus élevée que celle de l'environnement, alors des mouvements verticaux spontanés apparaissent au sein du nuage. On parle de convection humide (ou convection profonde). Le nuage ainsi formé est cumuliforme. Le niveau atmosphérique à partir duquel la température de la parcelle en ascension adiabatique devient plus élevée que la température de l'environnement s'appelle le \voc{niveau de convection libre}. Le niveau atmosphérique à partir duquel la température de la parcelle redevient plus faible que l'environnement s'appelle le \voc{sommet théorique du nuage}. Si le sommet théorique du nuage est très élevé, la formation de cumulonimbus, donc d'orage, est très probable. 

\end{finger}



