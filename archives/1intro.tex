\chapter{Un survol de l'atmosphère} \label{chap:int}

\dictum[Pierre-Simon de Laplace, 1797]{Un fluide rare, transparent, compressible et élastique, qui environne un corps, en appuyant sur lui, est ce que l'on nomme son atmosphère.}

\bk
La figure~\ref{fig:blue} illustre la présence d'une atmosphère très active sur Terre par les nuages qui y prennent naissance. Il ne s'agit que d'un exemple parmi tant d'autres pour appréhender l'atmosphère. C'est le but de ce chapitre d'introduction d'évoquer la diversité des points de vue pouvant être adoptés pour étudier l'atmosphère, un système complexe où se mêlent processus physiques, dynamiques, chimiques, biologiques, et même sociétaux. Sont également abordées dans ce chapitre quelques notions de base nécessaires pour la suite du cours.

\figsup{0.45}{0.25}{decouverte/cours_meteo/blue_50.png}{decouverte/cours_meteo/blueclouds_50.png}{La planète Terre avec et sans les nuages de son atmosphère. Les nuages couvrent très souvent au moins la moitié du globe. Construit d'après une image \ofg{Blue Marble} NASA du projet \ofg{Visible Earth}. Des versions haute-résolution des images planes et des explications complètes peuvent être trouvées aux adresses suivantes \url{http://visibleearth.nasa.gov/view_rec.php?id=2430} et \url{http://visibleearth.nasa.gov/view_rec.php?id=2431}.}{fig:blue} %%%http://visibleearth.nasa.gov/view_rec.php?id=2429

\mk
\section{Quelques définitions et généralités}

	\sk \subsection{Vocabulaire}
	\sk
L'objectif des sciences de l'atmosphère est d'étudier la structure et l'évolution de l'atmosphère en caractérisant et en expliquant les phénomènes qui s'y déroulent. Les sciences de l'atmosphère font principalement appel à des notions de physique, chimie, et mécanique des fluides.
\begin{description}
\item[\voc{Atmosphère}] Ensemble de couches, principalement gazeuses, qui entourent la masse condensée, solide ou liquide, d'une planète (voir également citation de Laplace en en-tête).
\item[\voc{Air}] Mélange gazeux constituant l'atmosphère terrestre.
\item[\voc{Aéronomie}] Science dont l'objet est la connaissance de l'état physique de l'atmosphère terrestre et des lois qui la gouvernent.
\item[\voc{Météorologie}] Discipline ayant pour objet l'étude des phénomènes atmosphériques et de leurs variations, et qui a pour objectif de prévoir à court terme les variations du temps.
\item[\voc{Climat}] Ensemble des conditions atmosphériques et météorologiques d'un pays, d'une région. Le climat peut également être défini comme un système thermo-hydrodynamique non isolé dont les composantes sont les principales « enveloppes » externes de la Terre : on parle également de~\voc{système climatique} [figure~\ref{fig:pluri}]. 
\begin{citemize} \item L'atmosphère : l’air, les nuages, les aérosols, \ldots 
\item L’hydrosphère : les océans, les rivières, les précipitations, \ldots 
\item La lithosphère : les terres immergées, les sols, \ldots 
\item La cryosphère : glace, neige, banquise, glaciers, \ldots 
\item La biosphère : les organismes vivants, \ldots
\item L’anthroposphère : l’activité humaine, \ldots 
\end{citemize} 
\end{description}

\figside{0.75}{0.35}{decouverte/cours_meteo/joussaume_pluri.png}{Schéma du système climatique présentant les différentes composantes du système : atmosphère, océans, cryosphère, biosphère et lithosphère, leurs constantes de temps et leurs interactions en termes d’échanges d’énergie, d’eau et de carbone. Source~:~S.~Joussaume \emph{in} Le Climat à Découvert, CNRS éditions, 2011}{fig:pluri}



\sk
\subsection{Grandeurs utiles}

	\sk
L'atmosphère est composée d'un ensemble de molécules. Pour la description de la plupart des phénomènes étudiés, le suivi des comportements individuels de chacunes des molécules composant l'atmosphère est impossible. On s'intéresse donc aux effets de comportement d'ensemble, ou moyen. Les principales variables thermodynamiques utilisées pour décrire l'atmosphère sont donc des grandeurs \voc{intensives} dont la valeur ne dépend pas du volume d'air considéré.
\begin{finger}
\item La \voc{température} $T$ est exprimée en K (kelvin) dans le système international. Elle est un paramètre macroscopique qui représente l'agitation thermique des molécules microscopiques. Les mesures de température usuelles font parfois référence à des quantités en \deg C, auxquelles il faut ajouter la valeur $273.15$ pour convertir en kelvins.
\item La \voc{pression} $P$ est exprimée en Pa dans le système international. La pression fait référence à une force par unité de surface ($1$~Pa correspond à l'unité~N~m$^{-2}$). Paramètre macroscopique, elle est reliée à la quantité de mouvement des molécules microscopiques qui subissent des chocs sur une surface donnée. Les mesures et raisonnements météorologiques font souvent référence à des quantités en hPa ou en mbar. Ces deux unités sont équivalentes : 1~hPa correspond à~$10^{2}$~Pa, 1~mbar correspond à $10^{-3}$~bar, ce qui correspond bien à 1~hPa, puisque 1 bar est $10^{5}$~Pa. La pression atmosphérique vaut $1013.25$~hPa (ou mbar) en moyenne au niveau de la mer sur Terre. On utilise parfois l'unité d'$1$~atm (atmosphère) qui correspond à cette valeur de~$101325$~Pa.
\item La \voc{masse volumique} ou densité~$\rho$ est exprimée en~kg~m$^{-3}$ dans le système international. Elle représente une quantité de matière par unité de volume. Elle vaut environ $1.217$~kg~m$^{-3}$ proche de la surface sur Terre.
\end{finger}


\sk
Les trois paramètres précités varient en théorie selon les trois directions de l'espace. On constate cependant que, pour une unité de longueur donnée, leurs variations selon la verticale sont beaucoup plus significatives que leurs variations selon l'horizontale. On peut donc définir une structure moyenne en fonction de l'altitude dont sera toujours relativement proche la structure verticale en chaque jour et chaque région de la planète. 

\figsup{0.48}{0.35}{\figfrancis/WH_vert_struct}{\figpayan/LP211_Chap1_Page_03_Image_0001.png}{[Gauche] Structure verticale de la pression, la densité et du libre parcours moyen des molécules (distance moyenne parcourue par une molécule avant de subir un choc sur une autre molécule). Noter l'échelle logarithmique en abscisse : une droite sur ce schéma dénote donc une variation exponentielle des quantités avec l'altitude. [Droite] Plus haut dans l'atmosphère, la variation verticale de la pression est dépendante pour plusieurs ordres de grandeur avec l'activité solaire. Les courbes indiquées correspondent respectivement à une très faible activité solaire (température de la thermopause de 600 K) et une très forte activité solaire (température de la thermopause de 2000K).}{fig:presvert}

\sk
Pression et densité décroissent de façon approximativement exponentielle selon l'altitude~$z$ [figure \ref{fig:presvert}] $$ P\sim P_0 \, e^{-z/H} $$ où $H$ est appelée \voc{hauteur d'échelle} et~$P_0$ une valeur de pression de référence. Cette loi de variation découle du fait que la pression atmosphérique à une altitude~$z$ est due au poids de la colonne d'air située au-dessus de l'altitude~$z$. En pratique sur Terre, la pression est divisée par deux environ tous les 5 km. On évalue la masse de l'atmosphère terrestre à~$5 \times 10^{18}$~kg, soit environ un millionième de la masse de la Terre. La moitié de la masse de l'atmosphère se situe au dessous de~$5500$~m, les~$2/3$ au dessous de~$8400$~m, les~$3/4$ au dessous de~$10300$~m, les~9/10~au dessous de~$16100$~m. Si l'on considère que les neuf dixièmes de l’atmosphère sont situés dans les $16$~premiers kilomètres, l’atmosphère ne forme donc qu'une mince pellicule gazeuse en comparaison des $6367$~km du rayon terrestre. On dit que l'on peut faire l'\voc{approximation de la couche mince}.

\sk
\subsection{Structure verticale : couches atmosphériques}

\sk
Les variations verticales de température sont très différentes des variations de pression et de densité: la température décroît et augmente alternativement avec l'altitude, de façon quasi-linéaire [figure \ref{fig:tempvert}], en restant comprise entre environ~$200$ et~$300$~K. Cette structure verticale de la température permet de diviser l'atmosphère en un certain nombre de couches aux propriétés différentes, dont les noms comportent le suffixe \emph{-sphère}. La limite supérieure d'une couche atmosphérique donnée porte un nom similaire, où le suffixe \emph{-sphère} est remplacé par le suffixe \emph{-pause}. Par exemple, la limite entre la troposphère et la stratosphère s'appelle la \voc{tropopause}. Les couches atmosphériques en partant de la surface vers l'espace sont détaillées ci-dessous.

\sk
\begin{description} 
\item[La \voc{troposphère}] s'étend jusqu'à environ 11 km d'altitude et contient 80\% de la masse de l'atmosphère. La température y décroit en moyenne de 6.5\deg C par kilomètre (nous verrons pourquoi dans un chapitre ultérieur). La troposphère est une couche relativement bien mélangée sur la verticale (échelle de temps de quelques jours), sauf en certaines couches minces, appelées \voc{inversions}, où la température décroit peu ou même augmente avec l'altitude. La troposphère est la couche où ont lieu la plupart des phénomènes météorologiques acessibles à l'expérience humaine (par exemple, les nuages montrés en figure~\ref{fig:blue}). La partie inférieure de la troposphère contient la \voc{couche limite atmosphérique} située juste au dessus de la surface, d'épaisseur variant de quelques centaines de~m à 3 km et définie comme la partie de l'atmosphère influencée par la surface sur de courtes échelles de temps (typiquement un cycle diurne). La couche limite atmosphérique est le siège de mouvements turbulents organisés au cours de l'après-midi qui opèrent un mélange des espèces chimiques qui y sont émises.
\item[La \voc{stratosphère}] est située au dessus de la troposphère. L'altitude au-dessus du sol de la tropopause peut varier entre~$5$ et~$15$~km. Contrairement à la troposphère, la stratosphère contient très peu de vapeur d'eau (à cause des températures très basses rencontrées à la tropopause) mais la majorité de l'ozone~O$_3$. L'absorption par l'ozone du rayonnement solaire \voc{ultraviolet}, de longueur d'onde moindre que le rayonnement visible et plus énergétique, explique que la température dans la stratosphère est d'abord isotherme, puis augmente avec l'altitude jusqu'à un maximum à la stratopause. Cette structure verticale très stable inhibe fortement les mouvements verticaux, ce qui explique que la stratosphère soit organisée en couches horizontales (comme l'indique l'étymologie de son nom). Le temps de résidence de particules dans la stratosphère est très long à cause de l'absence de nuages et précipitations.
\item[La \voc{mésosphère}] voit sa température décroître selon la verticale. Contrairement à la troposphère, elle ne contient pas de vapeur d'eau et contrairement à la stratosphère, elle ne contient que peu d'ozone. Elle se situe sur Terre à des altitudes entre~$50$ et~$85$~km. La mésopause est souvent le point le plus froid de l'atmosphère terrestre, la température peut y atteindre des valeurs aussi basses que~$130$~K. 
\item[La \voc{thermosphère}] s'étend jusque des altitudes très élevées (800 km) et voit sa température contrôlée par l'absorption du rayonnement solaire ultraviolet. La température dans la thermosphère varie souvent d'un facteur deux suivant l'activité solaire et l'alternance jour-nuit. Les aurores surviennent dans cette couche atmosphérique. Les missions spatiales \ofg{basse orbite} telles que la Station Spatiale Internationale sont localisées au milieu de la thermosphère.
\item[L'\voc{exosphère}] est située au-dessus de la thermosphère à partir d'une altitude d'environ~$800$~km sur Terre. Il s'agit de la zone où l'atmosphère subit un \voc{échappement} : les molécules peuvent s'échapper vers l'espace sans que des chocs avec d'autres molécules ne les renvoient dans l'atmosphère. L'exosphère constitue la dernière zone de transition entre l'atmosphère et l'espace.
\end{description}

\figsup{0.49}{0.3}{\figfrancis/WH_stdatm}{\figpayan/LP211_Chap1_Page_05_Image_0001.png}{Structure verticale idéalisée de la température correspondant au profil moyenné global annuel. Voir figure~\ref{fig:presvert} pour la distinction entre figure de gauche et figure de droite}{fig:tempvert}

\sk
\footnotesize
D'autres couches atmosphériques sont définies non pas à partir de la température mais à partir des propriétés électriques de l'atmosphère terrestre. On fait référence ici au vent solaire, qui est un flux de particules chargées (ions et électrons) formant un plasma qui s’échappe en permanence du Soleil vers l’espace interplanétaire
\begin{description}
\item[L'ionosphère] Comme son étymologie l'indique, l’ionosphère est une région de notre haute atmosphère contenant des ions et des électrons formés par photo-ionisation des molécules neutres qui s’y trouvent. C’est le Soleil, et plus particulièrement ses rayonnements énergétiques ultraviolets et X, mais aussi les particules du vent solaire et le rayonnement cosmique, qui sont à l’origine de cette ionisation de la haute atmosphère. L’ionosphère se situe entre~$50$ et~$1000$ kilomètres d’altitude (elle s'étend donc de la mésosphère à la thermosphère). L’ionosphère est habituellement divisée horizontalement en différentes couches, baptisées D, E et F dans lesquelles l’ionisation croît avec l’altitude. Ces couches proviennent des différences de pénétration dans l’atmosphère des rayonnements solaires selon leur énergie.
\item[La magnétosphère] Notre planète génère son propre champ magnétique, un peu à la manière d’une dynamo. C’est la différence de vitesse entre la rotation de la planète et de son coeur liquide qui, par induction, génère ce champ magnétique. Ce champ magnétique protège la Terre des agressions extérieures comme les rayons cosmiques et les particules énergétiques du vent solaire. Cette zone protégée s'appelle magnétosphère. Elle démarre au dessus de l’ionosphère, à plusieurs milliers de kilomètres de la surface du sol, et s’étend jusqu’à 70 000 kilomètres environ du côté du Soleil. Du côté opposé, la queue de la magnétosphère s’étire sur plusieurs millions de kilomètres. Les contours de la magnétosphère évoluent continuellement sous l’action du vent solaire et de sa variabilité.
\end{description}
\normalsize

\mk \section{Composition atmosphérique}
	
	\sk \subsection{Un mélange de gaz parfaits}
	\sk
On appelle \voc{gaz parfait} un gaz suffisament dilué pour que les interactions entre les molécules du gaz, autres que les chocs, soient négligeables. L'air composant l'atmosphère peut être considéré en bonne approximation comme un mélange de gaz parfaits\footnote{On peut en général considérer que le gaz est parfait si~$P < 1$~kbar. C'est le cas dans la plupart des atmosphères planétaires rencontrées. Il n'y a guère qu'au coeur des planètes géantes, où la pression dépasse cette limite, que l'approximation du gaz parfait doit être abandonnée.} notés~$i$, dont le nombre de moles est~$n_i$ pour un volume donné~$V$ d'air à la température~$T$. Chaque espèce gazeuse composant l'air est caractérisée par une \voc{pression partielle}~$P_i$ qui est définie comme la pression qu'aurait l'espèce gazeuse si elle occupait à elle seule le volume~$V$ à la température~$T$. Chacune de ces espèces gazeuses~$i$ se caractérise par une même température~$T$ et vérifie l'équation d'état du gaz parfait $$ P_i \, V = n_i \, R^* \, T $$ où~$R^*$=8.31 J~K\mo~mol\mo~est la constante des gaz parfaits (produit du nombre d'Avogadro et de la constante de Boltzmann). La pression totale de l'air~$P$ est, d'après la loi de Dalton, la somme des pressions partielles~$P_i$ des espèces gazeuses composant le mélange $P=\Sigma P_i$. En faisant la somme des lois du gaz parfait appliquées pour chacune des espèces gazeuses, on obtient $$ P \, V = \big( \Sigma n_i \big) \, R^* \, T $$ ce qui montre qu'un mélange de gaz parfaits est aussi un gaz parfait. Cette équation permet de relier la pression totale~$P$ à la température~$T$, mais présente l'inconvénient de contenir les grandeurs \voc{extensives}~$V$ et $n_i$ qui dépendent du volume d'air considéré. Il reste donc à donner une traduction intensive à la loi du gaz parfait pour un mélange de gaz. La masse totale contenue dans le volume~$V$ peut s'écrire $m=\Sigma n_iM_i$ où $M_i$ est la masse molaire du gaz $i$. En divisant l'équation précédente par $m$, et en utilisant la définition de la masse volumique~$\rho = m / V$, on obtient $$ \frac{P}{\rho} = \frac{\Sigma n_i}{\Sigma n_iM_i} \, R^* \, T $$ Or, d'une part, la \voc{masse molaire de l'air} composé d'un mélange de gaz~$i$ est $$ \boxed{ M=\frac{\Sigma n_iM_i}{\Sigma n_i} }$$ et d'autre part, on peut définir une \voc{constante de l'air sec} de la façon suivante $$ R=\frac{R^*}{M} $$ On a alors l'équation des gaz parfaits pour l'air atmosphérique qui permet de relier les trois paramètres intensifs importants : pression~$P$, température~$T$ et densité~$\rho$ $$ \boxed{ P = \rho \, R \,T } $$ L'état thermodynamique d'un élément d'air est donc déterminé uniquement par deux paramètres sur les trois~$P$, $T$, $\rho$. En météorologie par exemple, on travaille principalement avec la pression et la température qui sont plus aisées à mesurer que la densité. Les valeurs numériques de~$M$, et donc~$R$, dépendent de la planète considérée et de sa composition atmosphérique. 



\sk
\subsection{Composition moléculaire}

\sk
La concentration, au sens où elle est définie dans la plupart des cours de physique / chimie, est une quantité très peu utilisée en sciences de l'atmosphère. La composition chimique de l'atmosphère s'exprime préférentiellement en utilisant le \voc{rapport de mélange volumique}, soit la proportion d'un volume d'air occupée par un gaz particulier. L'air étant un gaz parfait, ce rapport de mélange volumique est simplement égal au rapport du nombre de molécules/atomes du gaz sur le nombre total de molécules d'air $$ \boxed{ r_i=\frac{n_i}{\Sigma n_k} } \qquad \textrm{\footnotesize (parfois également noté $q_i$)} $$ D'après la loi de Dalton, il correspond également au rapport entre la pression partielle du gaz considéré avec la pression totale du mélange. Le rapport de mélange n'est exprimé en pourcentage que pour les composés les plus abondants. Pour les \voc{gaz traces}, moins abondants, on exprime le rapport de mélange en parties par million (1 ppmv = $10^{-6}$), ou par milliards (1 ppbv = $10^{-9}$), voire le pptv tel que~1 pptv = $10^{-12}$. Dire que le rapport de mélange de CO$_2$ au sol est d’environ 380 ppmv signifie que sur un million de molécules d’air, 380 sont des molécules de CO$_2$. %On utilise également parfois le rapport de mélange massique, soit la proportion d'une masse d'air occupée par un gaz particulier. 

\begin{table}
\centering
\begin{tabular}{lcc}
%\toprule
%\makecell[l]{\textbf{Constituant}} &
%\makecell{\textbf{Masse}\\\textbf{Molaire}} &
%\makecell{\textbf{Rapport}\\\textbf{de Mélange}} \\
%\midrule
\hline
\textbf{Constituant} &
\textbf{Masse molaire} &
\textbf{Rapport de mélange} \\
\hline
Azote (N$_2$) & 28 & 78\% \\
Oxygène (O$_2$) & 32 & 21\% \\
Argon (Ar) & 40 & 0.93\% \\
\textbf{Vapeur d'eau (H$_2$O)} & 18 & 0-5\% \\
\textbf{Dioxyde de Carbone (CO$_2$)} & 44 & 380 ppmv \\
Néon (Ne) & 20 & 18 ppmv \\
Hélium (He) & 4 & 5 ppmv \\
\textbf{Méthane (CH$_4$)} & 16 & 1.75 ppmv \\
Krypton (Kr) & 84 & 1 ppmv \\
Hydrogène (H$_2$) & 2 & 0.5 ppmv \\
\textbf{Oxide nitreux (N$_2$O)} & 56 & 0.3 ppmv \\
\textbf{Ozone (O$_3$)} & 48 & 0-0.1 ppmv \\
%\bottomrule
\hline
\end{tabular}
\caption{\emph{Principaux composants de l'atmosphère. Les gaz à effet de serre sont indiqués en gras.}}
\label{tab:compos}
\end{table}


\sk
L'azote et l'oxygène dominent largement la composition de l'atmosphère terrestre (tableau \ref{tab:compos}), suivis par l'argon et d'autres gaz rares beaucoup moins abondants. Les rapports de mélange de vapeur d'eau et d'ozone sont très variables : la vapeur d'eau est présente surtout dans la troposphère, avec un maximum près de la surface et dans les tropiques, alors que l'ozone est principalement présente dans la stratosphère. Un certain nombre de gaz traces sont émis régulièrement au niveau de la surface, par des phénomènes naturels ou les activités humaines. Leur répartition dépend alors beaucoup de leur \voc{durée de vie} dans l'atmosphère. Le CO$_2$ qui est très stable est bien mélangé. Le méthane, qui a une durée de vie d'une dizaine d'années, est bien réparti dans la troposphère mais son rapport de mélange varie dans la stratosphère. Des polluants à durée de vie courte (quelques jours) comme l'ozone troposphérique, se retrouveront surtout au voisinage des sources. Les activités humaines ont également contribué à modifier le rapport de mélange de certains de ces gaz (par exemple, le~CO$_2$).

\sk
\begin{finger}
\item En employant les formules obtenues à la section précédente, et les informations ci-dessus sur la composition de l'atmosphère terrestre, il est possible de déterminer des valeurs numériques utiles pour la suite du cours
\begin{citemize}
\item La masse molaire de l'air est~$M = 28.966$~g~mol$^{-1}$ (on emploie souvent~$M \simeq 29$~g~mol$^{-1}$). 
\item La constante de l'air sec est~$R = 287$~J~K$^{-1}$~kg$^{-1}$.
\end{citemize}  
\item Dans toute discussion de la composition atmosphérique, il est important de faire la distinction entre composés minoritaires et majoritaires [figure~\ref{fig:minor}]. Alors que les composés majoritaires suivent une distribution verticale en accord avec l'état énergétique et dynamique de l'atmosphère globale, les composés minoritaires peuvent avoir des comportements très différents qui dépendent à la fois des mécanismes photochimiques de production et de perte ainsi que des phénomènes de transport.
\item La composition de l'air donnée ici est valide sur les premiers~$80$ à~$100$ kilomètres d'altitude, à part quelques constituants mineurs. On appelle cette région l'\voc{homosphère}, elle correspond approximativement à la troposphère, la stratosphère et la mésosphère (Figure \ref{fig:tempvert}). Dans l'homosphère, l'atmosphère est un mélange homogène de différents gaz. Au dessus de cette altitude, le libre parcours moyen des molécules devient très grand et on a une \ofg{décantation} où les éléments plus légers dominent progressivement aux altitudes élevées. On parle alors d'\voc{hétérosphère}; elle regroupe approximativement la thermosphère et l'exosphère. 
\end{finger}
%Au niveau du sol, l'atmosphère standard sèche est caractérisée par une pression d’environ 1013 hPa et une concentration totale de 2,69 x 1019 molécule~cm$^{-3}$ lorsque la température est de 273~K.

\figun{0.9}{0.4}{\figpayan/LP211_Chap1_Page_06_Image_0001.png}{Composition de l’atmosphère~: des espèces en très faibles quantités jouent un rôle très important. Sur la figure sont données quelques mesures de constituants minoritaires dans l’homosphère. Les courbes en trait fin correspondent aux concentrations résultant de rapports de mélange volumiques constants de~$10^{-1}$ à~$10^{-13}$. (Source: Kockarts, Aéronomie, 2000).}{fig:minor}

\figun{1}{0.35}{decouverte/cours_dyn/composition.png}{Objets du système solaire présentant une atmosphère substantielle et leurs caractéristiques.}{fig:composition}

\sk
Les compositions atmosphériques sont très distinctes selon les planètes du système solaire considérées [figure~\ref{fig:composition}]. L'atmosphère terrestre est unique parmi les atmosphères des autres planètes du système solaire. Elle est riche en azote et oxygène et pauvre en \carb, contrairement aux atmosphères de Venus et Mars, qui contiennent plus de~$90\%$ de \carb. On pourrait penser que la Terre a acquis son atmosphère pendant sa formation à partir des gaz présents dans la nébuleuse solaire. Une telle atmosphère serait alors primaire, et contiendrait des gaz de composition cosmique, c'est-à-dire similaire aux abondances chimiques du système solaire. Or, les gaz dominants dans le système solaire sont l'hydrogène et l'hélium. Ces gaz légers sont pratiquement absents dans notre atmosphère, car la gravitation terrestre est trop faible pour les retenir. Les planètes géantes comme Jupiter ou Saturne ont conservé ces gaz primordiaux dans leur atmosphère, au contraire des planètes internes du système solaire Venus, Terre et Mars qui ont des atmosphères de composition bien différente. Si la Terre a eu une telle atmosphère primaire pendant sa formation, elle l'a perdu rapidement. L'atmosphère actuelle doit donc être secondaire. L'évolution de sa composition résulte en partie de l'apparition de la vie [figure~\ref{fig:vie}].
%% parler des puits de carbone et des carbonates. comparaison entre Vénus et Mars.

\figun{0.75}{0.32}{\figpayan/vie.png}{Evolution parallèle de l’atmosphère et de la vie.}{fig:vie}

\sk
\subsection{Aérosols et hydrométéores}

%%\footnote{Cette partie est inspirée d'éléments trouvés dans le cours de S. Jacquemoud de \emph{Méthodes physique en télédétection.}}

\sk
Les \voc{aérosols} sont constitués de petites particules solides ou liquides en suspension dans les basses couches de l’atmosphère. Environ trois milliards de tonnes de particules sont injectés chaque année dans l’atmosphère par les processus naturels ou les activités humaines. On distingue plusieurs types d'aérosols.
\begin{citemize}
\item[\emph{poussières d'origine désertique}] Il s'agit de la première source mondiale d’aérosols. Elles sont soulevées par des vents violents lors des tempêtes de sable. Les grosses particules retombent rapidement au sol alors que les plus petites forment un nuage sec qui peut s’élever jusqu’à 4-6 km d’altitude et s’étendre sur des milliers de kilomètres. On peut retrouver en Europe ou en Amérique des particules en provenance d'Afrique.
\item[\emph{aérosols solubles dans l’eau}] Ils peuvent être d'origine naturelle (substances organiques émises par la végétation) ou liés à l'activité industrielle (sulfates, nitrates).
\item[\emph{aérosols d'origine marine}] Ils sont formés à partir des bulles résultant du déferlement des vagues et des courants marins. Outre le dioxyde de carbone, les bulles transportent quantité de substances, notamment des particules de sel microscopiques (NaCl) qui participent à la formation de brumes. L'éclatement de ces bulles à la surface des océans donne naissance à un très grand nombre de gouttelettes (parfois une centaine pour une bulle d'un diamètre de l'ordre du millimètre) qui ne se brisent pas et sont à l'origine des aérosols marins (également appelés embruns).
\item[\emph{aérosols carbonés}] Ils sont présents dans les régions tropicales et boréales en raison de nombreux feux de forêt. Par exemple les brûlis de la végétation intertropicale en période sèche occasionnent chaque année des brumes sèches qui disparaissent une fois les pluies revenues.
\item[\emph{aérosols de sulfates}] Ils sont d'origine volcanique. Le dioxyde de soufre SO$_2$ émis lors des éruptions volcaniques produit ces fines particules d'acide sulfurique (SO$_2$ + H$_2$O~$\rightarrow$~H$_2$SO$_4$) qui s'entourent de glace et forment avec les cendres un écran empêchant le rayonnement solaire d'arriver jusqu'au sol.
\end{citemize}
La plupart des aérosols se trouvent dans la troposphère où ils résident en moyenne une semaine. En raison de leur petite taille, les aérosols peuvent être transportés sur de longues distances. Ils sont en général ramenés au sol par les précipitations (pluie, neige). Les aérosols de plus petite taille ($0.01-0.1$~$\mu$m) jouent un rôle important de \voc{noyaux de condensation} dans la formation des nuages en favorisant la condensation de vapeur d'eau en gouttelettes d'eau et/ou de cristaux de glace. Les aérosols de plus grande taille (0.1-1.0~$\mu$m), les plus nombreux, interceptent la lumière du Soleil. La stratosphère contient aussi des aérosols (principalement d'origine volcanique) jusqu'à 18-20 km d’altitude. Contrairement aux aérosols troposphériques, leur concentration est relativement uniforme et leur durée de vie beaucoup plus longue, de plusieurs mois à plusieurs années.  

\sk
Certaines molécules peuvent s’agréger pour former des particules liquides ou solides. Dans l'atmosphère terrestre, ceci concerne principalement l'eau à l'état liquide ou solide dans l'atmosphère qui participe à la formation d'\voc{hydrométéores}. Ce sont des particules d’eau liquide (gouttelettes d'eau) et/ou solide (cristaux de glace) suspendues dans l'atmosphère dont la taille varie de~$1$~$\mu$m à~$1$~cm. Les brumes et nuages sont formés de ces fines gouttelettes d'eau en suspension dans l'atmosphère, qui apparaissent dès que le seuil de saturation de l'air en vapeur d'eau est dépassé (ces mécanismes sont détaillés dans un chapitre ultérieur). Les nuages bas et intermédiaires sont constitués de gouttelettes d’eau liquide ; les nuages d’altitude de cristaux de glace de différentes formes géométriques. Certains sont accompagnés de précipitations lorsque les gouttes ou cristaux sont assez gros pour former de la pluie, neige, grêle ou verglas. %La figure~\ref{fig:droplet} donne un ordre d'idée des tailles respectives des noyaux de condensation, des gouttelettes nuageuses et des gouttes de pluie.
%\figside{0.4}{0.15}{decouverte/cours_meteo/gouttes.png}{Taille comparée des noyaux de condensation (\emph{cloud condensation nuclei}), des gouttelettes de brume ou de nuage (\emph{moisture droplets}), des gouttes de pluie (\emph{raindrop})}{fig:droplet}


\mk
\section{Echanges énergétiques dans l'atmosphère}

\sk
L'atmosphère, et plus généralement le système climatique dont elle fait partie, échangent de l'énergie sous diverses formes. Ce sera le but des chapitres suivants que de détailler certains de ces échanges énergétiques. On donne ici un aperçu de la diversité des types d'énergie impliqués, ce qui permet d'illustrer la complexité du système atmosphérique (voir également les figures~\ref{fig:pluri} et~\ref{fig:flux}). 

%\figside{0.6}{0.4}{decouverte/cours_meteo/giec2007_gcm.jpg}{Evolution des phénomènes et éléments du système climatique intégrés dans les modèles numériques de climat. FAR correspond à 1990. SAR correspond à 1995. TAR correspond à 2001. AR4 correspond à 2007. Figure tirée du 4ème rapport du GIEC en 2007.}{fig:gcm}

\sk
\subsection{\'Energie radiative}\label{sec:energrad}

\sk
L'énergie \voc{radiative} est relative au rayonnement électromagnétique, dont les caractéristiques sont rappelées dans un chapitre ultérieur. L'atmosphère reçoit de l'énergie sous forme radiative principalement par le rayonnement lumineux reçu du soleil (maximum d'énergie émise dans les longueurs d'onde visible) et le rayonnement lumineux réémis par la surface terrestre chauffée par le Soleil (maximum d'énergie émise dans les longueurs d'onde \voc{infrarouge}, plus grandes que les longueurs d'onde visibles et moins énergétiques). Le flux d'énergie instantané, c'est-à-dire la puissance par unité de surface, reçu par la Terre (surface + atmosphère) du Soleil est~$1368$~W~m$^{-2}$. On appelle cette valeur la \voc{constante solaire}. Le rayonnement solaire reçu au sommet de l'atmosphère en un point donné de la surface varie en fonction de la latitude, de l'heure de la journée et de la saison considérée (c'est-à-dire la position de la Terre au cours de sa révolution annuelle autour du Soleil). %Le flux moyen reçu par le sommet de l'atmosphère est quatre fois plus faible que la constante solaire~:~$342$~W~m$^{-2}$. 

\sk
L'énergie radiative peut être absorbée par les espèces atmosphériques et convertie en énergie interne pour l'atmosphère. L’absorption du rayonnement solaire est dû à 
\begin{finger}
\item l’ozone (O$_3$), le dioxygène (O$_2$) et la vapeur d’eau (H$_2$O) dans les courtes longueurs d’onde (UV / visible),
\item la vapeur d’eau, le dioxyde de carbone (CO$_2$), le méthane (CH$_4$), l’oxyde nitreux (N$_2$O) et l’ozone (O$_3$) dans les longues longueurs d’ondes (infrarouge et micro-onde).
\end{finger}
Le rayonnement incident peut également être diffusé par les gaz et les aérosols présents dans l'atmosphère. Ce dernier phénomène prévaut pour les longueurs d'onde visible, puisque l'absorption du rayonnement solaire incident par l'atmosphère est relativement faible : on dit que l’atmosphère est relativement \voc{transparente} au rayonnement solaire dans le visible. L'ensemble de ces phénomènes sont responsables de la différence d'énergie radiative reçue au sommet de l'atmosphère et au niveau de la mer [figure~\ref{fig:radiation}]. 

%\figsup{0.45}{0.2}{\figpayan/LP211_Chap2b_Page_18_Image_0001.png}{\figpayan/LP211_Chap1_Page_07_Image_0004.png}{[Gauche] Spectre du rayonnement solaire (UV, Visible, IR) à l’extérieur de l’atmosphère terrestre et au niveau du sol (après traversée de l’atmosphère). Le rayonnement au sommet de l'atmosphère correspond à la courbe rouge. Le rayonnement diffusé par l'atmosphère correspond à la courbe bleu foncé. Le rayonnement absorbé par l'atmosphère correspond à la courbe bleu clair. Le rayonnement résultant au sol correspond à la courbe jaune. [Droite] Figure similaire, mais les espèces gazeuses responsables des bandes d'absorptions sont indiquées sur la figure.}{fig:radiation}

\sk
Le rayonnement émis par la surface terrestre, principalement dans l'infrarouge, est également absorbé par les espèces précitées (vapeur d'eau, CO$_2$, CH$_4$) et réémis à la fois vers l'espace et vers la surface. Ainsi, une partie du rayonnement émis par la surface est \ofg{piégée}, n'est pas évacuée vers l'espace et contribue à augmenter la température de la surface terrestre. Ce phénomène est désigné par le terme d'\voc{effet de serre} et les gaz qui en sont responsables s'appellent les \voc{gaz à effet de serre}. Les nuages peuvent également induire un effet de serre. Du point de vue radiatif, les nuages jouent d'ailleurs un double rôle dans le bilan énergétique de la Terre : ils réfléchissent le rayonnement du soleil vers l’espace dans le visible (donc refroidissent l'atmosphère) et exercent un effet de serre (et donc réchauffent l'atmosphère). 

\sk
L'activité humaine et les processus naturels modifient les rapports de mélange des espèces radiatives présentes dans l'atmosphère (gaz ou aérosols), en particulier les gaz à effet de serre. L'homme modifie ainsi le bilan énergétique de la planète, donc le fonctionnement du climat : on parle de \voc{changement climatique} [figure~\ref{fig:giecrad}].

%\figside{0.45}{0.45}{decouverte/cours_meteo/ges.jpg}{Rapports de mélange du dioxyde de carbone~CO$_2$, méthane~CH$_4$ et oxyde nitreux~N$_2$O au cours des 10 000 dernières années (larges figures) et depuis 1750 (petits inserts). Les données indicatrices des changements de la composition de l’atmosphère au cours du dernier millénaire mettent en évidence l’augmentation rapide des gaz à effet de serre qui est imputable principalement à la croissance économique depuis 1750. Les gaz sont bien mélangés dans l’atmosphère et leurs concentrations depuis 1750 reflètent les émissions provenant des sources à travers le monde. Les mesures proviennent de carottes glaciaires (différentes couleurs sont utilisées pour les diverses études scientifiques utilisées) et de campagnes atmosphériques (lignes rouges). Les forçages radiatifs comparés à la valeur de 1750 pour chacun des gaz représentés sont reportés sur la droite des larges figures. Figure tirée du 4ème rapport du GIEC en 2007.}{fig:giecrad}

\sk
\subsection{\'Energie mécanique}

La \voc{dynamique atmosphérique}, c'est-à-dire les vents, fait intervenir l'énergie cinétique et potentielle (dont la somme forme l'énergie mécanique). L'atmosphère ne reçoit pas partout la même quantité d'énergie~: de par sa sphéricité, la Terre reçoit plus d'énergie radiative du Soleil en chaque instant dans les régions équatoriales et tropicales qu'aux pôles. L'énergie émise par le système Terre (principalement dans les longueurs d'onde infrarouge) suit un comportement différent, où les différences selon la latitude sont moins marquées. Il en résulte un excédent d'énergie dans les basses latitudes et un déficit d'énergie dans les hautes latitudes. Cette différence provoque des courants atmosphériques de grande échelle qui ont tendance à répartir l'énergie des régions excédentaires en énergie vers les régions déficitaires en énergie~[figure~\ref{fig:hadley}]. De manière plus générale, la différence de température dans l'air est une des clés de tout mouvement dans l'atmosphère. Ce n'est cependant pas la seule : la rotation de la Terre est d'une importance cruciale pour expliquer comment les vents s'organisent sur Terre. Des développements plus approfondis sont proposés dans un chapitre ultérieur.

%\figside{0.5}{0.2}{decouverte/cours_meteo/energiedyn.png}{Schéma représentant les latitudes où l'atmosphère est excédentaire ou, au contraire, déficitaire en énergie. La courbe bleue représente l'énergie radiative reçue du Soleil, principalement dans les courtes longueurs d'onde (noté \emph{shortwave} sur la figure, correspond au rayonnement visible et ultraviolet). La courbe rouge représente l'émission par la surface terrestre, principalement dans les longues longueurs d'onde (noté \emph{longwave} sur la figure, correspond au rayonnement infrarouge). Une circulation atmosphérique de grande échelle se met en place entre les régions excédentaires (équatoriales et tropicales) et déficitaires (hautes latitudes).}{fig:hadley}

\sk
\subsection{\'Energie latente}

\sk
Une partie de l’énergie reçue par la Terre est transférée vers l’atmosphère par l’intermédiaire de processus qui ne sont ni radiatifs ni dynamiques. Les transferts de \voc{chaleur latente}, relatifs aux changements d'état de l'eau, jouent un rôle particulièrement important sur Terre. L'\voc{évaporation} (ou vaporisation) de l'eau consomme de la chaleur latente afin de briser les liaisons hydrogène qui existent à l'état liquide pour passer à l'état gazeux (ou vapeur). Ceci fait baisser l’énergie interne et contribue donc à un refroidissement de l'atmosphère. A l'inverse, lors de la \voc{condensation}, de la chaleur latente est dégagée et contribue à un chauffage de l'air. Le passage de l’état liquide à l’état solide, comme lors de la formation de neige ou de cristaux de glace dans les cirrus, s’accompagne également d’un dégagement de chaleur latente. En résumé
\begin{citemize}
\item consomment de l'énergie : fusion ($S \rightarrow L$), vaporisation ($L \rightarrow G$), sublimation ($S \rightarrow G$) ; 
\item dégagent de l'énergie : solidification ($L \rightarrow S$), condensation ($G \rightarrow L$), condensation solide ($G \rightarrow S$).
\end{citemize}

\sk
\subsection{Contribution de la chimie}

\sk
La chimie atmosphérique participe aux échanges d'énergie de façon directe par le caractère exothermique ou endothermique des réactions chimiques, mais également de façon indirecte dans la mesure où elle opère un couplage entre les diverses formes d'énergie dans l'atmosphère. Un exemple des couplages associés à la chimie atmosphérique est donné par la figure~\ref{fig:chimie}.
\begin{citemize}
\item Tout d'abord, la chimie contribue à la conversion de l'énergie radiative en énergie interne par le biais des processus \voc{photochimiques}, qui sont des réactions chimiques faisant intervenir le rayonnement par le biais des photons qui le constituent (d'où l'étymologie du nom). 
\item Ensuite, les réactions chimiques et photochimiques modifient la composition atmosphérique des espèces minoritaires dont certaines, comme l'ozone, ont un rôle radiatif central. 
\item Le fait que les espèces qui réagissent soient transportées par les vents introduit une complexité supplémentaire par l'intrication des constantes de temps chimiques, radiatives et dynamiques. 
\item Enfin, la cinétique chimique dépend de la température atmosphérique qui est influencée par les échanges d'énergie sous forme radiative, mécanique, latente. 
\end{citemize}

\figun{0.7}{0.45}{\figpayan/LP211_Chap1_Page_10_Image_0001.png}{Atmosphère et chimie, exemple de l’ozone stratosphérique. Figure inspirée de Brasseur, The role of the stratosphere in global change, 1993.}{fig:chimie}
