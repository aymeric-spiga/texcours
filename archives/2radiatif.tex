\chapter{Rayonnement électromagnétique et émission thermique} \label{chap:rad} 
\dictum[The Beatles, 1970]{Here comes the sun, and I say it's allright}

		\bk
Comment déterminer les processus dynamiques, physiques, chimiques à l'oeuvre dans l'atmosphère ? Il faut commencer par faire le point sur les sources d'énergie pour l'atmosphère, les océans et la surface. La principale source d'énergie pour l'atmosphère et le système climatique de la Terre est le Soleil\footnote{Ce n'est pas le cas pour les géantes gazeuses Jupiter et Saturne où il existe un flux de chaleur interne significatif en regard du flux d'énergie reçu du Soleil. Ce flux est un reste de la contraction gravitationnelle au cours de la formation de ces géantes gazeuses.}. La figure~\ref{fig:flux} montre que d’autres sources existent mais en quantité réduite : l'énergie reçue par la géothermie, ou par les activités humaines, est~$4$ ordres de grandeur plus faible que la source solaire; celle reçue des étoiles~$8$ ordres de grandeur plus faible. L’énergie solaire est transmise principalement à la Terre au moyen du rayonnement électromagnétique~: on qualifie cette énergie de \voc{radiative}. %L'objet de ce chapitre est de s'intéresser au rayonnement électromagnétique et plus particulièrement au phénomène d'émission thermique.

\figside{0.75}{0.18}{decouverte/cours_meteo/fluxenergsurf.png}{Ordres de grandeur des flux énergétiques reçus à la surface de la Terre. Source~:~P.~von Balmoos \emph{in} Le Climat à Découvert, CNRS éditions, 2011}{fig:flux}



\mk \section{Description générale du rayonnement électromagnétique}

	\sk \subsection{Spectre électromagnétique}

		\sk
Les échanges d'\voc{énergie radiative} se font à distance par le biais du \voc{rayonnement électromagnétique}. Le rayonnement électromagnétique est composé d'une superposition d'ondes monochromatiques de longueurs d'onde~$\lambda$ se propageant à la vitesse de la lumière~$c$ (dans le vide~$c=3 \times 10^8$~m~s$^{-1}$). Le rayonnement électromagnétique parcourt la distance Terre-Soleil en $8$~minutes; à l'échelle des processus atmosphériques terrestres, la propagation des ondes électromagnétiques est si rapide qu'elle peut être considérée en première approximation comme immédiate. 

\sk
Les ondes composant le rayonnement électromagnétique peuvent être caractérisées indifféremment par leur \voc{longueur d'onde}~$\lambda$, leur \voc{fréquence}~$\nu = c / \lambda$ ou leur \voc{nombre d'onde}\footnote{Le nombre d'onde est souvent exprimé en cm$^{-1}$. Pour obtenir~$\overline{\nu}$ dans cette unité à partir de~$\lambda$ en microns, on utilise~$\overline{\nu} = 10^{4} / \lambda$.}~$\overline{\nu} = 1 / \lambda$. L'ensemble de ces ondes constitue le \voc{spectre} électromagnétique. Selon le principe de De Broglie, à chaque onde électromagnétique de fréquence~$\nu$ est associée une particule sans masse nommée \voc{photon} dont l'énergie est~$h \, \nu$ où $h = 6.63 \times 10^{-34}$~J~s est appelée la constante de Planck. Cette énergie est souvent exprimée en électron-volts eV ($1$~eV~$= 1.6 \times 10^{-19}$~J~s).

\sk
Le rayonnement visible occupe une bande très étroite du spectre aux longueurs d'ondes comprises entre 0.4 et 0.76~$\mu$m [figure~\ref{fig:spectrum}]. Lorsque l'on considère des longueurs d'ondes plus courtes (c'est-à-dire des fréquences plus élevées) que le rayonnement visible, on passe dans le domaine du rayonnement ultraviolet, puis celui des rayons X et gamma~$\gamma$. Lorsque l'on considère des longueurs d'ondes plus grandes (c'est-à-dire des fréquences plus faibles) que le rayonnement visible, on passe dans le domaine du rayonnement infrarouge, puis celui des micro-ondes et des ondes radio. Les photons les plus énergétiques correspondent aux rayons X; les moins énergétiques aux ondes radio.

\figsup{1}{0.1}{\figpayan/LP211_Chap2_Page_09_Image_0001.png}{\figpayan/LP211_Chap2_Page_09_Image_0002.png}{Classification du rayonnement électromagnétique en fonction de la longueur d'onde. On rappelle que 1~$\mu$m (micron) correspond à $10^{-6}$~m et 1~nm (nanomètre) correspond à $10^{-9}$~m.}{fig:spectrum}


	\sk \subsection{Mesures quantitatives~: grandeurs caractéristiques}

		\sk
La quantité de rayonnement émise par une source de rayonnement, ou reçue par une cible, dépend des paramètres~:
\begin{citemize}
\item longueur d'onde~$\lambda$ ;
\item temps d'exposition~$t$ ;
\item surface~$S$ de l'objet (source ou cible) ;
\item direction dans l'espace considérée, que l'on repère par l'angle~$\beta$ entre le rayonnement incident (ou émis) et la normale à la surface (appelé angle zénithal) ;
\item portion d'espace considérée, exprimée par un \voc{angle solide} $\omega$, l'équivalent bidimensionnel d'un angle\footnote{De même qu'un angle en radians est la longueur d'un arc de cercle divisée par le rayon, le stéradian est la surface d'une portion de sphère divisée par le rayon au carré. On a donc $4\pi$ stéradians sur tout l'espace: voir figure \ref{fig:radiance}} ;
\item propriétés physico-chimiques de l'objet, par exemple sa température (voir section~\ref{corpsnoir}).
\end{citemize}
On exprime cette quantité de rayonnement reçue ou émise sous la forme d'une énergie totale~$E\e{r}$ en Joules (J). Que l'on considère une source ou une cible, $E\e{r}$ est l'énergie transmise par le rayonnement (radiative).

\sk
L'énergie totale~$E\e{r}$ est une grandeur intégrée peu utilisée en pratique en sciences de l'atmosphère. On lui préfère les quantités décrites ci-dessous\footnote{Les noms anglais de ces quantités sont respectivement \emph{radiant flux} pour~$\Phi$, \emph{irradiance} pour~$F$, \emph{radiance} pour~$L$} qui décrivent la quantité de rayonnement émise ou reçue par unité de temps, surface, longueur d'onde, \ldots
\begin{finger}
\item \underline{unité de temps} Le \voc{flux énergétique}~$\Phi$ est l'énergie totale~$E\e{r}$ par unité de temps~$t$ (par seconde) $$ \Phi = \ddf{E\e{r}}{t} $$ C'est une puissance exprimée en Watts (W~$\equiv$~J~s$^{-1}$). Le flux énergétique~$\Phi$ est intégré sur toutes les longueurs d'onde, toutes les directions d'espace et sur l'intégralité de la surface de la source ou cible.
\item \underline{unité de temps + unité de surface} La \voc{densité de flux énergétique}~$F$ est le flux énergétique~$\Phi$ par unité de surface~$S$ de la source/cible $$ \boxed{ F = \ddf{\Phi}{S} } = \f{\dd^2 E\e{r}}{\dd S \, \dd t} $$ C'est un \voc{flux net} exprimé en~W~m$^{-2}$. On l'appelle également \voc{émittance}~$M$ pour une source et \voc{éclairement}~$E$ pour une cible. Cette quantité est intégrée sur toutes les longueurs d'onde et toutes les directions d'espace.
\item \underline{unité de temps + unité de surface + direction fixée} La \voc{luminance énergétique}~$L$ est la densité de flux énergétique dans une direction donnée de l'espace repérée par un angle~$\beta$ $$ L = \f{\dd F}{\cos\beta \, \dd \omega} = \f{\dd^2 \Phi}{\cos\beta \, \dd \omega \, \dd S} = \f{\dd^3 E\e{r}}{\cos\beta \, \dd \omega \, \dd S \, \dd t} $$ 
par unité d'angle solide~$\omega$ (figure \ref{fig:radiance}). C'est une quantité surtout utilisée pour les sources, parfois appelée radiance. La surface considérée~$\sigma$ est perpendiculaire à la direction d'émission~: on la relie à~$S$ par~$\dd \sigma = \cos\beta \, \dd S$. La luminance énergétique~$L$ en~W~m$^{-2}$~sr$^{-1}$ est intégrée sur toutes les longueurs d'onde.
\item \underline{longueur d'onde fixée (grandeurs spectrales)} Le flux énergétique \voc{spectral} ou monochromatique~$\Phi_{\lambda}$ est le flux énergétique~$\Phi$ par unité de longueur d'onde~$\lambda$ $$ \Phi_{\lambda} = \ddf{\Phi}{\lambda} = \f{\dd^2 E\e{r}}{\dd \lambda \, \dd t} $$ Cette quantité en~W~m$^{-1}$ est intégrée sur toutes les directions d'espace et sur toute la surface de la source ou cible. On peut également définir des équivalents spectraux~$F_{\lambda}$ et~$L_{\lambda}$ pour les quantités~$F$ et~$L$ $$ \boxed{F_{\lambda} = \ddf{F}{\lambda}} \qquad\qquad L_{\lambda} = \ddf{L}{\lambda} $$ et des quantités spectrales~$\Phi_{\nu}$, $F_{\nu}$ et $L_{\nu}$ à partir de la fréquence~$\nu$ $$ \Phi_{\nu} = \ddf{\Phi}{\nu} \qquad\qquad F_{\nu} = \ddf{F}{\nu} \qquad\qquad L_{\nu} = \ddf{L}{\nu} $$ Les quantités spectrales définies à partir de la fréquence sont parfois plus avantageuses, dans la mesure où la fréquence est indépendante du milieu matériel transparent où l'onde matérielle se propage\footnote{La longueur d'onde~$\lambda$ dépend de l'indice de réfraction~$n$ du milieu (pour l'air, $n$ est proche de~$1$) et de la longueur d'onde dans le vide~$\lambda_0$.}. Attention, les unités des quantités spectrales dépendent de la quantité référente : ainsi $L_{\lambda}$ est en~W~m$^{-3}$~sr$^{-1}$ et $L_{\lambda}$ est en~W~m$^{-2}$~sr$^{-1}$~s.
\end{finger}

\figun{0.6}{0.15}{\figfrancis/lum_emit.pdf}{Schéma montrant l'émittance~$M$ et la luminance~$L$ d'un élément de surface $dS$ d'une source. $M$ est l'intégrale du flux dans toutes les directions. $L$ est le flux émis dans une certaine direction par unité de surface perpendiculaire.}{fig:luminance}

\begin{figure} \begin{center} \input{\figfrancis/luminance.pdftex_t} \end{center} \caption{\emph{ Schéma en coordonnées sphériques de la luminance $L$ de l'élément de surface $dS$ d'une source située dans un plan $(Oxy)$. La luminance est définie pour chaque direction repérée par les angles $\theta$ et $\varphi$. L'angle solide élémentaire autour d'une direction donnée vaut $\dd \omega = \sin\theta \, \dd\theta \, \dd\varphi$ (rapport entre surface hachurée et $r^2$). 
%La relation avec le flux énérgétique~$\Phi$ émis par la source est $L = \dd^2 \Phi / \left( \dd\omega \, \dd S \, \cos\theta \right)$.
}} \label{fig:radiance} \end{figure}

\sk
\subsubsection{Exemples d'application}

\sk
Dans ce qui précède, on part de la quantité la plus intégrée possible, à savoir l'énergie totale~$E\e{r}$, pour parvenir par dérivation à des quantités moins complexes à appréhender en pratique. Le chemin inverse se fait par intégration (sommation continue). On considère ici quelques exemples illustratifs. 
\begin{finger}
\item Pour reprendre la situation de la figure~\ref{fig:luminance}, supposons que l'on dispose d'informations sur la luminance énergétique~$L$ d'une source plane de surface élémentaire~$\dd S$, c'est-à-dire la quantité de rayonnement émise dans chaque direction de l'espace. Afin de connaître le flux net~$F$ dans tout l'espace (que l'on peut noter également émittance~$M$ puisque l'on considère une source), il suffit de l'écrire comme une intégrale de la luminance sur toutes les directions d'un demi-espace: $$ F = \int_{2\,\pi} L \, \cos\beta \, \dd \omega $$ où $2\,\pi$ représente l'intégration sur un demi-espace. %Deux cas particuliers sont intéressants. Dans la limite d'un rayonnement rasant~$\beta=\pi/2$, la contribution au flux net est nulle. Par ailleurs, s
Si la luminance~$L$ est indépendante de la direction, c'est-à-dire que le rayonnement est \voc{isotrope}, l'intégration donne simplement~$F=\pi L$. Dans ce cas on parle d'une \voc{source lambertienne}.
%C'est le cas d'un réflecteur de Lambert.
%Jean-Henri Lambert (1728-1777) a observé que l’énergie émise par certaines sources (parmi toutes les types de sources à sa disposition) anisotropes diminue comme le cosinus de l'angle θ, autour de la direction perpendiculaire à la surface de la source. Cette variation de l'énergie émise est observée lorsque nous mesurons l'énergie thermique rayonnée par un orifice percé dans un four (ce qui nous ramène au corps noir défini plus loin), isolé thermiquement et dont la température interne est supérieure à la température externe. Dans ce contexte, l'orifice est appelé un émetteur Lambert et ne balaye un espace que de stéradian. Une source obéit à la loi de Lambert si l’énergie rayonnée depuis un point de cette source est la même dans toutes les directions (on dit que son intensité est isotrope et donc indépendante de l'angle d’où on observe cette source). Soit M la valeur de l’éclairement mesurée par un capteur. On peut facilement en déduire le flux énergétique de la source de surface S : Φ =S M
\item Suppons que l'on dispose cette fois d'informations sur la densité de flux énergétique~$F$ d'une source sphérique de rayon~$R$ (par exemple, le Soleil). Puisque l'on considère une source, $F$ peut également être appelée émittance et être notée~$M$. Si le rayonnement est \voc{uniforme}, c'est-à-dire qu'en chaque point la source émet le même flux énergétique~$\Phi$ par unité de surface~$\dd S$, alors on dispose de la relation suivante $$ \Phi = 4\,\pi\,R^2 \, M$$ et plus généralement pour une surface~$S$ qui est une source uniforme de rayonnement $$ \boxed{ \Phi = S \, M } $$ On notera qu'on fait les calculs en considérant seulement le côté extérieur de la surface, celui d'où nous regardons la source, car seule la moitié de l'énergie échangée par les points de la surface~$S$ est émise sous forme de rayonnement. L'autre moitié est échangée du côté intérieur de la surface avec le milieu constituant le corps. 
\item L'énergie transmise par le rayonnement, et toutes les grandeurs définies précédemment, varient généralement beaucoup avec la longueur d'onde étudiée. Supposons que l'on connaisse la luminance spectrale~$L_\lambda$ dans un petit intervalle~$\dd \lambda$ autour de la longueur d'onde~$\lambda$, et ce, pour toutes les longueurs d'onde~$\lambda$ du spectre électromagnétique. Par exemple, la luminance totale~$L$ est retrouvée par intégration des longueurs d'onde les plus courtes aux plus longues : $$L=\int_{\lambda} L_\lambda \, \dd\lambda = \int_\nu L_\nu \, \dd \nu $$ Il s'agit d'une des formes du \voc{principe de superposition}, qui indique que le rayonnement électromagnétiques se compose d'une superposition d'ondes monochromatiques. Une intégration similaire est effectuée par l'électronique embarquée dans un appareil photo qui traite les flux reçus par les capteurs dans une variété de longueurs d'onde en domaine visible, afin d'obtenir une image finale qui intègre toutes ces informations.
\end{finger}



\mk \section{Emission de rayonnement} \label{corpsnoir}

		\sk
Le Soleil qui se situe à une distance considérable dans le vide spatial nous procure une sensation de chaleur. De même, placer sa main sur le côté d'un radiateur en fonctionnement sans le toucher procure une sensation de chaleur instantanée qui ne peut être attribuée à un transfert convectif entre le radiateur et la main. Cet échange de chaleur est attribué au contraire à l'émission d'ondes électromagnétiques par la matière du fait de sa température; on parle d'émission de \voc{rayonnement thermique}. Tous les corps émettent du rayonnement thermique. La transmission de cette énergie entre une source et une cible ne nécessite pas la présence d'un milieu intermédiaire matériel. 
%Le but de cette section est d'en étudier les principales propriétés.


	\sk \subsection{Corps noir}

		\sk
On appelle \voc{corps noir} un objet dont la surface est idéale et satisfait les trois conditions suivantes~:
\begin{description}
\item[émetteur parfait] un corps noir rayonne plus d’énergie radiative à chaque température et pour chaque longueur d’onde que n'importe quelle autre surface,
\item[absorbant parfait] un corps noir absorbe complètement le rayonnement incident selon toutes les directions de l'espace et toutes les longueurs d'onde,
\item[source lambertienne] un corps noir émet du rayonnement de façon isotrope
\end{description}

\sk
Un corps noir est à l'équilibre thermodynamique avec son environnement. On peut montrer qu'un tel corps émet du rayonnement qui dépend seulement de sa température et non de sa nature. La définition du corps noir, et les développements théoriques qui l'accompagnent, sont partis du constat, fait notamment par les céramistes, qu'un objet placé dans un four à haute température devient rouge en même temps que les parois du four quelle que soit sa taille, sa forme ou le matériau qui le compose. Un exemple de source utilisée pour étudier expérimentalement le modèle du corps noir consiste à construire une enceinte chauffée, totalement hermétique, et y percer un trou pour y mesurer le flux énergétique émis [figure~\ref{fig:four}]

\figside{0.35}{0.15}{\figwallace/Radiation/radiation_Page_10_Image_0001.png}{L'énergie entrant par une petite fente dans une enceinte subit des réflexions sur la paroi jusqu'à ce qu'elle soit absorbée. L'ouverture dans la paroi d'une enceinte chauffée apparaît comme une source de type corps noir. Un absorbant presque parfait est aussi un émetteur presque parfait. Ce type de four a été employé au début du XXe siècle pour évaluer expérimentalement les prédictions théoriques de Planck. Source~: Wallace and Hobbs, Atmospheric Science, 2006.}{fig:four}


		\sk
L'émission de rayonnement par le corps noir est décrite par une luminance énergétique spectrale~$L_{\lambda}$, notée $B_\lambda$ dans ce qui suit\footnote{Correspond au nom anglais \emph{blackbody}}. La loi de variation de~$B_\lambda$ selon la température~$T$ est donnée par la \voc{loi de Planck}\footnote{La luminance spectrale $B_\nu$ est déterminée d'une façon similaire. La démonstration de la loi de Planck fait appel à des notions de quantification d'énergie et de thermodynamique statistique qui sont hors programme dans le cadre de ce cours.} $$ B_\lambda(T) = \frac{C_1 \, \lambda^{-5}}{\pi \, \left( e^{ C_2 / \lambda T}-1\right) } $$ où $C_1$ et $C_2$ sont des constantes. Comme le rayonnement du corps noir est isotrope, l'émittance spectrale du corps noir, obtenue par intégration sur toutes les directions de l'espace, vaut $ M_\lambda(T) = \pi \, B_\lambda(T) $. 

%\figun{0.5}{0.25}{\figfrancis/WH_BBrad}{Courbes de luminance spectrale d'un corps noir pour différentes températures. La courbe en pointillés indique la position du maximum en fonction de $T$.}{fig:BBrad} 
\figside{0.5}{0.25}{\figwallace/Radiation/radiation_Page_11_Image_0001.png}{Courbes de luminance spectrale d'un corps noir pour différentes températures. La courbe en pointillés indique la position du maximum en fonction de $T$. Source~: Wallace and Hobbs, Atmospheric Science, 2006.}{fig:BBrad} 

\sk
Les variations de la fonction~$B_\lambda$ sont illustrées sur la figure~\ref{fig:BBrad}. L'émission de rayonnement par le corps noir ne dépend que de la longueur d'onde~$\lambda$ et de la température~$T$ du corps. A une température donnée, le rayonnement émis est parfaitement déterminé pour chaque longueur d'onde; dans un domaine spectral particulier, le rayonnement émis ne dépend que de la température du corps noir.


		\paragraph{Variations selon la température} 

\begin{finger}
\item L'énergie émise dépend de la température du corps émetteur~: 
\begin{citemize}
\item quantitativement~: plus le corps est chaud, plus la quantité de rayonnement thermique est grande~: la luminance spectrale~$B_{\lambda}$ augmente avec la température $T$ quelle que soit la longueur d'onde.
\item qualitativement~: la \ofg{couleur} du corps dépend de sa température~: la longueur d'onde pour laquelle le rayonnement est maximal diminue quand la température augmente.
\end{citemize}
\item La dépendance en température de la forme des courbes sur la figure~\ref{fig:BBrad} est résumée par deux lois simples qui sont décrites à la section suivante~: la loi de Wien (position du maximum) et la loi de Stefan-Boltzmann (intégrale totale).  
\end{finger}

\paragraph{Variations selon la longueur d'onde} 

\begin{finger} 
\item Le rayonnement thermique est surtout significatif entre les longueurs d'onde~$0.1$ et~$100$~$\mu$m, soit le domaine visible et infrarouge. Pour le type de température usuellement rencontrées sur Terre, la contribution dans les longueurs d'onde visible est petite par rapport à la contribution dans l'infrarouge -- il faut atteindre des températures de plusieurs centaines de degrés Celsius pour qu'elle devienne significative, comme on peut le constater lorsqu'on porte à haute température un morceau de métal ou que l'on considère une coulée de lave fraîche.
\item La luminance énergétique~$B_{\lambda}$ tend vers 0 
\begin{citemize}
\item aux longueurs d'ondes très courtes, ce qui signifie que le rayonnement thermique comporte extrêmement peu des photons les plus énergétiques;
\item et aux longueurs d'onde très grandes, ce qui est attendu étant donné que l'énergie des photons tend vers~$0$ et que leur nombre n'est pas suffisant pour que la contribution énergétique soit significative.
\end{citemize}
\end{finger}



	\sk \subsection{Lois du corps noir}

		\sk \subsubsection{Loi de Wien : maximum d'émission thermique}

		\sk
On observe sur la figure \ref{fig:BBrad} que, lorsque~$T$ augmente, la maximum de la luminance spectrale~$B_\lambda$, appelé \voc{maximum d'émission}, se décale vers les longueurs d'onde courtes, c'est-à-dire correspond à des photons de plus en plus énergétiques. La loi exacte, appelée \voc{loi de déplacement de Wien}, s'obtient en dérivant $B_\lambda$ par rapport à $\lambda$, ce qui permet d'obtenir $$ \boxed{ \lambda\e{max} \, T = 2898 \quad (\mu\textrm{m~K}) } $$ où $\lambda_{max}$ est la longueur d'onde du maximum de luminance spectrale~$B_\lambda$. La longueur d'onde du maximum d'émission~$\lambda\e{max}$ est ainsi inversement proportionnelle à la température du corps émetteur. Une formulation alternative est que $\nu\e{max}$ est proportionelle à $T$.

\figun{0.6}{0.45}{/home/aymeric/Big_Data/BOOKS/pierrehumbert_pics/9780521865562c03_fig001.jpg}{Source~: R. Pierrehumbert, Principles of Planetary Climates, CUP, 2010.}{wvl} 



		\sk \subsubsection{Loi de Stefan-Boltzmann : flux net surfacique}

		\sk
La \voc{loi de Stefan-Boltzmann}\footnote{Joseph Stefan met expérimentalement en évidence en 1879 la dépendance de l'émittance en puissance quatrième de la température. Ludwig Boltzmann, à qui l'on doit également des résultats fondamentaux sur l'entropie et l'atomisme, prouve en 1884 le résultat par des arguments théoriques.} donne la valeur de l'intégrale sur toutes les longueurs d'ondes et dans tout l'espace\footnote{On entend par là toutes les directions du demi-espace extérieur au corps considéré.} de la courbe du corps noir, décrite par la loi de Planck et illustrée par les figures \ref{fig:BBrad} et \ref{fig:BBmax}. Cette loi donne donc l'expression d'une densité de flux énergétique~$F$ ou plus spécifiquement, puisque le corps noir est une source de rayonnement, d'une émittance totale~$M$. Cette dernière s'obtient tout d'abord avec une intégration par rapport à~$\lambda$ de la luminance énergétique spectrale~$B_\lambda$ donnée par la loi de Planck, afin d'obtenir la luminance énergétique~$B$. On déduit ensuite l'émittance totale~$M$ en intégrant selon toutes les directions de l'espace; comme le rayonnement du corps noir est isotrope, $M$ s'obtient à partir de~$B$ simplement en multipliant par~$\pi$. La loi de Stefan-Boltzmann établit que le flux net surfacique~$M$ émis par un corps noir ne dépend que de sa température par une dépendance type loi de puissance $$ \boxed{ M\e{corps noir} = \sigma \, T^4 } $$ avec~$\sigma=5.67 \times 10^{-8} \textrm{~W~m}^{-2}\textrm{~K}^{-4}$ appelée constante de Stefan-Boltzmann. La loi de Stefan-Boltzmann, comme la loi de Planck dont elle dérive, stipule que l'émittance~$M$ d'un corps pouvant être considéré en bonne approximation comme un corps noir ne dépend que de sa température et non de sa nature. Cette loi indique par ailleurs que l'émittance~$M$ augmente très rapidement avec la température -- de par la puissance quatrième impliquée.


	\sk \subsection{Lois des corps gris et émissivité}

		\sk
Le corps noir est un modèle idéal d'absorbant qu'en pratique on ne rencontre pas dans la nature. Par exemple, le charbon noir est un absorbant parfait, mais seulement dans les longueurs d'onde visible. La plupart des objets ressemblent néanmoins au corps noir, au moins à certaines températures et pour certaines longueurs d'onde considérées en pratique. Dans le cas d'un corps qui n'est pas un absorbant parfait, on parle d'un \voc{corps gris}. A température égale, un corps gris n'émet pas autant qu'un corps noir dans les mêmes conditions. Pour évaluer l'énergie émise par un corps gris par comparaison à celle qu'émettrait le corps noir dans les mêmes conditions, on définit un coefficient appelé \voc{émissivité} $\epsilon_\lambda$ compris entre~$0$ et~$1$ et égal au rapport entre la luminance spectrale du corps~$L_\lambda$ et celle du corps noir~$B_\lambda$ $$ \epsilon_\lambda=\frac{L_\lambda}{B_\lambda(T)} $$ En toute généralité, l'émissivité~$\epsilon_{\lambda}$ d'une surface à une longueur d'onde~$\lambda$ dépend de ses propriétés physico-chimiques, de sa température et de la direction d'émission\footnote{Par exemple, les métaux, matériaux conducteurs de l'électricité, ont une émissivité faible (sauf dans les directions rasantes) qui croît lentement avec la température et décroît avec la longueur d'onde ; au contraire, les diélectriques, matériaux isolant de l'électricité, ont une émissivité élevée qui augmente avec la longueur d'onde et se révèlent lambertiens sauf pour les directions rasantes où l'émissivité décroît significativement.}.

\sk
On peut également définir une émissivité totale intégrée~$\epsilon$ qui permet d'exprimer l'émittance~$M$ d'un corps gris $$ \boxed{\SB} $$ Des valeurs de l'émissivité totale~$\epsilon$ pour certains matériaux sont données dans le tableau~\ref{tab:emiss}~: l'eau, la neige, les roches basaltiques ont des émissivités proches de~$1$ et peuvent être considérées comme des corps noirs en bonne approximation. 

\begin{table}\label{tab:emiss}
\begin{center}
\begin{tabular}{|c|c|}
\hline
Matériau & Emissivité~$\epsilon$ \\
\hline
Aluminium & 0.02 \\
Cuivre poli & 0.03 \\
Nuages type cirrus & 0.10 à 0.90 \\
Nuages type cumulus & 0.25 à 0.99 \\
Cuivre oxydé & 0.5 \\
Béton & 0.7 à 0.9 \\
Carbone & 0.8 \\
Lave (volcan actif) & 0.8 \\
Neige âgée & 0.8 \\
Ville & 0.85 \\
Désert & 0.85 à 0.9 \\
Peinture blanche & 0.87 \\
Brique rouge & 0.9 \\
Herbe & 0.9 à 0.95 \\
Eau & 0.92 à 0.97 \\
Peinture noire & 0.94 \\
Forêt & 0.95 \\
Suie & 0.95 \\
Neige fraîche & 0.99 \\
\hline
\end{tabular}
\caption{\emph{Quelques valeurs usuelles d'émissivité à la température ambiante (pour un rayonnement infrarouge). Source~: Hecht, Physique, 1999 -- avec quelques ajouts d'après site CNES}}
\end{center}
\end{table}


\mk \section{Energie reçue du Soleil}

	\sk \subsection{Caractéristiques et domaine de longueurs d'onde}

		\sk
Le Soleil peut être considéré en bonne approximation comme un corps noir car il absorbe tout le rayonnement incident. Sa \ofg{couleur} est dûe à du rayonnement émis et, plus précisément, correspond aux longueurs d'onde où le maximum de rayonnement est émis. D'après la loi de Wien, le Soleil, dont l'enveloppe externe a une température autour de~$6000$~K, a donc un maximum d'émission situé dans le visible à $\lambda\e{max} = 0.5 \mu$m, proche du maximum de sensibilité de l'oeil humain [figure~\ref{fig:BBmax} haut]. Au contraire, la surface terrestre, dont la température typique est d'environ~$288$~K, voit son maximum d'émission situé dans l'infrarouge vers 10~$\mu$m, alors que le rayonnement émis dans les longueurs d'ondes visible est négligeable [figure~\ref{fig:BBmax} bas]. Un raccourci usuel est donc de dire que \ofg{la Terre émet du rayonnement (thermique) dans l'infrarouge alors que le Soleil émet dans le visible}. En toute rigueur, cette affirmation ne parle que du voisinage du maximum d'émission, où la contribution au flux intégré selon toutes les longueurs d'onde est la plus significative. Il est ainsi plus exact de dire que, dans l'atmosphère, la région du spectre où~$\lambda$ est inférieure à environ 4~$\mu$m est dominée par le rayonnement d'origine solaire, alors qu'au-delà, le rayonnement est surtout d'origine terrestre. Il n’y a pratiquement pas de recouvrement entre la partie utile du spectre du rayonnement solaire et celui d’un corps de température ambiante; ce fait est d'une grande importance pour les phénomènes de type effet de serre, qui sont abordés plus loin dans ce cours. On désigne ainsi souvent le rayonnement d'origine solaire par le terme \voc{ondes courtes} et le rayonnement d'origine terrestre par le terme \voc{ondes longues}.

\figsup{0.65}{0.2}{decouverte/cours_meteo/6000K.jpg}{decouverte/cours_meteo/earth.jpg}{Courbes de luminance spectrale d'un corps noir pour différentes températures correspondant notamment au Soleil (haut) et à la Terre (bas). La quantité représentée ici est l'émittance spectrale~$M_\lambda = \pi \, B_\lambda$. Noter la différence d'indexation de l'abscisse et l'ordonnée sur les deux schémas. Le rayonnement thermique émis par la Terre est plusieurs ordres de grandeur moins énergétique que celui émis par le Soleil et le maximum d'émission se trouve à des longueurs d'onde plus grandes (infrarouge pour la Terre au lieu de visible pour le Soleil). Source : \url{http://hyperphysics.phy-astr.gsu.edu/hbase/bbrc.html}.}{fig:BBmax}


	\sk \subsection{Constante solaire}

		\sk
La distance Soleil-Terre est beaucoup plus grande que les rayons de la Terre et du Soleil. Ainsi, d'une part, le rayonnement solaire arrive au niveau de l'orbite terrestre en faisceaux pratiquement parallèles. D'autre part, la luminance en différents points de la Terre ne varie pas. On peut par conséquent définir une valeur moyenne de la densité de flux énergétique du rayonnement solaire au niveau de l'orbite terrestre, reçue par le système surface~+~atmosphère. Elle est désignée par le terme de \voc{constante solaire} notée~$\mathcal{F}\e{s}$. Les mesures indiquent que
\[ \mathcal{F}\e{s} = 1368 \text{~W~m}^{-2} \qquad \text{pour la Terre} \]

\sk
La constante solaire est une valeur instantanée côté jour~: le rayonnement solaire reçu au sommet de l'atmosphère en un point donné de l'orbite varie en fonction de l'heure de la journée et de la saison considérée (c'est-à-dire la position de la Terre au cours de sa révolution annuelle autour du Soleil)\footnote{En réalité, la constante solaire~$\mathcal{F}\e{s}$ varie elle-même d'environ~$3$~W~m$^{-2}$ en fonction des saisons à cause de l'excentricité de l'orbite terrestre, qui n'est pas exactement circulaire. De plus, elle peut varier évidemment en fonction des cycles solaires, néanmoins sans influence majeure sur la température des basses couches atmosphériques (troposphère et stratosphère).}. On peut donc définir un \voc{éclairement solaire moyen} noté~$\mathcal{F}\e{s}'$ reçu par la Terre qui intègre les effets diurnes et saisonniers. Autrement dit, $\mathcal{F}\e{s}$~est l'éclairement instantané reçu par un satellite en orbite autour de la Terre~; $\mathcal{F}\e{s}'$ est la valeur que l'on obtiendrait si l'on faisait la moyenne d'un grand nombre de mesures instantanées du satellite à diverses heures et saisons. 

\figside{0.5}{0.2}{decouverte/cours_dyn/incoming.png}{Energie reçue du Soleil par le système Terre. Source~: McBride and Gilmour, \emph{An Introduction to the Solar System}, CUP 2004.}{fig:eqrad}

\sk
On admet ici que~$\mathcal{F}\e{s}'$ peut être calculé en considérant que le flux total reçu du Soleil l'est à travers un disque de rayon le rayon~$R$ de la Terre (il s'agit de l'ombre projetée de la planète, voir Figure~\ref{fig:eqrad}). A cause de l'incidence parallèle, le flux énergétique intercepté par la Terre vaut donc~$\Phi = \pi \, R^2 \, \mathcal{F}\e{s}$. L'éclairement moyen à la surface de la Terre est alors $$\mathcal{F}\e{s}' = \frac{\Phi}{4 \, \pi \, R^2}$$ le dénominateur étant l'aire de la surface complète de la Terre. On obtient ainsi
\[ \boxed{ \mathcal{F}\e{s}' = \frac{\mathcal{F}\e{s}}{4} } \]

		\sk
La valeur de la constante solaire peut s'obtenir par le calcul. Le soleil est considéré en bonne approximation comme un corps noir de température~$T_{\sun} = 5780$~K. D'après la loi de Stefan-Boltzmann, son émittance est $M = \sigma \, T_{\sun}^4$ donc le flux énergétique~$\Phi_{\sun}$ émis par le Soleil de rayon~$R_{\sun} = 7 \times 10^5$~km est~$\Phi_{\sun} = 4 \, \pi \, R_{\sun}^2 \, \sigma \, T_{\sun}^4$. Ce flux énergétique est rayonné dans tout l'espace~: à une distance~$d$ du soleil il est réparti sur une sphère de centre le soleil et de rayon~$d$, donc de surface~$4 \, \pi \, d^2$. A cette distance, l'éclairement~$\mathcal{F}$, c'est-à-dire la densité de flux énergétique reçue en W~m$^{-2}$, s'écrit donc
\[ \mathcal{F} = \frac{\Phi_{\sun}}{4 \, \pi \, d^2} = \frac{4 \, \pi \, R_{\sun}^2 \, \sigma \, T_{\sun}^4}{4 \, \pi \, d^2} = \sigma \, T_{\sun}^4 \, \left( \frac{R_{\sun}}{d} \right)^2 \]
Si l'on prend~$d$ égal à la distance Terre-Soleil, $\mathcal{F}$ définit ainsi la constante solaire~$\mathcal{F}\e{s}$.
%\[ \mathcal{F}\e{s} = \frac{{\mathcal{F}\e{s}}^{\text{Terre}}}{d\e{soleil}^2} \]

%Variation de la constante solaire : Bien que l’intensité du soleil ait subit des variations depuis la formation de la Terre, on peut s’attendre à ce qu’elle soit stable sur une période de 1000 ans. On mesure mal la constante solaire, mais les mesures récentes, même avec leurs incertitudes, semblent indiquer que le soleil ne peut pas expliquer le réchauffement récent. Notons toutefois que les simulations actuelles ne tiennent pas compte des fluctuations possibles du rayonnement solaire (négligeable a priori).
%%%% pas sûr du dernier point.

