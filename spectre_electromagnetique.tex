\sk
Les échanges d'\voc{énergie radiative} se font à distance par le biais du \voc{rayonnement électromagnétique}. Le rayonnement électromagnétique est composé d'une superposition d'ondes monochromatiques de longueurs d'onde~$\lambda$ se propageant à la vitesse de la lumière~$c$ (dans le vide~$c=3 \times 10^8$~m~s$^{-1}$). Le rayonnement électromagnétique parcourt la distance Terre-Soleil en $8$~minutes; à l'échelle des processus atmosphériques terrestres, la propagation des ondes électromagnétiques est si rapide qu'elle peut être considérée en première approximation comme immédiate. 

\sk
Les ondes composant le rayonnement électromagnétique peuvent être caractérisées indifféremment par leur \voc{longueur d'onde}~$\lambda$, leur \voc{fréquence}~$\nu = c / \lambda$ ou leur \voc{nombre d'onde}\footnote{Le nombre d'onde est souvent exprimé en cm$^{-1}$. Pour obtenir~$\overline{\nu}$ dans cette unité à partir de~$\lambda$ en microns, on utilise~$\overline{\nu} = 10^{4} / \lambda$.}~$\overline{\nu} = 1 / \lambda$. L'ensemble de ces ondes constitue le \voc{spectre} électromagnétique. Selon le principe de De Broglie, à chaque onde électromagnétique de fréquence~$\nu$ est associée une particule sans masse nommée \voc{photon} dont l'énergie est~$h \, \nu$ où $h = 6.63 \times 10^{-34}$~J~s est appelée la constante de Planck. Cette énergie est souvent exprimée en électron-volts eV ($1$~eV~$= 1.6 \times 10^{-19}$~J~s).

\sk
Le rayonnement visible occupe une bande très étroite du spectre aux longueurs d'ondes comprises entre 0.4 et 0.76~$\mu$m [figure~\ref{fig:spectrum}]. Lorsque l'on considère des longueurs d'ondes plus courtes (c'est-à-dire des fréquences plus élevées) que le rayonnement visible, on passe dans le domaine du rayonnement ultraviolet, puis celui des rayons X et gamma~$\gamma$. Lorsque l'on considère des longueurs d'ondes plus grandes (c'est-à-dire des fréquences plus faibles) que le rayonnement visible, on passe dans le domaine du rayonnement infrarouge, puis celui des micro-ondes et des ondes radio. Les photons les plus énergétiques correspondent aux rayons X; les moins énergétiques aux ondes radio.

\figsup{1}{0.1}{\figpayan/LP211_Chap2_Page_09_Image_0001.png}{\figpayan/LP211_Chap2_Page_09_Image_0002.png}{Classification du rayonnement électromagnétique en fonction de la longueur d'onde. On rappelle que 1~$\mu$m (micron) correspond à $10^{-6}$~m et 1~nm (nanomètre) correspond à $10^{-9}$~m.}{fig:spectrum}
