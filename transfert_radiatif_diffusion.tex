\sk
La diffusion est un phénomène macroscopique résultant de la réflexion, de la réfraction et de la diffraction du rayonnement incident, qui se produisent au niveau microscopique en raison des inhomogénéités du milieu matériel traversé. Par abus de langage, auquel les présentes notes n'échappent pas, on identifie souvent réflexion et diffusion. 

\sk
On distingue différents mécanismes de diffusion selon la taille relative des cibles (molécules ou particules) par rapport à la longueur d'onde du rayonnement électromagnétique incident [figure \ref{fig:diffsize}]. Les connaître permet de comprendre certains phénomènes atmosphériques perçus au quotidien [figure~\ref{fig:ciel}]. Les radiations solaires situées dans l'ultraviolet sont absorbées dans la haute atmosphère (notamment par l'ozone dans la stratosphère) si bien que l'on considère principalement les radiations visibles.
\begin{finger}
\item \underline{Taille des cibles petite devant la longueur d'onde du rayonnement incident} La \voc{diffusion Rayleigh} est la diffusion par les molécules\footnote{ou par des particules significativement petites devant la longueur d'onde, mais ce cas de figure est relativement rare en pratique} qui constituent l'atmosphère. La diffusion Rayleigh dépend fortement de la longueur d’onde incidente. Lord Rayleigh à démontré en 1873 que cette dépendance s’exprimait selon l'inverse de la puissance quatrième de la longueur d'onde
\[ \boxed{ \Sigma_\lambda^r \propto \lambda^{-4} } \]
\begin{citemize}
\item Cette dépendance en longueur d'onde a un effet très notable sur le rayonnement thermique reçu du Soleil, dont nous avons vu au précédent chapitre qu'il est maximum dans les longueurs d'onde visibles. Les molécules d'air de l'atmosphère diffusent plus les photons de courte longueur d'onde à cause de la loi en puissance quatrième de~$\lambda$~: ainsi, le violet et le bleu sont~$16$ fois plus diffusés que le rouge par le mécanisme de Rayleigh. C'est pour cette raison que l'on voit le ciel bleu depuis la surface~: il s'agit de la couleur émanant du rayonnement solaire incident diffusé en majorité par le mécanisme de Rayleigh\footnote{Le fait qu'on ne voit pas le ciel violet est dû à une moindre sensibilité de l'oeil à ces longueurs d'onde, ainsi qu'un moindre flux incident que dans le bleu d'après le spectre solaire.}. La figure~\ref{fig:diffsep} nous indique que~$6\%$ du rayonnement incident sont ainsi diffusés, soit une contribution d'environ~$20\%$ à l'albédo planétaire. 
\item La diffusion Rayleigh ne montre pas de direction préférentielle significative, à part une tendance légèrement supérieure à la diffusion vers l'arrière (rétrodiffusion) et vers l'avant [figure~\ref{fig:diffdir}]. Ceci explique que le ciel apparaisse bleu qu'on le regarde depuis la surface ou depuis un avion. Ceci explique également que la diffusion Rayleigh fasse apparaître le Soleil de la couleur la moins diffusée, à savoir jaune à rouge suivant l'importance de la diffusion, alors qu'il apparaîtrait blanc sans diffusion.
\item Au lever et au coucher du Soleil, lorsque la lumière solaire traverse une couche importante d'atmosphère, la diffusion Rayleigh est plus grande qu'en journée lorsque le Soleil est proche du zénith. La raison est purement géométrique, comme l'on peut s'en convaincre d'après la section~\ref{sec:efficace}~: si l'angle d'incidence~$\theta$ est plus grand, le facteur~$1/\cos\theta$ est plus grand, donc, pour une même section efficace de diffusion~$\Sigma_\lambda^r$, la variation du flux incident~$dL_\lambda$ est plus grande. Ainsi, le soleil est vu rouge le soir car plus de rayonnement incident dans les longueurs d'onde bleues est diffusé qu'en journée.
\item Au contraire du rayonnement thermique solaire, dont le maximum d'émission est dans le domaine visible, l'effet de la diffusion Rayleigh sur le rayonnement thermique émis par la Terre est négligeable, car ce dernier est situé dans l'infrarouge à des longueurs d'onde~$\lambda$ plus élevées pour lesquelles~$\Sigma_\lambda^r \sim 0$. 
\end{citemize}

\figside{0.6}{0.3}{\figfrancis/WH_diff_size}{Type de mécanisme de diffusion dominant en fonction de la longueur d'onde (en abscisse) et de la taille des particules (en ordonnée, l'unité est en $\mu$m). Figure adaptée de Wallace and Hobbs, Atmospheric Science, 2006.}{fig:diffsize}

\figsup{0.48}{0.2}{decouverte/cours_meteo/nuage_ciel.jpg}{decouverte/cours_meteo/ciel_rouge.jpg}{Ciel bleu et nuage blanc. Coucher de soleil rouge. Crédits photos: \url{http://www.meteofrance.com} et \url{http://www.exworld.fr}}{fig:ciel}

\item \underline{Taille des cibles grande devant la longueur d'onde du rayonnement incident} La diffusion par les particules les plus grosses, par exemple les gouttes de brume de quelques centaines de microns, les gouttes de pluie de l'ordre du mm, les cristaux de glace de quelques dizaines de microns, ou les poussières les plus grosses, peut être expliquée par les lois de l'\voc{optique géométrique}, les lois qui gouvernent le fonctionnement des lentilles convergentes/divergentes. Contrairement à la diffusion de Rayleigh, la diffusion est non sélective, c'est-à-dire qu'elle ne dépend pas de la longueur d'onde. Les gouttes d'eau de l'atmosphère diffusent toutes les longueurs d'onde de façon quasiment équivalente, ce qui produit un rayonnement blanc. Ceci explique pourquoi le brouillard et les nuages nous paraissent blancs. La réalité d'un nuage est parfois plus complexe~: ses propriétés radiatives dépendent de la taille des particules et leur nombre par unité de volume.
\item \underline{Taille des cibles grande devant la longueur d'onde du rayonnement incident} La diffusion par les particules de taille intermédiaire, par exemple les gouttes nuageuses ou les aérosols de plus petite taille (quelques microns), est plus délicate à étudier que les deux mécanismes précédemment cités. On parle de \voc{diffusion de Mie}. Les particules soulevées pendant une tempête de poussière sur Terre ou sur Mars causent par diffusion de Mie une couleur orangée au ciel. Suivant la taille et la nature de la particule interagissant avec le rayonnement, la diffusion de Mie peut avoir des caractéristiques très directionnelles [figure \ref{fig:diffdir}]. La section efficace~$\Sigma_\lambda^r$ de la diffusion de Mie suit une loi en l'inverse de~$\lambda^2$ avec la longueur d'onde~$\lambda$ du rayonnement incident. Ces variations sont donc moins sensibles à la longueur d'onde que dans le cas de la diffusion de Rayleigh.
\end{finger}

\figside{0.6}{0.25}{\figfrancis/WH_diff_dir}{Répartition de la probabilité de diffusion dans différentes directions, pour différents types de diffusion: (a) Rayleigh, (b) et (c) Mie avec une particule plus grande en (c).}{fig:diffdir}

